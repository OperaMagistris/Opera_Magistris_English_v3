%---------------------------------
%Coordinator: Vincent ISOZ
%Last Update: 2018-01-20
%Contact: info@sciences.ch
%Compiled on: TeXMaker 5.0.2 x64/MiKTeX 2.9 / Package Update 2018-01-20/Microsoft Windows 10 Creator Update x64/PC 16GB RAM (recommanded otherwise the document could not compiled correctly or not at all without deactivating the LOF and LOT!)

%Remarks: 

%R1. The tkz-tab and etex packages must be installed manually from the MikTeX Package Manager!!!

%R2. When you install first MiKTeX and after TeXMaker don't forget to reboot computer. Compile the document (probably will fail first time...). Do an update of packages using the MiKTeX package updater. Try compile again and use once again de MiKTeX package update. And it should work! If it still don't work i have notice that sometimes corrupted images cause the compilation to not be possible.

%R3. Try using a less as possible styles and math operators to avoid confusion (thanks).

%R4. Since 2015-08-29 the ctable package makes conflicts and therefore you can't compile the document with it anymore. We have commented it.

%R5. When the MikTeX Updater updates the BibTex package, then you have to remove all temporary compilation files of the book to be able to compile it again.

%R6. Above 2,600 pages the MikTeX compilation reach TeX memory limit. You will get the message:
%TeX capacity exceeded, sorry [main memory size=3000000]
%To solve this open a command Window or the PowerShell and enter:
%initexmf --edit-config-file=pdflatex
%Add the following line in the new document that appears on the screen:
%pool_size=5000000
%main_memory=6000000
%extra_mem_bot=2000000
%font_mem_size=2000000
%and save it (ctrl+s) and quit the editor. After rebuild the format with:
%initexmf --dump=pdflatex
%that's it!!!!
%R6. To remove all equation in Notepad++ with a RegEx, use:
% (?s)\\begin\{gather\}.*?\\end\{gather\}
%don't forget to activate the field ". match newlines"
%---------------------------------
\documentclass[12pt,a4paper,twoside,openright]{report}
%solve problem of figures position
\usepackage{float}
%text styles and some special operators
\newcommand{\NewTerm}[1]{\textcolor{red}{#1}}
\newcommand{\SeeChapter}[1]{\textcolor{gray}{#1}}
\newcommand{\Ima}{\text{Im}}
\newcommand{\mi}{\mathrm{i}}
\newcommand{\card}[1]{\ensuremath{\left\|#1\right\|}}
%package to write TexMaker logo
\usepackage{hologo}
%package to write the reycling logo
\usepackage{recycle}
%package for tree logo (avoid to print)
\usepackage{fontawesome}
%package for beautiful quote font
\usepackage{calligra}
%package to control margine and paper size
\usepackage[paperheight=297mm,paperwidth=210mm,top=2.5cm,bottom=2.5cm,left=2cm,right=2cm,bindingoffset=1cm]{geometry}
%packages pour l'internationalisation de la police et son rendu
\usepackage[utf8]{inputenc}
\usepackage[T1]{fontenc}
\usepackage{emerald} %package for nice font for book title
\renewcommand{\rmdefault}{ptm}%to darken the text
\usepackage[french,english]{babel}
%package for specials chars (wingdings like)
\usepackage{pifont}
%package for per-thousand symbol
\usepackage{textcomp}
%package for version history
\usepackage{vhistory}
%packages for headers and footers
\usepackage{fancyhdr}
%package to force page notes to have no idents and to be at bottom of page
\usepackage[hang,flushmargin,bottom]{footmisc} 
%package to have more deep levels than by default
\usepackage{enumitem}
%package to have enumerations on multiple columsn
\usepackage{multicol}
%package for colors in tables, title page and anywhere else
\usepackage[usenames,dvipsnames]{xcolor}
\definecolor{BrickRed}{rgb}{0.72,0,0}
%change color of sections
\usepackage{sectsty}
\allsectionsfont{\color{black!50}}

%package to make captions small, bold and second package to make multiple captions
\definecolor{ocre}{RGB}{243,102,25}
\usepackage[font={scriptsize,bf},labelfont={color=ocre,bf},skip=0pt,justification=centering]{caption}

%\renewcommand{\thefigure}{\textcolor{ocre}{\bfseries\itshape\thechapter.\arabic{figure}}}
%\renewcommand{\figurename}{\textcolor{ocre}{\bfseries\itshape Fig.}}
%\renewcommand{\thetable}{\textcolor{ocre}{\bfseries\itshape\arabic{table}}}
%\renewcommand{\tablename}{\textcolor{ocre}{\bfseries\itshape Table}}

\usepackage{subcaption}
%definition of progress circles for sections
\newlength\charwidth
\newlength\chwidth
\newcommand*\circled[1]%
  \settototalheight\chwidth{#1\,\%}%
  \ifdim\chwidth>\charwidth\let\charwidth\chwidth\fi
  \addtolength\charwidth{15pt}% twice inner sep plus half line width
  \tikz[baseline=(char.base)];
    \draw [line width=2pt, color=basecol] (char.north) arc (90:90-#1*3.6:.5\charwidth) coordinate (a);
    \draw [line width=2pt, color=othercol]  (a) arc (90-#1*3.6:-270:.5\charwidth);
  }%
}
\colorlet{basecol}{purple}
\colorlet{othercol}{purple!25}
%end of definition of progress circles

%some colors for the title page
\definecolor{titlepagecolor}{cmyk}{0.3,0.3,0.3,0.3}
\definecolor{namecolor}{cmyk}{1,.50,0,.10}
%toc/lof/lot stuff
\usepackage{minitoc}
\setcounter{minitocdepth}{2}
\usepackage{tocloft} %this line and above two is to minimize vertical space in LOF and to increase number alignement in LOF
\addtolength{\cftfignumwidth}{1em}
\addtolength{\cfttabnumwidth}{1em}
%various packages for tables complications
\usepackage{hhline} %to draw correctyl double vertical/horizotnal lines
\usepackage{dcolumn} %to align decimals
\usepackage{multirow} %to merge cells
\usepackage{colortbl} %for background colors
\usepackage{array} %to center text horizontally in cells
\usepackage{variations} %to build variation tables
\usepackage{spreadtab} %for formulas in tables
\usepackage{booktabs} %for special borders
\usepackage{slashbox} %for diagonals in cells
\usepackage{fancybox} %for rows/columns with bck colors
\usepackage{hhline} %for double horizontal lines
\usepackage{diagbox}
%\usepackage{diagbox}
\usepackage{rotating} %rotation of text
\usepackage{makecell}
\setlength{\textfloatsep}{0.1cm} %space below table (between table and text)
%macro to vertically center images in cells
\newcommand\cincludegraphics[2][]{\raisebox{-0.6\height}{\includegraphics[#1]{#2}}}
%\usepackage{ctable} %for long tables
\usepackage{longtable} %to repeat title row on multiple pages
\usepackage{pbox} %for carriage return in cells
%package for charts and images
\usepackage{graphicx}
\usepackage{bclogo} %four countries flags
%package for videos (flash animations)
\usepackage{media9}
%package for lettrine
\usepackage{lettrine,oldgerm,yfonts}
%packages to include eps into final pdf (must be declared after graphicx package
\usepackage{epstopdf}
\epstopdfsetup{update} % only regenerate pdf files when eps file is newer
\usepackage{pdfpages}
%packages for *.eps images and others
\usepackage{graphicx}
\usepackage{epsfig}
\usepackage{transparent}
\usepackage{eso-pic}
%for background images
\newcommand\BackImage[2][scale=1]{%
\BgThispage
\backgroundsetup{
  contents={\includegraphics[#1]{#2}}
  }
}
%load packages for text canvas (borders)
\usepackage{bclogo,environ,wrapfig}
\usepackage[many]{tcolorbox}
\usepackage{pst-all}
%page to use footnotes in tables
\usepackage{footnote}
%to force big footnotes to stay on the same page
\interfootnotelinepenalty=10000

%numbering Definitions
\newcounter{def}
\newcommand{\mydef}{%
        \stepcounter{def}%
        \thedef}
        
%numbering lines
\usepackage[modulo,right]{lineno}
%put personal paragraphs number or simply margin notes
\usepackage{marginnote}
\newcounter{importantparagraph}[chapter]
\newcommand{\myparagraph}{%
        \stepcounter{importantparagraph}%
        \theimportantparagraph}

%style for equations boxes
\newcommand*{\boxcolor}{orange}
\makeatletter
\renewcommand{\boxed}[1]{\textcolor{\boxcolor}{%
\tikz[baseline={([yshift=-1ex]current bounding box.center)}] \node [rectangle, minimum width=1ex,rounded corners,draw] {\normalcolor\m@th$\displaystyle#1$};}}
 \makeatother
%end for equation boxes style

%post-it definition
\definecolor{myyellow}{RGB}{242,226,149}
\usetikzlibrary{shadows}
\NewDocumentCommand\StickyNote{O{6cm}mO{6cm}}{%
\begin{tikzpicture}
\node[
drop shadow={
  shadow xshift=2pt,
  shadow yshift=-4pt
},
inner xsep=7pt,
fill=myyellow,
xslant=-0.1,
yslant=0.1,
inner ysep=10pt
] {\parbox[t][#1][c]{#3}{#2}};
\end{tikzpicture}%
}
%end of post-it definition


%package for flowchart/workflows and definitions of some items
\usepackage{tikz}
\usetikzlibrary{backgrounds,arrows.meta,calc,arrows,fadings,mindmap}
\usetikzlibrary{decorations.pathreplacing,decorations.markings,patterns}
\usetikzlibrary{shapes.geometric,shapes.gates.logic.US,trees,positioning,arrows}
\usepackage{tkz-tab} %to draw tikz tables
%tikzlibrary for the standard model particle figure
\usetikzlibrary{calc,positioning,shadows,shadows.blur,decorations.pathreplacing}

%for circle numbers (in section Automata Theory)
\newcommand*\circledtext[1]{\tikz[baseline=(char.base)]{
            \node[shape=circle,draw,inner sep=2pt] (char) {#1};}}
%package for karnaugh map
\usepackage{karnaughmap}
%\usepackage{karnaugh-map}
%for smileys
\usepackage{tikzsymbols}
%for bohr atoms
\usepackage{bohr}
%package for the font size of the morphisms venn diagram
\usepackage{scalefnt}
%to build matrix multiplication scheme
\usepackage{amsbsy}
\newcommand*{\clap}[1]{\hbox to 0pt{\hss#1\hss}}
\newcommand*{\mat}[1]{\boldsymbol{\mathrm{#1}}}
\newcommand*{\subdims}[3]{\clap{\raisebox{#1}[0pt][0pt]{$\scriptstyle(#2 \times #3)$}}}
\fboxrule=1pt
%to build star vote system
\usetikzlibrary{arrows,shapes.geometric,positioning,matrix}
\newcommand\score[2]{
\pgfmathsetmacro\pgfxa{#1+1}
\tikzstyle{scorestars}=[star, star points=5, star point ratio=2.25, draw,inner sep=1.3pt,anchor=outer point 3]
  \begin{tikzpicture}[baseline]
    \foreach \i in {1,...,#2} {
    \pgfmathparse{(\i<=#1?"yellow":"gray")}
    \edef\starcolor{\pgfmathresult}
    \draw (\i*1.75ex,0) node[name=star\i,scorestars,fill=\starcolor]  {};
   }
  \end{tikzpicture}
}
\usetikzlibrary{matrix}
\usetikzlibrary{shapes,arrows}
\usetikzlibrary{shapes.geometric, arrows}
\tikzstyle{startstop}=[rectangle,rounded corners,minimum width=3cm,minimum height=1cm,text centered,draw=black,fill=red!30]
\tikzstyle{io}=[trapezium,trapezium left angle=70,trapezium right angle=110,minimum width=3cm,minimum height=1cm, text centered,draw=black,fill=blue!30]
\tikzstyle{process}=[rectangle,minimum width=3cm,minimum height=1cm,text centered,text width=3cm,draw=black,fill=orange!30]
\tikzstyle{decision}=[diamond,minimum width=3cm,minimum height=1cm,text centered,draw=black,fill=green!30]
\tikzstyle{arrow}=[thick,->,>=stealth]
\tikzstyle{decision}=[diamond, draw,fill=blue!20,text width=4.5em,text badly centered,node distance=3cm, inner sep=0pt]
\tikzstyle{block}=[rectangle,draw,fill=blue!20,text width=5em,text centered,rounded corners,minimum height=4em]
\tikzstyle{line}=[draw,-latex']
\tikzstyle{cloud}=[draw,ellipse,fill=red!20,node distance=3cm,minimum height=2em]
%paragraphe identation by default
\setlength\parindent{0pt}
%package of report visual style
\usepackage{background}
\backgroundsetup{ contents= {\begin{tikzpicture}[remember picture, overlay] \draw [line width=0.3pt,color=gray,step=0.5cm] (current page.south west) grid (current page.north east); \end{tikzpicture} } scale=1, angle=0} 
\usepackage{atbegshi}% http://ctan.org/pkg/atbegshi
%for codes
\usepackage{listings,lstautogobble,xcolor}
%color definition for matlab script
\definecolor{mygreen}{RGB}{28,172,0} % color values Red, Green, Blue
\definecolor{mylilas}{RGB}{170,55,241}
\lstset{language=Matlab,%
    %basicstyle=\color{red},
    breaklines=true,%
    morekeywords={matlab2tikz},
    keywordstyle=\color{blue},%
    morekeywords=[2]{1}, keywordstyle=[2]{\color{black}},
    identifierstyle=\color{black},%
    stringstyle=\color{mylilas},
    commentstyle=\color{mygreen},%
    showstringspaces=false,%without this there will be a symbol in the places where there is a space
    numbers=left,%
    numberstyle={\tiny \color{black}},% size of the numbers
    numbersep=9pt, % this defines how far the numbers are from the text
    emph=[1]{for,end,break},emphstyle=[1]\color{red}, %some words to emphasise
    %emph=[2]{word1,word2}, emphstyle=[2]{style}, 
    autogobble=true   
}
%for algorithms
\usepackage[linesnumbered,ruled]{algorithm2e}
%load americal mathematical society packages for equations
\usepackage{amsmath,amsthm,amssymb,amsfonts}
\DeclareMathOperator{\sgn}{sgn} %must be after package ams!
\usepackage{gauss} %package for gauss elimination in linear algebra
\usepackage{xfrac} %package for small fractions like spin 1/2
\usepackage{gensymb} %for degree symbole
\usepackage{empheq} %for multiple equations box
\usepackage{etoolbox} %to create matrices with outbounds legends using bbordermatrix
\let\bbordermatrix\bordermatrix
\patchcmd{\bbordermatrix}{8.75}{4.75}{}{}
\patchcmd{\bbordermatrix}{\left(}{\left[}{}{}
\patchcmd{\bbordermatrix}{\right)}{\right]}{}{}
\AtBeginDocument{\AtBeginShipoutNext{\AtBeginShipoutDiscard}}
%\package to automatically resize equation to page width
\usepackage{resizegather}
\usepackage{mathtools} %for text above arrows
%package for linear system of equations
\usepackage{systeme}
%packages for special integral
\usepackage{wasysym} %closed inetegrals
\usepackage{esint} %intégrales de contour orientées (analyse complexe)
\usepackage{bigints} %write huge integral
\usepackage{yhmath} %write arc symbol above multiple letters
%package for unitary matrix
\usepackage{dsfont}
%definition for theorems
\theoremstyle{definition}
\newtheorem{theorem}{Theorem}[chapter]
\newtheorem{dem}{Proof}[theorem]
\newtheorem{corollary}{Corollary}[theorem]
\newtheorem{lemma}{Lemma}[theorem]
\newtheoremstyle{itexmp}
 {\topsep}
 {\topsep}
 {\normalfont}
 {0pt}
 {\itshape\bfseries}{.}
 { }
 {#1 \textit{#2}}
\theoremstyle{itexmp}
\newtheorem{exmp}{Example}[chapter]
\newcommand{\statedefn}[2]{
 \definecolor{shadethmcolor}{cmyk}{0.1,0.05,0,0}
 \definecolor{shaderulecolor}{cmyk}{0.73,0.19,0,0}
 \begin{defn}\label{#1}{#2}\end{defn}
}
%for annuities symboles
\usepackage{lifecon}
%for euro symbol
\usepackage{eurosym}
%for tensor calculus AND annuities notations
\usepackage{tensor}
%for quantum bra-ket symbols
\usepackage{braket}
%numbering of equations restart at each section
\numberwithin{equation}{section} 
%package for terms simplifications in equations
\usepackage[makeroom]{cancel}
%package for watermark
\usepackage{draftwatermark}
\SetWatermarkLightness{0.85}
\SetWatermarkAngle{25}
\SetWatermarkScale{2}
\SetWatermarkFontSize{1.5cm}
\SetWatermarkText{4th Edition Draft}
%packages for chapters color margin markers and definitions
\usepackage{background}
\usepackage{xifthen}
\usepackage{totcount}
\regtotcounter{chapter}

\backgroundsetup
{   contents={
        \begin{tikzpicture}[overlay]
            \pgfmathtruncatemacro{\mytotalchapters}{\totvalue{chapter} > 0 ? \totvalue{chapter} : 20}
            \pgfmathsetmacro{\mypaperheight}{\paperheight/28.453}
            \pgfmathsetmacro{\mytop}{-(\thechapter-1)/\mytotalchapters*\mypaperheight}
            \pgfmathsetmacro{\mybottom}{-\thechapter/\mytotalchapters*\mypaperheight}
            \ifcase\thechapter
                \xdef\mycolor{white}
                \or \xdef\mycolor{red}
                \or \xdef\mycolor{orange}
                \or \xdef\mycolor{yellow}
                \or \xdef\mycolor{green}
                \or \xdef\mycolor{blue}
                \or \xdef\mycolor{violet}
                \or \xdef\mycolor{Apricot}
                \or \xdef\mycolor{Magenta}
                \or \xdef\mycolor{GreenYellow}
                \or \xdef\mycolor{Sepia}
                \or \xdef\mycolor{SkyBlue}
                \or \xdef\mycolor{Aquamarine}
                \or \xdef\mycolor{LimeGreen}
                \or \xdef\mycolor{WildStrawberry}
                \or \xdef\mycolor{RawSienna}
                \or \xdef\mycolor{Purple}
                \or \xdef\mycolor{RedOrange}
                \or \xdef\mycolor{CadetBlue}
                \else \xdef\mycolor{black}
            \fi
            \ifthenelse{\isodd{\value{page}}}
            {\fill[\mycolor] ($(current page.north east)+(0,\mytop)$) rectangle ($(current page.north east)+(-0.5,\mybottom)$);}
            {\fill[\mycolor] ($(current page.north west)+(0,\mytop)$) rectangle ($(current page.north west)+(0.5,\mybottom)$);}
        \end{tikzpicture}
    },
    scale=1,
    angle=0
}
%package for image positions
\usepackage{wrapfig}
\usepackage{picins}
%package lipsum and blindtext for various tests
\usepackage{lipsum,blindtext}
%definition for numbering sections
\setcounter{secnumdepth}{5}
\setcounter{tocdepth}{5}
%definition for main TOC depth
%\setcounter{\tocdepth}{5}
%package to get lastpage number
\usepackage{lastpage}
\AtBeginShipout{
\ifnum\value{page}=\number\numexpr\getpagerefnumber{LastPage}-0\relax
\phantomsection\label{preLastPage}
\fi}
%package for chapter styles formatting (gray box) and keept it here otherwise it will make a render bug
\usepackage[Bjornstrup]{fncychap}
\ChNameVar{\Huge}
%for electronic circuit
\usepackage[compatibility]{circuitikz}
\input{electronic_symbols_macros.tex}
%package for small TOC
\usepackage{shorttoc}
%was supposed to be use for Creative Common logo but doesn't work since August 2015 update
%\usepackage[type={CC},modifier={by},version={3.0},]{doclicense}
%part for bibtex
\usepackage[backend=bibtex]{biblatex}
\bibliography{biblio}
%package to make index
\usepackage{makeidx,imakeidx}
%to make work the options below first do an index with the options commented and after compile again with the option uncommented!
\makeindex[options=-s my_index_style.ist -c]

%decoration for title page (if \titlepacagedecoration uncommented further below
\newcommand\titlepagedecoration{
\begin{tikzpicture}[remember picture,overlay,shorten >= -10pt]

\coordinate (aux1) at ([yshift=-15pt]current page.north east);
\coordinate (aux2) at ([yshift=-410pt]current page.north east);
\coordinate (aux3) at ([xshift=-4.5cm]current page.north east);
\coordinate (aux4) at ([yshift=-150pt]current page.north east);

\begin{scope}[black!40,line width=12pt,rounded corners=12pt]
\draw
  (aux1) -- coordinate (a)
  ++(225:5) --
  ++(-45:5.1) coordinate (b);
\draw[shorten <= -10pt]
  (aux3) --
  (a) --
  (aux1);
\draw[opacity=0.6,black,shorten <= -10pt]
  (b) --
  ++(225:2.2) --
  ++(-45:2.2);
\end{scope}
\draw[black,line width=8pt,rounded corners=8pt,shorten <= -10pt]
  (aux4) --
  ++(225:0.8) --
  ++(-45:0.8);
\begin{scope}[black!70,line width=6pt,rounded corners=8pt]
\draw[shorten <= -10pt]
  (aux2) --
  ++(225:3) coordinate[pos=0.45] (c) --
  ++(-45:3.1);
\draw
  (aux2) --
  (c) --
  ++(135:2.5) --
  ++(45:2.5) --
  ++(-45:2.5) coordinate[pos=0.3] (d);   
\draw 
  (d) -- +(45:1);
\end{scope}
\end{tikzpicture}
}

%dedication page
\newenvironment{dedication}
  {\clearpage           % we want a new page
   \thispagestyle{empty}% no header and footer
   \vspace*{\stretch{1}}% some space at the top 
   \itshape             % the text is in italics
   \raggedleft          % flush to the right margin
  }
  {\par % end the paragraph
   \vspace{\stretch{3}} % space at bottom is three times that at the top
   \clearpage           % finish off the page
  }

%for start page
\usepackage{authblk}
%package for international date format
\usepackage[yyyymmdd]{datetime}

%package hyperref for links and pdf metadata, Must be the last package!
\usepackage[unicode,pdfencoding=auto,hidelinks,pdfinfo={pdfauthor={ISOZ Vincent},pdftitle={Elementary Applied Mathematics - Sciences.ch},pdfsubject={Elementary Applied Mathematics},pdfkeywords={Arithmetics;Statistics;Geometry;Trigonometry;Quantitative Finance;Engineering;Theoretical Computing},pdfproducer={Scientifc Evolution Sàrl}, pdfcreator={ISOZ Vincent}}]{hyperref}
\definecolor{wine-stain}{rgb}{0.5,0,0}
\hypersetup{colorlinks,linkcolor=wine-stain,linktoc=all}

%\usepackage[author={Opera Magistris}]{pdfcomment}

\renewcommand{\dateseparator}{--}
%package to control globally space between items
\usepackage{enumitem}
\setlist[1]{itemsep=6pt}
%three packages for old-style background
\usepackage{eso-pic}
\usepackage{changepage}
\strictpagecheck

%*****************************************fancy quotes starts here
\definecolor{quotemark}{gray}{0.7}
\makeatletter
\def\fquote{%
    \@ifnextchar[{\fquote@i}{\fquote@i[]}%]
           }%
\def\fquote@i[#1]{%
    \def\tempa{#1}%
    \@ifnextchar[{\fquote@ii}{\fquote@ii[]}%]
                 }%
\def\fquote@ii[#1]{%
    \def\tempb{#1}%
    \@ifnextchar[{\fquote@iii}{\fquote@iii[]}%]
                      }%
\def\fquote@iii[#1]{%
    \def\tempc{#1}%
    \vspace{1em}%
    \noindent%
    \begin{list}{}{%
         \setlength{\leftmargin}{0.1\textwidth}%
         \setlength{\rightmargin}{0.1\textwidth}%
                  }%
         \item[]%
         \begin{picture}(0,0)%
         \put(-15,-5){\makebox(0,0){\scalebox{3}{\textcolor{quotemark}{''}}}}%
         \end{picture}%
         \begingroup\itshape}%
 \def\endfquote{%
 \endgroup\par%
 \makebox[0pt][l]{%
 \hspace{0.8\textwidth}%
 \begin{picture}(0,0)(0,0)%
 \put(15,15){\makebox(0,0){%
 \scalebox{3}{\color{quotemark}''}}}%
 \end{picture}}%
 \ifx\tempa\empty%
 \else%
    \ifx\tempc\empty%
       \hfill\rule{100pt}{0.5pt}\\\mbox{}\hfill\tempa\ \emph{\tempb}%
   \else%
       \hfill\rule{100pt}{0.5pt}\\\mbox{}\hfill\tempa\ \emph{\tempb}\ \tempc%
   \fi\fi\par%
   \vspace{0.5em}%
 \end{list}%
 }%
 \makeatother
 %*****************************************fancy quotes ends here
%package for some advanced plots
\usepackage{pgfplots}
\usetikzlibrary{patterns}

\usepackage{titlesec}
%this is a test to make chapter title bigger
%\titleformat{\chapter}{\bf\huge}{\thechapter}{1em}{}

%to make beautiful section titles
\newcommand\titlebar{%
\tikz[baseline,trim left=3.1cm,trim right=3cm] {
    \fill [BrickRed!25] (4cm,-1ex) rectangle (\textwidth+3.1cm,2.5ex);
    \node [
        fill=BrickRed,
        anchor= base east,
        rounded rectangle,
        minimum height=3.5ex] at (4.5cm,0) {
        \textbf{\thesection.}
        %before is it was (but the compiler don't managed same anymore)
        %\textbf{\arabic{chapter}.\thesection.}
    };
}%
}
\titleformat{\section}{\LARGE}{\titlebar}{2cm}{} %distance from section text to number
%to have chapter number in front of sections and subsections
%\renewcommand*{\thesection}{\arabic{section}}

%push memory limits of LaTeX
%\usepackage{morewrites}
%if the above "morewrites" does not work...
\usepackage{scrwfile}


\begin{document}
	\sloppy %to force texts and equations to respects pages margins
	\pageref*{LastPage} %the star is to avoid the page number to be a link
	\raggedbottom
	\pagenumbering{roman}
	\newgeometry{margin=2.5cm}
%page de couverture
	\begin{titlepage}
	\newgeometry{left=2cm,bottom=2cm,top=2cm} %defines the geometry for the titlepage
	\BackImage[scale=1]{img/quantum.jpg}
	%\pagecolor{titlepagecolor}
	\noindent
	\color{gray}
	{\normalsize The free 6,000 pages super-quick, super-painless undergraduate transportable Book}\\
	{\normalsize on Elementary Applied Mathematics for Engineers (EAME). Built for people, not for profit!}
	\color{white}
	\makebox[0pt][l]{\rule{1.3\textwidth}{1pt}}\\
	\par
	\noindent
	\scalebox{2.2}{\fontsize{32pt}{0pt}\selectfont \textbf{Opera Magistris}} \\ \textcolor{namecolor}	{\textsf{Original distribution}}
	\vfill
	\noindent
	{\huge \textsf{3rd Edition}}
	\vskip\baselineskip
	\noindent
	\textsf{\{ISBN 978-2-8399-0932-7\} January 2014}
	%DOI: 10.13140/RG.2.2.31296.02562
	\vskip\baselineskip
	Compiled with {\huge \LaTeXe} on \TeX maker
	\vskip\baselineskip
	{\small
EAME-3 work is licensed under a Creative Commons "Attribution
3.0 Unported" license.\\ paternity; no commercial usage; sharing of conditions identical\\
\url{https://creativecommons.org/licenses/by-nc-sa/4.0/}}\\[2pt]
\includegraphics[width=1cm]{img/icons/share2.eps}\includegraphics[width=1cm]{img/icons/remix2.eps}
\includegraphics[width=1cm]{img/icons/by2.eps}
\includegraphics[width=1cm]{img/icons/nc-eu2.eps}
\includegraphics[width=1cm]{img/icons/sa2.eps}\\
2001-2017: \textit{One book to rule them (almost) all!} \href{https://www.facebook.com/sharer/sharer.php?u=http://www.sciences.ch/htmlen/latex/LaTeX_SciencesCh.pdf}{\faThumbsOUp{}}
\\\\
	
	%\titlepagedecoration
	\end{titlepage}
	\restoregeometry %restores the geometry
	\nopagecolor %use this to restore the color pages to white (or other color depending on your choice with \pagecolor{yellow!20} for example)
%end of title page
	\newpage\null\thispagestyle{empty}\newpage
	\restoregeometry
	%old style background
	%\AddToShipoutPicture{\checkoddpage
	%\ifoddpage
    %\put(0,0){\includegraphics[width=\paperwidth,height=\paperheight]{img/odd_page.jpg}}
	%\else
	%     \put(0,0){\includegraphics[width=\paperwidth,height=\paperheight]{img/even_page.jpg}}
	%\fi
 	%}r
	\begin{titlepage}

\newcommand{\HRule}{\rule{\linewidth}{0.5mm}} % Defines a new command for the horizontal lines, change thickness here

\center % Center everything on the page
 
%----------------------------------------------------------------------------------------
%	HEADING SECTIONS
%----------------------------------------------------------------------------------------

\textsc{\Large Original Distribution}\\[1.5cm] % Name of your university/college
\textsc{\Huge \textbf{Opera Magistris}}\\[0.5cm] % Major heading such as course name
\textsc{\large 3rd Edition}\\[0.5cm] % Minor heading such as course title

%----------------------------------------------------------------------------------------
%	TITLE SECTION
%----------------------------------------------------------------------------------------

\HRule \\[0.4cm]
{ \Large \bfseries Compendium on \\ Elementary Applied Mathematics for Engineers }\\[0.4cm] % Title of your document
\HRule \\[1.5cm]
 
%----------------------------------------------------------------------------------------
%	AUTHOR SECTION
%----------------------------------------------------------------------------------------

%\begin{minipage}{0.4\textwidth}
%\begin{flushleft} \large
%\emph{Co-editors:}\\
%Léon \textsc{HARMEL}\\
%Vincent \textsc{ISOZ}\\
%\end{flushleft}
%\end{minipage}
%~
%\begin{minipage}{0.4\textwidth}
%\begin{flushright} \large
%\emph{Supervisors:} \\
%F.D.C. Tigrou % Supervisor's Name Felis domesticus catus Tigrou
%\end{flushright}
%\end{minipage}\\[2cm]

% If you don't want a supervisor, uncomment the two lines below and remove the section above
%\Large \emph{Author:}\\
%John \textsc{Smith}\\[3cm] % Your name

%----------------------------------------------------------------------------------------
%	DATE SECTION
%----------------------------------------------------------------------------------------

{\large \today}\\[2cm] % Date, change the \today to a set date if you want to be precise

\vspace{120px}
\begin{center}
\textit{Scientific method for a better world and re-inform the people!}
\end{center}

%----------------------------------------------------------------------------------------
%	LOGO SECTION
%----------------------------------------------------------------------------------------

\BackImage[scale=1.75]{img/god.jpg}
%----------------------------------------------------------------------------------------

\vfill % Fill the rest of the page with whitespace

\end{titlepage}
	\begin{versionhistory}
		\vhEntry{3.10}{2018-01-20}{VI}{New topics added and a few corrections of typing/translation errors (see change log)}
		\vhEntry{3.9}{2017-11-13}{VI}{Some minor updates for the free version of the book (see change log)}
		\vhEntry{3.8}{2017-08-01}{VI}{Some major and critical updates and corrections (see change log)}
		\vhEntry{3.7}{2017-06-01}{VI}{French 3rd Edition translation into English finished!}
		\vhEntry{3.6}{2017-05-01}{VI}{French 3rd Edition translation into English. Translation progress $\sim 99\%$}
		\vhEntry{3.5}{2017-04-01}{VI}{French 3rd Edition translation into English. Translation progress $\sim 96\%$}
		\vhEntry{3.4}{2017-02-11}{VI}{French 3rd Edition translation into English. Translation progress $\sim 96\%$}
		\vhEntry{3.3}{2016-01-27}{VI}{French 3rd Edition translation into English. Translation progress $\sim 96\%$}
		\vhEntry{3.0}{2014-01-22}{VI}{French 3rd Edition published on the Internet as PDF ($4,888$ pages).}
		\vhEntry{2.0}{2005-03-10}{VI}{French 2nd Edition published on the Internet as PDF ($2,001$ pages).}
		\vhEntry{1.0}{2002-05-01}{VI}{French 1st Edition published on the Internet as HTML only.}
	\end{versionhistory}
	\begin{fquote}[Toni Morrison]If there's a book that you want to read, but it hasn't been written yet, then you must write it.
 	\end{fquote}
	\newpage\null\thispagestyle{empty}\newpage %création d'une nouvelle page en forcant la disparition du numéro de page
	\dominitoc
	\shorttoc{Contents}{0} % Only chapters
	\pagebreak
	\renewcommand{\contentsname}{Table of Contents}
	\tableofcontents
	\newpage\null\thispagestyle{empty}\newpage
	%global interparagraph
	\setlength{\parskip}{12pt}
	\pagenumbering{arabic}
	\clearpage %eliminate headers and footer of table of contents page
	\pagestyle{fancy} %du package fancyhdr!!!
	\renewcommand{\chaptermark}[1]{\markboth{\thechapter.\space#1}{}}
	\renewcommand{\sectionmark}[1]{\markright{#1}}
	\fancyhead[LE,RO]{\nouppercase\leftmark~(\nouppercase\rightmark)} %LE=Left Even,RO=Right Odd
	\fancyhead[LO,RE]{EAME v3.11-2014}
	\renewcommand{\footrulewidth}{1pt}
	\fancyfoot[LE,RO]{{\thepage}/\pageref*{LastPage} (\Acrobatmenu{GoBack}{Go back})}
	\fancyfoot[LO,RE]{\href{mailto:info@sciences.ch}{info@sciences.ch}}
	\fancyfoot[C]{}
	\let\cleardoublepage\clearpage

	\begin{dedication}
	{\LARGE Dedicated to \textbf{Mother Nature}}
	\begin{flushright}
	\includegraphics[scale=0.5]{img/intro/science.jpg}
	\end{flushright}
	\end{dedication}
	

	\chapter{Warnings}
	\minitoc
	\input{Chapter_Warnings.tex}
	
	
	\chapter{Acknowledgements}

	The ideas in this book have been developed and reinforced by many people. I have greatly benefited from my regular interactions with  students (university, engineering schools) and executives from all backgrounds, including CEOs, CFOs, PMs of many companies around the world, teaching sessions, developing company-specific programs, consulting, and even informal conversations. I am grateful to them for sharing their wisdom with me and inspiring many of the ideas in the book.

	This book and its companion website would not have been possible without the valuable support of the people mentioned below. They find here the expression of my gratitude (and for sure if some errors remains in this book this is obviously their fault...):
	
	\begin{itemize}		
		\item \textbf{HARMEL Léon }(†2012), graduate electrical engineer with a specialization in electronics and automation, responsible in the physical research laboratory at ACEC in Charleroi (BEL), for the provision of documentation that was used in the sections of Corpuscular Quantum Physics, Wave Quantum Physics,  Quantum Field Theory, Spinor calculus and General Relativity.
		
		\item \textbf{KOLANI Daname}, Ph.D. student in portfolio management at UM5-Rabat (MAR) for his help on the redaction of the R and MATLAB™ companion books, proofreading and complements of the section in this book about data mining and machine learning and translation of a few hundred pages of the original french website into this \LaTeX{} book.
		
		\item \textbf{LEGRAND Mathias}, Ph.D. of the École Centrale de Nantes (FRA) for his help on the translation of the first $550$  pages of the original french website into this \LaTeX{} book.
		
		\item \textbf{RICCHIUTO Ruben}, engineer degree in physics HES (B.Sc.) from the Engineering School of Geneva (CHE) and mathematician from the University of Geneva for his valuable help in plasma physics, electromagnetism, quantum physics, statistics, topology, quantum chemistry, fractals theory, analysis and many other areas affecting pure mathematics and computing.
		
		\item Regulars users of \href{http://www.les-mathematiques.net}{{\color{blue} Les Mathematiques.net}} and \href{http://www.futura-sciences.com}{{\color{blue} Futura-Sciences.com}} forum, for their valuable assistance in many areas of mathematics and physics. The debates and discussions that took place on the forums helps to constantly improve the educational aspect of this book.
		
		\item The \href{http://www.wikipedia.com}{{\color{blue} Wikipedia}} and \href{http://www.planetmath.com}{{\color{blue} PlanetMath}} websites to whom I am indebted to many borrow almost word by word (and this is mutual...).
		
		\item The professors Gabriel Nagy of Michigan State University, Liam Revell of Massachusetts Boston University, Jorge Cham (of PhD Comics) and the OpenStax team that have received \underline{and} read my multiple e-mails inquiries about images credits and texts credits but that never answered to me (not quite good examples of what should be the "scientific collaboration and spirit"...)
	\end{itemize}
	
	And thanks to all readers, webmasters and teachers for their websites and quality documents available for free and anonymously on the Internet and regular forum stakeholders. I sometimes verbatim recovered their explanations that do not require additions or corrections. It's probably needless to say that you should not assume that these people are in total agreement with the scientific purposes views expressed in this book; and are not responsible for any errors or obscurities that you might accidentally find in it.
	
	Thanks also to the few colleagues and customers who were willing to give me their comments to improve the content of this book. However, it is certain that it can still be improved on many points.
	
	I would like finally to thank especially all of my family for their continued support and my friends for their patience as I was almost completely absent, but I would like to send a special thanks to my Dad and Mom, for all of her incredibly help and support over the last months of translation of this book! I would like also to apologize to some of my customers and colleagues because as I answered very slowly to their e-mails and phones during thirteen months to better focus on the translation of this book. Thanks also to my girlfriend for always being there to take care of me when I forget to take care of myself...
	
	I would sincerely appreciate your comments, criticisms, corrections and suggestions for improving the text, for this purpose you can use the public guestbook associated to this PDF (for questions please use the forum!):
	\begin{center}
	\url{http://www.sciences.ch/htmlen/guestbook.php}
	\end{center}
	or if you want to have a private communication you can contact me by {\href{mailto:info@sciences.ch}{{\color{blue}email}}}. I will respond promptly (in three weeks most of time).
	
\chapter{Introduction}
	%To add line numbers on all paragraphes each 5 lines if necessary	
	%\linenumbers
	
	\minitoc
	\pagebreak
		%to make section start on odd page
	\newpage
	\thispagestyle{empty}
	\mbox{}
	\parpic[l][t]{%
	  \begin{minipage}{30mm}
	    \fbox{\includegraphics[width=80px,height=100px]{img/einstein.eps}}
	  \end{minipage}
	}		
	This book who first Edition has been published in 2001 is designed so that the knowledge required to read it is as basic as possible. It is not necessary to have a PhD to consult it, you just have to know reasoning, to think critically, to observe and have time...
	\begin{flushright}
	\textit{"Simplicity is the seal of truth and it radiates beauty"} \\
	 Albert Einstein
	\end{flushright}
	
	\section{Forewords}
	\lettrine[lines=4]{\color{BrickRed}N}o human endeavour has had more impact than Science\footnote{From Latin \textit{scientia} "knowledge, a knowing, expertness". Itself from \textit{sciens} (genitive scientis) that means "intelligent, skilled", present participle of \textit{scire} that means "to know" probably originally comes from "to separate one thing from another, to distinguish" related to \textit{scindere} "to cut, divide".} on our lives and our conception of the world and ourselves. Its theories, conquests and results are all around us.

	Omnipresent in the industry (aerospace, imaging, cryptography, transportation, chemistry, algorithmic, etc.) or in the services (banking, fintech, insurance, human resources, projects, logistics, architecture, communications, etc.), Applied Mathematics also appears in many other areas: surveys, risk modelling, data protection, politics, etc.  Applied Mathematics (also sometimes named "Mathematics Machinery") influence our lives (telecommunications, transport, medicine, meteorology, music, project management) and contribute to the resolution of current issues: energy, health, environment, climate, optimization, sustainable development, etc. much more than any soft skill techniques or methodology! They great success are their fabulous dispersion in the real world and their increasing integration in all human and artificial intelligence activities. We are going therefore to a situation where mathematicians and engineers will no longer have the monopoly of mathematics, but where almost any graduate job position will have to do advanced mathematics.

	As a former student in the field of engineering I have often regretted the absence of a single book fairly comprehensive, detailed (without going to the extreme...) and educational if possible free (!) and portable (being personally a fan of eBooks...) containing at least a non exhaustive idea of the overall program of Applied Mathematics in engineering schools with an overview of what is used for real in companies with more intuitive than rigorous proofs but with enough details to avoid unnecessary effort to the reader. Also a book that does not require the reader to adopt each time a new notation or terminology specific to the author when it is not outright to change to a foreign language... and where anyone can suggest improvements or additions (through the forum, guest-books or by e-mail).

	I was also frustrated during my studies to have quite often have to swallow "formulas" or "laws\footnote{"Laws" are \underline{descriptive} in Science not \underline{prescriptive} like in religions or in human legal system.}" supposedly (and wrongly) non-provable or too complicated as my teachers says or even disappointed by renowned authors books (where developments are left to the reader or as exercise and no real applications are even mention...). In this book predominates the will to never confuse the reader with empty sentences like "it is evident that...", "it is easy to prove that...", "we leave it to the reader as an exercise...", since all developments are presented in detail. But I'm not a purist of maths! I have only one ambition: to explain the easiest way possible.

	Although I have to admit that prove some mathematical relations presented within the engineering schools curriculum can not be done because of a lack of time in the official program or size limit in a book, I can not accept that a teacher or author tells his students (respectively, his readers) that certain laws are non-provable (because most of the time this is not true!) or that such or such proof is too complicated without giving a reference (where the student can find the information necessary to satisfy his curiosity) or at least a simplified but satisfactory proof.

	Moreover, I think that it is totally archaic today that some teachers continue to ask to their students to take a massive quantity of notes during classes. It would be much more favourable and optimal to distribute a course handout containing all the details in order to be able to concentrate on the essentials points with students, that is to say the oral explanations, interpretations, understanding, reasoning, critical thinking and practice rather than excessive and time-consuming blackboard copy... Obviously by giving a complete course handout some students will be brilliant by their absence but ... it is the better! Thus, those who are passionate can deepen subjects at home or at the university library, the weak do what they have to do and the rest (struggling students but workers) will follow the course given by the teacher to profit to ask questions rather than mindlessly copying a blackboard.

	Inspired on a learning model of an American scholar, whose I forgot the name (...), this book proposes and imposes the following properties to the reader: discover, memorize, cite, integrate, explain, restate, infer, select, use, decompose, compare, interpret, judge, argue, model, develop, create, search, reasoning, develop in a clear progressive teaching way to develop the analytic skills and openness.

	So, in my mind, this non-exhaustive book (and its associated companion PDFs) must be a substitute, free of charge for all students and employees around the World, to many references and gaps of the scholar system, allowing any curious student not to be frustrated for many years during his academic curriculum. Otherwise, the science of the engineer could have the aspect of a frozen science, apart from the scientific and technical developments, a heterogeneous accumulation of knowledge and especially of formulas which made he considered as a tasteless subproduct of mathematics and that brings companies and governments to many false results and bad decisions...
	
	This book has also been designed to meet the needs of executives, both finance as well as non-finance managers. Any executive who wants to probe further and grasp the fundamentals of strategic finance, strategic marketing or project management engineering and supply chain issues will benefit from its lecture. 
	
	This book has also for purpose to describes and explains how our Universe and our World (also other "worlds" in our Universe) works in a much more accurate, more complete and detailed way than any Holy book. This book will give you help the reader to understand how science is done and how scientists and engineers work, it will provide the reader models and quantification methods for the origin of species, of galaxies, of planets, of quantum phenomenon, of physics movements, of stellar physics, of extreme observable events and also extreme rare events and explains social strategies and modern technologies in a mathematical and provable way that everyone can check by himself and by exposing every-time the assumptions that any reasonable entity should take care of!
	
	Obviously Applied Mathematics is such an abundant topic that a book of this scale can only accommodate the basis (we just scratch the surface of gigantic topics!). Readers are certainly encourage to go beyond this (see the bibliography at the end of the book).

	Now, those who see Applied Mathematics only as a tool (what it also is), or as the enemy of religious beliefs, or as a boring school field school, are legion. However, it is perhaps useful to recall that, as Galileo said, «\textit{the book of nature is written in the language of mathematics}» (without wishing to do scientism!). When you go to China, you learn chinese. When you want go to the Universe you learn maths! Because maths are the language of the Universe. This is why maths are so fundamental and their are amazing as they apply to the whole Universe and across time. It is in this spirit that this book discusses Applied Mathematics for students in the Natural, Earth and Life sciences, as well as for all those who have an occupation related to the various subjects including philosophy or for anyone curious to learn about the involvement of science in everyday life.

	The choice to study engineering in this book as a branch of Applied Mathematics comes from the fact that the differences between all areas of physics (formerly known as "natural philosophy") and mathematics are so hardly notable that Fields medal (the highest award today in the field of mathematics) was awarded in 1990 to physicist Edward Witten, who used physical ideas to prove a mathematical theorem. This trend is certainly not fortuitous, because we can observe that all science, since it seeks to achieve a more detailed understanding of the subject it studies, always finish its trials in the pure mathematics (the absolute path by excellence ...). Thus, we can predict in a far future, the convergence of all the sciences (pure, exact or social) to the mathematics for the modelisation techniques (see for example the French PDF "\textit{L'explosion des mathématiques}" available in the download page of the companion website).

	It can sometimes seem to us difficult (due to irrational as obscure and unjustified fear of pure sciences in a large fraction of our contemporaries) to transmit the feeling of the mathematical beauty of nature, its deepest harmony and the well-oiled mechanics of the Universe, to those who know only the basics of algebra. The physicist Richard Feynman spoke a day of "two cultures": people who have and those who do not have sufficient understanding of mathematics to appreciate the scientific structure of nature. It is a pity that mathematics are necessary to deeply understand nature and that they also have a bad reputation. For the record, it is claimed that a King who asked Euclid to teach him geometry complained about its difficulty. Euclid replied, "There is no royal road". Physicists and mathematicians can not convert themselves to a different language. If you want to learn about nature, to appreciate its true value, you must understand its language. The nature is revealed only in this form and we can not be pretentious to the point of asking him to change this fact.

	In the same way, no intellectual discussion will allow you to communicate with a deaf person what you really feel while listening music. Similarly, all discussion of the world remain powerless to transmit an intimate understanding of the nature of those of the "other culture". Philosophers and theologians may try to give you qualitative ideas about the Universe. The fact that the scientific method (in the full sense of the term) can not convince the World of its strengths through its iterative process (the fact that it is rewritten and rewritten and rewritten by incremental improvement as in the DMAIC Six Sigma process\footnote{DMAIC (an acronym for Define, Measure, Analyze, Improve and Control) refers to a data-driven improvement cycle used for improving, optimizing and stabilizing processes and designs. The DMAIC improvement cycle is the core tool used to drive Six Sigma projects that we will study in-deep in the section of Industrial Engineering. However, DMAIC is not exclusive to Six Sigma and can be used as the framework for other improvement applications.}), is perhaps the fact of the limited horizon of some people who imagine that the human or another intuitive concept, sentimental or arbitrarily is the center of the Universe (anthropocentric principle).
	\begin{figure}[H]
		\centering
		\includegraphics[width=1.0\textwidth]{img/intro/scientific_method.jpg}
		\caption[Scientific Method Cyclic Process]{Scientific Method Cyclic Process (source: ?)}
	\end{figure} 
	
	\begin{tcolorbox}[title=Remark,colframe=black,arc=10pt]
	If you are an honest scientist, the huge majority of your ideas, even good
ideas (!), are going to be ruled out, not by new experiments but already by
inconsistency with old experiments. This is what really makes Science very
different, and what gives us an internal notion of right and wrong before new
experiments. So, quite contrary to the sense that some sceptical layperson
would get, the idea that you can just make up crap is wrong!
	\end{tcolorbox}
	
	So in science we have theory, experiment, and, in the middle between
them, "phenomenology". Phenomenologists are the ones who coax predictions out of theories, usually by simplifying the math and figuring out what can be measured, to which precision, and how (and, not rarely, also by whom).
	
	\subsection{Motivation and goals}
	Of course, in order to share this mathematical knowledge, it may seem paradoxical to increase, with our work, the long list of books already available in libraries, in commerce and on the Internet. Nevertheless, I must be able to present arguments that justifies the creation of such a book (and its associated website) as compared to books such as that of Feynman ($\sim 1,500$ pages), Landau ($\sim 3,000$ pages) or Bourbaki ($\sim 8,200$ pages) and Wikipedia/Wolfram webpages themselves or Khan Academy videos or OpenStax PDFs ($\sim 5,500$ pages) or even the \textit{Enzyklopädie der mathematischen Wissenschaften} ($\sim 20,000$ pages). So what do I think I can add to such a wealth of material? 
	\begin{enumerate}
		\item The great pleasure that we take to write this book ("keep the hand" and improve our skills) and have a detailed high quality compendium of tools for our customers and our students (and also all those around the World) for free.

		\item The passion for sharing knowledge \underline{for free} (battle again "copyright madness" (RIP Aaron Swartz!)) and without frontiers with a tool of quality as \LaTeX{} (at the opposite of Wikipedia that mixes \LaTeX{} and normal text and the awful and shameful content of Khan Academy\footnote{OpenStax has good undergraduate PDF - especially the example in their books - but there are between 40-60\% of missing proofs and the table of contents of their PDF and also the Index are not interactive... and major issue...: the content is limited only to undergraduate subjects}).
		
		\item Support free scientific education, critical thinking, and evidence-based understanding of the natural World. Furthermore it is clear that there is an undeserved appetite for people to understand and this book has been written for this purpose.
		
		\item Helping defenceless people who are being dragged into false beliefs by false statements or claims to acquire a certain level of critical method and investigation techniques.
		
		\item Write a modern 3rd millennial version of the "Almagest" hence the name "Opera Magistris" that means in English "Major Work".
		
		\item Because we can't wait as there are places in the world where the absence of teaching modern science and its methodology takes peoples to have believes that bring them to some dangerous and obscure paths.
		
		\item We want to offer Applied Mathematics in an enjoyable and easy-to-learn manner ("keep it simple and stupid" at the opposite of the $9$ Landau's graduate level books), because we believe that Applied Mathematics change the way we understand the Universe.
		
		\item This book was first written in French before (in year 2001) that the French version of Wikipedia had good mathematical content and long before Khan Academy or OpenStax did even exist.

		\item The quick updates/corrections opportunities (at the opposite of Khan Academy) and collaborations of a free e-book (with associated effective search tools) without having topics that disappears (at the opposite of Wikipedia).

		\item The content depending on readers requests/comments and on our interests (at the opposite of Khan Academy, OpenStax or Landau books)!
		
		\item At the opposite of Scientific publications (PRL or other similar) that sucks because don't give detailed proofs and sometimes turn in an infinite loop of references.
		
		\item The access to \LaTeX{} sources to everybody so nobody need to recreate the wheel and loose hundred or thousand of hours on redaction instead of innovation (at the opposite of Landau, Feynman and Bourbaki books)!

		\item Rigorous presentation with simplified detailed proofs of all presented concepts (at the opposite of Wikipedia, Khan Academy and OpenStax that focus only of the mathematical proofs of undergraduate concepts).

		\item The presentation of many advanced and detailed mathematical tools used in business and R\&D keeping in mind that the mathematical language seems eternal and to be one of the only common denominator between all countries in the World.

		\item The opportunity for students and teachers to reuse content by copy/paste (at the opposite of Landau, Feynman and Bourbaki books)!.

		\item Constant and fixed notation (at the opposite of Wikipedia, Khan Academy and OpenStax) throughout the book, for mathematical operators, a clear language on all topics (3.C. criterion: clear, complete and concise) and focus on the basics to make an important pedagogical work on the subjects (at the opposite of Landau's books).

		\item Gather as much information about pure and exact sciences in one electronic (portable), homogeneous and rigorous book (but that don't go as far as Landau's books).

		\item Release from all pseudo-truths, only truths\footnote{Here the word "truth" must not be taken in the sense of "absolute truth" that anyway doesn't exist in experimental science! But only as "something that works in a very reliable way as far as we know"!} that can be proven mathematically with detailed reasoning.

		\item Benefit from the development of teaching methods that use the Internet to search for the solution of mathematical problems.

		\item The dramatic improvement of automatic translation software and computing power that will make of this book, at least we hope, a reference in the fields of sciences.
		
		\item A PDF is better than a website as first all people that use the Internet since 1990 know that the huge majority of website disappear after ten years and secondly it is well known that some countries block Wikipedia and other knowledge website to keep their population in the ignorance (and block a PDF that can be shared in a e-mail is much more difficult).
		
		\item and... because Applied Mathematics are beautiful and especially when written in \LaTeX{} and illustrated (at the opposite of Landau or Bourbaki books whose illustrations and equation rendering quality are quite old and poor).
\end{enumerate}

	And also ... I believe that the results of individual research are the property of humanity and should be available to all those who explore anywhere the phenomena of nature. In this way the work of each benefit to all, and that is for all humanity that our knowledge cumulates and this is the trend that allows Internet.

	I do not hide that my contribution is limited largely to this day to that of a collector who gleans his information in the works of masters or publications or from anonymous web pages and who completes and argues developments and improved them when this is possible. Therefore some of the material in this book is original, and some comes from primary literature. However the vast majority of what we wrote is a rephrasing of results presented in the existing vast library of some (rare) fantastic books. For those who would accuse me of plagiarism, they should think on the fact that the theorems presented in most non-free books and commercially available have been discovered and written by their predecessors and their own personal contribution was also made, like mine, to put all this information in a clear and modern form a few hundred years later. In addition, it can be seen as doubtful that we ask to pay for access to a culture that is certainly the only truly valid and fair one in this world and where there is no patent or intellectual property rights.
	
	\begin{fquote}[Wilson Mizner]If you steal from one author, it's plagiarism; if you steal from many, it's research!
 	\end{fquote}

	This book also reflects my own intellectual limitations. Although I try to study as much science and math fields as possible, it is impossible to master them all. This book shows clearly only my own interests and experiences as consultant, but also my strengths and my weaknesses. I am responsible for the selection of inputs and, of course, of possible errors and imperfections.

	After attempting a strict (linear) order of presentation of the subject, I decided to arrange this book in a more pedagogical (thematic) way and always with practical examples o applications. It is in my opinion very difficult to speak of so vast subject in a purely mathematical order in only one human life, that is to say, when the concepts are introduced one by one, from those already known (where each theory, operator, tools, etc.. would not appear before its definition). Such a plan would require cutting the book, in pieces that are not more thematic. So I decided to present things in a logical order and not in order of need. Thus the reader will encounter, as the editor himself, to the extreme complexity of the subject.

	The consequences of this choice are the following:
	\begin{enumerate}
		\item Sometimes it will necessary to admit certain concepts, even to understand later.
	
		\item It will probably be necessary for the reader to go at least twice throughout the book. At the first reading, we apprehend the essential and at the second reading, we understand the details (I congratulate this who understand all the subtleties the first time).
	
		\item You must accept the fact that some topics are repeated and that there are many cross-references and complementary remarks.
	\end{enumerate}
	
	Some know that for every theorem and mathematical model, there are almost always several methods of proofs. I've always tried to choose the one that seemed the most simple (e.g. in relativity and quantum physics there is the algebraic and matrix formalism). The objective is to arrive at the same result anyway.
	
	This book being in its draft version, it necessarily has lacks on convergence controls, on continuity, grammar and others... (which will horrify some readers and mathematicians ...)! However, I have avoided (or, otherwise, I indicate it) the usual approximations of physics and the use of dimensional analysis, by using it as little as possible. I also try to avoid as much as possible subjects with mathematical tools that have not previously been presented and demonstrated rigorously.
	
	Finally, this presentation, that can still be improved, is not an absolute reference and contains errors. Any comment is welcome. I shall endeavour, as far as possible, to correct the weaknesses and make the necessary changes as soon as possible.
	
	However, while mathematics is accurate and indisputable, theoretical physics (its models), is still interpreted in the common vocabulary (but not in the mathematical vocabulary) and its conclusions all relative. I can only advise, when you read this book, to read by for yourself and not to be subjected to outside influences. You must have a very (very) critical mind, take nothing for granted and question everything without hesitation. In addition, the keyword of good scientist should be: "Doubt, doubt, doubt ... doubt still, and always checks.". We also recall that "nothing that we can see, hear, smell, touch or taste, is what it seems to be", therefore do not rely on your daily experience to draw hasty conclusions, be critical, Cartesian, rational and rigorous in your development, reasoning and conclusions!
	\begin{tcolorbox}[title=Remark,colframe=black,arc=10pt]
	One of the causes of anti-intellectualism probably lies actually in the inability of educational institutions to inculcate critical thinking at high-school. In particular universities, unable to develop critical thinking of their students. The Internet and its amalgamation of true and false also have their part of responsibility. In addition, some experts increase the problem by speaking publicly on issues outside their area of expertise!!!!
	\end{tcolorbox}
	I want to say to those who would try to find themselves the results of some developments of this book, do not worry if they do not success or if they doubt about their competences because of the time spent solving an equation or problem: some theories that seem obvious or easy today, have sometimes needed several weeks, months, even years, to be developed by mathematicians or leading physicists in the past!
	
	I also tried to ensure that this book is pleasing to the eye and to read through.
	
	Finally, I have chosen to write this work in the first person plural form: "we". Indeed, the mathematical physics is not a science that has been made or has evolve through individual work but with intensive collaboration between people connected by the same passion and desire of knowledge. Thus, by making use of "we", I would like pay tribute to the dead and missing scientists, to contemporary and future researchers for the work they will perform in order to approach the truth\footnote{It is clear from the above discussion that scientific research does not and can not lead to a knowledge of nature that is completely free from error. Rather it leads and is able to lead only to an unending process in which the degree of truth in our knowledge is continually increasing. Thus, with the further progress of science into new domains, it becomes possible for us to define the errors in older laws in more and more detail and in more and more respects, and in this way to delimit the domains of validity of these laws more precisely and more nearly completely.} and wisdom.
	
	\begin{center}
	\includegraphics[scale=0.8]{img/humour/pure_math_vs_applied_math.jpg}
	\end{center}

	%to make section start on odd page
	\newpage
	\thispagestyle{empty}
	\mbox{}
	\section{Methods}	
	\lettrine[lines=4]{\color{BrickRed}S}cience is the set of all systematic efforts (scrupulous observations and plausible assumptions until the evidence of the contrary) to acquire knowledge about our environment, to organize and synthesize them into testable laws and theories, whose main purpose is to explain the "how" of things (and NOT the why!) often by a four-step approach:
	
	\begin{itemize}
		\item[$-$] What do we have?
		\item[$-$] Where will we go?
		\item[$-$] What is our goal?	
		\item[$-$] Does it fit the data?
	\end{itemize}
	
	Scientists have to submit their ideas and results to independent verification and replication of their peers ("\NewTerm{peer-review}\index{peer-review}"). They must abandon or modify their conclusions when confronted with more complete or different evidences. The credibility of Science is based therefore on this self-correcting mechanism and this is what still makes in the 21st century that Science is not the best tool (as we do not know what will exist in the future...) but is has been proven as being the best investigation method for truth (ie not the "philosophical truth", but the "truth" as the best explanation at our actual level of knowledge) in comparison for all other actual existing methods or beliefs. The history of science shows that this system works very long and very well compared to all the others. In each area, progress has been spectacular. However, the system sometimes fails and has also to be corrected before small drifts accumulate.

	The downside is that scientists are humans. They have the imperfections of all humans, and especially, vanity, pride, anger and conceit. Nowadays, it happens that many people working on the same topic for a given time develop a common faith and believe they hold the truth. The leader of the faith is the Pope and distils his opinion. The Pope that plays the game, takes his miter and his pilgrim's staff to evangelize his fellow heretics. Until then, this makes smile. But, as in real religions, they are sometimes annoying to want to expand their opinion to those who do not believe. Some of these "churches" do not hesitate to behave like the Inquisition. Those who dare to express a different opinion are burned at every opportunity, during conferences, or at their place of work. Some young researchers, uninspired, prefer to convert to the dominant religion, to become clerics faster rather than innovative researchers or even iconoclasts. The great Pope write his Bible to disseminate his ideas, imposes it to read to students and newcomers. He formats then the thought of younger generations and ensures his throne. This is a medieval attitude that can block progress. Some Popes go so far that they believe be the pope in their specialization field automatically gives them the same throne in all other areas...

	This warning, and the reminders that will follow, must serve the scientific or any reader to ask himself by making good use of what we consider today as the good working/reasoning practices (we will discuss the principles of the Descartes method more below) to solve problems or develop theoretical models. And we do not use the models to build up some metaphysical system. We use the models to predict the outcomes of other measurements and to aid us in putting these to practical use!

	For this purpose, here is a summary table that provides the steps that should be followed by a scientific who works in mathematics or theoretical physics (for definitions, see just below)\label{methodology table}:

	\begin{table}[H]
	\begin{center}
		\definecolor{gris}{gray}{0.85}
			\begin{tabular}{|p{7.5cm}|p{7.5cm}|}
				\hline
				\multicolumn{1}{c}{\cellcolor{black!30}\textbf{Mathematics}} & 
  \multicolumn{1}{c}{\cellcolor{black!30}\textbf{Physics}} \\ \hline
				\textbf{1.} Expose formally or in common language the "hypothesis", the "conjecture" the "property" to prove (hypothesis are denoted H1., H2., etc. the conjectures CJ1., CJ2., etc. and the properties P1., P2., etc.). & \textbf{1.} Expose correctly in a formally or common language all the details of the "problems" to solve (problems are denoted P1., P2., etc.). \\ \hline
				\textbf{2.} Define the "axioms" (non-demonstrable, independent and non-contradictory) that will give the starting points and establish restrictions on development (the axioms are denoted A1., A2, etc.)\footnotemark. \newline\newline
In the same vein, the mathematicians defines the specialized vocabulary related to mathematical operators which will be denoted by D1., D2., etc. & \textbf{2.} Define (or state) the "postulates" or "principles" or the "hypothesis" and "assumptions" (supposedly unprovable...) that will give the starting point and establish restrictions on the developments (typically, assumptions and principles are denoted P1., P2., etc. and assumptions H1., H2., etc. trying to avoid the notation confusion between postulates and principles)\footnotemark. \\ \hline
				\textbf{3.} Once the Axioms laid, pull directly "lemmas" or "properties" whose validity follows directly and prepare the development of theorem supposed to validate departure hypothesis or conjectures (Lemmas being denoted L1., L2., etc. and properties P1., P2., etc.). & \textbf{3.} Once the "theoretical model" developed, check equations units for possible errors in the developments (such checks being marked VA1., VA2., etc.).\\ \hline
				\textbf{4.} Once the "theorems" (noted T1., T2., etc.) proved conclude on "consequences" (denoted C1., C2., etc.) and even properties (noted P1., P2., etc.). & \textbf{4.} Search for borderline cases (including "singularities") of the model to verify the validity intuitively (these borderline controls are denoted CL1., CL2., etc.).\\ \hline
				\textbf{5.} Test the strength (robustness) or usefulness of the conjectures or hypothesis by proving the reciprocal of the theorem or by comparing them with other examples of mathematical well-know theories to see if form together a coherent structure (examples being denoted E1., E2., etc.). & \textbf{5.} Experimentally test the theoretical model obtained and submit work to compare with other independent research teams. The new model should provide experimental results and never observed (predictions to falsify). If the model is validated then it is the official status of "theory".\\ \hline
				\textbf{6.} Possible remarks may be shown in a hierarchically structured order and noted R1., R2., etc. & \textbf{6.} Possible remarks may be shown in a hierarchically structured order and noted R1., R2., etc.			
				\\ \hline
		\end{tabular}
	\end{center}
	\caption{Methodology for Maths \& Physics Developments}
	\end{table}	
	
	\footnotetext[7]{Sometimes "properties", "conditions" and "axioms" are confused while the concept of axiom is much more accurate and profound.}
	\footnotetext[8]{You should not forget, however, that the validity of a model is not dependent on the realism of its assumptions but on the conformity of its implications with reality.}	
	
	\begin{tcolorbox}[title=Remark,colframe=black,arc=10pt]
	The fact that a question can be phrased in a grammatically correct English sentence doesn't make it meaningful, or entitle it to our serious attention. Nor, the fact that a word exist in the dictionary (like the word "soul") make it real...
	\end{tcolorbox}
	
	Proceed as in the above table is a possible workflow basis for people active in the field in mathematics or physics, or for any individual that wants to follow a rigorous critical thinking process\footnote{To educate and exercise yourself or students on the topic of critical thinking we strongly recommend \cite{parker2016looseleaf} and critical thinking tests like the Ennis-Weir test (free-writing test in which the test-taker evaluates, paragraph by paragraph, a given reasoned case) or the California Critical Thinking Disposition Inventory (CCTDI) that measures dispositions to use critical thinking skills like the Watson-Glaser Critical Thinking Appraisal (WGCTA).}. Obviously, proceed cleanly and traditionally as above takes a little or much more time than doing things no matter how (this is why most teachers do not follow these rules, as they would otherwise not have enough time to cover their entire course program) and this is one of the reasons why science takes a  majority of people outside of their comfort zone (as most people are looking to fix problems and interrogations in less than a few dozens of minutes).

	\begin{center}
	\includegraphics[scale=0.75]{img/intro/hypothesis_definitions.eps}
	\end{center}
	
	It must also be known to the reader that we insist on the fact that real scientists should no have emotions behind the subjects they study or speak about. They have to only use evidence (facts based on data, peer-review, reproducible experiences, consensus of scientific community) rather than emotional, biased\footnote{Among all major biases that we will introduce later, the Confirmation Bias - the tendency to search for, interpret, favour, and recall information that confirms or supports one's prior personal beliefs or values - is very likely the most common by scientifically illiterate people.}, subjective educational individual analysis that are not data driven. Obviously we may question the assumption that the used sensory probes detect something real or not? But professional scientists don't care about such a philosophical question. The only thing that matters (putting a part the fun of it...) is that these sensory probes give reliable, useful and consistent measures to test models that will help to build new tools for humanity!

	\begin{figure}[H]
		\centering
		\includegraphics[scale=0.5]{img/intro/consenus_religions.jpg}
		\caption[]{An example of absence of consensus in a small sample of religions}
	\end{figure}
	Notice also a funny shape of scientific $10$ commandments:
	\begin{enumerate}
		\item The phenomena you will observe\\
		And never measures you will falsify\\
		(attention to the confirmation bias\footnote{It's very strongly advised to read our introduction to cognitive biases in the section of Decision Theory page \pageref{cognitive bias}}!)
		
		\item Hypothesis you will proposed\\
		That with experiment you will test
		
		\item The experiment precisely you will describe and the data and algorithms you will provide\\
		Because your colleague will reproduce it\\
		(attention to the narrative discipline trap: the facts will be fitted to the desired results)
		
		\item With your results\\
		A theory you will build
		
		\item Parsimony you will use\\
		And the simplest hypothesis you will retain
		
		\item Ultimate truth will never be (epistemic humility)\\
		And always you will search for the truth
		
		\item From a non-refutable thesis you will refrain\\
		Because outside of the science it will remain
		
		\item All failures will be like a success\\
		Because science can confirm but also invalidate
		
		\item My authority I will not use (authority bias)\\
		To bias people opinions in fields where I have no proven expertise
		
		\item The Archimedean Oath and the Scientific Publication Rules I will respect\\
		As science must be transparent and responsible
	\end{enumerate}
	
	\begin{fquote}[Dara Ó Brian]Science knows it doesn't know everything, otherwise it would stop.
 	\end{fquote}

	\begin{tcolorbox}[colback=red!5,borderline={1mm}{2mm}{red!5},arc=0mm,boxrule=0pt]
	\bcbombe Caution! It is very easy to make new physical theories by just aligning words. This is named "\NewTerm{philosophy}\index{philosophy}" or "\NewTerm{rhetoric}\index{rhetoric}" and the Greeks thought of the atoms in this method. This can lead with a lot of luck to a true theory. Against it is much more difficult to make a "\NewTerm{predictive theory}"\index{predictive theory}, that is to say with equations that predict the outcome of an experiment (all physical theories must posit a correspondence between their mathematical apparatus and the physical world that they are attempting to describe).\\
	
	Moreover many philosophers reinvent arguments physicists have long known to be wrong. We have heard philosophers worry about paradoxes physicists solved ages ago, and we have heard philosophers deduce how natural laws should be while ignoring how natural laws are. In short, there are unfortunately many philosophers who don't notice when they are out of their depth. The same can be said of physicists, though. Some physicists draw on philosophical arguments more frequently than they like to admit. It's easy enough however for us to discard philosophy as useful -
because it is useless.
	\end{tcolorbox}
	
	\begin{fquote}[L. Aron Nelson]Science doesn't know everything, religion doesn't know anything!
 	\end{fquote}

	\begin{tcolorbox}[title=Remark,colframe=black,arc=10pt]
	What separates mathematics and physics is that in mathematics, the hypothesis is always true. Mathematical discourse is not a proof of an external seeking truth, but a target of consistency. What should be correct is just the reasoning. 
	\end{tcolorbox}

	When these rules are not respected, we speak of "\NewTerm{scientific fraud}"\index{scientific fraud} or of "\NewTerm{intellectual fraud}" (which often leads to being fired from his job but unfortunately we still not retired the diplomas when it happens). In general, scientific fraud itself comes in four main forms: plagiarism, fabrication of data and alteration of results unfavourable to the hypothesis, the omission of clear working hypotheses and collected datas, fallacious and biased arguments. To these frauds we can also add behaviours that pose problems regarding to the quality of work or more specifically to ethics, such as  submitting for example several times the same publication with only a few modifications, the omission of conflict of interest, the dangerous experiments, the non-conservation of primary data, etc.
	\begin{figure}[H]
		\centering
		\includegraphics[scale=0.5]{img/intro/peer_review.jpg}
		\caption[]{Source: \url{http://cartoonsbyjosh.co.uk}}
	\end{figure}	

	\subsection{Descartes' Method}
	Now we present the four principles of the Descartes' method which, as remind, is considered as the first scientific in history by his method of analysis:
	\begin{itemize}
		\item[P1.] Never accept anything as true that I obviously knew to be such. That is to say, carefully avoid precipitation and to understand nothing more in my judgements than what would appear so clearly and distinctly to my mind, that I had no occasion to doubt.
		
		\item[P2.] Divide each of the difficulties I have to examine into as many parts as possible (scrupulous observations and plausible hypothesis until evidence of the opposite), and that would be required to resolve them in the best way.
		
		\item[P3.] Driving my thoughts in order, beginning with the simplest objects and easiest to know, to go up gradually by degrees to the knowledge of the most compounds, and even assuming the order between those who not naturally precede each other.
		
		\item[P4.] Make everywhere so complete enumerations and so general reviews, that I'm sure not to omit anything.
	\end{itemize}
	\begin{figure}[H]
		\centering
		\includegraphics[scale=0.2]{img/intro/nullius_in_verba.jpg}
	\end{figure}
	\textit{Nullius in verba} (Latin for "on the word of no one" or "take nobody's word for it") is the motto of the Royal Society.  It is an expression of the determination of Fellows to withstand the domination of authority and to verify all statements by an appeal to facts determined by reproducible experiments and by cautious scientific peer-review\footnote{The process of peer-review generally applies to journal articles, but it is possible for a book to be peer-reviewed as well. Although many books go through some sort of editorial or review process, there is not an easy method for determining whether a book is peer reviewed. One method for locating peer-reviewed books is to take a look at book publications from university presses. Books published by university presses almost always go through a process of peer-review. Books from university presses are typically written by faculty members are who are under immense pressure to produce authoritative scholarly literature. The process of peer-review for university presses typically involves two or three independent referees who will initially review the manuscript. If the manuscript receives positive review, the university press will send it to their editorial board, who are all faculty members, for final review.}.

	\subsubsection{Blind studies}
	Scientific experiments\footnote{This text is a copy/paste of an article written by Manuel Gnida at \url{http://www.symmetrymagazine.org/article/the-facts-and-nothing-but-the-facts}} are designed to determine facts about our world using either "\NewTerm{retrospective studies}\index{retrospective studies}" based on the search of correlations by exploiting existing databases (hence looks backwards) or "\NewTerm{prospective studies}\footnote{Prospective studies have usually fewer potential sources of bias and confounding than retrospective studies.}\index{prospective studies}" based on the search of causalities using controlled/randomized/double-blinded experiments (respectively frequently abbreviated RCT or RCDBT where the "T" stand for "Trial" instead of using the word "Experiment" or "Study"\index{randomized controlled trial}). But in complicated analyses, there's a risk that researchers will unintentionally skew results to match what they were expecting to find. To reduce or eliminate this potential bias, scientists apply a method known as "\NewTerm{blind analysis}\index{blind analysis}".
	
	Blind studies are probably best known from their use in clinical drug trials\footnote{The simplest trial design is a "single-arm trial". In this design, a sample of individuals with the targeted medical condition is given the experimental therapy and then followed over time to observe their response. In a "double-arm trial" half of the people will receive a placebo (exposure group vs non-exposure group). In a "triple-arm trial" a third will receive a competitive treatment. In the general case we speak of "multi-arm trial".} (the term "triple-blinding" sometimes refers to this), in which patients are kept in the dark about - or blind to - whether they're receiving an actual drug or a placebo\footnote{In the science of drugs, a "pure placebo" is a treatment without any active substance; an "impure placebo" is a pharmacologically active product but has no effect on the pathology treated, or whose efficacy has \underline{not been sufficiently demonstrated}.} (indeed dummy pills, with no pharmacological activity at all, demonstrably improve health at a low statistical level of significance that is why double-blind drug trials must use placebos as controls) but also the doctors themselves (to avoid any behaviour bias) and the people who collects the data (to avoid any corruption). This approach helps researchers judge whether their results stem from the treatment itself or from the patients' belief that they are receiving it. But the method is also use in Gastronomy tasting or in forensic laboratories as well.
	
	Particle physicists and astrophysicists do blind studies, too. The approach is particularly valuable when scientists search for extremely small effects hidden among background noise that point to the existence of something new, not accounted for in the current model. Examples include the much-publicized discoveries of the Higgs boson by experiments at CERN's Large Hadron Collider and of gravitational waves by the Advanced LIGO detector.
	
	"\textit{Scientific analyses are iterative processes, in which we make a series of small adjustments to theoretical models until the models accurately describe the experimental data}" says Elisabeth Krause, a postdoc at the Kavli Institute for Particle Astrophysics and Cosmology, which is jointly operated by Stanford University and the Department of Energy's SLAC National Accelerator Laboratory. "\textit{At each step of an analysis, there is the danger that prior knowledge guides the way we make adjustments. Blind analyses help us make independent and better decisions}".
	
	Return on experience (REX) shows as expected that blind analyses need to be designed individually for each experiment. The way the blinding is done needs to leave researchers with enough information to allow a meaningful analysis, and it depends on the type of data coming out of a specific experiment.

	A common approach is to base the analysis on only some of the data, excluding the part in which an anomaly is thought to be hiding. The excluded data is said to be in a "black box" or "hidden signal box".

	Take the search for the Higgs boson. Using data collected with the Large Hadron Collider until the end of 2011, researchers saw hints of a bump as a potential sign of a new particle with a mass of about $125$ gigaelectronvolts. So when they looked at new data, they deliberately quarantined the mass range around this bump and focused on the remaining data instead.
	\label{evidence levels chart}
	\begin{figure}[H]
		\centering
		\includegraphics[width=0.75\textwidth]{img/intro/scientific_evidence_v2.jpg}
		\caption[Scientific evidence hierarchy]{Scientific evidence hierarchy (assuming each level is made \og at its best \fg{})}
	\end{figure}
	
	The figure above illustrates in the lowest level of evidence the obvious fact that we shouldn't trust the scientists (their opinions and beliefs) and even less corporate scientists, physicians, doctors and engineers (who are not scientists!) - science isn't driven by humans beliefs and opinions (!) - but only by the evidence provided by the data from reproducible independent experiments and the statistical analysis at a meta-analysis level in high H-index peer-review papers!
	
	\begin{tcolorbox}[title=Remark,colframe=black,arc=10pt]
	Testimonials or personal anecdotes have almost no values if the sample size is small and biased and if there is no direct way to measure the related event directly! This is why Historical Evidence\index{historicalevidence} based on only a few testimonial written in a unique thousands of year old book have no scientific value (even if there is a dozens of such books with concordant testimonials).
	\end{tcolorbox}	
	
	Those who quote Nietzche:
	\begin{fquote}[Friedrich Nietzsche]There are no facts, only interpretations!
 	\end{fquote}
 	also don't understand that they are at the level EL01 because is that claim itself a fact...????
	
	They used that data to make sure they were working with a sufficiently accurate model. Then they "opened the box" and applied that same model to the untouched region. The bump turned out to be the long-sought Higgs particle.

	That worked well for the Higgs researchers. However, as scientists involved with the Large Underground Xenon (LUX) experiment reported at the workshop, the "black box" method of blind analysis can cause problems if the data you're expressly not looking at contains rare events crucial to figuring out your model in the first place.
	
	LUX has recently completed one of the world's most sensitive searches for WIMPs - hypothetical particles of dark matter, an invisible form of matter that is five times more prevalent than regular matter. LUX scientists have done a lot of work to guard LUX against background particles-building the detector in a cleanroom, filling it with thoroughly purified liquid, surrounding it with shielding and installing it under a mile of rock. But a few stray particles make it through nonetheless, and the scientists need to look at all of their data to find and eliminate them.

	For that reason, LUX researchers chose a different blinding approach for their analyses. Instead of using a "black box", they use a process called "salting".

	LUX scientists not involved in the most recent LUX analysis added fake events to the data—simulated signals that just look like real ones. Just like the patients in a blind drug trial, the LUX scientists didn't know whether they were analysing real or placebo data. Once they completed their analysis, the scientists that did the "salting" revealed which events were false.

	A similar technique was used by LIGO scientists, who eventually made the first detection of extremely tiny ripples in space-time called gravitational waves.

	Not everyone in the scientific community is convinced that blinding is necessary. Blind analyses are more complicated to design than non-blind analyses and take more time to complete. Some scientists participating in blind analyses inevitably spend time looking at fake data, which can feel like a waste.
	
	Typically some quite famous medical doctors and engineers (all are mathematics haters because they were very bad in this field during their studies and don't understand how to apply advanced statistics nor how to read the corresponding analytical results!) castigates the "totally randomized double-blind religion" and a medicine/engineering that has passed from the humanist hands of caregivers/inventors to the cold ones of statisticians and "methodologists". For sure (...) we know since long time that the rule of thumb, feelings and beliefs work much better than statistical method...
	\begin{center}
		\includegraphics[scale=0.5]{img/intro/evidence_truth.jpg}
	\end{center}
	The reader must also keep in mind that we never claimed in this book that Meta-analysis or RCT (randomized control trial) are the golden rule of evidence based science. We just claim that they are the tools that seems actually - at the time we write these lines - to give the best results. Obviously they are not perfect (as the humans conducting the experiments are also not perfect beings...) and sometimes they have failed but anyone criticizing meta-analysis and RCT should provide quantitative based evidence that there are other methods (mentioning which one) that performs statistically significantly better!
	
	\begin{fquote}[L. Aron Nelson]Most people don't really want the truth; they just want reassurance that what they already believe, is the truth.
 	\end{fquote}
	
	\pagebreak
	\subsection{Research Integrity and Engineering/Scientific Ethics}
	Humans are trying at multiple levels to impose international standards of research integrity and engineering/scientific ethics.
	
	Let us introduce three of the most famous one:
	\subsubsection{Singapore Statement on Research Integrity}
	The first one is the Singapore Statement on Research Integrity, drafted at the Second World Conference on Research Integrity, which took place in Singapore from July 21 to 24, 2010, that is an important step toward promoting ethical conduct among scientists around the world. The 340 conference attendees included scientists, journal editors, academic and industry leaders, and representatives from government funding agencies and publishers from over 51 countries. Nanyang Technological University, the National University of Singapore, the Singapore Management University, and the Agency for Science, Technology, and Research hosted the gathering, with support from Singapore's Ministry of Education and National Research Foundation. The Singapore statement was drafted by conference co-chairs, Nicholas Steneck (University of Michigan) and Tony Mayer (Nanyang Technological University), and the incoming chair for the next World Conference, Melissa Anderson (University of Minnesota). In contrast to the First World Conference, which took place in Lisbon, Portugal in 2007 and focused on misconduct issues, the goal of the Second World Conference was to make a concerted effort to promote global research integrity. The Singapore Statement is the fruit of this endeavour.
	\begin{center}
		\includegraphics[scale=0.6]{img/intro/singapore_statement.jpg}
	\end{center}
	
	\subsubsection{European Code of Conduct for Research Integrity }
	The second one is the European Code of Conduct for Research Integrity that serves the European research community as a framework for self-regulation across all scientific and scholarly disciplines and for all research settings.
	
	The European Code of Conduct for Research Integrity  is a PDF of twenty A4 pages where only seven pages contains the code. The people interested to download it for free can just click on the following link:
	
	\begin{center}
	\url{https://www.allea.org/wp-content/uploads/2017/05/ALLEA-European-Code-of-Conduct-for-Research-Integrity-2017.pdf}
	\end{center}
	
	\begin{center}
		\includegraphics[scale=1]{img/intro/european_code.jpg}
	\end{center}
	
	\pagebreak
	\subsubsection{Archimedean Oath}
	The last one is inspired by the Hippocratic Oath, a group of students of the Ecole Polytechnique Fédérale de Lausanne in 1990 developed an oath of Archimedes expressing the responsibilities and duties of the engineer and technician. It was taken in various versions by other European engineering schools and could serve as basic inspiration oath for scientific researchers (however a few points are missing like as in medicine\footnote{Let us recall that contrary to a well spread misconception that doctors and surgeons or any medical personnel are obviously not scientists (we leave it to the reader to do an extensive fact checking if that surprises him)!}, to be struck off the order of scientists in the event of serious ethical error).

	"Considering the life of Archimedes of Syracuse which illustrated as of Antiquity the ambivalent potential of the technique, considering the responsibility increasing for the engineers and scientists with regard to the men and nature, considering the importance of the ethical problems that the technique and its applications raise, today, I pledge following and will endeavour to tend towards the ideal which they represent:
	\begin{enumerate}[label=\protect\circledbullet{\arabic*},leftmargin=15mm]
		\item I will practice my profession for the good of the people, in the respect of the Human Rights and of the Environment.

		\item I will recognize, being as well as possible informed to me, the responsibility for my acts and will not discharge me to in no case on others.

		\item I will endeavour to perfect my professional competences.

		\item In the choice and the realization of my projects, I will remain attentive with their context and their consequences, in particular from the point of view technical, economic, social, ecological... I will pay a detailed attention to the projects being able to have fine soldiers.

		\item I will contribute, in the measurement of my means, to promote equitable relationships between humans and to support the development of the countries lower-income group.

		\item I will transmit, with rigour and honesty, with interlocutors chosen with understanding, any information important, if it represents an asset for the company or if its retention constitutes a danger to others. In the latter case, I will take care that information leads to concrete provisions.

		\item I will not let myself dominate by the defense of my interests or those of my profession.

		\item I will make an effort, in the measurement of my means, to lead my company to take into account the concerns of this Oath.

		\item I will practice my profession in all intellectual honesty, with conscience and dignity.

		\item I promise it solemnly, freely and on my honour."
\end{enumerate}
	\begin{tcolorbox}[title=Remark,colframe=black,arc=10pt]
	Obviously, any engineer that has worked for Fortune 500 corporations knows that such an oath cannot be applied when the board committee and the managers are themselves not engineers and don't care about this oath (at the opposite of medicine where most of time all the hierarchy is made of doctor!). So the only possibility to apply this oath for engineers working is such structures is either to negotiate the respect of this oath during the hiring process (good luck...!) or to resign the contract... (what most engineers don't do and therefore they don't respect this oath...).
	\end{tcolorbox}
Sadly this oath should be completed with the "\NewTerm{Münich Declaration of the Duties and Rights of Journalists (1971)}\index{Münich declaration of the duties and rights of journalists}". That is, the essential duties of the scientist in gathering, reporting on and commenting on data consist in:
\begin{itemize}
	\item Respecting the truth no matter what consequences it may bring abut to him, and this is because the right of the public is to know the truth.

	\item Defending the freedom of information, of commentaries and of criticism.

	\item Publishing only such pieces of information the origin of which is known or – in the opposite case – accompanying them with due reservations; not suppressing essential information and not altering texts and documents.

	\item Not making use of disloyal methods to get information, photographs and documents.

	\item Feeling obliged to respect the private life of people.

	\item Correcting any published information which has proved to be inaccurate.

	\item Observing the professional secrecy and not divulging the source of information obtained confidentially.

	\item Abstaining from plagiarism, slander, defamation and unfounded accusations as well as from receiving any advantage owing to the publication or suppression of information.

	\item Never confusing the profession of journalist with that of advertiser or propagandist and not accepting any consideration, direct or not, from advertisers.

	\item Refusing any pressure and accepting editorial directives only from the leading persons in charge in the editorial office. Every journalist worthy of this name feels honoured to observe the above-mentioned principles; while recognising the law in force in each country, he does accept only the jurisdiction of his colleagues in professional matters, free from governmental or other interventions.
\end{itemize}

	\pagebreak
	\subsection{Scientific Publication Rules (SPR)}\label{scientific publicatons rules}
	It is impossible to have a constructive debate or analysis if the basis material is unusable. Sadly still in the 21st century it is quite easy to find Nobel price publications that were peer-reviewed\footnote{Some studies get published with no peer review at all, even studies from Nobel Prices (...), are called "predatory publishers" flood the scientific literature with journals that are essentially fake, publishing any author who pays.} and that are scientifically unusable (furthermore there is a huge issue with private scientific reviews that reject replication studies most of the time considering them as "boring", ie non-money profitable...). This is why we recall here the basic scientific publication rules for a publication be accepted by a real scientific peer-review committee:
	\begin{enumerate}[label=\protect\circledbullet{\arabic*},leftmargin=15mm]
		\item Use of \LaTeX{} for the writing of the publication
		
		\item All redaction files and raw data files must have ISO 9660 compliant names
		
		\item The publication should have a GUID (a unique Digital Object Identifier alike code)
		
		\item Put the publishing and peer-review dates  and times (ISO 8601 date/time format)
		
		\item Put the major and minor version of the publication (eg: v3.6 r58) and related taxonomy field keywords
		
		\item Put the experiment (development) period date (ISO 8601 date/time format)
		
		\item Write an abstract (brief summary of the goals, experiment, hypotheses, protocol and conclusions)
		
		\item Write an introduction

		\item All measurement\footnote{Measurement by the way must be recordable, reproducible and answer to one or multiple identified causalities.} units and mathematical notation must follow ISO 80000 standards 
		
		\item Use the "principle of precaution\footnote{The of expressions like "...(we) believe...", "....(we have) faith...", "...(we) think..." in the conclusions are forbidden.}" (use of conditional)
		
		\item Use "reactive responses", that is to say the make the confrontations between hypotheses / data, hypotheses / facts, hypotheses / observations 
		
		\item Use, when available, "leverage factors" to give substance and credit to the work by making reference to other corresponding publication on the same subject\footnote{This also the very important step of "personal review", that is to say a personal analysis of several tens / hundreds of scientific publications and that you have made one critical analysis that you use to build your own argument.}
		
		\item Material and Methods should be described in details. For theoretical papers, they should provide a link (URL) or reference where the full detailed proof can be found (if detailed proof is omitted in the original publication!). For experiments they should provide the detailed randomized double-blinded protocol\footnote{Double-blinded to avoid typical scandals like, among many others well-known examples, the Jacques Benveniste affair...}
		
		\item Put high resolution print-screens of charts (including the mandatory measurement errors and confidence/prediction intervals visible on the charts with the source code to generate them for reproducibility!) or photos
		
		\item Write the results and for experimental data always provide a statistical analysis to show if the effect seems significant or not (effect sizes\footnote{This indicator has for pros of not being dependent of the sample size.}, fluctuation intervals, averages, medians, standard deviations, standard errors, sample sizes, kurtosis, skewness and if the $p$-value is communicated then the power of the test must also \underline{always} be communicated!)
		
		\item Calculate the propagation of errors of measurement instruments
		
		\item Write the precautional conclusion to avoid HARKing\index{HARKing} (Hypothesizing After Results Are Known)\footnote{The conclusion for experimental results (reject null hypothesis or not) should be written before (!) the experiment is run and not changed afterwards to avoid human cognitive biases.}
		
		\item Give access to the raw data in a non-proprietary format to the scientific community\footnote{There are international standards for this purpose like Study Data Tabulation (SDTM), Analysis Data Model (ADaM) and Standard for Exchange of Nonclinical Data (SEND)}
		
		\item Give access to the scripts/codes (of Open Source softwares!) used for data analysis to the scientific community\footnote{To avoid case like the famous (and shameful) Reinhart-Rogoff error...}
		
		\item Give access to the \LaTeX{} sources of the publication to the scientific community
		
		\item Provide exact version (with minor release) of the softwares used for the research and to write the paper
		
		\item Put the bibliography with the references (ISO 690 - Numerical) and give the corresponding BibTeX file freely available
		
		\item Cite equivalent studies for meta-analysis\footnote{If there are no equivalent studies, then no meta-analysis are possible, then the results and conclusions don't reach any scientific consensus for recall!}
		
		\item Put the \% financial support of each sponsor (competing interests\footnote{Not only industrial and economic, but also religious like working for a non-secular university!}, funding sources)
		
		\item Submit the paper to the peer-review committee\footnote{Notice that if all scientists followed these rule, peer-review committees, with their bias and mood influence, would not be necessary anymore as the validation of a scientific paper could be quite easily automated.} (in single or double blind way\footnote{"single blind" is that the peer-reviews doesn't know the name of the authors, "double blind" is that neither the authors nor the reviewers know each others' identities.})
		
		\item List all actors (with position, grade from which university, e-mail\footnote{And eventually with gender type, country of origin and birth year for statistical purpose. Like: Albert Einstein (Researcher, PhD Physics ETHZ, a.einstein@ethz.ch, M, CHE, 1879)} and religious beliefs) and peer-reviewers (only name for that latter) of the paper
	\end{enumerate}
	Any publication that doesn't respect at least one of this rule cannot be considered as a "scientific" (ie "serious") publication! Many of the points above also applies to video content (typically TEDx videos, or any YouTube video where the speaker - in a comfortable monologue\footnote{People should not trust monologues - whether on radio, television or any social network - because there are no experts to counter the possible wrong arguments or ill-defined concepts that may lead a part of the audience to wrong speculative interpretations! In addition, humans under the stress of knowing that they are recorded are naturally prone to vocabulary errors and that's without counting the more than 200 cognitive biases in the brain which sometimes lead to the erroneous simplification of complex thoughts... !}... - doesn't cite the sources and meta-analysis when arguing or exposing its "personal experience", "opinions", or "expertise").
	
	The fact of being published in a peer-reviewed journal, however, does not guarantee the quality of an article! This is indeed an interesting fact that deserves to be underlined again and studied. Is it surprising ? Obviously not! All "publishers" know that the "reviewers" who judge the quality of articles submitted to journals are themselves researchers, caught up in the spiral of their own work and teachings, which have neither the desire nor the material possibility of check step by step all the calculations of an article in theoretical physics or all the references of an article in human sciences. This is not their mission. The vast majority of clearly erroneous articles are rejected by the system (the authors of the failed hoaxes do not suck and no one will know how many were foiled). Some, however, pass through the filters. It is obviously regrettable but perfectly well known to each concerned community. That of the hard sciences is not spared and notorious charlatans have managed to publish in respected and recognized journals. The hard sciences have not naturally been disqualified! These articles have simply been ignored for the vast majority of them (so unfortunately not all and especially those that do not detail all mathematical developments!): Neither read nor cited.
	
	\begin{tcolorbox}[title=Remark,colframe=black,arc=10pt]
	Even if is there is a consensus between scientists, a unique oriented study (which can be very important) can be used to influence the opinion of mainstream media, governments and people. This is why a study must always be repeated, peer-reviewed and meta-analysed by independent teams and laboratories.
	\end{tcolorbox}
	
	\begin{tcolorbox}[colback=red!5,borderline={1mm}{2mm}{red!5},arc=0mm,boxrule=0pt]
	\bcbombe Caution! Some people think that a "\NewTerm{scientific consensus}" refers to a large group of scientists who all agree that something is "true" (ie have enough evidence to be actually considered as the most accurate model). In reality, a scientific consensus is a large body of scientific studies that all agree with and support each other ("consensus of data"). The agreement among the scientists themselves is simply a by-product of the consistent evidence.
	\end{tcolorbox}
	
	A well known example of non-existing consensus are religions. Indeed, if someone argue that as the statistics don't lie the christian god must exist as it is the most followed religion in the world with $2$ billion Christians and that $2$ billion people can't be wrong, you can recall this same person that as there is $7$ billion people in the World, the $5$ other billion that not believe in the christian god cannot be wrong as... statistic don't lie... Same if you merge Muslims and Christians together, then only $55\%$ of the people in the World believe in a unique god and $55\%$ is statistically not enough to reach the scientific consensus that is at a threshold level of $95\%$...
	
	\begin{fquote}If a religion had even one true evidence beyond reasonable doubt, there would be no other religions!
 	\end{fquote}
	
	\begin{tcolorbox}[title=Remark,colframe=black,arc=10pt]
	It is possible to logically provide evidence of the non-existence of any gods with certain attributes, by showing an inconsistency between those attributes and either the definition of the god or other established facts. For many examples of this, see  \cite{martin2003impossibility}.
	\end{tcolorbox}
	
	\begin{center}
		\includegraphics[scale=0.6]{img/intro/scientific_papers.jpg}
	\end{center}
	It is then easy to understand why Internet web pages and YouTube videos (or any other similar platform) are not a reliable scientific sources according to the above protocol as:
	\begin{enumerate}
	   \item The peer-reviewers names are the huge majority of time not indicated
	   
	   \item Contributor/Editors are anonymous and can therefore not be identified (typically an issue of Wikipedia)
	   
	   \item The mathematical details are not provided (or even worst, there is not equations given at all!) so it is hard or even impossible to check by yourself if the reasoning is accurate
	   
	   \item The experiment exact protocol is not given so it is impossible to know if the results are fake or real
	   
	   \item No sources or cross-references are given
	   
	   \item The content is not in a reliable format (a video or a web page are not perennial and protected\footnote{In the 21st century a PDF for example should be protected against edition and should also be electronically signed} sources)
	   
	   \item The new presented theoretical models predict indeed what the previous one do, but doesn't predict anything new and is therefore not falsifiable
	   
	   \item The speaker on the video makes assumption that are not falsifiable (reference to god or to theories who mathematical details are not provided)
	   
	   \item etc.
	\end{enumerate}
	\begin{center}
		\includegraphics[scale=0.7]{img/intro/fake_science.jpg}
	\end{center}
	
	The strength of scientific evidence produced by different types of studies (for instance systematic reviews, meta-analysis, randomised control trials, observational research, animal studies, cell studies and expert opinions) can vary.  
	
	\begin{tcolorbox}[title=Remark,colframe=black,arc=10pt]
	The world is not systematically in line with our desires, our hopes, our prejudices, our philosophical aspirations, nor even with our apprehensions or our anxieties. It is not because it would be super cool to be able to move objects by the mere thought that this is necessarily possible. Likewise, it is not because he would be nice that a treatment for a rare disease exists that a therapeutic claim relating to this pathology is necessarily true. Our dreams, our hopes and our imagination are precious attributes of our humanity, which can put us in great danger if we forget that the associated intellectual productions are not due to us by reality. In the same vein, one thing is not necessarily true because it is conceivable that it is.
	\end{tcolorbox}
	
	This infographic will help you understand the advantages and limitations of different types of scientific evidence:
	\begin{figure}[H]
		\centering
		\includegraphics[width=1\textwidth]{img/intro/how_strong_is_the_scientific_evidence.pdf}
		\caption[Different types of scientific evidence]{Different types of scientific evidence (source: EUFIC)}
	\end{figure}
	Anyway, we have to keep in mind that the popularity of a idea has no a priori relevance; the arguments must relate to the quality of the evidence, not its notoriety!
	

	\pagebreak
	\subsection{Scientific Mainstream Media communication}\label{scientific mainstream media communication}
	The reader of mainstream media or also (even worst!) social networks must never trust a scientific study if the reference and peer-reviewed paper is not given as link\footnote{And this is even more true for every information or brain-candy statistics as we will see later during our studies of percentages.} (and if the paper does not respect the scientific publication rules that we have introduced earlier above page \pageref{scientific publicatons rules}!). The study must also not be taken as "absolute truth" by the reader if there is a consensus of the scientific community on only... A UNIQUE ... study/publication\footnote{Keep in mind that even a broken clock is right twice a day...}. The only way to be almost sure is to read the study itself and check if that latter respects the previous scientific publication rules, and this even more when you know that most mainstream medias are World champions in fall into the trap of cognitive biases (\SeeChapter{see section Game and Decision Theory page \pageref{cognitive bias}}) and data fallacies (\SeeChapter{see section Statistics page \pageref{data fallacies}}).
	
	A typical example of such an issue is a news that was taken by many international mainstream media on the Lyme borreliosis disease as following:
	\begin{figure}[H]
		\centering
		\includegraphics[scale=0.23]{img/intro/lyme_borreliose.jpg}
		\caption[Swiss TV publication about Lyme borreliosis treatment]{Swiss TV publication about Lyme borreliosis treatment\\ the 2017-01-08 (source: RTS App)}
	\end{figure}
	In summary what the "scientific journalist" (humm humm... I think it must be a new intern in fact...), of one of the main National Swiss Television (so a TV that has enough money to investigate correctly any news... at least in theory... in a country that assess to be number one in almost everything...), has published is a very bad (catastrophic) interpretation of the original scientific publication. The above article report that: "\textit{...a treatment applied during $3$ days not later than $72$ hour after the bite of the tick has revealed and efficiency of $100\%$...}. The most funny thing is that this article is provided by the Swiss Telegraph Agency (and afterwards relayed by Swiss TV) that says stupidly to be $100\%$ accurate (this is the result of a country that gives a journalist accreditation after only 50 days of training...).
	
	In reality (if medias did have read the publication until the end...) the study was stopped after $8$ weeks and it has been shown that the treatment has no better effect than a placebo...
	
	A second typical and recurrent mistake of the mainstream media is the confirmation bias (we will see the study of the biases later) of which here is a boring and ashamed example so much it is repeated (we may really think they do such errors on a given purpose ...):
	\begin{figure}[H]
		\centering
		\includegraphics[scale=0.23]{img/intro/miracle_lourdes.jpg}
		\caption[Publication of Swiss TV on a miracle in Lourdes]{Publication of Swiss TV about a miracle in Lourdes\\ the 2018-02-12 (source: App RTS/AFP)}
	\end{figure}
	Of course, any person with a minimum of general knowledge can quite simply check through existing meta-analysis that "miracles\footnote{A "miracle" is the term used by many people when they do not know how to explain an observed phenomenon. Eclipses were an example of a "miracle" not so long ago... If something can happen and does, it's hardly miraculous. So, a miracle must be something that can't happen. But, if a miracle does happen, then clearly, it can happen. And therefore it wasn't a miracle to begin with. So if life is full of miracles then you don't understand Nature and probabilities! Miracles have always been the snare of the ignorant and the refuge of the ambitious...}" are also occurring in hospitals and that in terms of "rates" of remission, Lourdes does not no better than mere chance in comparison to hospitals scattered all over the planet relative to this type of observation.

	It is therefore again shameful that one of the main Swiss national channel (a service that has enough money to properly investigate all information before relaying it ... at least in theory ... in a country that exclaims to be the number one in almost everything ...) has published biased information and this is even more shameful that this type of information comes from the AFP (Agence France Presse)...
	
	We give also another similar problematic example in the section of Population Dynamics that occurred during the COVID-19 pandemic in France in 2021 at page \pageref{cnews fallacy}.
	
	\subsubsection{Social Networks}
	About social networks and scientific communication ... A priori we could see as well on Facebook, YouTube, Twitter, Instagram, TikTok during discussions or debates that:
	\begin{itemize}
		\item If we simplify the scientific discourse (thinking that I may be a good idea...) on some hot topics\footnote{Topics where typically people will tell you that: "the facts are denied by my opinion" ...} we can be quickly enough be accused of distorting reality (this is the problem of indeed of simplification...!) or even to be condescending. And if you point out that you have simplified to make scientific illiterate people able to understand your argumentation, you will probably be accused of ad hominem attack. This is a situation afterwards where it is difficult or even impossible to restore confidence.
	
		\item If we do not simplify the scientific discourse (by using the vocabulary and quantitative methods of the field), or if we communicate the links to the scientific studies\footnote{Note that it seems to happen regularly that when links to the studies or meta-analysis  are provided, there is almost always people to say that they are financed by lobbyists, or the links they were chosen in the sense of the defended arguments, or even that as not all the links towards all the studies of the world are provided then the given one are not representative...} or theories themselves, we are accused of hiding the truth or make statistics lie (or even to be condescending again!) thanks to an abscond vocabulary and the use of too technical terms and tools. The result is even worse when access the scientific papers are not in free access!
	
		\item If we are talking about a subject on which we have no expertise or diploma, or which is not our field of activity, we get "repacked" quickly (rightly!)
	
		\item On some groups and social networks accounts, some messages from a discussion may be deleted by administrators, this does not permit the possibility to support arguments or counter-arguments or just completely biases the discussion because some answers disappear or just simply never appear (some stakeholders or messages are typically hidden or deleted by the administrator).
		
		\item A small percentage of people are very well educated but may be biased because their are indoctrinated since childhood and even worst (because quite difficult to detect)... some are just trolls (with postgraduate curriculum) who enjoy to trigger people on social networks just for the fun of reading the answers or... for the pleasure to analyse by curiosity the answers of people to their trolling.
		
		\item Statistically, $2.2\%$ of open social networks users like Facebook, TikTok, Snapchat, Instagram (this excludes automatically social networks like ResearchGate obviously!) have an IQ below $70$ points (and even some with higher IQ suffer of a sort of psychological paedomorphosis or of childhood indoctrination). In a country like France, that has in 2021 forty millions of daily active users, that makes $880,000$ users... and therefore quite a lot of uneducated, scientifically illiterate and non-rational people (taking into account that these kind of people are most of time jobless and therefore have much more to spend on social networks than PhD holder). That's why the mainstream social networks are frequented by plenty of WDKS representatives (World Dunning-Kruger Society).
	
		\item Finally keep in mind that all social networks do not support \LaTeX{}, so it is impossible to have real scientific discussion (that is to say... using mathematical equations or chemical formulas).
	\end{itemize}
	The goal here is not to provide a scientific solution to these problems (it would be relevant, however, to conduct some studies on the subject ...!). However ... tricks that seem to work quite well are blocking comments on the content published by scientists or by the scientific community. To intervene in exchanges only if and only if you are invited to communicate as an external expert (and not to intervene on your own initiative).
	\begin{center}
		\includegraphics[scale=0.8]{img/intro/opinions.jpg}
	\end{center}	
	Finally, let us mention the spurious common argumentation techniques that are particularly flagrant on social networks (and in other media in general too ...) from those who are impervious to the scientific method and statistical analyses and simulations and that are voluntarily (or not?) inspired from the principles of war propaganda that the historian Anne Morelle has stated:
	\begin{enumerate}
		\item We do not want war (ie we do not want any "changes")
		
		\item The opponents are the only one responsible for the war (it is he who forces the "unnatural changes")
		
		\item The leader of the opponents has the face of the devil
		
		\item This is a noble cause that we defend (eg a "public service") and not individual interests
		
		\item The opponents knowingly provokes atrocities (eg murders, dismissals, deaths, ...) and if we commit mistakes it is involuntarily
		
		\item The opponents uses unauthorized weapons (ie we do not understand the arguments of the opponents)
		
		\item We suffer very little loss, the opponents losses are huge (the current system is good so we must not change it because it will be worse)
		
		\item Artists and intellectuals support our cause
		
		\item Our cause has a sacred character (eg truth seeker, public service, etc.)
		
		\item Those who put in doubt our propaganda are traitors (ie they weaken the social cohesion, the national cohesion, try to make profit on the poorest, etc.)
	\end{enumerate}
	This is typically "post-truth" circumstances as defined by the Oxford dictionary\footnote{Post-truth: relating to or denoting circumstances in which objective facts are less influential in shaping public opinion than appeals to emotion and personal belief.}.
	
	We can also add to this list the famous: \og \textit{I did my own research} \fg{}. The guys searched two hundred hours on Twitter, YouTube and amateurs Blogs for scientific evidence by cherry picking and ignoring meta-analysis and this without the technical skills to understand scientific protocols and statistical (literally they pick up the first results on Google thinking that any text on the Internet or in a holy fictional book has a scientific value...).
	\begin{center}
		\includegraphics[width=0.6\textwidth]{img/intro/my_own_research.jpg}
	\end{center}
	No wonder, if the world is filled with empty and ridiculous opinions, there is nothing more capable of giving them course, than ignorance!
	
	Furthermore, keep in mind the "\NewTerm{Dunning-Kruger effect}\index{Dunning-Kruger effect}\label{Dunning-Kruger effect}" that is a type of cognitive bias in which people believe that they are smarter and more capable than they really are. Essentially, low ability people do not possess the skills needed to recognize their own incompetence. The first rule of Dunning-Kruger club is you don't know you're in Dunning-Kruger club!
	\begin{center}
		\includegraphics[width=0.7\textwidth]{img/intro/dunning_kruger_effect.jpg}
	\end{center}
	\begin{fquote}[Charles Darwin]Ignorance more frequently begets confidence than does knowledge: it is those who know little, not those who know much, who so positively assert that this or that problem will never be solved by science.
 	\end{fquote}
	
	\pagebreak
	\subsubsection{Expert Opinions}
	We must also be careful with the method mainstream media choose to interview "experts". 
	
	Indeed, in science it doesn't make really sense to invite an expert to speak especially if the latter:
	\begin{itemize}
		\item Use arguments without providing detailed evidence (name of peer-reviewed meta-analysis)
		
		\item Don't make difference between "opinions" and "scientific based evidence"
		
		\item Speaks about subjects outside of his field of specialization
		
		\item Don't work anymore since many years in laboratories
		
		\item Use his awards and books to give himself a legitimate expert position
		
		\item Use his laboratory team work to promote himself
		
		\item Is alone doing a monologue\footnote{As already said people should not trust monologues - whether on radio, television or any social network - because there are no experts to counter the possible wrong arguments or ill-defined concepts that may lead a part of the audience to wrong speculative interpretations! In addition, humans under the stress of knowing that they are recorded are naturally prone to vocabulary errors and that's without counting the more than 200 cognitive biases in the brain which sometimes lead to the erroneous simplification of complex thoughts... !} (typical of TEDx as already mentioned)
		
		\item Is interviewed by a scientifically illiterate journalist
		
		\item Is full of certitudes (the stupid are cocksure while the intelligent are full of doubt)
		
		\item Has a mental illness (not to be confused with a physical disease)
	\end{itemize}
	There a lot of famous examples (like Nobel Prices that have turned obsess by irrational subjects outside of their field of competence). However let us give two examples that we will meet later in this book again in different sections.
	
	 The first example is that Nosek's team that invited researchers to take part in a crowd-sourcing data analysis project. The setup was simple. Participants were all given the same data set and prompt: Do soccer referees give more red cards to dark-skinned players than light-skinned ones? They were then asked to submit their analytical approach for feedback from other teams before diving into the analysis.

	Twenty-nine teams with a total of $61$ analysts took part. The researchers used a wide variety of methods, ranging - for those of you interested in the methodological gore - from simple linear regression techniques to complex multilevel regressions and Bayesian approaches. They also made different decisions about which secondary variables to use in their analyses.

	Despite analysing the same data, the researchers got a variety of results. Twenty teams concluded that soccer referees gave more red cards to dark-skinned players, and nine teams found no significant relationship between skin color and red cards.
	\begin{figure}[H]
		\centering
		\includegraphics[width=0.8\textwidth]{img/arithmetics/repetability.jpg}
		\caption{Same data, different conclusions (purpose of meta-analysis)}
	\end{figure}
	The variability in results wasn't due to fraud or sloppy work. These were highly competent analysts who were motivated to find the truth, said Eric Luis Uhlmann, a psychologist at the Insead Business School in Singapore and one of the project leaders. Even the most skilled researchers must make sometimes subjective choices that have a huge impact on the result they find. That's why again it is quite a non-sense to give only one expert to speak in mainstream media in a monologue\footnote{As already said people should not trust monologues - whether on radio, television or any social network - because there are no experts to counter the possible wrong arguments or ill-defined concepts that may lead a part of the audience to wrong speculative interpretations! In addition, humans under the stress of knowing that they are recorded are naturally prone to vocabulary errors and that's without counting the more than 200 cognitive biases in the brain which sometimes lead to the erroneous simplification of complex thoughts... !}...
	
	The second recent example (obviously it's special case that must not be generalized!) was the spread of the COVID-19 in the first trimester of year 2020 in the USA. It has been asked to a few US experts what is their prognosis for the number of positive COVID-19 cases in March 29 in USA. This was summarised in the following figure:
	\begin{figure}[H]
		\centering
		\includegraphics[width=0.9\textwidth]{img/intro/covid19_experts.jpg}
		\caption{Experts opinions/estimation issue}
	\end{figure}
	For information, the number of positive cases known in March 29 in USA was in fact a bit above $100,000$. This is again a good example why a unique expert opinion or estimation - whatever the field - may not be always very reliable.
	
	Obviously the reader must keep in mind that the variability exists in the above two examples because the scientists were not authorized to use the "\NewTerm{Delphi method}\index{Delphi method}". Basically, the Delphi method is a process used to arrive at a group opinion or decision by surveying a panel of experts. Experts respond to several rounds of questionnaires, and the responses are aggregated and shared with the group after each round using statistical techniques (we will come back on that technique in the section Game and Decision Theory page \pageref{Delphi method}).

	%to make section start on odd page
	\newpage
	\thispagestyle{empty}
	\mbox{}
	\section{Vocabulary}
	\lettrine[lines=4]{\color{BrickRed}P}hysics and mathematics, like any field of specialization, has its own vocabulary. So that the reader is not lost in the understanding of certain texts he can read in this PDF, we have chosen to present here a few fundamentals words, abbreviations and definitions to know.
	
	Thus, the mathematician like to finish his proofs (when he thinks they are correct) by the abbreviation "Q.E.D." which means "Quod Erat Demonstrandum" (this is Latin).
	
	And during definitions (they are many in math and physics ...) scientist often use the following terminology:
	
	\begin{itemize}
	\item ... it is sufficient that ...
	
	\item ... if and only if ...
	
	\item ... necessary and sufficient ...
	
	\item ... means ...
	
	\item ... prove it ...
	\end{itemize}
	These five are not equivalent (identical in the strict sense). Because "it is sufficient that" correspond to a sufficient condition, but not to a necessary condition. Also it must be notice that these five are placed in the context of data analysis, data accuracy, reproduction and peer-review and not on any personal or common belief or also emotional aspect of a group of people (even if this group of people is more than a few billion individuals...)!
	\begin{center}
		\includegraphics[scale=0.55]{img/intro/an_old_age_argument.jpg}
	\end{center}

	\subsection{On Sciences}	
	It is important that we define rigorously the different types of sciences to which humans often refers. Indeed, it seems that in the 21st century a misnomer is established and that it became impossible for people to distinguish the "intrinsic quality" between a "science" and another one.

	\begin{tcolorbox}[title=Remark,colframe=black,arc=10pt]
Etymologically, the word "science" comes from the Latin "Scienta" (knowledge) whose root is the verb "scire" which means "to know".
	\end{tcolorbox}

This abuse of language is probably the fact that pure and accurate sciences lose their illusions of universality and objectivity, in the sense that they are self-correcting. This has for effect that some sciences are relegated to the background and try to borrow these methods, principles and origins to create confusion. We must therefore be very careful about the claims of scientificity in the human sciences, and this is also (or especially) true for the dominant trends in economics, sociology and psychology. Quite simply, the issues addressed by the human sciences are extremely complex, poorly reproducible, and empirical arguments supporting their theories are often quite low.

	By itself, however, science does not produce absolute truth. By principle, a scientific theory is valid as long as it can predict measurable and reproducible results that can be analysed statistically\footnote{So what we can read in all the various religious books existing around the world are not "scientific theories" but instead "speculative theories"!}. But the problems of interpretation of these results are part of natural philosophy.
	\begin{center}
		\NewTerm{\textbf{No scientific theory is proven or provable! It is simply not falsified as long as an experiment has not come to say otherwise!!}}
	\end{center}
	However, the scientific method is reliable enough so that Justice is not legitimate to take position on any scientific knowledge. And also the reader should keep in mind that actually, and as far as we know, it is impossible for the scientific method to prove that anything is true. At best it can improve the likelihood of something to be true. The scientific method builds knowledge on experimental evidence, and provable mathematics models, not axioms. But experimental evidence verifies or falsifies theories, it does not prove them!!!!!!!!
	\begin{tcolorbox}[title=Remark,colframe=black,arc=10pt]
	Hence keep in mind that a testimonial, or a simple sentence in a book or in a seminar even if said by a scientist has NO SCIENTIFIC VALUE, or no value for drawing conclusions, if it is not accompanied by data (typically proper, replicable, controlled scientific trials) and a mathematical detailed model! However, testimonials remains only useful by building speculative hypotheses and designing experiments and studies. If a speculative hypothesis is not accompanied by a methodology to test it experimentally, then it's still NOT science, but just free bullshit speculation! In a more elegant way we may say that it's: \textit{pseudoscientific rhetoric aimed at a mathematically unsophisticated audience}!
	\end{tcolorbox}
	If you don't get it (or you know people that don't get it), here is an illustrated example that may help to differentiate a "proof" and an "evidence":
	
	Let's imagine a chess board (a square board with black and white boxes in staggered rows) of $8$ boxes per side. The board contains $64$ boxes. 
	\begin{center}
		\includegraphics[scale=0.8]{img/intro/evidence_proof_initial_chess.jpg}
	\end{center}	
	If we remove to diagonally opposed boxes (two white boxes, in our case), there are $62$ boxes remaining. 
	\begin{center}
		\includegraphics[scale=0.8]{img/intro/evidence_proof_initial_chess_reduced.jpg}
	\end{center}
	We propose to find a way to cover up all the remaining boxes with dominoes exactly the size of two chess boxes. All the chess boxes must be covered, and no domino must exceed the chess board.
	\begin{center}
		\includegraphics[scale=1]{img/intro/evidence_proof_initial_chess_mapping.jpg}
	\end{center}
	The question is very simple: "is there at least one solution in which all the remaining chess boxes are covered, and no domino exceeds the chess board?"
	\begin{itemize}
		\item \textbf{Scientific Method:}
		
		An exact science is a science for which all theories are based on experiments. The scientific method consists thus to try all the possible ways to lay out the dominoes in order to cover up the chess board. After a certain number of tries, one might conclude that "It's impossible! I tried everything." Scientifically speaking, the evidence is made that the problem has no solution.
		
		But you will keep a doubt, deep in your mind. Because maybe if you systematically eliminate all layouts, you'll finally end up on a layout that works. This would break down the evidence of impossibility.
	
		\item \textbf{Mathematical method:}
		
		Each domino covers two chess boxes. A white one an a black one.
On $62$ remaining boxes, there are $32$ black ones and $30$ white ones.

		When $30$ dominoes will be placed on the board, all the white boxes will be covered and there will be $2$ black boxes not yet covered.

		Those two black boxes can not touch each other by a side. The best would be touching each other by a corner. A domino will thus be unable to cover both remaining boxes.
	
		The mathematical proof can not be argued and will never be contradicted. 
	\end{itemize}

	Given the diversity of phenomena to be studied, over the centuries there has been a growing number of disciplines such as chemistry, biology, thermodynamics, etc. All these disciplines that are a priori heterogeneous have common foundation physics, for language mathematics and for elementary principle the scientific method.
	
	\begin{fquote}[J. von Neuman]The sciences do not try to explain, they hardly even try to interpret, they mainly makes models. By a model is meant a mathematical construct which, with the addition of certain verbal interpretations, describes observed phenomena. The justification of such a mathematical construct is solely and precisely that it is expected to work-that is, correctly to describe phenomena from a reasonable wide area.
 	\end{fquote}

	Thus, a small memory refresh seems useful:

\textbf{Definitions (\#\mydef):}

\begin{itemize}
	\item[D1.] We define as "\NewTerm{pure science}"\index{pure science} any set of knowledge based on rigorous reasoning valid whatever the (arbitrary) elementary factor selected (when we say then "independent of sensible reality") and restricted to the minimum necessary. Only mathematics (often named the "queen of sciences") can be classified in this category. 

	\item[D2.] We define as "\NewTerm{exact science}"\index{exact science} or "\NewTerm{hard science}"\index{hard science}, any set of knowledge based on the study of an observation, observation that has been transcribed in symbolic form  and that can be reproduced and refuted (theoretical physics for example... sometimes...). Primarily, the purpose of exact sciences is not to explain the "why" but the "how". 
	
	And never forget... Science (especially physics) doesn't have to "make sense" it just has to make all the right, testable predictions (instrumentalism)! According to the philosopher Karl Popper, a theory is scientifically acceptable if, as presented, it can be "\NewTerm{falsifiable}\index{falsifiable}" (synonyms are "\NewTerm{refutable}\index{refutable}" or "\NewTerm{testable}\index{testable}"), i.e. subjected to experimental tests\footnote{That's why philosophy and naive human logic themselves can't "prove" anything. Because they strongly depend and vary from one socio-cultural context where people are born to another one (people from New-York have a quite different logic and philosophy from the Sentinelese tribe...). Without falsification science would be an anarchy of logically consistent but still useless models that simply suit someone's fancy.} (or  if it is possible to conceive of an observation or an argument which negates the statement in question). The "scientific knowledge" is then by definition the set of theories that have resisted to falsification (no amount of experimentation can ever prove you are right, but a single experiment can prove you are wrong!). Science is by nature subject to continuous questioning. 

	\begin{tcolorbox}[colback=red!5,borderline={1mm}{2mm}{red!5},arc=0mm,boxrule=0pt]
	\bcbombe Caution! There is no doubt that the exact sciences have yet an enormous prestige, even among their opponents because of their theoretical and practical success. It is certain that some scientists sometimes abuse of this prestige by showing a sense of superiority that is not necessarily justified. Moreover, it often happens that this same scientists exposed in the popular literature, very speculative ideas as if they were very approved, and extrapolate their results outside the context in which they were tested (and ... under the hypotheses they were checked once...).
	\end{tcolorbox}
	 
	
	\begin{center}
		\includegraphics[scale=0.35]{img/intro/evolution_is_just_theory.jpg}
	\end{center}

	\begin{tcolorbox}[title=Remark,colframe=black,arc=10pt]
The two previous definitions are often included in the definition of "\NewTerm{deductive sciences}"\index{deductive science} or even "\NewTerm{phenomenological science}"\index{phenomenological science}.
	\end{tcolorbox}
	
	\item[D3.] We define as "\NewTerm{engineering science}"\index{engineering science} or "\NewTerm{applied science}"\index{applied science} any set of knowledge or practices applied to the needs of human society such as electronics, chemistry, computer science, telecommunications, robotics, aerospace, biotechnology... 
	
	\begin{tcolorbox}[title=Remark,colframe=black,arc=10pt]
	A common confusion by the population is to argue that scientists are the reason why guns and bombs exists. In facts physicists (and other "real" scientists) focus on the theories, the understanding of Nature, not on the business or political applications. The people that converts theories into business applications are "engineers" and engineers are not scientists. Engineers just convert science knowledge into business application in a genius way. That's what engineers are!
	\end{tcolorbox}

	\item[D4.] We define as "\NewTerm{science}"\index{science} any body of knowledge based on studies or observations of events whose interpretation has not yet been transcribed and verified with mathematical rigour, characteristic of previous sciences, but using comparative statistics. We include in this definition: medicine (we should however be careful because some parts of medicine are studying phenomena using mathematical descriptions such as neural networks or other phenomena associated with known physical causes), sociology, psychology, history, biology, etc.
	
	Some teachers like to play with the word "science" as the acronym of (that's not stupid for college students): \textbf{S}olve, \textbf{C}reate, \textbf{I}nvestigate, \textbf{E}valuate, \textbf{N}otice, \textbf{C}lassify, \textbf{E}xperiment.
	
	\item[D5.] We define a (modern) "\NewTerm{scientist}"\index{scientist} as a professional owning a Doctorate and who systematically and daily gathers and uses reproducible research and experimental evidence, to make statistical hypothesis and test them using cutting edge scientific methods (like PhD level statistics or advanced reproducible numerical simulations), to gain and share peer-reviewed understanding and knowledge in famous scientific journals in a very specialized field of expertise.
	
	The definition therefore exclude the huge majority of engineers (who don't publish in peer-review papers and doesn't know how to analyse data using PhD statistical methods), laboratory assistants who just make experimental manipulations but don't make the hypothesis nor analyse the resulting measurements of their experiments but also medical doctors (see \cite{smith2004doctors} and \cite{freed2004doctors}) who only apply the results of scientific research but don't do any as specified above. Mathematicians are not excluded from this definition as their mathematical theories (proofs) are published in peer-reviewed papers and are reproduced (checked) independently by other mathematicians.

	\item[D6.] We define as "\NewTerm{soft science}"\index{soft science}, "\NewTerm{para-science}"\index{para-science} or "\NewTerm{pseudo-science}"\index{pseudo-science} any set of knowledge or practices that are currently based on non-verifiable and non-refutable facts (not scientifically reproducible) by experience or by mathematics. We include in this definition typically: astrology, theology, paranormal (which was demolished by zetetic science), graphology, justice\footnote{Indeed, for example in Switzerland, it is common that the cantonal Judge and the Federal Judge don't give the same judgement as that latter is non-scientific but rather subjectively based on the Judge experience of life}, etc. 
	
	As some scientists say: «\textit{It looks like science, it use the vocabulary of science... but that's not science at all.}»
	
	Especially pseudo-sciences are characterized by:
	\begin{itemize}
		\item They start with a conclusion (believe), then work backwards to confirm

		\item They make sometimes unfalsifiable, vague or unobservable claims 
		
		\item They relied heavily on anecdotes, personal experiences, and testimonials
		
		\item They are hostile to criticism

		\item They use circular reasoning/arguments and other logical fallacies

		\item They use vague jargon to confuse and evade

		\item They use subtle strategies to change people minds (especially children)
		
		\item They make people feel special and unique

		\item They do cherry picking only on favourable evidence

		\item They use non-reproducible/non-refutable methods with unrepeatable results

		\item They use bullshit-random language (technobabble) to impress the audience

		\item They use inconsistent and invalid logic
		
		\item They profess certainty, talk of "proof" with great confidence
		
		\item They claim there's a conspiracy to suppress or hide their ideas
		
		\item They go directly to the public, avoiding scientific scrutiny and consensus

		\item People working in the field are dogmatic and unyielding
		
		\item They highly monetize their books, videos (YouTube), conferences, etc.
	\end{itemize}
	\begin{figure}[H]
		\centering
		\includegraphics[width=0.8\textwidth]{img/intro/pseudo_science.jpg}
		\caption{Organized collection of irrational nonsense}
	\end{figure}
	And don't forget:
	\begin{center}
		\NewTerm{\textbf{What can be asserted without scientific evidence can be dismissed without scientific evidence!!}}
	\end{center}
	
	\item[D7.] We define as "\NewTerm{phenomenological science}" or "\NewTerm{natural sciences}"\index{natural science}, any science which is not included in the above definitions (history, sociology, psychology, zoology, biology, ...) 

	\item[D8.] "\NewTerm{Scientism}"\index{scientism} is an ideology that considers experimental science is the only valid mode of knowledge, or, at least, superior to all other forms of interpretation in the world. In this perspective, there is no philosophical, religious or moral truths superior to scientific theories. Only account what is scientifically proven (ie evidence beyond reasonable doubt!). 

	\item[D9.] "\NewTerm{Positivism}"\index{positivism} is a set of ideas that considers that only the analysis and understanding of facts verified by experience can explain the phenomena of the sensible world. Certainty is provided solely by the scientific experiment. He rejects introspection, intuition and metaphysical approach to explain any knowledge of the phenomena. \\
	
	What is interesting about this doctrine is that it is certainly one of the few that requires people to have to think for themselves and to understand the environment around them by continually questioning everything and by never accepting anything as granted (...). In addition, the real sciences have this extraordinary property that they give the possibility to understand things beyond what we can see. 
\end{itemize}

	But, science is science, and nothing more: a certain ordering, not too bad success, things that no longer leads to the metaphysics as the time of Aristotle, but that does not pretend to give us the whole story on reality or even the bottom of visible things.
	\begin{center}
		\NewTerm{\textbf{People must remember that Science is a robust cross-validated investigation method and NOT a belief system!!}}
	\end{center}
	\begin{tcolorbox}[title=Remark,colframe=black,arc=10pt]
	A quite common bias is to argue that "\textit{scientists believes in their actual models and the things they based their observations on}". That's totally wrong! Indeed, trained and real scientists are professionals (they are paid every month to do research and development - most of the time to found methods to reject existing theories - at least 30 to 70 hours per week after 5 to 7 years of graduate studies!) and at this level they are supposed to not be attached sentimentally to the models they're working on or things they used to make observations (even if some scientists - and sometimes some famous one - cannot control their cognitive biases!). The "\textit{actual models}" and "\textit{things they based their observations on}" are for professional scientists just "toy models" and "calibrated measurement tools" that seems actually to work well to explain and study our surrounding. When they have to build new models, this is because they are paid for! Not because they believe in them! There is absolutely no beliefs! Do you think that a garage owner really "believe" in it's wrench to work? He don't give a f**** about believing in it or not, it's just a well designed tool (maybe there is a better one?) that he uses to do his work conscientiously - sometimes with passion - to get his wage at the end of the month to pay his rent and food for his family!
	\end{tcolorbox}
	And by the way... also don't forget that:
	\begin{figure}[H]
		\centering
		\includegraphics[scale=0.5]{img/intro/bullshit_asimmetry_law.jpg}
		\caption[The Alberto Brandolini bullshit asymmetry law]{The Alberto Brandolini bullshit asymmetry law (source: Twitter)}
	\end{figure}	

	\subsection{Terminology}

The table of methods we presented earlier above (see page \pageref{methodology table}) contains terms that may perhaps seem unknown or barbarous for you. This is why it seems important to provide definitions of these and some other equally important that can avoid important confusion. 

\textbf{Definitions (\#\mydef):}

\begin{itemize}
	\item[D1.] Beyond its negative sense, the idea of "\NewTerm{problem}"\index{problem} refers to the first step of the scientific method. Formulate a problem is also essential for its resolution and allows to properly understand what is the problem and see what needs to be resolved. \\\\
	The concept of "problem" is intimately connected to the concept of "assumption" which will see the definition below. 

	\item[D2.] A "\NewTerm{hypothesis}\index{hypothesis}" is always, in the context of a theory already established or underlying, a supposition awaiting confirmation or refutation that attempts to explain a group of facts or predict the onset of new facts.\\\\
	Thus, a hypothesis can be at the origin of a theoretical problem that has to be resolved formally. 

	\item[D3.] The "\NewTerm{postulate}\index{postulate}" or  "\NewTerm{assumption}\index{assumption}" in physics corresponds frequently to a principle (see definition below) which admission is required to establish a proof (we mean that this is a non-provable proposition).\\\\
	The mathematical equivalent (but in a more rigorous version) of the assumption is the "axiom" for which we will see the definition below. 

	\item[D4.] A "\NewTerm{principle}"\index{principle} (close parent of "postulate") is a proposal accepted as a basis for reasoning or a general theoretical guide line for reasoning that needs to be performed. In physics, it is also a general law governing a set of phenomena and verified by the accuracy of its consequences. \\\\
The word "principle" is used with abuse in small classes or engineering schools by teachers not knowing (which is very rare), or unwilling (rather common), or that can't because lack of time (almost exclusively ) prove a relation.\\\\
The equivalent of the postulate or principle in mathematics is the "axiom" which we define as follows: 

	\item[D5.] An "\NewTerm{axiom}"\index{axiom} is a self-evident proposition or truth by itself which admission is necessary to establish a proof.\label{axiom} 
\end{itemize}

	\begin{tcolorbox}[title=Remarks,colframe=black,arc=10pt]
	\textbf{R1.} We could say that this is something we define as the truth for the speech that we argue, like a rule of the game, and that it does not necessarily a universal truth value in the sensitive world around us.\\

	\textbf{R2.} Axioms must always be independent (one should not be able to be proved from the other) and non-contradictory (sometimes we also say that they must be "consistent"). 
	\end{tcolorbox}	
	
\begin{itemize}
	\item[D6.] The "\NewTerm{corollary}"\index{corollary} is a term unfortunately almost non-existent in physics (wrongly!) and that is in fact a proposal resulting from a truth already demonstrated. We can also say that a corollary is and obvious and necessary consequence of a theorem (or sometimes of a postulate in physics). 

	\item[D7.] A "\NewTerm{lemma}"\index{lemma} is a proposal deduce from one or more assumptions or axioms and that for which the proof prepares this of a theorem.
\end{itemize}

	\begin{tcolorbox}[title=Remark,colframe=black,arc=10pt]
The concept of "lemma" is also (and this is unfortunate) almost used only in the field of mathematics. 
	\end{tcolorbox}	

\begin{itemize}
	\item[D8.] A "\NewTerm{conjecture}"\index{conjecture} is a supposition or opinion based on the likelihood of a mathematical result.\\\\
	Many conjectures have as little similar to lemmas, as they are checkpoints to obtain significant results.
	
	\item[D9.] Beyond its weak conjecture sense, a "\NewTerm{theory}"\index{theory} or "\NewTerm{theorem}"\index{theorem} is a set articulated around a hypothesis and supported by a set of facts or developments that give it a positive content and make the hypothesis well-founded (or at least plausible in the case of theoretical physics). 

	\item[D10.] A "\NewTerm{singularity}"\index{singularity} is an indeterminacy in a calculation That takes the appearance of a division by zero. This term is both used in mathematics and in physics. 

	\item[D11.] A "\NewTerm{proof}"\index{proof} is a set and sequence of mathematical procedures to use and follow to derive a result already known or not of a theorem, lemme or corollary (as already mentioned, this term should obviously be prohibited in experimental sciences as he doesn't highlight the underlying level of evidence!). 

	\item[D12.] If the word "\NewTerm{paradox}"\index{paradox} etymologically means: contrary to common opinion, it is not by pure taste for provocation, but rather for solid reasons. A "\NewTerm{sophism}"\index{sophism} meanwhile, is a deliberately provocative statement, a false proposition based on an apparently valid reasoning. Thus we speak about the "Zeno's paradox" when in reality it is only a sophism. The paradox is not limited to falsity, but implies the coexistence of truth and falsity, so that one can no longer distinguish true and the false. The paradox appears as an unsolvable problem an "\NewTerm{aporia}"\index{aporia}. 
	
\end{itemize}

	\begin{tcolorbox}[title=Remark,colframe=black,arc=10pt]
It should be added that the well-knows paradoxes, by the questions they raised, have permitted significant advances to science and led to major conceptual revolutions in mathematics as in theoretical physics (the paradoxes on sets and on infinity in mathematical, and those at the base of relativity and quantum physics).
	\end{tcolorbox}	

	%to make section start on odd page
	\newpage
	\thispagestyle{empty}
	\mbox{}
	\section{Science and Faith}
	\lettrine[lines=4]{\color{BrickRed}W}e will see that in Science, a theory is usually incomplete because it can not fully describe the complexity of the real world or because it does not predict what we don't know (excepted for Quantum Physics or General Relativity). It is thus for theories like the Big Bang (\SeeChapter{see section Astrophysics page \pageref{astrophysics}}) or the Evolution of species (\SeeChapter{see sections Populations Dynamics page \pageref{population dynamics} or Decision and Games Theory page \pageref{game and decision theory}}) because they are not reproducible in laboratories under identical conditions.  But some other theories are so accurate to predict physical phenomena that some people \underline{believe} that mathematics is the nearest language with some kind of gods (at least for those who believe in a divinity\footnote{Keep in mind that according to what we have seen before, the existence of a Divinity is an "Hypothesis" at the opposite of the Evolution of Species for example that is a "Theory"!}...) even if we know, as we have already mention it, that science is (should) be driven only by data and peer-review publications.
	\begin{center}
		\includegraphics[scale=0.4]{img/intro/science_we_trust.jpg}
	\end{center}	

	We should distinguish between different scientific currents: 
	\begin{itemize}
		\item "\NewTerm{Realism}"\index{realism} is a doctrine where physical theories have the aim to describe reality as it is in itself, in its unobservable components. 
	
		\item "\NewTerm{Instrumentalism}"\index{instrumentalism} is a doctrine where physical theories are only tools to predict observations but do not describe reality itself. 
	
		\item "\NewTerm{Fictionalism}"\index{fictionalism} is the doctrine where the content repository (principles and postulates) of physical theories is just an illusion, useful only to ensure the linguistic articulation of the fundamental equations. 
	\end{itemize}

	Even if today the scientific theories are sponsored by many specialists, alternative theories have valid arguments and we can not totally dismiss them. However, the creation of the World in six or seven days as described in some books is difficult to accept, and many believers recognize that a literal reading of religious books is not compatible with the current state of our knowledge and that it is more prudent to interpret it as a parable (even if their own books forbids explicitly such an intellectual exercise...!). If Science never provides definitive answer, it is no longer possible to ignore it!

	Faith (whether religious, superstitious, pseudo-scientific or other not data driven) on the contrary is intended to provide absolute truths of a different nature as it is a personal unverifiable belief (for example, Science requires proofs to be evidence based, religions requires beliefs to be proved...). This is why many people say that \textit{Science adjusts views based on what's observed when Faith is the denial of observation so that belief can be preserved}.... In fact, one of the functions of religion is to give meaning to the phenomena that can not be explained rationally with actual knowledge\footnote{This was the case with the rain, the thunder, diseases, stars, comets, earthquakes, volcanic eruptions, birth, evolution, etc. a few hundred years ago and is often designated by scientists under the name of "argument of ignorance"\index{argument of ignorance}}. Progress of knowledge trough Science therefore cause sometimes (...) questioning the religious dogma\footnote{When theists say that a god must exist because Earth is "perfectly" place for life, they assume that their god is restricted to placing life where it could naturally occur anyway...}. 
	
	\begin{fquote}SCIENCE is not defined by what we believe. IT IS NOT a belief system. It has no religious doctrine! It's just an intellectual investigation method that rejects claims that have no supporting experimental reproducible evidence.
 	\end{fquote}
	\begin{center}
		\includegraphics[scale=0.5]{img/intro/science_like_religion.jpg}
	\end{center}
	Conversely, except try to impose his own faith (which is nothing but a subjective and intimate personal conviction) to others, we must defy the natural temptation to characterize scientifically proven fact extrapolations of scientific models beyond their scope.

	The word "Science" is, as we have already mentioned above, increasingly used to argue that there is a scientific evidence where there is only a belief (some web pages like this proliferate always more and more and especially to get followers and a lot of clicks on the Internet). According to its detractors it is, for example, the case of the movement of Scientology (but there are many others). According to them, we should rather speak about "\NewTerm{occult sciences}"\index{occult science}.

	The occult sciences and traditional sciences exist since antiquity; they consist on a series of mysterious knowledge and practices designed to penetrate and dominate the secrets of nature. Over the past centuries, they have been progressively excluded from science. The philosopher Karl Popper has for a long time questioned himself about the nature of the demarcation between science and pseudoscience. After noticing that it is possible to find observations to confirm almost any theory, he proposes a methodology based on falsifiability. A theory must according to him, to deserve the adjective "scientific", guarantee the impossibility of some events. It becomes therefore refutable, so (and only then) capable of integrating science. It would suffice to observe any of these events to invalidate the theory, and therefore take the way to improving it.
	
	And also let us notice that major difference between scientific textbooks and religious books is that if you destroyed that latter, in a thousand year's time that wouldn't come back just as it was. Whereas if we took every science book and every fact and destroyed them all, in a thousand years they'd all be back. Because all the same tests would lead to the same results again!
	\begin{fquote}The reason there is a conflict between science and religions is that science keep disproving the things religions claims to be true or accurate.
 	\end{fquote}
	\begin{tcolorbox}[title=Remark,colframe=black,arc=10pt]
	If science, and over everything MATHEMATICS are the language of a God, then why are all holy books written without maths? Oh, umh... we know why!
	\end{tcolorbox}
	
	\begin{center}
		\includegraphics[scale=0.51]{img/intro/scientific_method_vs_faith_method.jpg}
	\end{center}
	
	This is important because every one of the endless series of pseudo-proofs of the existence of various gods that has been proposed, from antiquity to the present day, is automatically a failure because, as already mentioned, a logical deduction tells us nothing that is not already embedded in its premises. All logic can do for us is test the self-consistency of those premises. There is only one reliable way that humans have discovered so far to obtain knowledge they do not already possess: observation! And science is the methodical collecting of observations and the building and testing of models to describe those observations. Without falsification science would be an anarchy of logically consistent but still useless models that simply suit someone's fancy!
	
	\begin{fquote}Religion is the practice of using nonsense, in order to explain ignorance.
 	\end{fquote}
	
	Furthermore, religion has had thousands of years to provide evidence that any god exist. Yet can theists manage no better than: \og \textit{YOU JUST WAIT UNTIL YOU'RE DEAD} \fg{}. That tells a lot... all the more because if gods were real they wouldn't need people to argue their existence! And if theists really value faith over science then they can prove it easily: the next time they get sick they can go to a church instead of going to the hospital!
	
	\begin{fquote}A scientist will read dozens of books in his lifetime, but still known that he has a lot more to learn. A religious fanatic barely reads one book, and think that they know it all.
 	\end{fquote}

	The reader must also know that at the opposite of a common myth by scientifically illiterate people, any phenomenon or assumption (even supernatural!) if well defined can be tested by the scientific method. That's why we have (among many others) scientific evidence that prayers don't provide any therapeutic effect better than a placebo. See for example about supernatural stuff the following paper \textit{Study of the Therapeutic Effects of Intercessory Prayer (STEP) in cardiac bypass patients: a multicenter randomized trial of uncertainty and certainty of receiving intercessory prayer} \cite{benson2006study} for a quite good scientific protocol and analysis or the following paper \textit{Positive therapeutic effects of intercessory prayer in a coronary care unit population} \cite{byrd1988positive} for a typical bad scientific protocol and analysis. There exist also a few meta-analysis on this supernatural topic...
	
	\begin{fquote}It doesn't matter if a great scientist was religious. What matters is that none of them ever proved their god exists!
 	\end{fquote}
 	
 	
 	\begin{figure}[H]
		\centering
		\includegraphics[width=1.0\textwidth]{img/intro/timeline_religions.jpg}
		\caption[Evolutionary tree of religions]{Evolutionary tree of religions (author: Simon E. Davies)}
	\end{figure} 
	Facts enrages believers, who prompts them to solipsism. Their only way to defend beliefs that are not concordant with scientific data is to challenge reality itself (that's why before debating with them scientists have to agree with them about a definition of "reality" otherwise any debate is pointless)! So every time you argue with believers, because the can't address the facts so they attack epistemology. As it is as if their position is so weak that their god couldn't be real unless reality isn't...
	
	\begin{center}
		\includegraphics[scale=0.5]{img/intro/see_hear_spout.jpg}
	\end{center}
	
	
	\pagebreak
	\subsection{Baloney detection kit}
	Through their training, scientists are equipped with what Carl Sagan name the "\NewTerm{baloney detection kit}\index{baloney detection kit}" or "\NewTerm{bullshit detection kit}\index{bullshit detection kit}" that is a set of cognitive tools and techniques that fortify the mind against penetration by falsehoods and to draw boundaries between science and pseudoscience. It isn't merely a tool of science, it contains invaluable tools of healthy scepticism that apply just as elegantly, and just as necessarily, to everyday life. By adopting the kit, we can all shield ourselves against clueless guile and deliberate manipulation. 

	There are many version of these detection tool but here is an quite complete one (but still incomplete by construction) a proposed by Michael Shermer (founding publisher of \href{http://www.skeptic.com}{<Skeptic Magazine} and author of \textit{The Borderlands of Science}):
	
	\begin{enumerate}[label=\protect\circledbullet{\arabic*},leftmargin=15mm]
		\item \textit{\textbf{How reliable is the source of the claim?}}

		Pseudo-scientists often appear quite reliable, but when examined closely, the facts and figures they cite are distorted, taken out of context or occasionally even fabricated. Of course, everyone makes some mistakes. And as historian of science Daniel Kevles showed so effectively in his book The Baltimore Affair, it can be hard to detect a fraudulent signal within the background noise of sloppiness that is a normal part of the scientific process. The question is, Do the data and interpretations show signs of intentional distortion? When an independent committee established to investigate potential fraud scrutinized a set of research notes in Nobel laureate David Baltimore's laboratory, it revealed a surprising number of mistakes. Baltimore was exonerated because his lab's mistakes were random and nondirectional... So in science, there are no authorities and therefore claims are also not proofs (believing in something does not make it true!!!). At most, there are experts!

		\item \textit{\textbf{Does this source often make similar claims?}}

		Pseudo-scientists have a habit of going well beyond the facts. Flood geologists (creationists who believe that Noah's flood can account for many of the earth's geologic formations) consistently make outrageous claims that bear no relation to geological science. Of course, some great thinkers do frequently go beyond the data in their creative speculations. Thomas Gold of Cornell University is notorious for his radical ideas, but he has been right often enough that other scientists listen to what he has to say. Gold proposes, for example, that oil is not a fossil fuel at all but the by-product of a deep, hot biosphere (micro-organisms living at unexpected depths within the crust). Hardly any earth scientists with whom I have spoken think Gold is right, yet they do not consider him a crank. Watch out for a pattern of fringe thinking that consistently ignores or distorts data.

		\item \textit{\textbf{Have the claims been verified by another source?}}

		Typically pseudo-scientists make statements that are unverified or verified only by a source within their own belief circle. We must ask, Who is checking the claims, and even who is checking the checkers? The biggest problem with the cold fusion debacle, for instance, was not that Stanley Pons and Martin Fleischman were wrong. It was that they announced their  spectacular discovery at a press conference before other laboratories verified it. Worse, when cold fusion was not replicated, they continued to cling to their claim. Outside verification is crucial to good science.

		\item \textit{\textbf{How does the claim fit with what we know about how the world works?}}

		An extraordinary claim must be placed into a larger context to see how it fits. When people claim that the Egyptian pyramids and the Sphinx were built more than 10,000 years ago by an unknown, advanced race, they are not presenting any context for that earlier civilization. Where are the rest of the artefacts of those people? Where are their works of art, their weapons, their clothing, their tools, their trash? Archaeology simply does not operate this way.

		\item \textit{\textbf{Has anyone gone out of the way to disprove the claim, or has only supportive evidence been sought?}}

		This is the "confirmation bias" (we will come back on cognitive biases in the section of Decision Theory page \pageref{cognitive bias}), or the tendency to seek confirmatory evidence and to reject or ignore disconfirmatory evidence. The confirmation bias is powerful, pervasive and almost impossible for any of us to avoid. It is why the methods of science that emphasize checking and rechecking, verification and replication, and especially attempts to falsify a claim, are so critical. 

		\item \textit{\textbf{Does the preponderance of evidence point to the claimant's conclusion or to a  different one?}}

		The theory of evolution, for example, is "proved" through a convergence of evidence from a number of independent lines of inquiry. No one fossil, no one piece of biological or palaeontological evidence has "evolution" written on it; instead tens of thousands of evidentiary bits add up to a story of the evolution of life. Creationists conveniently ignore this confluence, focusing instead on trivial anomalies or currently unexplained phenomena in the history of life.

		\item \textit{\textbf{Is the claimant employing the accepted rules of reason and tools of research, or have these been abandoned in favour of others that lead to the desired conclusion?}} 

		A clear distinction can be made between SETI (Search for Extraterrestrial Intelligence) scientists and UFOlogists. SETI scientists begin with the null hypothesis that ETIs do not exist and that they must provide concrete evidence before making the extraordinary claim that we are not alone in the universe. UFOlogists begin with the positive hypothesis that ETIs exist and have visited us, then employ questionable research techniques to support that belief, such as hypnotic regression (revelations of abduction experiences), anecdotal reasoning (countless stories of UFO sightings), conspiratorial thinking (governmental cover-ups of alien encounters), low-quality visual evidence (blurry photographs and grainy videos), and anomalistic thinking (atmospheric anomalies and visual misperceptions by eyewitnesses).

		\item \textit{\textbf{Is the claimant providing an explanation for the observed phenomena or merely 
             denying the existing explanation?}}
	
		This is a classic debate strategy-criticize your opponent and never affirm what you believe to avoid criticism. It is next to impossible to get creationists to offer an explanation for life (other than "God did it"). Intelligent Design (ID) creationists have done no better, picking away at weaknesses in scientific explanations for difficult problems and offering in their stead. "ID did it." This stratagem is unacceptable in science.

		\item \textit{\textbf{If the claimant proffers a new explanation, does it account for as many phenomena as the old explanation did?}}
	
		Many HIV/AIDS sceptics argue that lifestyle causes AIDS. Yet their alternative theory does not explain nearly as much of the data as the HIV theory does. To make their argument, they must ignore the diverse evidence in support of HIV as the causal vector in AIDS while ignoring the significant correlation between the rise in AIDS among haemophiliacs shortly after HIV was inadvertently introduced into the blood supply.

		\item \textit{\textbf{Do the claimant's personal beliefs and biases drive the conclusions, or vice versa?}}

		All scientists hold social, political and ideological beliefs that could potentially slant their interpretations of the data (this is a "confirmation bias" also named "cherry picking" that is also by non-scientists the main cause of rejecting science results and tools), but how do those biases and beliefs affect their research in practice? Usually during the peer-review system, such biases and beliefs are rooted out, or the paper or book is rejected.  
	\end{enumerate}
	
	Some humans and especially believers (not just religious believers but on any subject for which there is no experimental evidence beyond reasonable doubt) are good at what is named "concordism". The underlying idea of concordism is to interpret texts or find random analogies in Nature (obviously having zero knowledge of combinatorics and probabilities) to justify that all the inventions or discoveries of modern science have been announced in their "holy" books or created by people belonging to the same belief group as themselves hundreds or thousands of years before others. Obviously all this masked under a hazy rhetoric, analogies that make no sense (compare statistical studies with simple texts ...), no statistical methodology, omitting to quote contradictory facts or conclusions, and with a total absence of cross-referencing sources and peer review! This obscurantism at its finest!!!

	It must be noted that this concordist approach is the prerogative of people with little education or indoctrinated since childhood (plus those suffering from mental disorders for various reasons) who are not able to identify their own biases, who are closed to any contradiction, and who are not capable of identifying the real complexity (multifactoriality) and the nuances of even a simple subject (they tend to binarize everything in True/False).
	
	\begin{fquote}You can beat 100,000 scholars with one fact, but you will very likely never beat an idiot even with 100 facts.
 	\end{fquote}
	
	\begin{figure}[H]
		\centering
		\includegraphics[width=1.0\textwidth]{img/intro/baloney_detection_toolkit.jpg}
	\end{figure}
	
	By fine tuning we can go more far about reasoning fallacies. Here is a most exhaustive list:
	\begin{enumerate}
		\item Ad hominem: An ad hominem argument attacks the messenger, not the message itself.

		\item Argument from authority: Argument that relies on the identity of an authority rather than the components of the argument itself.

		\item Argument from adverse consequences: Saying that because the implications of a statement being true would create negative results, it must not be true.

		\item Appeal to ignorance: If something is not known to be false, it must be true.

		\item Special pleading: Stating a universal principle, then insisting that it doesn't apply to your assertions for some reason.

		\item Begging the question/ assuming the answer: This occurs when a statement has an unproven premise. It is also named "circular reasoning" or "circular logic".

		\item Observational selection: Looking at only positive evidence while ignoring the negative and vice versa.

		\item Statistics of small numbers: Using small numbers in order to report large percentage increases.

		\item Misunderstanding of the nature of statistics: 	
Ignorance about central statistical assumptions and the definition of metrics (the confusion of correlation and causation, the sample size and hate of maths bias are well known example).

		\item Post hoc, ergo propter hoc: Basing an effect on a cause only on the basis of chronology.

		\item Excluded middle, or false dichotomy: Portraying an issue or argument as having only two options and no spectrum in between.

		\item Short-term vs. long-term: Assuming a current trend has remained constant throughout its history and will continue to do so in the future, even though no evidence suggests such an extrapolation is justified.

		\item Slippery slope, related to excluded middle: Saying something is wrong because it is next to or loosely related to something wrong.

		\item Suppressed evidence and half-truths: Drawing an unwarranted conclusion from premises that are at least in part correct.

		\item Weasel words: The usage of vague, non-specific references.
	\end{enumerate}
	
	In addition to teaching us what to do when evaluating a claim to knowledge, any good baloney detection kit must also teach us what not to do. It helps us recognize the most common and perilous fallacies of logic and rhetoric. Many good examples can be found in religion and politics, because their practitioners are so often obliged to justify two contradictory propositions.


	Finally, we would like to quote Antoine Lavoisier: «\textit{The physicist may also, in the silence of his laboratory and his cabinet, perform patriotic functions; he can thanks to his works reduce the mass of evils which afflict happiness and, had he not, contributed by the new roads that he opened to himself, only to delay of a few years, of a few days, the average life of humans, he could also aspire to the glorious title of benefactor of humanity.}»
	\begin{center}
		\includegraphics[scale=0.1]{img/humour/evidence_based.jpg}	
	\end{center}
	
	\pagebreak
	\section{Scientific communication backfire}
	\lettrine[lines=4]{\color{BrickRed}A}nother point that is important to highlight about science communication: Scientists, stop thinking explaining science will fix things and avoid people bias especially if you find yourself in a state of disbelief or evidence probably drives you crazy as there are nowadays many conspiracy about flat Earth, vaccines, climate change, etc. as mainstream media don't know how to communicate scientific papers.

	The reasons are the following and applied outside the case where people come listen to you or to other scientists in the context of a conference or seminar:
	\begin{enumerate}
		\item Most people don't want to listen about scientific method especially when they never asked you "is it true?", "is it the best method?", "is this not a bias?". If you use "your science" just to point out they are wrong about what they are saying or arguing you will just take them out of their comfort zone and make them even more hate science.
		
		\begin{figure}[H]
			\centering
			\includegraphics[scale=0.35]{img/intro/scientists_arent_arrogant.jpg}	
		\end{figure}
		
		\item Most humans are full of bias and they don't like to admit is it true as they assume the human is the top species of evolution and therefore cannot have such biases. So when you explain them they have biases, you just point out that they are not reliable. Speak about bias only if people ask you to do so.
		
		\item The huge majority of people believe than their personal experience is more robust than the hundred of years of peer-review, tests, checks of the "scientific method" that has seems so far, if not THE best, at least the best one we know nowadays.
	\end{enumerate}
	\begin{fquote}No amount of peer-reviewed evidence will ever persuade an idiot, all the more if the latter is scientifically illiterate and doesn't master top level statistical calculations and the subtleties of the scientific method.
 	\end{fquote}
 	
	Now let us quote some paragraphs of an excellent \href{http://www.slate.com/articles/health_and_science/science/2017/04/explaining_science_won_t_fix_information_illiteracy.html}{{\color{blue} article}} of Tim Requarth as it is almost perfect:
	

	«The theory many scientists seem to swear by is technically known as the deficit model, which states that people's opinions differ from scientific consensus because they lack scientific knowledge. In 2010, Dan Kahan, a Yale psychologist, essentially proved this theory wrong. He \href{http://www.nature.com/nclimate/journal/v2/n10/full/nclimate1547.html}{{\color{blue} surveyed }} over 1,500 Americans, classifying each person's "cultural worldview" on a scale that roughly correlates with politically liberal or conservative. He then assessed each person's scientific literacy with questions such as "True or False: Electrons are smaller than atoms". Finally, he asked them about climate change. If the deficit model were correct, Kahan reasoned, then people with increased scientific literacy, regardless of worldview, should agree with scientists that climate change poses a serious risk to humanity.
  
	That's not what he found. Instead, Kahan found that increased scientific literacy actually had a small negative effect: The conservative-leaning respondents who knew the most about science thought climate change posed the least risk. Scientific literacy, it seemed, increased polarization. In a later study, Kahan added a twist: He asked respondents what climate scientists believed. Respondents who knew more about science generally, regardless of political leaning, were better able to identify the scientific consensus-in other words, the polarization disappeared. Yet, when the same people were asked for their own opinions about climate change, the polarization returned. It showed that even when people understand the scientific consensus, they may not accept it.
	
	\begin{fquote}[Average social network user]Having no postgraduate education on this subject, nor any biases, and having done no research in a lab at a professional level whatsoever, I don't understand it. Therefore, it doesn't make sense and is false!
 	\end{fquote}

	The takeaway is clear: Increasing science literacy alone won't change minds. In fact, well-meaning attempts by scientists to inform the public might even backfire. Presenting facts that conflict with an individual's worldview, it turns out, can cause people to dig in further. Psychologists, aptly, dubbed this the "backfire effect".
	\begin{figure}[H]
		\centering
		\includegraphics[scale=0.35]{img/intro/explain_science.jpg}
		\caption[]{source: \url{https://www.ratbotcomics.com}, author: Dr. Jones}
	\end{figure}
	If scientists simply want to explain science to a curious audience, disseminate their research more broadly, or write for fun, this doesn't matter much. But if scientists are motivated to change minds-and many enrolled in science communication workshops do seem to have this goal-they will be sorely disappointed.

	That's not to say scientists should return to the bench and keep their mouths shut. They should just realize that closing the "information gap" isn't the goal. And instead, they need to learn how to communicate science strategically.

	There are obvious reasons why science communication is a necessary and worthwhile endeavour, but a huge one is that there's a politically motivated push to destabilize scientific authority. At a Heartland Institute conference last month, Lamar Smith, the Republican chairman of the House science committee, told attendees he would now refer to "climate science" as "politically correct science", to loud cheers. This lumps scientists in with the nebulous "left" and, as Daniel Engber pointed out here in Slate about the upcoming March for Science, rebrands scientific authority as just another form of elitism.

	Is it any surprise, then, that lectures from scientists built on the premise that they simply know more (even if it's true) fail to convince this audience? Rather than fill the information deficit by building an arsenal of facts, scientists should instead consider how they deploy their knowledge. They may have more luck communicating if, in addition to presenting facts and figures, they appeal to emotions. This could mean not simply explaining the science of how something works but spending time on why it matters to the author and why it ought to matter to the reader. Research also shows that science communicators can be more effective after they've gained the audience's trust. With that in mind, it may be more worthwhile to figure out how to talk about science with people they already know, through, say, local and community interactions, than it is to try to publish explainers on national news sites. And they might consider writing op-eds for their local papers, focusing on why science matters to their particular communities.

	Scientists can also learn to avoid certain pitfalls. I spoke with Gretchen Goldman, research director of the Union of Concerned Scientists' Center for Science and Democracy, which offers communication and advocacy workshops. A counter-intuitive lesson she's learned is that refuting stories that deny climate change by addressing each claim and explaining why it's wrong is not that productive. In fact, it could be counter-productive: "If you repeat the myth, that's the part people remember even if you immediately debunk it", she says. A better approach, she suggests, is to reframe the issue. Don't just keep explaining why climate change is real, explain how climate change will hurt public health or the local economy. Communication that appeals to values, not just intellect, research shows, can be far more effective.

	[...] But the obstacles faced by science communicators are not epistemological but cultural. The skills required are not those of a university lecturer but a rhetorician.

	So it's an admirable goal to communicate about science, but almost certainly destined to fail. This is because the way most scientists think about science communication - that just explaining the real science better will help - is quite wrong. It's so wrong that it may have the opposite effect of what they're trying to achieve. [...]»
	
	\begin{fquote}Silence is sometimes the best answer to the fools.
 	\end{fquote}
	
	\begin{center}
		\includegraphics[scale=0.25]{img/intro/lies_vs_comfort.jpg}	
	\end{center}
	
	\pagebreak
	\section{Is Science dogmatic?}
	We will here mainly repeat things that we have already mentioned earlier above. However as some scientifically illiterate people still think in this beginning of the 21st century that YouTube videos containing some rhetorical monologue\footnote{As already said people should not trust monologues - whether on radio, television or any social network - because there are no experts to counter the possible wrong arguments or ill-defined concepts that may lead a part of the audience to wrong speculative interpretations! In addition, humans under the stress of knowing that they are recorded are naturally prone to vocabulary errors and that's without counting the more than 200 cognitive biases in the brain which sometimes lead to the erroneous simplification of complex thoughts... !} or some books without proofs and without statistical data analysis constitute some sort of "evidence" it is necessary maybe to come back on a few topics but with a different perspective\footnote{And please don't forget that the purpose of Science is not the "Truth" but it's only a tool that explains quite well things that we see or feel following the best actual models. And also please don't forget that quoting a book, a famous scientist, a blog or a YouTube video is only at best an evidence of level 2 or 3!}!
	
	For Science, if something exist with evidence beyond reasonable doubt (BRD), at its actual state of knowledge, then it means it can be measured! If it cannot be measured or is ill-defined, well, then science can't provide evidence that it doesn't exist (don't forget that the purpose of Science is to refute models and if it fails to do so a long time enough then a Model may take the status of Theory!). In a simplistic and slightly out of context metaphor, this is equivalent to say that for the blind (taking away all their other senses) the world doesn't exist with strong evidence because they cannot see it nor feel it. Some people then say that Science is materialist! However this doesn't mean at all that Science has failed as an investigation method, however it just means that Science knows it doesn't know everything otherwise it would stopped and that the scientific method must be corrected constantly according to the new available evidences.
	
	Typically Science don't reject completely parapsychology powers or the existence of one of the thousand of deities created around the World by humans even if actually all experimental tests have rejected them with strong evidence (but not definitively!). The advocates of parapsychology or religions (or of some alternative medicines) may argue that Science can't prove nor disprove what it can't measure directly or indirectly. And they are very likely absolutely right! Science can only \underline{fail to reject with a given level of evidence} if something exist or not \underline{at its actual level of knowledge}. So if someone argue that flying unicorns exist without providing a protocol so that thousands of other people can check this fact in a reproducible way... then Science (scientists) humbly say us that: \textit{we don't have any evidence beyond reasonable doubt at our actual state of knowledge that unicorn exist or not}!
	
	So it's not Science that is dogmatic, nor the huge majority of the scientific community. But only some bad educated and biased scientists (nobody is perfect...).
	
	\begin{figure}[H]
		\centering
		\includegraphics[width=0.9\textwidth]{img/intro/science_dogmatic.jpg}
	\end{figure}
	
	Some humans don't like that scientists may not know everything about something, or that scientists do errors, and that science takes time and that even worst, these same people will maybe never have an answer to their main questions before their death. But that's how Science works actually!!! If someone found a more reliable and fast way to investigate the Universe and its phenomena than the actual state of the Scientific Method, then the majority of the scientific community - especially the academic part - is waiting to check it and if it works indeed better to adopt it!
	
	\begin{fquote}[Carl Sagan]Extraordinary claims require extraordinary evidence.
 	\end{fquote}

	 So some people may ask Science to be truly scientific, meaning, it should question itself regarding the certainty of their own postulates and instruments. But as we already said many times in the previous sections above, this is what researchers do in their daily work!!! They try to found new evidence to reject actual theories, models, methods, postulates or bad instruments... otherwise Science would stop and scientists would lost their job. However the process is slow and takes days, weeks, months, years, decades, centuries and sometimes even millenniums.
	 
	\begin{tcolorbox}[title=Remark,colframe=black,arc=10pt]
	Some people point out that the version of the Universe scientists currently have is only what their instruments, and especially their imagination allows them to understand. And they are right! We - all professional academic researchers - know that fact since centuries in Science and that's why every day we try to push the boundaries of knowledge (hence also of imagination) and to develop new instruments and methods to measure things that were not even known a few decades before! Keep in mind this is what we are paid for! If there is nothing new to discover we will all lost our jobs!\\
	
	So yes in its current state, Science cannot account for consciousness phenomena, synchronicity, or even near death experiences or spontaneous epiphanies, etc. But well educated professionals scientists won't claim they do or don't exist because we don't have yet any experimental reproducible evidence to support any of these positions! Scientists are just waiting for those who claim that such things exist - and even run quite good business with it - to provide us reproducible experimental evidence. Sadly so far all the people who made such claim and run business with them have failed to provide any evidence or were just simply debunked and that's all.\\
	
	Anyone should ask himself why evangelist pastors or any kind of miraculous healer (...) don't practice their art in the scientific controlled environment of  Hospitals rather that in their church (in front of a public in the huge majority biased, uneducated and scientifically illiterate) or with a small group of unknown and suspicious people behind a camera.......
	\end{tcolorbox}
	 
	People like Rupert Sheldrake, a PhD holder (from Cambridge) in biochemistry and retired researcher in the field of parapsychology who proposed the concept of morphic resonance... and also author of numerous rewarding books...... explores ten dogmas of Science that should be reconsidered according to his personal and subjective opinion... 
	
	Let us present below these ten dogmas and... we will comment them as Dr. Sheldrake doesn't have a PhD in Physics, nor in Cosmology but has a good sense of rhetoric especially when he does a monologue in front a public of neophytes (the reader and Dr. Sheldrake will be able to found the mathematical proof of our answers in the $8,000$ pages of this book if they are curious - as at the opposite of Dr. Sheldrake we like proofs and experimental evidence more than rhetorical monologue...):
	\begin{enumerate}
		\item \og Nature is mechanic: all the creatures and systems of nature are robots made to follow a given genetic program. \fg{}
		
		$\vartriangleright$ Our comment: We are not sure of how the word "Mechanics" is defined in this sentence (the basics of a debate is to agree on the meaning of words...) but obviously a biological system cannot be compared to a mechanical one (mechanical system doesn't evolve). However as we will prove it in this book Nature is Information based at all levels, Probabilistic based at the microscope level, and behaves Statistically and Mechanically at macroscopic level. So as most of the time in Science..., it's not as easy as it seems to be (it seems that Dr. Sheldrake has a binary vision of the World and of the Universe that is quite surprising given his supposed level of education).
		
		\item \og Matter is unconscious: Plants, stars, animals and elements are material things that are and cannot have a consciousness of themselves. \fg{}
		
		$\vartriangleright$ Our comment: First we should ask how is "consciousness" defined by Dr. Sheldrake. Secondly, who asserted that in the scientific community? Is there a written scientific consensus on this subject or it comes out from the imagination of Dr. Sheldrake?

		\item \og The laws of nature are fixed: At the moment of the Big Bang all the necessary constants until the end of time were established. The habits of nature do not evolve. \fg{}
		
		$\vartriangleright$ Our comment: What scientific consensus stated that? That's very likely wrong as actually we have a few mathematical-physics models that proves that the constants of the Universe may have changed and also experimental evidence that maybe the laws of Nature may have changed. But one thing is almost sure: there is no scientific consensus yet on that topic!

		\item \og The amount of nature and energy in the Universe is always the same. \fg{}
		
		$\vartriangleright$ Our comment: The scientific community has strong evidence beyond reasonable doubt of that statement for the \underline{observable} Universe. However the dynamic of the Universe disprove energy conservation at large scales (this is derived from Noether theorems!). Notice that we have however no evidence yet of this statement of energy conservation for the whole Universe (the observable and not observable one).

		\item \og Nature has no purpose: there is no design in nature in terms of intention and the process of evolution is mechanic. \fg{}
		
		$\vartriangleright$ Our comment: We have strong evidence beyond reasonable doubt indeed that there is no design as the existing design are flawed in many ways. And the process of evolution as proven mathematically in this book and experimentally (beyond any reasonable doubt) in laboratories is not mechanical but stochastic.

		\item \og Biologic Heritage: The plans to produce a living being are composed within the physical matter lodged in their genes. \fg{}
		
		$\vartriangleright$ Our comment: That's not completely correct. Experimental observations provide us that some basics of the plans are random and influenced by external and internal modifications. Again Dr. Sheldrake gives a binary view of a phenomenon which is far more complex. But this is something we know as typical of scientific illiterate people: they reduce something complex to something simple because they can't grasp the complexity of the World and of the Universe.

		\item \og Memory is kept in the brain as material prints: memory is made of proteins and nerve endings organized as a drawer within itself. \fg{}
		
		$\vartriangleright$ Our comment: If it was the case we wouldn't forget things... That's because the brain is far more complex and involves probabilities, Bayesian and stochastic process that we know why the human brain forgets things and has sometimes building issues...

		\item \og Mind is in the head: The mind has a physical connection with the head and the brain, relegating intellectual subordination on the rest of the body. \fg{}
		
		$\vartriangleright$ Our comment: What is the "Mind" for Dr. Sheldrake? As far as we know there is no scientific consensus about its definition. Furthermore is Mind what we observe in MNR scanners?

		\item \og Phenomena like telepathy are impossible: thoughts have no effect in the world because of number 8 on the list (the mind is in the head). \fg{}
		
		$\vartriangleright$ Our comment: What scientific consensus or community stated that? Actually Science has no evidence that telepathy works or exist, yes (!) - but no well educated scientist would say it is "impossible" (by the way any well educated scientist know that is better to avoid using the word "impossible" about an unknown future or unmeasurable phenomena).

		\item \og Only mechanical medicine works: It is merely by chance or placebo effects that traditional healing practices or natural remedies have any effect on people's health. \fg{}
		
		$\vartriangleright$ Our comment: Again... what scientific consensus or community stated that? That's not accurate. Maybe scientists agree on the fact that actually not other method than the Scientific Medicine has provided better odds ratio than other medicines. But if one day some people provide robust evidence that traditional healing practices or natural remedies work, then almost surely they will be promoted by the scientific academic community.
	\end{enumerate}
	
	If this is the best support Dr. Sheldrake has for his position, then this means that there is no support. It is entirely based on gish gallop and speculation, pointing to common misconceptions and popular opinions rather than evidence and examples of "dogma" in science.
	
	We can understand why someone reading or listening this kind of gish gallop may think that there is some basis of evidence. Gish gallop means making loads of claims without evidence, leaving the opposition the large job of checking each and every claim. Even so, when someone claims the "establishment" is dogmatic, immoral or whatsoever, we may sincerely hope that any reader will critically assess that person claims.
	
	\begin{figure}[H]
		\centering
		\includegraphics[width=0.7\textwidth]{img/intro/false_equivalence.jpg}
	\end{figure}
	
	\begin{flushright}
	Section quality score: \score{4}{5} 151 votes, 75.23\%
	\end{flushright}
	

\chapter{Arithmetic}

	\textit{\textbf{Mathematics is the ultimate form of forced art.}} (unknown)
	\minitoc
	\input{Chapter_Arithmetic.tex}
	
	
  \chapter{Algebra}

	\textit{\textbf{Algebra is the science of calculating the quantities or structures represented by letters.}} (Larousse)
	\minitoc
	\input{Chapter_Algebra.tex}
	
		
 \chapter{Analysis}

	\textit{\textbf{The analysis is the rigorous formulation of calculus.}}(Wikipedia)
	\minitoc
	\pagebreak
		%to force start on odd page
	\newpage
	\thispagestyle{empty}
	\mbox{}
	\section{Functional Analysis}\label{functional analysis}
\lettrine[lines=4]{\color{BrickRed}F}unctional analysis is the branch of mathematics and specifically of the analysis that is related to the study of function spaces. It takes its historical roots in the study of transformations such as the Fourier transform and in the study of differential equations and integrals. As such it encompasses so many areas that it is difficult to justify that it be a section of this book because it is rather a field of study. Moreover, it is because of this difficulty to accurately identify the area it covers that the reader will find the Fundamental Theorem of Calculus in the chapter of Integral and Differential Calculus rather than here... 

	Why do we use the term "analysis" in the particular case of functions? The reason lies in the historical study of various phenomena of nature and resolution of various technical problems and therefore mathematics, which often lead us to consider the variation of a parameter correlated with the variation of another or several other variables. To study these variations, many tools are available to each of us:
\begin{itemize}
	\item The engineer, for example, frequently use charts (in cartesian, polar or logarithmic coordinate system... concepts which are discussed further in more detail) to determine the mathematical relations (or "law") linking variables between them. Certainly, this kind of method is (sometimes ...) aesthetic but students know well how it is sometimes painful to transcribe measures points on a sheet of paper or on a computer and consultants know how dangerous can be a chart when not build in a scientific way. This is unfortunately a necessary step (but should avoid an abusive usage) to understand how our predecessors worked and got the results that help us today in our advances in theoretical physics.
	
	\item The mathematician and theoretical physicist usually hate to use the paper-pencil-scrawl methods. Nevertheless, the role of the mathematician or physicist is to develop new theories with mathematical axioms or principles which should require no usage of graphical representation nor access to experimental  measures that are often attached to it.
\end{itemize}

	\begin{tcolorbox}[title=Remark,colframe=black,arc=10pt]
Before starting to read what follows, it may be useful to remind the reader that the definition of the concept of "function" (and the basic properties thereto) is given in the section on Set Theory.
	\end{tcolorbox}	

	Function analysis is also strongly linked to Vector Calculus (and not only...). Thus for people who want to increase their knowledge about the fundamentals of function analysis we strongly recommend the reader to have a look to the Vector Calculus section.

\pagebreak
\subsection{Representations}

	We will see in what follows, firstly, how to represent different values related by tables and charts (yes! We must because it helps to understand more complicated stuff) and secondly how to mathematically analyse the properties of these representations only by using abstract mathematical tools.

	\textbf{Definition (\#\mydef):} A function is named "\NewTerm{univalent function}\index{univalent function}" or  "\NewTerm{unary function}\index{unary function}" if the number of its arguments (parameters or variables) is equal to one. In the case of a function of two arguments, we speak about a "\NewTerm{bivalent function}\index{bivalent function}" or "\NewTerm{bivalent function}\index{bivalent function}", etc. Formally a function is $n$-ary if:
	

\subsubsection{Tabular Representation}

Among the possible visual representation of functions, the most intuitive and the oldest is the one where we have in the column or the row of a table in an orderly way the values of the independent variable $x_1,x_2,...,x_n$ and the corresponding values, namely the "\NewTerm{transformed variables}\index{transformed variables}" of the function $y_1,y_2,...,y_n$ in another column or aligned row:

	\begin{table}[H]
	\begin{center}
		\definecolor{gris}{gray}{0.85}
			\begin{tabular}{|p{2cm}|p{2.5cm}|}
				\hline
				\multicolumn{1}{c}{\cellcolor{black!30}\textbf{$x$}} & 
  \multicolumn{1}{c}{\cellcolor{black!30}\textbf{$y=f(x)$}} \\ \hline
				\centering\arraybackslash\ $x_1$ & \centering\arraybackslash\ $y_1=f(x_1)$ \\ \hline
				\centering\arraybackslash\ $x_2$ & \centering\arraybackslash\ $y_2=f(x_2)$  \\ \hline
				\centering\arraybackslash\ $...$ & \centering\arraybackslash\ $...$  \\ \hline
				\centering\arraybackslash\ $x_n$ & \centering\arraybackslash\ $y_n=f(x_n)$  \\ \hline
		\end{tabular}
	\end{center}
	\caption[]{Values and corresponding transformed variables}
	\end{table}	
	
	In the expression:
	
	we say that the $a_1,a_2,...,a_n$ are the "\NewTerm{arguments}\index{arguments}" of $f$.

Such are, for example, tables of trigonometric functions, logarithmic tables, etc. and during the experimental study of certain phenomena tables which express the existing functional dependence between the measured physical quantities such as the readings of the temperature of the air stored in a meteorological station during one day.

Of course, this concept can be generalized to any multivalent function regardless its definition domain.

However, this method is laborious and does not permit to directly see the behaviour of the function and therefore a simple and attractive visual analysis of its qualitative properties. It still has the advantage of not requiring any special tools or advanced mathematics.

	\pagebreak
	\subsubsection{Graphical Representation}
	The natural, relative, real or purely imaginary numbers  (\SeeChapter{see section Numbers page \pageref{type of numbers}}) can all be represented as simply by points on a numerical infinite axis (straight line).

	To this purpose, we choose on this axis:
	\begin{enumerate}
		\item A point O named "\NewTerm{origin}\index{origin}"
		\item A positive direction, that we indicate by a horizontal arrow
		\item A unit of measure (usually represented by small vertical lines: the "\NewTerm{graduation}\index{graduation}")
	\end{enumerate}
Such that:
\begin{figure}[H]
\centering
\includegraphics[scale=0.75]{img/analysis/representative_1d.eps}
\caption{Typical representation example of an oriented infinite axis with origin}
\end{figure}
	In most cases we put (traditionally) the axis horizontally and choose the direction from left to right (at least when there is only on axis...).
	\begin{tcolorbox}[title=Remark,colframe=black,arc=10pt]
	The point (letter) $O$, frequently represents the number zero in mathematics but we might very well choose to put the origin elsewhere. For example, in physics, the point $O$ is often positioned at the location of the centroid of a system. 
	\end{tcolorbox}
	It is obvious that the fact that the sets of numbers that we discussed in the section Numbers are ordered implies that every number is represented by a single point on this axis. Thus, two distinct real numbers correspond two different points on the axis.

	Thus, there is a correspondence between all numbers and all the points of the axis (in the case of real or complex numbers, it corresponds not a number to each graduation, but a number at each \underline{point} of the axis!). Thus, each number represents a point or a unique graduation and back to each point or graduation is a single number which is the image.

\pagebreak
\paragraph{2D representations}\mbox{}\\\\
There are besides the one dimensional representations, other of higher dimensions (phew!...) like the "\NewTerm{planar representation}\index{planar representation}" that allow us to draw much more than simple points on a one-dimensional straight line but functions of one variable (but also points!). Let's see what this is and looks like:
\begin{figure}[H]
	\centering
	\includegraphics[scale=0.5]{img/analysis/cartesian_plane.jpg}
	\caption{Position of a point in a cartesian plane}
\end{figure}
In the above figure we have added two graduation that helps us to identify uniquely the position of the point $P$ given by $P(x,y)=P(+4,+3)=P(4,3)$. We then speak of "cartesian coordinates of a point".

In single-variable calculus, the functions that one encounters are functions of a variable (usually $x$ or $t$) that varies over some subset of the real number line (which we denote by $\mathbb{R}$). For such a function, say, $y = f (x)$, the graph of the function $f$ consists of the points 
	
These points lie in the Euclidean plane, which, in the Cartesian or rectangular
coordinate system, consists of all ordered pairs of real numbers $(a,b)$. We use the word "Euclidean" to denote a system in which all the usual rules of Euclidean geometry hold (\SeeChapter{see section Euclidean Geometry page \pageref{hilbert axioms}}). We denote the Euclidean plane by $\mathbb{R}^2$. where the exponent "$2$" represents the number of dimensions of the plane.

Thus we can for each of a variable $x$ on a horizontal axis, named commonly "\NewTerm{$x$-axis\index{$x$-axis}}" match a value $y$ through a function $f$ such that:
	
plotted on a vertical axis, named commonly the "\NewTerm{$y$-axis}\index{$y$-axis}" which passes through the junction defined by the origin $O$ such tat (arbitrary example):
\begin{figure}[H]
	\centering
	\includegraphics[scale=0.75]{img/analysis/representative_2d_planar.eps}
	\caption{Typical example of a planar representation with orthogonal axes, origin O and the $4$ quadrants}
\end{figure}
All points of the plane (that latter being denoted with the variations $X\text{O}Y$, $XY$ or $x\text{O}y$, $\text{O}xy$, $xy$) have for "\NewTerm{abscissa}\index{abscissa}" traditionally the $x$-values corresponding to the independent variable (horizontal axes by tradition) of the function and for "\NewTerm{ordinate}\index{ordinate}" the corresponding value of the function (vertical axes by tradition). All these generated what is named the "\NewTerm{planar graph}\index{planar graph}" of the function. If there is no confusion, we just say "\NewTerm{graph}\index{graph}".
\begin{figure}[H]
	\centering
	\includegraphics[scale=0.4]{img/analysis/vocabulary_planar_graph.jpg}
	\caption{Vocabulary planar graph}
\end{figure}
An interesting detailed example for middle schools students that may help is the planar representation of the following line equation:

That give in the range $x\in [-3.3,+3.3]$ the following graph:
\begin{figure}[H]
	\centering
	\includegraphics[scale=0.5]{img/analysis/line_equation_plan_representation.jpg}
\end{figure}
with its equivalent tabular representation for some arbitrary chosen points:
\begin{table}[H]
	\begin{center}
		\definecolor{gris}{gray}{0.85}
			\begin{tabular}{|p{2cm}|p{3.5cm}|p{2.5cm}|}
				\hline
				\multicolumn{1}{c}{\cellcolor{black!30}\textbf{$x$}} & 
  \multicolumn{1}{c}{\cellcolor{black!30}\textbf{$2x-3$}} &  \multicolumn{1}{c}{\cellcolor{black!30}\textbf{Point $(x,2x-3)$}}\\ \hline
				\centering\arraybackslash\ $-1$ & \centering\arraybackslash\ $2\cdot(-1)-3=-5$ & \centering\arraybackslash\ $(-1,-5)$ \\ \hline
				\centering\arraybackslash\ $0$ & \centering\arraybackslash\ $2\cdot (0)-3=-3$ & \centering\arraybackslash\ $(-1,-5)$  \\ \hline
				\centering\arraybackslash\ $+1$ & \centering\arraybackslash\ $2\cdot(+1)-3=-1$ & \centering\arraybackslash\ $(-1,-5)$  \\ \hline
				\centering\arraybackslash\ $+3$ & \centering\arraybackslash\ $2\cdot(+3)-3=+3$ & \centering\arraybackslash\ $(-1,-5)$  \\ \hline
		\end{tabular}
	\end{center}
	\caption[]{Values and corresponding transformed variables}
	\end{table}	
In the case of representation by a rectangular coordinate system (cartesian, polar or logarithmic) as the figure above, we can see that the entire coordinate plane is divided into four areas that by tradition we name "\NewTerm{quadrants}\index{quadrants}" as already mentioned just earlier.

	\begin{tcolorbox}[title=Remark,colframe=black,arc=10pt]
	When we wish to highlight a particular point on the graph representing the function, we draw most of time a small round as presented in the prior-previous figure for the point of coordinates $(x_n,y_n)$.
	\end{tcolorbox}	

Another classic case of plane graph representation  known by a large number of students is the plot of polynomials (\SeeChapter{see section Calculus page \pageref{polynomial}}) with real coefficients or trigonometric functions (\SeeChapter{see section Trigonometry page \pageref{trigonometry}}).

Indeed, to solve polynomial equations of the second degree (\SeeChapter{see section Calculus page \pageref{second order polynomials}}), it is common in small classes that the teacher asks his students in addition to give an algebraic expression of the roots of:
	
given by for recall (see section Calculus page \pageref{double root} for the proof):
	
a graphics resolution where the two roots (in the case where there are two distinct real roots) are given by the intersection of the parabola with the $x$-axis (of course, if the equation has no solution, there are no intersections...):

\begin{figure}[H]
\centering
\includegraphics[scale=0.75]{img/analysis/roots_parabola.eps}
\caption{Representation of roots on a planar graph}
\end{figure}

The graphical representation can be generalized to polynomial equations of the 3rd, 4th and 5th degree (we will prove much further, using Galois theory that it is not possible to get a general algebraic expression of the roots of a polynomial equation of the 5th degree and higher).

There is another well-known and interesting example of special graph because when most young people think that after high-school they will never do maths again, in Switzerland many employees are faced to calculate in spreadsheet softwares what we name the "coordinate wage" that is a "\NewTerm{step-wise function}\index{step-wise function}" defined in year 2013 by the government as:

	

Where $R$ is a minimal value defined also by the government as being equal to 25,800.- in 2013 and the wage is denoted by the letter $S$ (for \textbf{S}alary).

When we plot such a stepwise function with for example Maple 4.00b we get:

\begin{figure}[H]
\centering
\includegraphics[scale=0.6]{img/analysis/step_wise_function.eps}
\caption{Example of step-wise function for swiss coordinate wage with Maple 4.00b}
\end{figure}

And therefore it is obvious thank to this chart representation that the previous definition can be simplified as:

	

	That is much easier to write in any spreadsheet software or also with Maple 4.00b:

\texttt{>R:=258000;}\\
\texttt{>plot(min(max(R/8,S-7/8*R),17/8*R,S=3/4*R..100000);}

	Also, graphs\index{graphs}\index{charts} are as we know powerful qualitative tools in the field of statistics (\SeeChapter{see section Statistics page \pageref{statistics}}) but also of data mining (\SeeChapter{see section Numerical Methods page \pageref{data mining}}) as a starting point for data analysis (histograms, cheese, box plots, radar, scatter plots, etc.). The assumptions and ideas that are generated by graphical analysis can be investigated with advanced statistical tools (for a few hundred of examples see the \texttt{R} or MATLAB™ softwares companion book).

Below for example, a graph (histogram) taken from the Industrial Engineering section that is very common in the field of statistics and project management in the global industry:

\begin{figure}[H]
\centering
\includegraphics[scale=0.75]{img/analysis/six_sigma.eps}
\caption{Example of typical histogram in engineering companies (Six Sigma)}
\end{figure}

Histograms allow to observe distributions and determine qualitatively if it fits a particular theoretical model.

Graphics can also be used to observe changes over time (time series, control charts, residual analysis, etc.):

\begin{figure}[H]
\centering
\includegraphics[scale=0.75]{img/analysis/time_serie.eps}
\caption{Example of OHLC time series with moving averages in financial trading}
\end{figure}
or another type of OHLC (Open-High-Low-Close) trading plot:
\begin{figure}[H]
\centering
\includegraphics[scale=0.75]{img/analysis/plot_OHLC.jpg}
\end{figure}
There are different rules for the colors of an OHLC plot! We can first define a color depending if the close price is lower than the open price (the open price is always on the left and the closure price always on the right):
\begin{figure}[H]
\centering
\includegraphics[scale=0.75]{img/analysis/plot_OHCL_color_first_rule.jpg}
\end{figure}
Or with the following rules:
\begin{figure}[H]
\centering
\includegraphics{img/analysis/plot_OHCL_color_second_rule.jpg}
\end{figure}
And still many other charts... that we have already seen and other we will see throughout the pages of this book.

\paragraph{3D representations}\mbox{}\\\\
Of course, in the case of a trivalent function (three-dimensional), that is to say a parameter which depends on two other, the idea is the same as for 2D except that the number of quadrants doubles:

\begin{figure}[H]
\centering
\includegraphics[scale=0.75]{img/analysis/quadrants_3d.eps}
\caption[Quadrants in a 3D orthogonal system]{Quadrants in a 3D orthogonal system (source: Wikipedia)}
\end{figure}

This 3D method of representation and analysis of a trivalent function was time consuming at the beginning of the 20th century but with the help of computers in the end of the 20th century this time consuming problem was almost solved...

In 3D functional analysis, we will deal with functions of two or three variables (usually  $x, y, z$, respectively). The graph of the arrow of coordinates $(x, y, z)=(x,y,f(x,y))$, lies in Euclidean space. Since Euclidean can be 3-dimensional (and more or less for sure!), we denote it by $\mathbb{R}^3$.

Euclidean space has three mutually perpendicular coordinate axes ($x$, $y$ and $z$), and three mutually perpendicular coordinate planes: the $xy$-plane, $yz$-plane and $xz$-plane:

\begin{figure}[H]
\centering
\includegraphics[scale=0.75]{img/algebra/euclidian_planes.eps}
\caption{Mutually perpendicular planes in $\mathbb{R}^3$}
\end{figure}

The coordinate system shown above is known as a right-handed coordinate system, because it is possible, using the right hand, to point the index finger in the positive direction of the $x$-axis, the middle finger in the positive direction of the $y$-axis, and the thumb in the positive direction of the $z$-axis, as below:

\begin{figure}[H]
\centering
\includegraphics[scale=0.75]{img/algebra/right_hand.eps}
\caption{Right hand system}
\end{figure}

What we are going to represent now further below (special example), purists mathematicians would notice it as follows (it's nice to have seen at least once this notation as you could meet it in other books):
	
and let us see what it gives with  Maple 4.00b:

\texttt{>restart:}\\
\texttt{>with(plots):}\\
\texttt{>f:=(x,y)->12*x/(1+x\string^ 2+y\string^ 2);}\\
\texttt{>xrange:=-10..10;yrange:=-5..5;}\\
\texttt{>plot3d(f,xrange,yrange);}

This will give:

\begin{figure}[H]
\centering
\includegraphics[scale=0.75]{img/analysis/representation_grid_function.eps}
\caption{Grid representation of a 3D function with Maple 4.00b}
\end{figure}

Let us improve the visual by adding a shading interpolation color with warm color to high positions and cold colors to low positions:

\texttt{>plot3d(f,xrange,yrange, style=patchnogrid, grid=[80,50], shading=ZHUE, axes=FRAME, tickmarks=[3,3,3], labels=[`x`,`y`,`f(x,y)`], labelfont=[TIMES,BOLD,12], title=`Graphique rempli`, titlefont=[TIMES,BOLD,12], scaling=unconstrained, orientation=[-107,68]);}

This will give:

\begin{figure}[H]
\centering
\includegraphics[scale=0.6]{img/analysis/representation_shading_interp_function.eps}
\caption{Isolines representation of a 3D function with Maple 4.00b}
\end{figure}

Let us plot now the "\NewTerm{contour lines}\index{contour line}", also named "\NewTerm{isoline}\index{isoline}\label{isoline}" (or "\NewTerm{isoquant}\index{isoquant} in Econometry), that represents lines of the same height on the function surface\footnote{It is a cross-section of the three-dimensional graph of the function $f(x, y)$ parallel to the $x, y$ plane. In cartography, a contour line (often just named a "contour") joins points of equal elevation (height) above a given level, such as mean sea level. A contour map is a map illustrated with contour lines, for example a topographic map, which thus shows valleys and hills, and the steepness of slopes. The contour interval of a contour map is the difference in elevation between successive contour lines.} (see section of Differential Geometry page \pageref{isoline} for a rigorous definition):

\texttt{>plot3d(f,xrange,yrange,style=patchcontour);}

This will give:

\begin{figure}[H]
\centering
\includegraphics[scale=0.75]{img/analysis/representation_isoline.eps}
\caption{Shading interpolation representation of a 3D function with Maple 4.00b}
\end{figure}

It's not very nice so let us improve this a little bit:

\texttt{>plot3d(f,xrange,yrange,style=patchcontour,contours=[seq(-7+k/4,k=0..60)],\\
grid=[80,50],shading=ZHUE,axes=FRAME, tickmarks=[3,3,3],\\ scaling=unconstrained,orientation=[-107,68]);}

This will give:

\begin{figure}[H]
\centering
\includegraphics[scale=0.75]{img/analysis/representation_nice_3d_function.eps}
\caption{Better representation of a 3D function with Maple 4.00b}
\end{figure}

With a small rotation to view from above:

\texttt{>plot3d(f,xrange,yrange, style=patchcontour, contours=[seq(-7+k/4,k=0..60)], grid=[80,50], shading=ZHUE, axes=FRAME, tickmarks=[3,3,3], scaling=unconstrained, orientation=[-90,0]);}

\begin{figure}[H]
\centering
\includegraphics[scale=0.75]{img/analysis/representation_nice_3d_function_above.eps}
\caption[]{Above representation of a 3D function with Maple 4.00b}
\end{figure}

And in section view (side view):

\texttt{>plot(f(x,2),x=xrange);}

\begin{figure}[H]
\centering
\includegraphics[scale=0.5]{img/analysis/representation_nice_3d_function_section.eps}
\caption{Representation of a section of the pseudo-3D surface}
\end{figure}

Or with multiple sections views:

\texttt{>display([seq(plot(f(x,y),x=xrange),y=yrange)]);}

\begin{figure}[H]
\centering
\includegraphics[scale=0.5]{img/analysis/representation_nice_3d_function_multiple_sections.eps}
\caption{Representation of multiple sections of the pseudo-3D surface}
\end{figure}

The reader can also animate the graph above with the following command:

\texttt{>display([seq(display([plot(f(x,k/5),x=xrange),}\\ \texttt{textplot([6,5,cat('y=',convert(evalf(k/5,2),string))],font=[TIMES,BOLD,16])])}\\
\texttt{,k=-25..25)],insequence=true, title='Animation',titlefont=[TIMES,BOLD,18]);}

That's all for typical and simple example of standard manipulations of an engineer hired in a company and using graphics (in practice it will instead use MATLAB™ instead of Maple but the reader can refer to the free companion book on MATLAB™ with a few hundreds of pages graphics).

\paragraph{2D Vector representations}\mbox{}\\\\
It is also frequently made use of graphic representations in the context of analytical geometry to simplify analysis or to prove theorems with the help of visual representations (do not abuse of this method!).

Thus, we can easily introduce the concept of "norm" (\SeeChapter{see section Vector Calculus page \pageref{vector norm}}) in a very easy way by plotting the distance between two points (in 2D or in 3D) and applying the Pythagorean theorem that will be assumed to be known (\SeeChapter{see section Euclidean Geometry page \pageref{pythagorean theorem}}).

The main idea of a planar vector representation in physics and engineering labs is that  a point $P_1$ of coordinates $(x_1,y_1)$ that has some physical properties (typically a velocity) will be after a given time at the point $P_2$ of coordinates $(x_2,y_2)$ supposed to be always in the same plane. In this way, the straight line between $P_1$ and $P_2$ is a visualization of the "intensity" of the velocity (and implicitly of the force). When doing that for many points we get a planar representation of a planar vector field (for more example see the companion book on MATLAB™):

\begin{figure}[H]
\centering
\includegraphics{img/analysis/vector_field.jpg}
\caption[]{Typical planar vector field with MATLAB™}
\end{figure}


Now let us represent three points $P_1,P_2,P_3$ on a plane graph in which has been defined a referential as presented below:

\begin{figure}[H]
\centering
\includegraphics{img/analysis/vector_plane.jpg}
\caption[]{Scenario of three points in a plane}
\end{figure}

We can consider the straight line $\overline{P_1P_2}$ as a vector but not translated at the origin of the referential (\SeeChapter{Vector Calculus}).

If $x_1\neq x_2$ and $y_1\neq y_2$ (as in the figure above), the points $P_1,P_2,P_3$ are the vertices of a perpendicular triangle. By applying the Pythagorean theorem (\SeeChapter{see section Euclidean Geometry page \pageref{pythagorean theorem}}) we can easily calculate the metric distance $d$ as:
	
	On the figure, we see that:
	
	Since $\forall x \in \mathbb{R} \; \vert x \vert ^2 =x^2$, we can write:
	
	If $x_1=y_1=0$, we end up with a relation named "\NewTerm{norm}", "\NewTerm{module}" or "\NewTerm{distance}\index{distance}" that we have already defined as part of our study of Vector Calculus when the origin of the vector is translated on the origin of the referential (see section of the corresponding name page \pageref{vector norm}).
	
	\begin{theorem}
	Obviously, if we consider two points $P_1(x_1,y_1),P_2(x_2,y_2)$, we can determine if a third point $P_3(x_3,y_3)$ is on the mediator (\SeeChapter{see section Euclidean Geometry page \pageref{mediator}}) of the first two and for this that it is obviously sufficient that (by definition of the mediator!):
	
	\end{theorem}
	\begin{tcolorbox}[title=Remark,colframe=black,arc=10pt]
	We hesitated to put this proof in the section of Analytical Geometry but at then end we have decided that it was a nice example of showing how visual representation can help readers to better understand some subjects.
	\end{tcolorbox}	
	\begin{dem}
	As $(x_1,y_1),(x_2,y_2)$ are known, we can easily express an "\NewTerm{analytic expression}" property of the mediator that is that for each point on the mediator we have:
	
	where $a, b$ are therefore constants and wherein any point that satisfies this relation, which is in this case the equation of a straight line, lies on the mediator.
	\begin{flushright}
		$\blacksquare$  Q.E.D.
	\end{flushright}
	\end{dem}
	Furthermore, it is easy to see that the midpoint of the line segment that coincide with the mediator is given by:
	
	So we see that with a simple visual representation, we can achieve results that are sometimes (...) more obvious for students.
	
	Let us use this example to define some concepts on that we will come back further and do some reminders.
	
	\textbf{Definition (\#\mydef):} Any function of the form of a polynomial (\SeeChapter{see section Calculus page \pageref{polynomial}}) of degree $1$ with constant real coefficients:
	
	is the analytic expression of what we name a "\NewTerm{straight line}\index{straight line}\label{straight line}" "\NewTerm{linear equation}\index{linear equation}" of "\NewTerm{slope}\index{slope}" $a$ and "\NewTerm{intercept}\index{intercept}" $b$ (when $x=0$).
	
	Obviously, if:
	
	the line is horizontal if we graphically represent it since $y$ is constant for all $x$ and is equal to $b$. Conversely, if:
	
	the straight line will be vertical in the $x\text{O}y$ referential.
	
	\paragraph{Properties of visual representations}\mbox{}\\\\
	Depending on the type of graph we visualize (especially graphics planes) it is possible to extract some basic properties. Let us see the most important one to know for univariate functions:
	
	\begin{enumerate}
		\item[P1.] The graph of a function is "\NewTerm{symmetrical about the $y$-axis}\index{graph symmetric about the $y$-axis}" if the change in from $x$ to  $-x$ in the function does not change the value of $y$ such that:
		\begin{figure}[H]
		\centering
		\includegraphics{img/analysis/function_property_symetry_y.jpg}
		\caption{Example of symmetry through the $y$-axis of a function}
		\end{figure}
		
		\item[P2.] The graph of a function is "\NewTerm{symmetrical about the $x$-axis}\index{graph symmetric about the $x$-axis}" if the change from $y$ to $-y$ does not change the value of $x$ such that:
		\begin{figure}[H]
		\centering
		\includegraphics{img/analysis/function_property_symetry_x.jpg}
		\caption{Example of symmetry through the $x$-axis of a function}
		\end{figure}
		
		\item[P3.] The graph of a function is "\NewTerm{symmetrical about the origin $\text{O}$}\index{graph symmetrical about the origin}" if the simultaneous change of $y$ to $-y$ and from $x$ to $-x$ gives the following result (that is to say that the change in the sign of one variable change the sign of the other):
		\begin{figure}[H]
		\centering
		\includegraphics{img/analysis/function_property_symetry_o.jpg}
		\caption{Example of symmetry through the origin $\text{O}$ of a function}
		\end{figure}
		
		\item[P4.] Given a function $y=f(x)$, if we add a constant $c^{te} \geq 0$ to this function as:
		
		then the function $f(x)$ is shifted (or "translated") vertically upwards of a distance $c^{te}$ as presented in the figure below:
		\begin{figure}[H]
		\centering
		\includegraphics{img/analysis/function_property_positive_translated.jpg}
		\caption{Example of a positive vertical translation of a function}
		\end{figure}
		And conversely if $c^{te} \geq 0$ but:
		
		then the function $f(x)$  is obviously translated vertically downwards:
		\begin{figure}[H]
		\centering
		\includegraphics{img/analysis/function_property_negative_translated.jpg}
		\caption{Example of a negative vertical translation of a function}
		\end{figure}
		We can also consider horizontal translations of functions. Specifically, if we have still $c^{te}$, then $y=f(x)$ is translated horizontally to the right if we write:
		
		which graphically is represented by:
		\begin{figure}[H]
		\centering
		\includegraphics{img/analysis/function_property_negative_horizontal_translated.jpg}
		\caption{Example of negative horizontal translation of a function}
		\end{figure}
		and conversely, translated horizontally to the left, if we write:
		
			as shown in the graph below:
		\begin{figure}[H]
		\centering
		\includegraphics{img/analysis/function_property_positive_horizontal_translated.jpg}
		\caption{Example of positive horizontal translation of a function}
		\end{figure}
		To stretch or compress vertically a function, we simply multiply $y=f(x)$ by a constant $c^{te}>1$ and respectively $0\leq c^{te}<1$ as:
		
		and don't forget that if a function is linear then we have the special property $f(\lambda x)=\lambda f(x)$.
		This is graphically represented for the case $c^{te}>1$ by:
		\begin{figure}[H]
		\centering
		\includegraphics{img/analysis/function_property_upscaled.jpg}
		\caption{Example of vertical stretch of a function}
		\end{figure}
		and when $0\leq c^{te}<1$ by:
		\begin{figure}[H]
		\centering
		\includegraphics{img/analysis/function_property_downscaled.jpg}
		\caption{Example of vertical compression of a function}
		\end{figure}
		To stretch or compress a function horizontally, by the same way, we just need to multiply the variable $x$ by a constant by a constant $c^{te}>1$ and respectively $0\leq c^{te}<1$ as:
		
		This is graphically represented for the case $c^{te}>1$ by:
		\begin{figure}[H]
		\centering
		\includegraphics{img/analysis/function_property_upscaled_horizontal.jpg}
		\caption{Example of horizontal stretch of a function}
		\end{figure}
		and when $0\leq c^{te}<1$ by:
		\begin{figure}[H]
			\centering
			\includegraphics{img/analysis/function_property_downscaled_horizontal.jpg}
			\caption{Example of horizontal downscale of a function}
		\end{figure}
	\end{enumerate}
	
	\begin{tcolorbox}[title=Remark,colframe=black,arc=10pt]
	Translate, stretch, compress a function or apply it a symmetry is transforming it. The plot resulting from these transformations is named the "\NewTerm{transformed}\index{transformed graph}" from the initial plot.
	\end{tcolorbox}	
	
	\textbf{Definitions (\#\mydef):} We say that a function $f$ is (we simplify the definition using an univariate function):
	\begin{itemize}
		\item[D1.] A function is a "\NewTerm{constant function}\index{constant function}" on an interval $I$ if for each pair $(x_1,x_2)$ of elements of $I$ such that $x_1\neq x_2$, we have $f(x_1)=f(x_2)$. What we denote in a condensed manner by:
		
		
		\item[D2.] A function is an "\NewTerm{increasing function}\index{increasing function}" or an "\NewTerm{increasing function in the broadest sense}" on the interval $I$ if for each pair $(x_1,x_2)$ of elements of $I$ such that $x_1\leq x_2$, we have $f(x_1)\leq f(x_2)$. What we denote in a condensed manner by:
		
		
		\item[D3.] A function is an "\NewTerm{decreasing function}\index{decreasing function}" or an "\NewTerm{decreasing function in the broadest sense}" on the interval $I$ if for each pair $(x_1,x_2)$ of elements of $I$ such that $x_1\leq x_2$, we have $f(x_1)\geq f(x_2)$. What we denote in a condensed manner by:
		
		\begin{tcolorbox}[title=Remark,colframe=black,arc=10pt]
		A function is a "\NewTerm{monotonic function}\index{monotonic function}" or "\NewTerm{monotonic function in the broadest sense}" on an interval $I$ if it is increasing or decreasing in this interval.
		\end{tcolorbox}
		
		\item[D4.] A function is a "\NewTerm{strictly increasing function}\index{strictly increasing function}"  on the interval $I$ if for each pair $(x_1,x_2)$ of elements of $I$ such that $x_1\leq x_2$, we have $f(x_1)< f(x_2)$. What we denote in a condensed manner by:
		
		
		\item[D5.] A function is an "\NewTerm{strictly decreasing function}\index{strictly decreasing function}"  on the interval $I$ if for each pair $(x_1,x_2)$ of elements of $I$ such that $x_1\leq x_2$, we have $f(x_1)> f(x_2)$. What we denote in a condensed manner by:
		
		\begin{tcolorbox}[title=Remark,colframe=black,arc=10pt]
		A function is a "\NewTerm{strictly monotonic function}" on an interval $I$ if it is strictly increasing or decreasing in this interval.
		\end{tcolorbox}
	\end{itemize}
	
	\subsubsection{Analytical Representation}
	The analytical method of representation is by far the most used and consists of representing any function in an "\NewTerm{analytic expression}\index{analytic expression}" or "\NewTerm{closed form}\index{closed form}" which is a symbolic and abstract mathematical notation of all known mathematical operations that must be applied in a certain order to numbers and letters expressing constants or variables that we seek to analyse.
	
	Note that by "all known mathematical operations", we consider not only the mathematical operations seen in the chapter of Arithmetics (addition, subtraction, root extraction, etc.) but also all the operations that will be defined later in this book.
	
	If the functional dependence $y=f(x)$ is such that $f$ is an analytic expression, then we say that the "\NewTerm{function $y$ of $x$}" is "given analytically ". 

	Here are some examples of simple analytical expressions:
	
	When we have determined the equation of the mediator, we have obtained an analytical expression of the visual straight line that characterize it as a function of the type:
	
	which we recall, is the analytical expression of the equation of a straight line, also named "\NewTerm{linear equation}\index{linear equation}" or "\NewTerm{affine function}\index{affine function}", on a plane where two points are known $P_1(x_1,y_1),P_2(x_2,y_2)$, the slope is given by the ratio of vertical growth on the horizontal growth as:
	
	A friendly and trivial application is to prove analytically that two non-vertical lines are parallel if and only if they have the same slope. Thus, given two lines with the equations:
	
	The lines intersect at a point $(x, y)$ if and only if values of $y$ are equal for a certain $x$, that is to say:
	
	The last equation can be solved with respect to $x$ if and only if $a_2-a_2\neq 0$. We have therefore proved that the lines $y_1,y_2$ intersect if and only if $a_1\neq a_2$. Therefore, they do not intersect (are parallel) if and only if $a_1=a_2$.
	
	In a quite simple way by applying the Pythagorean theorem, it is not difficult (\SeeChapter{see section Analytical Geometry page \pageref{conics}}) to determine that the equation of a circle with center $C (h, k)$ has for equation (it is of use in mathematics not explain $y$ for the equation of the circle therefore the equation of the latter is much more visually aesthetic and speaking):
	
	In these examples the functions are expressed analytically by a single formula (equality between two analytical expressions) which defines at the same time the "natural domain of definition" of the functions.
	
	\textbf{Definition (\#\mydef):} The "\NewTerm{natural domain of definition}\index{natural domain of definition}\label{natural domain of definition}" of a function given by an analytical expression is the set of $x$ values for which the expression on the right-hand side has a definite value.
	
	For example the function:
	
	is defined for all values of $x$ except the value $x=1$ where we have a singularity (division by zero).
	\begin{tcolorbox}[title=Remark,colframe=black,arc=10pt]
	There are an infinite number of functions and we can not expose them all here, however we will meet more than a thousand on this entire book and should amply suffice to get an idea of their study.
	\end{tcolorbox}
	
	And we have the famous following "\NewTerm{table of variations}\index{table of variations}\label{table of variations}" that is also considered as an analytical tool and also used by some teachers to study the basics of the derivative $f'$ of a function $f$ (\SeeChapter{see section Differential and Integral Calculus page \pageref{differential calculus}}). For example with the function $x^3-3x^2+2$ (already seen in the previous mentioned section):

	\begin{minipage}{\linewidth}\centering
    \begin{variations}
     x      & \mI &    & 0 &    & 2 &    & \pI  \\
     \filet
     f'     & \ga +    & 0    &  -  &  0   & \dr+      \\
     \filet
     \m{f}  & ~  & \c  & \h{~} & \d & ~    &  \c       \\
     \end{variations}
	\end{minipage} 	
	
	Whose corresponding plot is:
	\begin{figure}[H]
		\centering
		\includegraphics{img/algebra/variation_plot_example.jpg}
		\caption[]{Plot of  function $x^3-3x^2+2$}
	\end{figure}
	
	\pagebreak
	\subsection{Functions}\label{functions}
	In mathematics, a function is a relation between a set of inputs and a set of permissible outputs with the property that each input is related to exactly one output.
	
	Functions of various kinds are the central objects of investigation in most fields of modern mathematics. There are many ways to describe or represent a function. Some functions may be defined by a formula or algorithm that tells how to compute the output for a given input. Others are given by a picture, named the "graph" of the function. In science, functions are sometimes defined by a table that gives the outputs for selected inputs. A function could be described implicitly, for example as the inverse to another function or as a solution of a differential equation.
	
	First remember the definitions already given earlier during our study of graphical representation of functions:
	
	\textbf{Definitions (\#\mydef):} We say that a function $f$ is (we simplify the definition using an univariate function):
	\begin{itemize}
		\item[D1.] A function is a "\NewTerm{constant function}\index{constant function}" on an interval $I$ if for each pair $(x_1,x_2)$ of elements of $I$ such that $x_1\neq x_2$, we have $f(x_1)=f(x_2)$. What we denote in a condensed manner by:
		
		
		\item[D2.] A function is an "\NewTerm{increasing function}\index{increasing function}" or an "\NewTerm{increasing function in the broadest sense}" on the interval $I$ if for each pair $(x_1,x_2)$ of elements of $I$ such that $x_1\leq x_2$, we have $f(x_1)\leq f(x_2)$. What we denote in a condensed manner by:
		
		
		\item[D3.] A function is an "\NewTerm{decreasing function}\index{decreasing function}" or an "\NewTerm{decreasing function in the broadest sense}" on the interval $I$ if for each pair $(x_1,x_2)$ of elements of $I$ such that $x_1\leq x_2$, we have $f(x_1)\geq f(x_2)$. What we denote in a condensed manner by:
		
		\begin{tcolorbox}[title=Remark,colframe=black,arc=10pt]
		A function is a "\NewTerm{monotonic function}\index{monotonic function}" or "\NewTerm{monotonic function in the broadest sense}" on an interval $I$ if it is increasing or decreasing in this interval.
		\end{tcolorbox}
		
		\item[D4.] A function is a "\NewTerm{strictly increasing function}\index{strictly increasing function}"  on the interval $I$ if for each pair $(x_1,x_2)$ of elements of $I$ such that $x_1\leq x_2$, we have $f(x_1)< f(x_2)$. What we denote in a condensed manner by:
		
		
		\item[D5.] A function is an "\NewTerm{strictly decreasing function}\index{strictly decreasing function}"  on the interval $I$ if for each pair $(x_1,x_2)$ of elements of $I$ such that $x_1\leq x_2$, we have $f(x_1)> f(x_2)$. What we denote in a condensed manner by:
		
		\begin{tcolorbox}[title=Remark,colframe=black,arc=10pt]
		A function is a "\NewTerm{strictly monotonic function}\index{strictly monotonic function}" on an interval $I$ if it is strictly increasing or decreasing in this interval.
		\end{tcolorbox}
	\end{itemize}
	And let us add now complementary definitions:
	
	\textbf{Definitions (\#\mydef):}
	\begin{enumerate}
		\item[D1.] We say that $y$ is a function of $x$ and we will write $y=f(x),y=\varphi(x)$, etc., if for every value of the variable $x$ belonging to a certain domain of definition (set) $D$, corresponds a value of the variable $y$ in another target domain of definition (set) $E$. What we denote in various ways (the third one being the most recommended):
		
		The variable $x$ is named "\NewTerm{independent variable}\index{independent variable}" or "\NewTerm{input variable}" or even "\NewTerm{exogenous variable}\index{exogenous variable}" and $y$ the "\NewTerm{dependent variable}\index{dependent variable}" or "\NewTerm{endogenous variable}\index{endogenous variable}".
		
		The dependence between the variables $x$ and $y$ is named a "\NewTerm{functional dependency}\index{functional dependency}". The letter $f$, which in the symbolic notation of functional dependence, indicates that we need to apply some operations to $x$ to obtain the corresponding $y$ value.
		
		Sometimes we write:
		
		rather than:
		
		In the latter case the letter $y$ expresses at the same time the value of the function and the symbol of operations applied to $x$.
		\begin{tcolorbox}[title=Remark,colframe=black,arc=10pt]
		As we saw it during our study in the section Set Theory, an application (or function) may be injective, surjective or bijective: 
		\begin{figure}[H]
			\centering
			\includegraphics[scale=0.75]{img/analysis/functions_type.jpg}
			\caption{Quick summary of applications/functions types}
		\end{figure}
		It is therefore necessary that the reader for whom these concepts are unknown goes in priority read these definitions.
		\end{tcolorbox}
		
		\item[D2.] The set of $x$ values (inputs) for which the value of the function $y$ is given by the function $f (x)$ is named the "\NewTerm{range of existence}\index{range of existence}" of the function or "\NewTerm{domain of definition}\index{domain of definition}\label{domain of definition}" of the function and denoted in this book by the letter $D$.
		
		The set of outputs of $f(x)$ is named the "\NewTerm{image}\index{image}" or sometimes the "\NewTerm{codomain}\index{codomain}" and denoted in this book by the letter $E$. When study of the point of view of the knowledge of the output values only, the set of $x$ is named the "\NewTerm{pre-image}".
		
		\item[D3.] A function $f(x)$ is named a "\NewTerm{periodic function}\index{periodic function}" if there is a constant $c^{te}$ such that the function's value does not change when we add (or subtract we) that constant to the independent variable such as:
		
		which corresponds to a translation along the $x$-axis. The smallest constant satisfying this condition is named the "\NewTerm{period}\index{period}" of the function. It is frequently denoted by the letter $T$ in physics.
		
		The most common periodic functions know by students and engineers are the trigonometric functions (see section of the corresponding name page \pageref{trigonometry}):
		\begin{figure}[H]
			\centering
			\includegraphics[scale=0.4]{img/analysis/periodic_function.jpg}
			\caption{Example of periodic function with period and amplitude}
		\end{figure}
		 \begin{tcolorbox}[title=Remark,colframe=black,arc=10pt]
		The sum of two periodic functions with different periods is not necessarily periodic and there is no general formula to get the period of a function that is the sum of $n$ other functions!
		\end{tcolorbox}
		
		\item[D4.] In differential calculus (\SeeChapter{see section Differential and Integral Calculus page \pageref{differential calculus}}), the expression:
		
		with $h\neq 0$ is of particular interest. We name it a "\NewTerm{growth quotient}\index{growth quotient}" (we discuss this in much more detail in our study of differential and integral calculus).
		
		\item[D5.] We use certain properties of functions for easy graphical representation and analysis or mathematical simplifications. In particular, a function $f (x)$ is named "\NewTerm{even function}\index{even function}\label{even function}" if:
		
		for all $x$ in its definition domain.
		
		That is to say as we already seen previously, it is symmetric relatively with the $y$-axis:
		\begin{figure}[H]
			\centering
			\includegraphics{img/analysis/function_property_symetry_y.jpg}
			\caption{Example of even function}
		\end{figure}
		A function $f (x)$ is named "\NewTerm{odd function}\index{odd function}\label{odd function}" if:
		
		for all $x$ in its definition domain.
		
		That is to say as we already seen previously, it is symmetric relatively with the origin:
		\begin{figure}[H]
			\centering	\includegraphics{img/analysis/function_property_symetry_o.jpg}
			\caption{Example of odd function}
		\end{figure}
		So, to summarize, an even function is a function that is independent of the sign of the variable and an odd function change of sign when we change the sign of the variable (the spiral of Cornus in the section Civil Engineering is a good practical example of odd function). This concept will be very useful to us to simplify some very useful developments in physics (such as Fourier transforms of odd or even functions for example, or the calculation of certain integrals!).
		\begin{theorem}
		Remember that this type theorem linking a general concept to a particular case and its opposite is often found in mathematics. We will see such examples in tensor calculus with the symmetric and antisymmetric tensor (\SeeChapter{see section Tensor Calculus page \pageref{symmetric tensor} and page \pageref{antisymmetric tensor}}) or in quantum physics with the Hermitian or non-Hermitian operators (\SeeChapter{see section Wave Quantum Physics page \pageref{hermitian operator} and page \pageref{non-hermitian operator}}).
		\end{theorem}
		\begin{dem}
		Let us write:
		
		Then:
		
		If we sum then we get:
		
		and by subtracting:
		
		So there is really and odd and even decomposition of any function!!!
		\begin{flushright}
			$\blacksquare$  Q.E.D.
		\end{flushright}
		\end{dem}
		Finally, it is important to note that:
		\begin{itemize}
			\item The product of two even functions is an even function
			\item The product of two odd functions is an even function
			\item The product of an even and odd function is an odd function
		\end{itemize}
		Let us see a short proof of the last property because we will need it in the chapter on Geometry.
		\begin{dem}
		Let $g(x)$ be an even function and $h(x)$ an odd function such as:
		
		Therefore:
		
		\begin{flushright}
			$\blacksquare$  Q.E.D.
		\end{flushright}
		\end{dem}
		\item[D6.] In general, if $f (x)$ and $g (x)$ are arbitrary functions, we use the terminology and notations given in the following table:
		\begin{table}[H]	
			\begin{center}
				\begin{tabular}{|c|c|}
				\hline
				  \rowcolor[gray]{0.75}Terminology&Value of the function\\
				  \hline
				  % after \\: \hline or \cline{col1-col2} \cline{col3-col4} ...
				  Sum $f+g$ & $(f+g)(x)=f(x)+g(x)$ \\\hline
				  Difference $f-g$ & $(f-g)(x)=f(x)-g(x)$ \\\hline
				  Product $f \cdot g$ & $(f \cdot g)(x)=f(x)g(x)$ \\\hline
				  Quotient $\displaystyle\frac{f}{g}$&$\left(\displaystyle\frac{f}{g}\right)(x)=\displaystyle\frac{f(x)}{g(x)}$ \\\hline
				\end{tabular}
			\end{center}
			\caption{Terminology about functions}
		\end{table}
		The definition domains of $f+g,f-g,f\cdot g$ are the intersection $I$ of the definition domain of $f (x)$ and g $(x)$, that is to say, the numbers which are common to both domains of definition. The domain of definition of $g/g$ is meanwhile the subset $I$ of all $x$ such that  $g(x)\neq 0$.
		
		\item[D7.] Let $y$ be a function of $f$ of $u$ such that $y=f(u)$ and $u$ a function $g$ of $x$ such that $u=g(x)$, then $y$ depends on $x$ and we have what we name a "\NewTerm{composite function}\index{composite function}\label{composite function}" that we denote:
		
		The last equality should be read "\NewTerm{$f$ round $g$}" and not confuse with the "round" symbol with the notation of the dot product that we have seen during our study of the section Vector Calculus page \pageref{dot product}.
		
		The domain of definition of the composite function is either identical to the entire domain of definition of the function $u=g(x)$ or the part of the domain in which the values of $u$ are such that the corresponding values $f (u)$ belong to the domain of definition of this function.
		
		Obviously the principle of composite function can be applied not only once, but an arbitrary number of times such that $y=f(g(h(t)))$ and so on...
		
		In computing science a function may compose with itself a given number of times $n$, such that $f(f(f(f(f...)))))=f^n$ that must not be confuse with the notation $f^2(x)$.
		
		If $u$ does not depend on another variable (or it is not itself a composite function), then we say that $f(x)$ is an "\NewTerm{elementary function}\index{elementary function}".

		Obviously there are an infinite number of elementary functions but most can be classified into families whose expression is similar to one of the following:
		
		\begin{itemize}
			\item "\NewTerm{Linear functions}\index{linear function}":
			
			The are simply functions representing straight lines of slope $a$ passing through the origin of the axis.
			
			\item "\NewTerm{Affine functions}\index{affine function}":
			
			The are simply functions representing straight lines of slope $a$ passing through the origin of the axis or not (linear function with a translation).
						
			\item "\NewTerm{Power functions}\index{power function}":
			
			where $m\in\mathbb{R}$. Functions involving roots are often named "\NewTerm{radical functions}\index{radical functions}".
			\begin{figure}[H]
				\centering
				\includegraphics{img/analysis/power_function.jpg}
				\caption{Different plots of simple power functions}
			\end{figure}
			
			\item "\NewTerm{Absolute value functions}\index{absolute value function}\label{absolute value plot}" (see section Arithmetic Operators page \pageref{absolute value} for the definition and the study of the "absolute value"):
			
			For example the plots with Maple 4.00b that we get with the command:\\
			
			\texttt{>plot([(x),(cos(x)),(x\string^2-3),(x\string^3-4*x\string^2+2*x)],x=-6..6,y=-4..3,\\
			thickness=3);}	
			
			\begin{figure}[H]
				\centering
				\includegraphics{img/analysis/pre_absolute_plot_functions.jpg}
			\end{figure}
			
			and taking the absolute value:\\
			
			\texttt{>plot([abs(x),abs(cos(x)),abs(x\string^2-3),abs(x\string^3-4*x\string^2+2*x)]\\
			,x=-6..6,y=-0.5..3,thickness=3);}
			\begin{figure}[H]
				\centering
				\includegraphics{img/analysis/post_absolute_plot_functions.jpg}
			\end{figure}
		
		\item "\NewTerm{Exponential functions}\index{exponential function}":
			
			where the famous $e^x$ is only a special case and also $a\in\mathbb{R}$.
			
			When $a\geq 0$ we have typically:
			\begin{figure}[H]
				\centering
				\includegraphics{img/analysis/exponential_functions.jpg}
				\caption{Different plots of simple exponential functions $(1^2,2^x,3^x,4^x)$ with Maple 4.00b}
			\end{figure}
			where $m$ is a positive number different from $1$ (otherwise it is simple a linear function):
			
			If $a<0,x\in\mathbb{R}$ the function is not defined. Indeed for $(-1)^(0.5)=\left\lbrace \mathrm{i},	-\mathrm{i}	\right\rbrace$ therefore it is an application from $f:\mathrm{R}\mapsto\mathbb{C}^2$ and as far as we know there is no nice way to represent it visually and anyway this is not a function in the traditional way.
			
			\item "\NewTerm{Logarithmic functions}\index{logarithmic function}":
				
			with $a\in\mathbb{R}^{+}$ and that by construction of the logarithm (see further below) are of the type $f:\mathbb{R}^{+}\mapsto \mathbb{R}$.
			We have typically:
			\begin{figure}[H]
				\centering
				\includegraphics{img/analysis/logarithm_functions.jpg}
				\caption{Different plots of logarithm $\ln(x)=\ln_e(x)$ in green and $\log_{10}(x)$ in red with Maple 4.00b}
			\end{figure}
			
			\item "\NewTerm{Periodic/Trigonometric functions}\index{period function}\index{trigonometric function}":
			
			We already defined previously what is a periodic function. For the trigonometric functions the reader can see below a plot of the main one but for more details it is strongly recommended to read the section Trigonometry page \pageref{trigonometry}:
			\begin{figure}[H]
				\centering
				\includegraphics[scale=0.9]{img/analysis/trigonometric_functions.jpg}
				\caption{Different plots of trigonometric functions with Maple 4.00b}
			\end{figure}
			
			\item "\NewTerm{Polynomial functions}\index{polynomial function}":
			
			
			where as we already know $a_0,a_1,...,a_n$ are constant numbers named "\NewTerm{coefficients}\index{coefficients}" and $n$ is a positive integer that we name "\NewTerm{degree of the polynomial}\index{degree of a polynomial}". Obviously this function is defined for all values of $x$, that is to say, it is define on an infinite interval.
			
			If follows that functions the power functions of the type $x^m$ and linear functions of the type $f(x)=x$ are a subclass of polynomial for $m\in \mathbb{N}$.
			
			We have already study more deeply polynomials in the section Calculus with their main properties but let us give us again the plot of some of them as recall: 
			\begin{figure}[H]
				\centering
				\includegraphics{img/algebra/polynomials.jpg}
				\caption[Some polynomials plotted with R.3.2.1]{Some polynomials plotted with R.3.2.1 (see our \texttt{R} companion book)}
			\end{figure}
			We will see and study in this book some famous polynomials as: Legendre polynomials (\SeeChapter{see section Quantum Chemistry page \pageref{legendre polynomial}}), Bernoulli polynomials (\SeeChapter{see section Sequences and Series page \pageref{bernoulli polynomials}}), Bernstein polynomials (\SeeChapter{see section Numerical Methods page \pageref{bernstein polynomial}}), Hermite polynomials (\SeeChapter{see section Functional Analysis page \pageref{hermite polynomial}}), ...
			
			\item "\NewTerm{Rational fractions}\index{rational fractions}" are polynomials divisions (\SeeChapter{see section Calculus page \pageref{polynomials division}}):
			
			\begin{tcolorbox}[title=Remark,colframe=black,arc=10pt]
			Obviously two rational fractions are equal, if one is obtained from the other by multiplying the numerator and denominator by the same polynomial.
			\end{tcolorbox}
			The rational function:
			
			is not defined at $x^2=5 \Leftrightarrow x=\pm \sqrt{5}$. It is asymptotic (see further below) to $\frac{x}{2}$ as $x$ approaches infinity:
			\begin{figure}[H]
				\centering
				\includegraphics{img/analysis/rational_function.jpg}
				\caption[Example of rational function]{Example of rational function $f(x) = \frac{x^3-2x}{2(x^2-5)}$}
			\end{figure}
			The rational function:
			
			 is defined for all real numbers, but not for all complex numbers, since if x were a square root of -1 (i.e. the imaginary unit or its negative), then formal evaluation would lead to division by zero!
			 
			 A constant function such as is a rational function since constants are polynomials. Every polynomial function $f(x) = P(x)$ is a rational function with $Q(x) = 1$. The power functions $f(x)=x^m$ are also rational functions when $m\in\mathbb{N}$.
			 
			 \item "\NewTerm{Algebraic functions}\index{algebraic function}" are defined by the fact that the function $f(x)$ is the result of addition, subtraction, multiplication, division, of variables put to an integer or non-integer power. Therefore most of the functions defined previously can be included in this definition: linear functions, affine function, power function, polynomial function, rational functions.
			 
			 \item A "\NewTerm{piecewise function}\index{piecewise function}" is a function defined by different formulas on different parts of its domain. The absolute value is a famous example of a piecewise-defined function because the formula changes with the sign of $x$:
			 
			 
			 \item A "\NewTerm{step function}\index{step function}" $f:[a,b]\in \mathbb{R}$ is defined if and only if there exists a subdivision $(a_i)_{0\leq i \leq n}$ of $[a, b]$ such that $a_0=a$ and $a_n=b$ and $(\lambda_0,...,\lambda_n)\in \mathbb{R}^n$ such as:
			 
			\begin{figure}[H]
				\centering
				\includegraphics{img/analysis/step_function.jpg}
				\caption{Example of a step function}
			\end{figure}
			Such functions can be found in signal processing and also in statistics for survival analysis.
		\end{itemize}
		
		However, there are a very large number of other elementary functions that will meet in the individual sections of this book. Examples include the "Bessel functions" (\SeeChapter{see section Sequences and Series page \pageref{bessel functions}}), the "Lipschitz functions" (\SeeChapter{see section Topology page \pageref{lipschitz functions}}), the "Dirac functions" (\SeeChapter{see section Differential and Integral Calculus page \pageref{dirac function}}), the "distribution functions" (\SeeChapter{see section Statistics page \pageref{distribution function}}), the "Euler gamma function" (\SeeChapter{see section Differential and Integral Calculus page \pageref{gamma euler function}}), etc. The reader will notice that the Dirac function also belongs to the family of distribution functions.
	\end{enumerate}
	
	Here is a quite good summary (non exhaustive but good!):
	\begin{figure}[H]
		\centering
		\includegraphics{img/analysis/functions.jpg}
		\caption[Visual representation of various functions]{Visual representation of various functions (source: ?)}
	\end{figure}
	
	\subsubsection{Limits and Continuity of Functions}\label{limits}
	We will now consider ordered variables of a special type, which we define by the relation "the variable tends to a limit." In what will follow, the concept of limit of a variable will play a fundamental role, being intimately related to the basic notions of mathematical analysis, derivatives, integrals, etc.
	
	\textbf{Definition (\#\mydef):} The number $a$ is named the "\NewTerm{limit}\index{limit}" of variable magnitude $x$, if for any arbitrarily small positive number $\varepsilon$ we have:
	
	If the number $a$ is the limit of the variable $x$, we say that "\NewTerm{$x$ tends to the limit $a$}".

	We can also define the concept of limit from geometrical considerations (this can help to better understand ... but not always ...):
	
	The constant number $a$ is the limit of the variable $x$, if for any given neighbourhood, no matter how small, of center $a$ and of radius $\varepsilon$, we can find a value $x$ such that all the points corresponding to the following values of the variable belong to this neighbourhood (notions that we defined earlier). We represent geometrically this as:
	\begin{figure}[H]
		\centering
		\includegraphics{img/analysis/limit_geometric_representation.jpg}
		\caption{Geometric concept of limit in $\mathbb{R}^1$}
	\end{figure}
	\begin{tcolorbox}[title=Remarks,colframe=black,arc=10pt]
	\textbf{R1.} It should be trivial that the limit of a constant value is equal to this constant, since the inequality $|x-c^{te}|=|c^{te}-c^{te}|=0<\varepsilon $ is always satisfied for an arbitrary $\varepsilon>0$.\\
	
	\textbf{R2.}  Not all variable have limits. For example $y=\sin(x)$ as this trigonometric function fluctuates between $[-1,+1]$ from $[-\infty,+\infty]$.
	\end{tcolorbox}
	
	\textbf{Definition (\#\mydef):} A variable $x$ tends to infinity if for any positive chosen $M$, we indicate one value of $x$ from which all successive values of the variable $x$ (values in the neighbourhood of the previous chosen value) satisfy the inequality $|x|>M$. Formally:
	
	\begin{itemize}
		\item A variable $x$ "\NewTerm{tends to $+\infty$}" if for any positive chosen $M>0$, we indicate one value of $x$ from which all the successive values of the variables $x$ satisfies the inequality $M<x$.
	
		It is typically the type of consideration that we have for divergent sequences (divergent to infinity) where for a given term of value $M$ of the sequence all the other terms are greater ant $M$.
		
		
		\item A variable $x$ "\NewTerm{tends to $-\infty$}" if for any negative chosen $M<0$, we indicate one value of $x$ from which all the successive values of the variables $x$ satisfies the inequality $x<M$.
		
	\end{itemize}
	\textbf{Definition (\#\mydef):} Given $y=f(x)$ a function defined in a neighbourhood of $a$ or on some point of this neighbourhood. The function $y=f(x)$ tends to the limit $b$ (that is to say $y\rightarrow b$) when $x$ tends to $a$ (that is to say $x a$) if for any positive number $\varepsilon$ as small as possible, we can indicate  positive number $\delta$ such that all $x$ different from $a$ satisfying the inequality $|x-a|<\delta$ also satisfy $|f(x)-b|<\varepsilon$. Formally a function has a limit $b$ on $a$ when in a domain $E$ if:
	
	The inequality $|x-a|<\delta$ gives the possibility to have the distance from which we come with our $x$ without taking care of the direction (left or right) as we take for measurement of distance the absolute values. Indeed on a system of axis representing ordinates values, we can, for a given value, coming from the left or from the right (if necessary you can imagine a bus coming to a bus stop that can from the left or from the right only since the absolute distance from it to the bus stop is less than or equal to $\delta$).
	
	If $b$ is the limit of the function $f (x)$ when $x\rightarrow a$ we then write in this book in any case:
	
	Obviously the above definition is available when $a=\pm \infty$ or/and $b=\pm \infty$!	
	
	To define the direction from which we come from by applying the limit, we use a special notation (recall that this will give us the information of which side of the road comes our bus from...). Thus, if $f (x)$ tends towards the limit $b_1$ when $x$ approaches a number $a$ by taking only values smaller than $a$, then we write:
	
	(notice the small $-$ subscript) and we name $b_1$ the "\NewTerm{left limit}\index{left limit}" of the function $f (x)$ at point $a$ (because remember that the horizontal axis goes from left to right from $-\infty$ to $+\infty$, so small values compared to a given value, are on the left).
	
	If $x$ takes values greater than $a$, then we will write:
	
	(notice the small $+$ subscript) and we name $b_2$ the "\NewTerm{right limit}\index{right limit}" of the function $f (x)$ at point $a$.
	
	In the figure below we have for example:
	
	\begin{figure}[H]
		\centering
		\includegraphics{img/analysis/limits.jpg}
		\caption{Left and Right limit examples}
	\end{figure}
	It is not always easy (or even possible) to calculate limits of some functions. Let us see some typical examples:
	\begin{tcolorbox}[colframe=black,colback=white,sharp corners]
	\textbf{{\Large \ding{45}}Examples:}\\\\
	E1. Let us prove that:
		
	is true. For this purpose we have to prove that for any small $\varepsilon$ the inequality:
	
	will be satisfied as soon as $|x|>M$ where $M$ is defined by the choice of $\varepsilon$. The previous inequality is obviously equal to:
	
	which is satisfy if we have $x$:
	
	We admit that the example and the method can be discussed.... But in fact it is only an application of the Hospital rule (ratio of the derivatives) already proved in the section of Differential and Integral Calculus. The reader must also know that we will see also other techniques to determine limits further below with better examples.\\
	
	E2. Now using Taylor series and change of variables consider we want to calculate:
	
	The method is quite to intuitive. Indeed, first we do a change of variable:
	
	Now consider the Taylor series about $x=0$ for the function $f(x)=\sqrt{1+ax}$. We have:
	
	Which gives:
	\end{tcolorbox}
	
	\begin{tcolorbox}[colframe=black,colback=white,sharp corners]
	
	as a Taylor expansion about $x=0$. Applying this to our limit we see that:
	
	E3. We want to calculate the limit of:
	
	How can we deal with something like this? An idea is the to remember that it also implicitly means:
	
	Hence:
	
	And to answer what is the value of $\theta$, we refer to the plot of the function $\tan(\theta)$ and we see then that the first corresponding value is $\pi/2$, therefore:
	
	\end{tcolorbox}
	
	The signification of the symbols $x\rightarrow -\infty$ and $x\rightarrow +\infty$ makes obvious the signification of the expressions:
	
	and:
	
	that we denote formally by:	
	
	We have defined the case where the function $f (x)$ tends to a certain limit $b$ when $x\rightarrow a_{+,-}$ or $x\rightarrow \pm \infty$. Now let us consider the case where the function tend to infinity when the variable $x$ change in a certain way.
	
	We then have typically and obviously:
	
	Or when we need to indicate the direction:
	
	If the function $f(x) \rightarrow +\infty$ when $a \rightarrow +\infty$ then we write:
	
	And as we have four possibilities for the sign, we write:
	
	that is to say the four following possibilities:
	
	And once again don't forget as we already mentioned before that some function such as for example $f(x)=\sin(x)$ don't have any finite limit when $x\rightarrow \pm \infty$. Then we say that the function is just "bounded" (\SeeChapter{see section Set Theory page \pageref{closed bounded interval}}).
	\begin{figure}[H]
		\centering
		\includegraphics{img/analysis/limite_delucq.jpg}
	\end{figure}
	Now that we've roughly an overview of the concept of limit, we will give an extremely important definition that has a very important place of many areas of high-level mathematics, theoretical physics and computing science (numerical methods).
	
	\textbf{Definition (\#\mydef):} Given a function $f(x)$ and one of its subdomain (or whole one) $E$ (most of time $E \subseteq \mathbb{R}$ and $x_0\in E$, we say that we have a "\NewTerm{continuous function}\index{continuous function}" on $x_0$ if and only if:
	
	That is to say more formally (you have to be able to read the fact that we are going close in an infinitely small way of a limit an this allows the continuity):
	
	In other words: a function is continuous if for every point $x_0$ in the domain $E$, we can make the images of that point ($f(x_0)$) and another point ($f(x)$) arbitrarily close (of a distance $\varepsilon$) if we move the other point ($x$) close enough (distance $\delta$) to our given point.
	
	The latter relation will be generalized a little bit in the section of Topology and completed with the concept of... "uniform continuity"!
	\begin{tcolorbox}[title=Remarks,colframe=black,arc=10pt]
	\textbf{R1. }$f$ is "\NewTerm{continuous on the left}\index{continuous on the left}" or respectively "\NewTerm{continuous on the right}\index{continuous on the right}", if we add to the definition above the condition $x>x_0$, respectively $x<x_0$.\\
	
	\textbf{R2.} A continuous function with a continuous inverse function is named a "\NewTerm{homeomorphism}\index{homeomorphism}".\\
	
	\textbf{R3.} Instead of saying when necessary that a function is not continuous on $x_0$ or on a given domain, some practitioners prefer to say that the function has an "\NewTerm{oscillation}\index{oscillation}".
	\end{tcolorbox}	
	We have the following trivial corollaries:
	\begin{enumerate}
		\item[C1.] $f(x)$ is continuous on $x_0$ if and only if $f(x)$ is continuous on the left right and on the right left.
		
		\item[C2.] $f(x)$ is continuous on $E$ if and only if $f(x)$ is continuous on any point of $E$.
	\end{enumerate}
	
	\paragraph{Limit laws}\mbox{}\\\\
	We now take a look at the "\NewTerm{limit laws}\index{limit laws}", the individual properties limits in the univariate case. The proofs will be omitted as it is quite intuitive but any reader can request us the proof of one of them if needed!
	
	Let $f(x)$ and $g(x)$ be defined for all $x\neq a$ over some open interval containing $a$. Assume that $L$ and $M$ are real numbers such that:
	
	Let $c^{te}$ be a constant. Then, each of the following statements holds:		
	\begin{itemize}
		\item The sum law for limits gives:
		
		
		\item The difference law for limits gives:
		
		
		\item Constant multiple law for limits:
		
		
		\item Product law for limits:
		
		
		\item Quotient law for limits:
		
		for $M\neq 0$.
		
		\item Power law for limits:
		
		for every positive integer $n$.
		
		\item Root law for limits:
		
		for all $L$ if $n$ is odd and for $L\geq 0$ if $n$ is even.
	\end{itemize}
	
	\subsubsection{Asymptotes}
	The term "\NewTerm{asymptote}\index{asymptote}" is used in mathematics to precise possible properties of an infinite branch of curve which growth tends to an infinitesimal value.
	
	In analytic geometry, an asymptote of a curve is simply said to be a line such that the distance between the curve and the line approaches zero as they tend to infinity. In some contexts, such as algebraic geometry, an asymptote is defined as a line which is tangent to a curve at infinity.
	\begin{tcolorbox}[title=Remark,colframe=black,arc=10pt]
	The word asymptote is derived from the Greek and means "not falling together".
	\end{tcolorbox}	
	
	\textbf{Definitions (\#\mydef):}
	
	\begin{enumerate}
		\item[D1.] When the limit of a function $f(x)$ tends to a constant 
$c^{te}$ when $x \rightarrow \pm \infty$, then the graphical representation of this function leads us to draw a horizontal line that we name "\NewTerm{horizontal asymptote}\index{horizontal asymptote}" which equation is satisfies:
		
		
		\item[D2.] When the limit of a function $f(x)$ tends to  
$\pm \infty$ when $x \rightarrow a_{+,-}$, then the graphical representation of this function leads us to draw a vertical line that we name "\NewTerm{vertical asymptote}\index{vertical asymptote}" which equation is satisfies:
		
		Vertical asymptotes is the typical symptom of a division by zero in a fraction and has a very important place in physics. The syndrome is also named a "\NewTerm{singularity}\index{singularity}".
		\begin{tcolorbox}[colframe=black,colback=white,sharp corners]
		\textbf{{\Large \ding{45}}Example:}\\\\
		The graph of the function:
		
		 has the straight line of $x=1$ and $y=0$ as horizontal asymptote:
		 \begin{figure}[H]
			\centering
			\includegraphics{img/analysis/asymptote_vertical_horizontal_example.jpg}
			\caption{Graphical representation of a  horizontal and vertical asymptote}
		\end{figure}
		\end{tcolorbox}
		
		\item[D3.] The straight line of equation is an "\NewTerm{oblique asymptote}\index{oblique asymptote}" of a curve of the function $f (x)$ if:
		
		the values of $a$ and $b$ can be easily found using the following relations:
		
		
		\begin{tcolorbox}[colback=red!5,borderline={1mm}{2mm}{red!5},arc=0mm,boxrule=0pt]
		\bcbombe Caution! A curve may have two distinct oblique asymptotes in $+\infty$ and $-\infty$.
		\end{tcolorbox}
		
		To find a possible oblique asymptote, one must already be certain that the function $f(x)$ admits an infinite limit in $+\infty$ or $-\infty$ then only we look for the limits at $-\infty$ and $+\infty$ of  $f (x) / x$ and $f(x)-ax$.
		
		Three typical cases can be considered for oblique asymptotes:
		\begin{enumerate}
			\item The representative curve of $f(x)$ has for asymptotical direction the affine equation $y=ax$:
			
			\begin{tcolorbox}[colframe=black,colback=white,sharp corners]
			\textbf{{\Large \ding{45}}Example:}\\\\
			The graph of the function:
			
			 has the straight line of $y=x$ as oblique asymptote:
			 \begin{figure}[H]
				\centering
				\includegraphics{img/analysis/asymptote_oblique_affine_example.jpg}
				\caption{Graphical representation of an oblique affine asymptote}
			\end{figure}
			\end{tcolorbox}
			
			\item The representative curve of $f(x)$ has an infinite branch (this branch has not close form asymptote) and the only one thing we can say is that $x$-axis is the direction of this asymptote. Such an asymptote exists when:
			
			\begin{tcolorbox}[colframe=black,colback=white,sharp corners]
			\textbf{{\Large \ding{45}}Example:}\\\\
			The functions $f(x)=\sqrt{x}$ (in red) or $\ln(x)$ (in green) have a limit $f(x)/x$ equal to $0$ and both have a "parabolic branch"  of direction following the $x$-axis:
			 \begin{figure}[H]
				\centering
				\includegraphics{img/analysis/asymptote_parabolic_branche_example_x.jpg}
				\caption[]{Graphical representation of an parabolic branch example following $x$-axis}
			\end{figure}
			\end{tcolorbox}
			
			\item The representative curve of $f(x)$ has an infinite branch (this branch has not close form asymptote) and the only one thing we can say is that $y$-axis is the direction of this asymptote (we then also speak of "parabolic branch"):
			
			\begin{tcolorbox}[colframe=black,colback=white,sharp corners]
			\textbf{{\Large \ding{45}}Example:}\\\\
			The function $f(x)=x^2$ has an infinite $f(x)/x$ limit and therefore has a parabolic branch of direction following the $y$-axis.
			 \begin{figure}[H]
				\centering
				\includegraphics{img/analysis/asymptote_parabolic_branche_example_y.jpg}
				\caption[]{Graphical representation of a parabolic branch example following $y$-axis}
			\end{figure}
			\end{tcolorbox}
			
			\item A function $f(x)$ is say to have a "\NewTerm{curvilinear asymptote}\index{curvilinear asymptote}" if it satisfies:
			
			for $n>1 $where for recall $P_n(x)$ is a polynomial of degree $n$.
			\begin{tcolorbox}[colframe=black,colback=white,sharp corners]
			\textbf{{\Large \ding{45}}Example:}\\\\
			The function :
			
			has a curvilinear asymptote that is:
			
			Indeed:
			
			 \begin{figure}[H]
				\centering
				\includegraphics{img/analysis/asymptote_curvilinear_example.jpg}
				\caption[]{Graphical representation of curvilinear asymptote}
			\end{figure}
			\end{tcolorbox}
		\end{enumerate}
	\end{enumerate}
	There are many techniques for finding limits that apply in various conditions. It's important to know all these techniques, but it's also important to know when to apply which technique. Some basic techniques who doesn't involve derivation are:
	\begin{figure}[H]
		\centering
		\includegraphics{img/analysis/finding_limits.jpg}
		\caption[Basic techniques for finding limits]{Basic techniques for finding limits (source: Khan Academy)}
	\end{figure}
	
	
	\pagebreak
	\subsubsection{Concavity/Convexity of a function}
	We will define now a property which at first sight may seem of no interest as it is so trivial but which we shall find in the section of Statistics for the demonstration of an important relation named "Jensen inequality" and which is of major importance Finance and Insurance for the valuation of options and premiums (\SeeChapter{see section Economy page \pageref{finance convex function}}) and especially for the application of the Jensen's inequality (\SeeChapter{see section Statistics page \pageref{jensen inequality}}).

	Consider the following figure:
	\begin{figure}[H]
		\centering
		\includegraphics{img/analysis/concavity_convexity.jpg}
		\caption[]{Graphical representation of curvilinear asymptote}
	\end{figure}
	\textbf{Definition (\#\mydef):} In mathematics, a real function of a real variable is say to a "\NewTerm{convex function}\index{convex function}\label{convex function}" if, viewed from below, its graph is convex (in bump); we mean that if $A$ and $B$ are two points of the graph of the function, the segment $[AB]$ is entirely situated above the graph. It is the same to say that the "\NewTerm{epigaph}" (the set of points above the graph) is a concave set. Conversely, a function whose graph, as seen from below, is seen as a cave, is say to be a "\NewTerm{concave function}\index{concave function}\label{concave function}". It is the same to say that the "\NewTerm{hypograph}" (the set of points below) is a convex set.
	
	By specifying by the values of the function what are the points $A$ and $B$ above, we get often an equivalent definition of the convexity of a function: a function defined on a real interval $I$ is convex when, for any $x_1$ and $x_2$ of $I$ and all $t$ in $[0,1]$ we have:
	
	When the inequality is strict, then we obviously speak of a "\NewTerm{strictly convex function}".
	
	\begin{tcolorbox}[title=Remarks,colframe=black,arc=10pt]
	Without proof, just by looking to the above chart, we will assume quite obvious that convexity implies $f''(x)\geq 0$ for all $x$. Just as before, strict convexity occurs when
the inequality is strict.
	\end{tcolorbox}

	By extension (common sense from my point of view), a function $f$ is concave if $-f$ is convex (which is trivial with the pay-off function - see section Economy page \pageref{finance convex function} - profile of options from seller or buyer point of view ).
	
	\begin{tcolorbox}[colframe=black,colback=white,sharp corners]
	\textbf{{\Large \ding{45}}Examples:}\\\\
	E1. Consider $f(x)=x^{2}$ . The first derivative of $f(x)$ is given by $\frac{\mathrm{d}}{d\mathrm{d} x} f=2 x$ and its second derivative by $\frac{\mathrm{d}^{2}}{\mathrm{d} x^{2}} f=2$. Since this is  always strictly greater than $0,$ we have proven that $f(x)=x^{2}$ is strictly convex.\\
	
	E2. $f(x)=\log (x)$ . The first derivative is $\frac{\mathrm{d}}{\mathrm{d} x} f=\frac{1}{x}$ and its second derivative is given by $\frac{\mathrm{d}^{2}}{\mathrm{d} x^{2}} f=-\frac{1}{x^{2}}$. Since this is negative for all $x>0$, we have proven that $\log(x)$ is a concave function over $\mathbb{R}_{+}$ .
	\end{tcolorbox}
	
	\subsubsection{Euler theorem for homogeneous functions}
	We have to introduce now a definition and a theorem that will be very important for quite advanced concepts in our study of Physics.
		
	\textbf{Definition (\#\mydef):} A "\NewTerm{homogeneous function}\index{homogeneous function}\label{homogeneous function}" is one with multiplicative scaling behaviour: if all its arguments are multiplied by a factor, then its value is multiplied by some power of this factor. For example, a homogeneous real-valued function of two variables $x$ and $y$ is a real-valued function that satisfies the condition $f(t x,t y)=t ^{k}f(x,y)$ for some constant $k$ and all real numbers $t\in \mathbb{R}^*$. The constant $k$ is named  the "\NewTerm{degree of homogeneity}".
	
	More generally, if $f:\mathbb{R}^n\mapsto \mathbb{R}$ is a function between two vector spaces over a field $F$, and $k$ is an integer, then $f$ is said to be homogeneous of degree $k$ if:
	
	or more commonly written:
	
	
	\begin{tcolorbox}[colframe=black,colback=white,sharp corners]
	\textbf{{\Large \ding{45}}Example:}\\\\
	The function $f(x,y)=x^{2}+y^{2}$ is homogeneous of degree $2$. Indeed:
	
	\end{tcolorbox}
	
	Now let us suppose $f:\mathbb{R}^n\mapsto \mathbb{R}$ is continuously differentiable on $\mathbb{R}$. We know that a function is homogeneous of degree $k$ if:
	
	Differentiating both sides with respect to $t$, we get:
	
	by the total exact differential chain rule of $f$. 
	
	So the "\NewTerm{Euler theorem for homogeneous functions}\index{Euler theorem for homogeneous functions}\label{Euler theorem for homogeneous functions}" can be summarized as following:
	
	\begin{tcolorbox}[title=Remark,colframe=black,arc=10pt]
	Remember that we have proved in the section of Differential and Integral Calculus (see page \pageref{total exact differential}) that:
	
	Therefore:
	
	\end{tcolorbox}
	
	Then if we choose to set with the special case $k=1$ (it's the case that will interest us the most in Physics) then the above becomes:
	
	or explicitly for $n=2$:
	
	if $f(tx)$ is homogeneous of degree $1$.
	
	This is an important result we will need in Lagrangian Mechanics that will be useful for the study of the Lagrangian of a free particle in Special Relativity and in Quantum Cosmology.
	
	
			
	\pagebreak
	\subsection{Logarithms}\label{logarithms}
	We hesitated to put the definition of logarithms in the section Calculus. After a moment of reflection, we decided it was better to put it in this section because to understand it well, we must be aware of the concept of limits, of definition domain and of the power function. We hope that our choice will suit you best.
	
	Given the power (bijective) function of any base where $a \in \mathbb{R}_{+}^{*}/1$ (we exclude $1$ otherwise it is not bijective) and denoted for recall by:
	
	for which it corresponds to each real number $x$, exactly one positive number $a^x$ (the image set of the function is in $\mathbb{R}$) such as the powers calculations rules are applicable (\SeeChapter{see section Calculus page \pageref{power rules calculations}}).
	
	We know that for such a function that if $a>1$, then $f (x)$ is an increasing and positive (monotone) in $\mathbb{R}$, and if $0<a<1$, then $f(x)$ is positive and decreasing (monotone) in $\mathbb{R}$.
	
	\begin{tcolorbox}[title=Remarks,colframe=black,arc=10pt]
	\textbf{R1.} If $a>1$, when $x$ decreases to negative values, the graph of $f (x)$ approaches the $x$-axis. Thus, the $x$ axis is a horizontal asymptote. When $x$ increases in positive values, the graph rises quickly. This type of change is characteristic of the "\NewTerm{law of exponential of growth}\index{law of exponential of growth}" and $f(x)$ is sometimes named "\NewTerm{growing function}\index{growing function}"... If $0<a<1$, when $x$ increases, the graph tends asymptotically to the $x$-axis. This type of variation is known as an "\NewTerm{exponential decay}\index{exponential decay}".\\
	
	\textbf{R2.} By studying $a^x$, we exclude the case where $a\leq 0$ and $a=1$. Notice that if $a<0$, then $a^x$ is not a real number for many values of $x$ (we recall that the whole image set is forced to $\mathbb{R}$ in our previous definition). If $a=0$, the $a^0=0$ is not defined. Finally, if $a=1$, then $a^x=1$ for all $x$ and the graph of $f(x)$ is a horizontal line.
	\end{tcolorbox}
	As the power function $f (x)$ is bijective then there exists an inverse function $f^{-1}(x)$ and is named "\NewTerm{logarithm function}\index{logarithm function}" of base $a$ and is denoted by:
	
	and therefore:
	
	if and only if $y=a^x$.
	
	More generally it is defined by:
	
	
	Considering $\log_a(x)$ as an exponent, we have the following properties:	
	\begin{table}[H]
	\begin{center}
		\begin{tabular}{|c|c|}
			  \hline
			  \rowcolor[gray]{0.75}Properties & Justification \\ \hline
			  $\log_a1=0$ & $a^0=1$ \\ \hline
			  $\log_aa=a$ & $a^1=a$ \\ \hline
			  $\log_aa^x$ & $a^x=a^x$ \\ \hline
			  $a^{\log_a(x)}=x$ & $a^{\log_a(x)}=a^y=x$ \\
			  \hline
		\end{tabular}
		\end{center}
		\caption{Properties of the logarithm in base $a$}
	\end{table}
	\begin{tcolorbox}[title=Remarks,colframe=black,arc=10pt]
	\textbf{R1.} The word "logarithm" means "number of logos", "logos" meaning "reason" or "ratio".\\
	
	\textbf{R2.} The logarithm and power functions are defined by their bases (the number $a$). When using a power of $10$ as a base ($10, 100, 1000, ...$) then we speak of "\NewTerm{common system}\index{common system}" because they have for $\log$ successive integers.\\
	
	\textbf{R3.} The integer part of the logarithm is named the "\NewTerm{characteristic}"\index{characteristic of a logarithm}.
	\end{tcolorbox}
	There are two types of logarithms that we find almost exclusively in mathematics and physics: the logarithm of base $10$, logarithm of base $e$ (the latter often named "\NewTerm{natural logarithm}\index{natural logarithm}") and logarithm of base $2$ for information theory.
	
	First the on in base $10$ (the most used on graphical representations):
	
	abusively noted:
	
	and the base (Eulerian) $e$:
	
	historically noted:
	
	the "$n$" meaning "Napierian".
	
	\begin{tcolorbox}[title=Remark,colframe=black,arc=10pt]
	Historically, it is John Napier (1550-1617) whose name was Latinized "Napier" that we own the study of logarithms and the name of "natural logarithms" which aimed to facilitate greatly the time for manual calculations.
	\end{tcolorbox}
	In English for the logarithm function in base-$10$ logarithmic we need to calculate:
	
	ask the following question: at what power $n\in \mathbb{R}$ should we raise $10$ to get $x$?
	
	Formally, this consist to solve the equation:
	
	or written in another way:
	
	with $x$ being known and therefore in base $10$:
	
	The logarithm in base $10$ is used a lot in graphical representations in the scientific perspective when we look at amplitudes variations. For example with Maple 4.00b  we have for two sine function  having respectively for their respective mean the same amplitude variation of $50\%$ visible below that do not highlights necessarily this fact trivially:
	
	\texttt{>plot({10+0.5*10*sin(x),100+100*0.5*sin(x)},x=1..10);}
	
	\begin{figure}[H]
		\centering
		\includegraphics{img/analysis/two_sinus_for_comparison_without_logarithm_scale.jpg}
		\caption[]{Plot with Maple 4.00b with two sine functions having same amplitude change compared to their average}
	\end{figure}
	While in logarithmic scale, this gives:
	
	\texttt{>with(plots):\\
	>logplot({10+0.5*10*sin(x),100+100*0.5*sin(x)},x=1..10);}
	
	\begin{figure}[H]
		\centering
		\includegraphics{img/analysis/two_sinus_for_comparison_with_logarithm_scale.jpg}
		\caption[]{Same plot with Maple 4.00b but with the $y$-axis in logarithm (base $10$) scale}
	\end{figure}
	For the logarithmic function in Eulerian base $e$ it is necessary to calculate:
	
	to ask ourselves the following question: at what power $n\in \mathbb{R}$ we must raise the number $e$ to get $x$?
	
	Formally this consists to solve the equation:
	
	with $x$ being known and therefore:
	
	Technically, we say that the exponential function (see below for details):
	
	is the inverse bijection of the $\ln (x)$ function.
	\begin{figure}[H]
		\centering
		\includegraphics{img/analysis/bijection_ln_x_exp_x.jpg}
		\caption{Graphical representation of the correspondence between the natural logarithm and the exponential}
	\end{figure}
	But what is that "Eulerian" number also named "\NewTerm{Euler number}\index{Euler number}\label{Euler number}"? Why do we find so often in physics and mathematics? Let us first determine the origin of its value:
	
	with $\alpha \in \mathbb{N}$ and when $\alpha \rightarrow +\infty$.
	\begin{tcolorbox}[title=Remark,colframe=black,arc=10pt]
	The second term of the equality is typically the type of expression that we find in compound interest in finance (\SeeChapter{see section Economy page \pageref{compound interest}}) or in any other type of identical increase factor. And what interests us in this case is when this type of increase tends to infinity.
	\end{tcolorbox}
	The interest we have to pose the problem as in this way is that if we do tend $\alpha \rightarrow +\infty$ the function written above tends to $e$ and this function has the special property of being calculable more or less easily for historical reasons using Newton's binomial.
	
	So according to the development of the Newton binomial (\SeeChapter{see section Calculus page \pageref{binomial coefficient development}}) we can write:
	
	This development is similar to the Taylor expansion (\SeeChapter{see section Sequences and Series page \pageref{taylor series}}) of some given functions for particular cases of development values (hence the reason why we find this eulerian number in many places that we will later).
	
	By performing some algebraic transformations that should now be obvious to the reader, we find:
	
	We see in this last equality that the function $\left(1+\frac{1}{\alpha}\right)^\alpha$ is increasing when $\alpha$ increases. Indeed, when we move from $\alpha$ to the value $\alpha+1$ each term of this sum increases:
	
	Let us prove now that the variable $\left(1+\frac{1}{\alpha}\right)^\alpha$ is bounded. By seeing that:
	
	So we get by analogy with the extended expression of Newton binomial determined just previously the following order relation:
	
	On the other hand:
	
	We then can write the inequality:
	
	The underlined terms constitute a geometric sequence of reason $q=1/2$ (\SeeChapter{see section Sequences and Series page \pageref{geometric sequence}}) and whose first term is $1$. If follows using the result obtained in the section of Sequences ans Series, that we can write:
	
	Therefore, we have:
	
	We have therefore proved that the function $\left(1+\frac{1}{\alpha}\right)^\alpha$ is bounded.
	
	The limit:
	
	tends to this limited value that is the number $e$ whose value is:
	
	The prior previous relation is also know in the following form after an obvious change of variable:
	
	\begin{tcolorbox}[title=Remark,colframe=black,arc=10pt]
	As we have proved it in the section Numbers, this number is irrational.
	\end{tcolorbox}
	We can then define the "\NewTerm{natural exponential function}\index{natural exponential function}\label{natural exponential function}" (reciprocal of the natural logarithm function) by:
	
	also sometimes denoted by:
	
	The number $e$ and the function that determines it are very useful. We find them in all areas of mathematics and physics and thus in almost all the chapters of this book.
	
	As we have proved it in the section of Differential and Integral Calculus the functions $e^x$ has for remarkable property that its derivative is equal to itself:
	
	and this is used a lot for the resolution of differential equations in physics and finance.
	
	Logarithms have several properties. Here are the most important one in our point of view (we are referring to a given base $X$) and that are very useful in physics, electronics, chemistry and so on...
	
	Let us begin. First:
	
	If we put $X^m=a$ and $X^n=b$ we get:
	
	If we have the special case when $a=b$ then:
	
	Now let us try to express:
	
	in another way. For this we put first:
	
	which leads us to the development:
	
	Now let us try to express:
	
	with $n\in \mathbb{N}^{*}$ in another way. For this we put first:
	
	which leads us to the development:
	
	There is  also another relation used a lot of time in physics in respect to the change of logarithm basis. The first relation is trivial and follows from the algebraic properties of logarithms:
	
	the second one:
	
	is a bit less trivial and requires perhaps a proof (we used it for our study of continued fractions in the section Number Theory).
	\begin{dem}
	We first use the equivalent equations (of the first relation above):
	
	and we proceed as follows:
	
	What finally brings us to:
	
	\begin{flushright}
		$\blacksquare$  Q.E.D.
	\end{flushright}
	\end{dem}
	
	\pagebreak
	\subsection{Convolutions}\label{convolution}
	A convolution is a mathematical operation on two functions to produce a third function, that is typically viewed as a modified version of one of the original functions, giving the integral of the point-wise multiplication of the two functions as a function of the amount that one of the original functions is translated.
	
	There are different type of convolutions and as always, we will focus in this book only on the one that are actually used in other sections of this books.
	
	\subsubsection{Continuous and Discrete Linear Convolution Product}
	\textbf{Definition (\#\mydef):} The "\NewTerm{continuous convolution}\index{continuous convolution}\label{continuous convolution}" of two continuous signals $x(t)$ and $h(t)$ is defined as:
	
	
	\begin{tcolorbox}[title=Remark,colframe=black,arc=10pt]
	The convolution is also sometimes denoted with different symbols:
	
	or:
	
	\end{tcolorbox}
	We will now prove some properties of the convolutions on focusing only on those use in the other sections of this book!
	\begin{itemize}
		\item[P1.] Convolution is commutative:
		
		\begin{dem}
		By making the change of variable $\lambda=t-\tau$, in one form of the definition of convolution:
		
		it becomes:
		
		proving that convolution is commutative.
		\begin{flushright}
			$\blacksquare$  Q.E.D.
		\end{flushright}
		\end{dem}
		
		\item[P2.] Convolution is associative:
		
		\begin{dem}
		The proof is easier to understand if we consider a limited integral (but you can change the bounds to the infinity one and you will fall back on the general result):
		
		 The proof may not be obvious for many readers. So we recommend to see the equivalent proof for the discrete version further below.
		\begin{flushright}
			$\blacksquare$  Q.E.D.
		\end{flushright}
		\end{dem}
		
		\item[P3.] Convolution is distributive:
		
		\begin{dem}
		
		\begin{flushright}
			$\blacksquare$  Q.E.D.
		\end{flushright}
		\end{dem}
		
		\item[P4.] Relation with differentiation:
		
		\begin{dem}
		
		\begin{flushright}
			$\blacksquare$  Q.E.D.
		\end{flushright}
		\end{dem}
	\end{itemize}
	We will assume as obvious that these properties also apply to the discrete convolution that we will introduce now!
	
	Typically, $y(t)$ is the output of a system characterized by its impulse response function $h(t)$ with input $x(t)$.
	
	\textbf{Definition (\#\mydef):} The "\NewTerm{discrete convolution}\index{discrete convolution}" of two discrete signals $x[n]$ and $h[n]$ is defined as:
	
	If $h[m]$ is finite, e.g.:
	
	the convolution becomes
	
	If the system in question were a causal system in time domain:
	
	the above would become:
	
	This is also written:
	
	\begin{tcolorbox}[colframe=black,colback=white,sharp corners]
	\textbf{{\Large \ding{45}}Examples:}\\\\
	E1. Consider:
	
	We get visually (it's not obvious!):
	\begin{figure}[H]
		\centering
		\includegraphics[scale=0.6]{img/analysis/discrete_convolution.jpg}
		\caption{Discrete convolution example}
	\end{figure}
	The details steps are given by first considering that:
	
	and the fact that:
	And we have for recall:
	
	And in tabular form:
	\begin{table}[H]
	\centering
		\begin{tabular}{lccccc}
		$n$: & $0$ & $1$ & $2$ & $3$ & $4$ \\
		$m=0$ & ${\color{blue}{2}}\cdot {\color{red}{3}}=6$ & ${\color{blue}{1}}\cdot {\color{red}{3}}=3$ & ${\color{blue}{3}}\cdot {\color{red}{3}}=9$ & $0$ & $0$ \\
		$m=1$ & $0$ & ${\color{blue}{2}}\cdot {\color{red}{2}}=4$ & ${\color{blue}{1}}\cdot {\color{red}{2}}=2$ & ${\color{blue}{3}}\cdot {\color{red}{2}}=6$ & $0$ \\
		$m=2$ & $0$ & $0$ & ${\color{blue}{2}}\cdot {\color{red}{1}}=2$ & ${\color{blue}{1}}\cdot {\color{red}{1}}=1$ & ${\color{blue}{3}}\cdot {\color{red}{1}}=3$ \\ \hhline{|=|=|=|=|=|=|}
		$c[n]$ & $6$ & $7$ & $13$ & $7$ & $3$
		\end{tabular}
	\end{table}
	The $c[n>4]$ being all equal to zero.\\
	\end{tcolorbox}
	\begin{tcolorbox}[colframe=black,colback=white,sharp corners]
	
	
	The final result will then be written:
	
	Such a calculations can obviously be done very simply with softwares like \texttt{R} for example (see the corresponding companion book).\\
	
	E2. A well known example is the convolution of two gaussians (\SeeChapter{see section Statistics page \pageref{sum of two random normal variables}}) that also result in a... gaussian. And obviously the discrete version of the convolution also give a gaussian as illustrated below with \texttt{R} (see the corresponding companion book for more details):
	\begin{figure}[H]
		\centering
		\includegraphics[scale=0.55]{img/analysis/discrete_convolution_gaussians_R.jpg}
	\end{figure}
	where we have as the reader have noticed above, the following convolution $\mathcal{N}(5,20)*\mathcal{N}(10,3)$ and we see obviously that the result gives a gaussian of $\mathcal{N}(5+10=15,\sqrt{20^2+3^2}=15)$.
	\end{tcolorbox}
	Let us now prove that the discrete convolution is also associative for the discrete convolution (as promised earlier!):
	\begin{dem}
	
	\begin{flushright}
		$\blacksquare$  Q.E.D.
	\end{flushright}
	\end{dem}
	However, in image processing, we often consider convolution in spatial domain where causality does not apply.
	
	If $h[m]=h[-m]$ is symmetric (almost always true in image processing),
	then replacing $m$ by $-m$ we get:
	
	We see that now the convolution is the same as the {\em correlation} of the two functions. 
	
	If the input $x[m]$ is finite (always true in reality), i.e.:
	
	its index $n+m$ in the convolution has to satisfy the following for $x$ to be in the valid non-zero range:
		
	or correspondingly, the index $n$ of the output $y[n]$ has to satisfy:
	
	When the variable index $m$ in the convolution is equal to $k$, the 
	index of output $y[n]$ reaches its lower bound $n=-k$; when $m=-k$, 
	the index of $y[n]$ reaches its upper bound $n=N+k-1$. In other words,
	there are $N+2k$ valid (non-zero) elements in the output:
	
	
	Assume the size of the input signal $x[n]$ is $N$ ($n=0,\cdots,n=N-1$) and the size of $h$ is $M=2k+1$ (usually an odd number), then the size of the resulting convolution $y=x*h$ is $N+M-1=N+2k$. However, as it is usually desirable for the output $y$ to have the same size as the input $x$, we can drop $k$ components at each end of $y$. When the size of $h$ is even,we can drop $k$ components at one end and $k-1$ from the other of $y$.
	
	The code segment for this $1$D convolution $y=x*h$ is given below. 
	
	In particular, if the elements of the kernel are all the same (an average operator or a low-pass filter), then we can speed up the convolution process while sliding the kernel over the input signal by taking care of only the two ends of the kernel.
	
	\subsubsection{Matrix Convolution}\label{matrix convolution}
	In image processing, all of the discussions above for one-dimensional convolution are generalized into two dimensions!
	
	"\NewTerm{Matrix convolution}\index{matrix convolution}" is the treatment of a matrix by another one which is named the "\NewTerm{kernel}". Most of the time, the convolution matrix filter uses a first matrix which is the image to be treated. The image is a bi-dimensional collection of pixels in rectangular coordinates. The used kernel $h$ of always \underline{odd} $k\times k$ dimensions  depends on the effect you want!
	
	
	For example, if we have the following matrices:
	
	Then (don't forget that the columns and row of the kernel matrix are flipped!):
	
	That can be illustrated as following ($h$ has already been double-flipped in the figure below):
	\begin{figure}[H]
		\centering
		\includegraphics[width=0.8\textwidth]{img/analysis/matrix_convolution.jpg}
		\caption{Matrix Convolution}
	\end{figure}
	The full example can be run and reproduced with a free software like R:
	\begin{figure}[H]
		\centering
		\includegraphics[width=1.0\textwidth]{img/analysis/matrix_convolution_r.jpg}
	\end{figure}
	Or for people who may prefer a slow implementation in C++:
	\begin{lstlisting}[language={C++}, caption={C++ for matrix convolution}]
	// find center position of kernel (half of kernel size)
	kCenterX = kCols / 2;
	kCenterY = kRows / 2;

	for(i=0; i < rows; ++i)              // rows
	{
	    for(j=0; j < cols; ++j)          // columns
	    {
	        for(m=0; m < kRows; ++m)     // kernel rows
	        {
	            mm = kRows - 1 - m;      // row index of flipped kernel
	
	            for(n=0; n < kCols; ++n) // kernel columns
	            {
	                nn = kCols - 1 - n;  // column index of flipped kernel
	
	                // index of input signal, used for checking boundary
	                ii = i + (kCenterY - mm);
	                jj = j + (kCenterX - nn);
	
	                // ignore input samples which are out of bound
	                if( ii >= 0 && ii < rows && jj >= 0 && jj < cols )
	                    out[i][j] += in[ii][jj] * kernel[mm][nn];
	            }
	        }
	    }
	}
	\end{lstlisting}

	\pagebreak
	\subsection{Integral Transforms}
	An "\NewTerm{integral transform}\index{integral transform}" is an operator that maps functions from one space to another. Formally:
	
	Now the practical motivation for an integral transform is to reduce the complexity of the problem i.e the mathematical operations will be much easier to handle in the image space (typically the resolution of differential equations!).
	
	However, as much as it is fun to do work in the image space, one has to be able to interpret the results in the original space. To do so requires the study of the operator $K$. Usually one knows a priori the nature of the function $f$ by the nature of the problem one is dealing. Hence the study of integral transforms is the study of the operator $\mathcal{T}$. Two properties come very easily:
	
	To ensure invertibility, one has to show that the kernel space only contains the null function.

	The Fourier and Laplace transforms are for example continuous (integral) transforms of continuous functions (even if there exist a discrete version of the Fourier transform!).

	The Laplace transform maps a function $f(t)$ to a function $\mathcal{F}(s)$ of the complex variable $s$, where:
	
	Since the derivative:
	
	maps to $s\mathcal{F}(s)$, the Laplace transform $\mathcal{L}$ of a linear differential equation is an algebraic equation. Thus, the Laplace transform is useful for, among other things, solving linear differential equations.
	
	If we set the real part of the complex variable $s$ to zero, $\sigma=0$, the result is the Fourier transform $\mathcal{F}(\mathrm{i}\omega)$ which is essentially the frequency domain representation of $f(t)$ (note that this is true only if for that value of $\sigma$ the formula to obtain the Laplace transform of $f(t)$ exists, i.e., it does not go to infinity).
	
	The $\mathcal{Z}$-transform is essentially a discrete version of the Laplace transform and, thus, can be useful in solving difference equations, the discrete version of differential equations. The $\mathcal{Z}$-transform maps a sequence $f[n]$ to a continuous function $F(z)$ of the complex variable $z=re^{\mathrm{i}\Omega}$.
	
	If we set the magnitude of $z$ to unity, $r=1$, the result is the Discrete Time Fourier Transform (DTFT) $\mathcal{F}(\mathrm{i}\Omega)$ which is essentially the frequency domain representation of $f[n]$.
	
	\begin{tcolorbox}[title=Remarks,colframe=black,arc=10pt]
	\textbf{R1.} The three integral transformations mentioned above (Fourier, Laplace, $\mathcal{Z}$) are only a sample of what exists in practice. Let us also mention the Hartley transform, the Mellin transform, the Weierstrass transform, the Hankel transform (Fourier-Bessel), the Abel transform, the Hilbert transform, the Gauss-Weierstrass transform, etc.\\
	
	\textbf{R2.} For information we will calculate cases and prove properties of the Fourier, Laplace, $\mathcal{Z}$ and Hilbert transforms only used in concrete applications in the industry and mainly useful in other chapters of this book! Indeed a rather general presentation would require a few hundred pages and it is not the objective of this book, as you already know it, to prove mathematical properties not associated with concrete cases.
	\end{tcolorbox} 
	
	\subsubsection{Fourier Transform} \label{fourier transform analysis}
	So we have already introduced Fourier Transforms in the section of Sequences and Series (page \pageref{fourier transform}), we will come back here more in details on this topic in (we wish) are more structured way...
	
	\paragraph{Continuous Time Fourier Transform}\mbox{}\\\\	
	The Fourier expansion coefficient $X[k]$ of a continuous periodic  signal $x_T(t)=x_T(t+T)$ is:
	
	and the Fourier expansion of the signal is:
	
	which can also be written as:
	
	where $X(k\omega_0)$ is defined as:
	
	
	When the period of $x_T(t)$ approaches infinity $T \rightarrow +\infty $, the 
	periodic signal $x_T(t)$ becomes a non-periodic signal $x(t)$ and the following 
	will result:
	\begin{itemize}
	
	\item Interval between two neighbouring frequency components becomes zero:
	
	
	\item Discrete frequency becomes continuous frequency:
	
	
	\item Summation of the Fourier expansion becomes an integral:
	
	the second equal sign is due to the general fact:
	
	
	\item Time integral over $T$ becomes over the entire time axis:
	
	\end{itemize}
	
	In summary, when the signal is non-periodic $x(t)=\lim_{T\rightarrow +\infty}x_T(t)$, the Fourier expansion becomes Fourier transform. The forward transform (analysis) is:
	
	and the inverse transform (synthesis) is:
	
	
	Comparing Fourier coefficient of a periodic signal $x_T(t)$ with Fourier spectrum of a non-periodic signal $x(t)$:
	
		
	The spectrum of a time signal can be denoted by $X(\omega)$ or $X(f)$ to emphasize the fact that the spectrum represents how the energy contained in the signal is distributed as a function of frequency $\omega$ or $f$. Moreover, if $X(f)$ is used, the factor $1/2\pi$ in front of the inverse transform is dropped so that the transform pair takes a more symmetric form. On the other hand, as Fourier transform can be considered as a special case  of Laplace transform when the real part $\sigma$ of the complex argument $s=\sigma+\mathrm{i}\omega=\mathrm{i}\omega$ is zero:
	
	it is also natural to denote the spectrum of $x(t)$ by $X(\mathrm{i}\omega)$.
	
	Ok this done let us see now some important example for physics (especially quantum physics) and signal processing!
	\begin{tcolorbox}[colframe=black,colback=white,sharp corners]
	\textbf{{\Large \ding{45}}Examples:}\\\\
	E1. Consider the unit impulse function (Dirac function):
	
	Therefore:
	
	and if $a=0$ we have then:
	
	E2. If the spectrum of a signal $x(t)$ is a delta function in frequency domain $X(\mathrm{i}\omega)=2\pi\;\delta(\omega)$, the signal can be found to be:
	
	i.e.:
	
	E3. We consider:
	
	The spectrum is:
	
	This is the sinc function with a parameter $a$.\\
	
	Note that the height of the main peak is $2a$ and it gets taller and narrower as	$a$ gets larger.
	
	\end{tcolorbox}
	
	
	\begin{tcolorbox}[colframe=black,colback=white,sharp corners]
	 Also note:
	
	When $a$ approaches infinity, $x(t)=1$ for all $t$, and the spectrum becomes:
	
	Recall that the Fourier coefficient of $x(t)=1$ is:
	
	which represents the energy contained in the signal at $k=0$ (DC component at zero frequency), and the spectrum $X(\mathrm{i}\omega)=X[k]/\omega$ is the energy density or distribution which is infinity at zero frequency.\\
	
	The integral in the above transform is an important formula to be used frequently later:
	
	which can also be written as:
	
	Switching $t$ and $f$ in the equation above, we also have:
	
	representing a superposition of an infinite number of cosine functions of all
	frequencies, which cancel each other any where along the time axis except at
	$t=0$ where they add up to infinity, an impulse. \\
	
	E3. Let us now consider:
	
	The spectrum of the cosine function is:
		
	\end{tcolorbox}
	
	\begin{tcolorbox}[colframe=black,colback=white,sharp corners]
	
	The spectrum of the sine function:
	
	can be similarly obtained to be:
	
	Again, these spectra represent the energy density distribution of the sinusoids, while the corresponding Fourier coefficients:
	
	and:
	
	represent the energy contained at frequency $\omega=\omega_0$.
	\end{tcolorbox}
	
	
	
	
	\subparagraph{Properties of Fourier Transform}\mbox{}\\\\	
	The properties of the Fourier transform are summarized below. For some of them already proved in the section of Sequences and Series we will not give the poof. But for others we will do it again! 
	
	In the following, we assume 
	$\;\;{\cal F}(x(t))=X(\mathrm{i}\omega)$ and ${\cal F}(y(t))=Y(\mathrm{i}\omega)$.
	
	\begin{itemize}
	
	\item[P1.] Linearity:
	
	
	\item[P2.] Time shift:
	
	\begin{dem} Let $t'=t\pm t_0$, i.e., $t = t' \mp t_0$, we have:
	\begin{eqnarray}
	{\cal F}(x(t \pm t_0))&=&\int\limits_{-\infty}^{+\infty} x(t\pm t_0)
		e^{-\mathrm{i}\omega t} \mathrm{d}t
		=\int\limits_{-\infty}^{+\infty} x(t')e^{-\mathrm{i}\omega(t'\mp t_0)} \mathrm{d}t'
		\nonumber \\
		&=& e^{\pm \mathrm{i}\omega t_0}
		\int\limits_{-\infty}^{+\infty} x(t')e^{-\mathrm{i}\omega t'} \mathrm{d}t'=X(\mathrm{i}\omega)e^{\pm \mathrm{i}\omega t_0}
		\nonumber
	\end{eqnarray}
	\begin{flushright}
		$\blacksquare$  Q.E.D.
	\end{flushright}
	\end{dem}
	
	\item[P3.] Frequency shift:
	
	\begin{dem} Let $\omega'=\omega\pm \omega_0$, i.e., $\omega = \omega'\mp\omega_0$,
	we have:
	
	\begin{flushright}
		$\blacksquare$  Q.E.D.
	\end{flushright}
	\end{dem}
	
	\item[P4.] Time reversal:
	
	\begin{dem}
	
	Replacing $t$ by $-t'$, we get:
	
	\begin{flushright}
		$\blacksquare$  Q.E.D.
	\end{flushright}
	\end{dem}
	
	\item[P5.] Even and Odd Signals and Spectra:
	
	If the signal $x(t)$ is an even (or odd) function of time, its spectrum $X(\mathrm{i}\omega)$ is an even (or odd) function of frequency:
	
	and:
	
	\begin{dem} If $x(t)=x(-t)$ is even, then according to the time reversal property, we have:
	
	i.e., the spectrum $X(\mathrm{i}\omega)=X(-\omega)$ is also even. Similarly, if $x(t)=-x(-t)$ is odd, we have:
	
	i.e., the spectrum $X(\mathrm{i}\omega)=-X(-\omega)$ is also odd.
	\begin{flushright}
		$\blacksquare$  Q.E.D.
	\end{flushright}
	\end{dem}
	
	\item[P6.] Time and frequency scaling:
	
	\begin{dem}
	Let $u=at$, i.e., $t=u/a$, where $a>0$ is a scaling factor, we have:
	
	Note that when $a<1$, time function $x(at)$ is stretched, and $X(\mathrm{i}\omega/a)$ is compressed; when $a>1$, $x(at)$ is compressed and $X(\mathrm{i}\omega/a)$ is stretched.	This is a general feature of Fourier transform, i.e., compressing one of the $x(t)$ and $X(\mathrm{i}\omega)$ will stretch the other and vice versa. In particular, when $a\rightarrow 0$, $x(at)$ is stretched to approach a constant, and $X(\mathrm{i}\omega/a)/a$ is compressed with its value increased to approach an impulse; on the other	hand, when $a \rightarrow +\infty$, $ax(at)$ is compressed with its value increased to approach an impulse and $X(\mathrm{i}\omega/a)$ is stretched to approach a constant.
	\begin{flushright}
		$\blacksquare$  Q.E.D.
	\end{flushright}
	\end{dem}
	
	\item[P7.] Complex Conjugation:
	
	
	\begin{dem} Taking the complex conjugate of the inverse Fourier transform, we get:
	
	Replacing $\omega$ by $-\omega'$ we get the desired result:
	
	\begin{flushright}
		$\blacksquare$  Q.E.D.
	\end{flushright}
	\end{dem}
	We further consider two special cases:
	\begin{itemize}
	\item If $x(t)=x^*(t)$ is real, then:
	
	i.e., the real part of the spectrum is even (with respect to frequency $\omega$), and the imaginary part is odd:
	
	\item If $x(t)=-x^*(t)$ is imaginary, then:
	
	i.e., the real part of the spectrum is odd, and the imaginary part is even:
	
	\end{itemize}
	
	If the time signal $x(t)$ is one of the four combinations shown in the table (real even, real odd, imaginary even, and imaginary odd), then its spectrum $X(\mathrm{i}\omega)$ is given in the corresponding table entry:
	\vskip0.2in
	\begin{table}[H]
		\centering
		\begin{tabular}{c||c|c} \hline
			& if $x(t)$ is real		& if $x(t)$ is imaginary	\\ 
			& $X_r$ even, $X_i$ odd	& $X_r$ odd, $X_i$ even \\ \hline \hline
		if $x(t)$ is Even	&			&		\\
		$X_r$ and $X_i$ even	& $X_i=0$, $X=X_r$ even & $X_r=0$, $X=X_i$ even	\\ \hline
		if $x(t)$ is Odd	& 			&		\\
		$X_r$ and $X_i$ odd	& $X_r=0$, $X=X_i$ odd	& $X_i=0$, $X=X_r$ odd	\\ \hline
		\end{tabular}
	\end{table}
	Note that if a real or imaginary part in the table is required to be both even 
	and odd at the same time, it has to be zero.
	
	These properties are summarized below:
	\vskip 0.1in
	\begin{table}[H]
		\centering
		\begin{tabular}{l|l|l} \hline
		  & $x(t)=x_r(t)+\mathrm{i}x_i(t)$	& $X(\mathrm{i}\omega)=X_r(\mathrm{i}\omega)+\mathrm{i}X_i(\mathrm{i}\omega)$	\\ \hline
		1 & real $x(t)=x_r(t)$ 		& even $X_r(\mathrm{i}\omega)$, odd $X_i(\mathrm{i}\omega)$ \\
		2 & real and even $x(-t)=x_r(t)$ 	& real and even $X_r(\mathrm{i}\omega)$ \\
		3 & real and odd $x(-t)=-x_r(t)$ 	& imaginary and odd $X_i(\mathrm{i}\omega)$ \\
		4 & imaginary $x(t)=x_i(t)$  	& odd $X_r(\mathrm{i}\omega)$, even $X_i(\mathrm{i}\omega)$ \\ 
		5 & imaginary and even $x(-t)=x_i(t)$ 	& imaginary and even $X_i(\mathrm{i}\omega)$ \\
		6 & imaginary and odd $x(-t)=-x_i(t)$ 	& real and odd $X_r(\mathrm{i}\omega)$ \\ \hline
		\end{tabular}
	\end{table}
	
	As any signal can be expressed as the sum of its even and odd components, the first three items above indicate that the spectrum of the even part of a real signal is real and even, and the spectrum of the odd part of the signal is  imaginary and odd. 
	
	\item[P8.] Symmetry (or Duality):
	
	
	Or in a more symmetric form:
	
	\begin{dem} As ${\cal F}(x(t))=X(\mathrm{i}\omega)$, we have:
	
	Letting $t'=-t$, we get:
	
	Interchanging $t'$ and $\omega$ we get:
	
	or:
	
	In particular, if the signal is even:
	
	then we have:
	
	\begin{flushright}
		$\blacksquare$  Q.E.D.
	\end{flushright}
	\end{dem}
	For example, the spectrum of an even square wave is a sinc function, and the spectrum of a sinc function is an even square wave. 
	
	\item[P9.] Multiplication theorem:
	
	
	\begin{dem} 
	
	\begin{flushright}
		$\blacksquare$  Q.E.D.
	\end{flushright}
	\end{dem}
	
	\item[P10.] Parseval's equation\index{Parseval's equation} for the Fourier transform (for more details see page \pageref{Parseval theorem}):
	
	In the special case when $y(t)=x(t)$, the above becomes the Parseval's equation:
	
	where:
	
	is the energy density function, commonly named "\NewTerm{power spectrum}\index{power spectrum}\label{power spectrum}", representing how the signal's energy is distributed along the frequency axes. The total energy contained in the signal is obtained by integrating $S(\mathrm{i}\omega)$ over the entire frequency axes.
	
	The Parseval's equation also indicates that the energy or information contained in the signal is reserved, i.e., the signal is represented equivalently in either the time or frequency domain with no energy gained or lost!
	
	The latter relation is more commonly written:
	
	or more simply:
	
	\begin{tcolorbox}[colframe=black,colback=white,sharp corners]
	\textbf{{\Large \ding{45}}Example:}\\\\
	We have already proved in the section of Sequences and Series that the Fourier transform of the square pulse was given by (see page \pageref{fourier transform pulse square}):
	
	Hence:
	
	\end{tcolorbox}
	
	\begin{tcolorbox}[colframe=black,colback=white,sharp corners]
	The energy spectrum of the pulse we have just calculated shows a great similarity with the Fraunhofer diffraction pattern due to a narrow slit (\SeeChapter{see section Wave Optics page \pageref{fraunhofer diffraction}}). In reality, it is more than a similarity because it is possible to prove in physics that any diffraction pattern is the Fourier transform of the object that is the cause!
	\end{tcolorbox}
	
	
	\item[P11.] Correlation:
	
	The "\NewTerm{cross-correlation}\index{cross-correlation}" of two real signals $x(t)$ and $y(t)$ is defined as (notice that it is strictly equivalent to the definition of the continuous convolution seen just earlier above at page \pageref{continuous convolution}):
	
	Specially, when $x(t)=y(t)$, the above becomes the "\NewTerm{auto-correlation}\index{auto-correlation}" of signal $x(t)$:
	
	Assuming ${\cal F}(x(t))=X(\mathrm{i}\omega)$, we have ${\cal F}(x(t-\tau))=X(\mathrm{i}\omega)e^{-\mathrm{i}\omega\tau}$ and according to multiplication theorem, $R_x(\tau)$ can be written as:
	
	i.e.:
	
	that is, the auto-correlation and the energy density function of a signal $x(t)$ are a Fourier transform pair.
	
	\item[P12.] Convolution Theorems:
	
	The "\NewTerm{convolution theorem}\index{convolution theorem}\label{convolution theorem}" states that convolution in time domain corresponds to multiplication in frequency domain and vice versa:
	
	
	Let us start with the proof of ($a$)!
	
	\begin{dem}
	
	\begin{flushright}
		$\blacksquare$  Q.E.D.
	\end{flushright}
	\end{dem}
	And now let us continue with the proof of ($b$)!

	\begin{dem}
	
	
	\begin{flushright}
		$\blacksquare$  Q.E.D.
	\end{flushright}
	\end{dem}
	
	\item[P13.]  Time Derivative\label{fourier transform time derivative}:
	
	\begin{dem} 
	Differentiating the inverse Fourier transform $X(\mathrm{i}\omega)$ with respect to $t$ we get:
	
	Repeating this process we get:
	
	\begin{flushright}
		$\blacksquare$  Q.E.D.
	\end{flushright}
	\end{dem}
	
	\item[P14.] Time Integration\label{fourier transform time integration}:
	
	First consider the Fourier transform of the following two signals:
	
	
	According to the time derivative property above:
	
	we get:
	
	and:
	
	Why do the two different functions have the same transform?
	
	In general, any two function $f(t)$ and $g(t)=f(t)+c^{te}$ with a constant difference $c^{te}$ have the same derivative $\mathrm{d}\;f(t)/\mathrm{d}t$, and therefore they have the same transform according the above method. This problem is obviously caused by the fact that the constant difference $c^{te}$ is lost in the derivative operation.
	
	To recover this constant difference in time domain, a delta function 
	needs to be added in frequency domain. Specifically, as function $\mathrm{sgn}(t)$ does not have DC component, its transform does not contain a delta:
	
	To find the transform of $u(t)$, consider:
	
	and:
	
	The added impulse term $\pi \delta(\omega)$ directly reflects the constant $c=1/2$ in time domain.
	
	Now we show that the Fourier transform of a time integration is:
	
	
	\begin{dem}
	
	First consider the convolution of $x(t)$ and $u(t)$:
	
	Due to the convolution theorem, we have:
	
	\begin{flushright}
		$\blacksquare$  Q.E.D.
	\end{flushright}
	\end{dem}
	
	\item[P15.] Frequency Derivative:
	
	\begin{dem} We differentiate the Fourier transform of $x(t)$ with
	respect to $\omega$ to get:
	
	i.e.:
	
	Multiplying both sides by $\mathrm{i}$, we get:
	
	Repeating this process we get:
	
	\begin{flushright}
		$\blacksquare$  Q.E.D.
	\end{flushright}
	\end{dem}
	
	\end{itemize}
	
	\pagebreak
	\subparagraph{Usual Fourier transforms}\label{usual Fourier transforms}\mbox{}\\\\
	There are in math and physics many Fourier transforms of signals we see quite frequently (but not exclusively). Furthermore, all Fourier transforms proved below will be used in the various chapters on Physics, Engineering, Atomistic, Social Mathematics, etc of this book. So, as in any formula booklet, we propose you the most Fourier transforms but with... the proofs!
	\begin{enumerate}
	
	\item Impulse:
	
	As shown above:
	
	or:
	
	It is therefore immediate that the inverse Fourier transform of the complex exponential is the Dirac delta:
	
	
	\item Unit Step:
	
	As shown above:
	
	
	\item Constant:
	
	As shown above:
	
	This is a useful formula.
	
	\item Complex exponential:
	
	The spectrum of a complex exponential can be found from the above due to the frequency shift property:
	
	Let us write this more explicitly:
	
	Now let us try to solve this using the physicist way... Using what we have just seen before (the inverse Fourier transform of the Dirac pulse), we have:
	
	it comes obviously after rearranging (multiplying both sides by $2\pi$):
	
	
	\item Sinusoids:
	
	
	Therefore:
	
	Similarly, we have:
	
	
	\item Exponential decay (right-sided):
	
	Therefore:
	
	
	\item Exponential decay (left-sided):
	
	Due to the time reversal property, we also have (for $a>0$):
	
	or:
	
	
	\item Exponential decay (two-sided):
	
	As the two-sided exponential decay is the sum of the right and left-sided 
	exponential decays, its spectrum of $x(t)$ is the sum of their spectra due 
	to linearity:
	
	
	\item Comb function:
	
	The comb function is defined as:
	
	Its Fourier series coefficient is:
	
	and its spectrum is:
	
	We see that the spectrum of an impulse train with time interval $T$ is also an impulse train with frequency interval $\omega_0=2\pi/T$. Also, according to the definition of the Fourier transform, we have:
	
	Therefore we have this equation:
	
	which can be compared with the equation in continuous case:
	
	
	
	\item Square wave (for another detailed derivation see page \pageref{fourier transform pulse square}):
	
	A square wave or rectangular function of width $a$ can be considered as the  difference between two unit step functions:
	
	and due to linearity, its Fourier spectrum is the difference between 
	the two corresponding spectra:
	
	
	\item Sinc function:
	
	The spectrum of an ideal low-pass filter is:
	
	and its impulse response can be found by inverse Fourier transform:
	
	
	\item Triangle function:
	
	As $x(t)$ is an even function, its Fourier transform is:
	
	Alternatively, as the triangle function is the convolution of two square functions
	($a=1/2$), its Fourier transform can be more conveniently obtained according to the
	convolution theorem as: 
	
	
	\item Gaussian function:
	
	The Fourier transform of a Gaussian or bell-shaped function $x(t)=e^{-\pi t^2}$ is:
	
	Here we have used the identity:
	
	We see that the Fourier transform of a bell-shaped function is also a bell-shaped function:
	
	Note that the area underneath either $x(t)$ or $X(\mathrm{i}\omega)$ is unity. Moreover, due to the property of time and frequency scaling, we have:
	
	(Note that if $a=1/\sqrt{2\pi \sigma^2}$, then $a\;x(at)$ above is a normal 	distribution with variance $\sigma^2$ and mean $\mu=0$.) If we let $a \rightarrow \infty$, $x(t)$ becomes narrower and taller and  approaches $\delta(t)$, and its spectrum $e^{-\pi (f/a)^2}$ becomes wider and approaches constant $1$. On the other hand, if we rewrite the above as:
	
	and let $a \rightarrow 0$, $x(t)$ approaches $1$ and $X(\mathrm{i}\omega)$ approaches $\delta(\omega)$.
	
	\end{enumerate}
	
	Now that we have seen quite a number of properties and usual Fourier Transform, let us go back to our heat equation\index{heat equation} determined in the section of Thermodynamics (see page \pageref{heat equation}) in the form:
	
	And let us also be interested in the case where:
	
	To simplify the writing, we will write the differential equation in the following form:
	
	Let us take the Fourier transform relatively to $x$ of this equality. Let us recall that for this purpose we have proved in the section of Sequences and Series and just earlier at page \pageref{fourier transform time derivative} that (time derivative property):
	
	Let us put to simplify the notations:
	
	As we take the Fourier transform with respect to $x$, we can take out the partial derivative of the integral of the Fourier transform such as:
	
	Our differential equation is then reduced to:
	
	Thus explicitly:
	
	We see then immediately with the simplified version that a particular solution is:
	
	The constant has to be determined by the initial condition:
	
	Therefore:
	
	Then it comes by doing the inverse Fourier transform in $x$:
	
	Now let's make a small change of notation by putting:
	
	Then we have:
	
	We then fall much more quickly on the same integral that we obtained in the section of Thermodynamics in our study of the heat equation, the finishing work being the same, the reader can refer to it!

	We had also proved in the section Thermodynamics that we obtained:
	
	We then notice that the Fourier transform is not only a tool for analysing a frequency domain signal but also for solving some differential equations more quickly.
	
	But as many times in mathematics, we must be careful using such a tool. They may be some trap and subtilities. Let us see on famous example!

	We know that the set of solutions of the differential equation:
	
	we $y$ is a continuous function from $\mathbb{R} \mapsto \mathbb{C}$ is made of the set of constant functions!
	
	Let us try to solve this equation using the Fourier transform. By taking the Fourier transform on the left and on the right we get:
	
	An error would be to believe that we can divide left and right by $\omega $ and thus get:
	
	In this case, taking the inverse transform would get $ y = 0 $ as the only solution to the equation, then we would lose all other solutions.
	
	In fact, what we must remember is that the Fourier transform is defined on the space of "temperate distributions" and that therefore, as long as we decide to use this integral transform to solve the differential equation  above, we also decide, implicitly, to look for solutions in this space. Now in the space of temperate distributions, the equation $\omega\mathcal{F}(y)=0$ possesses an infinity of solutions that are given by (without proof) $c\cdot \delta$ where $c$ is a complex number and $\delta$ is the Dirac distribution. As a result:
	
	and taking the inverse transform we get indeed:
	
	We therefore conclude that when we solve differential equations with the Fourier transform it must be remembered that in the space of temperate distributions the usual algebraic calculation rules are to be handled with care. This remark is obviously valid only for people who know the Fourier transform but have only a vague idea of what is a temperate distribution (which is normally the case for engineers....
			
	\paragraph{Discrete Time Fourier Transform}\mbox{}\\\\
	A discrete-time signal can be considered as a continuous signal $x(t)$ 
	sampled at a rate $F=1/t_0$ or $\Omega=2\pi/t_0$, where $t_0$ is the 
	sampling period (time interval between two consecutive samples). The
	corresponding sampling function (comb function) is:
	
	The sampling process can be represented by:
	
	where $x[m]=x(mt_0)$ is the value of $x(t)$ at $t=mt_0$. The Fourier transform of this discrete signal (treated as a special case of continuous signal) is:
	
	This is the forward Fourier transform (analysis) of a discrete signal $x_s(t)$. The spectrum $X(\mathrm{i}\omega)$ is periodic with period $\Omega=2\pi F=2\pi/t_0$:
	
	as :
	
	
	To get back the time signal $x[m]$ from its spectrum:
	
	we multiply the equation by $e^{\mathrm{i}\omega nt_0}/\Omega$ and integrate both sides with respect to $\omega$ over the period $\Omega=2\pi F=2\pi/t_0$	to obtain the inverse Fourier transform (synthesis):
	
	Note that here we used:
	
	which can be compared this with:
	
	To summarize, the spectrum of a given discrete signal:
	
	can be found by the "\NewTerm{forward discrete Fourier transform}\index{forward discrete Fourier transform}\index{discrete Fourier transform}" to be:
	
	and the signal can be expressed by inverse Fourier transform:
	
	It is interesting to compare this discrete time Fourier transform pair with the Fourier series expansion (the Fourier transform of a periodic signal): 
	
	
	with discrete spectrum:
	
	We see symmetry between these two different forms of Fourier transform. If the  signal $x(t)=x(t+T)$ is periodic, its spectrum $X(\mathrm{i}\omega)$ is discrete, the coefficients of the Fourier series with interval $\omega_0=2\pi/T$. On the other hand, if $x(t)$ is discrete with interval $t_0=2\pi/\Omega$, its spectrum $X(\mathrm{i}\omega)=X(\mathrm{i}\omega+\Omega)$ is periodic.
	
	In particular, if the unit of time is so chosen that the sampling period is $t_0=1$, then $\Omega=2\pi/t_0=2\pi$, and the forward Fourier transform of a discrete signal becomes:
	
	The inverse transform becomes:
	
	The spectrum $X(\mathrm{i}\omega)=X(\mathrm{i}\omega+2\pi)$ is periodic.
	
	\begin{tcolorbox}[title=Remark,colframe=black,arc=10pt]
	The spectrum of a time signal (continuous or discrete) can be denoted by $X(\mathrm{i}\omega)$ or $X(f)$ to emphasize the fact that the spectrum represents how the energy contained in the signal is distributed as a function of frequency $\omega$ or $f$. Moreover, if $X(f)$ is used, the factor $1/2\pi$ in front of the inverse transform is dropped so that the transform pair takes a more symmetric form. On the other hand, as Fourier transform of discrete signal can be considered as a special case of Z transform when the real part of $s=\sigma+\mathrm{i}\omega$ is zero, i.e., $z=e^s=e^{\mathrm{i}\omega}$:
	
	it is also natural to denote the spectrum of $x[n]$ by $X(e^{\mathrm{i}\omega})$.
	\end{tcolorbox}
	
	
	\paragraph{Properties of Discrete Fourier Transform}\mbox{}\\\\
	As a special case of general Fourier transform, the discrete time transform  shares all properties (and their proofs) of the Fourier transform discussed above, except now some of these properties may take different forms. In the following, we always assume ${\cal F}[x[m]]=X(e^{\mathrm{i}\omega})$ and ${\cal F}[y[m]]=Y(e^{\mathrm{i}\omega})$. 
	
	\begin{enumerate}
	\item[P1.] Linearity:
	
	
	\item[P2.] Time Shifting:
	
	\begin{dem}
	
	If we let $m'=m-m_0$, the above becomes:
	
	\begin{flushright}
		$\blacksquare$  Q.E.D.
	\end{flushright}
	\end{dem}
	
	
	\item[P3.] Time Reversal:
	
	
	
	\item[P4.] Frequency Shifting:
	
	
	\item[P5.] Differencing:
	
	Differencing is the discrete-time counterpart of differentiation.
	
	\begin{dem}
	
	\begin{flushright}
		$\blacksquare$  Q.E.D.
	\end{flushright}
	\end{dem}
	
	
	\item[P6.] Differentiation in frequency:
	
	
	\begin{dem}
	Differentiating the definition of discrete Fourier transform with respect to 	$\omega$, we get:
	
	\begin{flushright}
		$\blacksquare$  Q.E.D.
	\end{flushright}
	\end{dem}
	
	\item[P7.] Convolution Theorems:
	
	The convolution theorem states that convolution in time domain corresponds to multiplication in frequency domain and vice versa:
	
	
	Recall that the convolution of periodic signals $x_T(t)$ and $y_T(t)$ is:
	
	Here the convolution of periodic spectra $X(f)$ and $Y(f)$ is similarly defined as:
	
	
	Proof of ($a$): 
	
	
	Proof of ($b$):
	
	\begin{flushright}
		$\blacksquare$  Q.E.D.
	\end{flushright}
	
	\item[P8.] Parseval's relation\index{Parseval's relation} for the discrete Fourier transform:
	
	\end{enumerate}
	
	\begin{tcolorbox}[colframe=black,colback=white,sharp corners]
	\textbf{{\Large \ding{45}}Examples:}\\\\
	E1. The discrete Fourier transform of the Dirac delta:
	
	E2. The spectrum of:
	
	is:
	
	If the signal is two-sided:
	
	Due to the time reversal property, its spectrum is:
	
	E3. Consider a LTI system (Linear Time-Invariant) with impulse response:
	
	and input:
	
	The output $y[n]$ can be found in either time domain by convolution or in frequency domain by multiplication. In time domain, we have:
	
	\end{tcolorbox}
	
	\begin{tcolorbox}[colframe=black,colback=white,sharp corners]
	When $a=b$, we have:
	
	In frequency domain, we first find the spectra of both $x[n]$ and $h[n]$ to be:
	
	and the spectrum of the output is:
	
	To find $y(n)$ in time domain by inverse transform of $Y(e^{\mathrm{i}\omega})$, we use partial fraction expansion to rewrite the above as:
	
	By equating the coefficients of $e^{-\mathrm{i}\omega}$ and the constants, we get:
	
	which can be solved to get:
	
	In this form, $Y(\mathrm{i}\omega)$ can be easily inverse transformed to yield:
	
	same as the result from convolution. Again when $a=b$, we have:
	
	But since:
	
	by the frequency differentiation property, we have:
	
	and the output in time domain is obtained as:
	\end{tcolorbox}
	
	\begin{tcolorbox}[colframe=black,colback=white,sharp corners]
	
	Note that the time-shifting property is used due to the factor $e^{\mathrm{i}\omega}$. Also note that $u[n+1]$ (starting at $n=-1$) is replaced by $u[n]$ (starting at $n=0$) as $n+1=0$ when $n=-1$.\\
	
	E4. The impulse response of a discrete LTI system is:
	
	where $|a|<1$ so that the system is stable. The output $y[m]$ of the system with an input:
	
	can be found in three different ways.
	\begin{itemize}
		\item Time domain convolution: 
		The output is the convolution of $x[m]$ and $h[m]$:
		
		
		\item The eigenequation method:\\ 
	
		We first get the frequency response function from $h[m]$:
		
		which is the eigenvalue of the system when the input is a complex exponential $e^{\mathrm{i}n\omega}$. Now the system's response to: 
		
		can be found to be:
		
	\end{itemize}
	
	\end{tcolorbox}
	
	\begin{tcolorbox}[colframe=black,colback=white,sharp corners]
	\begin{itemize}
		\item Frequency domain multiplication:
		
		If we find the spectra of both $h[m]$ and $x[m]$ in the frequency domain, the spectrum of $y[m]$ can be found by multiplication. We already know:
		
		We next find the spectrum of $x[m]$:
		
		Now the spectrum of the output $y[m]$ can be found:
		
		and the output $y[m]$ is obtained by inverse Fourier transform:
		\begin{eqnarray}
		y[m] &=& \frac{1}{2\pi} \int\limits_0^{2\pi} \left[\frac{\pi}{1-ae^{-\mathrm{i} \omega}}
		 \sum_{k=-\infty}^{+\infty} \left[\delta(\omega-2k\pi-\frac{2\pi}{N})+\delta(\omega-2k\pi-\frac{2\pi}{N})\right]\right]
			e^{\mathrm{i}m\omega} \mathrm{d}\omega
			\nonumber \\
		 &=& \frac{1}{2}e^{\mathrm{i}2\pi m/N}\frac{1}{1-ae^{-\mathrm{i}2\pi /N}}
			+\frac{1}{2}e^{-\mathrm{i}2\pi m/N}\frac{1}{1-ae^{\mathrm{i}2\pi /N}}	
			\nonumber
		\end{eqnarray}
	\end{itemize}
	
	The physical meaning of this result will be clear if we write $H(2\pi/N)$ in polar form:
	
	and the output becomes:
	
	That is, the output of the system is also a sinusoidal signal of the same 
	frequency as the input, but with different magnitude $r$ and a phase angle
	$\theta$. For example, if $N=4$, we have:
	
	and the output is:
	
	\end{tcolorbox}
	
	
	
	\StickyNote[2.5cm]{\LARGE To finish depending on donations}[6.5cm]
	
	\pagebreak
	\subsubsection{Laplace Transform}\label{Laplace transform}
	The Laplace transform (LT), as we have already mention it, is a generalization of the Fourier transform (TF), however, although it is so called in his honour because he used it in his work on the theory of probability, seems to have been originally discovered by Leonhard Euler. The Laplace transform also appears in all branches of mathematical physics (mechanical engineering, electronics, quantitative finance, etc.) and is used extensively in order to solve  differential equations that arise in many modelling situations of real life.
	
	As for the Fourier transform, the Laplace transformation allows us to get rid of differentiations. The transform can do this because it has the wonderful property of converting the operation of differentiation into the far simpler one of multiplication. That is, it transforms a differential equation into an algebraic equation. This process is analogous to how logarithms transform multiplication into the simpler operation of
addition (because $\log (xy)=\log(x)+\log(y)$).

	A power series about an origin is any series that can be written in the following form:
	
	where $a_n$ are  numbers and $n$ is a non-negative integer. One can think of $a_n = a(n)$ as a function of $n$ for each non-negative integer $n = 0, 1, 2, \ldots$. In order to give birth to Laplace transformation technique, we  make some associations. The discrete variable $n$ is converted into a real variable $t$. The coefficient term $a_n$ is written as $f(t)$. The term $x^n$ can equivalently be written as $e^{(\ln (x^t))}$. Finally, summation notation can be replaced by its continuous analogue, that is, integration. By doing so, we have following:
	 
For convergence, it is obviously important to have following condition for the above integral (yes think for this about the original sum!):
	 
	Therefore:
	 
	Since $0<x<1 $ so it implies that:
	
	Thus $\ln(x)$ has to be negative for the integral to converge, in this regard, we suppose $\ln(x)=-s$ where $s>0$. Thus, the final integral takes the form:
	
	In this way, we can say that Laplace Transform is simply stretching a discrete (infinite series)  into a continuous (integration) analogue. 
	
	Let us recall that if $s=\mathrm{i}\omega$ (the real part of $s$ is purely imaginary), Laplace transform becomes Fourier transform! In general, any continuous time signal $x(t)$ can be Laplace transformed to get:
	
	provided the integral converges, i.e., the function $X(s)$ exists. This general form of "\NewTerm{bilateral Laplace transform}\index{bilateral Laplace transform}\index{Laplace transform}\label{bilateral Laplace transform}" is related to the Fourier transform:
	
	i.e., Laplace transform of a generic function is Fourier transform of the same function multiplied by $e^{-\sigma t}$. This exponential factor has the effect of forcing the signals to converge. This is why the Laplace transform can be applied to a broader class of signals than the Fourier transform, including exponentially growing signals shown in the following two cases:

	\begin{itemize}
		\item Right sided:
		
		The Fourier transform does not exist as the signal grows exponentially when $t\rightarrow +\infty$, i.e., the transform integral does not converge (not integrable). However, its Laplace transform exists if $\Re[s]=\sigma>a$ (i.e., $\sigma-a>0$), as the modified signal $x(t)e^{-\sigma t}=e^{-(\sigma-a)t}u(t)$ will converge.
		
		\item Left sided:
		 
		Again the Fourier transform does not exist as the signal grows exponentially when $t\rightarrow -\infty$, i.e., the transform integral does not converge. But if $\Re[s]=\sigma<a$ (i.e., $\sigma-a<0$), the modified signal $x(t)e^{-\sigma t}=e^{-(\sigma-a)t}u(-t)$ will converge and its Laplace transform exists.
	\end{itemize}

	In Fourier transform, both the signal $x(t)$ in time domain and its spectrum $X(\mathrm{i}\omega)$ in frequency domain are a one-dimensional (1D) complex function. However, the Laplace transform $X(s)$ of the 1D signal $x(t)$ is a complex function defined over a two-dimensional {\em complex plane}, called the s-plane,	spanned by the two variables $\sigma$ (for the horizontal real axis) and $\omega$ (for the vertical imaginary axis). 
	
	In particular, if this 2D function $X(s)=X(\sigma+\mathrm{i}\omega)$ is evaluated along the imaginary axis $\Re[s]=\sigma=0$, it becomes a 1D function $X(\mathrm{i}\omega)$, the	Fourier transform of $x(t)$. Graphically, the Fourier transform, the spectrum of the signal, can be found as the cross section of the 2D function $X(s)=X(\sigma+\mathrm{i}\omega)$ along the line $\Re[s]=\sigma=0$.
	
	Before we go further another way to introduce the Laplace may be useful and help the reader. But if you have understand the above introduction you can go over this alternate presentation.
	
	The Fourier transform works quite well only when the argument of the integral does not diverge, where the kernel (the multiplication by $e^{-\mathrm{i}\omega t}$) is purely complex, and its integral sweeps the set of reals (bilateral infinity $]-\infty,+\infty[$).

	Typically, mathematicians, that like to generalize stuff...., have generalized the Fourier transform to functions kernel that are not purely imaginary and whose integral could be one-sided $[0,+\infty[$. Thus, the Laplace transform converges for a larger set of functions than the Fourier transform.
	
	Indeed, let us see a simple example with a diverging function:
	
	This last integral does not converge (at least as far as I know ...) whatever the value of $\alpha$ different from zero!
	
	Taking a generalized version of the Fourier transform and denoting it (we already know that it is the bilateral Laplace transform):
	
	with (we know that already!):
	
	where in general, the convergence of the integral is not guaranteed for all $s$. We then name the "\NewTerm{abscissa of absolute convergence}" of the Laplace transform the set of values that $\sigma$ must take for the integral to converge.
	
	Let us take again the non convergent Fourier transform seen just above but where the function is defined only for the positive times only (unilateral Laplace transform):
	
	The result converges if $\omega\geq 0$ (which is always the case in physics) for $t>0$ and if and only if $\sigma>1$. Then in this case we have:
	
	
	\paragraph{Inverse Laplace Transform}\mbox{}\\\\
	The inverse Laplace transform can be obtained from the corresponding Fourier	transform:
	
	The inverse Fourier transform of the above is:
	
	Multiplying both sides by $e^{\sigma t}$, we get the inverse Laplace transform:
	
	As we want to represent the inverse transform in terms of $s$ (instead of $\omega$), we realize that:
	
	We also realize that the integral in the inverse transform is along a vertical line in the $s$-plane from $\mathrm{i}\omega=-\mathrm{i}\infty$ to $\mathrm{i}\omega=+\mathrm{i}\infty$ while the other variable $\sigma$ is fixed. The "\NewTerm{inverse Laplace transform}\index{inverse Laplace transform}" above can therefore be written as:
	
	Now we have the Laplace transform pair:
	
	
	The forward (bilateral) and inverse Laplace transform pair can also be represented as:
	
	
	\pagebreak
	\paragraph{Region of Convergence}\mbox{}\\\\
	An essential issue of Laplace transform of $x(t)$ is whether the transform $X(s)$ even exists, and under what condition it exists. To see this, consider the following examples.
	
	\begin{tcolorbox}[colframe=black,colback=white,sharp corners]
	\textbf{{\Large \ding{45}}Examples:}\\\\
	The Fourier transform of a signal $x(t)=e^{-at}u(t)$ is:
	
	This integral does not converge unless $a>0$. In other words, only when the signal $x(t)$ decays (instead of grows) exponentially, will its Fourier transform  $X(\mathrm{i}\omega)$ exist:
	
	
	Now consider Laplace transform of the same signal:
	
	Similar to the Fourier transform, for this integral to converge, i.e., for Laplace transform $X(s)$ to exist, it is necessary for $\sigma=\Re[s]$ to satisfy:
	
	in which case the Laplace transform is:
	
	As a special case where $a=0$, $x(t)=u(t)$ and we have:
	
	When $ \Re[s]=\sigma=0$, Laplace transform $X(s)$ becomes Fourier transform $X(\mathrm{i}\omega)$. \\
	
	E2. The non-causal version of the signal above is $x(t)=-e^{-at}u(-t)$ and its Laplace transform is:
	
	\end{tcolorbox}
	
	\begin{tcolorbox}[colframe=black,colback=white,sharp corners]
	Only when:
	
	will this integral converge and Laplace transform $X(s)$ exists
	
	Again as a special case when $a=0$, $x(t)=-u(-t)$ we have
	
	Comparing the two examples above we see that two different signals can have identical Laplace transform $X(s)$ but $s$ may have to satisfy different conditions for $X(s)$ to exist. In general, the set of all $s$ values satisfying the conditions for the integral of Laplace transform to converge is called the "\NewTerm{region of convergence}\index{region of convergence}" (ROC) in the complex s-plane. In the first case above, the ROC is $\Re[s]>0$, and in the second case, the ROC is $\Re[s]<0$.\\
	
	E3. Consider the signal:
	
	The Laplace transform of this signal is:
	
	Following example 1, we see that the conditions for the three integrals to converge are, respectively:
	
	i.e., when $\Re[s]>-1$, $X(s)$ exists and can be written as:
	
	\end{tcolorbox}
	
	\begin{tcolorbox}[colframe=black,colback=white,sharp corners]
	E4. Let us consider:
	
	As the Laplace integration converges independent of $s$, the ROC is the entire $s$-plane. In particular, when $T=0$, we have:
	
	\end{tcolorbox}
	\paragraph{Zeros and Poles of Laplace Transform}\mbox{}\\\\
	All Laplace transforms in the above examples are rational, i.e., they can be written as a ratio of polynomials of variable $s$ in the general form:
	
	where $N(s)$ is the numerator polynomial of order $M$ with roots $s_{z_k}, (k=1,2, \cdots, M)$, and $D(s)$ is the denominator polynomial of order $N$ with roots $s_{p_k}, (k=1,2, \cdots, N)$. In general, we assume the order of the numerator polynomial is lower than that of the denominator polynomial, i.e.,  $M < N$. If this is not the case, we can always expand $X(s)$ into multiple terms so that $M<N$ is true for each of terms.	
	\begin{tcolorbox}[colframe=black,colback=white,sharp corners]
	\textbf{{\Large \ding{45}}Example:}\\\\
	Consider:
	
	As the order of the numerator $M=2$ is higher than that of the denominator $N=1$, 
	we expand it into the following terms
	
	and get
	
	Equating the coefficients for terms $s^k$ $(k=0, 1, \cdots, M)$ on both sides, we get
	
	Solving this equation system, we get coefficients
	
	and
	
	Alternatively, the same result can be obtained by carrying out a long-hand division
	
	\end{tcolorbox}
	
	\textbf{Definitions (\#\mydef):}
	\begin{enumerate}
		\item[D1.] Any complex value $s_z$ of $s$ for which $H(s)|_{s=s_z}=H(s_z)=0$ is a "\NewTerm{zero}" of $H(s)$.

		\item[D2.]  Any complex value $s_p$ of $s$ for which $H(s)|_{s=s_p}=H(s_p)=\infty$ is a "\NewTerm{pole}\index{pole}" of $H(s)$.
	\end{enumerate}
	Obviously, all roots of the numerator polynomial $N(s)$ are zeros of $H(s)$ and all roots of the denominator polynomial $D(s)$ are poles of $H(s)$. Moreover, if the order of $D(s)$ exceeds the order of $N(s)$ (i.e., $N>M$), then $H(\infty)=0$, i.e., there is a zero at infinity. On the other hand, if the order of $N(s)$ exceeds that of $D(s)$ (i.e., $M>N$), then $H(\infty)=\infty$, i.e, there is a pole at infinity. On the $s$-plane zeros and poles can be indicated by $o$ and $x$ respectively. Most essential behaviour properties of an LTI system can be obtained graphically from the ROC and the zeros and poles of its transfer function $H(s)$ on the $s$-plane.
	
	\paragraph{Properties of region of convergence}\mbox{}\\\\
	Whether the Laplace transform $X(s)$ of a function $x(t)$ exists depends on whether or not the transform integral converges:
	
	which in turn depends on the duration and magnitude of $x(t)$ as well as the real part of $s$ $\Re[s]=\sigma$. When $x(t)$ is right sided (i.e., $x(t)=0$ for $t<t_0$), it may have infinite duration for $t>0$, and the larger $\sigma$ the more quickly $x(t)e^{-\sigma t}$ decays as $t \rightarrow +\infty$. On the other hand, if $x(t)$ is left sided (i.e., $x(t)$ for $t<t_0$), it may have infinite duration for $t<0$, and the smaller $\sigma$ the more quickly $x(t)e^{-\sigma t}$ decays as $t \rightarrow -\infty$. The imaginary part of $s$ $Im[s]=\mathrm{i}\omega$ determines the frequency of a sinusoid which is bounded and has no effect on the convergence of the integral. Based on these observations, we can get the following properties for the ROC:	
	\begin{itemize}
	
	\item If $x(t)$ is absolutely integrable and of finite duration, then the ROC is the entire $s$-plane (the Laplace transform integral is finite, i.e., $X(s)$ exists, for any $s$).
	
	\item The ROC of $X(s)$ consists of strips parallel to the $\mathrm{i}\omega$-axis in the $s$-plane.
	
	\item If $x(t)$ is right sided and $\Re[s]=\sigma_0$ is in the ROC, then any $s$ to	the right of $\sigma_0$ (i.e., $\Re[s]>\sigma_0$) is also in the ROC, i.e., ROC is a right sided half plane.
	
	\item If $x(t)$ is left sided and $\Re[s]=\sigma_0$ is in the ROC, then any $s$ to the left of $\sigma_0$ (i.e., $\Re[s]<\sigma_0$) is also in the ROC, i.e., ROC is a left sided half plane.
	
	\item If $x(t)$ is two-sided, then the ROC is the intersection of the two one-sided ROCs corresponding to the two one-sided parts of $x(t)$. This intersection can be either a vertical strip or an empty set.
	
	\item If $X(s)$ is rational, then its ROC does not contain any poles (by definition $X(s)|_{s=s_p}=\infty$ dose not exist). The ROC is bounded by the poles or extends to infinity.
	
	\item If $X(s)$ is a rational Laplace transform of a right sided function $x(t)$, then the ROC is the half plane to the right of the rightmost pole, and if $X(s)$ is a rational Laplace transform of a left sided function $x(t)$, then the ROC is the half plane to the left of the leftmost pole.
	
	\item A signal $x(t)$ is absolutely integrable, i.e., its Fourier transform $X(\mathrm{i}\omega)$ exists, if and only if the ROC of the corresponding Laplace transform $X(s)$ contains the imaginary axis $\Re[s]=0$ or $s=\mathrm{i}\omega$.
	
	\end{itemize}
	
	\begin{tcolorbox}[colframe=black,colback=white,sharp corners]
	\textbf{{\Large \ding{45}}Examples:}\\\\
	E1. Consider the Laplace transform of a two-sided signal:
		$x(t)=e^{-b|t|}$:
	
	\end{tcolorbox}
	\begin{tcolorbox}[colframe=black,colback=white,sharp corners]
	The Laplace transform of the two components can be obtained from the two examples discussed earlier above. Then we get:
	
	and let $b=-a$,  we then have:
	
	Combining the two components, we have:
	
	Whether $X(s)$ exists or not depends on $b$. If $b>0$, i.e., $x(t)$ decays exponentially as $|t| \rightarrow +\infty$, then the ROC is the strip between $-b$ and $b$ and $X(s)$ exists. But if $b<0$, i.e., $x(t)$ grows exponentially as  $|t| \rightarrow +\infty$, then the ROC is an empty set and $X(s)$ does not exist.\\
	
	E2. Given the following Laplace transform, find the corresponding signal:
	
	Given the two poles $s_{p_1}=-1$ and $s_{p_2}$ of the expression, there are three associated ROCs: 
	\begin{itemize}
		\item The half plane to the right of the rightmost pole $s_{p_2}=-1$, with the corresponding right sided time function:
		
		\item The half plane to the left of the leftmost pole $s_{p_1}=-2$, with the corresponding left sided time function:
		
		\item The vertical strip between the two poles $-1 < \Re[s] < -2$, with the corresponding two sided time function:
		 
	\end{itemize}
	In particular, note that only the first ROC includes the $\mathrm{i}\omega$-axis and the corresponding time function has a Fourier transform. Fourier transform of the other two functions do not exist.
	\end{tcolorbox}
	
	\pagebreak
	\paragraph{Properties of Laplace Transform}\label{properties of Laplace Transform}\mbox{}\\\\ 
	The Laplace transform has a set of properties in parallel with that of the Fourier transform. The difference is that we need to pay special attention to the ROCs. In the following, we always assume:
	
	and:
	
	
	\begin{enumerate}
		\item[P1.] Linearity:
		
		While it is obvious that the ROC of the linear combination of $x(t)$ and $y(t)$ should be the intersection of the their individual ROCs $R_x \cap R_y$ in which both	$X(s)$ and $Y(s)$ exist, we note that in some cases when zero-pole cancellation occurs, the ROC of the linear combination could be larger than $R_x \cap R_y$, as  shown in the example below.
		
		\begin{tcolorbox}[colframe=black,colback=white,sharp corners]
		\textbf{{\Large \ding{45}}Example:}\\\\
		Assume:
		
		and:
		
		then:
		
		\end{tcolorbox}
		
		\item[P2.] Time Shifting:
		
		
		\item[P3.] Shifting in $s$-Domain:
		
		Note that the ROC is shifted by $s_0$, i.e., it is shifted vertically by $Im[s_0]$ (with no effect to ROC) and horizontally by $\Re[s_0]$. If $R_x$ is $\Re[s]>0$, the new ROC is $\Re[s+s_0]>0$, i.e., $\Re[s]>-\Re[s_0]$.
		
		\item[P4.] Time Scaling:
		
		Note that the ROC is horizontally scaled by $1/a$, which could be either positive ($a>0$) or negative ($a<0$) in which case both the signal $x(t)$ and the ROC of its Laplace transform are horizontally flipped. 
		
		\item[P5.] Conjugation:
		
		\begin{dem} 
		
		\begin{flushright}
			$\blacksquare$  Q.E.D.
		\end{flushright}
		\end{dem}
		
		\item[P6.] Convolution:
		
		Note that the ROC of the convolution could be larger than the intersection of $R_x$ and $R_y$, due to the possible pole-zero cancellation caused by the convolution, similar to the linearity property.
		
		\begin{tcolorbox}[colframe=black,colback=white,sharp corners]
		\textbf{{\Large \ding{45}}Example:}\\\\
		Assume:
		
		then:
		
		\end{tcolorbox}
		
		\item[P7.] Differentiation in Time Domain:
		
		This can be proven by differentiating the inverse Laplace transform:
		
		Again, multiplying $X(s)$ by $s$ may cause pole-zero cancellation and therefore the resulting ROC may be larger than $R_x$. For example, when $x(t)=u(t)$ and $X(s)=1/s$ with $\Re[s]>0$ as the ROC, $\mathrm{d} x(t)/\mathrm{d}t=\delta(t)$ and with $sX(s)=1$ whose ROC is the entire $s$-plane. In general, we have:
		
		
		\begin{tcolorbox}[colframe=black,colback=white,sharp corners]
		\textbf{{\Large \ding{45}}Example:}\\\\
		The ROC of ${\cal L}[\delta(t)]=1$ is the entire s-plane, and we have:
		
		and more generally:
		
		\end{tcolorbox}
		
		\item[P8.] Differentiation in $s$-Domain:
		
		This can be proven by differentiating the Laplace transform:
		
		Repeat this process we get:
		
		
		\item[P9.] Integration in Time Domain:
		
		This can be proven by realizing that:
		
		and therefore by convolution property we have:
		
		Also note that as the ROC of ${\cal L}[u(t)]=1/s$ is the right half plane $\Re[s]>0$, the ROC of $X(s)/s$ is the intersection of the two individual ROCs $R_x \cap \{\Re[s]>0\}$, except if pole-zero cancellation occurs (when $x(t)=\mathrm{d}\delta(t)/\mathrm{d}t$ with $X(s)=s$) in which case the ROC is the entire $s$-plane.
	\end{enumerate}
	
	\paragraph{Usual Laplace transforms}\mbox{}\\\\
	As we always do in this book, let us see some common usual Laplace Transforms that are useful in some physics and finance well known problems (and also used as example in the MATLAB™ companion book):
	\begin{enumerate}
		\item $\delta(t)$, $\delta(t-\tau)$
		
		
		Moreover, due to time shifting property, we have:
		
		
		\item $u(t)$ (Heaviside function\index{Heaviside function}), $t\;u(t)$, $t^n\;u(t)$
		
		Due to the property of time domain integration, we have:
		
		Let us have another more explicit approach as the previous one may be a bit hard.... The function $u(t)$ (for recall it is the Heaviside function defined as $H(t)=0,\forall t<0$) for the usage of the unilateral Laplace transform as it non-null only for $t\geq 0$. Then we have:
		
		This integral converse only if $\sigma \geq 0$ Hence:
		
		
		Applying the $s$-domain differentiation property to the above, we have:
		
		Or more explicitly (it sometimes good to see different ways...!) using integration by part:
		
		By the first term we also see that we must have $\sigma>0$. Which gives us:
		
		
		And in general (useful especially in Mechanical Engineering even if so far we were able to avoid its usage):
		
		Indeed, using integration by parts:
		
		From the first term we also see that we must have $\sigma >0$. Hence it remains:
		
		In then comes under the condition $\sigma>0$:
		
		
		\item $e^{-at}u(t)$, $te^{-at}u(t)$
		
		Applying the $s$-domain shifting property to:
		
		we have:
		
		If you don't like using the shift property for that proof, here is a more explicit and long way to get the same result (we change the notation a bit to avoid any confusion):
		
		We see that the integral will converge if and only if:
		
		Therefore:
		

		Applying the same property to:
		
		we have:
		
		
		\item $e^{-\mathrm{i}\omega_0 t}u(t)$, $\sin(\omega_0 t)u(t)$, $\cos(\omega_0 t)u(t)$ 
		
		We could make it by integration by parts but it's quite long. The tick is letting $a=\pm \mathrm{i}\omega_0$ in:
		
		we get:
		
		and therefore:
		
		and:
		
		
		\item $t\;\cos(\omega_0 t)u(t)$, $t\;\sin(\omega_0 t)u(t)$ 
		
		Letting $a=\pm \mathrm{i}\omega_0$ in:
		
		we get:
		
		Furthermore we have:
		
		and:
		
		
		\item $e^{-at}\cos(\omega_0 t) u(t)$,  $e^{-at}\sin(\omega_0 t) u(t)$  
		
		Applying $s$-domain shifting property to:
		
		and:
		
		we get, respectively:
		
		and:
		
		
		\item \label{Laplace pair for finance and telegrapher equation}Let us see an important transform pair that we will find invaluable when we get to transients in transmission lines and in advanced quantitative finance\footnote{ Notice that the entire development that will follow is inspired from the excellent book: \textit{Transients for Electrical Engineers} \pageref{nahin2018transients}.}. Specifically, if we define the "\NewTerm{Gauss error function}\index{Gauss error function}\index{error function}":
		
		then we will derive here the pair of unilateral Laplace transform:
		
		\begin{dem}
		We will start by computing the Laplace transform of:
		
		How do we know we have to start from this? Because this function appears in the book \textit{Analytical Theory of Heat} (1822) of Joseph Fourier where he already did the job before us but in a different painful way\footnote{You can found it in the English translation of 1878 \cite{fourier2003analytical} at the page 384 for $\mathbb{R}^3$.}!
		
		In any case what we have is:
		
		Next, let us make the change of variable:
		
		and so:
		
		Therefore:
		
		With this our unilateral Laplace transforms becomes:
		
		Or:
		
		where:
		
		Now, concentrate on the integral, alone. Multiplying out the exponent
	of the integrand, we have:
		
		where:
		
		The integral $I(b)$, despite its perhaps complicated appearance, can be evaluated as follows.
		
		Differentiating  with respect to $b$ we get:
		
		Now, make the change of variable:
		
		which leads to:
		
		and so:
		
		Thus:
		
		which the primitive gives:
		
		where $c^{te}$ is some constant. We can determine it easily with (\SeeChapter{see section Statistics page \pageref{Gauss integral}}):
		
		the reader may have indeed recognized the Gauss integral here.
		
		So:
		
		And putting into:
		
		We get:
		
		And putting also back into:
		
		Which give us indeed the pair\label{laplace pair for telegrapher equation} (that will be useful to us for the study of the Telegrapher equation):
		
		Which is equivalent:
		
		Finally, recall that we have proved during our studies of main Laplace transform properties the (\SeeChapter{see section Analysis page \pageref{properties of Laplace Transform}}) following pair (integration in time domain):
		
		Therefore:
		
		In the integral above let us make the following change of variable:
		
		Therefore:
		
		and so:
		
		In the first integral in the square brackets we recognize the half on the Gauss integral. Therefore:
		
		In the second one we recognize the Gauss error function (see page \pageref{error function}), therefore:
		
		Therefore our pair:
		
		can be rewritten:
		
		and so, at last, we have the claimed pair of the beginning!
		\begin{flushright}
			$\blacksquare$  Q.E.D.
		\end{flushright}
		\end{dem}
	\end{enumerate}
	
	\paragraph{Solving differential systems}\mbox{}\\\\
	 Let us see well known simple academic companion example of the application of the Laplace transform to deal with differential equations.
	 
	 Let us consider that a voltage $x(t)$ is applied as the input to a resistor $R$, a capacitor $C$ and an inductor $L$ connected in series (the case without the use of the Laplace transform is introduced in details in the section of Electrical Engineering page \pageref{rlc circuit}). The output $y(t)$ is the voltage across one of the three elements. The system can be described by a differential equation in time domain:
	
	or an algebraic equation in $s$-domain:
	
	and the overall impedance of the circuit is defined as the ratio between voltage $V(s)$ and the current $I(s)$:
	
	which is composed of the individual impedance of the three elements
	

	\begin{table}[H]
		\centering
		\begin{tabular}{l|c|c|c} \hline
			& resistor $R$ & capacitor $C$ & inductor $L$  \\ \hline
		time domain & $i=\frac{v}{R}$ & $i=\frac{1}{C}\frac{\mathrm{d}v}{\mathrm{d}t}$ & $v=\frac{1}{L}\frac{\mathrm{d}i}{\mathrm{d}t}$ 
		\\ \hline 
		$s$-domain    & $V_R=IR$ & $V_C=I/Cs$ & $V_L=IsL$	\\ \hline
		impedance $Z=V/I$   &    $R$   &   $1/sC$   &   $sL$    \\ \hline
		\end{tabular}
	\end{table}	
	If the output is the voltage across one of the three elements ($V_L$, $V_R$, or $V_C$), the transfer function $H(s)$ can be easily obtained by treating the series circuit as a voltage divider: 
	\begin{itemize}
		\item Output is voltage across the capacitor $v_C(t)$
		
		\item Output is voltage across the resistor $v_R(t)$
		
		\item Output is voltage across the inductor $v_L(t)$
		
	\end{itemize}
	If we define
	
	the common denominator of the transfer functions can be written in standard (canonical) form:
	
	with two roots
	
	and the transfer functions above can be written in standard forms:
	\begin{itemize}
		\item 
		
		with two poles $p_1, p_2$ and no zeros. 
		\item 
		
		with two poles $p_1, p_2$ and one zero at the origin. 
		\item 
		
		with two poles $p_1, p_2$ and two repeated zeros at the origin. 
	\end{itemize}
	The three transfer functions behave like low-pass, band-pass and high-pass filter, respectively. Moreover, when the common real part $-\zeta \omega_n$ of the two complex conjugate poles is small (i.e., $0<\zeta < 0.5$), there will be a narrow pass-band around $\omega=\omega_n$ for all three transfer functions. The magnitude and phase angle of the corresponding frequency response function $|H(\mathrm{i}\omega)|$ can be 
	qualitatively determined in the s-plane, as to be discussed later.
	
	\paragraph{Unilateral Laplace Transform}\mbox{}\\\\
	The Laplace transform so far discussed is the "bilateral Laplace transform" as it can be applied to left sided signals as well as right sided ones. Now we will consider the "\NewTerm{unilateral Laplace transform}\index{unilateral Laplace transform}\index{Laplace transform}" of an arbitrary signal $x(t)$ defined as:
	
	where for recall $\cal U$ and $u(t)$ are another typical notation of the Heaviside function.
	
	This definition always assumes $x(t)=0$ for $t<0$. When the unilateral Laplace transform is applied to find the transfer function $H(s)={\cal UL}[h(t)]$ of an LTI system, it is always assumed to be causal. And the ROC is always right sided in $s$-plane.
	
	By definition, the unilateral Laplace transform of any signal $x(t)=x(t)u(t)$ is identical to its bilateral Laplace transform. However, when $x(t) \ne x(t)u(t)$, the two Laplace transforms are different. 
	
	\label{properties of unilateral Laplace Transform}Unilateral Laplace Transform shares all the properties of bilateral Laplace transform, except some of the properties are expressed in different forms. Here we only consider the differentiation in time domain:
	
	\begin{dem}
	
	\begin{flushright}
		$\blacksquare$  Q.E.D.
	\end{flushright}
	\end{dem}
	We can further get Laplace transform of higher order derivatives
	
	and in general:
	

	\StickyNote[2.5cm]{\LARGE To finish depending on donations}[6.5cm]
	
	\pagebreak
	\subsubsection{$\mathcal{Z}$-Transform}
	Laplace transforms (whose Fourier transform is for recall a special case) is applicable only for functions say ... "continuous" to make it simple.... But since the advent of computing the vast majority of functions (signals) are samples at a time interval $T_s$ such that the functions are actually discrete.

	From then on it becomes useful to define a transformation similar to the Laplace transform but in the discrete case! Thus, the Laplace transform becomes a special case of the $\mathcal{Z}$-transform when the sampling period $T_s$ tends to $0$.

	The discrete Fourier transform of a discrete signal $x[n]$ is defined for recall as:
	
	provided $x[n]$ is absolutely summable:
	
	Obviously some signals may not satisfy this condition and their Fourier transform do not exist. To overcome this difficulty, we can multiply the given $x[n]$ by an exponential function $e^{-\sigma n}$ so that $x[n]$ may be forced to be summable for certain values of the real parameter $\sigma$. Now the discrete time Fourier transform becomes:
	
	The result of this summation is a function of a complex variable defined as:
	
	This is the "\NewTerm{forward (bilateral) $\mathcal{Z}$-transform}\index{forward $\mathcal{Z}$-transform}" of the discrete signal $x[n]$:
	
	
	There is another way to introduce the $\mathcal{Z}$-transform that result to the same result but with different notation and that is more explicit. As it may help the reader to better understand, let us show that but by limiting ourselves to positive times only (the case of interest in engineering is the unilateral $\mathcal{Z}$-transform!).

So we know that a continuous function $f(t)t$ can be discretized (sampled) in a way to take at each time step $nT_s$ with $n\in \mathbb{N}$ the value of the function of interest, multiplied by a Dirac pulse on this same point such that:
	
	Now let us calculate the Laplace transform of this discretized function at the limit and using the fact that the $f(nT_e)$ are henceforth independent from the time and the linearity property of the Laplace transform:
	
	As we have shown above in our study of the usual Laplace transforms, we have:
	
	Then we have:
	
	That is usage as we have just seen before to write as following and define as the "\NewTerm{unilateral $\mathcal{Z}$-transform}" (on which we will come back later below):
	
	with:
	
	That's it for the second approach!
	
	Now, given the $\mathcal{Z}$-transform $X(z)$, the original time signal can be obtained by the inverse $\mathcal{Z}$-transform, which can be derived from the corresponding discrete Fourier transform. As shown above, we have:
	
	Now $x[n]e^{-\sigma n}$ can be obtained by the inverse Fourier transform:
	
	Multiplying both sides by $e^{\sigma n}$, we get:
	
	To represent the inverse transform in terms of $z$ (instead of $\omega$), we note:
	
	i.e.:
	
	and the "\NewTerm{inverse $\mathcal{Z}$-transform}\index{inverse $\mathcal{Z}$-transform}" can be obtained as:
	
	Note that the integral with respect to $\omega$ from $0$ to $2\pi$ becomes an integral with respect to $z=e^{\sigma+\mathrm{i}\omega}$ in the complex $z$-plane, along a circle with a fixed radius $e^\sigma$ and a varying angle $\omega$ from $0$ to $2\pi$. Now we have the $\mathcal{Z}$-transform pair:
	
	
	The forward and inverse $\mathcal{Z}$-transform pair can also be represented as:
	
	In particular, if we let $\sigma=0$, i.e., $z=e^{j\omega}$, then the $\mathcal{Z}$-transform becomes the discrete-time Fourier transform:
	
	This is the reason why sometimes the discrete Fourier spectrum is expressed as a function of $e^{\mathrm{i}\omega}$.
	
	
	Different from the discrete-time Fourier transform which converts a 1-D signal $x[n]$ in time domain to a 1-D complex spectrum $X(e^{\mathrm{i}\omega})$ in frequency domain, the $\mathcal{Z}$-transform $X(s)$ converts the 1-D signal $x[n]$ to a complex function defined over a 2-D complex plane, named the  "\NewTerm{$z$-plane}\index{$z$-plane}", represented in polar form by radius $|z|=|e^{\sigma+\mathrm{i}\omega}|=e^\sigma$ and angle 
	$\angle z=\angle(e^{\sigma+\mathrm{i}\omega})=\omega$. 
	
	In particular, when this 2-D function $X(z)=X(e^{\sigma+\mathrm{i}\omega})$ is evaluated along the unit circle $|z|=e^0=1$ corresponding to $\sigma=0$, it becomes a 1-D periodic function $X(e^{\mathrm{i}\omega})$, the discrete Fourier transform of $x[n]$. 
	
	\pagebreak
	\paragraph{Transfer function of LTI system}\mbox{}\\\\
	The output $y[n]$ of a discrete LTI (Linear Time-Invariant) system with input $x[n]$ can be found by convolution (see page \pageref{convolution}):
	
	where $h[n]$ is the impulse response function of the system. In 	particular, if the input is a complex exponential:
	
	then the output $y[n]$ can be found to be:
	
	This is the eigenequation with the complex exponential $x[n]=z^n=e^{sn}$ being the eigenfunction of any discrete LTI system, corresponding to its eigenvalue defined as:
	
	which is the $\mathcal{Z}$-transform of its impulse response $h[n]$, called the "\NewTerm{transfer function}\index{transfer function}" of the LTI system. In particular, when $\sigma=0$, i.e., $z=e^s=e^{\mathrm{i}\omega}$, the transfer function $H(z)$ becomes the frequency response function, the Fourier transform of the impulse response:
	
	
	\paragraph{Conformal mapping between $s$-plane to $z$-plane}\mbox{}\\\\
	The $s$-plane and the $z$-plane are related by a conformal mapping (ie. conserves the angles) specified by the analytic complex function:
	
	where :
	
	Even if, as far as i know, the is no important application of mapping, it is more a mathematical curiosity the may help some students to understand how the $s$-plane and $z$-plane are related together.
	
	The mapping is continuous, i.e., neighbouring points in $s$-plane are mapped to neighbouring points in $z$-plane and vice versa. Consider the mapping of these specific features: 
	\begin{itemize}
		\item The origin $s=0$ of $s$-plane is mapped to $z=e^0=1$ on the real axis in $z$-plane.
		\item Each vertical line $\Re[s]=\sigma_0$ in $s$-plane is mapped to a circle $|z|=e^{\sigma_0}$ centered about the origin in $z$-plane. In particular,
			\begin{itemize}
			\item Leftmost vertical line $\Re[s]=\sigma=-\infty$ is mapped as the origin $|z|=e^{-\infty}=0$
			\item Imaginary axis $\Re[s]=0$ is mapped as the unit circle $|z|=e^0=1$
			\item Rightmost vertical line $\Re[s]=\sigma={+\infty}$ is mapped as a circle of infinite radius $|z|=e^{{+\infty}}={+\infty}$.
			\end{itemize}
		\item Each horizontal line $\Im[s]=\mathrm{i}\omega_0$ in $s$-plane is mapped to $\angle{z}=\omega_0$, a ray from the origin in $z$-plane of angle $\omega_0$ with respect to the positive horizontal direction. 
		\item A right angle formed by a pair vertical and horizontal lines in $s$-plane is conserved by the mapping, as the corresponding circle and ray in $z$-plane also form a right angle (in fact any angle is conserved, an important property of the conformal mapping).
	\end{itemize}
	The infinite range $-\infty < \omega < {+\infty}$ for frequency $\omega$ along a vertical line $\Re[s]=\sigma_0$ in $s$-plane is mapped repeatedly to a finite range $0 \le \omega < 2\pi$ around a circle $|z|=e^{\sigma_0}$ in $z$-plane, corresponding to the conversion of a continuous signal $x(t)$ with non-periodic spectrum $X(\mathrm{i}\omega)$ for $-\infty < \omega < {+\infty}$ to a discrete signal $x[n]$ with periodic spectrum $X(e^{\mathrm{i}\omega})$ for $0 \le \omega < 2\pi$.
	
	\paragraph{Region of Convergence}\mbox{}\\\\
	Whether the $\mathcal{Z}$-transform $X(z)$ of a signal $x[n]$ exists depends on the complex variable $z=e^s$ as well as the signal itself. $X(z)$ exists if and only if the argument $z$ is inside the region of convergence (ROC) in the $z$-plane, which is composed of all $z$ values for the summation of the $\mathcal{Z}$-transform to converge. The ROC of the $\mathcal{Z}$-transform is determined by $|z|=|e^s|=e^{\sigma}$ (a circle), the magnitude of variable $z$, while the ROC for the Laplace transform is determined by $\sigma=\Re[s]$, (a vertical line), the real part of $s$. 
	
	\begin{tcolorbox}[colframe=black,colback=white,sharp corners]
	\textbf{{\Large \ding{45}}Examples:}\\\\
	E1. Let us see our first $\mathcal{Z}$-transform with at the same time its corresponding region of convergence. For this purpose, let us first recall the geometric series (see page \pageref{sum of powers}):
	
	for $|x|<1$. The $\mathcal{Z}$-transform of the following right sided signal $x[n]=a^n u[n]$ is:
	
	\end{tcolorbox}
	
	\begin{tcolorbox}[colframe=black,colback=white,sharp corners]
	
	If we put $a=1$, we simply get the $\mathcal{Z}$-transform of the Heaviside function (as a simple geometric series: $1+z^{-1}+z^{-2}+\ldots$\\
	
	For this summation to converge, i.e., for $X(z)$ to exist, it is necessary to have $| az^{-1} |<1$, i.e., the ROC is $|z| > |a|$. As a special case when $a=1$, $x[n]=u[n]$ and we have:
	
	E2. Let us see now a simple trivial example of an inverse of the following given $\mathcal{Z}$-transform:
	
	Comparing this with the definition of $\mathcal{Z}$-transform:
	
	we get:
	
	In general, we can use the time shifting property:
	
	to inverse transform the $X(z)$ given above to $x[n]$ directly.\\
	
	E3.  Let us determine the obvious inverse transform of:
	
	As we know that:
	
	which converges if the ROC is $|z|>|a|$, i.e., $|az^{-1}|<1$ then we get:
	
	\end{tcolorbox}
	
	\pagebreak
	\paragraph{Zeros and Poles of $\mathcal{Z}$-Transform}\mbox{}\\\\
	All $\mathcal{Z}$-transforms in the above examples are rational, i.e., they can be written as a ratio of polynomials of variable $z$ in the general form:
	
	where $N(z)$ is the numerator polynomial of order $M$ with roots $z_{z_k}, (k=1,2, \cdots, M)$, and $D(z)$ is the denominator polynomial of order $N$ with roots $z_{p_k}, (k=1,2, \cdots, N)$. In general, we assume the order of the numerator polynomial is lower than that of the denominator polynomial, i.e.,  $M < N$. If this is not the case, we can always expand $X(z)$ into multiple terms so that $M<N$ is true for each of terms.
	
	The "\NewTerm{zeros}\index{zeros}" and "\NewTerm{poles}\index{poles}" of a rational $X(z)=N(z)/D(z)$ are defined as:
	
	\textbf{Definitions (\#\mydef):} 
	\begin{enumerate}
		\item[D1.]  Each of the roots of the numerator polynomial $z_z$ for which:
		
		is a "\NewTerm{zero}" of $X(z)$.
	
	  	If the order of $D(z)$ exceeds that of $N(z)$ (i.e., $N>M$), then $X({+\infty})=0$, i.e., there is a zero at infinity:
	  	
	
		\item[D2.] Each of the roots of the denominator polynomial $z_p$ for which :
		
		is a "\NewTerm{pole}" of $X(z)$.
	
		If the order of $N(z)$ exceeds that of $D(z)$ (i.e., $M>N$), then $X({+\infty})={+\infty}$, i.e, there is a pole at infinity: 
	  
	\end{enumerate}
	Most essential behaviour properties of an LTI system can be obtained graphically from the ROC and the zeros and poles of its transfer function $H(z)$ on the $z$-plane.
	
	\paragraph{Properties of $\mathcal{Z}$-Transform}\mbox{}\\\\
	The $\mathcal{Z}$-transform has a set of properties in parallel with that of the Fourier transform (and Laplace transform). The difference is that we need to pay special attention to the ROCs. In the following, we always assume:
	
	and:
	
	
	\begin{itemize}
		\item Linearity:
		
		While it is obvious that the ROC of the linear combination of $x[n]$ and $y[n]$ should be the intersection of the their individual ROCs $R_x \cap R_y$ in which both $X(z)$ and $Y(z)$ exist, note that in some cases the ROC of the linear combination could be larger than $R_x \cap R_y$. For example, for both $x[n]=a^n u[n]$ and $y[n]=a^n u[n-1]$, the ROC is $|z|>|a|$, but the ROC of their difference $a^n u[n]-a^n u[n-1]=\delta[n]$ is the entire $z$-plane.
		
		\item Time shifting:
		
		\begin{dem}
		
		Define $m=n-n_0$, we have $n=m+n_0$ and:
		
		The new ROC is the same as the old one except the possible addition/deletion of the origin or infinity as the shift may change the duration of the signal.
		\begin{flushright}
			$\blacksquare$  Q.E.D.
		\end{flushright}
		\end{dem}
		A slightly different approach lead us to a different result that is more explicit and by the way more useful especially in business applications (we change the notation a bit to be in conformity with the tradition in the financial field). First let us consider the special case (shift of one unit only in the positive direction\footnote{This is then a "unilateral $\mathcal{Z}$-transform" and as we will see it on page , the properties then differ slightly from the bilateral version.}):
		
		Repeating exactly the same type of procedure we get quite simply in the general way for $m>0$:
		
		
		\item Time Expansion (Scaling):
		
		
		The discrete signal $x[n]$ cannot be continuously scaled in time as $n$ has to be an integer (for a non-integer $n$ $x[n]$ is zero). Therefore $x[n/k]$ is defined as:
		
		\begin{tcolorbox}[colframe=black,colback=white,sharp corners]
		\textbf{{\Large \ding{45}}Example:}\\\\
		If $x[n]$ is ramp:
		\begin{table}[H]
		\centering
		\begin{tabular}{c|cccccc} \hline
		 $n$ & 1 & 2 & 3 & 4 & 5 & 6 \\ \hline 
		 $x[n]$ & 1 & 2 & 3 & 4 & 5 & 6 \\ \hline 
		\end{tabular}
		\end{table}
		
		then the expanded version $x[n/2]$ is :
		\begin{table}[H]
		\centering
		\begin{tabular}{c|cccccc} \hline
		 $n$ & 1 & 2 & 3 & 4 & 5 & 6 \\ \hline
		 $n/2$ & 0.5 & 1 & 1.5 & 2 & 2.5 & 3 \\ \hline
		 $m$ &	     & 1 &     & 2 &     & 3 \\ \hline
		 $x[n/2]$ & 0 & 1 & 0 & 2 & 0 & 3 \\ \hline 
		\end{tabular}
		\end{table}
		
		where $m$ is the integer part of $n/k$.
		\end{tcolorbox} 
		
		\begin{dem}
		 The $\mathcal{Z}$-transform of such an expanded signal is:
		
		Note that the change of the summation index from $n$ to $m$ has no effect as the terms skipped are all zeros.
		\begin{flushright}
			$\blacksquare$  Q.E.D.
		\end{flushright}
		\end{dem}
		
		\item Convolution:
		
		The ROC of the convolution could be larger than the intersection of $R_x$ and $R_y$, due to the possible pole-zero cancellation caused by the convolution.
		
		\item Time Difference:
		
		\begin{dem}
		
		Note that due to the additional zero $z=1$ and pole $z=0$, the resulting ROC is the same as $R_x$ except the possible deletion of $z=0$ caused by the added pole and/or addition of $z=1$ caused by the added zero which may cancel an existing pole.
		\begin{flushright}
			$\blacksquare$  Q.E.D.
		\end{flushright}
		\end{dem}
		
		\item Time Accumulation:
			
		\begin{dem}
		The accumulation of $x[n]$ can be written as its convolution with $u[n]$:
		
		Applying the convolution property, we get:
		
		as ${\cal Z}[u[n]]=1/(1-z^{-1})$.
		\begin{flushright}
			$\blacksquare$  Q.E.D.
		\end{flushright}
		\end{dem}
		
		\item Time Reversal:
		
		\begin{dem}
		
		where $m=-n$.
		\begin{flushright}
			$\blacksquare$  Q.E.D.
		\end{flushright}
		\end{dem}
		
		\item Scaling in $z$-domain:
		
		
		\begin{dem}
		
		In particular, if $a=e^{j\omega_0}$, the above becomes:
		
		The multiplication by $e^{-\mathrm{i}\omega_0}$ to $z$ corresponds to a rotation by  angle $\omega_0$ in the $z$-plane, i.e., a frequency shift by $\omega_0$. The rotation is either clockwise ($\omega_0>0$) or counter clockwise ($\omega_0<0$) corresponding to, respectively, either a left-shift or a right shift in frequency domain. The property is essentially the same as the frequency shifting property of discrete Fourier transform.
		\begin{flushright}
			$\blacksquare$  Q.E.D.
		\end{flushright} 
		\end{dem}
		
		\item Conjugation:
		
		\begin{dem}
		Complex conjugate of the $\mathcal{Z}$-transform of $x[n]$ is
		
		Replacing $z$ by $z^*$, we get the desired result.
		\begin{flushright}
			$\blacksquare$  Q.E.D.
		\end{flushright}
		\end{dem}
		
		\item Differentiation in $z$-domain:
		
		\begin{dem}
		
		i.e.:
		
		\begin{flushright}
			$\blacksquare$  Q.E.D.
		\end{flushright}
		\end{dem}
		\begin{tcolorbox}[colframe=black,colback=white,sharp corners]
		\textbf{{\Large \ding{45}}Example:}\\\\
		Taking derivative with respect to $z$ of the right side of:
		
		we get:
		
		Due to the  property of differentiation in $z$-domain, we have:
		
		Note that for a different ROC $|z|<|a|$, we have:
		
		\end{tcolorbox} 
	\end{itemize}
	
	\paragraph{Usual $\mathcal{Z}$-Transforms}\mbox{}\\\\
	\begin{enumerate}

		\item $\delta[n]$, $\delta[n-m]$
		
		Due to the time shifting property, we also have:
		
		
		\item $u[n]$, $a^n u[n]$, $n a^n u[n]$
		
		
		Due to the scaling in $z$-domain property, we have:
		
		Or for those that don't like using this property, we have simply using the previous result:
		
		Applying the property of differentiation in $z$-domain to the above, we have:
		
		
		\item $u[n-k]$

		Using the time shifting property that is for recall given by:
		
		we have:
		
		\begin{tcolorbox}[colframe=black,colback=white,sharp corners]
		\textbf{{\Large \ding{45}}Example:}\\\\
		A useful case for a simple business application that we will use later is:
		
		\end{tcolorbox}
		
		\item $e^{\pm jn\omega_0}u[n]$, $\cos[n\omega_0]u[n]$, $\sin[n\omega_0]u[n]$
		
		Applying the scaling in $z$-domain property to ${\cal Z}[u[n]]=1/(1-z^{-1})$, we have:
		
		and similarly, we have:
		
		Moreover, we have:
		
		Similarly we have:
		
		
		\item $r^n \cos[n\omega_0]u[n]$, $r^n \sin[n\omega_0]u[n]$
		
		Applying the $z$-domain scaling property to the above, we have:
		
		and:
		
	
	\end{enumerate}
	
	
	\paragraph{Unilateral $\mathcal{Z}$-Transform}\mbox{}\\\\
	The "\NewTerm{unilateral $\mathcal{Z}$-transform}\index{unilateral $\mathcal{Z}$-transform}" of an arbitrary signal $x[n]$ is defined as:
	
	where for recall $\cal U$ and $u[n]$ are another typical notation of the Heaviside function.
	
	Some of the properties of the unilateral $\mathcal{Z}$-transform that differ slightly from the bilateral $\mathcal{Z}$-transform are listed below:
	\begin{enumerate}
		\item[P1.] Time Advance:
		
		where we have assumed $m=n+1$.
		
		\item[P2.] Time Delay:
		
		where $m=n-1$. Similarly, we have:
		
		where $m=n-2$. In general, we have:
		
		
		\item[P3.] Convolution:
		
		If both $x[n]$ and $y[n]$ are causal, i.e., $x[n]=y[n]=0$ for $n<0$, the unilateral and bilateral $\mathcal{Z}$ transforms are identical.
		
		\item[P4.] Time Difference:
		
		\begin{dem}
		
		\begin{flushright}
			$\blacksquare$  Q.E.D.
		\end{flushright}
		\end{dem} 
		
		
		\item[P5.] Time Accumulation:
		
		
		\item[P6.] Initial Value Theorem:
		
		If $x[n]=x[n]u[n]$, i.e., $x[n]=0$ for $n<0$, then:
		
		\begin{dem}
		
		All terms with $n>0$ become zero as $z^{-n}=1/z^n \rightarrow 0$ as 
		$z \rightarrow {+\infty}$, except the first one which is always $x[0]$.
		\begin{flushright}
			$\blacksquare$  Q.E.D.
		\end{flushright}
		\end{dem}		
		
		\item[P7.] Final Value Theorem:
		
		If $x[n]=x[n]u[n]$, i.e., $x[n]=0$ for $n<0$, then:
		
		\begin{dem}

		i.e.:
		
		Letting $z\rightarrow 1$ in the above, we get:
		
		where $x[-1]=0$.
		\begin{flushright}
			$\blacksquare$  Q.E.D.
		\end{flushright}
		\end{dem}
	\end{enumerate}
	
	
	\StickyNote[2.5cm]{\LARGE To finish depending on donations}[6.5cm]
	
	\pagebreak
	\subsubsection{Hilbert Transform}\label{hilbert transform}
	The information about the Hilbert transform is often scattered in most textbooks about signal processing. Their authors frequently use mathematical formulas without explaining them thoroughly to the reader at the opposite of the Fourier, Laplace or $\mathcal{Z}$-transform. Our purpose is to make a more stringent presentation of the Hilbert transform but still with the signal processing application in mind.
	
	\textbf{Definition (\#\mydef):} The "\NewTerm{Hilbert transform}\index{Hilbert transform}" of a function $f(t)$ is defined for all $t$ by:
	
	when the integral exists.
	
	It is normally not possible to calculated the Hilbert transform as an ordinary improper integral because of the pole at $\tau=t$. However, the $P$ in front of the integral denotes the "\NewTerm{Cauchy principal value}\index{Cauchy principal value}" which expanding the class of function for which the integral definition above exist. It can be defined as following:
	
	\textbf{Definition (\#\mydef):} The "\NewTerm{Cauchy principal value}\index{Cauchy principal value}" is a method for assigning values to certain improper integrals (\SeeChapter{see section Differential and Integral Calculus page \pageref{improper integral}}) which would otherwise be undefined. Depending on the type of singularity in the integrand $f$, the Cauchy principal value is defined according to the following rules:
	\begin{itemize}
		\item For a singularity at the finite number $b$:
		
		where $b$ is a point at which the behaviour of the function $f$ is such that:
		
	
		\item For a singularity at infinity:
		
		where:
		
	\end{itemize}
	In some cases it is necessary to deal simultaneously with singularities both at a finite number $b$ and at infinity. This is usually done by a limit of the form:
	
	The Cauchy principal value can also be defined in terms of contour integrals of a complex-valued function $f(z)$; $z = x + \mathrm{i}y$, with a pole on a contour $C$. Define $C(\varepsilon)$ to be the same contour where the portion inside the disk of radius $\varepsilon$ around the pole has been removed. Provided the function $f(z)$ is integrable over $C(\varepsilon)$ no matter how small $\varepsilon$ becomes, then the Cauchy principal value is the limit:
	
	\begin{tcolorbox}[colframe=black,colback=white,sharp corners]
	\textbf{{\Large \ding{45}}Example:}\\\\
	Consider the ill-defined expression:
	
	As the primitive of $1/x$ is equal to $\ln(x)$ there is obviously a problem... So the idea is to write instead:
	
	but as the $1/x$ function is odd, we can write this:
	
	that is obviously equal to $0$! So $0$ is the principal value of the initial integral.
	\end{tcolorbox}
	
	\begin{tcolorbox}[title=Remark,colframe=black,arc=10pt]
	Sadly different authors use different notations for the Cauchy principal value of a function $f$, among others:
	
	\end{tcolorbox}

	\StickyNote[2.5cm]{\LARGE To finish depending on donations}[6.5cm]
	
	\pagebreak
	\subsection{Functional dot product (inner product)}\label{functional dot product}
	The "\NewTerm{functional dot product}\index{function dot product}\index{orthogonality of functions}\index{functions orthogonality}" (very strong analogy with the dot product in seen in the section Vector Calculus) may seem unnecessary when examined for the first time outside of an application context or only as generalization purpose, but in fact it has many practical applications. We will make such direct use in the section of Wave Quantum Physics and Quantum Chemistry, or even more important in the context of trigonometric polynomials through the Fourier series and transforms (\SeeChapter{see section Sequences and Series page \pageref{fourier series}}) that we find everywhere in contemporary physics and computer science.
	
	However, if the reader has not travelled the section of Vector Calculus and the part treating the vector dot product, we would highly recommend reading it otherwise what follows may be a little incomprehensible.
	
	We put ourselves in the space $\mathcal{C}([a,b],\mathbb{R})$ of continuous functions in the interval $[a, b]$ into $\mathbb{R}$ with the inner product defined by (we find here again the specific notation of the dot product in its functional - ie integral - version as we had mentioned during our definition of the vector dot product in the section of Vector Calculus):
	
	
	A family of orthogonal polynomials, as we can make the analogy with the dot product in the section Vector Calculus, is therefore a polynomial family $(p_0,...,p_n,...)$ such as:
	
	if $j \ne k$. We recall that an orthogonal family is a free family. We also saw in the section of Vector Calculus that in the space $\mathcal{C}([a,b],\mathbb{C})$ the only possible coherent choice was:
	
	We name the two previous relations "\NewTerm{$L^2$-dot product}\index{dot product!$L^2$-dot product}".
	Remember now that if we want to show something is an inner product, we need to show three things for all $f,g\in \mathbb{R}^2$ and $\alpha\in \mathbb{R}$:
	\begin{enumerate}
		\item Symmetry: $\langle f|g\rangle=\langle g|f\rangle$ (or, if the field is the complex numbers $\mathbb{C}^n$, $\langle f|g\rangle=\overline{\langle g|f\rangle}$, i.e. "conjugate symmetry.)
	
		\item Linearity: $\langle \alpha f|g\rangle=\alpha \langle f|g\rangle$. Notice this also implies $\langle f|\alpha g\rangle=\alpha \langle f|g\rangle$ ($\bar{\alpha}$ in the complex case) by symmetry
	
		\item Positive-definite: $\langle f,f\rangle\geq 0$ with equality if and only if $f=0$, the zero function
	\end{enumerate}
	The first two properties follow directly from the definition of an integral. For the third property, it's quite obvious (the integral of $f^2$ can be zero only if $f$ is zero).
	\begin{tcolorbox}[title=Remark,colframe=black,arc=10pt]
	If such a dot product exist, we can then the Cauchy-Schwarz inequality also applies to it. We name it the "\NewTerm{Cauchy-Schwarz inequality for integrals}\index{Cauchy-Schwarz inequality for integrals}".
	\end{tcolorbox}
	
	Therefore we can build the "\NewTerm{$L^2$-norm}\index{norm!$L^2$-norm}":
	
	\begin{tcolorbox}[title=Remark,colframe=black,arc=10pt]
	We will see further below that the definition above is not the most general one as especially physicists and engineers say that functions are orthogonal under a more constraint situation!
	\end{tcolorbox}
	The development that will follow will remind us the Gram-Schmidt procedure (\SeeChapter{see section Vector Calculus page \pageref{gram-schmidt procedure}}) to build an orthogonal family.
	
	\begin{theorem}
	Given $(p_0,...,p_n,...)$ a family of linearly independent polynomial defined on $[a,b]$ and $V$ the vector space defined by this family. The family $(y_0,...,y_n,...)$ defined by recurrence in the following way:
	
	and $y_0=p_0$ is orthogonal and generates $V$.
	\end{theorem}
	\begin{dem}
	Let us show by induction on $n$ that $(y_0,y...,y_n,...)$ is an orthogonal family which generates the same space as $(p_0,...,p_n,...)$. The assertion holds for $n=0$. Let us suppose that the assertion holds for $n\geq 0$, for $0\leq k\leq n$ we have:
	
	$(y_0,...,y_n,...)$ is therefore orthogonal. Finally, the equality:	
	
	\begin{flushright}
		$\blacksquare$  Q.E.D.
	\end{flushright}
	\end{dem}
	
	\addcontentsline{toc}{paragraph}{Orthogonality of trigonometric functions}
	Let us see a first example very important is signal processing and statistics that is relatively to frequency analysis:
	\begin{tcolorbox}[colframe=black,colback=white,sharp corners]
	\textbf{{\Large \ding{45}}Example:}\\\\
	Let us consider the very important example in modern physics that is the set of continuous $2\pi$-periodic function denoted $P_{2\pi}$ that forms a vector space (\SeeChapter{see section Set Theory page \pageref{vector space}}).\\
	
	We define the dot product of two functions of this set by:
	
	The aim of this definition is to build an abstract functional basis $P_{2\pi}$ on which we can break down any $2\pi$-periodic function!!!\\
	
	The simplest idea is then to use the trigonometric functions sine and cosine:
	
	The relations below show that the basis chosen above are orthogonal and therefore form a free family, plus it's a generating family of the vector space $P_{2\pi}$ because as we have seen in our study of Fourier series (\SeeChapter{see section Sequences and Series page \pageref{fourier series}}), we have the following values:
	
	where $\delta_{km}$ is the kronecker symbol (\SeeChapter{see section Tensor Calculus page \pageref{kronecker symbol}}).\\
	
	Therefore it is also an orthogonal basis but not orthonormal. If we want to normalized the vectors of the basis we just need obviously to take:
	
	\end{tcolorbox}
	\begin{tcolorbox}[title=Remark,colframe=black,arc=10pt]
	If the reader remembers that for a random variable $X$ defined on $\mathbb{R}$, the mean was calculated as (\SeeChapter{see section Statistics page \pageref{expected mean continuous variable}}):
	
	The we can assimilate:
	
	where:
	
	to the expected mean of the function $g(x)$! Analogy sometimes very useful in practice!
	\end{tcolorbox}
	\addcontentsline{toc}{paragraph}{Orthogonality of complex exponential functions}
	Let us see now another example that is an extension of the previous one and that has also a great importance in signal processing but also in quantum physics and quantum chemistry:
	\begin{tcolorbox}[colframe=black,colback=white,sharp corners]
	\textbf{{\Large \ding{45}}Example:}\\\\
	Let us consider a basis of complex functions of the form $r e^{\mathrm{i}n\varphi}$ with $n\in\mathbb{Z}$. We therefore can write:
	
	We get:
	
	It is obvious that if we take for basis functions of the type:
	
	then we have an orthonormal basis (and not just an orthogonal one).
	\end{tcolorbox}
	\addcontentsline{toc}{paragraph}{Orthogonality of Bessel functions}\label{orthogonality of bessel functions}
	Another example that will be useful for us in the section of Wave Mechanics for our study of the ideal circular membrane of a drum.
	\begin{tcolorbox}[colframe=black,colback=white,sharp corners]
	\textbf{{\Large \ding{45}}Example:}\\\\
	We have proved in the section Sequences and Series that the Bessel function\index{Bessel function} $J_p(x)$ satisfies the following differential equation (Bessel's equation):
	
	which can be written as:
	
	The variable $p$ need not be an integer as we will see it in the section of Mechanical Engineering with the study case of the self-buckling column.\\
	
	It turns out to be useful to define a new variable $t$ by $x = a t$, where $a$ is a constant which we will take to be a zero of $J_p$, i.e. $J_p(a) = 0$. Let us define:
	
	which implies:
	
	and substituting into (\ref{eq}) gives:
	
	since $x \mathrm{d}/\mathrm{d}x$ is equivalent to $t \mathrm{d}/\mathrm{d}t$.
	We can also write down the equation obtained by picking another zero, $b$. Defining:
	
	which implies:
	
	we have then:
	
	To derive the orthogonality relation, we multiply (\ref{eq1}) by $v$, and (\ref{eq2}) by $u$. Subtracting and dividing by $t$ gives:
	\end{tcolorbox}
	
	\begin{tcolorbox}[colframe=black,colback=white,sharp corners]
	
	The first two terms in (\ref{combine}) can be combined as:
	
	since the extra terms present in (\ref{totalderiv}), but not in (\ref{combine}), when the derivatives are expanded out are equal and opposite and so cancel. Hence we have:
	
	We next integrate this over the range of $t$ from $0$ to $1$ ($0$ since the Bessel function is not defined for $t<0$ and to $1$ since it's the place where there is a zero by construction for recall!), which gives:
	
	The integrated term vanishes at the lower limit because $t=0$, and it also vanishes at the upper limit because $u(1) = v(1) = 0$, see (\ref{u10}) and (\ref{v10}). Hence, if $a \ne b$, (\ref{int01}) gives:
	
	which, using (\ref{ut}) and (\ref{v10}), can be written
	
	This is the desired orthogonality equation. Remember we require that $a$ and $ b$ are distinct zeroes of $J_p$, so both Bessel functions in (\ref{orthog}) vanish at the upper limit.
	\end{tcolorbox}
	
	\pagebreak
	\addcontentsline{toc}{paragraph}{Orthogonality of Hermite polynomial}
	Another that will be useful for us in the section of Wave Quantum Physics during our study of the harmonic oscillator:
	\begin{tcolorbox}[colframe=black,colback=white,sharp corners]
	\textbf{{\Large \ding{45}}Example:}\\\\
	We will introduce in the section of Wave Quantum Physics the following "physicist Hermite polynomial\index{Hermite polynomials}\label{hermite polynomial}":
	
	Therefore (see the plot in the section of Wave Quantum Physics):
	
	where we notice almost immediately that (useful for further below):
	
	And we need to prove that they are orthogonal (or even better: orthonormal!).\\
	
	For this purpose we introduce the weight function $w(x)=e^{-x^2}$ therefore:
	
	So we get (we use the Gauss integral as proved in the section Statistics page \pageref{Gauss integral}):
	
	\end{tcolorbox}
	
	\begin{tcolorbox}[colframe=black,colback=white,sharp corners]
	Finally, using Kronecker symbol (\SeeChapter{see section Tensor Calculus page \pageref{kronecker symbol}}):
	
	\end{tcolorbox}
	
	\addcontentsline{toc}{paragraph}{Orthogonality of Laguerre polynomial}\label{orthogonality of Laguerre polynomial}
	Now let us see a last example useful for our study of the non-rigid rotator in Quantum Chemistry (radial part solution of the hydrogenoid Schrödinger equation):
	\begin{tcolorbox}[colframe=black,colback=white,sharp corners]
	\textbf{{\Large \ding{45}}Example:}\\\\
	If $L_{m}(x)$ and $L_{n}(x)$ are Laguerre's polynomials (\SeeChapter{see section page \pageref{Laguerre polynomials}}) $m,n$ being positive integers) then:
	
	where
	
	Proof: The generating function for Laguerre's polynomial gives:
	
	We now multiply both sides of the above by $e^{-x}$ and integrate both sides from $0$ to $+\infty$ with respect to $x$, which gives:
	
	Therefore:
	
	\end{tcolorbox}
	
	\begin{tcolorbox}[colframe=black,colback=white,sharp corners]
	
	When $m\neq n$, equating coefficients of $t^{n} s^{m}$ on both sides of the above relation gives:
	
	When $m=n$, equating coefficients of $t^{n} s^{n}$ from both sides of the prior previous relation gives:
	
	Combining we get:
	
	where:
	
	\end{tcolorbox}
	
	\begin{tcolorbox}[title=Remark,colframe=black,arc=10pt]
	\addcontentsline{toc}{paragraph}{Orthogonality of Legendre polynomial}We have already proved in the section of Calculus at page \pageref{legendre polynomials} that the Legendre polynomials (useful for our study of Quantum Chemistry) are orthogonal.
	\end{tcolorbox}
	From what we have seen above we deduce that:
	
	is in fact generalized by:
	
	where $w(x)$ is the "\NewTerm{weight function}\index{weight function}". 
	
	So the engineer, physicist, mathematician must always be careful when he see in a textbook a sentence of the type: \textit{these functions are orthogonal}. Indeed the author/redactor should instead read: \textit{these functions are orthogonal with a given weight}.
	
	\subsubsection{Cauchy-Schwarz inequality for integrals}\label{Cauchy-Schwarz inequality for integrals}
	Let us now provide a detailed proof of what we have mentioned earlier above: the 	"\NewTerm{Cauchy-Schwarz inequality for integrals}\index{Cauchy-Schwarz inequality for integrals}\label{Cauchy-Schwarz inequality for integrals}". As this will be an important result with quite important application in Statistics and Quantum Physics.
	
	\begin{theorem}
	Let $f$ and $g$ be real functions which are continuous on the closed interval $[a,b]$. Then:
	
	\end{theorem}
	\begin{dem}
	For the proof we use the following clever trick! We put $\forall x\in\mathbb{R}$:
	
	Hence using relative sizes of definite integrals:
	
	Using the linear combination of integrals property:
	
	Let us put:
	
	where:
	
	The quadratic equation $Ax^2+2Bx+C$ is non-negative for all $x$. It follows (using the same reasoning as in Cauchy's Inequality) that the discriminant $(2B)^2-4AC$ of this polynomial must be non-positive.
	
	Thus:
	
	and hence the result!
	\begin{flushright}
		$\blacksquare$  Q.E.D.
	\end{flushright}
	\end{dem}
	
	\begin{flushright}
	\begin{tabular}{l c}
	\circled{100} & \pbox{20cm}{\score{4}{5} \\ {\tiny 47 votes,  71.49\%}} 
	\end{tabular} 
	\end{flushright}
	
	%to make section start on odd page
	\newpage
	\thispagestyle{empty}
	\mbox{}
	\section{Complex Analysis}\label{complex analysis}

	\lettrine[lines=4]{\color{BrickRed}B}efore starting this section on the study of differential and integral calculus in the generalized case of complex numbers, I should point out that I used many illustrations inspired by the PDF of E.~Hairer (with his permission). This text also contains many sentences and developments taken, homogenized and simplified from the same PDF (at the risk to make some purist readers climbing to the curtains...) according to the notations and educational objectives of the rest of this book.

	The subject of the complex analysis is the study of functions $\mathbb{C} \mapsto \mathbb{C}$ and their differentiability (which is different from that in $\mathbb{R}^2$). The "holomorphic functions" (that is to say differentiable in a subset of $\mathbb{C}$) have as we will see it later surprising and elegant properties that can be reused in the situation of the special case of functions in $\mathbb{R}^2$ (remember that $\mathbb{C}$ is a generalization of $\mathbb{R}^2$ ) that have important applications in advanced physics (we will use the results of this section for our study of quantum physics and also for some applications of fluid mechanics and also for advanced models in financial options pricing).

	Before we start let us first explain the interest of Complex Analysis in a simplified way!

	We studied in the chapter Algebra a part of the Differential and Integral Calculus with some useful and important theorems in physics and engineering. However, staying in $\mathbb{R}$ or $\mathbb{R}^2$ the list of theorems runs out somehow and we end up finding much relevant tools in practice that allows to simplify the integration calculation that we can sometimes found in industrial applications. So, when we remember that $\mathbb{R} \subset \mathbb{C}$ (thus the set of complex numbers generalizes the set of real numbers) and that we can also build a correspondence $\mathbb{R}^2 \mapsto \mathbb{C}$ as we shall see, then new theorems appear with very interesting results that can be exploit for the integrals in $\mathbb{R}$ or $\mathbb{R}^2$!! It is because of this reason that the engineer needs to know Complex Analysis!
	
	After studying this particular field of mathematics, it is common to say that the shortest path between two truths of the real domain often requires to pass trough the complex domain...

	\subsection{Linear Applications}

	A good introduction to complex analysis and its representation is to look at first (for educational purposes mainly) the special case of complex linear applications. Let us see this!

Let $U \subset \mathbb{C}$ be a set and $V \subset \mathbb{C}$ another set. A function that associates to each $z \in U$ an $w \in V$ such that:
	
is a "\NewTerm{complex function}\index{complex function}":
	
What is important is to remember (\SeeChapter{see section Numbers page \pageref{complex numbers}}) that we can identify:
	
and:
	
We have thus two functions of two real variables $x, y$:
	
which are the coordinates of the point $w$.

\textbf{Definition (\#\mydef):} An application is named "\NewTerm{$\mathbb{C}$-linear}\index{$\mathbb{C}$-linear function}" if for example a function of the type:
	
where $c$ is a fixed complex number and $z$ an arbitrary complex number, satisfying:
	

That is to say that $f(z)$ must me additive and homogeneous or just briefly when this two properties are satisfied we say that $f(z)$ is a "\NewTerm{linear map}\index{linear map}".

We have seen and proved in the section on Numbers during our study of complex numbers, that the multiplication of two complex numbers could be equivalent to an orthogonal rotation followed by a scaling and that this same multiplication could be represented in matrix form! Or the transcription into a matrix form involves as we saw in the section on Linear Algebra automatically  the property of linearity!

So the reader can easily check that a matrix of rotation/scaling is an example of an $\mathbb{C}$-linear application (on request we can detail) that we will now write:
	
Which can be typically represented as follows (we can clearly see a rotation and a scaling which conserve the angles and proportions):

\begin{figure}[H]
\centering
\includegraphics[scale=0.75]{img/analysis/clinear_application.eps}
\caption{$\mathbb{C}$-linear application example}
\end{figure}

It is the fact that the proportions and the angles are kept that makes a complex function $\mathbb{C}$-linear. Otherwise, we would say that the function is $\mathbb{R}$-linear.

So a matrix equation is $\mathbb{C}$-linear if and only if it is of the form:
	
Let us see some examples of quite remarkable $\mathbb{C}$ non-linear functions.

	\begin{tcolorbox}[colframe=black,colback=white,sharp corners]
\textbf{{\Large \ding{45}}Examples:}\\\\
E1. Consider the function:
	
In real coordinates this gives:
		
So let's look what this function do with the points of the complex plane which are coincident with the vertical lines of this same plane (which take us to write $x=a$). Then we have:
	
	and eliminating y, we find the equation of a parabola or rather a family of  parabolas (for several values of $b$) which are open to the left of the pictured complex plane.\\

	Here is a picture representation of the complex plane on which we have drawn a cat head:
	\begin{figure}[H]
		\begin{center}
			\includegraphics[scale=0.75]{img/analysis/c_linear_image_cat.eps}
		\end{center}	
		\caption{Complex representation of the image of the example function}
	\end{figure}
	\end{tcolorbox}

	\begin{tcolorbox}[colframe=black,colback=white,sharp corners]
and if we look at the corresponding pre-image complex plane  then we have two heads of cats that appear:
	\begin{figure}[H]
		\begin{center}
			\includegraphics[scale=0.75]{img/analysis/c_linear_pre_image_cat.eps}
		\end{center}	
		\caption{Pre-image representation of the example function}
	\end{figure}
The appearance of these two heads of cats is that this function has 2 possible pre-images for each image point (so it is a surjective function - see section Set Theory page \pageref{surjective application}).\\

Here is a nice Maple 17.00 script by Carl Ebehart to check this (shame that this can not be done in an easier way in Maple):\\

\texttt{> complextools[gridimage] := proc(p)\\
local llhc, width, height, xres, yres, clrs, V, H, i,j,k,l,pz,x,y,z,f,g,xtcs,ytcs,opts,margs;\\
llhc := [-1, -1];\\
width := 2;height := 2;\\
xres := .25;yres := .25;\\
xtcs := 1; ytcs := 1;\\
clrs := [red, black];\\
opts := NULL;\\
opts := op(select(type,[args],`=`));\\
margs:= remove(type, [args] ,`=`) ;\\
if nops(margs) >1 and margs[2] <> `` then llhc := margs[2] fi:\\
if nops(margs) >2 and margs[3] <> `` then width := margs[3] fi:\\
if nops(margs) >3 and margs[4] <> `` then height := margs[4] fi:\\
if nops(margs) >4 and margs[5] <> `` then xres := margs[5] fi:\\
if nops(margs) >5 and margs[6] <> `` then yres := margs[6] fi:\\
if nops(margs) >6 and margs[7] <> `` then xtcs := margs[7] fi:\\
if nops(margs) >7 and margs[8] <> `` then ytcs := margs[8] fi:}
	\end{tcolorbox}

	\begin{tcolorbox}[colframe=black,colback=white,sharp corners]
\texttt{if nops(margs) >8 and margs[9] <> `` then clrs := margs[9] fi:\\
z:= x + I*y;\\
pz := evalc(p(z));\\
f := unapply(evalc( Re(pz)),x,y); g := unapply(evalc(Im(pz)),x,y);\\
V:= plot( [
seq([seq(op([[f(llhc[1] + i*xres ,llhc[2]+(j-1)*yres/ytcs),g(llhc[1] + i*xres ,llhc[2]+(j-1)*yres/ytcs)], [f(llhc[1] + i*xres , llhc[2] +j*yres/ytcs),g(llhc[1] + i*xres , llhc[2] +j*yres/ytcs)]]),
j=1..ytcs*height/yres)], i = 0 .. width/xres)
],color=clrs[1]);\\
H := plot( [
seq([seq(op([[f(llhc[1]+(j-1)*xres/xtcs,llhc[2] + i*yres),
g(llhc[1]+(j-1)*xres/xtcs,llhc[2] + i*yres)], 
[f(llhc[1] +j*xres/xtcs, llhc[2] + i*yres),
g(llhc[1] +j*xres/xtcs, llhc[2] + i*yres)]]),
j=1..xtcs*width/xres)], i = 0 .. height/yres)
],color=clrs[2]);\\
plots[display]([V,H],scaling=constrained,opts);
end:\\
with(complextools);}\\

\texttt{>plots[display]([seq(plots[display]([gridimage(z->z), gridimage(z->z\string^2)]),i=10)],insequence=true);}
	\begin{figure}[H]
		\begin{center}
			\includegraphics[scale=0.75]{img/analysis/c_linear_maple_transform.eps}
		\end{center}	
		\caption{Practical Maple 17.00 example of simple $C$-linear application}
	\end{figure}
	\end{tcolorbox}
	
	\begin{tcolorbox}[colframe=black,colback=white,sharp corners]
E2. Another interesting feature is the "\NewTerm{Cayley transformation}\index{Cayley transformation}" used in some areas of physics and defined as:
	
having as domain definition: $\mathbb{C}/\left\lbrace 1\right\rbrace$.\\

	We notice that this is an involutive function since:
	
	and as we have proved in the section of Proofs Theory that any involution function is both injective and surjective, then the Cayley transform is a bijective function.\\

This function transforms the imaginary axis $\mi y$ in unit circle (and vice versa as it is involutive). Let us see that:
	
where:
	
satisfies:
	
That is to say:
	
This is the equation of a circle as proven in the section of Analytical Geometric.
	\end{tcolorbox}

\pagebreak
	\begin{tcolorbox}[colframe=black,colback=white,sharp corners]
	E3. As another example of function, consider the "\NewTerm{Joukovski transformation}\index{Joukovski transformation}" defined by:
	
If the definition domain is built in polar coordinates look at how a circle or ellipse transforms with this function:

	\begin{figure}[H]
		\begin{center}
			\includegraphics[scale=0.75]{img/analysis/joukovski_pre_image.eps}
		\end{center}	
		\caption{Transformation into polar coordinates of an ellipse with the example function}
	\end{figure}
Then the image plane will be:
	\begin{figure}[H]
		\begin{center}
			\includegraphics[scale=0.5]{img/analysis/joukovski_image.eps}
		\end{center}	
		\caption{Result of the Joukovski transformation in polar coordinates}
	\end{figure}
It thus transforms the circles respectively centered at $0$ and the rays passing through $0$ into a family cofocal ellipses and hyperbole . To prove this fact, we use the complex polar coordinates (Euler formula) seen in the section on Numbers (\SeeChapter{see chapter Arithmetics page \pageref{euler formula}}):
	\end{tcolorbox}

	\begin{tcolorbox}[colframe=black,colback=white,sharp corners]
	
and:
	
Then we have:
	
therefore:
	
and we immediately see that (\SeeChapter{see section Trigonometry page \pageref{remarkable trigonometric identities}}):
	
has the form of the equation of an ellipse (\SeeChapter{see section Analytical Geometry page \pageref{analytical expression ellipse}}) and we also have:
	
	which is the equation of a hyperbola (\SeeChapter{see section Analytical Geometry page \pageref{hyperbola}}).\\

	This function is useful in case we cleverly place a circle through the point $z=1$ (as in the case of the first figure) the plan represented in polar coordinates with a dotted line might looks like an airplane wing. This allowed a time ago in aerodynamics (but the technique is obsolete today)  to transpose the study of a vector field of an airplane wing profile to the study of a circle profile and to do after the Joukovski transformation.
	\end{tcolorbox}
	
	\pagebreak
	\begin{tcolorbox}[colframe=black,colback=white,sharp corners]
Indeed, let us see a part of this still with Maple 4.00:\\

\texttt{> assume(x,real,y,real);}\\
\texttt{> z:=x+I*y;}\\
\texttt{> F:=1/2*(z+1/z);}\\
\texttt{> u:=Re(F);}\\
\texttt{> u:=evalc(u);}\\
\texttt{> v:=Im(F);}
\texttt{> v:=evalc(v);}\\
\texttt{> with(plots):with(plottools):}\\
\texttt{> p1:=disk([0,0],1,color=black):}\\
\texttt{> p2:=implicitplot({seq(v=b8,b=-10..10)},x=-4..4,y=-2..2,color=black):}\\
\texttt{> display([p2,p1],scaling=constrained);}\\

We thus get:

	\begin{figure}[H]
		\begin{center}
			\includegraphics[scale=0.5]{img/analysis/joukovski_application.eps}
		\end{center}	
		\caption{Important application example of the Joukovski function}
	\end{figure}

	\end{tcolorbox}
	Let us see a last example that shows an electric dipole with its electric field and potential lines (\SeeChapter{see section Electrostatics page \pageref{equipotentials}}) can bee seen as the emergence of the $\mathbb{C}$-linear function $1/z$:

	\begin{tcolorbox}[colframe=black,colback=white,sharp corners]
\textbf{{\Large \ding{45}}Examples:}\\\\
E4. Always with Maple 4.00b we write:\\

\texttt{>assume(x,real,y,real);}\\
\texttt{> z:=x+I*y;}\\
\texttt{> F:=1/z;}\\
\texttt{> u:=Re(F);u:=evalc(u);}\\
\texttt{> v:=Im(F);v:=evalc(v);}\\
\texttt{> with(plots):}\\
\texttt{> p1:=implicitplot({seq(u=a,a=-5..5)},x=-1..1,y=-1..1,numpoints=1000):}\\
\texttt{> p2:=implicitplot({seq(v=b,b=-5..5)},x=-1..1,y=-1..1,numpoints=1000,}\\
\texttt{color=green):}\\
\texttt{> display([p1,p2],scaling=constrained);}\\
	\end{tcolorbox}
	
	\pagebreak
	\begin{tcolorbox}[colframe=black,colback=white,sharp corners]
	\begin{figure}[H]
		\begin{center}
			\includegraphics[scale=0.5]{img/analysis/dipole.eps}
		\end{center}	
		\caption{Another important application of a complex application}
	\end{figure}
	\end{tcolorbox}
	
	\subsection{Holomorphic Functions}\label{holomorphic functions}
	The definition of the derivative with respect to a complex variable is naturally formally identical to the derivative with respect to a real variable.
	
	We then have, if the function $f(z)$ is differentiable in $z_0$:
	
	and we say (abusively in this book) that function is "\NewTerm{holomorphic}\index{holomorphic function}" (while in $\mathbb{R}$ we say "differentiable") or "\NewTerm{analytical}\index{analytical function}" in its domain or in a subset of it if it is differentiable at any point.
	
	In other words a holomorphic functions is a complex-valued function of one or more complex variables that is complex differentiable in a neighbourhood of every point in its domain. The existence of a complex derivative in a neighbourhood is a very strong condition, for it implies that any holomorphic function is actually infinitely differentiable and equal to its own Taylor series.
	
	\begin{tcolorbox}[title=Remarks,colframe=black,arc=10pt]
	\textbf{R1.} A complex function is derived like a real function, we just have to put $z$ as being $x$... at the condition of what we will see in what follows is respected!\\
	
	\textbf{R2.} In fact if the function is holomorphic in a subset of the complex plane, we will see a little further below in our study of the convergence of power series that this is than always an open subset.
	\end{tcolorbox}
	Equivalently, we say that the function $f$ is $\mathbb{C}$-differentiable if the following limit exists in  $\mathbb{C}$:
	
	Let us now present and prove a central theorem for complex analysis named "\NewTerm{Cauchy-Riemann theorem}\index{Cauchy-Riemann theorem}"!
	
	If the function:
	
	is $\mathbb{C}$-differentiable on $z_0=x_0+\mathrm{i}y_0$, then we have:
	
	which is somewhat the equivalent to the Schwarz theorem (limited to $\mathbb{R}$) proved in the section of Differential and Integral Calculus. The above two relations are named "Cauchy conditions". So these are the two conditions that must verify a complex function to be differentiable on $z_0$. Thus, it is possible to use these relations to examine the points where the function is not analytic.
	
	\begin{theorem}
	If these conditions are satisfied (what will prove right below), then we deduce that $u$ and $v$ must both harmonic functions of $x$ and $y$.
	\end{theorem}
	\begin{dem}
	As:
	
	by choosing:
	
	with $x \in \mathbb{R}$ we get:
	
	and as $x$ approaches a small value $\mathrm{d}x$, we have (\SeeChapter{see section Differential and Integral Calculus page \pageref{differential calculus}}):
	
	by choosing:
	
	with $y \in \mathrm{R}$, we get:
	
	and when $y$ tends to a small value we have (\SeeChapter{see section Differential and Integral Calculus page \pageref{differential calculus}}):
	
	So now we have:
	
	But remember we proved in the section of Integral and Differential Calculus the following theorem:
	
	Therefore:
	
	Therefore using directly Schwartz theorem:
	
	Which can be written:
	
	A trivial solution is obviously to have:
	
	Therefore the right to write:
	
	By identifying real and imaginary parts, we finish the proof!
	\begin{flushright}
		$\blacksquare$  Q.E.D.
	\end{flushright}
	\end{dem}
	So for $f$ to be differentiable in the complex domain $\mathbb{C}$ (holomorphic) at a point, it is sufficient that it be differentiable as a function of two real variables ($\mathbb{R}^2$-differentiable on $(x_0,y_0)$) and that its partial first derivatives at this point satisfy the Cauchy-Riemann equations.
	
	But, for it to be $\mathbb{C}$-differentiable, Cauchy-Riemann's equations must valid at all points of the complex plane (we sometimes speaks about "\NewTerm{complete functions}\index{complete functions}") and not only in a subdomain thereof! Otherwise, it contains therefore "\NewTerm{singularities}\index{singularities}" and we then speak of "\NewTerm{meromorphic function}\index{meromorphic function}\label{meromorphic function}" (which is therefore a holomorphic function excepted on singularities points).
	
	The Gamma function studied in the section of Differential and Integral Calculus (see page \pageref{gamma euler function}) is such a well-known function:
	
	\begin{figure}[H]
		\centering
		\includegraphics[scale=0.6]{img/analysis/gamma_meromorphic.jpg}
		\caption[Gamma function is meromorphic in the whole complex plane]{Gamma function is meromorphic in the whole complex plane (source: Wikipedia)}
	\end{figure}
	\begin{tcolorbox}[title=Remark,colframe=black,arc=10pt]
	Geometrically, we will prove later that a holomorphic function has a possible interpretation in the sense that it is a conformal transformation (angles conservation).
	\end{tcolorbox}
	
	Notice therefore that if $f (z)$ is $\mathbb{C}$-differentiable it can be developed as Taylor series (\SeeChapter{see section Sequences and Series page \pageref{taylor series}}):
	
	Note an important thing too. If we rewrite:
	
	as following:
	
	Then we say that $f$ is "\NewTerm{irrotational}\index{irrotational}" (\SeeChapter{see section Vector Calculus page \pageref{irrotational}}) since the first relation can be seen as:
	
	which is an important analogy! Finally, the second equation:
	
	also let us say by analogy (but it stops at a simple analogy!) that the bivariate function $f$ is non-divergent (\SeeChapter{see section Vector Calculus page \pageref{divergence vector field}}) what is good mnemonic way to remember this equation.
	
	Let's also show something else in evidence. If we take the two Cauchy-Riemann equations:
	
	and that we derivate them once again we get:
	
	and that we sum these two relations, we get then:
	
	It is the same with v. Then we have:
	
	And we know very well this form of equations (Maxwell-Poisson equation in the section of Electrodynamics and Newton-Poisson equation in the section of Astronomy...). This is a wave equation also named "\NewTerm{Laplace equation}\index{Laplace equation}" (nothing to do with that of the same name seen in our study of the hydrostatic!) and given by the scalar Laplacian (\SeeChapter{see section Vector Calculus page \pageref{scalar laplacian}}):
	
	Then it is traditional to say that $u$ is harmonic and of course we can get the same result with $v$! Well obviously ... we knew it, since we have already studied in the section Numbers that the real and imaginary parts of a complex number could be put in trigonometric form.

	Thanks to this discovery, Riemann opened the application of holomorphic functions in many problems of physics, since these equations are satisfied by the gravitational potential (Newton-Poisson equation in the section of Astronomy page \pageref{newton-poisson equation}) by electric and magnetic fields (Maxwell-Poisson equation in the section of Electrodynamics page \pageref{maxwell-poisson equation}) by heat balance (no examples yet in this book) and by movements without rotational of certain fluids (no examples either in this book yet).

	
	\begin{tcolorbox}[colframe=black,colback=white,sharp corners]
	\textbf{{\Large \ding{45}}Example:}\\\\
	The potential of a dipole can be described by the following holomorphic function:
	
	The figure below:
	\begin{figure}[H]
		\centering
		\includegraphics{img/analysis/holomorphic_dipole_plot.jpg}
		\caption{Plane representation of a well known holomorphic function...}
	\end{figure}
	shows level-curves (iso-curves) of the given harmonic functions $u (x, y)$ and $v (x, y)$ as real and complex parts of the function $f (z)$ of this example.
	\end{tcolorbox}
	
	
	\pagebreak
	\subsubsection{Orthogonality of real and imaginary iso-curves}
	We will now prove a friendly property that have the functions that satisfy the Cauchy conditions (i.e. that are analytic functions!). Indeed, remember that we have already seen above the function:
	
	which gave the following diagram:
	\begin{figure}[H]
		\begin{center}
			\includegraphics[scale=0.75]{img/analysis/c_linear_image_cat.eps}
		\end{center}	
		\caption{Reminder of plane representation of a complex function seen earlier}
	\end{figure}
	
	\begin{theorem}
	Well he functions satisfying the conditions Cauchy have the simple following geometrical property following: the lines whose real part of the function is constant $\mathcal{R}(f(z))=c^{te}$ and lines whose imaginary part is constant $\mathcal{I}(f(z))=c^{te}$  are orthogonal to each other (think to the trigonometric form of complex numbers it helps to better visualize!).
	
	In other words, the analytical complex functions are transformation functions of an area of the plane into a new plane where the angles are preserved. Then we say that the function is a "\NewTerm{complete transformation}\index{complete transformation}".
	\end{theorem}
	
	\begin{dem}
	For the proof remember that we have proved in section of Vector Calculus that  gradient of a function $f$ of $\mathbb{R}^2$ is given by:
	
	and as part of our study of isolines in the section of Differential Geometry that the tangent vector to isolines of the function $f$ will always be parallel to the vector of the plane:
	
	and that the latter two vectors are perpendicular, such that:
	
	Now assimilate the tangent (parallel) vector $\vec{t}_u$ to the real isolines:
	
	with:
	
	and the normal vector to the imaginary isolines:
	
	with the gradient $v$ of components:
	
	Using the Cauchy conditions proved above, we have for this last relation:
	
	By comparing:
	
	we therefore see that $\vec{t}_u$ and $\vec{\nabla}(v)$ are parallel (collinear). And since $\vec{t}_u$ is colinear the real isolines and that $\vec{\nabla}(v)$  is perpendicular to the imaginary isolines we finished our proof.
	\begin{flushright}
		$\blacksquare$  Q.E.D.
	\end{flushright}
	\end{dem}
	The reader may take as an example the function:
	
	mathematically and schematically detailed earlier above! But to change a little bit, consider an example that will accompany us throughout the rest of this section and that is the following holomorphic function:
	
	That gives us with Maple 4.00b:
	
	\texttt{
	>assume(x,real,y,real);\\
	> z:=1/(1+(x+I*y)\string^2);\\
	> F:=1/z;\\
	> u:=Re(F);\\
	> u:=evalc(u);\\
	> v:=Im(F);\\
	> v:=evalc(v);\\
	> with(plots):\\
	> p1:=implicitplot({seq(u=a,a=-5..5)},x=-5..5,y=-5..5,numpoints=1000):\\
	> p2:=implicitplot({seq(v=b,b=-5..5)},x=-5..5,y=-5..5,numpoints=1000,color=green):\\
	> display([p1,p2]);
	}
	
	which gives:
	\begin{figure}[H]
		\begin{center}
			\includegraphics{img/analysis/holomorphic_isoclines.jpg}
		\end{center}	
		\caption{Representation of an important holomorphic function with its isolines}
	\end{figure}
	
	\subsection{Complex Logarithm}
	We need for all functions built into $\mathbb{R}$ found their equivalent in $\mathbb{C}$ while knowing that if we reduce the case of $\mathbb{R}$ to $\mathbb{C}$ we must get back on our feet!
	
	To do this, let us start with the most classical and academic function which is the logarithm and also the only one function for which we will need the complex version in other sections of this book. As always we will focus only on the properties that we will need later for practical applications and nothing more!
	
	In the same way that we built the logarithm as being by definition by the inverse function of the natural exponential $e^x$ in the section of Functional Analysis, we first start from:
	
	where $z$ is a complex number and we will define the complex logarithm that must be reduced to the natural logarithm if $z$ has no imaginary part!
	
	So by definition the complex logarithm will be:
	
	and in this entire book, the complex logarithm will be differentiated by the real logarithm by a capital L for the first letter!
	
	Let us write $z$ and $w$ in the Euler form as viewed in the section Numbers:
	
	Then we have:
	
	By correspondence, we find immediately
	
	with $k \in \mathbb{Z}$. Therefore we get:
	
	Therefore:
	
	or more explicitly:
	
	So if $w$ has no imaginary part, we fall back on our feet since $\text{arg} (w)$ becomes zero.
	
	A big difference is highlighted between the logarithm of the complex and real numbers: the complex numbers logarithms can take several values because of the argument!!
	
	For a function to have an inverse, it must map distinct values to distinct values, i.e., be injective. But the complex exponential function is not injective, because $e^{z+2\pi \mathrm{i} }= e^z$ for any $z$, since adding $\mathrm{i}\theta$ to $w$ has the effect of rotating $e^z$ counter-clockwise $\theta$ radians. So all the points of the form $z+\mathrm{i}k\theta$  are all mapped to the same number by the exponential function. So the exponential function does not have an inverse function in the standard sense.
	
	There are two solutions to this problem.
	
	\begin{enumerate}
		\item One is to restrict the domain of the exponential function to a region that does not contain any two numbers differing by an integer multiple of $2\pi \mathrm{i}$: this leads naturally to the definition of "\NewTerm{branches}\index{branches}" of $\text{Log}(w)$, which are certain functions that single out one logarithm of each number in their domains.
		
		\item Another way to resolve the indeterminacy is to view the logarithm as a function whose domain is not a region in the complex plane, but a "Riemann surface" that covers the punctured complex plane in an infinite-to-$1$ way.
	\end{enumerate}
	Branches have the advantage that they can be evaluated at complex numbers. On the other hand, the function on the Riemann surface is elegant in that it packages together all branches of $\text{Log}(w)$ and does not require an arbitrary choice as part of its definition.
	
	We can see this with Maple 4.00b easily:
	
	\texttt{>plot3d([r*cos(f),r*sin(f),f],r=0..1,f=-2*Pi..2*Pi,axes=boxed,style=patch,\\
	shading=ZHUE);}
	
	which gives:
	\begin{figure}[H]
		\begin{center}
			\includegraphics{img/analysis/complex_logarithm.jpg}
		\end{center}	
		\caption{Complex Logarithm plot with Maple 4.00b}
	\end{figure}
	
	For this reason, one cannot always apply $\text{Log}$ to both sides of an identity $e^{z_1}=e^{z_2}$ to deduce $z_1=z_2$ . Also, the identity $\text{Log} (z_1z_2)= \text{Log}(z_1) + \text{Log}(z_2)$ can fail: the two sides can differ by an integer multiple of $2\pi \mathrm{i}$.
	
	For each non-zero complex number $w = x + i\mathrm{y}$, the principal value $\text{Log}(w)$ is the logarithm whose imaginary part lies in the interval $[-\pi,+\pi]$. The expression $\text{Log}(0)$ is left undefined since there is no complex number $z$ satisfying $e^z = 0$.
	
	Then the principal value of the complex logarithm can be defined by (\SeeChapter{see section Trigonometry page \pageref{trigonometry}}):
	
	We see also obviously that the function $\text{Log}(w)$ is discontinuous at each negative real number (we can see it on the figure above), but continuous everywhere else in $\mathbb{C}^*$.
	
	Riemann surfaces\index{riemann surfaces} can be thought of as deformed versions of the complex plane: locally near every point they look like patches of the complex plane, but the global topology can be quite different. For example, they can look like a sphere or a torus or a couple of sheets glued together.
	
	The main point of Riemann surfaces is that holomorphic functions may be defined between them. Riemann surfaces are nowadays considered the natural setting for studying the global behaviour of these functions, especially multi-valued functions such as the square root and other algebraic functions, or the logarithm.
	
	Basically, a Riemann surface is simply just a surface, as far as the shape is concerned. Any normal surface you can think of (for example, plane, sphere, torus, ...) are all Riemann surfaces. We say it's Riemann surface, is due to the context, is that we define the surface using complex functions, and for use in studying complex functions.
	
	\subsection{Complex Integral Calculus}
	We have seen just above how to check if a complex function $f (z)$ was differentiable (it must at least respect the Cauchy-Riemann equations) at any point.
	
	Now let us see the opposite case... the integration that is absolutely fascinating in complex plane!
	
	We have obviously taking again the notations of the section of Differential and Integral Calculus:
	
	either in explicit form:
	
	Well once this expression established, let us give a little explanation about how to read it:
	\begin{enumerate}
		\item We know that $u$ and $v$ are dependent both in the general case of $x$ and $y$.
		
		\item We know that $ u $ and $ v $ represent (see examples at the beginning of this section) closed or open curves and also straight lines when $ x $ (or respectively $y$) is fixed and that the other associated variable varies!
	\end{enumerate}
	So each term have an integral in the above expression is in fact a line integral on a family of open or closed curves (including a specific case that is straight lines...)!
	
	This integral can be evaluated using the Green's theorem in the plane (\SeeChapter{see section Vector Calculus page \pageref{green theorem}}) if we consider the particular case of a closed curvilinear path such as:
	
	Let us first study the real part:
	
	Indeed we proved (it is strongly advised to read again this Green's theorem) in the section of Vector Calculus that:
	
	What will be written in our situation:
	
	However, if the function is holomorphic and thus satisfies the Cauchy-Riemann equations we get immediately:
	
	Thus our integral is reduced in the particular case of a closed path:
	
	and... reusing Green's theorem for this imaginary part:
	
	However, if the function is holomorphic (for reminder that is to say differentiable at every point of the complex plane or an open subset of it) and thus satisfies the Cauchy-Riemann equations we get immediately:
	
	and we thus obtain the "\NewTerm{Cauchy theorem}\index{Cauchy theorem}", or "\NewTerm{Cauchy-Goursat theorem}\index{Cauchy-Goursat theorem}" for its generalized version for non continuous functions, which says that if a function is holomorphic (thus satisfying the Cauchy-Riemann equations) and integrated on a closed contour then:
	
	As a corollary (without proof), any function that satisfies the above relation is holomorphic (in the whole complex plane or an open subset of it).
	
	This result gives the possibility in certain fields like quantum physics fields (we think of the Yukawa potential that is not yet treated in this book in detailed) to calculate complicated real definite integrals using the above property. The idea is when choosing the closed contour of the path integral to play to make the real definite integral only as a part only of the path (by generalizing to the complex case) and by equality with zero we deduce its value thanks to the other parts of the integrals of the chosen path (parts that are obviously simple to calculate).
	
	In other words, the idea is to calculate by difference! The difficulty residing in practice in finding the function $f (z)$ and the closed contour that permits to make appear the function $f (x) $ of the researched definite integral...
	
	Using this result, let us make a very important academic example which will be useful later (but who has no connection with the case of calculating a real definite integral).
	\begin{tcolorbox}[colframe=black,colback=white,sharp corners]
	\textbf{{\Large \ding{45}}Example:}\\\\
	Let us calculate:
	
	For this purpose, we will use the simplification that consist to remember (\SeeChapter{see section Numbers page \pageref{euler formula}}) that:
	
	Therefore:
	
	We can then write the path integral as:
	
	Or as on a closed path differentiable at any point (without nodes) the angle to make a full turn will necessarily be between $0$ and $+\pi$. It comes then:
	
	\end{tcolorbox}
	Before we continue by noticing a very interesting and important fact that we will detail later formally: An integral (we do not speak of primitive but of integral!) of the type $1/x$ in $\mathbb{R}$ would not be calculable. But now if we generalize the concept of $\mathbb{C}$, we see that we get go around the singularity via a path integral that enclose the singularity. And ... and ... in our previous calculation $z$ might have only the real value and not the imaginary one (so $z$ reduce to $x$). So the integral of $1/x$ becomes calculable and has a result in the set of complex numbers which is remarkable!
	
	Some mathematicians interpret this by figuring that $1/x$ is a flat projection of a three-dimensional space in which the imaginary axis is perpendicular to the plane $\mathbb{R}^2$ (see figures below). Hence the fact that $1/x$ can integrated in the set $\mathbb{C}$.
	
	Finally, let us indicate that $1/z$ is holomorphic on the whole complex plane except on $0$ (the derivative being the same as $1/x$). Then the function $1/z$ is thus not $\mathbb{C}$-differentiable!
	
	This being done, let us do an important and similar case with the following path integral:
	
	where $z_0$ is a constant complex number. Let us write:
	
	We can then write if we make one turn counter clockwise:
	
	which is valid only if our integration path avoids $z_0$ what otherwise there is a singularity. This latter integral is a little simplistic generalization of the previous one.
	
	Now let us show the important theorem that interests us since the beginning of this section using many proven results so far!
	
	We know that if a function $f (z)$ satisfies the Cauchy-Riemann equations, the if we carefully avoid the value $z_0$ (as in the above calculations), the expression:
	
	is differentiable at all points except in on $z_0$ (where the expression is no longer holomorph) is name a "\NewTerm{singularity}\index{singularity}".

	Indeed, take a holomorphic function $f (z)$ satisfying Cauchy-Riemann equation and subtract a constant ($f(z_0)$) does not change the fact that the expression (in this case the numerator in previous relationship) remain holomorphic. Finally, multiply it by a fraction (the denominator of the above equation) which is also holomorph gives a holomorphic function. But singularities can then appear, we then speak of "\NewTerm{meromorphic functions}\index{meromorphic functions}" (this is the ratio of two holomorphic functions).
	\begin{tcolorbox}[title=Remark,colframe=black,arc=10pt]
	A meromorphic function is a function holomorphic in the whole complex plane, except possibly on a set of isolated points each of which is a pole (singularity) for the function (see further below for the concept of pole/singularity). The gamma function (see the plot in the Differential and Integral calculus section) is a famous example of meromorphic function!
	\end{tcolorbox}	
	So if we take the path integral on a closed path avoiding $z_0$, the Cauchy theorem gives us immediately (remember the proof above):
	
	However, this can also be written after rearrangement of terms:
	
	Therefore:
	
	But we have proved above that:
	
	Then we get the result named "\NewTerm{Cauchy's integral theorem}\index{Cauchy's integral theorem}", or more rarely "\NewTerm{Cauchy formula}\index{Cauchy formula}" (of which there is a generalized result we will prove later below):
	
	In fact, in practice all the subtlety is to be able to take back a given holomorphic function $g(z)$ (which therefore satisfies the Cauchy-Riemann equations) by manipulating it in a form of the type:
	
	when its possible... then the calculation of its path integral (closed path) becomes extremely simple since it will be equal to:
	
	by the Cauchy's integral theorem!
	
	\begin{tcolorbox}[title=Remark,colframe=black,arc=10pt]
	So we know how to calculate the value of a path integral of an expression that is not holomorph but for which the numerator is holomorph! 
	\end{tcolorbox}
	
	\begin{tcolorbox}[colback=red!5,borderline={1mm}{2mm}{red!5},arc=0mm,boxrule=0pt]
	\bcbombe Caution! The sign of the value of a path integral will depend on the direction in which its integration path will be done. If the direction is straightforward (that is to say "counter-clockwise") its sign will be positive; if on the contrary the direction is clockwise his sign will be negative. You probably think that this information is irrelevant since this value is usually zero. Yes... it is, but we will see later the importance of this information when referring to the calculation of what we name the "residuals".
	\end{tcolorbox}
		
	\begin{tcolorbox}[colframe=black,colback=white,sharp corners]
	\textbf{{\Large \ding{45}}Example:}\\\\
	An important application example is named the "\NewTerm{Gauss' mean value theorem}\index{Gauss' mean value theorem}" that states given $f(z)$ an analytic function on and inside a disk $C_r:|z-z_0|=r$, then:
	
	Indeed, from the Cauchy's integral relation we have (we simply make the change of variable $z=z_0+re^{\mathrm{i}\theta}$):
	
	\end{tcolorbox}
	 There is a similar relation for the derivative $f'(z_0)$ to that given by the Cauchy's integral theorem. Let us see this:
	
	Therefore:
	
	thereby continuing, we have:
	
	In short, we therefore note that:
	
	which is the "\NewTerm{Generalized Cauchy's integral theorem}\index{Generalized Cauchy's integral theorem}\index{Cauchy's integral}".
	
	This result is very powerful because it shows that holomorphic functions are infinitely differentiable (because of the denominator), that is to say analytical, and it is much more difficult to find an equivalent theorem with such simple conditions for real functions.
	
	If we now return to our Taylor expansion of a complex function:
	
	um ... and what do we see here? Well this !:
	
	It follows the following relation named "\NewTerm{Laurent series in positive powers}\index{Laurent series in positive powers}" (there is a more generalized version will be prove later below):
	
	that gives us the formal expression of a complex function in the form of infinite series of integer powers near a point $z_0$ of the complex plane with therefore:
	
	Remembering that $d^{n}f(z_0)/\mathrm{d}z^n$ can be written equivalently $f^n(z_0)$, we see that all the two previous relations gives us the Taylor series expansion that we had obtained in real analysis (\SeeChapter{see section Sequences and Series page \pageref{taylor series}}) and that was:
	
	Thus, the Taylor series in $\mathbb{R}$ are a special case of Laurent series that are in $\mathbb{C}$!!!
	
	This result is quite remarkable because it also shows that we can use the path integral in the complex plane for calculating the coefficients $c_n$ of the Laurent series instead of calculating the derivatives of order $n$ of the function $f$ if these latter are too complicated to determine. Or vice versa... calculate a simple derivation instead of calculating a headache type path integral (typically the case in physics) using the fact that:
	
	The only unfortunate point being that the latter relation is calculable only if we can put the function in path line integral in the form:
	
	where $n$ is a positive or null integer. This is honestly far from to be easy in most cases! The idea would be to find a general path for line integral, valid for any function $f (z)$ such that the denominator (which additionally contains a singularity on $z_0$) disappears. That would be ideal ... but we need a track ... and it will come from the study of the convergence of series of complex powers. Let's see what it is with a qualitative approach!
	
	\pagebreak
	\subsubsection{Convergence of a complex series}
	We saw in the section of Sequences and Series that many real functions could be expressed in Maclaurin series (special case of the Taylor series on $x_0=0$) in the form:
	
	We also showed, by example the only, that this series expansion of infinite powers was valid for some real functions only in a certain domain of definition named "radius of convergence".
	
	Even if this radius of convergence can be determined more or less easily in each case, there are some baffling examples that could not in the early 19th century be understood without complex analysis.
	
	Let's see a simple example to understand what kind of problem it is. Consider for this the two functions:
	
	and before continuing our example, recall that we have proved in the section of Sequences and Series the relation:
	
	relative to a geometric series, that is to say a series whose terms are of the type:
	
	Therefore it comes immediately if $n \rightarrow +\infty$ and $q \in ]-1,+1[$:
	
	if $u_0=1$, we get:
	
	So if we change the notation, we have\label{sum of powers}
	
	Then it comes immediately:
	
	Therefore the two previous functions $g(x)$ and $h(x)$ are defined for a infinite series expansion in powers only in radius of convergence $x \in ]-1,+1[$.
	
	We would get the same result by making a Maclaurin series expansion!
	
	We see that there is trivially for $g(x)$ two singularities that are $x=\left\lbrace -1,+1\right\rbrace$ by cons, basically we do not see trivial singularities for $h (x)$ if we reason only in $\mathbb{F}$ so it can be hard for the latter function to understand the origin of the radius of convergence!
	
	Indeed, if we plot these two functions in $\mathbb{R}$ with Maple 4.00b we get respectively:
	\begin{figure}[H]
		\begin{center}
			\includegraphics{img/analysis/g_h_example_functions.jpg}
		\end{center}
	\end{figure}
	hence the problem of why there is still implicitly a radius of convergence $x \in ]-1,+1[$ for $h (x)$???
	
	An even more blatantly way to highlight the problem, is to show the approach of these two functions by a Maclaurin series expansion with ten terms.
	
	For $g(x)$ we get for example:
	
	\texttt{>with(plots):\\
	>xplot:= plot(1/(1-x\string^2),x=-5..5,thickness=2,color=red):\\
	>tays:= plots[display](xplot):\\
	>for i from 1 by 2 to 10 do\\
		tpl:= convert(taylor(1/(1-x\string^2), x=0,i),polynom):\\
		tays:= tays,plots[display]([xplot,plot(tpl,x=-5..5,y=-2..2,\\
		color=black,title=convert(tpl,string))])\\
		od:\\
	>plots[display]([tays],view=[-5..5,-2..2]);}
	
	\begin{figure}[H]
		\begin{center}
			\includegraphics{img/analysis/g_function_expansion_inspection.jpg}
		\end{center}	
		\caption[]{Plane representation of the function $g$ to visualize the problem}
	\end{figure}
	where we see well that the Maclaurin series (or expression in power series) does not converge outside $x \in ]-1,+1[$  which can be intuitive because of both singularities.
	
	For $h(x)$ we have by cons:
	
	\texttt{
	> with(plots):\\
	> xplot:= plot(1/(1+x\string^2),x=-5..5,thickness=2,color=red):\\
	> tays:= plots[display](xplot):\\
	> for i from 1 by 2 to 10 do\\
		tpl:= convert(taylor(1/(1+x\string^2), x=0,i),polynom):\\
		tays:= tays,plots[display]([xplot,plot(tpl,x=-5..5,y=-2..2,\\
		color=black,title=convert(tpl,string))])\\
	od:\\
	> plots[display]([tays],view=[-5..5,-2..2]);\\
	}
	
	\begin{figure}[H]
		\begin{center}
			\includegraphics{img/analysis/h_function_expansion_inspection.jpg}
		\end{center}	
		\caption[Divergent Maclaurin series]{Surprisingly, here the Maclaurin series (in black) does not converge}
	\end{figure}
	where we see well that the Maclaurin series (or the expression in power series) does not converge either outside $x \in ]-1,+1[$ which was unsettling and against-intuitive at the beginning of the history of real analysis.
	
	Today even a high school student knows that he can also think in $\mathbb{C}$ and that $\mathbb{R} \subset \mathbb{C}$. So the real analysis is just a special case and restricted of the field of complex analysis. The fact to extend the domain of a given analytic function is named "\NewTerm{analytic continuation}\index{analytic continuation}". As we will see just now analytic continuation often succeeds in defining further values of a function, for example in a new region where an infinite series representation in terms of which it is initially defined becomes divergent!
	
	The singularity for $h (x)$ in $\mathbb{C}$ comes that latter is then  written:
	
	and there are therefore two singularities $z=\left\lbrace{-\mathrm{i},+\mathrm{i} }\right\rbrace$ that we see well if we represent:
	
	with Maple 4.00b (fortunately we now have the equivalent of a microscope in mathematics with Maple...):
	
	\texttt{>plot3d(abs(1/(1+(re+I*im)\string^2)),re=-3..3,im=-3..3,view=[-2..2,-2..2,-2..2]\\
	,orientation=[-130,70],contours=50,style=PATCHCONTOUR,axes=frame,\\
	grid=[100,100],numpoints=10000);}
	
	\begin{figure}[H]
		\begin{center}
			\includegraphics{img/analysis/g_inspection_in_C.jpg}
		\end{center}	
		\caption{Complex representation of the function $h$ to highlight the reason for the divergence}
	\end{figure}
	where we can see the two singularities on the imaginary axis and the function $h (x)$ on the real axis (between the two peaks). So when we develop a function in power series, we conclude that the radius of convergence is defined by the whole complex plane and not by the traditional axis of the real analysis.
	
	This makes it more natural to understand why we were talking in section of Sequences and Series of "radius" as seen from above, we have in the complex plane:
	\begin{figure}[H]
		\begin{center}
			\includegraphics{img/analysis/h_various_radius_convergence.jpg}
		\end{center}	
		\caption{Representation of the various convergence of radius of $h(z)$}
	\end{figure}
	hence the fact that we are talking sometimes about (open) convergence disk and sometimes of (open) convergence radius. Moreover, we notice on the chart that the domain of convergence is convex (any couple of points of the domain can be connected by a straight line that is in the area of convergence).
	\begin{tcolorbox}[title=Remark,colframe=black,arc=10pt]
	Let us Recall that a subset, interval or "open" disc means that we do not take its border as we have seen in the section Topology.
	\end{tcolorbox}
	Then we understand better why the Taylor series does not converge trivially for $h(x)$: it must converge on the whole disc of the complex plane and not just converge on the real axis!
	
	From all this we deduce that our Laurent series in positive powers proved above:
	
	not necessarily converge, unsurprisingly... on the whole complex plane (just like the Taylor series on the real line as this is the equivalent!) but sometimes only in a opened subdomain (convex?) of this plane around $z_0$ (which in the particular example taken above was obviously: $0$).
	
	With our function $h(x)$ expressed using a development of Maclaurin with 5 terms, we see immediately with Maple 4.00b that on the borders of the square inscribed in the disc of convergence, the series does not converge and we're guessing the start of the two singularities:
	
	\texttt{>plot3d(abs(1-(re+I*im)\string^2+(re+I*im)\string^4-(re+I*im)\string^6+(re+I*im)\string^8),\\
	re=-0.7..0.7,im=-0.7..0.7,view=[-1.5..1.5,-1.5..1.5,0..1.5]\\
	,orientation=[-130,70],contours=50,style=PATCHCONTOUR,axes=frame,\\
	grid=[100,100],numpoints=10000);}
	
	\begin{figure}[H]
		\begin{center}
			\includegraphics{img/analysis/h_zoom_on_complex_representation.jpg}
		\end{center}	
		\caption{Focus on the complex representation to understand the reason for the divergence}
	\end{figure}
	a little outside the disc of convergence, we obviously have a little bit nonsense:
	
	\texttt{>plot3d(abs(1-(re+I*im)\string^2+(re+I*im)\string^4-(re+I*im)\string^6+(re+I*im)\string^8),re=-3..3,\\
im=-3..3,view=[-1.5..1.5,-1.5..1.5,0..1.5],orientation=[-130,70]\\
	,contours=50,style=PATCHCONTOUR,axes=frame,grid=[100,100],numpoints=10000);}
	
	\begin{figure}[H]
		\begin{center}
			\includegraphics{img/analysis/h_divergence.jpg}
		\end{center}	
		\caption[]{This diverges ... (stalactites ???)}
	\end{figure}
	There is still something interesting to try ... since we are now on a plane, not a straight line right (axis), it is possible for us to make the Taylor expansion around a singularity $z_0$ by deforming the disk in a convex crown/ring simply connected as shown below (the crown/ring being the simplest simply convex geometry arising from the deformation of a disk):
	\begin{figure}[H]
		\begin{center}
			\includegraphics{img/analysis/h_representation_transformation_disc_in_crow.jpg}
		\end{center}	
		\caption{Representation of the deformation of a disc in a crown/ring}
	\end{figure}
	The advantage of this is to deform the area of convergence on the whole complex plane by avoiding (bypassing) all the singularities. Thus, unlike the Taylor series that are only valid on an interval of the $x$-axis, we would have a new type of series describing a function absolutely everywhere, that is to say before AND after (so around...) singularities!
	
	So obviously we will require that in the deformed crown above the function is always holomorph and analytical (as in the initial convex disc). Before determining what we are going down (generalized Laurent series!), we must first do a study of the decomposition of path integral:
	
	\pagebreak
	\subsection{Path Decomposition}
	The path integrals as given previously can also be written in another form almost classical and used many times in the literature.
	
	Let us see this. First, remember that we have just proved in the special case of a holomorphic function that:
	
	But a closed path can be seen as a path having a round trip:
	\begin{figure}[H]
		\begin{center}
			\includegraphics{img/analysis/closed_path.jpg}
		\end{center}	
		\caption{Representation of a closed path with round trip}
	\end{figure}
	Therefore we can write:
	
	And now comes what interest us... for this purpose let us focus one  of the path integral of the type:
	
	We already well known (1st form of notation) that any complex number $z$ of the type:
	
	can be (2nd form of notation) written as (Euler form):
	
	and to integrate on a path, nothing prevent us to choose a path where $r$ (the module) would be fixed and $\theta$ variable (we could not have the possibility to do this with the 1st form because by modifying the imaginary or real part, we can't get be guarantee to get a nice smooth curve but this is possible with the Euler form of a complex number)!
	
	Therefore we have:
	
	We write then naturally:
	
	and as:
	
	Therefore:
	
	That we often find in the following form in the literature:
	
	
	
	\subsubsection{Inverse Path}
	If $C$ is a curve going from a point $P$ to a point $Q$, then we denote by $C^-$  the same curve but travelled from $Q$ to $P$.
	
	Let us parametrized $C^-$:
	
	If $C(t)$ it is the curve defined on $[a, b]$, then we define the curve $C^-(t)$ on $[a, b]$ by:
	
	Indeed we have with this parametrization:
	
	and when $t$ increases from $a$ to $b$, $a + b - t$ decreases from $b$ to $a$. $C^-$ is therefore only $C$ but travelled in the opposite direction.
	
	We then have using the last proof:
	
	Let us put:
	
	Therefore:
	
	Then we have:
	
	Therefore if $C^-$ and $C$ are the paths of the same function but travel in the opposite direction, we have by taking our conventional notation (caution! In the second term it is implicit that the parametrization is different from the first one!):
	
	Therefore:
	
	this is why we often say that the sign of the value of a line integral will depend on the direction in which its integration path is travel. If the direction is straightforward (that is to say "counter-clockwise") its sign will be positive; if on the contrary the direction is clockwise its sign will be negative (\SeeChapter{see section Differential and Integral Calculus page \pageref{closed path orientation}}).
	
	\pagebreak
	\subsection{Laurent Series}
	This last relation obtained, we can return to the deformation of our disc of convergence in a crown. We recall that initially the idea is to have the analytical expression of a function as an infinite series of powers in a limited area around a singularity point and all this... in the purpose to be able to calculate for physicists complex path integrals through a method using the properties of complex series!
	
	Let's start with the point (2) that is to say have an infinite power series for a path integral, which will take us more easily to point (1) that is to say get the an analytical expression of a function around a singularity point, by zooming on our crown:
	
	\begin{figure}[H]
		\begin{center}
			\includegraphics{img/analysis/crown.jpg}
		\end{center}	
		\caption[]{Zoom on our crown from our starting example}
	\end{figure}
	We therefore have if the function $f$ is analytic and holomorphic in the crown of outer radius $R$ and inside radius $r$, the following path curvilinear integral in the crown as we proved above (we change notation: $z=z'$ and $z_0=z$):
	
	
	therefore we denote now by $z$ the point where we want to know the function and $z'$ variable of which $f$ depends. This notation change will be justified later for a purely practical reason.
	
	The crown can be broken down into four paths:
	
	If both segments $C_c$ and $-C_c$ are infinitely close, they then correspond to the same path travelled once in a positive direction and once in the negative direction. As we have proved just above that:
	
	It therefore follows that:
	
	Which brings us to write:
	
	where we have put a "+" between the last two terms, because as we shall see immediately, the convergence criterion associated with the traditional notation in this field of study, makes automatically emerge the sign "-".
	
	For the two integral $f_1,f_2$, we know that the fraction can be written as a geometric series as already seen above. Effectively, starting from (now you will understand why we changed the notation):
	
	by assimilating:
	
	where as we have seen, the convergence requires that:
	
	so that $x$ is lower in absolute value to $1$.
	
	We then see the infinite geometric series appearing:
	
	Therefore:
	
	To come back to:
	
	we have in any point $z$ inside the circle of radius $R$ whose border is described by the variable $z'$ and of center $z_0$ the convergence that is assured because:
	
	Then we can write:
	
	Integrating term by term integration, we highlight the development (already known):
	
	with the definition of coefficients $c_n$, where $n$ is a positive or null integer:
	
	This development may do think to the development of Taylor in the sense that only positive (or zero) powers of $(z'-z_0)$ appear, but this is not Taylor development in the case of the crown! Indeed, $c_n$ can not be written this time as:
	
	since, by assumption, $f(z)$ is assumed analytic in the crown only and may therefore very well not be inside the small circle of radius $r$, in particular on $z_0$, in which case $f^{n}(z_0)$ may simply not exist (let us repeat that $z$ is strictly constrained to be in the crown, therefore $r<\vert z \vert <R$). We will see later what happens when $f (z)$ is holomorphic in this disk and that, in particular, $z_0$ is not a singular point.
	
	We still need to treat $f_2$. We then do the same type of development as for $f_1$, with the difference that now:
	
	when $z'$ browse the small circle of radius $r$. To make a geometric series appear, we must write this time:
	
	Therefore:
	
	So we have:
	
	Integrating term by term, we highlight the (new) development:
	
	with:
	
	By changing $n$ in $-n$ in the summation for $f_2$, we have for the sum $f_1(z)+f_2(z)$:
	
	with at this time two distinct $c_n$:
	
	We will now see that these two relations can be combined into one!
	
	For this purpose if we observe well the last two relations, we find that they do not depend at all of $z$ (!) and this is normal since the $c_n$ are the coefficients of the series expansion of $f(z)$ and these are the same at any point of the domain of definition of the function where it is analytic!
	
	So the two contours (circles) can be merged into only one circle since it is located in the crown and has for center $z_0$:
	
	Furthermore, the attentive reader will have noticed that this contour does not even need to be a circle finally! It may be any geometry as long as it is closed and is located in an analytical area!
	
	Thus, we get the two relations:
	
	The two previous relation define the "\NewTerm{general Laurent series}\index{general Laurent series}". It is remarkable and differs from a Taylor series in the sense that it contains all the positive and negative integer powers and the coefficients $c_n$ can a priori not be expressed with the derivatives of $f$.
	
	The power series of $n\geq 0$ is named "\NewTerm{regular part}\index{regular part of a power series}", the negative powers is commonly named "\NewTerm{main part}\index{main part of a power series}".
	
	The series of negative powers converges uniformly everywhere outside $\gamma_r$, that of positive powers within $\gamma_R$. In total the development of Laurent converges uniformly in the common area, which is the crown and therefore also on the unique path $\gamma$.
	
	Let us now show a point that we have mentioned above. If the circle contains no singularity, then all the coefficients:
	
	are zero. First note that $-n-1$ is a positive or zero integer, which we will denote by $p$ such as:
	
	We then have the following integrand along a closed path:
	
	But, if we remove the singularity that requires $f(z')$ is holomorphic (and in anyway this is required by all initial developments of the Laurent series).
	
	As $(z'-z_0)^p$ is polynomial with positive and not null integer powers and that as we know any polynomial satisfying these conditions is differentiable at least once without showing singularity. Thus this term is also holomorphic.
	
	Assuming that the product of two holomorphic functions is holomorphic and that contour $\gamma$ is closed, then we have using the following result proved above (for a holomorphic function):
	
	the following immediate consequence:
	
	if there is no singularity in the small circle of the crown. We then fall back again on a development with only positive powers, the $c_{n\geq 0}$ this time being equal to:
	
	according to the generalized Cauchy's integral theorem proved earlier above. Conversely, we see well that this is the main part (when it exists!) which contains the information on the fact that $f$ is not a priori holomorphic in the small disk. The existence of negative powers shows that $f$ is clearly not bounded on $z_0$.
	
	The classification of singularities of a function will be precisely based on the consideration of the characteristics of the main part of the Laurent  development centered on a singular point of this function.
	\begin{tcolorbox}[colframe=black,colback=white,sharp corners]
	\textbf{{\Large \ding{45}}Example:}\\\\
	Let us see to what looks like the Laurent series of our famous example function:
	
	on a simply convex domain that would be the crown rounding the singularity  $\mathrm{i}$ for example (we could have taken the second singularity $-\mathrm{i}$ but we had to choose one to not repeat twice the explanations below...). This is equivalent therefore to search the power series development of $z-\mathrm{i}$. \\
	
	We will proceed as following:
	
	For what will follow we will use:
	
	\end{tcolorbox}
	
	\pagebreak
	\begin{tcolorbox}[colframe=black,colback=white,sharp corners]
	The second fraction can be expressed as a geometric series if as we have already seen:
	
	Therefore it comes:
	
	Let us multiply both sides of this equality by $-i / $2 and then divide them by $z - i$ (the second term in the denominator of the original fraction) for obtain the left term:
	
	and for the right term:
	
	Finally we have the following geometric series:
	
	We see then on this Laurent series around $\mathrm{i}$ of the holomorphic function $f(z)$ that the following coefficient appears:
	
	and then we have with Maple 4.00b:\\
	
	\texttt{>plot3d(abs(-I/2*1/((re+I*im)-I)-(I/2)\string^2-(I/2)\string^3*(re-I*im)-\\
	(I/2)\string^4*(re-I*im)\string^2-(I/2)\string^5*(re-I*im)\string^3),\\
	re=-1.5..1.5,im=-1.5..1.5,view=[-2..2,-2..2,-1..2],\\
	orientation=[-130,70],contours=50,style=PATCHCONTOUR,axes=frame,\\
	grid=[100,100],numpoints=10000);}
	\end{tcolorbox}
	
	\pagebreak
	\begin{tcolorbox}[colframe=black,colback=white,sharp corners]
	This gives the following figure:
	\begin{figure}[H]
		\centering
		\includegraphics{img/analysis/laurent_series_representation.jpg}
		\caption{Laurent series representation of $f(z)$ with Maple 4.00b}
	\end{figure}
	where we see that the Laurent series allows us to express $f (z)$ in a neighbourhood close to the singularity $\mathrm{i}$ by taking five terms.\\
	
	Ditto if we make the sum of the two Laurent series for the two singularities with seven terms:\\
	
	\texttt{>plot3d(abs(-I/2*1/((re+I*im)-I)-(I/2)\string^2-(I/2)\string^3*(re-I*im)-\\
	(I/2)\string^4*(re-I*im)\string^2-(I/2)\string^5*(re-I*im)\string^3 -(I/2)\string^6*(re-I*im)\string^4\\
	-(I/2)\string^7*(re-I*im)\string^5+I/2*1/((re+I*im)+I)+(I/2)\string^2+
(I/2)\string^3*(re+I*im)+(I/2)\string^4*(re+I*im)\string^2+(I/2)\string^5*(re+I*im)\string^3\\
	+(I/2)\string^6*(re+I*im)\string^4
+(I/2)\string^7*(re+I*im)\string^5),re=-1.5..1.5, im=-1.5..1.5, view=[-2..2,-2..2,-1..2],orientation=[130,70], contours=50, style=PATCHCONTOUR, axes=frame,grid=[100,100],\\
numpoints=10000);}\\

	This gives the image visible on the next page:
	\end{tcolorbox}
	
	\pagebreak
	\begin{tcolorbox}[colframe=black,colback=white,sharp corners]
	\begin{figure}[H]
		\centering
		\includegraphics{img/analysis/sum_of_two_laurent_series.jpg}
		\caption{Sum of the two Laurent series of $f(z)$ for both singularities with Maple 4.00b}
	\end{figure}
	\end{tcolorbox}
	
	\subsection{Singularities}
	We have seen just before that it was possible to calculate the path integral of a function, on condition of analyticity, on the outline of a singularity. Our goal will now be to enhance this approach.
	
	We have already mentioned and highlighted in our previous proofs that the integrant in the "Cauchy's integral theorem" was of the form:
	
	where $f(z)$ is well defined in $z_0$.
	
	The point $z-z_0$ is of course a singularity of $f(z)$ and it is not defined there.
	
	As we saw during our proof of Laurent series, $f(z)$ can be expressed as a Laurent series in the form a positive power Laurent series in a convergence disk (or what remains the same: as a series of Laurent in a crown not centered on a singularity...) in the form:
	
	Before continuing, it is customary in mathematics to define a small conventional vocabulary regarding this time the possible singularities of $f(z)$!
	
	Let us first recall that we know, and that we have proved, that all information on the singularities of $f (z)$ are contained in the main part of the Laurent series (negative powers) defined on the crown surrounding $z_0$:
	
	The following classification focus on "\NewTerm{isolated singularities}\index{isolated singularities}", that is to say, a singular point where $f(z)$ is analytic everywhere in the neighbourhood excepted on $z_0$. This classification, as we will see permits us to distinguish three types of singular points, will be useful when developing the theory of residues further.
	
	\textbf{Definitions (\#\mydef):}
	\begin{enumerate}
		\item[D1.] When the limit of the function $\vert f(z) \vert$ exists on $z_0$, we say that the singularity is a "\NewTerm{removable singular point}\index{removable singular point}" or "\NewTerm{apparent singularity}\index{apparent singularity}".
		
		For example:
		
		does not seem to be defined on $z=z_0=0$ but we have a numerator having a Laurent series without negative powers (therefore a simple Taylor series). It then comes into by doing the Maclaurin series (that is to say the Taylor series on $z=z_0=0$...):
		
		We then see that $f (z)$ finally has no term with negative power and therefore we have eliminated the singularity (or that it contains simply no singularities... which can easily be check with Maple 4.00b).
		
		\item[D2.] When on $z_0$ the limit $\vert f(z) \vert $ does not exist we speak about "\NewTerm{essential singularity}".
		
		For example, $z_0=0$ is an essential singularity for the function:
		
		Indeed, if $z$ approaches zero coming from the positive real axis $\mathbb{R}_+$, the function diverges, more precisely, it tends to $+\infty$. If $z$ comes from $\mathbb{R}_-$, the function tends to zero as illustrated by the following Maple 4.00b plot:
		
		
		\texttt{>plot3d(abs(exp(1/(re+I*im))),re=-5..5,im=-5..5,\\
		view=[-3..3,-3..3,-0.5..3],orientation=[-130,70],contours=50,\\
		style=PATCHCONTOUR,axes=frame,grid=[100,100],numpoints=10000)\\
		>plot3d(abs(exp(1/(re+I*im))),re=-5..5,im=-5..5,\\
		view=[-3..3,-3..3,-0.5..3],orientation=[-130,70],contours=50,\\
		style=PATCHCONTOUR,axes=frame,grid=[100,100],numpoints=10000)}
		\begin{figure}[H]
			\centering
			\includegraphics{img/analysis/plot_essential_singularity.jpg}
			\caption{Essential singularity example with $e^{1/z}$ in Maple 4.00b}
		\end{figure}
		Indeed:
		
		So an equivalent way of defining an essential singularity, is to say that there are an infinite number of terms with negative powers in the main part of the Laurent series.
		
		\item[D3.] When on $z_0$ the limit of $\vert f(z) \vert$ is $+\infty$, we speak about a "\NewTerm{pole}\label{pole}".
		
		This is the last category (as far as we know...) in which we can store a function that is not classifiable neither in the first nor in the second definition above.
		
		So another equivalent way of defining a "pole", is to say that there is a finite number of terms with negative powers in the main part of the Laurent series. If the number of terms is $k$, then we speak of "\NewTerm{pole of order $k$}".
		
	\begin{tcolorbox}[title=Remarks,colframe=black,arc=10pt]
	\textbf{R1.} We sometimes say that an essential singularity is a " \NewTerm{pole of order $+\infty$}".\\
	
	\textbf{R2.} A pole of order 1 is named a "\NewTerm{simple pole}". One of order 2 is named a "\NewTerm{double pole}" and so on...
	\end{tcolorbox}
	\end{enumerate}
	If we come back on our example:
	
	We have proved previously that the Laurent series of the function was:
	
	This function has therefore a trivial pole of order $1$ on $z_0=\mathrm{i}$ and also on $z_0=-1$ because in this latter case this infinite series diverge to $+\infty$ and we can easily check this with the following Maple 17.00 command:
	
	\texttt{>sum(-(I*(1/2))\string^n*(-2*I)\string^(n-2), n = 0 .. infinity)}
	
	\subsection{Residue Theorem}\label{residue theorem}
	Consider a function $f(z)$ whose pole is of order less or equal to $k$.
	
	Let us make it  analytic:
	
	that is to say that we have take a function $f(z)$ that we have made analytic after elimination of the poles supposed in finite number - order - less than or equal to $k$ on $z_0$. 
	
	This function $\phi(z)$ has therefore a Laurent series development  in a disc center on $z_0$.
	
	As we have prove it previously, we can therefore by using the following relation:
	
	write:
	
	Using $f(z)$ under the integral it comes:
	
	You must deeply analyse this relation and understand that it link together the integral of a function having singularities with the value on one point of an analytical function having no more singularities!!!
	
	This latter relation can be rewritten by rearranging terms:
	
	And by expressing $\phi^{(k)}(z_0)$ by using (this is authorized because this latter function is analytical) the fact that by definition:
	
	We get obviously:
	
	Therefore by making $\phi(z)$ explicit again:
	
	This latter relation is valid only for ONE isolated singularity (in case you forget!) and where $k$ is equal at least to $1$!

	Mathematicians therefore define:
	
	as being the residue of the function $f(z)$ at the point being an isolated singularity of order $k$. Or respectively:
	
	where the path integral is centered on $z_0$.
	
	Now notice that the term on the right of the equality in the previous relation correspond to the coefficient $c_{-1}$ of the Laurent series. Indeed:
	
	Therefore:
	
	\begin{tcolorbox}[title=Remark,colframe=black,arc=10pt]
	Therefore it comes that on an isolated singularity that can be eliminated, the residue is null because as we saw it before, the path integral rounding a domain without singularity is equal to zero!
	\end{tcolorbox}
	To resume, the relation:
	
	is very interesting for the physicist... because this is a very elegant way for him to calculate the path integral of a non analytic function $f(z)$ having a unique isolated singularity and this just by knowing the order of its poles!
	
	For example if a function $f(z)$ has only a pole of order $1$, we have therefore:
	
	and we replace therefore $z_0$ by the desired value in the parenthesis $(z-z_0)$ and after we calculate the limit between brackets!
	
	Now to go more fare, let us remind that outline of the path integral:
	
	and the curvilinear path of the integral:
	
	are in fact combined (identical) and the coefficients $c_n$ do not depend on $z$! The only constraint on the path is that is closed and in an analytical domain centered on one point.
	
	So if we have several isolated singularities, surrounded by connected curvilinear paths as shown below on the complex plane of a function having a pole of order 3 (i.e. three non-removable singularities $z_0,z_1,z_2$):
	\begin{figure}[H]
		\centering
		\includegraphics{img/analysis/multiple_surrounded_singularities.jpg}
		\caption{Multiple isolated singularities surrounded by curvilinear paths}
	\end{figure}
	then we have still only one closed curvilinear path but whose different isolated singularities are connected by cross which as we know: the paths that are opposed  cancel themselves! And let us remind that the coefficients are the same throughout on all the path since it is on an analytical domain.
	
	We then have the generalized version of the residue theorem for a function $f$ with $n$ isolated singularities:
	
	with a rigorous approach that is specific to engineers ... who sometimes write this latter relation as following:
	
	where $r$ is therefore a residue. This is an important result in the field of solving differential equations associated with some inverse Laplace transforms (\SeeChapter{see section Functional Analysis page \pageref{Laplace transform}}). This intermediate result will give us the possibility to get an another one a little further of major importance for the section of Corpuscular Quantum Physics.
	\begin{tcolorbox}[colframe=black,colback=white,sharp corners]
	\textbf{{\Large \ding{45}}Example:}\\\\
	Let us take again our famous function:
	
	We know it has a pole of order $1$ on $z_0=\mathrm{i}$ and a pole of order $1$ on $z_0=-\mathrm{i}$. So if we take this time Laurent series with a path that surrounds the two singularities (and not only one) then we have a function with a pole of order $2$.
	
	It comes then for this particular case:
	
	with $n$ being equal to $2$.
	
	Then we have:
	
	and:
	
	\end{tcolorbox}
	
	\pagebreak
	\begin{tcolorbox}[colframe=black,colback=white,sharp corners]
	We can easily check this with Maple 4.00b:\\
	
	\texttt{>readlib(singular):\\
	>singular(1/(1+z\string^2),z);\\
	>readlib(residue):\\
	>residue(1/(1+z\string^2),z=-I);\\
	>residue(1/(1+z\string^2),z=I);\\}

	and therefore:
	
	In fact in this case, the residue theorem gives zero because the function has no poles to infinity which is true since in our example:
	
	Physicists meanwhile say that... the force does make any work on this path ...!
	\end{tcolorbox}
	
	\subsubsection{Pole at infinity}
	We have say before that any function that did not have poles at infinity had therefore the sum of the residues of all the poles that are equal to zeros. This result is very important in physics and merits to be study!
	
	It is almost trivial to recognize the number of poles... but to recognize the poles that are at infinity there are many times traps we can easily fall in.
	
	Let us consider the expression $f(z)\mathrm{d}z$. If $z$ is at the neighbourhood of the infinity then $1/z$ is near $0$. Let us write:
	
	Then we have:
	
	Then the residue at infinity is such that:
	
	with:
	
	Therefore with:
	
	The latter relation we will be indispensable to us in the section of Corpuscular Quantum Physics Corpuscular to build the relativistic Sommerfeld model of hydrogenous atom because we will need to calculate a path integral with a pole.
	
	Let's see an example with the function that accompanies us since the beginning of this section. That is to say:
	
	Therefore it comes:
	
	And we recognize immediately the initial function, in absolute value, and that has therefore no pole on $0$. Therefore $f(z)$ has no pole at infinity.
	
	\begin{flushright}
	\begin{tabular}{l c}
	\circled{100} & \pbox{20cm}{\score{3}{5} \\ {\tiny 14 votes,  61.43\%}} 
	\end{tabular} 
	\end{flushright}
	
	%to force start on odd page
	\newpage
	\thispagestyle{empty}
	\mbox{}
	\section{Topology}\label{topology}
	\lettrine[lines=4]{\color{BrickRed}T}opology is an extremely broad field of mathematics for which it is difficult to define precisely the object so the areas where it applies are varied (real line topology, graphs topology, differential topology, complex topology, symplectic topology, etc.). 

We mainly make a distinction with:
	\begin{itemize}
		\item "\NewTerm{General topology}" that establishes the foundational aspects of topology and investigates properties of topological spaces and investigates concepts inherent to topological spaces. It includes point-set topology, which is the foundational topology used in all other branches (including topics like compactness and connectedness).
		
		\item "\NewTerm{Algebraic topology"} tries to measure degrees of connectivity using algebraic constructs such as homology and homotopy groups.
		\item "\NewTerm{Differential topology"} is the field dealing with differentiable functions on differentiable manifolds. It is closely related to differential geometry (see section of the same name in the chapter about Geometry page \pageref{differential geometry}) and together they make up the geometric theory of differentiable manifolds.
		\item "\NewTerm{Geometric topology}" primarily studies manifolds and their embeddings (placements) in other manifolds. A particularly active area is low dimensional topology, which studies manifolds of four or fewer dimensions. This includes knot theory, the study of mathematical knots.	
	\end{itemize}
	\begin{tcolorbox}[title=Remark,colframe=black,arc=10pt]
	A "\NewTerm{manifold}\index{manifold}" is a higher dimensional analogue of a curve or a surface. A manifold of dimension $n$ is a space, you can think of a collection of points, that locally looks like $\mathbb{R}^{n}$ for some integer $n$. All curves in the two dimensional plane that do not intersect locally look like part of a line. All smooth surfaces in $\mathbb{R}^3$, that is surfaces that do not have sharp kinks, edges (boundaries), or points all locally look like the two dimensional plane. Thus such surfaces are two dimensional manifolds. One should imagine being able to tear any small piece off the smooth surface, then being able to stretch it, push down any "hills" and push up any "valleys" to end up with a flat piece of the $2$-plane. Therefore a manifold is a space that is locally Euclidean\\
	
	The picture to have in your mind when thinking about manifolds is the relation between a globe and a map. Small pieces of the globe can always be described by maps (pieces of the 2-plane). Moreover, the entire globe can be covered by a collection of maps: an atlas.
	\end{tcolorbox}
	What we can say at first is that in its foundations Topology is very closely related to the set theory, the study of convergence of sequences and series, functional analysis, analysis complex, the differential and integral calculus, vector calculus and the geometry to mention only the most important cases that the reader can already found in this book.

	The origin of Topology comes from the problems that laid to the progress of functional analysis in the rigorous study of continuous functions, their differentiability, their limits at a point (finite or note), the existence of extremums, etc. in higher-dimensional spaces (in fact, implicitly, the goal for topology is to create tools that easily allow to study the properties of functions in all dimensions). All these concepts, needed a rigorous mathematical definition of the intuitive idea of proximity, especially when doing operations on such functions.

	We will try in this section to identify the basis of the structures that allow us to speak about limits and continuity and this only for curiosity as consultants in R\&D and financial engineering we never saw a business application where the subjects below are absolutely necessary to develop a new business or solve a problem. 
	
	The majority of examples that we will take in this section will be in $\mathbb{R}$ (the $\mathbb{R}$ straight line to be more exact...) because it is the most used one by the engineers (and most of times the only one!) and the one we will have to use for the sections on Graph Theory, Statistics, Differential and Integral Calculus and also on Fractals. When we restrict our study of Topology on $\mathbb{R}$ we then speak of "\NewTerm{Real Analysis}".

	\subsection{General Topology}

General topology is the branch of topology dealing with the basic set-theoretic definitions and constructions used in topology. It is the foundation of most other branches of topology, including differential topology, geometric topology, and algebraic topology.

The fundamental concepts in point-set topology are "\NewTerm{continuity}", "\NewTerm{compactness}", and "\NewTerm{connectedness}".  
	\begin{itemize}
		\item Continuous functions take \underline{nearby} points to nearby points.
		
		\item Compact\label{compact} sets are those that can be covered by finitely many sets of \underline{arbitrarily small} size. 
		
		\item Connected sets are sets that cannot be divided into two pieces that are \underline{far apart}. 
	\end{itemize}		
	The words \underline{nearby}, \underline{arbitrarily small}, and \underline{far apart} can all be made precise by using open sets. If we change the definition of open set, we change what continuous functions, compact sets, and connected sets are. Each choice of definition for open set is named a "\NewTerm{topology}". A set with a topology is named a "\NewTerm{topological space}\index{topological space}\label{topological space}".

	"\NewTerm{Metric spaces}\index{metric space}\label{metric space}" are an important class of topological spaces where distances can be assigned a number named a "\NewTerm{metric}\label{metric}". Having a metric simplifies many proofs, and many of the most common topological spaces are metric spaces.
	
	An "\NewTerm{inner product}" (\SeeChapter{see section Vector Calculus page \pageref{inner product}}) induces a "\NewTerm{norm}"  (\SeeChapter{see section Vector Calculus page \pageref{vector norm}}) and the norm induces a metric space. 
	
	Therefore we understand better what we will study in this section that can be summarized by the following figure:
	\begin{figure}[H]
		\centering
		\includegraphics{img/analysis/topological.jpg}
	\end{figure}

	\subsubsection{Topological Spaces}
	Topological spaces form the conceptual foundation on which the concepts of limit, continuity or equivalence are defined.
	
	The framework is general enough to be applied in many different situations: finite sets, discrete sets, geometry spaces, $n$ dimensional numerical spaces and most complex functional areas. These concepts appear in almost all branches of mathematics, they are therefore central to the modern view of mathematics.
	
	\textbf{Definition (\#\mydef):} Consider a non-empty set $X$ (the length of a plastic ruler for example). A "\NewTerm{topology $\mathcal{T}$}" or "\NewTerm{topological space $(x,\mathcal{T})$}" on $X$ is a family $\mathcal{T}$ of parts of $X$ (of length of our rule...) named "\NewTerm{open $V$}"  (as the open intervals seen in the section  of Functional Analysis) such that the following axioms are true:
	\begin{enumerate}
		\item[A1.] The empty set $\varnothing$ and $X$ are considered as open $V$ and must belong to the family of the topology $\mathcal{T}$ (these only both open sets define what we name the "\NewTerm{trivial topology}" that is the most minimal one satisfying all the axioms):
		
		In other words, if we imagine our plastic ruler, the measure zero (strictly speaking: the empty set) must belong to the topology defined on the ruler and the ruler itself (seen as a subset).
		
		\item[A2.] Any finite intersection of open of $\mathcal{T}$ will be an open of $\mathcal{T}$:
		
		
		\item[A3.] Any union of open of $\mathcal{T}$ will be an open of $\mathcal{T}$:
		
		\begin{tcolorbox}[title=Remarks,colframe=black,arc=10pt]
		\textbf{R1.} Mathematicians frequently note by $O$ the family of open sets and by $F$ the family of closed sets. Convention we will not follow in this book.\\

		\textbf{R2.} The close sets of a topology are complementary of open sets. Therefore, the family of  close sets contains among other $X$ and the empty set $\varnothing$...\\
	
		\textbf{R3.} There is no difference between part and subset of a set.
		\end{tcolorbox}
		
		\item[A4.] The couple $(X,\mathcal{T})$ is a "\NewTerm{Hausdorff space}" or "\NewTerm{separate space}" if moreover the property named "\NewTerm{Hausdorff axiom}" is verified:
				
	\end{enumerate}
	\begin{tcolorbox}[title=Remarks,colframe=black,arc=10pt]
		\textbf{R1.} A well known example of topological space is $\mathbb{R}$ provided with the set $F$ generated by the open intervals (by the union law), that is to say the intervals of the type $] a, b [$.\\
		
		\textbf{R2.} We will see a very concrete and beautiful + nice application of Hausdorff spaces in our study of fractals in the chapter Theoretical Computing.
	\end{tcolorbox}
	
	\textbf{Definition (\#\mydef):} 
	\begin{enumerate}
		\item[D1.] If we denote by $(X, V)$ a topological space, $V$ designating the open sets of $X$, a "\NewTerm{base}", in the topological sense, of $(X, V)$ is a part $B$ of $V$ such that any open set of $V$ is a an union of open sets of $B$ (this is the same idea as vector spaces but in fact applied to sets ... nothing bad and difficult! If you want an example see the section of Measure Theory).
		
		\item[D2.] In topology a subset $A$ of a topological space $X$ is named "\NewTerm{dense}\index{dense set}\label{dense set}" (in $X$) if for every point $x$ in $X$ either belongs to $A$ or is a limit point of $A$. Informally, for every point in $X$, the point is either in $A$ or arbitrarily "close" to a member of $A$. For instance, every real number is either a rational number or has one arbitrarily close to it. Therefore $\mathbb{Q}$ is dense in $\mathbb{R}$.
	\end{enumerate}
	
	\pagebreak
	\subsection{Metric Space and Distance}\label{distance}
	\textbf{Definition (\#\mydef):} A "\NewTerm{metric space}" denoted by $(X, d)$ or sometimes $X_d$ (or even sometimes just $X$ if the type of distance $d$ cannot not be confused) is by definition a set $X$ with provided with an application:
	
	named "\NewTerm{distance}\index{distance}" or "\NewTerm{metric}", which satisfies the following axioms:
	\begin{itemize}
		\item[A1.] Positivity:
		
		
		\item[A2.] Separation:
		
		\item[A3.] Triangular inequality:
		
		
		\item[A4.] Symmetry: 
		
	\end{itemize}
	\begin{tcolorbox}[title=Remarks,colframe=black,arc=10pt]
		\textbf{R1.} Some readers will probably see immediately that some of these properties have already been seen in other sections of this book during our study of distances between functional points and during our study of norms (triangle inequality proved in the section of Vector Calculus - the symmetry, positivity, the separation have already been study in the section of Functional Analysis).\\
		
		\textbf{R2.} Some authors omit the axiom A1 which is strictly correct as it trivially follows from A3.\\
	\end{tcolorbox}
	The "distance function" of $\forall x,y \in X$ is thus usually denoted in the more possible sense in mathematics (at least as far as we know):
	
	we will see three examples much more further below with a schema.
	
	\textbf{Definition (\#\mydef):} If we do not impose the axiom A2, we say that $d$ is a "\NewTerm{semi-distance}" on $X$ and if we allow a semi-distance $d$ to take the value $+\infty$, we prefer to say that $d$is a "gap".
	\begin{tcolorbox}[title=Remarks,colframe=black,arc=10pt]
		\textbf{R1.} If a distance $d$ satisfies the property:
		
		property more restrictive that the triangle inequality in some spaces, we say that $d$ is "\NewTerm{ultrametric}".\\
		
		An example of ultrametric distance is the family tree (...):
		
		\begin{figure}[H]
			\centering
			\includegraphics{img/analysis/family_tree.jpg}
			\caption{Example of ultrametric distance with an orgchart}
		\end{figure}
		We have the following distances:
		
		We note that the distances above do not add up, but we have by cons:
		
		Therefore:
		
		
		\textbf{R2.} Let $(X, d)$ be a metric space and consider $F=\varnothing$ a part of the set $E$. The metric space $(F,\delta)$ where $\delta$ denotes the restriction $d_{F \times F}$ of $d$ is named "\NewTerm{metric subspace}" of $(X, d)$ (we should check that the distance $d$ is equivalent to the distance $\delta$). In this case, we also say that $F$ is provided with the distance induced by this of $X$. We therefore simply note $d$ the induced distance.
	\end{tcolorbox}

	\pagebreak
	Let us give now some examples:
	\begin{tcolorbox}[colframe=black,colback=white,sharp corners]
	\textbf{{\Large \ding{45}}Examples:}\\\\
	E1. If we take for $X$ the plane, or the three-dimensional space of Euclidean geometry and a unit of length, the "distance" in the usual sense is a distance within the meaning of the 4 axioms mentioned above. In these spaces, the three points $A, B, C$ satisfy as we have proved it in the section Vector Calculus:
	
	with other inequality obtained by circular permutation of $A, B, C$. These inequalities are well known, for example between the side lengths of a triangle.\\
	
	E2. If we take $X=\mathbb{R}^n$, $n \in \mathbb{N} \geq 1$ and that we equip $\mathbb{R}^n$ of an Euclidean vector space structure (and not non-Euclidean!) and we take two points:
	
	in $\mathbb{R}^n$, the distance is then given by (we have already proved this in the sections of Functional Analysis and Vector Calculus):
	
	\end{tcolorbox}
	\label{euclidean topology}This latter distance satisfies the five axioms of distance and we name it the "\NewTerm{Euclidean distance}" and is often denoted $L_2$. We can take (it is an interesting property for the general culture), any relation of the form:
	
	is also a distance in $\mathbb{R}^n$ (without proof) named " \NewTerm{$p$-norm}". In the particular case with $n=1$, we have of course:
	
	This is the usual distance on $\mathbb{R}$ and is often denoted $L_2$.
	
	Mathematicians are even stronger by generalizing ever more (the proof has little interest for now in this book) the prior-previous relation (taking into account the definition of the distance) in the form:
	
	which is named "\NewTerm{Hölder distance}". Also sometimes denoted for $L_p:\mathbb{R}^n\mapsto\mathbb{R}$:
	
	\begin{tcolorbox}[title=Remark,colframe=black,arc=10pt]
	Following the intervention of a reader we would like to point out that strictly speaking the above inclusion should be noted $[1,+\infty[ \subset \mathbb{\overline{R}}$ where $\mathbb{\overline{R}}$ is the achieved line (also valid for precision for the Minkowski inequality below).
	\end{tcolorbox}	
	As for the triangle inequality, then given by (\SeeChapter{see section Vector Calculus page \pageref{triangle inequality}}):
	
	The generalization, by the verification of the existence of the Hölder distance, gives us true "\NewTerm{Minkowski inequality}":
		

	Let us continue with our examples:
	\begin{tcolorbox}[colframe=black,colback=white,sharp corners]
	\textbf{{\Large \ding{45}}Examples:}\\\\
	E3. If we take $X=\mathbb{C}$ we will consider the distance:
	
	Therefore if $z=a+\mathrm{i}b=(a,b)$ and $z'=a'+\mathrm{i}b'=(a',b')$ we have the module that the same manner as norm in $\mathbb{R}^2$, forms a distance:
	\\
	
	E4. Let us consider $E=\varnothing$ an arbitrary set. Let us write:
	
	It is quite check that this distance satisfies the five axioms and that is furthermore an ultrametric distance. This distance is named "\NewTerm{discreet distance}" and the reader will notice that, by analogy, we choosed to express this distance by the Dirac symbol $\delta$ (this is not innocent !!) rather than the traditional $d$.
	\end{tcolorbox}
	
	\pagebreak
	\subsubsection{Equivalent Distances}
	Sometimes two different distances $d$ and $\delta$ on the same set $E$ are quite similar so that the related metric spaces $(E,d),(E,\delta)$ have the same properties for certain mathematical objects defined by $d$ on one hand, and by $\delta$ on the hand. There are several concepts of equivalences for example first (before the others that require mathematical tools that we have not yet defined):

	\textbf{Definition (\#\mydef):} Let $d$ and $\delta$ be two distances on the same set $E$, $d$ and $\delta$ are named "\NewTerm{equivalent distances}" if there are two real constants $c>0,C>0$ such that:
	
	Therefore:
	
	with $c\leq C$. We note this equivalence by:
	
	The advantage of this definition is the following: if we have convergence for one of the metric, then we have convergence for the other too. More clearly:
	
	verbatim:
	
	
	\subsubsection{Lipschitz Functions}\label{lipschitz functions}
	With respect to the above definitions, we can now assign some additional properties to functions such as we had define in the section of Set Theory or Functional Analysis and analysed (in part...) in the section of Differential and Integral Calculus. The idea is also mainly to build a set of tools enabling the study of differential properties of non differentiable functions.
	
	Let $(E, d)$ and $(F,\delta)$ be metric spaces, and $f:E \rightarrow E$ a function. We define the following properties:
	\begin{enumerate}
		\item[P1.] We say that $f$ is an "\NewTerm{isometry}" if (it is rather intuitive ...!):
		
		
		\item[P2.] If we take the usual distance, the $L$-Lipschitz or "\NewTerm{Lipschits function of order $L$}" is then defined by on a given interval by:
		
		that we can also write:
		
		or what remains the same: all line drawn between two arbitrary points of the graph must have a bounded and finite slope coefficient (derivative) between $L$ and $-L$.
		
		Any such $L$ is referred to as a "\NewTerm{Lipschitz constant}" for the function $f$. As $L$ can be define on intervals (not necessarily the whole domain of definition) smallest constant is sometimes named the "\NewTerm{best Lipschitz constant}".
		
		Otherwise, one can equivalently define a function to be Lipschitz continuous if and only if there exists a constant $L$ such that, for all $x\neq y$:
		
		For real-valued functions of several real variables, this holds if and only if the absolute value of the slopes of all secant lines are bounded by $k$. 
		
		Therefore all Lipschitz must be continuous and any function $f$ that has a bounded $L$ value is more restrictive than just simply being continuous! 
		
		\begin{tcolorbox}[colframe=black,colback=white,sharp corners]
		\textbf{{\Large \ding{45}}Examples:}\\\\
		E1. The function $f(x)=\sin (x)$ is $1$-Lipschitz as the derivative of the cosine is between $-1$ and $1$.\\
		
		E2. The function $f(x)=x^2$ is locally Lipschitz as for any interval closed and finite interval we can found a bound $L$ but is not globally Lipschitz as when $x\rightarrow \pm\infty$ then the derivative has also no bounds.\\
		
		E3. The function $f(x)=|x|$ has no derivatives on $x=0$ in the ordinary sense. But its derivative in the Lipschitz sense on $x=0$ is given by the a closed interval denoted $\partial_L f(0)=[-1,1]$ given by the bound of $L$ as the ordinary derivatives is less than or equal to $L$ in absolute value!\\
		
		This last example show us that the notion of a local minimum of a function $f(x)$ in the ordinary sense is generalized with Lipschitz condition. As we can now simply define the condition of local minimum of a non-smooth function $f(x)$ as:
		
		rather than in the ordinary sense (more restrictive):
		
	\end{tcolorbox}
		
		
		Schematically for a Lipschitz continuous function, there is a double cone (shown in white) whose vertex can be translated along the graph, so that the graph always remains entirely outside the cone.
		\begin{figure}[H]
			\centering
			\includegraphics{img/analysis/lipschitz.jpg}
			\caption[Example of Lipschitz function]{Example of Lipschitz function (source: Wikipedia)}
		\end{figure}
		
		Intuitively, a Lipschitz continuous function is therefore limited in how fast it can change: there exists a definite real number such that, for every pair of points on the graph of this function, the absolute value of the slope of the line connecting them is not greater than this real number; this bound is named a "Lipschitz constant" of the function (or "modulus of uniform continuity"). For instance, every function that has bounded first derivatives is Lipschitz.
		
		\item[P3.] If $L=1$, we say that the function $f(x)$ is a "\NewTerm{short map}". If $-1<L<1$, we say that $f(x)$ is "\NewTerm{strictly contracting}\label{strictly contracting}".
		
		\item[P4.] We say that two metric spaces are "\NewTerm{isometric spaces}" if there is a surjective isometry of one over the other (which is quite natural in geometry ...).
	\end{enumerate}
	\begin{tcolorbox}[title=Remarks,colframe=black,arc=10pt]
	\textbf{R1.} An isometry is always injective as:
	
	but in general it is not surjective.\\
	
	\textbf{R2.} If $(E,d)$ and $(F,\delta)$ are isometric, of the point of view of the theory of metric spaces they are not discernible, as all their properties are the same, but their elements can be  of very different nature (sequences in one and functions in the other).\\
	\end{tcolorbox}
	
	\pagebreak
	\subsubsection{Continuity and Uniform Continuity}\label{continuity and uniform continuity}
	As we already see it in the section of Functional Analysis, a continuous function is, roughly speaking, a function for which small changes in the input result in small changes in the output and that permits the analysis of limits. Otherwise, a function is said to be a discontinuous function. Formally it was defined by:
	
	In other words remember that this mean that a function is continuous if for every point $x_0$ in the domain $E$, we can make the images of that point ($f(x_0)$) and another point ($f(x)$) arbitrarily close (of a distance $\varepsilon$) if we move the other point ($x$) close enough (distance $\delta$) to our given point.
	
	Hence it is not continuous if:
	
	
	The previous definition is not so good as it make usage of a special case of distance (the absolute value). It is therefore more common to generalize by writing:
	
	
	Now let us state a more restrictive definition!
	
	\textbf{Definition (\#\mydef):} A function $f(x)$ is "\NewTerm{uniformly continuous}" if it satisfies:
	
	with $\lambda=\varepsilon/L$ and $L\neq 0$. In other words, if we can bring two points as close as we want in a space, so can we in the other way (which ensures somehow the derivation.
	
	Hence it is not uniformly continuous if:
	
	
	If we compare the two relations:
	
	the only difference is the order of the quantifiers. Indeed, for something to be continuous, you can check "one $x$ at a time", so for each $x$, you pick a $\varepsilon$ and then find some $\varepsilon>0$ that depends on both $x$ and $\varepsilon$ so that $|f(x)-f(x_0)|<\varepsilon$ if $|x-x_0|<\delta$. If we want uniform continuity, we need to pick first a $\varepsilon$, then find a $\delta$ which is good for ALL the $x$ values we might have.
	
	As the previous definition are not quite easy for everybody let us see the engineer version of these two definitions:
	
	\pagebreak
	\textbf{Definitions (\#\mydef):}
	\begin{enumerate}
		\item[D1.] A function $f(x):E \rightarrow \mathbb{R}$ is "\NewTerm{continuous}" at a point $x_0\in A$ if, for all $\varepsilon>0$, there exists a $\delta>0$ such that whenever $|x-x_0|<\delta$ (and $x\in E$) it follows that $|f(x)-f(x_0)|<\varepsilon$.
		
		\item[D2.] A function $f(x):E \rightarrow \mathbb{R}$ is "\NewTerm{uniformly continuous}" on $A$ if, for all $\varepsilon>0$, there exists a $\delta>0$ such that whenever $|x-x'|<\delta$ (and $(x,x')\in E$) it follows that $|f(x)-f(x')|<\varepsilon$.
	\end{enumerate}
	Therefore we see better the difference: continuity is defined at a point $x_0$, whereas uniform continuity is defined on a set $E$. Roughly speaking, uniform continuity requires the existence of a single $\delta>0$ that works for the whole set $E$, and not near the single point $x_0$.
	
	From this definition wee that any uniformly continuous function is continuous but the reciprocity is not true (any uniformly continuous function is not necessarily continuous):
	
	If the chosen distance is known (for example the absolute value for scalar functions such that $d=|\cdot|$ and $\delta=|\cdot|$) the previous definition notation change obviously a little bit:
	
	
	\begin{tcolorbox}[colframe=black,colback=white,sharp corners]
	\textbf{{\Large \ding{45}}Example:}\\\\
	The function $f(x) = x^2$ is continuous but not uniformly continuous
on the interval $E = [0,+\infty[$.\\

	We prove first that our function $f(x)$ is continuous on $E$. Remember first that:
	
	In our case we can therefore write and check that:
	
	
	Let us choose $x_0=a-1$ with $a>1$ and $\delta=\min(1,\varepsilon/2a)$ (note that $\delta$ depends on $x_0$ since $a$ does). Choose $x \in S$. Assume $|x-x_0|<\delta$. Then $|x-x_0|<1$ so $x<x_0+1$ so $x,x_0<a$ so:
	
	We prove now that $f(x)$ is not uniformly continuous on $E$, i.e.:
	
	Let $\varepsilon=1$. Choose $\delta>0$. Let $x_0=1/\delta$ and $x=x_0+\delta/2$. Then $|x-x_0|=\delta/2<\delta$ but:
	
	as required.
	\end{tcolorbox}
	
	\subsection{Opened and Closed Set}
	\textbf{Definition (\#\mydef):} Consider a set $E$ with a distance $d$. A subset $U$ of $E$ is named "\NewTerm{open subset}" if, for each element of $U$, there is a non-null distance $r$ for which all the elements of $E$ whose distance from this element is less than or equal to $r$, belong to $U$, which gives in mathematical language:
	
	In topology, an open subset is then only an abstract concept generalizing the idea of an open interval in the real line.
	\begin{tcolorbox}[title=Remark,colframe=black,arc=10pt]
	For recall, the symbol "|" means in this context: satisfies the property...
	\end{tcolorbox}	
	In practice, however, open sets are usually chosen to be similar to the open intervals of the real line. The notion of an open set provides a fundamental way to speak of nearness of points in a topological space, without explicitly having a concept of distance defined. 
	\begin{tcolorbox}[colframe=black,colback=white,sharp corners]
	\textbf{{\Large \ding{45}}Example:}\\\\
	The points $(x, y)$ satisfying $x^2 + y^2 = r^2$ are colored blue. The points $(x, y)$ satisfying $x^2 + y^2 < r^2$ are colored red. The red points form an open subset of the plane $\mathbb{R}^2$. The blue points form a boundary set. The union of the red and blue points is a closed set.
	\begin{figure}[H]
		\centering
		\includegraphics{img/analysis/opened_set.jpg}
	\end{figure}
	\end{tcolorbox}
	This definition may perhaps seem complicated but in fact, its real meaning is simpler than it seems. In fact, according to this definition, an open set in a topological space is nothing more than a set of contiguous points and without borders.
	
	The lack of border comes from the condition $r\neq 0$. Indeed, by reductio ad absurdum, if an open set $U$ had an edge, then for each point on it (the edge) it would still be possible to find a point not belonging to $U$ as close as we want from it. It follows that the distance $r$ becomes necessary therefore zero.

	\textbf{Definitions (\#\mydef):} 
	\begin{enumerate}
		\item[D1.] A "\NewTerm{closed subset}" is an "\NewTerm{open with edge}"

		\item[D2.] A "\NewTerm{neighbourhood}" of a point $E$ is a subset of $E$ containing an open subset containing this point.
	\end{enumerate}
	The definition of an open set can be simplified by introducing an additional concept, that of "open ball":
	
	\subsubsection{Balls}
	Given $x$ an element of $E$:
	
	\textbf{Definition (\#\mydef):} An "\NewTerm{open ball of center $x$ and radius $r>0$}" or "\NewTerm{metric ball of radius $r$ centered at $x$ without border}" is the subset of all the points of $E$ whose the distance $x$ is less than $r$, that we write in general:
	
	An open set can also be defined as a set for which it is possible to define an open ball at each point.
	
	Typically in the real plane where $d$ is the euclidean distance:
	
	\begin{figure}[H]
		\centering
		\includegraphics{img/analysis/open_set.jpg}
		\caption{An open ball of radius $r$, centered at the point $x$}
	\end{figure}
	An open set can also be defined as a set for which it is possible to define an open ball at each point.
	\begin{tcolorbox}[title=Remarks,colframe=black,arc=10pt]
	\textbf{R1.} The open such defined, form what we name an "\NewTerm{induced topology}" by the distance $d$ or also a "\NewTerm{metric topology}".\\
	
	\textbf{R2.} We name an "\NewTerm{open cover}" $U$ of $E$, a set of open of $E$ whose union is equal to $E$. In other words: A collection of open sets that collectively cover another set.\\
	
	Formally, if:
	
	is an indexed family of open sets $U_\alpha$, then $C$ is a cover of $X$ if:
	
	Visually in a naive way this gives:
	\begin{figure}[H]
		\centering
		\includegraphics{img/analysis/open_cover.jpg}
	\end{figure}
	\end{tcolorbox}	
	\textbf{Definition (\#\mydef):} A "\NewTerm{closed ball}" is similar to an open ball but differs in the sense that we include the elements located at a distance $r$ from the center:
	
	\begin{tcolorbox}[title=Remark,colframe=black,arc=10pt]
	For $0<r<r'$ the inclusions $_oB_x^r \subset B_x^r \subset B(x,r')$ are direct consequences of the definition of the open and closed ball.
	\end{tcolorbox}
	\begin{tcolorbox}[colframe=black,colback=white,sharp corners]
	\textbf{{\Large \ding{45}}Example:}\\\\
	The usual distance in $\mathbb{R}$ is given by $d(x,y)=|x-y|$. The balls are there simple intervals. For $x \in \mathbb{R}$ and $r\in \mathbb{R}_{+}^{*}$, we have:
	
	\end{tcolorbox}
	\textbf{Definition (\#\mydef):} A "\NewTerm{sphere}" is given by:
	
	\begin{tcolorbox}[title=Remark,colframe=black,arc=10pt]
	Since by definition $r>0$, open and closed balls are not empty because they contain at least their center. By cons, a sphere may be empty!
	\end{tcolorbox}
	\begin{tcolorbox}[colframe=black,colback=white,sharp corners]
	\textbf{{\Large \ding{45}}Example:}\\\\
	With $\mathbb{R}^n,\mathbb{C}^n$ we have seen in the previous examples we could set different distances. To distinguish them, we denote then by:
	
	So in $\mathbb{R}^2$ the closed balls with center O and of radius unit equivalent to the previous three formulations, have the following shapes (remember that $0<r\leq 1$ in this example):
	\begin{figure}[H]
		\centering
		\includegraphics{img/analysis/shape_some_distances.jpg}
		\caption{Examples of closed balls of unit radius with different distances}
	\end{figure}
	\end{tcolorbox}
	For example in statistics (see the section of the same name) we also use (among a lot of others) the Chi-2 distance given by:
	
	Or always in (multivariate) statistics (have a look to the Statistics section but also to the Industrial Engineering one) the "\NewTerm{Mahalanobis distance}"\index{Mahalanobis distance}\label{Mahalanobis distance}:
	
	\begin{tcolorbox}[title=Remarks,colframe=black,arc=10pt]
	The proof that the Mahalonobis distance is indeed a distance is straightforward. Indeed, without loss of generality, let us rewrite $x-\mu=\vec{x}$. Then:
	
	But as we have proved it in the section Statistics at page \pageref{positive semi-definitiveness of covariance matrix}, $\Sigma$ is positive semi-definite, hence $\Sigma^{-1}$ is most of time invertible (\SeeChapter{see section Statistics page \pageref{positive semi-definite matrix not always invertible}}). When it's invertible, all its eigenvalues are positive, then $\Sigma^{-1}$ is also positive semi-definite (\SeeChapter{see section Linear Algebra page \pageref{inverse definite positive matrix is also definite positive}}). This implies that:
	
	Therefore the Mahalonobis distance in indeed a distance. Another way to see it as engineer is just to consider that as $\Sigma^{-1}\vec{x}$ simply gives just another vector $\vec{y}$ such that:
	
	This it's just the euclidean distance between two different vectors!
	\end{tcolorbox}	
	For any other semi-positive definite matrix $A$ other than the variance-covariance matrix we also define the "\NewTerm{elliptic metric}"\index{elliptic metric}\label{elliptic metric}:
	
	
	Or in the Error Correcting Codes section we use the Hamming distance given by:
	
	and so on...
	
	\subsubsection{Partitions}
	Now that we have defined the concepts of balls, we can finally (almost) rigorously define the concepts of open and closed intervals (which in a space of more than one dimension are named "partitions") that we have so often used in the section of Functional Analysis and Differential and Integral Calculus.
	
	\textbf{Definition (\#\mydef):} Let $(X,d)$ of a metric space. We say that a subset $A$ of $X$ is "bounded" if there is a closed ball $_fB_{r_0}^r(X)$ such that $A \subseteq _fB_{r_0}^r(X)$:
	
	Given the previous note on balls inclusions, it is clear that we can replace the word "closed" by"open". Moreover the triangle inequality implies that the bounded character of $A$ does not depend on the choice of $x_0$ (with a $x_0^{\prime}$ we simply need to replace $r$ by $r'=r+d\left(x_0,x_0^{\prime}\right)$).
	\begin{tcolorbox}[colframe=black,colback=white,sharp corners]
	\textbf{{\Large \ding{45}}Example:}\\\\
	The odd–even topology is the topology where $X = \mathbb{N}$ and:
	
	the unbounded partitions of radius $r<1$. \\
	
	 Therefore we see that unless $P$ is trivial, at least one set in $P$ contains more than one point, and the elements of this set are topologically indistinguishable: the topology does not separate points!
	\end{tcolorbox}
	
	\textbf{Definitions (\#\mydef):}
	\begin{enumerate}
		\item[D1.] Let $X$ be a set and $(Y,d)$ a metric space. If $X$ is a set, we say that a function $f: X\mapsto Y$ is "bounded" if its image $f (X)$ is bounded (the case of the sine or cosine function, for example).
		
		\item[D2.] Given $(E, d)$ a metric space, and given $A$ a non-empty subset of $E$. For any $u\in E$ we note $d(u, A)$ and name "\NewTerm{distance $u$ to $A$}", the positive real number:
		
		We extend the concept by writing:
		
		If $A$ and $B$ are two parts (subsets) of $E$ we have respectively (perhaps this is more understandable in this way for some readers...)
		
		The reader must take care here to interpret $d(A,B)$ as the infinimum of the distance between the sets $A$ and $B$, because the distance between the parties does not always define a distance in the usual way on the part for example of $\mathcal{P}(\mathbb{R})$.
		
		Indeed, if we take again our famous example:
			
	 	we have $d(A,B)=0$ when $n \rightarrow 0$ while $A\neq B$.
	 	\begin{tcolorbox}[title=Remarks,colframe=black,arc=10pt]
		\textbf{R1.} If the reader has well understood the definition of "parts" (and especially the previous example) he has probably noticed that it does not necessarily always exist a $a\in A$ such that $d(u,A)=d(u,a)$. Accordingly, we write:
		
		Moreover, if such an $\alpha$ exists, it is obviously not necessarily unique.\\
		
		\textbf{R2.} It should be remembered that this distance also meets the $5$ axioms of distances (we can give the proof on request)!
		\end{tcolorbox}	
	
		\item[D3.] Given $(E, d)$ a metric space, and let $A$ be a part (subset) of $E$. We name "\NewTerm{adhesion}" of $A$ and denote by $\text{adh} (A)$ the subset of $E$ defined by:
		
		For example, the adhesion of the part (subset) of rational numbers $\mathbb{Q}$ (part $A$) of $\mathbb{R}$  (the metric space $E$) is a subset of $\mathbb{R}$ itself since any real number is the limit of a rational.
		
		Especially, since $\forall u\in A:\quad =+\infty$, we have $\text{adh}(\varnothing)=\varnothing$, and since $\forall u\in E:\quad d(u,E)=0$, we have $\text{adh}(E)=E$.
		
		\begin{tcolorbox}[title=Remarks,colframe=black,arc=10pt]
		\textbf{R1.} Any element of the set $\text{adh}(A)$ is named "adherent point" of $A$.\\
		
		\textbf{R2.} We say that a part $A$ of $E$ is and "\NewTerm{closed part}" if it is equal to its adherence.\\
		
		\textbf{R3.} We say that a part $A$ of $E$ is an "\NewTerm{open part}" if its complementary relatively to $E$:
		
		is closed.
		\end{tcolorbox}	
	\end{enumerate}
	It follows from the definitions that (without proof):
	
	and:
	
	with some properties (supposed as very obvious but we can give the proof on request):
	\begin{enumerate}
		\item[P1.] If $A\subset E$ and $B\subset E$ satisfies $A\subset B$, then we have:
		
		
		\item[P2.] For all $A\subset E$, any $u\in E$ we have:
		
		The latter property has for corollary (obvious and therefore without proof excepted on request):
		If for any $u\in E$, we have $d(u,A)=d(u,B)$ and $A,B\neq \varnothing$, we then have:
		
	\end{enumerate}
	
	\subsubsection{Formal Ball}
	The concept of distance from a point to a set gives the possibility to extend the notions of ball and sphere see previously. We will see now the basis concepts of a "\NewTerm{formal ball}" also named "\NewTerm{generalized ball}".
	
	\begin{enumerate}
		\item[D1.] Given $A\neq \varnothing$ and given a $r>0$. We name "\NewTerm{generalized open ball}" of center $A$ of radius $r$, the following set:
		
		and respectively "\NewTerm{generalized closed ball}":
		
		and respectively  a "\NewTerm{generalized sphere}":
		
		
		\item[D2.] Given $(E, d)$ a metric space and let $A, B$ be two non-empty parts (subsets) of $E$. We denote by $g (A, B)$ and name "\NewTerm{gap}" of $A$ to $B$, the real number greater than or equal to zero such that:
		
		\begin{tcolorbox}[title=Remark,colframe=black,arc=10pt]
		The triangle inequality $g(A,B)\leq g(A,C)+g(C,B)$ is not valid in the context of gaps. To prove it, a single example that contradicts this inequality is sufficient.
		\end{tcolorbox}
		\begin{tcolorbox}[colframe=black,colback=white,sharp corners]
		\textbf{{\Large \ding{45}}Example:}\\\\
		In $\mathbb{R}$ let us take $A=\{0,1\},B=\{2,3\},C=\{1,3\}$ then we have:
		
		\end{tcolorbox}	
	\end{enumerate}
	
	\subsubsection{Diameter}
	\textbf{Definition (\#\mydef)}: Given $(E, d)$ a metric space and $A$ a non-empty part (subset) of $E$. We denote $\text{diam}(A)$ and name "\NewTerm{diameter}" of $A$, the positive non-zero real number:
	
	Every non-empty part (subset) $A$ of a metric space satisfying $\text{diam}<+\infty$ will also be say "bounded".
	\begin{tcolorbox}[title=Remark,colframe=black,arc=10pt]
	We consider the empty set $\varnothing$ as bound set of diameter $A$.
	\end{tcolorbox}	
	If the whole metric space $(E,d)$ is bounded, we say that the distance $d$ is bounded. For example, the discrete distance is limited, the usual distance on $\mathbb{R}$ is not.
	
	We also have the following properties (the first two are usually trivial, the third one comes from the definition of the diameter itself):
	\begin{enumerate}
		\item[P1.] $\text{diam}(A)=0 \Leftrightarrow A=\{a\}$ or $A=\varnothing$
		
		\item[P2.] $A\subset B \Rightarrow \text{diam}(A)\leq \text{diam}(B)$
		
		\item[P3.] $\text{diam}(_fB_x^r)\leq 2r,\text{diam}(_oB_x^r)\leq 2r,\text{diam}(S_x^r)\leq 2r$
		
		\begin{tcolorbox}[colback=red!5,borderline={1mm}{2mm}{red!5},arc=0mm,boxrule=0pt]
		\bcbombe Caution!!! Concerning the latter property, the reader must take the habit of thinking with the Euclidean distance. The first common pitfall is to think that the second diameter (that of the open ball) should be strictly less but that would be forgetting that the board has no thickness strictly speaking! 
		\end{tcolorbox}
		
		There is also often a understanding problem with $\text{diam}(S_x^r)\leq 2r$. To be convinced just take the discrete distance (that for two points that are not confused is equal to $1$, otherwise $0$). Thus, in a metric space where we take $S_x^r$ with $r=1$, we have indeed $\text{diam}(S_x^1)\leq 1$ (that is an interesting case because almost completely counter-intuitive).
		
		\item[P4.] $\text{diam}(A\cup B)\leq \text{diam}(A)+g(A,B)+\text{diam}(B)$
		
		To be convinced, in $\mathbb{R}$ take $A=B$, then we have (trivial strict inferiority):
		
		
		\item[P5.] $A$ is bounded if and only if: $\exists r>0,\exists x\in E:\quad A\subset _oB_x^r$
	\end{enumerate}

	\textbf{Definition (\#\mydef):} We name "\NewTerm{Hausdorff excess}" or "\NewTerm{Hausdorff distance}" from $X$ to $Y$:
	
	that we found often in the literature with the more condensed notation:
	
	or much more explicitly:
	
	\begin{figure}[H]
		\centering
		\includegraphics{img/analysis/hausdorff_distance.jpg}
		\caption[]{Components of the calculation of $d_H$ between $X$ and $Y$ (source: Wikipedia)}
	\end{figure}
	\begin{tcolorbox}[colframe=black,colback=white,sharp corners]
	\textbf{{\Large \ding{45}}Example:}\\\\
	Let us take $X\subset \mathbb{R}^2$ as the unit radius circle centered at the origin and $Y$ to the square circumscribing it. Elementary geometry concepts obviously leads to finding that the Hausdorff distance between the circle and the square is therefore:
	\begin{figure}[H]
		\centering
		\includegraphics{img/analysis/hausdorff_distance_highschool_example.jpg}
		\caption{High-school example of a Hausdorff distance in the plane}
	\end{figure}
	technically:
	
	\end{tcolorbox}
	\begin{tcolorbox}[title=Remark,colframe=black,arc=10pt]
	We have generally $e(X,Y)\neq e(Y,X)$ and these quantities may not be finite.
	\end{tcolorbox}
	
	\subsection{Varieties}\label{varieties}
	We now introduce the "varieties". These are topological spaces that are "locally as $\mathbb{R}^2$" (our space for example ...), that is locally euclidean.
	
	\textbf{Definitions (\#\mydef):}
	\begin{enumerate}
		\item[D1.] A "\NewTerm{topological variety of dimension $n$}" is a Hausdorff space $M$ such that for every $p\in M$ there exists an open neighbourhood $U\subset M$ with $p\in U$, an open neighbourhood $U' \subset \mathbb{R}^n$ and a homeomorphism such that:
		
		\item[D2.] A "\NewTerm{homeomorphism}" between two spaces is a continuous bijection whose inverse is also continuous.

		\item[D3.] The pairs $(U,\varphi)$ are named "\NewTerm{maps}", $U$ being the "\NewTerm{domain of the map}" and $\varphi$ the "\NewTerm{coordinate application}". Instead of "map" sometimes we say also "coordinate system".

		\begin{tcolorbox}[title=Remark,colframe=black,arc=10pt]
		We will denote by $\dim(M)$ the dimension of a topological variety. Therefore:
		
		\end{tcolorbox}
	
		\item[D4.] Given $M$ be a topological variety of dimension $n$. A family $A$ of maps of $M$ family is named a  "\NewTerm{atlas}" if for each $x\in M$, there is exists a map $(U, \varphi)\in A$ such as $x\in U$.
	\end{enumerate}
	If $(U_1,\varphi_1),(U_2,\varphi_2)$ are two maps of $M$ such as $U_1\cap U_2\neq \varnothing$, then the application of map changes:
	
	\begin{figure}[H]
		\centering
		\includegraphics{img/analysis/homeomorphism_of_maps.jpg}
		\caption{Maps homeomorphism}
	\end{figure}
	is obviously also a homeomorphism. More "geometrically" it looks like this:
	\begin{figure}[H]
		\centering
		\includegraphics[scale=0.6]{img/analysis/maps_homeomorphism.jpg}
		\caption[Intuitive maps homeomorphism]{Intuitive maps homeomorphism (source: ?)}
	\end{figure}
	
	\pagebreak
	\subsubsection{Subvariety}
	\textbf{Definition (\#\mydef):}A subset of a variety is itself a variety named a "\NewTerm{subvariety}\index{subvariety}\label{subvariety}". 

	\begin{tcolorbox}[colframe=black,colback=white,sharp corners]
	\textbf{{\Large \ding{45}}Example:}\\\\
	A sphere of the three-dimensional Euclidean space $\mathbb{R}^3$ is an (algebraic) variety since it is defined by a polynomial equation. For example and is obviously smooth and locally euclidean:
	
	defines the sphere of radius $1$ centered at the origin. Its intersection with the $xy$-plane is a circle given by the system of polynomial equations:
	
	Hence the circle is itself an algebraic variety, and a subvariety of the sphere, and of the plane as well.
	\end{tcolorbox}
	

	\pagebreak
	\subsubsection{Surfaces Homeomorphism}
	\textbf{Definition (\#\mydef):} In the mathematical field of topology, a "\NewTerm{homeomorphism} or "\NewTerm{topological isomorphism}" is a continuous function between topological spaces that has a continuous inverse function. Roughly speaking, a topological space is a geometric object, and the homeomorphism is a continuous stretching and bending of the object into a new shape. Thus, a square and a circle are homeomorphic to each other, but a sphere and a torus are not. 
	
	More formally, remember that an application $\varphi: X \mapsto Y$ between two topological spaces is named a homeomorphism if it has the following properties:
	\begin{enumerate}
		\item $\varphi$ is a bijection (one-to-one and onto)
		
		\item $\varphi$ is continuous
		
		\item The reciprocal function $\varphi^{-1}$ is continuous 
	\end{enumerate}
	
	Can we say that a square (being a special map in $\mathbb{R}^2$) is homeomorph to circle (being another special map in $\mathbb{R}^2$), or a torus to a cup of tee... If this is possible we must be able to find a closed form bijective expression between the two surfaces.
	
	As the pure theoretical concepts are very not friendly in our point of view let us begin with a two dimension special case. Let us show (prove) first that we can transform all interior points of  square of side $1$ into all interior points circle of radius $1$. This is represented by the  know figure:
	\begin{figure}[H]
		\centering
		\includegraphics{img/analysis/isomorphism_circle_square.jpg}
	\end{figure}
	Such mappings have particular interest in industrial design or just simply for communication purposes (Photoshop effects or Statistics charts deformation as we do many times in the \texttt{R} Software):
	\begin{figure}[H]
		\centering
		\includegraphics[scale=0.5]{img/analysis/chessboard_isomorphic_circle_square.jpg}
	\end{figure}
	or for defishing fish eyes captors, picture or security mirrors:
	\begin{figure}[H]
		\centering
		\includegraphics[scale=1.25]{img/analysis/defishing.jpg}
	\end{figure}
	Recall that we defined unit disc as the set:
	
	If we think of the unit disc as a continuum of concentric circles with radii growing from zero to one, we can parametrize the unit disc as the set:
	
	In doing so, we introduced a parameter $t$ this is the distance of point $(u,v)$ to the origin.
	\begin{figure}[H]
		\centering
		\includegraphics{img/analysis/continnum_disc.jpg}
	\end{figure}
	In analogy to the circular continuum of the unit disc, one can write the square region $[-1,1] \times [-1,1]$ as the set:
	
	In other words, the square can be considered as a continuum of concentric shrunken FG-squircles (\SeeChapter{see section Analytical Geometry page \pageref{fg squircle}}).
	\begin{figure}[H]
		\centering
		\includegraphics{img/analysis/continnum_square.jpg}
	\end{figure}

	Topologists denote the proof that the interior points of two geometries are homeomorph in this special case using the following notation
	
	We will now show that $\mathring{\mathcal{D}}\mapsto \mathring{\mathcal{D}}$ as it is the most common case in practice in our point of view. That is to say:
	\begin{figure}[H]
		\centering
		\includegraphics{img/analysis/mapping_circle_to_square.jpg}
	\end{figure}
	We can establish a correspondence between the unit disc and the square region by mapping every circular contour in the interior of the disc to a squircular contour in the interior of the square. In other words, we map contour curves in the circular continuum of the disc to those in the squircular continuum of the square. This can be done by equating the parameter $t$ of both sets to get the equation:
	
	We name this equation the "\NewTerm{squircularity condition}" for mapping a circular disc to a square region.
	
	It is easy to derive the FG-Squircular mapping by combining the squircularity condition:
	
	That we can also write:
	
	Using radial to cartesian coordinates (\SeeChapter{see section Vector Calculus page \pageref{polar coordinates}}):
	
	Therefore by equivalence we get:
	
	After substitution of parameter $t$, we get:
	

	In other words, this is a radial mapping that converts circular contours on the disc to squircular contours on the square.
	
	We shall now derive the inverse equations for the FG-Squircular mapping. But as it is boring to write in \LaTeX{} and it is not used to much in practice we will omit the latter for the moment.
	
	\subsubsection{Differential Varieties}
	\textbf{Definitions (\#\mydef)}:
	\begin{enumerate}
		\item[D1.] A "\NewTerm{differentiable variety}" is a topological space $M$ where the applications $\varphi$ are of class $\mathcal{C}^{+\infty}$.

		\item[D2.] A "\NewTerm{diffeomorphism}" is an application where $\varphi: U\mapsto U' $ where $U,U'$ are open domains of $\mathbb{R}^n$ and if $\varphi$ is a homeomorphism and furthermore $U,U'$ are differentiable!
		\begin{tcolorbox}[title=Remark,colframe=black,arc=10pt]
		"Differentiable" in this context will always mean of class $\mathcal{C}^{+\infty}$.
		\end{tcolorbox}

		\item[D3.] Given a topological variety $M:=M^n$ (to simplify the notations), two maps $(U_1,\varphi_1),(U_2,\varphi_2)$ of $M$ are named \NewTerm{compatible maps} (more precisely: compatible of class $\mathcal{C}^{+\infty}$ if one of these two properties is satisfied:
		\begin{enumerate}
			\item[P1.] $U_1\cap U_2\neq \varnothing$ and the application $\varphi_2\circ \varphi_1^{-1}$ of map changes is a diffeomorphism.

			\item[P2.] $U_1\cap U_2 =\varnothing$
		\end{enumerate}
		An atlas $A$ of $M$ is differentiable if all maps of $A$ are compatible between them.
	\end{enumerate}
	\begin{tcolorbox}[title=Remark,colframe=black,arc=10pt]
	Given a differentiable atlas, it is sometimes necessary to complete it: we say that a map of $M$ is compatible with a differentiable atlas if it is compatible with every map of $A$. An atlas of $A$ is a "\NewTerm{maximal atlas}" if every compatible map of $A$ belongs already to $A$. A maximal atlas is named a "\NewTerm{differentiable structure}".
	\end{tcolorbox}

	\begin{flushright}
	\begin{tabular}{l c}
	\circled{70} & \pbox{20cm}{\score{4}{5} \\ {\tiny 12 votes,  71.67\%}} 
	\end{tabular} 
	\end{flushright}

	%to make section start on odd page
	\newpage
	\thispagestyle{empty}
	\mbox{}
	\section{Measure Theory}\label{measure theory}
	\begin{tcolorbox}[colback=red!5,borderline={1mm}{2mm}{red!5},arc=0mm,boxrule=0pt]
	\bcbombe Caution! The level of abstraction and of motivation required for reading and understanding this section is quite high for engineers (target audience of this book for recall). The reader should be comfortable with the concepts seen in the section Set Theory as well as the one one Topology. We also apologize for the actual lack of figures. 
	\end{tcolorbox}
	\lettrine[lines=4]{\color{BrickRed}T}he measure, in the topological sense, will allow us to generalize the elementary notion of measure of a segment or area (in the Riemann sense, for example) and is inseparable from the new theory of integration that will build Lebesgue from the years to 1901-1902 and we will address here to build mathematical tools much more powerful than the simple Riemann integral (\SeeChapter{see section Differential and Integral Calculus page \pageref{riemann integral}}) with practical and numerical example in MATLAB\textsuperscript{TM}.
	
	The philosophers of science who developed measurement theory were largely concerned with epistemic questions like: we can't observe correlations between physical objects and real numbers, so how can the use of real numbers be justified in terms of things we can observe? Indeed, privileges a single unit of mass, involves real numbers in the facts of
mass. Why is the latter bad?: 
	\begin{enumerate}
		\item Real numbers are abstract and therefore causally inert
		
		\item Real numbers don't fundamentally exist
		
		\item Real numbers are constructed entities, and constructed entities can't be involved  in fundamental facts
	\end{enumerate}

	The Measure Theory will also allow us to rigorously define the concept of measurement (no matter what is the measure) and so return to the important results of the study of probabilities (\SeeChapter{see section Probabilities page \pageref{probabilities}}). Indeed, we will see (we will define the vocabulary that follows just now further below) why $(U, A, P)$ is a "\NewTerm{probability space}" where $A$ is in fact a "\NewTerm{tribe}" on $U$ and $P$ a measure on the measurable space $(U , A)$.
	
	\subsection{Measurable Spaces}
	When in mathematics we calculate derivatives, primitives or simply count stuff, we carry implicitly a measure of an object or set of objects. Rigorously, mathematicians want to define how the measured thing can be structured, how to make a measurement of it and the properties resulting!
	
	\textbf{Definitions (\#\mydef):}
	\begin{enumerate}
		\item[D1.] Let $E$ be a set, a "\NewTerm{tribe}" on $E$ is a family $\mathcal{A}$ (this notation comes from the fact that many people speak of "\textbf{A}lgebra sets" instead of "tribe") of subsets of $E$ satisfying the following axioms:
		\begin{enumerate}
			\item[A1.] $E\in \mathcal{A}$ (see examples below - $E$ being one of the possible elements of $\mathcal{A}$).
			
			\item[A2.]  If $A$ is a member of a tribe then:
			
			This means that $\mathcal{A} $ is "\NewTerm{stable by transition to complementary}". This axiom implies that the empty set is always an element of a tribe!
			
			\item[A3.] For any sequence $(A_n)$ of elements of $\mathcal{A}$ we have:
			
			 We then say that $\mathcal{A}$ is then "\NewTerm{stable by countable union}".
		\end{enumerate}
		For example, the graduating from a simple ruler of measurement ... satisfies these three axioms!
	\begin{tcolorbox}[title=Remarks,colframe=black,arc=10pt]
	\textbf{R1.} We write $E\in \mathcal{A}$ because we consider with this notation $E$ not anymore as a subset of $\mathcal{A}$  but as an element of $\mathcal{A}$!\\
	
	\textbf{R2.} The uncountable cases are typical of topology, statistics or integral calculus!
	\end{tcolorbox}	
	
	\item[D2.] The pair $(E,\mathcal{A})$ is named "\NewTerm{measurable space}" and we say that the elements of $\mathcal{A}$ are "\NewTerm{measurable sets}".
	
	\item[D3.] If in the third axiom we require that $\mathcal{A}$ is stable under finite (uncountable) union then we impose the more general notion of "\NewTerm{$\sigma$-algebra}\label{sigma algebra}". Thus, a tribe is necessarily contained in an $\sigma$-algebra (but the opposite is not true just because the axiom is stronger) such that we can write:
	
	\begin{tcolorbox}[title=Remark,colframe=black,arc=10pt]
	In the field of probabilities, $E$ is assimilated to the Universe of events and $\mathcal{A}$ to a family of events and we speak then of "\NewTerm{probabilistic space}" or simply of... "\NewTerm{measurable space}".
	\end{tcolorbox}	
	\end{enumerate}
	\begin{tcolorbox}[colframe=black,colback=white,sharp corners]
	\textbf{{\Large \ding{45}}Examples:}\\\\
	E1. Given $E=\{1,2\}$ a set of cardinal 2... The only two tribe $\mathcal{A}$ that satisfy the three axioms are:
	
	There are no other tribes for the set $E$ as these two (the normal one, and the maximum one), because we must not forget that the union of each elements of the tribe must also be in the tribe (axiom A3), and also the complement of a member (axiom A2).\\
	
	We also see from this example that if $E$ is set then $\{E,\varnothing\}$ is indeed a tribe!\\
	
	E2. The set of parts of $E$, denoted $\mathcal{P}(E)$ is also a tribe (dixit previous example).
	\end{tcolorbox}
	A tribe $\mathcal{A}$ is also "\NewTerm{stable by the union of the finite complementaries}". Indeed, if $(A_n)$ is a sequence of elements of $\mathcal{A}$ we have (trivial when taking for example the previous first example):
	
	A tribe is also "\NewTerm{stable by finite intersection}", that is to say (trivial also by taking the previous first example):
	
	which brings to the property that a tribe is stable by finite unions and intersections. Especially, if we take two elements of a tribe $A,B\in \mathcal{A}$, then $A\setminus B\in \mathcal{A}$ with for recall (\SeeChapter{see section Set Theory page \pageref{symmetric difference}}):
	
	\begin{tcolorbox}[title=Remark,colframe=black,arc=10pt]
	Most readers should probably easily see with the previous first example that if $(\mathcal{A}_i)_I$ is a family of tribes on $E$ the $\bigcap_I \mathcal{A}_i$ is also a tribe (the verification is almost immediate).
	\end{tcolorbox}	
	Well it is nice to play with potatoes and sub-potatoes... and their complementary but let us continue...
	
	\textbf{Definition (\#\mydef):} Given $E$ a set and $\mathcal{B}$ a family of subsets of $\mathcal{P}(E)$ such that $\mathcal{B}\subset \mathcal{P}(E)$. We denote by definition:
	
	the "\NewTerm{generated tribe}" by $\mathcal{B}$. Therefore $\sigma(\mathcal{B})$ is by definition the smallest tribe containing $\mathcal{B}$ (and by extension the smallest tribe of $E$).

	Below are three small examples that gives the opportunity to check if what precedes has been well understood and that also gives the possibility to highlight important results for what will follow:
	\begin{tcolorbox}[colframe=black,colback=white,sharp corners]
	\textbf{{\Large \ding{45}}Examples:}\\\\
	Given a set $E$ and $A\subset E,A \neq E$ and also $\mathcal{B}=\{A\}$ then (when $A$ is seen as a subset of $E$ as given by the statement of a family of subsets!):
	
	
	E2. If $\mathcal{A}$ is a tribe on $E$ then:
	

	E3. Given $E=\{1,2,3,4\}$ and $A=\{\{1,2\},{3}\}$ we then have (take care because now $A$ is a family of parts (subsets) and not only a unique subset!) the following generated tribe:
	
	Rather than determining this tribe by seeking the smallest tribe $\mathcal{P}(E)$ containing $A$ (which would be laborious) we play with the axioms defining a tribe to easily find it.
	
	So therefore we find well in $\sigma(A)$ at least the obligatory empty set $\{\varnothing\}$ and also:
	
	following the axiom A1 and:
	
	itself by the definition of $\sigma(A)$ and the complementaries of:
	
	following the axiom A2 and also the unions:
	
	following the axiom A3.
	\end{tcolorbox}
	\textbf{Definition (\#\mydef)}: Let $E$ be a topological space (\SeeChapter{see section Topology page \pageref{topological space}}). We denote by $\mathcal{B}(E)$ the tribe generated by the open sets of $E$. $\mathcal{B}(E)$ is named the "\NewTerm{borelian tribe}" of $E$. The elements of $\mathcal{B}(E)$ are named the "\NewTerm{borelians}" of $E$. 
	
	\begin{tcolorbox}[title=Remarks,colframe=black,arc=10pt]
	\textbf{R1.} The notion of borelian tribe is especially interesting because it is necessary for the definition of "Lebesgue tribe" and afterwards to the "Lebesgue measure" that will lead us to define the  famous "Lebesgue integral"!\\
	
	\textbf{R2.} The tribe $\mathcal{B}(E)$ being stable by going to the complementary, it also contains all closed subsets.\\
	
	\textbf{R3.} If $E$ is a topological space with a finite basis, $\mathcal{B}(A)$ in generated by the opens of the basis.
	\end{tcolorbox}	
	\begin{tcolorbox}[colframe=black,colback=white,sharp corners]
	\textbf{{\Large \ding{45}}Example:}\\\\
	If $\mathbb{R}$ designates the space provided of real numbers with the Euclidean topology (\SeeChapter{see section Topology page \pageref{euclidean topology}}), the family of open intervals with rational bounds is a "\NewTerm{countable base}" (given the bounds...) of $\mathbb{R}$ and therefore generates $\mathcal{B}(\mathbb{R})$ . Same thing for $\mathbb{R}^d$ with for countable basis the family of open spaces with rational bounds.
	\end{tcolorbox}
	\begin{theorem}
	Let us now consider a dense set (\SeeChapter{see section Topology page \pageref{dense set}}) in $\mathbb{R}$. The following families generate $\mathcal{B}(\mathbb{R})$:
	
	\end{theorem}
	\begin{dem}
	Given (the family of open subsets):
	
	We have obviously:
	
	Furthermore:
	
	Therefore the intervals of the type $[a,b[$ with $a$ and $b$ in $\mathcal{S}$ also belongs to $\sigma(\mathcal{F})$. Therefore, if we generalize, with $x<y$, it exists a sequence $(a_n)$ of elements of $\mathcal{S}$ decreasing to $x$ and a sequence $(b_n)$ of elements of $\mathcal{S}$ increasing to $y$ such that:
	
	which bring in the same way as $E\in \mathcal{A}$ that $\mathcal{B}(\mathbb{R})\subseteq\sigma(\mathcal{F})$. Other cases can be treated analogously.
	\begin{flushright}
		$\blacksquare$  Q.E.D.
	\end{flushright}
	\end{dem}
	\begin{theorem}
	Given $(E, \mathcal{A})$ a measurable space and $A\subseteq E$ (and $A\in \mathcal{A}$) (where $A$ is therefore considerate as a subset and non as an element!). The family $\{A\cap B| B\in \mathcal{A}\}$ is a tribe on $A$ named "\NewTerm{trace tribe}" of $\mathcal{A}$ on $A$, that we will denote by $A\cap \mathcal{A}$. Furthermore, if $A\in \mathcal{A}$, the trace tribe is formed by the measurable elements contained in $A$.
	\end{theorem}
	\begin{dem}
	We will do a proof by the example (... yes it is not a real proof...). For this we check the three points that define a tribe:
	\begin{enumerate}
		\item $A=(E\cap A) \Rightarrow A\in (A\cap \mathcal{A})$
		
		\item Given $B\in\mathcal{A},A \setminus (A\cap B)=A\setminus B=A \cap B^c=A \cap B^c$ and therefore $A\setminus (A\cap B)\in (A\cap \mathcal{A})$
		\begin{tcolorbox}[colframe=black,colback=white,sharp corners]
		\textbf{{\Large \ding{45}}Example:}\\\\
		Given $E=\{1,2,3\}$ then (a tribe among others - do not forget the stability by union!):
		
		Let us choose $A=\{1,2\},B=\{2,3\}$ (it is obvious that $\{A\cap B|B\in \mathcal{A}\}$ is a tribe on $A$). Then:
		
		and we have well $\{1\}\in A$ and also $A\in (A\cap \mathcal{A})$.
		\end{tcolorbox}
		
		\item Given $(A\cap B_n)$ a sequence of elements of $A\cap \mathcal{A}\; (B_n\in \mathcal{A})$ then:
		
		The last statement of the proposition will be supposed as obvious (if not, let us know!).
	\end{enumerate}
	\begin{flushright}
		$\blacksquare$  Q.E.D.
	\end{flushright}
	\end{dem}
	Given now $E$ a set, $\mathcal{C}$ a family of subsets of $E$ and $A\subseteq E$ non empty. We denote by $A\cap \mathcal{C}$, the trace $\mathcal{C}$ on $A$ and $\sigma_A(A\cap \mathcal{C})$ the tribe generated on $A$. Therefore:
	
	\begin{tcolorbox}[colframe=black,colback=white,sharp corners]
	\textbf{{\Large \ding{45}}Example:}\\\\
	Given the set $E=\{1,2,3,4\},\mathcal{C}=\{\{1,2\},\{3\}\},A=\{3,4\}$ then:
	
	and let us check that $A\cap \sigma(\mathcal{C})=\sigma_A(A\cap \mathcal{C})$:
	
	So the equality is satisfied!
	\end{tcolorbox}
	A trivial corollary of this equality is that if we consider a topological space $ E$ and $A\subseteq E$ with the induced topology, then:
	
	We will study more in details $\sigma$-algebra in measurement theory but first let us recall that a tribe (sometimes named "algebra of sets") on $E$ must satisfy the following properties:
	\begin{enumerate}
		\item[P1.] Has to contain $E$
		\item[P2.] Must be stable by the complementary
		\item[P3.] Must be stable by countable union or intersection
	\end{enumerate}
	and a $\sigma$-algebra on $E$ is less restrictive than a tribe as it has to satisfy:
	\begin{enumerate}
		\item[P1.] Has to contain $E$
		\item[P2.] Must be stable by the complementary
		\item[P3.] Must be stable by finite (uncountable) union or intersection
	\end{enumerate}
	Let us recall (\SeeChapter{see section Set Theory page \pageref{symmetric difference}}) that if we have $E$ that is a set, then for every $A,B\subseteq E$ we define the symmetric difference $A\Delta B$ between $A$ and $B$ by:
	
	Trivial properties are as follows:
	\begin{enumerate}
		\item[P1.] A $\sigma$-algebra is stable by symmetric difference ($A,B\in \mathcal{A}$ we have $A\Delta B\in \mathcal{A}$)
		
		\item[P2.] $A\Delta B=B\Delta A$
		\item[P3.] $A^c\Delta B^c=A\Delta B$
			
		\item[P4.] $A\Delta B=(A\cup B)\setminus (A\cap B)$
	\end{enumerate}
	\begin{theorem}
	If $\mathcal{B}$ is a $\sigma$-algebra over $E$, then $(\mathcal{B},\Delta,\cap)$ is a "\NewTerm{Boolean ring}" (or "Boolean algebra" but be careful with the term "algebra" here which can cause confusion with the corresponding structure in Set theory) with $\varnothing$ and $E$ as neutral "additive" element ($\Delta$) and respectively" multiplicative" ($\cap$).
	\end{theorem}
	\begin{tcolorbox}[title=Remark,colframe=black,arc=10pt]
	For reminders on the items listed in the preceding paragraph, the reader can refer to the section of Set Theory page \pageref{set theory} and the subsection of Boolean Algebra (\SeeChapter{see section Formal Logic Systems page \pageref{boolean algebra}}).
	\end{tcolorbox}	
	\begin{dem}
	The "addition" $\Delta$ is associative because developing we get (this can verified by an arrow diagram if needed - the "potatoes"):
	
	and the latter expression is stable by permutation (commutation) of $A$ and $C$ (same method of verification). Therefore:
	
	We check that $\varnothing$ is neutral with respect to the symmetric difference (the proof that $E$ is neutral with respect to inclusion is obvious). It is trivial that:
	
	$(\mathcal{B},\Delta,\varnothing)$ is therefore well an Abelian group with respect to the law $\Delta$ (symmetric difference).
	
	Finally $\cap$ is distributive with respect to $\Delta$. Indeed:
	
	What makes $(\mathcal{B},\Delta,\varnothing)$ is indeed a ring (furthermore of a commutative ring!).
	\begin{flushright}
		$\blacksquare$  Q.E.D.
	\end{flushright}
	\end{dem}

	\pagebreak
	\subsubsection{Monotone Classes}
	\textbf{Definition (\#\mydef)}: Let $E$ be a set. A "\NewTerm{monotone class}" on $E$ is a family $\mathcal{C}$ of subsets of $E$ satisfying the following axioms:
	\begin{enumerate}
		\item[A1.] $E\in \mathcal{C}$
		\item[A2.] $A,B\in \mathcal{C}$ and $A\subseteq B \Rightarrow \setminus A\in \mathcal{C}$
		\item[A3.] If $(A_n)$ is an increasing sequence (take care to the word "increasing"!) of elements of $\mathcal{C}$ then $\displaystyle\bigcup_{i=1}^{+\infty} A_i\in \mathcal{C}$ (stable by countable increasing union).
	\end{enumerate}
	\begin{tcolorbox}[title=Remarks,colframe=black,arc=10pt]
	\textbf{R1.} An increasing sequence of sets is: $A_1\subseteq A_2\subseteq A_3 ...$\\
	
	\textbf{R2.} The first two axioms imply that $\mathcal{C}$ is by complementary.\\
	
	\textbf{R3.} The three axioms together leads to that the monotonous class is stable by decreasing intersection. A way to check this to take the complement of each element of the increasing sequence to fall back on the decreasing sequence and vice versa.
	\end{tcolorbox}
	Every $\sigma$-algebra is a monotone class, because $\sigma$-algebras are closed under arbitrary countable unions and intersections.
	
	Therefore:
	
	In the same way as for the tribes, if we consider a family $(\mathcal{C}_i)_I$ of monotone class on $E$. Then $\bigcap_I \mathcal{C}_i$ is a monotone class (the proof is verified immediately by the three previous axioms).
	\begin{tcolorbox}[colframe=black,colback=white,sharp corners]
	\textbf{{\Large \ding{45}}Example:}\\\\
	Given $E$ a set, $\mathcal{P}(E)$ is a monotone class on $E$. More generally, a tribe is a monotone class.

	Equivalently to tribes, let us consider a set $E$ and $\mathcal{C}\subseteq \mathcal{P}(E)$. Given $\mathcal{S}$ the family of all monotone class containing $\mathcal{C}$, $\mathcal{S}$ is not empty because $\mathcal{P}(E)\in \mathcal{S}$. We denote by:
	
	the monotone class generated by $\mathcal{C}$. Therefore $\mathcal{M}(\mathcal{C})$ is the smallest monotone class containing $\mathcal{C}$ (and satisfying obviously the previous axioms).
	\end{tcolorbox}
	\begin{tcolorbox}[title=Remark,colframe=black,arc=10pt]
	If $E$ is a set and $\mathcal{C}\subseteq \mathcal{P}(E)$ then $\mathcal{C}\subseteq \sigma(\mathcal{C})$, as $\sigma(\mathcal{C})$ is a monotone class (and also a tribe) containing $\mathcal{C}$ and therefore contains also $\mathcal{M}(\mathcal{C})$ (see the examples with tribes).
	\end{tcolorbox}	
	 \begin{theorem}
	Given $E$ as set. If $\mathcal{C}$ is a family of parts of $E$ that we impose as stable by finite intersection then $\mathcal{C}=\sigma{\mathcal{C}}$ (we then have to prove that the smallest tribe of $\mathcal{C}$ is equal to the smallest monotone class of $\mathcal{C}$. If we do not impose that $\mathcal{C}$ is stable by finite intersection we would not have necessarily the equality!
	\end{theorem}
	\begin{dem}
	As already said: $\mathcal{M}(\mathcal{C})\subseteq \sigma(\mathcal{C})$ (which is trivial). We will prove first that $\mathcal{M}(\mathcal{C})$ is a tribe on $E$. For this it is sufficient to show that $\mathcal{M}(\mathcal{C})$ is (also) stable by countable union (and not necessarily by an increasing sequence of elements!).
	
	Let us considerate following families for the proof:
	
	By the previous definitions $\mathcal{M}_1\subseteq \mathcal{C}$ but $\mathcal{C}$ being (imposed) stable by finite intersections implies that $\mathcal{C}\subseteq \mathcal{M}_1$ and therefore (it is the same reasoning as for the tribes):
	
	$\mathcal{M}_1$ is a monotone class, indeed $E\in \mathcal{M}_1$, if $A_1,A_2 \in \mathcal{M}_1$ and that $A_1\subseteq A_2$ (second axiom) then:	
	
	and therefore (which supports the fact that the other elements $(A_n)$ satisfy the previous relation):
	
	If $(A_n)$ is an increasing sequence of elements of $\mathcal{M}_1$ then:
	
	as $(A_n\cap B)$ is an increasing sequence.
	
	Therefore $\mathcal{M}_1$ is indeed a monotone class and by $\mathcal{C}\subseteq \mathcal{M}_1 \subseteq \mathcal{M}(\mathcal{C})$, we therefore have:
	
	The latter equality implies $\mathcal{C}\subseteq \mathcal{M}_2$. As for $\mathcal{M}_1$, we show that  $\mathcal{M}_2$ is a monotone class and therefore $\mathcal{C}= \mathcal{M}_2$, which means by extension that $\mathcal{M}(\mathcal{C})$ is therefore stable by finite intersections.
	
	$\mathcal{M}(\mathcal{C})$ being stable by complementary this take us to that $\mathcal{M}(\mathcal{C})$  is, we just proved it, stable by finite unions (but we want to prove that it is stable by countable union!).
	
	Given now a sequence $(A_n)$ of elements of $\mathcal{M}(\mathcal{C})$. We consider the sequence:
	
	$(B_n)$ is an increasing sequence of elements of $\mathcal{M}(\mathcal{C})$, therefore:
	
	but:
	
	Therefore:
	
	Therefore $\mathcal{M}(\mathcal{C})$ is stable by countable union and finally $\mathcal{M}(\mathcal{C})$ is a tribe. But as $\mathcal{C}\subseteq \mathcal{M}(\mathcal{C})$ this brings us to $\mathcal{M}(\mathcal{C})=\sigma(\mathcal{C})$.
	\begin{flushright}
		$\blacksquare$  Q.E.D.
	\end{flushright}
	\end{dem}
	We will later see some important applications of this theorem (but first we want to improve the above text with figure and more simple and practical examples!).
	
	\begin{flushright}
	\begin{tabular}{l c}
	\circled{50} & \pbox{20cm}{\score{4}{5} \\ {\tiny 17 votes,  76.47\%}} 
	\end{tabular} 
	\end{flushright}
		
\chapter{Geometry}

	\textit{\textbf{Geometry is the mathematical discipline that focuses on the rigorous study of spaces and forms}}. (Larousse)
	\minitoc
	\pagebreak
	\input{Chapter_Geometry.tex}
	
   \chapter{Mechanics}

	\textit{\textbf{Mechanics is the branch of physics that relates to the study of forces and their actions in abstract form}}. (Larousse)
	\minitoc
	\input{Chapter_Mechanics.tex}
	
\chapter{Electromagnetism}

	\textit{\textbf{Electrodynamic is the field of physics that study the dynamic action of electric currents and the propagation of electromagnetic waves.}}
	\minitoc
	\pagebreak
		%to force start on odd page
	\newpage
	\thispagestyle{empty}
	\mbox{}	
	\section{Electrostatics}
	\lettrine[lines=4]{\color{BrickRed}S}o far we have focus only on the gravitational interaction and the characteristic quantity of matter, named "mass" associated with it. We discussed the electromagnetic interaction, analysing macroscopic phenomena such as friction, cohesion, elasticity, the forces of contact, etc. Now we look at electronic forces and the characteristic of matter named "\NewTerm{electric charge}\index{electric charge}" associated with them. The electromagnetic interaction binds matter in all its observable forms. It is this that holds the electrons to the nucleus in the atom, which holds together the atoms in molecules, molecules into objects and even your nose to your face...
	
	The "\NewTerm{electric charge}" generate the "\NewTerm{electric force}\index{electric force}" or "\NewTerm{Coulomb force}\index{Coulomb force}\label{coulomb force}" and we are only beginning to understand this force thanks to quantum field theory (see further below in this book). The electric charge is a fundamental concept, which can not be described in terms of more simple and fundamentals concepts. We know it by its effects and unfortunately not by what it is (this was also the case for the mass before the discovery of the Higgs Boson).
	
	Experience has also shown that even if the electric charge is an additive quantity such as the mass, however, it also has negative value (and not exclusively positive as know nowadays for the mass). Thus, in the current language, and as confirmed by experience, two identical sign electric charges repel and two opposite sign electric charges attract (we will see a schematic figure of this further below).
	\begin{figure}[H]
		\centering
		\includegraphics[scale=0.9]{img/electromagnetism/electrostatic_cat.jpg}
		\caption[]{Electric charges exist all around us. They can cause objects to be repelled from each other or to be attracted to each other (credit: modification of work by Sean McGrath)}
	\end{figure}
	Let us now see the classic force that is associated with the electric charge:
	
	\subsection{Electric Force}\label{electric force}
	It has experimentally been established by Coulomb that a reference point particle undergoes a force $\vec{F}$ of an intensity proportional to its charge $q$, when placed in the neighbourhood of one or more electrical charges $Q_i$ in a medium of absolute electrical permittivity $\varepsilon$ (permittivity to the electric field of course...!) given by in vector notation and non-relativistic:
	
	where $\vec{r}_i$ is the vector position of the sample charge $Q_i$ and $\vec{r}$ of the reference point charge particle relatively to a same orthonormal vector basis.
	
	In other words, two point electric charged particles attract or repel each other in a force directly proportional to their electric charge and inversely proportional to the square of the distance that separates them.
	
	In the case of a system with two particles separated by a distance $r$, we have the same simplified relation and we fall back on the most common form of the electric force or "\NewTerm{Coulomb force}\index{Coulomb force}" as given in most books (as scalar and non-relativistic):
	
	Hence recalling Newton's second law:
	
	Frequently, these relation is defined as the "\NewTerm{Coulomb's law}\index{Coulomb's law}" in most schools and admitted as unprovable. In fact, it is not! This relation can be proved as we will see in the study of quantum fields physics (\SeeChapter{see section Quantum Field Theory page \pageref{yukawa potential}}) using the Klein-Gordon equation in the context of a potential field with spherical symmetry (proof performed by Yukawa).
	
	\begin{tcolorbox}[title=Remark,colframe=black,arc=10pt]
	Don't forget that as $2\pi$ appears frequently in many theorems of physics and mathematics (more than $\pi$ alone), the value $4\pi$ is frequently denoted by $2\tau$ as by definition $\tau=2\pi$.
	\end{tcolorbox}
	In either formulation, Coulomb's law is fully accurate only when the objects are stationary, and remains approximately correct only for slow movement. These conditions are collectively known as the electrostatic approximation. When movement takes place, magnetic fields that alter the force on the two objects are produced. The magnetic interaction between moving charges may be thought of as a manifestation of the force from the electrostatic field but with Einstein's theory of relativity taken into consideration.
	\begin{tcolorbox}[title=Remark,colframe=black,arc=10pt]
	For the relativistic form of Coulomb's law, the reader is referred to the section Special Relativity where it is proved that (vector form):
	
	\end{tcolorbox}
	The value of electric permittivity in vacuum is in turn given experimentally by the "\NewTerm{dielectric constant}\index{dielectric constant}" or simple "\NewTerm{electric constant}\index{electric constant}":
	
	and relatively to medium considered, we define a "\NewTerm{relative dielectric permittivity}\index{relative dielectric permittivity}" $\varepsilon$ that makes it easier to determine the properties of a material with respect to the electric field so that we have the "\NewTerm{absolute electrical permittivity}\index{absolute electrical permittivity}":
	
	It should be mentioned that some authors define the permittivity of vacuum from the speed of light and the magnetic permeability of vacuum (\SeeChapter{see section Magnetostatics page \pageref{magnetic permeability of vacuum}}). Therefore, the value of the electrical permittivity of vacuum is obviously correct by definition. But this only makes sense once known the Maxwell's theory and this will be presented and proved later in the section Electrodynamics (we follow the steps in the historical scientific discoveries).
	
	 It also appears in the Coulomb force constant, as the "\NewTerm{Coulomb constant}\index{Coulomb constant}":
	  
	The factor into parentheses in:
	
	depends only on the distribution of charges $Q_i$ in the volume and the absolute electrical permittivity of the medium $\varepsilon$. Since its value varies from one place to another and depends on the position vector $\vec{r}$ of the reference electric charge, it forms a set of vectors, which property is this of a multitude of electrical field lines hence the use of term "\NewTerm{electric field}\index{electric field}".
	
	The set of these vectors $\vec{E}$ carries therefore the name "electric field", at the point $\vec{r}$ ,developed in the electric charge distribution $Q_i$:
	
	Engineers often use another notation that allows to characterize only the geometry of the field regardless the environment (medium) and for this purpose they introduce the concept of "\NewTerm{displacement field}\index{displacement field}":
	
	where $\chi$ is the "\NewTerm{electric susceptibility}\index{electric susceptibility}". We will meet this vector again in the section Electrodynamics during our synthesis of Maxwell's equations.
	
	Coulomb force acting on the sample charge $q$, is then written in a conventional way (hence the electric field can simple be interpreted as the force per unit charge!):
	
	One configuration is of particular interest: two separated point charges of opposite charge. In the limit of vanishing separation, it is named "dipole". Its field fundamentally differs from that of just a single charge even though it is just the sum of the charge. The dipole as a concept is extremely important throughout electrodynamics. It is applied for example explaining the emission of electromagnetic radiation or as a model for molecules.
	
	Let us first consider the case of opposite electric charges. For the given problem we have $\vec{r}_1=-d/2\vec{e}_x$ and $\vec{r}_2=d/2\vec{e}_x$. So the charges lie on the $x$-axis with a separation $d$. Remember that $\vec{e}_x$ is the unit vector in $x$ direction (\SeeChapter{see section Vector Calculus page \pageref{vector calculus}}). The electric field is then given by:
	
	Imagine we are interested in the magnitude and the direction of the field only about the line of the $y$-axis. To figure it out both, we simply calculate:
	
	We see that the electric field has only a component in $x$-direction. Because of the symmetric choice of the coordinate system we could have guessed this in the first place.
	
	The magnitude is therefore given by the norm of the electric field:
	
	Let us now consider the case of equal charges (positive for example):
	
	The direction of the field is in this case always parallel to the $y$-axis but changing sign at $y=0$. Its magnitude is given by:
	
	We find that for equal charges the magnitude of the electric field decreases for large $y$.
	
	With Maple 4.00b we can easily plot these two magnitudes:
	
	\texttt{>plot({1/(0.5\string^2+y\string^2)\string^(3/2),1/(y\string^2)*1/((1/(2*y))\string^2+1)\string^(3/2)},y=-5..5);}
	\begin{figure}[H]
		\centering
		\includegraphics{img/electromagnetism/dipole_profile.jpg}
		\caption{Dipole profile for $q_1=-q_2$ and $q_1=q_2$ along $y$-axis}
	\end{figure}
	This is how looks like some lines of electric field of two identical charges particles ($q_1=q_2$) with a nice perspective effect:
	\begin{figure}[H]
		\centering
		\includegraphics[scale=1]{img/electromagnetism/3d_dipole.jpg}
		\caption{3D dipole profile with $q_1=q_2$}
	\end{figure}
	
	\pagebreak
	\subsection{Electric Potential}\label{electric potential}
	 Given two points $A$ and $B$ in a region of space where there is an electric field $\vec{E}(x,y,z)$ and given $\Gamma$ a path connecting these two points. So, in the particular case where the source of a field $\vec{E}$ is a sphere or a punctual body and we put  a charge as its neighbourhood, we have for the work done by the force to move the charge from point $A$ to point $B$:
	 
	Moreover, this work is as we shall see, equivalent to the potential energy. We thus define the "\NewTerm{potential difference}\index{potential difference}" or simply the "\NewTerm{potential}\index{potential}" by the relation:
	
	and therefore:
	
	In other words, the "difference in electric potential" between two points (i.e., voltage) in a static electric field is defined as the work needed per unit of charge to move a test charge between the two points. 

	The work can then be written\label{electrostatic potential energy}:
	
	\begin{tcolorbox}[title=Remarks,colframe=black,arc=10pt]
	\textbf{R1.} The potential is often named "\NewTerm{voltage}\index{voltage}" by electricians, electrical engineers and other engineers and the unit of measurement of the potential that is the "\NewTerm{Volt}\index{Volt}" denoted by [V].\\
	
	\textbf{R2.} The potential difference can either exists between two terminals of opposite charge $(+, -)$, or between two terminals of the type $(+, \text{neutral})$ or also $(-, \text{neutral})$. The latter two cases represents typically the configuration used by trains, trams, storms and almost all electromechanical appliances.
	\end{tcolorbox}
	\begin{theorem}
	We will now show in the more general framework that exists that the stationary vector field $\vec{E}$ derivates from a potential field:
	\end{theorem}
	\begin{dem}
	Given a charge $Q$ located relative to a reference frame by the vector $\vec{r}_Q$. Then in each point of space there exists a field $\vec{E}$ such as:
	
	let us develop this expression:
	
	If $\vec{E}$ is a stationary potential field then, there must be a potential $\Phi(x,y,z)$ of this field which satisfies:
	
	Let us look to the potential $\Phi$ exists for a Coulomb field. Then we must have for the field in $x$:
	
	therefore:
	
	and if we do the same development for each component, we also get the same result. Thus the electric potential is a scalar field and not a vector field (the electric field is it obviously a vector field)!
	\begin{flushright}
		$\blacksquare$  Q.E.D.
	\end{flushright}
	\end{dem}
	The potential $\Phi(x,y,z)$ is named in the case of a Coulomb field of "\NewTerm{Coulomb potential}\index{Coulomb potential}" and is conventionally chosen such that:
	
	As we can see it by the prior-previous expression of $\Phi(x,y,z)$, $c^{te}$ is an arbitrary constant, which imposes in the case of absence of charges that:
	
	Which finally gives us:
	
	Giving for all components:
	
	that we write in a more condensed way\label{derivation of electric field by the potential}:
	
	A common concept that you can found in many electromagnetic compatibility standards for humans (especially) but also for machines is the "\NewTerm{step voltage}\index{step voltage}" that is just the difference in potential measured between two point distant of $1$ meter. In other words it's just the electric field value. Near some high-voltage installations, the step voltage can kill humans or other animals.
	
	The resistance of the human body is roughly between $1,000$ and $1,000$ ohms. Then we have the following typical table:
	\begin{figure}[H]
		\centering
		\includegraphics{img/electromagnetism/step_voltage_table.jpg}
		\caption{Various voltage values relative to human body health}
	\end{figure}
	Mobile phone 3G and 4G antenna seem to have a step voltage (depending on various countries regulations) between $11\;[\text{V}\cdot\text{m}^{-1}]$ and $58\;[\text{V}\cdot\text{m}^{-1}]$ at the day we write these lines. This is a higher value than the step voltage of a hair dryer, a toaster, a mixer, a fridge or even in comparison of Earth natural step voltage that is over land in average around\footnote{According to the study Atmospheric Electricity, Grupo de Eletricidade Atmosférica (ELAT), Instituto Nacional de Pesquisas Espaciais, June 2005.} $120\;[\text{V}\cdot\text{m}^{-1}]$.
	
	\begin{tcolorbox}[title=Remark,colframe=black,arc=10pt]
	The same developments and results (and those that follow) are applicable regarding the gravitational potential field. However, they rarely are made in the literature or schools because the human being does not control (at least until now) the gravitational field with an ease and intensity equivalent to that of the electric field...
	\end{tcolorbox}
	It therefore follows that:
	
	and so for example for a spherical symmetric potential (case that we will find in many other sections of this book), it comes:
	
	
	\pagebreak
	\subsubsection{Path Independence}
	Let us prove now that the potential difference between two points $A$ and $B$ is independent of the path $\Gamma$ travelled such as we did for the gravitational potential field in the section of Classical Mechanics.

	Given $\Gamma$ a path between two points $A$ and $B$ and a field $\vec{E}$ and let us make so that we can express the field on $x$, $y$ and $z$ with respect to a single variable $t$ (which has nothing to do with time...) that would account for its magnitude change for  any movement between these two points:
	
	So with $A$ corresponding to the value $t_1$ of the parametrization and $B$ to the value $t_2$. 
	
	But, we know that (\SeeChapter{see section Differential and Integral Calculus page \pageref{total exact differential}}) that:
	
	Therefore it comes:
	
	Therefore:
	
	This last expression shows that $U$ is independent of the path $\Gamma$ regardless of the way we parametrize it.
	
	The Coulomb field is therefore a "\NewTerm{conservative vector field}\index{conservative vector field}" Indeed, if we consider a closed path $\Gamma$ and $A$ and $B$ are two confused points of the part then the potential difference will be zero!
	
	Notice that sometimes we also say that the gradient of the potential is conservative.

	\pagebreak
	\subsection{Equipotential and Field lines}
	Now we can from what we have built, define the "equipotentials" and "field lines".
	
	Given a Coulomb field defined relatively to a given repository (reference frame). Then at each point $(x, y, z)$ of space, we can associate an electric field vector $\vec{E}(x,y,z)$ and an electric potential.
	
	\textbf{Definition (\#\mydef):} The "\NewTerm{field lines}\index{field lines}" is a family of curves for which the vector $\vec{E}(x,y,z)$ is tangent and constant at each point and the "\NewTerm{equipotentials}\index{equipotentials}\label{equipotentials}" as the lines for which the potential $U (x, y, z)$ are also constant.
	
	\begin{theorem}
	In this case, and that is what we will show, all field lines are perpendicular to all equipotentials if the vector field derived from a potential.
	\end{theorem}
	Let us use the property conservation of the Coulomb field for the proof:
	\begin{dem}
	
	As we are in the presence of an electric field, this therefore derives from a potential as we know. This implies that if the field is not null the potential there is also not. So in the line integral:
	
	one of the terms is zero! This is not the electric field $\vec{E}$ as we in presence of one, which discredits the potential $U$ and as the charge moves $\mathrm{d}\vec{r}$ is not null either. Then write the line integral in another way:
	
	Therefore:
	
	we can then conclude that the equipotential are perpendicular to the lines of the electric field and vice versa. This is what was to be proved.
	\begin{flushright}
		$\blacksquare$  Q.E.D.
	\end{flushright}
	\end{dem}
	Here are examples of level lines including field lines and equipotential lines obtained using Maple 4.00b (we will show in our study of differential equations how to get the mathematical functions of the field lines):
	\begin{figure}[H]
		\centering
		\includegraphics{img/electromagnetism/field_and_equpotential_lines_01.jpg}
		\caption[]{Left: a single charge - Right: two charges of the same sign}
		\includegraphics{img/electromagnetism/field_and_equpotential_lines_02.jpg}
		\caption[]{Left: two charges of opposite signs - Right: four charges of the same sign}
	\end{figure}
	\begin{tcolorbox}[title=Remark,colframe=black,arc=10pt]
	Apart from the opposite charges, we recall that the same results are applicable to the masses with the gravitational field!
	\end{tcolorbox}
	Two applications of these results are very important (for which we will limit ourselves to the study of the most important properties):
	\begin{enumerate}
		\item The determination of the field lines and equipotential lines for an infinite straight wire such as we can consider in approximation in the electric circuits or high voltage overhead lines (to determine the influence of fields from wires with their environment - this study is part of the branch of electro-engineering we name EMC for "\NewTerm{ElectroMagnetic Compatibility}\index{ElectroMagnetic Compatibility}"). The results can also be used to determine the "\NewTerm{voltage step}\index{voltage step}" for some rectilinear systems which determines for a given distance, the potential per meter for which a mammal may be killed by electroshock near such a wire. An extension (which I do not wish to deal even if the subject is exciting but very controversial) is also the influence of this type of potential on the functioning of the human brain in the case of the use of portable phones (antennas transmitting a potential) or about houses near high voltage power lines...
		\begin{tcolorbox}[title=Remark,colframe=black,arc=10pt]
		We will determine in the section of Magnetostatics the Biot and Savart law giving the magnetic field for such a wire carrying a given current intensity.
		\end{tcolorbox}
		
		\item The determination of the field lines and equipotential of an electric dipole has a huge importance in chemistry. We will also see what is the dynamics of it when immersed in a uniform electric field and the energy of interaction between dipoles (as is often the case in chemistry).
	\end{enumerate}
	
	\subsubsection{Infinite straight wire}
	Given:
	
	We have:
	
	by making use of the concept of linear charge density as we have defined it in the section of Principles of Mechanics, we have:
	
	Let us consider an infinite line (wire) of negligible section, and carrying a linear continuous density charge $\gamma$. The goal is then to calculate the electric field and potential at any point $M$ of the space outside this line (wire) in order to know the influences of charges of this line (wire) on the environment by considering the influence of the electric field (if the charges were moving we should also take into account the influence of the magnetic field, which we will do in the section Magnetostatics).
	
	For this, the method is to cut the line (wire) into small elements of line (wire) $\mathrm{d}l$, each element carrying a charge load $\mathrm{d}q$.  The electric field created by the charge load on $P$ at point $M$ located at a distance $x$ and of orthogonal projection $H$ on the line is:
	
	The trick now is to take the symmetric $P'$ of $P$ with respect to $H$ (the orthogonal projection of $M$ on the wire)
	\begin{figure}[H]
		\centering
		\includegraphics{img/electromagnetism/infinite_straight_wire.jpg}
		\caption{Configuration for the analysis of the electric field of an infinite straight wire}
	\end{figure}
	for which we have identically:
	
	The total field is therefore:
	
	But we have:
	
	Therefore:
	
	As we might expect, the latter relation shows that the field is perpendicular to the line (to the wire...).
	The norm of $\mathrm{d}\vec{E}(M)$ is:
	
	This relation has three dependent variables $r$, $\mathrm{d}l$, $x$. The norm of the total field on a point is equal to the sum of the norms of all the vectors $\mathrm{d}\vec{E}(M)$ of the entire length of the wire since all vectors have the same direction.
	
	For this calculation, we will make a change of variable, and put $r$, $\mathrm{d}l$, $x$ according to the angle $\alpha$ between the line (wire) and the vector $\overrightarrow{PM}$. In the rectangle triangle $HMP$:
	
	if we take the origin of $z$ in $H$. We also have:
	
	and:
	
	therefore:
	
	
	and therefore:
	
	The potential is easily deduced by taking the primitive of $E$ since:
	
	Then we have:
	
	The constant is undetermined since as $r$ approaches infinity, $U$ tends therefore to zero and leads to an infinite constant. This uncertainty is mainly due to the approximation of the infinite wire length.

	\subsubsection{Electric Rigid Dipole}\label{electric rigid dipole}
	An important and interesting disposition of electric charges is that forming an "\NewTerm{electric dipole}\index{electric dipole}" rigorously named "\NewTerm{rigid electric dipole}\index{rigid electric dipole}" or "\NewTerm{electrostatic dipole}\index{electrostatic dipole}". It consists of two equal and opposite electric charges $+ q, -q$ separated by a very small distance. We will seek to determine the potential and the electric field at a point $M$ of the dipole environment.
	
	To determine this let us consider an electric charge $q_i$ on a point $A_i$ and a very distant point $M$ from $A_i$. Let us take an Galilean reference frame in O:
	\begin{figure}[H]
		\centering
		\includegraphics{img/electromagnetism/elecrostatic_dipole_study_configuration.jpg}
		\caption{Electric field at a distant point $M$ of the dipole}
	\end{figure}
	The electric potential created at the $M$ by the charge $q_i$ is:
	
	In the triangle $\text{O}A_i M$, the distance $\overline{A_i M}$ can be written with the cosine theorem:
	
	The potential becomes:
	
	At great distance, $r$ becomes much higher than $r_i$, the quantity:
	
	tends to zero. So we can make a Maclaurin development (\SeeChapter{see section Sequences and Series page \pageref{usual maclaurin developments}}) of $(1+u)^{-1/2}$ at the neighbourhood of $u=0$. To avoid making heavy calculations, we will limited ourselves to the order two on $r$:
	
	therefore:
	
	Keeping only the terms of the second order in $r$:
	
	The potential becomes:
	
	We kept in the expression of the potential three terms. The term $U_i^0$ is the potential created by a charge that would be in O. In other words, at the zero order, the potential created by a charge located at a point near O is identical to the potential created by a charge that would be in O. The terms $U_i^1,U_i^2$ are correction terms, at the first order and second order respectively. We notice that these terms vary on $1/r^2$ and $1/r^3$, thus decreasing faster than the first. These two terms are thus more effective at smaller distances.
	We see that the terms $U_i^1,U_i^2$ involve the quantity $q_ir_i$. This quantity is what we define as the "\NewTerm{dipole moment}\index{dipole moment}" of the electric rigid dipole:
	
	\begin{tcolorbox}[title=Remark,colframe=black,arc=10pt]
	The dipole moment is expressed in Coulomb meter, but for convenience (...) it is expressed in Debye [D] by some engineers.
	\end{tcolorbox}
	The potential created at great distance by a discrete charge distribution is obtained by summing all individual contributions:
	
	Which can also be written:
	
	By definition, $U_0(M)$ is unipolar or monopolar term, $U_1(M)$ is the dipolar term, $U_2(M)$ is the quadrupolar term. If the electric charge distribution is a globally zero, as is the case of an ideal atom or of an ideal non-ionized molecule, it only remain the multipolar contributions.
	
	Let us return to the particular case of the dipole:
	
	The term monopolar term is zero, since the sum of the electric charges is zero. If we neglect the terms higher than the first order, there remain the dipole contribution.
	
	The angles $\theta_1$ and $\theta_2$ of the dipole are complementary, therefore $\cos(\theta_1)=-\cos(\theta_2)$. But as $q_1=-q_2=q$, the product $q_i\cos(\theta_i)$ is constant.
	
	If the two charges of the dipole are at a constant distance from each other and equidistant from the origin O. We will put that $r_1=r_2=d$.

	The potential is therefore reduce to:
	
	
	where $a$ is simply the constant distance between the two electric charges.
	It is customary in the case of the study of the electric dipole to write the above relation as:
	
	where $\vec{p}$ is the definition of the dipolar moment and:
	
	Let us recall now that we proved at the beginning of this section that:
	
	and as we saw in section of Vector Calculus, the gradient in spherical coordinates leads us to write:
	
	therefore:
	
	To determine the equation of the equipotential, remember that these lines (or "surfaces" when in space) are obtained through by the constraint:
	
	Therefore:
	
	with:
	
	The electric field must by definition be tangential to the field lines, thus parallel to the elementary displacement:
	
	Since $E_\phi=0$, we have:
	
	So finally there remains only:
	
	Which is a differential equation that can be easily integrate:
	
	Which is equivalent to write:
	
	The plot of the field and equipotential lines then gives in spherical coordinates (remember that the vertical component is zero by symmetry):
	\begin{figure}[H]
		\centering
		\includegraphics{img/electromagnetism/dipole_electric_equipotentials.jpg}
		\caption{Polar plot of the field lines and equipotentials of an electric dipole}
	\end{figure}
	Even if in an electric dipole the two charges are equal and opposite, giving a zero net charge, the fact they are slightly displaced is sufficient to produce a non identically zero electric field. In atoms, the center of mass of the electrons coincides with the core, and therefore the average electric dipole moment of the atom is zero. But if an external field is applied, the movement of electrons is distorted and the electron mass center is displaced by a given distance from the core. The atom is then polarised and becomes an electric dipole of moment $\vec{p}$. This moment is proportional to the external applied field $\vec{E}$.

	As we did not found in Maple 17.00 how to do the polar plot above here is another Maple 17.00  code to have fun with a dipole:
	
	\texttt{>V:=1/sqrt((x-1)\string^2+y\string^2+z\string^2)-2/sqrt((x+1)\string^2+y\string^2+z\string^2):\\
	>with(LinearAlgebra); with(VectorCalculus); with(plots);\\
	>with(plottools);\\
	>Efield := Gradient(-V, [x, y, z]);\\
	>fieldplot3d([Efield[1], Efield[2], Efield[3]],x =-1.5..1.5, y =-1.5..1.5, z =-1.5..1.5);\\
	>NormEfield := Normalize(Efield,2);\\
	>p1:=sphere([1, 0, 0],0.75,color=red);\\
	>p2:=sphere([-1, 0, 0],1.5,color=blue);\\
	>p3:=fieldplot3d(NormEfield,x=-4.5..4.5,y=-4.5..4.5,z =-4.5..4.5,color=black);\\
	>display([p1, p2, p3], scaling = constrained);
	}

	Giving at first (obviously the dimensions are fictitious):
	\begin{figure}[H]
		\centering
		\includegraphics{img/electromagnetism/dipole_electric_field_maple.jpg}
		\caption{Electric field lines for two charges of opposite signs with Maple 17.00}
	\end{figure}
	Or to get the view of the potential with the equipotentials:

	\texttt{>z:= 0;\\
	>plot3d(V,x =-1.5..1.5, y=-1.5..1.5, style=patchcontour,contours=200);
	}

	\begin{figure}[H]
		\centering
		\includegraphics{img/electromagnetism/dipole_potential_well.jpg}
		\caption{Potential well and equipotentials of the dipole with Maple 17.00}
	\end{figure}
	Or to get a 2D profile of the electric field and equipotentials:
	
	\texttt{>p4:=implicitplot({seq(V=(1/10)*b,b=-10..10)},x=-5..5,y=-4..4);\\
	>p5:=fieldplot([NormEfield[1],NormEfield[2]],x=-5 .. 5,y=-4..4); 
	\\>display([p4, p5], scaling = constrained);\\
	}
	\begin{figure}[H]
		\centering
		\includegraphics{img/electromagnetism/dipole_potential_field_profile_maple.jpg}
		\caption{2D profile of electric field and potential of the dipole with Maple 17.00}
	\end{figure}
	\begin{tcolorbox}[title=Remark,colframe=black,arc=10pt]
	Molecules also may have a permanent electric moment. Such molecules are named "\NewTerm{polar molecules}\index{polar molecules}" For example, in the HCl molecule the electron of the hydrogen atom spends more time to move around the chlorine that around the hydrogen atom. Also, the center of negative charges does he not coincide with that of the positive charge of the center and the molecule has a dipole moment. By cons, in the molecule $\text{CO}_2$, all the atoms are aligned, and the resulting electric dipole moment is zero for reasons of symmetry.
	\end{tcolorbox}
	When an electric dipole is positioned in an electric field, a force is exerted on each of the charges of the dipole. The resulting force is:
	
	Let us consider the particular case where the electric field is directed along the $x$-axis and where the dipole is oriented parallel to this field. If we only consider the quantities:
	
	with $a$ being the distance between the two charges, and therefore:
	dipole is oriented parallel to this field. If we only consider the quantities:
	
	This result show that an electric dipole oriented parallel to the field tends to move in the direction in which the field increases (as the gradient thereof). We note that if the electric field is uniform, the resultant force on the dipole is zero!!!
	
	The potential energy of the dipole is:
	
	If we use the equation:
	
	to describe the uniform electric field and if $\theta$ is the angle between the dipole and the electric field, the last factor $(U_+ - U_-)/a$ is just the component $E_a=E\cos(\theta)$ of the field $\vec{E}$ parallel to $a$. So:
	
	or:
	
	The potential energy is minimum for $\theta=0$, which shows that the dipole is in equilibrium when it is oriented parallel to the field.

	These configurations of a dipole placed in an electric field have very important applications. For example, the electric field of an ion in a solution polarized the solvent molecules that surround the ions and they are oriented as in the figure below:
	\begin{figure}[H]
		\centering
		\includegraphics{img/electromagnetism/dipole_solvant.jpg}
		\caption{Example of what happens in a solution with an ion}
	\end{figure}
	In a solvent with polar molecules such as water, ions of an electrolyte in solution are surrounded by a number of these molecules due to the dipole-charge interaction. This phenomenon is named "solvation" of the ion, specifically "hydration" if the solvent is water.
	
	For example with Salz (NaCl):
	\begin{figure}[H]
		\centering
		\includegraphics{img/electromagnetism/nacl_hydration.jpg}
		\caption[Hydration of Salz]{Hydration of Salz (source: ?)}
	\end{figure}
	These oriented molecules become more or less dependant of the ion, increasing its effective mass and decreasing its effective charge, which is partially obscured by the molecules (screening). The net effect is that the mobility of a ion in an external field is reduced. Similarly, when a gas or liquid, which  molecules are permanent dipoles is placed in an electric field, the molecules as a result of the force couple applied due to the electric field, tend to align with their dipoles parallel. We then say that the substance was "\NewTerm{polarized}\index{polarized}".

	It may therefore be interesting to determine the vector electric field produced by a dipole rather than its potential. The electrostatic field created at a point $M$ by the doublet is obtained by performing the vector sum of the fields created in this point by positive $P$ and negative $N$ charges, hence:
	
	The distribution of charges being invariant by rotation about the $z$ axis of the doublet, the topography is independent of the azimuthal angle $\pi$ of the spherical coordinates. We can then represent it in any meridian plane passing through the axis $NP$. The field $\vec{E}$ is the given by:
	
	Having:
	
	vectorially, we get:
	
	The dot product being the multiplication of components one by one, we have:
	
	Hence:
	
	Finally:
	
	So by a limited Maclaurin series development as we did at the beginning:
	
	By introducing:
	
	We can simplify the notations:
	
	It may also be relevant to calculate the energy of interaction between two electric dipoles. If we denote by $\vec{p}_1$ the dipole moment, we can write:
	
	If we denote by $\vec{p}_2$ the moment of the second dipole and if we use the relation:
	
	we find that the interaction energy between two dipoles is:
	
	We can draw out several important conclusions from this result. The energy of interaction $E_{P,1,2}$ is symmetric relatively to the two dipoles, because the permutation of $\vec{p}_1$ and of $\vec{p}_2$ leaves it unchanged. This is an expected result. The interaction between two dipoles is not central as it depends of the angles which the vector position or the unitary vector $\vec{u}_r$ do with $\vec{p}_1$ and $\vec{p}_2$.
	
	An atom, molecule or a ion, whose dipole moment is zero in the fundamental state, acquire a dipole moment under the action of the applied non-uniform electric field as we have seen since the opposing signs charges are solicited in opposed directions. The centers of gravity of positive and negative charges do not coincide anymore, it appears an "\NewTerm{induced dipole moment}\index{induced dipole moment}". 

	In an experiment with linear approximation for weak excitatory electric fields, the induced dipole moment is proportional to the applied field $\vec{E}$, which we translate by  (it is in fact an approximation of the Langevin-Debye relation we will prove later):
	
	The quantity $\alpha$, which physical dimension is that of a volume is the "\NewTerm{polarizability}\index{polarizability}" of the structure. The dipole-dipole electrostatic interaction was introduced by J.D. Van der Waals in 1873, in the case of molecules, to interpret real deviations from the ideal gas.
	
	The Van der Waals forces are repulsive as the distance between molecules is very low because they oppose the interpenetration of electron clouds, what we express by introducing their volume (covolume).

	By cons, they are attractive when the distance is sufficient. We attribute this attraction to three types of interactions involving rigid or induced dipoles:
	\begin{enumerate}
		\item The forces between polar molecules (rigid dipoles), say to be "\NewTerm{Keesom forces}\index{Keesom forces}".

		\item The forces between a polar molecule (rigid dipole) and a polarizable molecule (induced dipole) named "\NewTerm{Debye forces}\index{Debye forces}".

		\item	The average forces between induced dipoles which appear even when the molecules are not polar, named "\NewTerm{London forces}\index{London forces}".
	\end{enumerate}
	In all these three cases, the electrostatic energy is negative (attraction) and varies as $r^{-6}$. To prove this statement, let us calculate the interaction energy between two rigid dipoles, of dipolar moments $\vec{p}_1$ and $\vec{p}_2$:
	
	with:
	
	and:
	
	Therefore:
	
	This is the van der Waals interaction potential between two dipoles atoms where $C_6$ is the "\NewTerm{Van der Waals constant}\index{Van der Waals constant}".
	
	Hence:
	
	Thus, the radial dependence of the force is indeed in $r^{-7}$. This very rapid decay of the Van der Waals force with distance helps explain its very short range and therefore its influence when the medium is sufficiently dense. However the decay is slower than the chemical bond that decrease exponentially (\SeeChapter{see section Quantum Chemistry page \pageref{molecular chemistry}}).
	\begin{tcolorbox}[title=Remark,colframe=black,arc=10pt]
	The interaction between polar molecules, of the Keesom type, is made very high in the presence of hydrogen, because the latter, thanks to its small size, also interacts with the atoms of other molecules. It is that one which is at the origin of the "\NewTerm{hydrogen bonding}\index{hydrogen bonding}".
	\end{tcolorbox}
	
	\subsection{Electric Flux}
	Given $\vec{V}$ a vector field and $S$ a surface named "\NewTerm{Gauss surface}\index{Gauss surface}" in space. If we divide this area into a number $N$ of smaller surfaces $\mathrm{d}S$ each traversed by a field $\vec{V}_i$ and having a unit perpendicular vector $\vec{n}_i$ (special case) on their surface, then we can form the sum of:
	
	When $N$ tends to infinity and all $\mathrm{d}S$ to zero, we get for this sum:
	
	The value of this integral thus gives the flow $\Phi_V$ of the field $\vec{V}$ through the surface $S$ delimited by a domain $\Lambda$ and where:
	
	In the case of the electrostatic field, we write:
	
	This expression, define the "\NewTerm{electric flow}\index{electric flow}" or also named "\NewTerm{electric flux}\index{electric flux}".	
	
	The inevitable question that arises is then: what is its physical meaning? The flux of a fluid is the amount of fluid (especially the quantity of volume) passing through a surface by second. Then there is a flux (flow) of something. About the electrical flux (flow), from the classical point of view, nothing flows, the electric field is already established and is static, but is flow through the surface. The value of the electric field at any point in space is the field intensity at that point, while the flow can be considered the quantity of field which passes through the surface $S$. 

	There is a hundred years, physicists identified the flow with the number of electric field lines passing through the surface. But the least we can say is that the simplistic view that the field lines have a distinct reality and that we can count is misleading. We will see during our study of Quantum Field Theory, in the section of the same name, that the latter supports a virtual photons current that is the nature of electromagnetic interactions. Despite this, the physicists did not hurry to associate the flow of virtual photons of the 20th century to the continuous electric field lines of the 19th century. Whatever its nature, the concept of flow is powerful and of great practical use, both in electricity and magnetism.
	
	As will we prove it in the framework of Maxwell's equations (\SeeChapter{see section Electrodynamics page \pageref{maxwell equations}}), solving this integral gives in the general case (this is the "Gauss law for electrostatic\index{Gauss law for electrostatic}" also named "Gauss theorem for electrostatic"):
	
	In the case where the surface is not closed and reduce to a plane, the previous close surface integral is reduced to:
	
	
	\subsubsection{Capacitor}\label{capacitor}
	A capacitor (originally known as a condenser) is a passive two-terminal electrical component used to store electrical energy temporarily in an electric field. The forms of practical capacitors vary widely, but all contain at least two electrical conductors (plates) separated by a dielectric (i.e. an insulator that can store energy by becoming polarized). The conductors can be thin films, foils or sintered beads of metal or conductive electrolyte, etc. The non-conducting dielectric acts to increase the capacitor's charge capacity. Materials commonly used as dielectrics include glass, ceramic, plastic film, air, vacuum, paper, mica, and oxide layers. Capacitors are widely used as parts of electrical circuits in many common electrical devices. Unlike a resistor, an ideal capacitor does not dissipate energy. Instead, a capacitor stores energy in the form of an electrostatic field between its plates.

	When there is a potential difference across the conductors (e.g., when a capacitor is attached across a battery), an electric field develops across the dielectric, causing positive charge $+Q$ to collect on one plate and negative charge $-Q$ to collect on the other plate. If a battery has been attached to a capacitor for a sufficient amount of time, no current can flow through the capacitor. However, if a time-varying voltage is applied across the leads of the capacitor, a displacement current can flow.

An ideal capacitor is characterized by a single constant value, its capacitance. Capacitance is defined as the ratio of the electric charge $Q$ on each conductor to the potential difference $U$ between them. The SI unit of capacitance is the farad [F, which is equal to one coulomb per volt ($1$ [C$\cdot$V$^{-1}$]). Typical capacitance values range from about $1$ [pF] to about $1$ [mF].

	The larger the surface area of the "plates" (conductors) and the narrower the gap between them, the greater the capacitance is. In practice, the dielectric between the plates passes a small amount of leakage current and also has an electric field strength limit, known as the breakdown voltage. The conductors and leads introduce an undesired inductance and resistance.

	Capacitors are widely used in electronic circuits for blocking direct current while allowing alternating current to pass. In analogue filter networks, they smooth the output of power supplies. In resonant circuits they tune radios to particular frequencies. In electric power transmission systems, they stabilize voltage and power flow.
	\begin{figure}[H]
		\centering
		\includegraphics[scale=0.25]{img/electromagnetism/capacitors.jpg}
		\caption[Some capacitive dipoles]{Some capacitive dipoles (source: \url{http://www.e-style.ch}, author: Martin Bircher)}
	\end{figure}
	As a direct application of Gauss's theorem, very useful in electronics and for engineers, consider a large thin a flat sheet, wearing a homogeneous surfacic electronic charge density $\sigma$ and immersed in an environment of absolute electrical permittivity $\varepsilon$. In the area close to its center, the electric field resulting from all the fields of the fields is normal, uniform, constant and go away from the sheet. Let us consider a Gaussian surface with a cylinder shape limited by the bases $S_1=S_2=S$ and its tubular surface $S_3$ symmetrical with respect to the sheet. It therefore encloses an electric charge $\sigma S$. It follows that:
	
	and as $E=E_1=E_2$ and $E_3=0$, we find:
	
		Finally, the electric field of a large and thin loaded flat sheet is:
	
	If we put face to face two identical plates but with opposite charges, the algebraic sum will of course be:
	
	Excepted of the extremities where the side effect is important, the overall field is everywhere the vector sum of uniform fields from two opposing thin loaded layers. We name such a system a "\NewTerm{plane and parallel capacitor}\index{plane and parallel capacitor}":
	\begin{figure}[H]
		\centering
		\includegraphics{img/electromagnetism/plane_capacitor.jpg}
		\caption{Schematic diagram of plane and parallel capacitor}
	\end{figure}
	The result is remarkable because it is independent of the distance between the planes (in fact never forget it is an approximation for very small distances!). The calculation of electric potential is therefore simplified. Thus:
	
	Thus, the capacitance of the plane and parallel capacitor is consequently:
	
	Let us now recall that we have shown previously that:
	
	Since each plate of a parallel and plane capacitor contributes equally, the total electric field between the plates would be :
	
	The potential difference is:
	
	Solving for $Q$ yields:
	
	The plates are oppositely charged, so the attractive force $F_\text{att}$ between the two plates is equal to the electric field produced by one of the plates times the charge on the other:
	 
	This is therefore the force between the plates of a parallel (infinite) plate capacitor.
	
	The plates of a cylindrical capacitor are two infinite rolled cylinders (or very long relatively to their diameter), coaxial of respective radius $R_1$ and $R_2$. It is therefore the very important case of the coaxial cable (which dielectric is often polyethylene) that we can found in many laboratories (and not only!):
	\begin{figure}[H]
		\centering
		\includegraphics{img/electromagnetism/capacitors_cylindrical.jpg}
		\caption{Schematic diagram of cylindrical capacitor}
	\end{figure}
	Let us see a second academic example that is the "\NewTerm{cylindrical capacitor}\index{cylindrical capacitor}":

	By Gauss theorem, we know that:
	
	And since the field $\vec{E}$ is collinear at every point of the surface $\vec{S}$, it immediately comes by knowing the expression of the surface of the cylinder (\SeeChapter{see section Geometric Shapes page \pageref{cylinder}}):
	
	But:
	
	Therefore:
	
	Or often written in the form of the "\NewTerm{coaxial capacity per unit length}\index{coaxial capacity per unit length}\label{coaxial capacity per unit length}":
	
	Let us also calculate the capacity of a spherical capacitor which corresponds in a first approximation to some of Van Der Graaf generators that we have in the labs of some schools, museums or even research centers...
	
	A "\NewTerm{spherical capacitor}\index{spherical capacitor}" consists of two concentric spheres of radius $R_1$ and $R_2$ with $R_1<R_2$.
	\begin{figure}[H]
		\centering
		\includegraphics{img/electromagnetism/capacitors_spherical.jpg}
		\caption{Schematic diagram of spherical capacitor}
	\end{figure}
	We now have immediately:
	By Gauss theorem, we know that:
	
	And therefore since the field is assumed to be collinear at any point on the surface it comes immediately knowing the expression of the surface of a sphere:
	
	But:
	
	Then we have:
	
	Then we have:
	
	And that's all for the classical examples...
	
	Well! We have just seen that the capacity was defined as:
	
	or in alternative current notation (when the current or the voltage varies as function of the time, it is the usage to write them in lowercase):
	
	We then have for the instantaneous power (\SeeChapter{see section Electrokinetics page \pageref{electromotive force}}):	
	
	Assuming an ideal capacitor (which does not dissipate energy by Joule effect) we get:
	
	So by integration in a given time interval from $0$ to $t$ we have for the second term:
	
	This energy is always positive and is stored in electrostatic form in the capacitor.
	
	In the context of a sine wave alternative current, the average power will be equal to zero. We can generalize this by assuming that a perfect capacitor does not dissipate power no Joule effect.
	\begin{tcolorbox}[title=Remark,colframe=black,arc=10pt]
	Some scientific experiments requiring enormous energy use thousands of giant capacitors charged in the long term to accelerate particles or to make operate megajoules LASER. However we can not actually store the surplus of electric power of some power plants, which is why we use this surplus to pull back the water in the dams that can use their pool as a reserve of potential energy to produce a complement to electricity at the time of peak consumption (inverse transformation). We can also found huge capacitor that have for purpose to manage critical voltage collapse in a changing environment (this are the big cylinder in transformation stations).
	\end{tcolorbox}
	
	
	\paragraph{Dielectric strength}\mbox{}\\\\
	The term "\NewTerm{dielectric strength}\index{dielectric strength}" denoted $\mathcal{R}$ and that for SI units volts per meter [V$\cdot$m$^{-2}$] has the following meanings:
	\begin{itemize}
		\item Of an insulating material, the maximum electric field that a pure material can withstand under ideal conditions without breaking down (i.e., without experiencing failure of its insulating properties).
		
		\item For a specific configuration of dielectric material and electrodes, the minimum applied electric field (i.e., the applied voltage divided by electrode separation distance) that results in breakdown.
	\end{itemize}
	The theoretical dielectric strength of a material is an intrinsic property of the bulk material and is independent of the configuration of the material or the electrodes with which the field is applied. This "intrinsic dielectric strength" corresponds to what would be measured using pure materials under ideal laboratory conditions. At breakdown, the electric field frees bound electrons. If the applied electric field is sufficiently high, free electrons from background radiation may become accelerated to velocities that can liberate additional electrons during collisions with neutral atoms or molecules in a process named "avalanche breakdown". Breakdown occurs quite abruptly (typically in nanoseconds), resulting in the formation of an electrically conductive path and a disruptive discharge through the material. For solid materials, a breakdown event severely degrades, or even destroys, its insulating capability.

	The factors affecting apparent dielectric strength are typically:
	\begin{itemize}
		\item it decreases with increased sample thickness.
		\item it decreases with increased operating temperature.
		\item it decreases with increased frequency.
		\item for gases (e.g. nitrogen, sulphur hexafluoride) it normally decreases with increased humidity.
		\item for air, dielectric strength increases slightly as humidity increases
	\end{itemize}
	For a capacitor used in electronics, if we exceed the dielectric strength value, we observe the destruction of the element. This maximum value of the voltage applied to the terminals, is named sometimes the "\NewTerm{breakdown voltage}\index{breakdown voltage}" of the capacitor and denoted $U_c$ . We can define the dielectric strength of the material (medium) as:
	
	\begin{tcolorbox}[colframe=black,colback=white,sharp corners]
	\textbf{{\Large \ding{45}}Example:}\\\\
	Two common dielectric strength values that have to be know by students are:
	
	\end{tcolorbox}
	When we talk of dielectric strength, we also speak of the dielectric is an insulator or a substance that does not conduct electricity and is polarizable by an electric field. In most cases, the dielectric properties are due to the polarization of the substance. When the dielectric (in this case, air is the dielectric) is placed in an electric field, the electrons and protons of its atoms are reoriented and, in some cases, at the molecular level, a polarisation is induced (as we have seen in our study of the dipoles above). This polarization produces a potential difference, or voltage, between both terminals of the dielectric; it then stores the energy which becomes available when the electric field is removed. The effectiveness of a dielectric is its relative ability to store energy compared to that of vacuum. It is expressed by the relative electric permittivity $\varepsilon_r$, determined relative to that of a vacuum $\varepsilon_0$. The dielectric strength is then ability of a dielectric to withstand electric fields without losing its insulating properties. An effective dielectric releases a big fraction of the energy it had stored when the electric field is reversed.
	
	\subsection{Electrostatic potential energy}\label{electrostatic potential energy}
	Let us consider two electric charges $q_1,q_2$. The first is supposed to be at rest and fixed; the second is brought from infinity to a distance $a$ from $q_1$ (the same reasoning was applied to the gravitational field in the section of Classical Mechanics!). Suppose that the two charges are of the same sign. As $q_1,q_2$ tend to repel each other, we have to provide a potential energy $E_p$ to take $q_2$ (infinitely slowly) of $q_1$. The work $\mathrm{d}W$ done by the electrostatic force at any point is by definition as we know:
	
	The potential energy of the system is:
	
	as the force $F$ is resistive (hence the origin of the sign "$-$").

	Therefore:
	
	We then get the potential energy at a point ($x$ in the numerator is simplified with an $x$ in the denominator) to a given sign convention:
	
	If we have $q_2=Zq_1$ where $Z \in\mathbb{N}$ and we denote $q_1=e$ then we fall back on a well known way to write the previous relation:
	
	If the path is not linear we have obviously as in Classical Mechanics with the electric field and using the notations seen at the beginning of the section:
	
	Notice that:
	
	can also be written as:
	
	
	\begin{tcolorbox}[colback=red!5,borderline={1mm}{2mm}{red!5},arc=0mm,boxrule=0pt]
	\bcbombe Caution! When we do physics, we have to be sure what potential energy we are talking about. This is a major problem! If you take for example the potential energy due to gravitational force, it can take any value depending on the reference point. If the reference is at sea level, a point below the sea level will have a negative potential energy, by cons if the reference is the center of the Earth, there will be only positive potential energies. That is why we are writing rather the potential energy in the form of height difference relative to a reference in mechanics. For the potential energy of the electron, we must know with what it interacts. If it is with a negative charge, the product of the electric charge values will be positive and therefore the potential energy of interaction will be positive, if it interacts with a positive charge, the product will be negative and the potential energy will be negative. In short, we must know what we are talking about. This is an example where words are important in physics too (same problems as the foundations of mathematics!).\\
	
	In general, if the potential energy decreases when the distance increase, the force is repulsive if it increases when the distance increase, the force is attractive.
	\end{tcolorbox}
	
	\begin{flushright}
	\begin{tabular}{l c}
	\circled{90} & \pbox{20cm}{\score{3}{5} \\ {\tiny 75 votes,  66.13\%}} 
	\end{tabular} 
	\end{flushright}

	%to force start on odd page
	\newpage
	\thispagestyle{empty}
	\mbox{}	
	\section{Magnetostatics}
	\lettrine[lines=4]{\color{BrickRed}M}agnets are known since ancient times (without that it was known at that the origin of their properties) under the name "magnetite": black stones found near the city of Magnesia (Turkey). This is from this stone that also comes the current name of the "magnetic field". The Chinese were the first to use the properties of the individual magnets more than 1,000 years before for compasses. They consisted of a magnetite needle resting on straw floating on the water contained in a graduated container.
	
	Magnetostatics is the study of magnetic fields in systems where the currents are steady (not changing with time). It is the magnetic analogue of electrostatics, where the charges are stationary. The magnetization need not be static; the equations of magnetostatics can be used to predict fast magnetic switching events that occur on time scales of nanoseconds or less. It is even a good approximation when the currents are not static - as long as the currents do not alternate rapidly.
	
	As for the electric field, a good/better understanding of the origin of this field can be done through modern theories such as wave quantum physics and quantum field theory. The beginner reader will have therefore to be patient, as for the study of the electric field, until to get in this book the knowledge to study these modern theories.
	
	The quantitative study of the interactions between magnets and currents was made by the physicists Biot and Savart in 1820 only (nobody knew before that the compass was influenced by the current inside the core of the Earth). They measured the amplitude of the oscillations of a magnetic needle according to its distance to a straight current. They found that the force acting on a pole is directed perpendicular to the direction connecting the center to this conductor wire and it varies inversely with the distance. This is the first case that we will study.
	
	The idea of the experiment was the following:
	\begin{figure}[H]
		\centering
		\includegraphics[scale=0.9]{img/electromagnetism/biot_savart.jpg}
	\end{figure}
	\pagebreak
	Or in real life with a much simple configuration:
	\begin{figure}[H]
		\centering
		\includegraphics{img/electromagnetism/biot_savart_simple.jpg}
	\end{figure}
	Given a movement of electric charges (i.e.: a current $I$) generating in the space a vector field whose effects are measurable and whose properties differ from those of the electrostatic field. We deduce the existence of a new vector field we name (temporarily) "\NewTerm{magnetic field}\index{magnetic field}" and which we denote by $\vec{B}$.
	
	The physical units of the magnetic field will naturally been deduced from the time we will be able to connect this magnetic field to something known as a Force for example . This is what we will see further below during our study of the "Laplace's force law".
	
	The simplest case of study consisting of an indefinite straight conductor wire (example that we can also assimilate to a simple move of charges without having necessarily a wire to carry them) carrying a current $I$ (see the section Electrokinetics for the concept of "current") shows that the magnetic field lines are circles having for axis the conductor wire itself:
	\begin{figure}[H]
		\centering
		\includegraphics{img/electromagnetism/ampere_wire.jpg}
		\caption{Magnetic field around a straight infinite wire}
	\end{figure}
	The direction of $\vec{B}$ is usually defined by the intermittance of "\NewTerm{Ampère's observer}\index{Ampère's observer}", that is to say, an observer that would be positioned along the wire, so that the current goes to its feet towards its head and who would look at the point $M$ where we evaluate the magnetic field. The field $\vec{B}$ is directed from the right to the left of this observer. This is represented many times by the following had figure:
	\begin{figure}[H]
		\centering
		\includegraphics{img/electromagnetism/ampere_observer.jpg}
		\caption{Ampere observer hand rule}
	\end{figure}
	It is said that is was experimentally established by Biot and Savart in 1820 that the norm of the magnetic field $\vec{B}$ at the distance $r$ of the wire is proportional to the current $I$ running through it and inversely proportional to $r$ such that:
	
	This relation is traditionally considered as the pillar of the study of the magnetic field.
	
	The proportionality coefficient $k$ depends as always to selected unit system (as for other study fields). For the set of all its consequences, it is advantageous to write the above expression in a form that makes appear the perimeter of the circle of radius $r$. So we put:
	
	\begin{tcolorbox}[title=Remark,colframe=black,arc=10pt]
	Don't forget that as $2\pi$ appears frequently in many theorems of physics and mathematics (more than $\pi$ alone), this value is frequently denoted by the letter $\tau$.
	\end{tcolorbox}
	Thus we obtain the value of the magnetic field at a distance $r$ from a straight conductor wire through which a constant current $I$ pass through:
	
	where $\mu_0$ is a new constant which we name "\NewTerm{magnetic permeability of vacuum}\index{magnetic permeability of vacuum}\label{magnetic permeability of vacuum}" (again as for the electrical permittivity, there is a "\NewTerm{relative magnetic permeability}\index{relative magnetic permeability}") and whose value is given as usual in this book with the other physical constants in the section Principles of Mechanics chapter.
	
	The units of this constant, even if given in the section Principias of the Mechanics chapter, will be deducted automatically as soon as we have managed to link the magnetic field with the concept already known of "Force" (see below). This is what we will see when we will study the "\NewTerm{Laplace's force law}".
	
	\subsection{Ampere's theorem}\label{Ampère's law}
	In classical electromagnetism, the "\NewTerm{Ampère's theorem}\index{Ampère's theorem}" or "\NewTerm{Ampère's circuital law}\index{Ampère's circuital law}" or simply "\NewTerm{Ampère's law}\index{Ampère's law}" relates the integrated magnetic field around a closed loop to the electric current passing through the loop.
	
	It is interesting to calculate the "\NewTerm{circulation of the magnetic field}\index{circulation of the magnetic field}" $\vec{B}$ in vacuum (or not) along a contour $\Lambda$ that turns once in the positive direction (anti-clockwise) around the wire oriented in the direction of the current (Ampere observer):
	
	
	\begin{tcolorbox}[title=Remark,colframe=black,arc=10pt]
	The field is collinear along the path as we have seen previously hence the fact that the dot product can be written as a simple product of norms.
	\end{tcolorbox}
	We obtain then by definition the "\NewTerm{Ampère's law}\index{Ampère's law}" (or erroneously named "\NewTerm{Ampère's theorem}\index{Ampère's theorem}" because this result is not provable ... at least as far as we know...):
	
	where the current $I$ in a high symmetry system can be assimilated to a simple algebraic sum of currents snared by the path such that we have:
	
	Warning!!! This is not because the circulation of the magnetic field is zero in a region of space that the magnetic field is zero at any point!
	\begin{tcolorbox}[title=Remarks,colframe=black,arc=10pt]
	\textbf{R1.} The Ampere's law will give us the possibility to determine the fourth Maxwell equation that we will prove in the section of Electrodynamics.\\
	
	\textbf{R2.} The prior-previous relation is sometimes wrongly named "Ampere theorem" when in fact this result is not provable as already mentioned. Some physicists, however, use the fourth Maxwell equation to prove the prior-previous relation but then this is the snake biting its tail...
	\end{tcolorbox}
	The expression that we obtained can be simplified even more if we introduce a new physical being named "\NewTerm{magnetic field intensity}\index{magnetic field intensity}" or more commonly "\NewTerm{magnetic excitation}\index{magnetic excitation}" and which is denoted by the letter $\vec{H}$ (which is intrinsically independent of the propagation medium!).
	
	If we consider that we are always in the vacuum where there is no magnetic dipole then we define it in the vacuum by:
	
	Therefore, we are often taken to speak about "\NewTerm{magnetic induction}\index{magnetic induction}" for $\vec{B}$ and "\NewTerm{magnetic field}\index{magnetic field}" for $\vec{H}$. But both are happily confused following the authors/professors/teachers and especially depending on the contexts (as will be the case also in this book). When we deal with magnets that have intrinsic magnetization by the properties of the material that are made from, we denote distinctly the external magnetic field by:
	
	which is a more general form of the previous relation. It is therefore usual to define the "\NewTerm{magnetic susceptibility}\index{magnetic susceptibility}" as being the dimensionless ratio:
	
	Thus, the magnetic susceptibility indicates the amplitude with which a material responds magnetically to the presence of a magnetic excitation. We then get by this definition, the relation between relative magnetic permeability and magnetic susceptibility:
	
	Therefore:
	
	where $\mu$ is named the "\NewTerm{absolute magnetic permeability}\index{absolute magnetic permeability}".
	
	It is customary to name materials that have a positive magnetic susceptibility "\NewTerm{paramagnetic materials}\index{paramagnetic materials}" (contributing to increase the magnetic field) and those with a negative magnetic susceptibility "\NewTerm{diamagnetic materials}\index{diamagnetic materials}" (tend to oppose to the magnetic field ). We will see later the Langevin theoretical models to explain quantitatively with a relatively good approximation the two phenomena (in both cases the magnetic susceptibility has a value which is very low).
	
	So finally we can also write the Ampere's Law as following:
	
	The interest of the Ampere's law and of the concept of circulation of the magnetic field may appears like this more evident.
	
	This latter relation obviously is obviously very useful in theoretical physics because it will allow us to determine other important powerful results. Otherwise, in practice, the physicist or electrician/electrical engineer will often be faced with having to use electromagnets for small and medium experiences which he may wish to recalibrate the nominal values, or even solenoids.
	
	
	\subsubsection{Infinitely long solenoid}\label{solenoid}
	Also a particularly important application in electronics and electrical engineering is the calculation of the induction field in a coil of wire through which pass a current that we will consider as constant in a first time. This is nothing more than an induction coil more technically named an "\NewTerm{inductance}\index{inductance}". Let's see what it is:
	
	A solenoid is a coil of wire formed by a conductor wire wound helically through which pass a current of intensity $I$. In what follows, we assume that the induction field $\vec{B}$ of a solenoid is zero between the coils and parallel to the axis of the solenoid.
	
	Let us consider the following figure and let us look in approximation only to the inside part of the solenoid admitting that the external field is zero by the infinite length of it and that the joins of the coils are perfect (not letting dissipate any magnetic field)...:
	
	\begin{figure}[H]
		\centering
		\includegraphics{img/electromagnetism/infinite_solenoid.jpg}
		\caption{Infinite solenoid (coil of wire)}
	\end{figure}
	Let us apply the Ampere law to a rectangular path $abcd$. Therefore:
	
	The first integral of the right member gives $B\cdot h$ where $B$ is the norm of $\vec{B}$ inside the solenoid and $h$ the length of the segment $\overline{ab}$. We can notice that the segment $\overline{ab}$, even if it is parallel to the axis of the solenoid, don't need not coincide with it.
	
	The second and fourth integral is equal zero because for these two segments $\vec{B}$ and $\mathrm{d}\vec{l}$ are everywhere perpendicular: as $\vec{B}\circ\mathrm{d}\vec{l}$ is zero everywhere, the two integrals are equal to zero. The third integral is also equal to zero since the segment is calculated outside the solenoid where we assumed that the magnetic field is equal to zero as we consider the solenoid to be perfect.	
	
	Thus, the integral $\oint\vec{B}\circ\mathrm{d}\vec{l}$ for the entire rectangular path is equal to $B\cdot h$ such that:
	
	but the current $I$ is the sum of the currents $I_0$ passing through the $N$ turns contained in the path of integration. But in electronics, we used to work with the value $n$ (we choose the lowercase letter by analogy to Thermodynamics where lowercase letters represent densities) which is the number of turns per unit length:
	
	Therefore we have:
	
	That is:
	
	\begin{figure}[H]
		\centering
		\includegraphics{img/electromagnetism/solenoide_analogy_dipole_magnet.jpg}
		\caption[Linear solenoid analogy with (dipole) magnet]{Linear solenoid analogy with (dipole) magnet (source: ?)}
	\end{figure}
	Even if this relation has been established for an ideal infinite solenoid, it gives a pretty good magnitude (but not exact!) of magnetic induction field for interior points near the center of a real solenoid! This relation also shows that the magnetic field is independent in approximation of the solenoid diameter and that this latter is uniform across the section thereof!!! In laboratories, a solenoid is a convenient device for producing a uniform field induction in the same way that the plane capacitor is used to produce a uniform electric field.
	
	If we approach two solenoids (or magnetic needle) free to orient themselves we notice that :
	\begin{itemize}
		\item Two poles of the same nature repel each other
		\item Two poles of different nature attract
	\end{itemize}
	This property is used to determine the nature of the poles of any magnet: it suffices to approach a compass towards one of the ends of the magnet and observe which pole is attracted. The unknown magnet pole will then be of a different nature.
	
	Here is an illustration of the magnetic actions between two magnets:
	\begin{figure}[H]
		\centering
		\includegraphics[scale=0.8]{img/electromagnetism/magnets.jpg}
	\end{figure}
	
	\pagebreak
	\subsubsection{Toroidal coils}
	The toroidal coil is another important example of the application of Ampere's Law. Indeed, we especially find this configuration in low power electronics (e.g. computers) where the inductors are mostly toroidal or in the energy production with the famous Tokomak that are (very ...) schematically reduced to toroidal coils.
	\begin{figure}[H]
		\centering
		\includegraphics[scale=0.5]{img/electromagnetism/toroidal_coil.jpg}
		\caption{Toroidal coil}
	\end{figure}
	For symmetry reasons, it is clear that the magnetic induction lines form concentric circles inside the coil. Let us apply Ampere's law to the radius $r$ of integration of the circular path:
	
	That is to day:
	
	If follows then:
	
	Thus, unlike $B$ inside a solenoid, $B$ is not constant within the toroidal coil.
	\begin{figure}[H]
		\centering
		\includegraphics[scale=0.9]{img/electromagnetism/toroidal_field.jpg}
		\caption{Toroidal field}
	\end{figure}
	
	\pagebreak
	\subsubsection{Electromagnet}
	Let us determine for example (important and interesting one!) the magnetic field in the air gap of length $L_a$ and section $S_a$ of an electromagnet of length $L_{\text{Fe}}$ and section $S_{\text{Fe}}$ as shown below:
	\begin{figure}[H]
		\centering
		\includegraphics{img/electromagnetism/electromagnet_rectangular.jpg}
		\caption{Schematic rectangular Electromagnet}
	\end{figure}
	The Ampere's law gives us in vacuum:
	
	in the case of the electromagnet, we can write that the circulation of the field is the sum of the circulation of the air gap field and the magnet field itself:
	
	where $N$ corresponds to the number of current loops surrounding the magnet and which allows the production of the magnetic field.
	
	Let us recall that we have by definition:
	
	Therefore:
	
	If the airgap is not to big $L_\text{Fe}>>L_\text{airgap}$ we can write:
	
	Therefore:
	
	Finally:
	
	The relation is the same for an electromagnet having two coils! The reader will have perhaps notice on the way that this relation can also be used experimentally if we seek to determine the value of the relative magnetic permeability of the iron when all other parameters are known.
	
	\paragraph{Strength of a magnet or electromagnet}\mbox{}\\\\
	If we have know the norm of the magnetic field $B$ produced by a magnet at its surface, we can calculate with a certain approximation the force required to detach it from an iron surface.
	
	For this, we will denote by $F$ the necessary force to take off the magnet from a distance $d$ of an iron surface. We will assume the distance $d$ to be sufficiently small distance so that we can accept that the in the entire volume between the magnet and the iron, the magnetic field is constant.
	
	Thus, the work done by the force $F$ is (\SeeChapter{see section Classical Mechanics page \pageref{energy work power}}):
	
	This work has been turned into magnetic field energy in the volume created between the magnet and iron. The volumetric energy density due to the magnetic field of air being (\SeeChapter{see section Electrodynamics page \pageref{poynting vector}}):
	
	The volume of the space created between the magnet and the iron being equal to $V=Sd$ where $S$ is the surface of the magnet that was bonded to iron. We then have the following dimensional equivalence:
	
	Hence we deduce the contact force for small values of $d$:
	
	where $B$ is the limit value of the magnetic field which causes our material to stick to the magnet (so that by raising the magnet, the associated material follows).
	
	If we look at a lifting electromagnet of radius $0.75$ [m] capable of lifting $200$ [kg]:
	\begin{figure}[H]
		\centering
		\includegraphics{img/electromagnetism/lifting_electromagnet.jpg}
		\caption{Lifting electromagnet}
	\end{figure}
	Then we have:
	
	We can also use roughly the same calculation to determine the magnetic field of the electromagnet of the famous worldwide following playful toy known by people passionate in physics:
	\begin{figure}[H]
		\centering
		\includegraphics{img/electromagnetism/playful_magnetic_toy.jpg}
		\caption{Playful lifting electromagnet}
	\end{figure}
	
	\pagebreak
	\subsection{Maxwell-Ampere Relation}
	Given $\vec{J}$ the current density at any point of the space in the case of a three-dimensional distribution and let $S$ be a closed surface which is based on any contour $\Gamma$. The current $I$ that pass through $\Gamma$ is of course given by:
	
	According to Ampere's law, the flow of the magnetic field along $\Gamma$ is equal to this integral. It may therefore take here, given the selected contour $\Gamma$, an infinite continuous variable values. On the other hand, the Stokes' theorem (\SeeChapter{see section Vector Calculus page \pageref{stokes theorem}}) give us that:
	
	Therefore:
	
	and finally we conclude that:
	
	We can make a bold comparison of this result with the relation below (proven in the section Electrodynamics), by extension of the static and dynamic electric charge:
	
	which is nothing else than the first Maxwell equation (\SeeChapter{see section Electrodynamics page \pageref{first maxwell equation}}). Therefore, as we have seen in the section Electrostatics, we have:
	
	By analogy, the idea is to put (this assumption is verified a little further below in the text by the remarkable results obtained):
	
	relation that we can write in a more elegant way by assuming the current not dependent on the observer's position in space and collinear with perpendicular vector perpendicular of the crossing surface:
	
	where $\Gamma$ represents the perimeter of the wire in which the current $I$ flows.
	
	\subsection{Biot-Savart law}\label{biot savart law}
	Form the latest development, we get:
	
	Remember that at the last stage of our previous developments (we have specified it implicitly) the integration path is perpendicular to the electric current! But the magnetic field can not be zero at any point of the current line. Therefore, we are led to write what is hidden:
	
	The above relation allows us thus, by extension, to write in a more general form:
	 
	which is nothing else than the "\NewTerm{Biot-Savart law}\index{Biot-Savart law}" often presented first in high-school classes as starting point from the study of magnetism (originally it was experimentally determined in 1820 by Jean-Baptiste Biot and Félix Savart with the mathematical help of Simon de Laplace).
	
	This latter relation may as well be written (very important form):
	
	Therefore:
	
	Here we fall back on the non-relativistic approximation of the magnetic field as we have determined it in our study of Special Relativity (\SeeChapter{see section Special Relativity page \pageref{relativistic electrodynamics}}), where we have prove that:
	
	Another important form of the expression for the magnetic field is:
	
	As the current density $\vec{J}$ is collinear $\mathrm{d}\vec{l}$, we can write:
	
	Therefore:
	
	An important remark is necessary to our level of our discussion in the context of pre-university academic studies, mathematical formulations of magnetic and electric fields are considered as unprovable laws from which we later deduce the Maxwell's equations (furthermore the developments are not of the most aesthetic and rigorous type). The experimental aspect of such important relations can give a negative image of theoretical physics to students. It is therefore clear that at university, we have an approach somewhat less pragmatic.
	
	Indeed, we postulate the Schrödinger equation (\SeeChapter{see section Wave Quantum Physics page \pageref{classical one dimensional schrodinger equation}}) which we use to prove the non-relativistic formulation of Coulomb's law with the Yukawa theory (\SeeChapter{see section Quantum Field Theory page \pageref{yukawa potential}}). Then, during the study of relativity (\SeeChapter{see section Special Relativity page \pageref{relativistic electrodynamics}}), we determine the relativistic form of Coulomb's law. Then we admit the existence of the magnetic field whose expression is experimentally given by the Lorentz force (see further below in this section) and by the properties of Lorentz transformations and knowledge of the relativistic expression of the Coulomb's law, we determine the expression of relativistic magnetic field. Then, by non-relativistic approximation, we fall back on the Biot-Savart law. This approach is much more welcomed by students but not necessarily accessible to all levels.
	
	Now let us come back on the Biot-Savart law. An important example in astrophysics of the Biot-Savart law in the context of plasma jets accretion disks are the unique circular current loops (we also have to add to this the Lorentz force in the relativistic framework for understanding the dynamics of these jets).
	
	\subsubsection{Magnetic field for a current loop}
	The figure below is a good example of a current loop seen from profile:
	\begin{figure}[H]
		\centering
		\includegraphics{img/electromagnetism/current_loop.jpg}
		\caption{Magnetic field for a current loop}
	\end{figure}
	So we have a circular loop of radius $R$ traversed by an electric current of intensity $I$. The objective is to calculate $\vec{B}$ at a point $P$ on the axis of the loop.
	
	The vector $\mathrm{d}\vec{l}$ corresponding to an elementary electric current at the top of the loop exits perpendicularly from the plane of the page (screen). The angle $\theta$ between this vector and $\vec{r}$ is therefore $\pi/2$. The plane formed by $\mathrm{d}\vec{l}$ and $\vec{r}$ is normal to the figure. The vector $\mathrm{d}\vec{B}$ produced by this elementary electric current is normal to this plane by the form of the Biot-Savart law. It is therefore in the plane of the figure and at a right angle with the vector $\vec{r}$ as shown on the figure.
	
	Let us decompose $\mathrm{d}\vec{B}$ into two parts: the first $\mathrm{d}\vec{B}_{||}$ is along the axis of the loop and the second $\mathrm{d}\vec{B}_{\perp}$ is perpendicular to this axis. Only the component $\mathrm{d}\vec{B}_{||}$ contributes to the total magnetic induction at the point $P$. This is because the components $\mathrm{d}\vec{B}_{||}$ of all elementary currents are positioned on the axis and they are added directly. As for the components $\mathrm{d}\vec{B}_{\perp}$, they are directed in different directions perpendicular to that axis so that, by symmetry, their contribution is zero on this axis (be really careful with this special case!).
	
	 We get:
	
	It is a scalar integral performed on all elementary currents. We get by the Biot-Savart law:
	
	In addition, we have according to the figure above:
	
	Combining these relations, we get:
	
	The figure reveals that $r$ and $\alpha$ are not independent variables. We can express them in function of the new variable $x$, the distance between the center of the loop and the point $P$. The relations between these variables are:
	
	Substituting these values in the expression of $\mathrm{d}B_{||}$, we obtain:
	
	We notice that for all the elementary currents, $I$, $R$, $x$ have respectively the same values. Integrating this differential gives:
	
	An important point of this relation is therefore $x=0$ we get:
	
	Another important application of the Biot-Savart law consists to take the previous example, but for any continuous falt shape and considered as punctual and we would like to know the value of the field elsewhere than on the axis of symmetry . The results will be very useful when we will study the Corpuscular Quantum Physics and therefore the magnetic properties of metals.
	
	\subsubsection{Magnetic field for an infinite wire}\label{magnetic field for an infinite wire}
	Let us also prove (it is an interesting example!) that from the Biot-Savart law:
	
	we can get for an infinite straight wire of radius $R$ the relation:
	
	for $r>R$ that we get already earlier with the Ampere's theorem (which shows the equivalence between the two ways of calculating!).
	
	Let us choose for the infinite straight wire below $x$ as variable:
	\begin{figure}[H]
		\centering
		\includegraphics{img/electromagnetism/infinite_straight_wire_2.jpg}
		\caption{Infinite straight wire}
	\end{figure}
	We then have from the figure above:
	
	hence:
	
	By integrating:
	
	For the remaining part of the development, the trick is to use the following configuration:
	\begin{figure}[H]
		\centering
		\includegraphics{img/electromagnetism/infinite_straight_wire_3.jpg}
		\caption{Piece of the infinite wire}
	\end{figure}
	Therefore:
	
	Which gives us:
	
	After simplification:
	
	and so when the wire length tends is very big relatively to $l$ we get for the magnetic field outside the wire ($d>R$):
	
	Inside of the cable, the circulation integral around a circular path of radius, $r<R$, is the same:
	
	However, in this case, the smaller Amperian loop does not enclose all of the current flowing through the cable. We are told that the current density, $J$, is uniform in the cable. We can thus determine the current per unit area (i.e. the current density) that flows through the whole cable, and use that to determine how much current flows through the surface with area $\pi r^{2}$ that is defined by the Amperian loop:
	
	And by definition:
	
	Finally, we can apply Ampère's Law to determine the magnitude of the magnetic field inside the cable:
	
	Therefore:
	
	Hence:
	
	and we find that the magnetic field is zero at the center of the cable $(r=0)$, and increases linearly up to the edge of the cable $(r=R)$.
	
	\pagebreak
	\subsection{Magnetic dipole}
	The magnetic dipole has as its electrostatic counterpart enormous importance in the study of magnetic properties of materials for which it allows to develop good theoretical models.
	
	Before reading what follows, we strongly advise the reader (this even more than an advice) to read absolutely everything about the development of the rigid electric dipole moment in the section of Electrostatic. Indeed, most calculations that follow have the same reasoning, mathematical developments and approximations to some tiny nuances. We do not therefore wish to make the same intermediate calculations already present when calculating the electric dipole moment (however, if you really have trouble on reading what follow, we are ready to complete the missing as always in this book).
	
	A magnetic dipole is the limit of either a closed loop of electric current or a pair of poles as the dimensions of the source are reduced to zero while keeping the magnetic moment constant. It is a magnetic analogue of the electric dipole, but the analogy is not complete. In particular, a magnetic monopole, the magnetic analogue of an electric charge, has never been observed. Moreover, one form of magnetic dipole moment is associated with a fundamental quantum property: the spin of elementary particles.
	\begin{figure}[H]
	\centering
	\begin{subfigure}{.5\textwidth}
	  \centering
	  \includegraphics[scale=0.5]{img/electromagnetism/magnetic_dipole_as_a_current_loop.jpg}
	  \caption[Magnetic dipole as a current loop]{Magnetic dipole as a current loop (source: Wikipedia)}
	\end{subfigure}
	\begin{subfigure}{.5\textwidth}
	  \centering
	  \includegraphics[scale=0.7]{img/electromagnetism/magnetic_dipole_as_a_magnet.jpg}
	  \caption{Magnetic dipole as a magnet}
	\end{subfigure}
	\caption{Microscopic dipoles point of views}
	\end{figure}
	Some planets as the Earth have a magnetic field similar to magnetic dipole (but quite not homogeneous and not static):
	\begin{figure}[H]
		\centering
		\includegraphics{img/electromagnetism/magnetic_dipole_as_earth.jpg}
		\caption{Earth view as a magnetic dipole}
	\end{figure}
	
	The magnetic dipole, however, has a significant difference in relation to the practical case we impose us as part of its study ... there are no two electric charges! Indeed, electric charges at rest emit in a first approximation (this is experimental and theoretical fact...) an intrinsic magnetic field far too low to be considered interesting in the context of the study of the magnetic properties of materials. However, we should clarify something interesting (very cool), the elementary Coulomb electric charges are modelled sometimes (wrongly!) by physicists as a sphere rotating on itself (the "spin") and are represented as a superposition of small circular loops (oh... a loop we already know something like this) that are infinitely small so an observer in the rest frame (center of the electric charge) can interpret the overall Coulomb electric charge as a current moving in the different current loops, thereby inducing an intrinsic magnetic field (not pretty!?).
	
	In short, let us consider a flat loop (hep... another loop again...), of any shape, center on O, through which travel a permanent and constant current $I$ and where one of the points of the loop is denoted by $P$. We want to calculate the magnetic field generated by this coil at any point $M$ in space, located at a distance $r$ of the loop (specifically, at large distances compared to the size of the loop).
	\begin{tcolorbox}[title=Remark,colframe=black,arc=10pt]
	Personally there are some steps in the calculation that I find... well... very far unconvincing... but... there are so many approximations anyway that these problematic steps can bee see as details only... hummm...
	\end{tcolorbox}
	We put:
	
	\begin{figure}[H]
		\centering
		\includegraphics{img/electromagnetism/dipole_configuration_study.jpg}
		\caption{Magnetic dipole study configuration}
	\end{figure}
	So we will use the Biot-Savart law:
	
	under the assumption that the point $M$ is located at great distance from the loop. What gives us the right to write:
	
	But as $\mathrm{d}\vec{l}=\mathrm{d}\vec{\rho}$ therefore:
	
	Let us evaluate the term $\vec{r}'/r^3$ for points $M$ located far away from the current loop (at the denominator we used the cosine theorem as during our study of the rigid electric dipole moment in the section Electrostatics):
	
	where we did like the rigid electric dipole moment a Taylor limited development to order $1$. 
	\begin{tcolorbox}[title=Remark,colframe=black,arc=10pt]
	The last approximation is very rough in the sense that it is a clever choice of the terms to neglect to achieve a visually aesthetic result and to define the magnetic dipole moment (see further below) ...
	\end{tcolorbox}
	Putting this expression into the Biot-Savart law, we get:
	
	Let us evaluate each term involved in the parenthesis separately:
	\begin{enumerate}
		\item $\displaystyle\oint\mathrm{d}\vec{\rho}\times\vec{u}=\left(\oint\mathrm{d}\vec{\rho}\right)\times\vec{u}=\left(\vec{\rho}(P_0)-\vec{\rho}(P_0)\right)\times\vec{u}=\vec{0}\times\vec{u}=\vec{0}$

	since the vector $\vec{u}$ is independent of the point $P$ on the current loop and we do a curvilinear integration along the entire turn, returning to the starting point.

		\item $-\displaystyle\dfrac{1}{r}\oint \mathrm{d}\vec{\rho}\times\vec{\rho}$

		By the properties of the vector product:
		
		But since $\mathrm{d}\vec{\rho}$ and $\vec{\rho}$ are perpendicular and in a same plane, we have $\vec{\rho}\times\mathrm{d}\vec{\rho}$ which is the infinitesimal surface $\mathrm{d}S'$ of a rectangle and it represents nothing as the 
abscissa is curved relatively $O$. Indeed:
		\begin{figure}[H]
			\centering
			\includegraphics{img/electromagnetism/perpendicularite_vectors_magnetic_dipole.jpg}
		\end{figure}
		So we can write:
		
		where $\vec{n}$ is the vector normal to the plane of the loop (base vector of the $z$-axis). This result is generally valid regardless of the surface.

		Hence:
		
		
		\item $\displaystyle\oint\mathrm{d}\vec{\rho}\times\vec{u}(\vec{u}\circ\vec{\rho})=-\vec{u}\times\oint\mathrm{d}\vec{\rho}(\vec{u}\circ\vec{\rho})$ by the properties of the cross product (\SeeChapter{see section Vector Calculus page \pageref{cross product}}).
	\end{enumerate}
	We will use these relations to calculate the unknown integral of the start of our study. If we decompose the vector $\vec{\rho}$ and $\vec{u}$ in the base $(\vec{e}_1,\vec{e}_2)$ generating the plane of the loop, we get:
	
	as $\mathrm{d}\rho_3=0$ and $\rho_3=0$.

	We also have:
	
	Hence:
	
	Let us recall that:
	
	In the form of components (only the third term is non-zero since $\mathrm{d}\rho_3=0$ and $\rho_3=0$), we have:
	
	hence:
	
	Which brings us to write:
	
	Therefore:
	
	We notice that the latter relation is equal to:
	
	Then finally:
	
	By bringing together these results we obtain for the magnetic field:
	
	We see therefore an important quantity that appears because it fully describes the spire seen from a great distance, namely the "\NewTerm{magnetic local dipole moment}\index{magnetic dipole moment}\label{magnetic local dipole moment}" or simply "\NewTerm{magnetic dipole moment}":
	
	also often denoted by $\vec{\mathcal{M}}$ by some practitioners. The magnetic moment has then for units $[\text{A}\cdot\text{m}^2]=[\text{J}\cdot\text{T}^{-1}]=[\text{N}\cdot\text{m}\cdot \text{T}^{-1}]$. 
	
	We then have the prior-previous relation that can be written:
	
	By making use of the property of the double vector product (\SeeChapter{see section Vector Calculus page \pageref{grassman rule}}):
	
	We then get another form of expression of the approximate magnetic field created by a dipole:
	
	It is in the latter form that we found the most often the expression of the magnetic moment in the literature.
	
	This relation has to be compared (for fun) with the expression for the electric field for a rigid electric dipole:
	
	and thus we see that there is perfect correspondence.
	
	We still arrived to put this in a fairly identical and aesthetics and form after some approximations...

	We have also:
	
	Therefore:
	
	The origin of the magnetic field of any material must be microscopic. Using the Bohr model of the atom (\SeeChapter{see section Corpuscular Quantum Physics page \pageref{bohr model}}) we can convince ourselves that atoms (at least some) have an intrinsic magnetic dipole moment. Indeed, the Bohr model of the Hydrogen atom consists of an electron of electric charge $q=-e$ in circular motion around a central nucleus (a proton with an electric charge $q=+e$) with a period $T=2\pi\omega$.
	
	If we look over long time scales compared to $T$, it is as if there was an electric current:
	
	So we have indeed a sort of circular loop, of mean radius equal to the average distance to the proton, ie the Bohr radius $r_0$ (\SeeChapter{see section Corpuscular Quantum Physics page \pageref{bohr radius}}). The Hydrogen atom therefore have an intrinsic magnetic moment of:
	
	where $\vec{b}$ is the momentum of the electron and $q / 2m$ the "\NewTerm{gyromagnetic factor}\index{gyromagnetic factor}" (this result is very important for the Langevin model of diamagnetism!). This reasoning can be generalized to other atoms. Indeed, a set of electric charges in rotation about an axis will produce a magnetic moment proportional to the total angular momentum. This happens even if the total electric charge is zero (material or neutral atom): what matters is the (scalar) existence of a current.

	\begin{tcolorbox}[title=Remark,colframe=black,arc=10pt]
	We deduce then immediately that for a non-conductive ring of radius $R$ having a linear charge density $\rho$, turning at a rate $\omega=2\pi/T$ we then have:
	 
	So this is the magnetic moment of a rotating non-conductive ring!
	\end{tcolorbox}	
	
	So in fact, we can qualitatively explain the magnetic properties of materials based on the orientation of the magnetic moments of the atoms that compose them:
	\begin{itemize}
		\item "\NewTerm{Non-magnetic materials}\index{non-magnetic materials}": These are materials where the magnetic moments are randomly distributed, there is no intrinsic magnetic field.

		\item "\NewTerm{Diamagnetic materials}\index{diamagnetic materials}": these are the materials that subjected to a magnetic field, have their magnetic moments that opposed to it and are (marginally)  repelled by magnets. They therefore induce a magnetic moment opposite to the magnetic field direction.
		
		\item "\NewTerm{Paramagnetic materials}\index{paramagnetic materials}": it is the materials for which moments can be oriented in the direction of an external magnetic field and can therefore be magnetized so (attracted) momentarily. They therefore induce a magnetic moment in the direction of the magnetic field.

		\item "\NewTerm{Ferromagnetic materials}\index{Ferromagnetic materials}": are materials whose moments are already oriented in a particular direction, permanently (natural magnets).
	\end{itemize}
	\begin{tcolorbox}[title=Remark,colframe=black,arc=10pt]
	Earth is known to have a dipolar magnetic field where the magnetic north pole corresponds to the geographic South Pole (at a given angle value). At the macroscopic level, the explanation for the existence of the magnetic field observed on the stars is still far from satisfactory. The theory of "dynamo effect" trying to explain the fields observed by the presence of currents, mainly azimuthal, in the core of stars. Several known facts remain partially unsolved:
	\begin{itemize}
		\item The magnetic cycles: the Sun has a large-scale magnetic field similar to that of Earth, approximately dipolar. However, there is a polarity inversion every $11$ years. For the Earth, geophysicists were able to show that there had been a reversal about $700,000$ years before.
		
		\item Non-alignment with the angular momentum of the celestial body: it is of the order of ten degrees for the Earth, it is perpendicular for Neptune!
	\end{itemize}
	\end{tcolorbox}
		Let us consider now a non-conducting disc of radius $R$, charge $Q$ and uniform charge density $\sigma$ is rotating about an axis passing through its center and perpendicular to its plane with an angular velocity $\omega$. To determine it's magnetic moment (not a bad approximation of a non-relativistic Black Hole accretion disc), we divide the disk into many small rings of width $\mathrm{d}r$. The current produced by the ring is the total charge contained divided by the period for one rotation. Thus:
	
	Factoring in the area of each ring gives
	
	Therefore:
	
	Or in vector form we have then the magnetic moment of a rotating disc:
	
	
	Let us now consider a non-conductive sphere shell of radius $R$ rotating a constant speed $\omega$ around the $Z$ axis with a uniform surface electric charge $\sigma$
	
	We split the sphere into thin segments (each corresponding to a ring), each one having for magnetic moment:
	
	where $\mathrm{d}S$ is the area enclosed by the segment (keep in mind that the projected radius is $r=R\sin(\theta)$):
	
	The current $\mathrm{d}I$ on each segment is :
	
	The charge on each segment is:
	
	Where $\mathrm{d}s$ is the surface area of the segment (\SeeChapter{see section Geometric Shapes page \pageref{sphere}}):
	
	This results in:
	
	We now have an expression for $\mathrm{d}I$:
	
	Now we can calculate the moment of each segment:
	
	Integrating this over the whole sphere gives:
	
	Hence the magnetic moment for a spherical shell:
	
	\begin{tcolorbox}[title=Remark,colframe=black,arc=10pt]
	It seems that the result is the same for a ball instead than for a spherical shell. If it's true, it's a useful example for some special exotic astronomical objects.
	\end{tcolorbox}
	
	To see a last pretty practical case of the magnetic dipole, remember that in the section Electrostatics, we had shown that in spherical coordinates we had for the electric dipole:
	
	And as we have shown there is perfect correspondence between the magnetic and electric dipole moment, then we can immediately write:
	
	We can then deduce the norm of the magnetic field vector:
	
	Therefore at the North Pole we have:
	
	and at the equator:
	
	The magnitude of the total field at the pole is twice as strong as at the equator (most people think that intuitively it is the opposite)!
	
	We can also write from the previous results:
	
	also sometimes denoted by the letter $m$ (yes... its obvious to refer the first letter of "magnetism").
	
	\begin{tcolorbox}[colframe=black,colback=white,sharp corners]
	\textbf{{\Large \ding{45}}Example:}\\\\
	Thus, we can have fun to calculate the magnetic moment of the Earth at the geomagnetic equator. What gives, knowing that the Earth's geomagnetic field  at equator is in average equal to $32$ [$\mu$T]:
	
	\end{tcolorbox}
	For the Earth (that is far from a perfect solid dipole at the opposite of Neutron Star), measurement give for the magnetic field amplitude:
	\begin{figure}[H]
		\centering
		\includegraphics{img/electromagnetism/earth_magnetic_field.jpg}
		\caption[Earth's magnetic field]{Earth's magnetic field (source: \url{http://geomag.usgs.gov})}
	\end{figure}
	and for the purely horizontal component (tangential to the surface):
	\begin{figure}[H]
		\centering
		\includegraphics[scale=0.9]{img/electromagnetism/earth_magnetic_field_horizontal.jpg}
	\end{figure}
	and for the purely vertical component:
	\begin{figure}[H]
		\centering
		\includegraphics[scale=0.9]{img/electromagnetism/earth_magnetic_field_vertical.jpg}
	\end{figure}
	But in fact, the Earth is far from being a perfect dipole as it is a complex non-homogeneous non-isotropic celestial object as shown in the figure below:
	\begin{figure}[H]
		\centering
		\includegraphics[scale=0.9]{img/electromagnetism/earth_magnetic_field_non_dipole.jpg}
	\end{figure}
	and furthermore the (main) magnetic north pole of Earth move across the ages (this is named the "\NewTerm{secular variation of Earth's magnetic north pole}\index{secular variation}"):
	\begin{figure}[H]
		\centering
		\includegraphics[scale=0.8]{img/electromagnetism/earth_magnetic_secular_variation.jpg}
	\end{figure}
	
	\pagebreak
	\subsubsection{Magnetic torque}\label{magnetic torque}
	We have just introduced earlier the dipolar magnetic moment $\mu_l$. For a magnet, the magnetic moment (denoted $\mu$) is indirectly defined a quantity that determines the torque it will experience in an external magnetic field such that by definition:
	
	Thus, if a magnetic dipole (or a bar magnet) is placed in a uniform magnetic field in oblique orientation, it experiences no force but experiences a torque. This torque tends to align the dipole moment along the direction of magnetic field!
	
	We also have that a magnetic dipole moment may be defined as the torque acting on a magnetic dipole placed perpendicular to a uniform magnetic field of unit strength as in this case we have $\vec{\tau}=\vec{\mu}$.

	In the magnetic pole representation of a material, a magnetic material can be identified as two magnetic poles some distance apart. The magnetic moment of a material bar is then given by the product of the "\NewTerm{magnetic pole strength}\index{magnetic pole strength}" denoted $q_m$ and the distance between the poles $L$. Then for a magnetic bar where $\vec{n}$ is the unitary vector collinear to the magnet bar length, we have experimentally:
	
	That is in scalar form:
	
	This is the famous relation of the torque for a magnetic bar in a uniform constant magnetic field having two poles of magnetic strength $q_m$ and $-q_m$:
	\begin{figure}[H]
		\centering
		\includegraphics[scale=1]{img/electromagnetism/torque_bar_magnet.jpg}
		\caption{Torque on a bar magnet in a uniform magnetic field}
	\end{figure}
	The lines of action of both forces $\vec{F}_1$ and $\vec{F}_2$ are different, therefore, these two forces form a couple (or torque) which tends to rotate the magnet along the direction of magnetic field strength. 
	
	Thus, torque acting on the bar magnet is max, when it is placed perpendicular to the magnetic field.

	The work done in rotating the dipole through a small angle $\mathrm{d}\theta$:
	
	If the dipole is rotated from $\theta=0_1$ to $\theta=\theta_2$, total work done is given by:
	
	If $\theta_1=\pi/2$ and $\theta_2=\theta$, then:
	
	This work done is stored in the form of potential energy i.e:
	
	We will see again this relation further below.
	
	\pagebreak
	\subsection{Lorentz law (Lorentz force)}\label{lorentz force}
	In electrostatics, we calculated the force exerted by one or a collection of electric charges at rest (...) on a stationary or moving electric charge. The electric force was for recall written as following:
	
	In the most general case, where the acting electric charges are moving, the force they exert on a punctual electric charge $q$ placed in a given point in space is the sum of two terms: one that is independent of the relative speed $\vec{v}$ of this electric charge, the other that depends on it. Here's how is written this relation:
	
	that is the "\NewTerm{Lorentz law}\index{Lorentz law}" or "\NewTerm{Lorentz force}\index{Lorentz force}".
	
	To prove this relation, we will put two assumptions, but first it is important to inform the reader that this proof requires not necessarily obvious mathematical tools (most reader will have to read the section of Analytical Mechanics and Wave Quantum Physics to understand it well):
	\begin{enumerate}
		\item[H1.] Given a non-relativistic point particle of mass $m$, of coordinates $x_i$ with $i=1,2,3$ and velocity $\dot{x}_i$ still with $i=1,2,3$. We assume that it is subjected to a force $\vec{F}$ that satisfies Newton's equations:
		
		with the following commutation relations (\SeeChapter{see sections of Analytical Mechanics page \pageref{poisson bracket} and Wave Quantum Physics page \pageref{commutators and anticommutators}}):
		
	 	The reader must realize that the last relation is an assumption (hypothesis) and it is not equivalent to the commutation rules that we saw in the section of Wave Quantum Physics between positions and linear momentum!

		\item[H2.] There exist fields $\vec{E}(\vec{x},t)$ and $\vec{B}(\vec{x},t)$ , not depending on the speed, such that:
		
		and which satisfy the Maxwell equations (\SeeChapter{see section Electrodynamics page \pageref{maxwell equations}}):
		
	\end{enumerate}
	At a classical level, we express the commutation assumptions by using the Poisson correspondence commutation-brackets (\SeeChapter{see section Analytical Mechanics page \pageref{poisson bracket}}), that is:
	
	with (recall):
	
	Now we define a vector potential $\vec{A}$ (\SeeChapter{see section Electrodynamics page \pageref{magnetic vector potential}}) such that:
	
	then the hypothesis of commutation ($\{x_i,p_j\}=\delta_{ij}$)  can be written for $i\neq j$:
	
	So we can say that $\vec{A}$ depends only on $\vec{x}$ and $t$ and since it commutes identically to $p_j$.
	
	Moreover, we know that classical mechanics admits a Lagrangian formulation (equivalent to Newton's equations) for which the equations of mechanics become (\SeeChapter{see section Analytical Mechanics page \pageref{canonical moments}}):
	
	where $L$ is the Lagrangian of the system. Therefore, with:
	
	we can integrate the relation:
	
	and we get:
	
	 Just as the electric scalar potential $\phi$ is a potential energy per unit charge, the magnetic vector potential $\vec{A}$ is potential energy per unit length element of current (but current has a direction, so the magnetic vector potential has to have three components, that's why it's a vector too...) that is [J$\cdot$m$^{-1}$A$^{-1}$]. It is an old tradition sometimes to use a derivate unit name the "\NewTerm{Weber}\index{Weber}" such that:
	 
	The "-" sign of the constant of integration of the vector potential is justified to be consistent with what we will see during our study of gauge theory (\SeeChapter{see section Electrodynamics page \pageref{gauge theory}}).

	The second Lagrange equation $p_j=m\dot{x}_j+qA_j(\vec{x},t)$ gives us then:
	
	By developing a little bit:
	
	and:
	
	For the set of all coordinates, this gives under condensed form and using the tools of vector analysis:
	
	So what must be absolutely noticed here is that it is the potential vector that appear in the Lagrangian and not the magnetic field!!! Therefore it's the potential vector that is at the origin of the variation of energy of a particle in an electromagnetic field and not the magnetic field!!!!!!!!
	
	Therefore:
	
	or written differently:
	
	We thus fall back well on the expression of the Lorentz force where $\vec{E}$ and $\vec{B}$ are given by:
	
	as we will see it in gauges theory. Certainly the proof is far from being obvious, but is possible.
	
	Many engineer have tested this result in school laboratory and know very well the following image (purple light is emitted along the electron path, due to the electrons colliding with gas molecules in the bulb and not because of bremsstrahlung that we will discuss later):
	\begin{figure}[H]
		\centering
		\includegraphics{img/electromagnetism/electron_lorentz_beam.jpg}
		\caption{Beam of electrons moving in a circle,\\ due to the presence of a magnetic field}
	\end{figure}
	You can also see this happening in photographs taken of charged particle tracks through a device known as the Wilson bubble chamber (we will come back on that one the section of Wave Quantum Physics as this detection procedure seems to be incompatible with Heisenberg's inequalities!): 
	\begin{figure}[H]
		\centering
		\includegraphics{img/electromagnetism/bubble_chamber.jpg}
		\caption[Bubble Chamber]{Bubble Chamber (source: CERN)}
	\end{figure}
	In principle, this is still how physicists frequently determine charge to mass ratios of particles produced in particle physics experiments.
	
	If a charged particle moves in a uniform magnetic field with its velocity at some
arbitrary angle with respect to $\vec{B}$, its path is a helix:
	\begin{figure}[H]
		\centering
		\includegraphics{img/electromagnetism/helicoidal_movement.jpg}
	\end{figure}
	Spiralling tracks as visible in the figure above are a common feature of bubble chamber pictures, and they are caused by electrons $e^{-}$ (or positrons  $e^{+}$, which - key point - have the same mass).

	What a spiral tells us is that an electron loses energy at a considerable rate as it travels through a bubble chamber liquid. All other charged particles, unless they collide with a nucleus, very gradually slow down - get more curved - as they lose energy by ionisation (making bubbles in the bubble
chamber).

	Electrons are able to lose energy more quickly by another process known as "bremsstrahlung" (braking radiation). This process, which is a consequence of the fact that all accelerated charges radiate, is important for electrons because they have small masses. We will study mathematically this phenomenon in the section of Electrodynamics.
	
	\pagebreak
	\subsubsection{Magnetic Vector Potential}\label{magnetic vector potential}
	To get some experience with the magnetic vector potential, let us look first at what it is for a uniform magnetic field $\vec{B}_0$. This is calculation that will be useful during our study of Quantum Physics.

	Taking our $z$-axis in the direction of $\vec{B}_0$, we must have:
	
	By inspection, we see that one possible solution of these equations is:
	
	Or we could equally well take:
	
	Still another solution is a linear combination of the two:
	
	It is clear that for any particular field $\vec{B}$, the vector potential $\vec{A}$ is not unique; there are many possibilities.
	
	The third solution, more general, shows that $\vec{A}$ is perpendicular to the $z$-axis in this special examples. 
	
	So there are actually (infinitely) many vector fields $\vec{A}$ whose curl will equal an arbitrary magnetic flux density $\vec{B}$. In other words, given some vector field $\vec{B}$, the solution $\vec{A}$ to the differential equation $\vec{\nabla}\times\vec{A}=\vec{B}$ is not unique!

	To completely (i.e. uniquely) specify a vector field, we need to specify both its divergence and its curl and as we will see it in the section of Electrodynamics the divergence can satisfy the following gauge equation:
	
	Therefore finally we can write among other relations that we will also prove in the section of Electrodynamics that:
	
	Otherwise the vector potential for a uniform field can also be obtained in another way. The circulation of $\vec{A}$ on any closed loop $\Gamma$ can be related to the surface integral of $\vec{\nabla}\times\vec{A}$ by Stokes' theorem (\SeeChapter{see section Vector Calculus page \pageref{stokes theorem}}):
	
	But the integral on the right is equal to the flux of $\vec{B}$ through the loop, so:
	
	So the circulation of $\vec{A}$ around any loop is equal to the flux of $\vec{B}$ through the loop. If we take a circular loop, of radius $r$ in a plane perpendicular to a uniform field $\vec{B}$, the flux is just:
	
	Let us see another way to derive that result. We want to try one of the possibilities for the vector potential to be:
	
	It can be seen that:
	
	The first term gives $3 \vec{B}$ since the $\vec \nabla \circ \vec{r}=3$. The second term is zero because $\vec{\nabla}\circ \vec{B}=0$. The third and the fourth terms are calculated as follows:
	
	because the field is uniform.
	

	Adding all the four terms, result follows. It can be seen that the Coulomb gauge condition is satisfied. The divergence of the vector potential is given by:
	
	because the first term vanishes as the field is uniform while the second term vanishes because the curl of position vector is zero. Thus the expression satisfies Coulomb gauge condition. If the magnetic field is along the $z$ direction, we can take either of the following expressions for the vector potential components:
	
	Both these expressions are valid expressions for the vector potential and they differ by gradient of a scalar field. It can be checked that if we add a term $\vec\nabla (\psi)$ to the first term where:
	
	we get the second expression.
	
	Now let us consider another interesting example. For an infinite wire along the $z$ axis, the magnetic field, using cylindrical coordinates, is (see page \pageref{magnetic field for an infinite wire}):
	
	where $r$ is the distance from the axis of the wire. We want to know the expression of the corresponding magnetic vector potential.
	
	We know that the magnetic field has cylindrical symmetry and is directed along the circumferential direction and must satisfy:
	
	Thus the curl of the vector potential only has $\phi$ component:
	
	By symmetry, since the wire is infinite, the derivative with respect to $z$ must be zero and we have:
	
	which gives:
	
	where we have explicitly added gradient of an arbitrary scalar field.

	Now suppose we look inside the wire, taking the wire's radius to be $R$. At a distance $r$ from the axis, only current within that radius contributes to the field, so we get from Ampère's law (see page \pageref{Ampère's law}) using a circular loop within the wire (see page \pageref{magnetic field for an infinite wire}):
	
	Using exactly the same method as earlier we then get:
	
	Again, we get $\vec \nabla \circ \vec{A}=0$ and $\vec \nabla \times \vec{A}=\vec{B}$, so it looks like we're done.
	
	Now let us move to another case! We know that the field inside a solenoid is along its axis ($z$ direction) and is zero outside. If we take a circular path of radius $s$ centered along its axis, the flux through the circular area is (see page \pageref{solenoid}):
	
	The line integral of the vector potential is:
		
	\begin{figure}[H]
		\centering
		\includegraphics[scale=1]{img/electromagnetism/vector_potential_solenoid.jpg}
	\end{figure}
	The vector potential is thus given by for $s<R$:
	
	Though the field outside is zero, the vector potential does not vanish outside the solenoid! This is because, if we take a circle of radius $s>R$, the flux through the circular area is:
	
	flux being contributed only from inside the solenoid. Thus for $s>R$ as:
	
	We then get:
	
	Hence:
	
	which falls off as inverse of distance from the axis.
	
	Now an obvious question is if does vector potential have any physical significance?
	
	It may be seen from the above example that the vector potential remains non-zero outside the solenoid even though the magnetic field itself has become zero. It turns out that the vector potential is not just a mathematical artefact but has physical reality!!!
	
	If you are familiar with the Young’s double slit experiment (\SeeChapter{see section Wave Optics page \pageref{young interference experiment}}) done with a beam of coherent light. The experiment is not really restricted to light wave but can be performed with matter wave such as a beam of electrons. We learn from quantum theory that like light exhibits dual behaviour, that of wave as well as particles known as photon, a wavelength is associated with material particles as well. This is known as "de Broglie wavelength" (\SeeChapter{see section Wave Quantum Physics page \pageref{de Broglie wavelength}}) and is given by the ratio of the Planck’s constant to the momentum of the particle.
	
	The experiment is performed with an electron beam in place of light. What is done is to put a small solenoid just beyond the slits between the slits and the screen. Initially the solenoid does not carry any current and its dimensions are small enough so that it does not disturb the interference pattern produced on the screen because of the phase difference between electron waves arriving at the screen from the slits.
	
	The current in the solenoid is switched on. The solenoid is of very small dimensions so that most of the electron beam passes outside the solenoid. Since the magnetic field outside is zero, the electron beams does not experience any force due to the magnetic field and should reach the screen undeflected. This should not then affect the interference pattern. However, what is found is that the pattern on the screen shifts suggesting a change in the phase relationship. It can be shown in quantum mechanics that the agent responsible for this phase change is the vector potential which is non-zero outside the solenoid though the magnetic field is zero.
	
	\subsubsection{Work of Magnetic Field}
	Let us stop a moment on the expression of the Lorentz force. We see with this relation that a stationary (or not) electric charge in an electric field will experience a force that will give him the necessary impulsion to make vary its kinetic energy (zero or non-zero at the start). This fact is however not valid for the magnetic field. Indeed, when we put a stationary electric charge in a magnetic field, this latter will not be influenced by magnetic field strength and therefore will not see its kinetic energy vary. If the electric charge has a non-zero initial velocity, it follows that the magnetic field will change the components of the velocity vector but not its norm!!! So, we used to say that "the magnetic field does not work" (in the sense that the magnetic field is not make move an electric charge that is initially at rest or change the norm of its speed).
	Let's see how we can prove mathematically that the magnetic field does not work.
	\begin{dem}
	We know that for an electric charged particle immersed in a magnetic field, we have:
	
	hence:
	
	and let us express the temporal variation of the kinetic energy:
	
	and substituting in it the derivative of the speed by prior-previous relation, we get:
	
	The kinetic energy of the particle therefore does indeed not change because the magnetic field is conservative in the special case of stationary magnetic (and hence electric) fields.
	
	This result also mean obvious that $\nabla{\vec{F}}=\vec{0}$. But if fact as we will prove it in the section of Electrodynamic this is a special case!! Indeed we will prove in the section of Electrokinetics the Faraday's induction law:
	
	Therefore it is immediate that $\nabla{\vec{F}}$ will not be null anymore (and hence $\vec{F}$) if $\vec{B}$ is not static.
	\begin{flushright}
		$\blacksquare$  Q.E.D.
	\end{flushright}
	\end{dem}
	Now, if we are interested only in the second term of this relation, we can get to prove the "Laplace's force law"!

	So we have considering only the second term of the Lorentz law:
	
	where $\rho$ is the volumetric electric charge density. If $\vec{v}$ and $\mathrm{d}\vec{l}$ are supposed parallels we can write that:
	
	A current density allows us to calculate the inertial speed of the electric charge carriers in a conductor. The number of conduction electrons in a wire is equal to:
	
	where $n$ is the number of conduction electrons per unit volume and $Al$ the volume of the wire (and therefore $A$ is the area of the section of the conductor).
	
	A quantity of electric charges $q=\rho A l$ pass trough the surface section a wire in a time $t$ given by:
	
	The intensity of current $I$ being defined by:
	
	we get that:
	
	From:
	
	We can now draw that:
	
	Finally, we find that:
	
	which is the "\NewTerm{Laplace's force law}\index{Laplace's force law}\label{Laplace law}" or "\NewTerm{Lorentz force}\index{Lorentz force}" and therefore is derived from the Lorentz law. We then deduce the units of the magnetic field (force per unit current and unit length of the source):
	
	where T is a commonly tolerated unit in use named the "Tesla" in honour of Nikola Tesla one of the greatest spirit of all time with Albert Einstein. Therefore this unit implicitly contains the Coulomb unit which is then at the origin of the magnetic field! Knowing finally explicitly the units of the magnetic field, we can determine the units of the magnetic permeability constant by taking back the relation:
	
	He then comes for the units of the constant magnetic permeability:
	
	So we see that it has for units a force by square current intensity.

	Having done this, let us see some important cases of application of the Lorentz law:
	
	\pagebreak
	\subsubsection{Classical Hall effect}
	Previously, we have studied the action of a magnetic induction on a wire-like circuit having for purpose to find the expression of the magnetic forces applied to the same area of this circuit. Now let us turn our attention to the conductivity electrons themselves, placing us in the case of the figure below:
	\begin{figure}[H]
		\centering
		\includegraphics{img/electromagnetism/hall_effect.jpg}
		\caption{DC current through a metallic ribbon for the study of classical Hall effect}
	\end{figure}
	where a metallic ribbon is traversed by a continuous current $\vec{I}$. The vector of current density $\vec{J}$ is constant and parallel to the long sides $\overline{PQ}$ or $\overline{RS}$ of the ribbon.
	
	Let us imagine then that the ribbon is immersed in a uniform magnetic field perpendicular to the planes $PQ$ and $RS$ (following the $z$-axis). The mobile electric charges of volumetric density $\rho$ contained in a volume element $\mathrm{d}V$ is therefore subject to the magnetic force:
	
	This force change the trajectories of mobile electrons and, during a transitional period, causing their accumulation on the front edge of the ribbon while an excess positive charge appears on the back edge.

	This phenomenon produces an additional electric field parallel to $\overline{RP}$ which exerts on the mobile charges of volume $\mathrm{d}V$ an electric force:
	
	The two forces are then opposing each other and the coulomb force tends to take back the electron paths in their initial position. A steady state is established gradually.
	\begin{tcolorbox}[title=Remark,colframe=black,arc=10pt]
	In fact, every time we talk about steady in physics, we lie a little bit. It is in fact just a stable equilibrium and in general, the system oscillates around its equilibrium position. After a while, a system like the conductor involved in our example shows negligible oscillations. Physics is sometimes also a matter of approximations...
	\end{tcolorbox}
	When this state is reached, the current density is again parallel to $\overline{PQ}$ and the electric and magnetic forces above are vectorially opposed. So we have:
	
	with:
	
	In some books, the cross product is made explicit in the form of its components such as:
	
	since the other components are zero (the current density is parallel to the ribbon and the magnetic field is perpendicular).

	Now, as we have proved it in the section Electrokinetics we have
	
	therefore:
	
	We then define the "\NewTerm{Hall coefficient}\index{Hall coefficient}" by:
	
	$R_H$ can be used at equilibrium for the measurement of $J_x$ if we suppose $v_x=c^{te}$ and that the potential $U$ (and therefore implicitly $E_x$) and external field $B_z$ are known.  

	Notice that by construction we also have $R_H\propto \rho$ and this is used sometimes by engineer to determiner the density of electrical charge carriers in a new material sample.
	\begin{tcolorbox}[title=Remark,colframe=black,arc=10pt]
	We also tale of "\NewTerm{Hall's resistance}\index{Hall's resistance}". It is simply the ratio of the Hall voltage on the current through the sample. It must not be confuse with the Hall resistance $R_H$. 
	\end{tcolorbox}
	In a two-dimensional semiconductor, the Hall effect is also measurable. By cons, at sufficiently low temperature, we observe a series of trays for the Hall resistance as a function of the magnetic field. These plates appear to specific resistance values, and this, independently of the sample used. This is the subject of the "\NewTerm{quantum Hall effect}\index{quantum Hall effect}" that we will not discuss in this chapter.

	Under scalar form the relation of the "Hall effect", is written:
	
	We can also express it by expliciting the potential difference that corresponds by definition to the electric field.

	If $l$ is the width of the ribbon, we have:
	
	If $e$ is its thickness, the current $I$ that pass through it is equal to:
	
	Given the relative positions of the various vectors, the relation expressing the Hall effect is equivalent to:
	
	More aesthetically and in traditional form, the voltage of the Hall effect is given by:
	
	with:
	
	which is the "\NewTerm{Hall constant}\index{Hall constant}". It is inversely proportional to the density of the free electric charge carriers and in the context of metals, it is negative.

	In other field of study such as the semiconductor, we write the Hall voltage in the following traditional form:
	
	where $q$ is the electron charge, and $n$ the traditional notation (sic!) of the carrier density within the study of semiconductors.

	We then in the latter field the Hall constant which is defined by:
	
	What has however made the reputation of the Hall effect, besides the fact that this result is enormously used to make magnetic fields sensors of all kinds (as Hall effect sensors operate without physical contact with the magnets), this is that for certain types of semiconductor this Hall constant is positive!!!! This would mean with the standard models we have available to us until now, there would be positive charges that would generate the current... and at the time of the establishment of this experience for semi-conductors this was inexplicable. At the time of  Edwin Herbert  Hall this experiment was used to check whether it was positive or negative charges that were moving and Hall concluded by testing it on conductive metals that only the negative electricity where moving in the conductors.
	\begin{tcolorbox}[title=Remark,colframe=black,arc=10pt]
	The most well known Hall probe by the population is the speedometer on bikes (odometer), which operates on the basis of attaching a small magnet on one of the spokes of a wheel and whose passage in front of the Hall sensor produces a signal processed by the electronic odometer.
	\end{tcolorbox}
	We will prove later that by using quantum theory in the context of semiconductors (\SeeChapter{see section Electrokinetics page \pageref{semiconductors}}) positive charges may yet appear under certain conditions and cause a current!
	
	\subsubsection{Larmor radius}
	A very interesting case of laboratory study is the movement of a charge in a uniform magnetic field. For this study, consider a particle of mass $m$ and electric charge $q$ placed in a uniform magnetic field with an initial velocity $\vec{v}_0$.

	We have following Lorentz' law:
	
	We will take advantage of the fact that the magnetic force is zero in the direction of the magnetic field.

	We'll break down the speed in two components, one parallel and one perpendicular to the magnetic field such as:
	
	The equation of motion is then:
	
	The trajectory remains rectilinear uniform in the direction of the magnetic field! In other words, if the speed of the charged particle was initially zero in the direction of the field then it will remain zero!
	
	Let us now consider a Cartesian coordinate system with the $Z$ axis is given by the direction of the magnetic field such that $\mathrm{d}v_z/\mathrm{d}t=0$. The equation of motion is therefore written only on two components, since:
	
	hence:
	
	A very simple special solution to these two differential equations in non relativistic framework is obviously:
	
	So where we chose an initial speed following the $x$-axis. By integrating, we get:
	
	where the integration constants were chosen as zero (arbitrary choice). The trajectory is thus a circle of radius:
	
	perpendicular to the magnetic field and named "\NewTerm{Larmor radius}\index{Larmor radius}", described with the pulsation:
	
	named "\NewTerm{gyro-synchrotron pulsation}\index{gyro-synchrotron pulsation}" This circle is travelled in the conventional positive direction for negative charges.
	\begin{tcolorbox}[title=Remark,colframe=black,arc=10pt]
	The movement is circular only if the particle, initially, therefore has no speed in the direction of the magnetic field. If it has one, it will keep it (the magnetic field has no action in that direction).
	\end{tcolorbox}
	The problem of such a configuration to build an accelerator is that if we increase the energy of the particle (by adding a synchronized electric field on the gyro-synchrotron pulsation and collinear to the movement), its speed increases and the radius Larmor also. But the "cyclotron" which is based on this system has a limited radius since it is difficult to maintain a constant magnetic field over a large area:
	\begin{figure}[H]
		\centering
		\includegraphics[scale=0.5]{img/electromagnetism/cyclotron.jpg}
		\caption[Cyclotron principle]{Cyclotron principle (source: \url{http://femto-physique.fr}, author: Jimmy Roussel)}
	\end{figure}

	Even more difficult, in the relativistic case, the pulse is then written with the Fitzgerald-Lorentz factor (\SeeChapter{see section Special Relativity page \pageref{fitzgerald lorentz factor}}):
	
	We then see that we need to adjust the pulse of the electric field to the pulse of rotation when the sped increase: the accelerator is then now a "\NewTerm{synchrotron}\index{synchrotron}".
	
	To solve the problem of increasing radius, whereas we use a "Synchrotron" consisting of a single vacuum tube having straight sections containing the accelerating cavities and curved sections equipped with magnets creating at each instant the magnetic field adapted to the particle velocity. This technique, that it is easy to talk about but very difficult to put into practice, is the most used today. The CERN LHC is part of the family of synchrotrons.

	From this relation, it is easy to have the kinetic energy of the particle:
	
	It is based on this relation that work the "\NewTerm{Dempster mass spectrometers}\index{Dempster mass spectrometers}". It is using this technique that researchers discovered in 1920 that atoms of the same chemical element does not necessarily have the same mass. The different varieties of atoms of the same chemical element, varieties that differ in mass, are the "isotopes" (\SeeChapter{see section Nuclear Physics page \pageref{isotope}}):
	\begin{figure}[H]
		\centering
		\includegraphics[scale=0.5]{img/electromagnetism/dempster_mass_spectrometers.jpg}
		\caption[Dempster mass spectrometers principle]{Dempster mass spectrometers principle (source: \url{http://femto-physique.fr}, author: Jimmy Roussel)}
	\end{figure}
	The Larmor radius is the greatest distance that a particle can travel in the transverse direction before being deflected from its path. This corresponds to a kind of trapping distance. Unless it receive additional kinetic energy, a charged particle is thus trapped in a magnetic field.
	
	It is interesting to notice that more the kinetic energy of a particle is high (large mass or large transverse speed) and the Larmor radius is big. Conversely, the higher the magnetic field and the smaller is the Larmor radius (hence the fact we need hug magnetic field at CERN to keep high speed particles into a circular trajectory).

	We come back on these concepts in the section Electrodynamics, where after studying the Maxwell equations, we will make some developments for the betatron type accelerator.
	\begin{tcolorbox}[title=Remark,colframe=black,arc=10pt]
	Confining the plasma in a tokamak is based on this property that charged particles have to describe a helical path around a magnetic field line. Hence the need to use a torus... in the simplest case.
	\end{tcolorbox}
	To close this subject, and following the comment of a reader, let us develop a little more in detail the solution of the both differential equations seen previously:
	
	simplifying the notation, they become:
	
	and we will take as initial conditions:
	
	The trick is to put:
	
	The system of differential equations the becomes (we drop the $Z$ component):
	
	and therefore:
	
	And by doing the same for $Y$, we get:
	
	The both differential being identical it is sufficient to solve one to have the other solutions. The roots of the characteristic equation are (\SeeChapter{see section Differential and Integral Calculus page \pageref{characteristic equation}}):
	
	Since the discriminant is negative (the expression that is for recall under the root), then we have(\SeeChapter{see section Differential and Integral Calculus page \pageref{first order lde with constant coefficients}}) the homogeneous solution is then:
	
	Hence:
	
	And as there is no second member to our two differential equations, the homogeneous solution is then the general solution. Therefore:
	
	and same:
	
	Since we have put $X=\dot{x}$, $Y=\dot{y}$ and that we have $\ddot{x}=\omega\dot{y}$, then it comes:
	
	Thus explicitly:
	
	Since this equation must be valid for all time $t$, let us consider the case where $t = 0$, the equation is reduced to:
	
	By doing the same for $Y$ (always with $t = 0$), we get:
	
	Taking into account the initial conditions for speed, we get:
	
	So:
	
	\begin{tcolorbox}[title=Remark,colframe=black,arc=10pt]
	By putting $v_{0y}$ to be zero, we fall back on the simple special solution we proposed above with respect to speed.
	\end{tcolorbox}
	To continue, we take the primitive to find the position coordinates. It comes then:
	
	For the initial conditions:
	
	to be satisfied, we see pretty quickly that we must have:
	
	Thus finally:
	
	So if we take the $Y$ component of the velocity to be zero, we have:
	
	Which is not more the path of a circle because of the initial conditions chosen for this development. To fall back on the circular path it require the initial condition on $Y$ to be:
	
	Trapped particles in magnetic fields are found in the Van Allen radiation belts around Earth, which are part of Earth's magnetic field. These belts were discovered by James Van Allen while trying to measure the flux of cosmic rays on Earth (high-energy particles that come from outside the solar system) to see whether this was similar to the flux measured on Earth. Van Allen found that due to the contribution of particles trapped in Earth's magnetic field, the flux was much higher on Earth than in outer space. Aurorae, like the famous aurora borealis (northern lights) in the Northern Hemisphere, are beautiful displays of light emitted as ions recombine with electrons entering the atmosphere as they spiral along magnetic field lines. Aurorae have also been observed on other planets, such as Jupiter and Saturn.
	\begin{figure}[H]
		\centering
		\includegraphics[scale=1]{img/electromagnetism/van_allen_belt.jpg}
		\caption[Van Allen radiation belts around Earth]{Van Allen radiation belts around Earth (source: OpenStax)}
	\end{figure}
	
	\pagebreak
	\subsubsection{Energy of a magnetic dipole}
	Thanks to the fact that we now have the units of the magnetic field and those of the magnetic permeability constant and the Lorentz law, we will now be able to determine by dimensional analysis and intuition the total energy of a static (oriented!) magnetic dipole which will be very useful for us for the theory of paramagnetism developed further below.

	Let us consider for this a rigid magnet in the form of cylinder of length $L$ and of negligible radius that can be regarded as a north / south dipole (a "bar magnet") immersed in a constant and homogeneous magnetic field in the plane perpendicular to the axis of rotation of the dipole:
	\begin{figure}[H]
		\centering
		\includegraphics{img/electromagnetism/simple_static_magnetic_dipole.jpg}
		\caption[]{schematic diagram for the study of the static magnetic dipole}
	\end{figure}
	The experience shows that when the dipole is collinear with the magnetic field, it does not move. It follows that the force on one end depends in proportion to the sine of the angle with the magnetic field such that:
	
	At the dimension level we have so fare:
	
	We would have to get rid of Amps [A] by already making intervene at least obviously what characterizes a magnetic dipole... and we have already determined earlier above: its magnetic moment $\mu_l$ which units are for recall:
	
	It seems quite natural enough to write to go a little further:
	
	This now gives at the dimensional level:
	
	We then a unit of length in more. Then it seems quite natural to introduce the dipole length such that (the force should logically be equal at every point of the dipole so there is no reason to introduce here half of the dipole length!):
	
	Now to get back to the energy of the magnetic dipole, we consider that it is zero when the dipole is initially in a position perpendicular to the magnetic field. Using the common approximation as what infinitesimal movement of the two ends is given by for small angles:
	
	Thus, the elementary work (energy) to turn the dipole is (we multiply by two because we have to sum the two forces acting on each pole) by denoting by $B$ the norm of the magnetic field:
	
	where we see that the dipole length is no longer involved. In fact we must not forget that it is in the magnetic dipole moment $\mu_l$ that we have the equivalent area of the dipole.

	It comes by integration for a given final angle:
	
	Therefore:
	
	Let us see now a classic and academic case that we can get from this result and that we will see again during our study of the spin in the section of Wave Quantum Physics.

	So we have proved earlier above that an electric charged particle is deflected by a force given by the equation of the Lorentz' law:
	
	It follows that if the field has only a constant component in $z$ and the speed only one component in $x$, this will cause a helical motion in the plane perpendicular to the field as we have already proved above in our study of the Larmor radius.

	Let us consider now an electric charged particle launched at uniform speed along an $x$-axis between two poles of opposing magnets which generate a vertical and heterogeneous magnetic field and let us look only to the deflection along $z$ of the trajectory of the particle.
	
	From the perspective of the $z$-axis the particle can be considered in uniform acceleration (\SeeChapter{see section Classical Mechanics page \pageref{kinematics of rectilinear motion}}):
	
	Since we assume that the initial position along $z$ of the particle is midway between the two poles and that its initial velocity along the $z$-axis such as:
	
	Then we have:
	
	Since the magnetic field does not work (see proof earlier above) and that implies that the kinetic energy remains constant, we can write that the time is simply the ratio of the horizontal distance travelled by the particle by the speed module:
	
	Let us recall that we have proved just earlier above that the potential energy of a magnetic dipole was in a constant and homogeneous field given by:
	
	Then in comes in our case:
	
	And as we can associate a force to a potential energy, it comes by assuming that the dipole magnetic moment is constant along the $z$-axis and the magnetic field is still somewhat a little bit inhomogeneous (happy mix of assumptions ... but we still do engineering physics in this book!):
	
	And finally:
	
	So we see in any case that $z$ can take a continuous spectrum of values that depends on the magnetic dipole moment of the particle. Now, as we shall see in the section of Wave Quantum Physics and Relativistic Quantum Physics during our study of quantum operators and especially that of the orbital angular momentum and of spin, an experiment named "\NewTerm{Stern-Gerlach experiment}\index{Stern-Gerlach experiment}" has shown that this was not the case for some particles or atoms for which the $z$ are clearly discrete, what classical physics seemed unable to explain!!!!
	\begin{figure}[H]
		\centering
		\includegraphics[scale=0.9]{img/electromagnetism/stern_gerlach_experiment.jpg}
		\caption[Stern-Gerlarch experiment principle and result]{Stern-Gerlarch experiment principle and result (source: Wikipedia, source: JohnDC$^\dagger$)}
	\end{figure}
	Stern and Gerlach were astonished by what they saw. Not only were these "free" electrons deflected by the magnetic field, the pattern of their deflection was itself totally unexpected and mysterious. They found that the beam separated into two distinct parts. The basic experimental set-up is shown below.

	This pointed to two things. First, the "free" electrons must have some kind of built-in magnetic moment because they are interacting with the magnetic field. Second, this magnetic moment is no ordinary dipole moment - the electrons are not acting like tiny little magnetic spheres shooting through the magnetic field.
	
	\pagebreak
	\subsection{Langevin treatment of Diamagnetism and Paramagnetism}
	We have define earlier what were diamagnetic and paramagnetic materials. In 1905 Langevin proposed after trial and error the theories of diamagnetism and paramagnetism (in 1906 Pierre Weiss proposed the theory of ferromagnetism) under some strong and simplified assumptions but that still work quite well to explain some material behaviours. We will see here what are the developments used to explained these behaviours.
	
	\subsubsection{Langevin model of diamagnetism}
	The purpose of this model is be able to explain the negative magnetism that is opposed to the magnetic excitation such that we can observe it in practice. This model is rough relatively to the quantum model but it is interesting for two main reasons: the first is that it already gives enough concepts to the reader to start the study of this subject if not familiar with quantum theory, the second being that it is a good training model (in the academic sense) because it shows how trials and successive approximations can lead to something relatively acceptable in practical perspective.
	\begin{tcolorbox}[title=Remark,colframe=black,arc=10pt]
	This model has its place in the section Magnetostatics because the excitation field used in the model is assumed to be constant!
	\end{tcolorbox}
	For this, we consider the classical model of Langevin (the quantum model giving the same result as we will see in the corresponding section), where the electron is regarded as travelling a circular orbit of radius $r$ and then can be assimilated to an electric current in a loop producing an electromotive force (\SeeChapter{see section Electrokinetics page \pageref{electromotive force}}):
	
	that we can assimilate to an electric field such that (\SeeChapter{see section Electrokinetics page \pageref{total emf of the circuit}}):
	
	Therefore it comes:
	
	We then have by denoting $m_e$ the rest mass of the electron and $q_e$ its electric charge:
	
	hence:
	
	The application of an external magnetic excitation will have for effect to change the magnetic dipole moment $\mu_l$ of quantity $\Delta \mu_l$. But, we have prove earlier above that the magnetic dipole moment was given by:
	
	Then it comes for the electron:
	
	Then, the very clever and rough trick (in the approximate sense of the term relatively to the experience and to the quantum model developed a few years later) is to take into consideration the fact that the electron in classical point of view can be considered as a point object that can move in a sphere of radius $R$ given in the case of a mono-electronic atom and not just in a planar circular radius $r$ perpendicular to the direction of the magnetic excitation field.

	In this case, we then have of course:
	
	and we will consider that the three coordinates are independent and identically distributed random variables (so already the theoretical model is very rough but it's better than nothing...). Therefore, it follows that their expectation is equal such as:
	
	Using the property of linearity of the expected mean (\SeeChapter{see section Statistics page \pageref{properties of the mean}}) and the notation used by the physicist, it is then written:
	
	and as the coordinates are considered like random variables identically distributed, we also have:
	
	It follows immediately that:
	
	And so if we are interested only to the average  disk radius containing all orbits perpendicular to the direction of the magnetic excitation field directed along the $z$-axis, then it comes:
	
	Therefore, finally, for an electron in all possible orbits of a limited sphere:
	
	where $\langle R^2 \rangle$ can be explicitly calculated with the quantum model (with wave functions to be precise!).

	For an atom containing $Z$ electrons, we will do the rough assumption that a simple sum of the effects is valid ...:
	
	From a macroscopic point of view, the number of atoms in a unit volume will be the ratio of the density of the material divided by the atomic weight of the element in question multiplied by Avogadro's number:
	
	It comes then by unit volume:
	
	and it is this result that is assimilated to the magnetic susceptibility as it's written in the form of the "\NewTerm{Langevin's diamagnetic susceptibility relation}\index{Langevin's diamagnetic susceptibility relation}":
	
	At unit level, we have well:
	
	So it is a coefficient without units as expected.... and it is because this coefficient is negative that we assimilate this model to diamagnetism (by definition). The reader will have perhaps noticed that if the magnetic excitation is zero, the susceptibility is also zero... which is an expected minimum behaviour of the theoretical model. But by cons this latter does not depend on the temperature (the influence thereof is anyway almost negligible in standard laboratory temperature range).

	The agreement experiment / theory is excellent for spherically symmetric elements in the order of $\pm 10\%$ error. For non-spherical elements the error often reaches $50\%$ with respect to the experience.
	
	\pagebreak
	\subsubsection{Langevin model of paramagnetism}
	Langevin tried (with varying degrees of success here too) to explain the paramagnetism with the same underlying ideas, but however by opting for a completely different mathematical approach to ensure a positive outcome... (do it yourself physicist way of life...). What Langevin also knew it is that paramagnetism highly dependent on the temperature according to experimental studies of ferromagnetic materials, it was therefore necessary to choose an approach emphasizing the temperature and at the time there were no $10,000$ ways to do this! It follows that this model opens also the door to the theory of ferromagnetism!

	As at the starting point at that time we naturally assumed the (Maxwell-)Boltzmann distribution proved in the section of Statistical Mechanics that describes for recall the distribution of detectable particles that do not interact with any constraints on the number of particles state... (at the time of Langevin there were only this model available...):
	
	where as the reader maybe noticed it (\SeeChapter{see section Statistical Mechanics page \pageref{maxwell distribution}}) the potential energy $\mu_i$ is taken to be zero, so that all energy is in the form of kinetic energy.
	
	It is important to notice that the latter writing  of the distribution function, as we have seen in the section Statistical Mechanics, ignores the normalization constant to make it a true probability density function (but we will obviously calculate that constant a little further below).

	We have also proved just earlier that the magnetic potential energy of a dipole was given by:
	
	Now, let us recall that we have proved in the section of Geometric Shapes that the surface of a sphere element was given by:
	
	It comes therefore (the reader may refer to the figure of the section of Geometric Shapes) that for a crown of the sphere defined by two parallel planes, which middle contains the origin of the sphere is then given by (if needed and on readers request we can draw another figure) :
	
	Why are we talking about this? Well because a portion of the total number of magnetic dipoles included in an angle range $\mathrm{d}\theta$ is as we see above proportional to a surface element as:
	
	We then have that this number is given by (to a given unknown constant factor $K$):
	
	The corresponding proportion (so it is also a probability) over all angles, for a given angle, is then given after normalization obviously by:
	
	We saw during our study of the magnetic dipole moment that $\vec{\mu}_l$ contributes to the magnetic field. If we have a volumetric density of $n$ magnetic dipoles, then we have a contribution of around $n\vec{\mu}_l$ if they are all oriented in the same direction. But if we project the vector of the magnetic moment on the direction of the magnetic field, the contribution of dipoles will then be written:
	
	but as there are many dipoles in different directions and that do many various angles with the field, then we have to take the average such that the contribution to the magnetic field is proportional to:
	
	And using simply the properties of density statistical functions (\SeeChapter{see section Statistics page \pageref{distribution function}}), the expected mean contribution is given by:
	
	and as the denominator is just a normalization constant, we can get it out of the first integral:
	
	To integrate, let us do the little simplification of writing by putting:
	
	This gives us:
	
	and let us put:
	
	Then we have:
	
	It comes then:
	
	The primitive numerator is known to us because it is part of the usual primitive proved in detail in the section of Differential and Integral Calculus! The integral in the denominator is trivial:
	
	The function:
	
	is often named "\NewTerm{Langevin's function}\index{Langevin's function}" with for recall:
	
	\begin{figure}[H]
		\centering
		\includegraphics{img/electromagnetism/langevin_function_plot_maple.jpg}
		\caption[]{Plot of Langevin's function with Maple 4.00b}
	\end{figure}
	The Langevin function is equal $0$ when its parameter is $0$ and approaches $1$ when its argument tends to infinity. So the system eventually saturate when the magnetic field increases, which corresponds to the experimental behaviour of paramagnetic materials. By cons the increase in temperature therefore decrease factor $\gamma$ and makes the Langevin's function tend to $0$ and has for effect to cancel the alignment of dipoles.

	For small values of the parameter $\gamma$, the Langevin's function can be considered as linear as we see on the plot above.
	To simplify the expression, we'll calculate the Taylor approximation of the hyperbolic cotangent using detailed proved presented in the section of Sequences and Series and when the hyperbolic cotangent argument is to recall strictly less than $1$ in absolute value:
	
	Then we have:
	
	We have therefore the magnetic field which is proportional to:
	
	We see that the factor:
	
	is dimensionless. Indeed:
	
	So we can consider it is the paramagnetic susceptibility and write the "\NewTerm{Langevin's relation of the paramagnetic susceptibility}\index{Langevin's relation of the paramagnetic susceptibility}":
	
	better known under the name "\NewTerm{Curie's law}\index{Curie's law}" and shows that shows that the magnetic susceptibility is inversely proportional to the temperature (good but obviously this law becomes false at low temperatures and we must then derive empirically the Curie-Weiss law).
	
	As a little summary can always be useful here is one:
	\begin{table}[H]
	\begin{center}
		\definecolor{gris}{gray}{0.85}
		\renewcommand{\arraystretch}{2.6}
			\begin{tabular}{|c|c|c|c|c|}
				\hline
				\cellcolor{black!30}\textbf{Property} & 
\cellcolor{black!30}\textbf{Diamagnetic} & \cellcolor{black!30}\textbf{Paramagnetic} & \cellcolor{black!30}\textbf{Ferromagnetic}    \\ \hline
		\cellcolor{black!30}\textbf{Magnetic induction} & $B<B_0$ & $B>B_0$ & $B \gg B_0$\\ \hline
		\cellcolor{black!30}\textbf{Susceptibility $\chi$} & $\chi<0$ & $\chi \geq 0$ & $\chi \gg 0 \;(10^2-10^3)$\\ \hline
		\cellcolor{black!30}\textbf{Temperature dependence} & No & $\chi_d=\dfrac{C}{T}$ & $\chi_f=-\dfrac{C}{T-T_c}$\\ \hline
		\cellcolor{black!30}\textbf{Relative permeability $\mu_r$} & $\mu_r<1$ & $\mu_r>1$ & $\mu_r \gg 1 \;(10^2-10^3)$\\ \hline
		\cellcolor{black!30}\textbf{Magnetising vector} & Opposite to $\vec{H}$ & In the direction of $\vec{H}$ & In the direction of $\vec{H}$\\ \hline
	\end{tabular}
	\end{center}
	\caption{Summary of common magnetic properties}
	\end{table}
	
	\begin{flushright}
	\begin{tabular}{l c}
	\circled{90} & \pbox{20cm}{\score{4}{5} \\ {\tiny 48 votes,  79.17\%}} 
	\end{tabular} 
	\end{flushright}

	%to force start on odd page
	\newpage
	\thispagestyle{empty}
	\mbox{}		
	\section{Electrodynamics}\label{electrodynamics}
	\lettrine[lines=4]{\color{BrickRed}C}lassical electrodynamics is a branch of theoretical physics that studies the interactions between electric charges and currents using an extension of the classical Newtonian model. The theory provides an excellent description of electromagnetic phenomena whenever the relevant length scales and field strengths are large enough that quantum mechanical effects are negligible. For small distances and low field strengths, such interactions are better described by quantum electrodynamics.
	
	In this section we will build a set of equations that can sum up themselves our entire knowledge about electrostatics and magnetostatics. These equations, at the number of four, are named "\NewTerm{Maxwell-Heaviside equations}\index{Maxwell-Heaviside equations}" (denomination that we will shorten as many other books under the name "\NewTerm{Maxwell's equations}\index{Maxwell's equations}") and will allow us to approach the branch of physics named "\NewTerm{electrodynamics}\index{electrodynamics}" and therefore electromagnetic waves (i.e.: light!).
	
	Electrodynamics is a pillar of the electronic revolution! Without this theory: no radio, no phones or cell phones, no computers, no satellites, no satellites, no electric motor, we would still at the technological state of the late 19th century without it!
	\begin{tcolorbox}[title=Remark,colframe=black,arc=10pt]
	It is very important to understand what will follow! Some developments will be reused in the section of Special Relativity, Quantum Field Theory, etc. Furthermore, the reader should also read in parallel the section of Special Relativity to better understand the ins and outs of certain results and the provenance of some mathematical tools.
	\end{tcolorbox}
	We assume before tackling mathematical models that everyone admits at the beginning of this third millennium than gamma rays, radio waves, microwaves, visible light (and not visible light) are electromagnetic waves (EM) of different frequencies:
	\begin{figure}[H]
		\centering
		\includegraphics{img/electromagnetism/spectrum.jpg}
	\end{figure}
	With a non-exhaustive summary of the current applications of the wave frequencies in this early 21st century:
	\begin{figure}[H]
		\centering
		\includegraphics{img/electromagnetism/spectrum_applications.jpg}
	\end{figure}
	Frequency allocation to the business economy, civil and military industry is the role of the International Telecommunication Union.
	
	\subsection{Maxwell Equations}\label{maxwell equations}
	As already mentioned Maxwell's equations are a set of partial differential equations that form the foundation of classical electrodynamics, classical optics, and electric circuits and encompass the knowledge of almost all 19th century about the subject. These fields in turn underlie modern electrical and communications technologies. Maxwell's equations describe how electric and magnetic fields are generated and altered by each other and by charges and currents.
	
	If there is a some gods (concept that remains to be proved beyond reasonable evidence) many people say that probably god (we don't know the god of what religion...) wrote the Maxwell Equations first and let there be light: and there was light!
	
	\subsubsection{First Maxwell Equation (constant electric flow)}\label{first maxwell equation}
	Let us define $\vec{\mathcal{E}}$ a vectors field in space. Consider a closed surface $S$ in this field. Then at each point $(x, y, z)$ belonging to the surface corresponds a vector of the field:
	
	In this case the Ostrogradsky's theorem give us (\SeeChapter{see section Vector Calculus page \pageref{gauss ostrogradsky theorem}}):
	
	with $V$ being the volume delimited by the closed surface $S$ (so-named for recall "\NewTerm{Gauss surface}\index{Gauss surface}").
	\begin{tcolorbox}[title=Remark,colframe=black,arc=10pt]
	The Ostrogradsky theorem is verified since there is no singularities in the volume $V$!
	\end{tcolorbox}
	Before continuing let us also recall that in the case of the Ostrogradsky theorem the normal vector $\vec{n}$ is conventionally directed outwardly from the surface.
	
	In the particular case of an electric field, we get some very interesting results. Indeed, given a charge $Q$ located relatively to a frame by the vector $\vec{r}_Q$. Then, we already saw in the section of Electrostatic that in every point in space, there is a field $\vec{E}$ such that:
	
	therefore:
	
	As we can see it, the field $\vec{E}$ has a singularity on $(r_{Q_x},r_{Q_y},r_{Q_z})$. Let us consider a Gauss surface such as the charge $Q$ lies outside of this surface! Within the volume $V$ delimited by the surface $S$ the field $\vec{E}$ then has no singularities. So we can calculate the divergence (\SeeChapter{see section Vector Calculus page \pageref{divergence vector field}}) of $\vec{E}$:
	
	Therefore we get:
	
	So if we calculate the flow through this surface we find (see the section of Vector Calculus for a detailed description of the "nabla" operator represented by the symbol "del"):
	
	The flow is equal to zero!
	
	In the case where the charge $Q$ is located inside the Gauss surface $\vec{\nabla}\circ\vec{E}$ is no longer defined on $(r_{Q_x},r_{Q_y},r_{Q_z})$ we have then:
	
	With $\Phi_B$ being the flow $\vec{E}$ on a small ball  $B$ around the punctual charge $Q$.
	
	In this case:
	
	as the divergence is defined everywhere on $V-B$. So it remains:
	
	But in the case of a sphere, it is quite easy to calculate:
	
	We have:
	
	hence the "\NewTerm{first Maxwell equation}\index{first Maxwell equation}" or "\NewTerm{Gaussian law}\index{Gauss law for the electric field}\label{gauss law}" (or "\NewTerm{Gauss theorem}\index{Gauss theorem for the electric field}") with a slightly condensed notation:
	
	where $\rho_Q$ is the charge density in Coulombs per $\text{m}^3$. On the left we have the integral form of the first Maxwell's (electro-magnetostatics) equation and its differential form on the right.
	
	This equation suggests that the flow of the electric field through a closed surface (hence the circle on the integral) is equal to a given dimensional factor, to the total charge enclosed in the latter surface.
	\begin{tcolorbox}[title=Remark,colframe=black,arc=10pt]
	The integral of the last relation is a line integral (thus evaluated on a curve). In the field electrodynamics, line integrals  apply very often on paths or closed surfaces hence the indication of a circle superimposed on the integral symbol and being named "\NewTerm{circulation of the vector field}\index{circulation of the vector field}".
	\end{tcolorbox}	
	If we now express this equation in terms of the electrical potential for which we have proved in the section Electrostatics that:
	
	We get:
	
	We can write the above relation more aesthetically using the scalar Laplacian (\SeeChapter{see section Vector Calculus page \pageref{scalar laplacian}}) such that we get the relation:
	
	named "\NewTerm{Maxwell-Poisson equation}\index{Maxwell-Poisson equation}\label{maxwell-poisson equation}".
	
	\subsubsection{Second Maxwell Equation (non-existence of magnetic monopole)}
	In the particular case of a magnetic field, we also get some very interesting results.
	
	Indeed, given a current $I$ located relatively by a reference frame by the vector $\vec{r}_I$. Then at each point $\vec{r}$ in space, we proved in the section Magnetostatics that there is a field $\vec{B}$ such that:
	
	Therefore:
	
	As we can see it, the field $\vec{B}$ has a singularity on $(r_{I_x},r_{I_y},r_{I_z})$. Let us consider then a Gaussian surface such that the current $I$ is outside this surface!

	Within the volume $V$ delimited by the surface $S$ then field $\vec{B}$ has no singularities anymore. We can therefore calculate the divergence (\SeeChapter{see section Vector Calculus page \pageref{divergence vector field}}) of $\vec{B}$:
	
	Therefore:
	
	If we calculate the flow through this surface, we then find for the magnetic flux:
	
	The flow is equal to zero!
	
	In the case where the current $I$ is located within the Gauss surface, $\vec{\nabla}\circ\vec{B}$ is no longer defined in $(r_{I_x},r_{I_y},r_{I_z})$ we have then:
	
	with $\Phi_{B'}$ being the flow of $\vec{B}$ on a small sphere $B'$ surrounding partially the straight conductor carrying the current $I$. In this case:
	
	as the divergence is defined everywhere on $V-B'$. Therefore it remains only that:
	
	But in the case of a sphere, it is easy to calculate:
	
	We then have the Gauss's law for the magnetic field:
	
	Indeed, in the case of the magnetic field, $\mathrm{d}\vec{S}$ and $\vec{B}$ are therefore perpendicular:
	
	\begin{tcolorbox}[title=Remark,colframe=black,arc=10pt]
	From which we can deduce that $\vec{B} \perp\vec{E}$ !
	\end{tcolorbox}
	Therefore, given a Gaussian surface in a magnetic field, then the magnetic flux through this surface is equal to:
	
	relation which is the "\NewTerm{second Maxwell equation}\index{second Maxwell equation}" or "\NewTerm{Gauss law for magnetism}\index{Gauss law for magnetism}\label{gauss law for magnetism}" (also sometimes named "\NewTerm{Gauss theorem for magnetism}"). On the left we have the integral form of the fourth Maxwell's equation and its differential form on the right.
	
	This second equation is equivalent to say that there is no "\NewTerm{magnetic monopole}\index{magnetic monopole}" in nature, that is to say, at any positive pole, we have to find a negative one (from a magnet the field lines do not diverge). However, the second equation just add the idea (demonstrated by Dirac) that if it was possible to find a monopole in the Nature, it would be the point source of the magnetic field. We will see this a little further in detail below.
	
	\subsubsection{Third Maxwell Equation}\label{third maxwell equation}
	We shall see in the section of Electrokinetics (because we need some concepts that we have not study yet), that the variation of the magnetic flux in time through a conductive loop induces a voltage in the loop given by the "\NewTerm{Faraday's law}\index{Faraday's law}" or "\NewTerm{Lenz-Faraday's law}\index{Lenz-Faraday's law}":
	
	and we have already proved in the section of Electrostatics that:
	
	where the last equality is valid only in the special case where the pat is collinear to the electric field.
	\begin{figure}[H]
		\centering
		\includegraphics[scale=0.5]{img/electromagnetism/lorenz_law.jpg}
		\caption{Lorenz law illustration}
	\end{figure}
	\begin{tcolorbox}[title=Remark,colframe=black,arc=10pt]
	We will see in the section of Electrokinetics that it is not quite correct to denote the potential with $U$ as above because in fact, Faraday's law expresses the electromotive force EMF (electromotive potential) denoted by $e$ and this potential is not conservative unlike the electrostatic Coulomb potential (for which the integral over a closed path is zero as we have already proved it in the section of Electrostatics).
	\end{tcolorbox}
	For an element $\mathrm{d}\vec{l}$ of a circuit $\Gamma$ it comes:
	
	the sign change is here justified by the Lenz law that said that the induced current (and magnetic flow associated with it) has such a direction that it opposes to the change in the flow through the circuit (see figure above).
	
	If we develop this relation, using Stokes' theorem (\SeeChapter{see section Vector Calculus page \pageref{stokes theorem}}) which is for recall:
	
	Then we have:
	
	Where, as we will see it in the section of Electrokinetics, the electric field above is not the simple Coulomb field but the sum of a Coulomb field and an electromotive field (implicitly generated by Biot-Savart force).
	
	We then have:
	
	And if the surface element does not move in space and only the magnetic field varies with time, then we have:
	
	Therefore:
	
	A trivial solution is then to say that:
	
	We then get finally:
	
	This is the "\NewTerm{third Maxwell equation}\index{third Maxwell equation}" or "\NewTerm{Faraday-Maxwell law}\index{Faraday-Maxwell law}" sometimes still named "\NewTerm{induction law}\index{induction law}". On the left we have therefore the integral form of the third Maxwell's equation and its differential form on the right.
	
	The third equation therefore affirms that a variation of the magnetic field produces an electric field in a conductor loop. We then say that the term with the partial derivative of the magnetic field is the "\NewTerm{magnetic coupling term}\index{magnetic coupling term}". This equation is based on the Faraday's theory.
	\begin{tcolorbox}[title=Remark,colframe=black,arc=10pt]
	Often in the scientific literature, the potential $U (t)$ is simply denoted by a tiny $u$.
	\end{tcolorbox}
	Faraday's law of induction is typically used by small portable devices such as below PEG (Personal Energy Generator) to charge portable electronic devices:
	\begin{figure}[H]
		\centering
		\includegraphics{img/electromagnetism/peg.jpg}
		\caption[Photo of a PEG with a mobile phone]{Photo of a PEG (right) with a mobile phone (left)}
	\end{figure}

	\paragraph{Betatron}\mbox{}\\\\
	Among the many examples of application of the third law that we will see in other sections of this book, there is one that is particularly nice because it is reminiscent of modern physics on a large scale (even if in reality we are very far).
	
	Thanks to the Maxwell equations and the relations proved in the section Magnetostatics, we can make a small non-exhaustive theoretical study of the physical principle underlying one of the oldest non-linear particles accelerator.
	
	One of the first non-linear methods that comes to mind is to accelerate a charged particle through magnetic induction. This type of accelerator is named a "\NewTerm{betatron}\index{betatron}" (the idea is that it accelerates the electrons as fast as the one that appears in the beta decay...) and was conceptualized in the 1930s.	
	
	The betatron is a particle accelerator that injects electrons into a vacuum torus (in white on the photo below) submitted to a magnetic field that will be considered here as homogeneous between the two magnets (in red in the photo below) to obtain intense X-rays or gamma rays useful in certain professional applications (medicine, analysis of structures, etc.). This accelerator is limited by the magnetic field that it can produce or support.
	\begin{figure}[H]
		\centering
		\includegraphics{img/electromagnetism/betatron.jpg}
		\caption{Photo of a Betatron}
	\end{figure}
	For this theoretical study, we will first use the result proved in the section Magnetostatics when we study the Larmor radius: an electron moving in a magnetic field will have a circular path which is perpendicular to the magnetic field.
	
	Then we will also need the third Maxwell equation in the integral form:
	
	which - for recall - said that a change in the magnetic field produces an electric field in a conductor loop (or in a charged particle movement that can be assimilated to a conductor loop!!!!).
	
	We have:
	
	and as the path is circular in the betatron as we have proved in our study of the Larmor radius in the section Magnetostatics, we have:
	
	Now, as the electric field is tangential to the circular path of the electrons and that they go in the opposite direction of that field (direction that is constant at any point of the path), we have since electrons travel in circles:
	
	But we also have:
	
	Then it comes:
	
	therefore:
	
	We would like to calculate the kinetic energy that the negatively charged particle acquires after several rounds. This latter is equal to the work done by the electric field to move the load on the circular path (remember that the magnetic field don't do "work"!!!).
	
	As we have proved in the section of Electrostatics, we have along an electric (constant) field line:
	
	Therefore it comes as the charge travels $N$ times the circumference of the betatron:
	
	\begin{tcolorbox}[colframe=black,colback=white,sharp corners]
	\textbf{{\Large \ding{45}}Example:}\\\\
	Let us consider a sinusoidal magnetic field of magnitude $B_0=1 [\text{T}]$ at a frequency of $f=50$ [Hz], therefore a period $T$ of $20$ [ms]. This means that in $5$ [ms] the magnetic field passes from a maximum to a zero value.\\

	Let us also consider that we have a betatron with a circular path of $1$ [m] and the electron can remain approximately $480,000$ rounds on this circular trajectory with this specific radius without deviating too much (the equivalent of about $3,000$ [km] travelled). The electron is injected with an energy of $2$ [MeV] in the vacuum torus (energy that is already very close to that of the speed of light!).\\
	
	Let us first calculate the initial radius for the trajectory according to the relativistic Larmor radius. For this, first we need the speed corresponding to the energy of $2$ [MeV]:
	
	After some elementary algebraic operations, we find:
	
	Therefore the initial Larmor radius is equal to:
	
	We then for during all the time of the acceleration that will take the electron to a Larmor radius of $1$ [m] a kinetic energy gain of:
	
	Which corresponds in electron volts to:
	
	Hence the energy that was measured experimentally at that time. This energy also corresponds to a speed which is very close to that of light. Thus, with the same calculation as above, we get:
	
	speed reached only in a hundredths of a second!
	\end{tcolorbox}
	\begin{tcolorbox}[title=Remark,colframe=black,arc=10pt]
	Therefore in reality, the centrifugal force increases gradually as the electron acquires kinetic energy (and therefore the speed). This force must be compensated by increasing the Lorentz force accordingly.
	\end{tcolorbox}
	
	\subsubsection{Fourth Maxwell Equation}\label{fourth maxwell equation}
	The fourth Maxwell equation is in our point of view the most important one. It is a generalization of the Ampere's law which has already been presented in the section Magnetostatics and for which we got the circulation $C_B$ of the magnetic field:
	
	Remember that the third Maxwell's equation tells us that the variation of a magnetic field gives generated to an electric field. We can therefore assume that the converse may be probably true (need experimentation to be checked!).
	
	A typical place where we can observe a variation of an electric field, for example, is the capacitor (\SeeChapter{see section Electrokinetics page \pageref{capacitor}}).
	
	We know that:
	
	and that the electric field between two parallel planes, of surface $S$, bearing electric charges $Q_+=-Q_-$, uniformly distributed is given by (\SeeChapter{see section Electrostatics page \pageref{capacitor}}):
	
	where $\sigma$ is the surfacic density.

	This result is independent of the distance $l$ between the planes. The first Maxwell equation give us:
	
	The capacity of a condensator is defined by (\SeeChapter{see section Electrostatics page \pageref{capacitor}}):
	
	we obtained in the particular case of plane and parallel capacitor that the capacity was equal to:
	
	Therefore it comes:
	
	and using the fact that the electrostatic potential is the electric field multiplied by a distance that will be taken in this case as the distance $D$ between the two planes of the capacitor, we have:
	
	As the electric field may change with time, it is often customary to put a tiny $i$ for the electric current variable (this is a tradition that we will see again the section Electrokinetics) and as between the two capacitor planes there only vacuum, then we speak of "\NewTerm{displacement current}\index{displacement current}", which is why the latter relation is often denoted as follows:
	
	By expressing the above expression by using the surface current density, we get:
	
	If the electric field is not uniform in space and thus depends on the spatial coordinates, we'll use the partial derivatives such as:
	
	The displacement current generates a magnetic field calculable using Ampere's law:
	
	In all phenomena where we observe a charge displacement, we can assume that there is creation of a displacement current which is superimposed on the current flow because of capacitive effects in the material. We write therefore:
	
	where we have (recall of the section of Electrostatic and Magnetostatics):
	
	Furthermore, the Stokes theorem provides us that:
	
	therefore:
	
	and we get from this finally:
	
	This is the "\NewTerm{fourth Maxwell equation}\index{fourth Maxwell equation}" or "\NewTerm{Maxwell-Ampere equation}\index{Maxwell-Ampere equation}". On the left we have the integral form of the fourth Maxwell's equation and its differential form on the right.
	
	The fourth and last Maxwell equation combines the creation of a magnetic field to any variation of an electric field and / or the presence of an electric current (the presence of an electric current being a sufficient condition but not necessary in view of the second term). We then say that the term with the partial derivative of the electric field is the "\NewTerm{term of electrical coupling}\index{term of electrical coupling}".
	
	To summarize a bit we have therefore the following four Maxwell equations named "\NewTerm{local forms of Maxwell's equations}\index{local forms of Maxwell's equations}" in differential form (when the integrals are not indicated):
	
	In the case where $\mu_r,\varepsilon_r\neq 1$, that is to say if we do not work in vacuum but in the matter, we write the local Maxwell equations in the following form:
	
	where $\vec{D}$ is (for recall) named the "\NewTerm{displacement field}\index{displacement field}" or "\NewTerm{electric induction}" and (for recall) $\vec{H}$ is the "\NewTerm{magnetic excitation}\index{magnetic excitation}". We also have $\chi$ that is the "\NewTerm{electric susceptibility}\index{electric susceptibility}" and $\chi_m$ the "\NewTerm{magnetic susceptibility}\index{magnetic susceptibility}". Traditionally $\varepsilon_r$ is named the "\NewTerm{relative permittivity}\index{relative permittivity}" and $\chi_r$ the "\NewTerm{relative permeability}\index{relative permeability}".
	
	\begin{tcolorbox}[colback=red!5,borderline={1mm}{2mm}{red!5},arc=0mm,boxrule=0pt]
	\bcbombe Caution!! $\vec{E}$ is a reaction of the vacuum to the field $\vec{D}$. This is due to the permittivity vacuum constant  set in the integral (at least that's one way of seeing the thing...).
	\end{tcolorbox}
	
	But in the vacuum and in the case where we consider an absence of charges, we get:
	
	This result is important because it expresses the possibility of the propagation of the electric and magnetic field even in the absence of sources!!! We will use these equations to determine the electromagnetic wave equations further below.
	\begin{tcolorbox}[title=Remark,colframe=black,arc=10pt]
	It is possible to express the Maxwell equations under relativistic form... but in reality, as we have already noticed, the equations are unchanged! Indeed, Maxwell's equations are already relativists. This is not surprising, because the vectors of electric and magnetic fields, the photons, travel at the speed of light. At this speed, relativity is the Queen and a correct theory could only be relativistic. However, we can express the latter equations using tensor mathematical notations (see below our demonstration of the tensor of the electromagnetic field). In this form the four equations become incredibly simple and compact (only one extremely short equation). Formulated in this way, the electric and magnetic fields are written as a single field named of course the "\NewTerm{electromagnetic field}\index{electromagnetic field}". It is a tensor field as we shall see later.
	\end{tcolorbox}
	In integral form the four Maxwell equations are written:
	
	For information Maxwell himself didn't like electrodynamics the way he conceived it because he couldn't come up with an underlying mechanical model. Back then, the standard of beauty was a mechanical clockwork universe, but in Maxwell's theory electromagnetic fields just are - they are not made of anything else, no gears or notches, no fluids or valves. Maxwell was unhappy about his own theory because he thought that only \og \textit{when a physical phenomenon can be completely described as a change in the configuration and motion of a material system, the dynamical explanation of that phenomenon is said to be complete.} \fg{} Maxwell tried for many years to explain electric and magnetic fields by something that would fit the mechanistic world-view. Alas, in vain.

	\subsubsection{Magnetic Monopoles}
	Notice that opting for the natural measurement system where $c=1$, then we have to Maxwell's equations in vacuum:
	
	since as we prove it further below, in vacuum:
	
	Then the transformation:
	
	brings the second pair of preceding equations the first one!!!! This symmetry of Maxwell's equations is named "\NewTerm{duality}\index{duality}" and that's a clue that suggests that the electric and magnetic fields are only the unified parts of a whole.
	
	Moreover, if we introduce the following complex field:
	
	the duality (taking the real part only), then is written:
	
	the pair of Maxwell's equations indicated above is then reduced to (we use the property of linearity of the cross product) only one pair of equations which we must not forget to take only the real part:
	
	However, this symmetry does not extend to the Maxwell equations with sources expressed in the natural system by:
	
	but half of the time it does not work (make the substitution of $\vec{\mathcal{E}}$ you will see that you always get one of the equations on the pair that is consistent and the other not!). The trick then is to separate the two densities in their respective imaginary and real part (as we can see, generalize any physics theory to complex numbers always bring us to very interesting stuff as for maths!!!):
	
	We then get (always without forgetting to take the real parts and not forgetting that we are in natural units):
	
	then we simply have to put $\rho_m=\vec{j}_m=0$. These equations are certainly charming but their generalization brings nothing new because no magnetic charge expressed as:
	
	and named "\NewTerm{magnetic monopole}\index{magnetic monopole}" was observed to this date. In an experimental context, we say then that $\rho_Q,\vec{j}$ are real such that we have well:
	
	The previous system rewritten as following is named "\NewTerm{Symmetrized Dirac-Maxwell equations}\index{symmetrized Dirac-Maxwell equations}" (in natural units):;
	
	To see if we have the right to write the relations above, remember first that if we take the divergence of the curl of the electric field, we know that this operation (\SeeChapter{see section Vector Calculus page \pageref{differential operators identities}}) is always equal to zero regardless of the function considered:
	
	Then we have the following result:
	
	Hence:
	
	Therefore:
	
	Thus after rearrangement:
	
	So is the fact of falling back on a continuity equation (identical in the form to the thermodynamics continuity equation, fluid mechanics continuity equation, electric field continuity equation or even that of the quantum probability continuity equation, etc.) that would have brought Dirac to complete the four Maxwell equations we have written just above.
	
	Remember that current is the movement of charge. The continuity equation says that if charge is moving out of a differential volume (i.e. divergence of current density is positive) then the amount of charge within that volume is going to decrease, so the rate of change of charge density is negative. Therefore the continuity equation amounts to a conservation of charge.
	
	\subsection{Charge conservation equation}\label{charge conservation equation}
	So we proved so far the four Maxwell's equations which are the foundations of classical electrodynamics.

	The Maxwell's equations can be divided into two groups:
	\begin{enumerate}
		\item Equations without sources:
		

		\item Equations with sources\footnote{The fourth Maxwell equation as denoted here is sometimes named the "\NewTerm{Oersted's law}\index{Oersted's law}"} (in vacuum below):
		
	\end{enumerate}
	Deriving the first equation with sources in respect to time:
	
	by taking the divergence of the second we get:
	
	The divergence of a curl (rotational) is always zero as we have prove it in the section of Vector Calculus and therefore the last expression is zero. But since a reader asked us, we detail this result more explicitly simplifying a bit:
	
	but, $\vec{\nabla}\circ\vec{B}=0$ and therefore:
	
	After simplification and using natural units (\SeeChapter{see section Principia page \pageref{natural system units}}) we get:
	
	which is named the  "\NewTerm{charge conservation equation}\index{charge conservation equation}" or "\NewTerm{equation of continuity}\index{equation of continuity}" and said that in two near instants $t+\mathrm{d}t$, the variation $\mathrm{d}Q$ of the charge contained in a closed surface defining a system can only be exclusively  attributed to a charge exchange with the outside.
	
	This equation is very important because it involves in the study of relativity, the load is a translation invariant quantity.
	
	\subsection{Gauge Theory}\label{gauge theory}
	Before you start reading this subsection it is of primary importance for the reader to go for a ride in the Algebra chapter of this book, in which there is a section Vector Calculus where we prove some differential vector operators that are of first importance in physics and especially for gauge theory and where we also study their main properties.
	
	What will follow is very important because besides the fact that we are going to naturally present a new vector field (the vector potential) which is essential in certain equations of relativistic quantum physics (see section of the same name page \pageref{relativistic quantum physics}) we will reuse this gauges approach in the section of Wave Quantum Physics where the consequences are much more important!
	
	Given the known relation:
	
	There exists by the properties of the divergence and curl (rotational) operators (\SeeChapter{see section Vector Calculus page \pageref{differential operators}}) a "\NewTerm{vector potential $\vec{A}(\vec{r},t)$}\index{vector potential}" such that:
	
	thus satisfying then (the divergence of the curl - rotational - of a vector field is always zero because it is a mathematical property proven in the section of Vector Calculus):
	
	and in the context of magnetostatic named "\NewTerm{magnetic potential}\index{magnetic potential}". 
	\begin{tcolorbox}[title=Remark,colframe=black,arc=10pt]
	The vector potential is therefore a potential... and a vector! Similarly as we have define a potential $U$ which derivates from an electric field $\vec{E}$, we can thus define a potential $\vec{A}(\vec{r},t)$ for the field $\vec{B}$. But for technical reasons (coming from the expression of the curl - rotational - of $\vec{E}$ and of $\vec{B}$ in the Maxwell's equations), the potential vector $\vec{A}(\vec{r},t)$ is not that simple as $U$ and can not be expressed as a simple scalar, we must use a vector potential.
	\end{tcolorbox}
	\begin{figure}[H]
		\centering
		\includegraphics[scale=0.9]{img/electromagnetism/magnetic_potential.jpg}
		\caption{Vector Magnetic Potential $\vec{A}$}
	\end{figure}
	Thus, a depiction of the $\vec{A}$ field around a loop of $\vec{B}$ flow (as would be produced in a toroidal inductor) is qualitatively the same as the $\vec{B}$ field around a loop of current.
	
	The reader will have notice that at the opposite of the electric potential they can be no magnetic field but still a magnetic potential.
	
	If we take the relation $\vec{B}=\vec{\nabla}\times\vec{A}$ in the Maxwell equation $\vec{\nabla}\times\vec{E}=-\partial \vec{B}/\partial t$ we get:
	
	We put now (the notation $\vec{F}$ has no relation with the Newtonian force!):
	
	and we use the mathematical properties of the curl (rotational) and gradient operators to write a new relation (the "$-$" sign is an anticipation of what will follow):
	
	hence\label{electric field with potential vector}:
	
	where $\phi(\vec{r},t)$ is a "\NewTerm{scalar potential}\index{scalar potential}" that is well know to us as being the electric potential $U$. 
	\begin{tcolorbox}[title=Remarks,colframe=black,arc=10pt]
	\textbf{R1.} The field $\vec{F}$ seems to obey the same properties as that of the gravitational field (Newton-Poisson law), but it seem to be only a curiosity (units and other mathematical properties are not equivalent).\\
	
	\textbf{R2.} The reader probably easily see that if the vector potential is zero, then we fall back on (\SeeChapter{see section Electrostatic page \pageref{derivation of electric field by the potential}}):
	
	which reinforces the assumptions of previous developments (and that's not ...)
	\end{tcolorbox}
	In addition, the fields $\vec{E}$ and $\vec{B}$ remain unchanged if we do in the previous relations the following replacements (the terms cancel trivially):
	
	where $\psi$ is an arbitrary function of $\vec{r}$ and $t$.
	
	We name such a transformation a "\NewTerm{gauge change}\index{gauge change}". The freedom on the choice of potentials allow us to impose a constraint that we name "\NewTerm{gauge constraint}\index{gauge constraint}".
	
	There are several ways of forming this constraint among which we distinguish two very common in physics!

	Thus we will use either the "\NewTerm{Lorentz gauge}\index{Lorentz gauge}" by imposing:
	
	or the "\NewTerm{Coulomb gauge}\index{Coulomb gauge}" by imposing:
	
	holds. Thus, $\psi$ should satisfy:
	
	The relation:
	
	which is named "\NewTerm{Poisson equation of the potential vector}\index{Poisson equation of the potential vector}".
	
	Similarly, to show that it is always possible to impose the Lorenz condition, we simply need to find $\psi$ in the above equations:
	
	such that the relation (Lorenz gauge):
	
	is satisfied. Therefore, $\psi$ must satisfy:
	
	Or in other words and in a more condensed manner:
	
	where the operator:
	
	is by definition named the "\NewTerm{Alembertian}\index{Alembertian}\label{alembertian}" (we will often encounter this term from now as well in the section of Electrodynamics and Wave Quantum Physics) that is also invariant under Lorentz transformation as we will prove it in the section of Special Relativity.
	
	Reporting the equations:
	
	in the other two Maxwell equations in vacuum:
	
	we get, by making appear the Laplacian of a vector field $\Delta \vec{A}$ by on of the properties of the curl, gradient and divergence vector operators (\SeeChapter{see section Vector Calculus page \pageref{differential operators identities}}):
	
	the following relations:
	
	the latter relation being named the "\NewTerm{arbitrary gauge}\index{arbitrary gauge}".
	
	For the Lorenz gauge, these last two equation simplify to (feel free to contact us if you do not see how):
	
	that we name "\NewTerm{wave equations for electromagnetic potentials}\index{wave equations for electromagnetic potentials}" in analogy with the wave equations of electric and magnetic fields that we will determine further below.
	For the Coulomb gauge, the same equations simplify to:
	
	knowing that $c^2=1/(\varepsilon_0\mu_0)$ we can write the two wave equations of electromagnetic potentials in the form:
	
	Let us now put $A^0=\phi/c$ (to homogenize the units) such that we define a "\NewTerm{four-vector potential}\index{four-vector potential}" which allows us to write vectorially the two above relations in unified manner:
	 
	\begin{tcolorbox}[title=Remark,colframe=black,arc=10pt]
	The fact that the Alembertian of the four-vector potential  is expressed from the four-vector current which is contravariant (\SeeChapter{see section Special Relativity page \pageref{four-vector current}}) bring us to put that the four-vector potential is itself contravariant!
	\end{tcolorbox}
	Relation that we will we denote is a more condense form as follows:
	
	where $j^\alpha$ is name "\NewTerm{four-vector current}\index{four-vector current}".
	\begin{tcolorbox}[title=Remark,colframe=black,arc=10pt]
	We will see again this four-vector during our determination of the electromagnetic field tensor further below (except that we will be in natural units but this does not change the idea...).
	\end{tcolorbox}
	The four-vector potential as defined above leads us to be able to write the (quadrivergence) Lorenz gauge by making use of tensor notation:
	
	Which ultimately allows you to write the Lorenz gauge in the covariant form below:
	
	It is therefore an equation of the form of the Klein-Gordon equation for a massless particle (\SeeChapter{see section Relativistic Quantum Physics page \pageref{free Klein-Gordon equation}}). So we can say in a sense that the invariance of the electromagnetic gauge is connected to the fact that the mass of the photon is zero!
	\begin{tcolorbox}[title=Remark,colframe=black,arc=10pt]
	It is useful to notice that the fact that writing $\dfrac{1}{c}\dfrac{\partial }{\partial t}=\partial_0$ (with or without natural units where $c=1$) is a notation that will also be adopted in our study of the Dirac equation (\SeeChapter{see section Relativistic Quantum Physics page \pageref{dirac equation}}) or also in quantum field theory (except that there will be an imaginary part).
	\end{tcolorbox}
	These notations finally lead us to be able to write:
	
	We obtain the continuity equation:
	
	that is tensor equivalent form of the following relation (see prof earlier above):
	
	To summarize roughly:
	
	A given number of physical effects are modelized, depending on the case, by fields that can be scalar, vector, tensor, or spinor fields that we therefore name "gauges". A given number of physical phenomena seems to comply with conditions of say of "symmetry", vis-à-vis these gauges. This symmetry is expressed by what we therefore call a "gauge invariance".

	For example, the field that modelize well the electromagnetic field is as we have seen it, a four-vectors field consisting of a scalar potential $\phi$ (whose gradient is the electric field $\vec{E}$) and a vector-potential $\vec{A}$ (whose curl is the magnetic field $\vec{B}$). It is this quadrivectoriel field that helps us to modelize the electromagnetic field that is named "gauge".

	It turns out that we were getting exactly the same physical effects on a system of electric charged particles if we replace this gauge by another gauge by adding it a gauge  constraint (typical example between the Lorenz gauge or Coulomb gauge as seen above). The invariance of the laws of physics when switching from a gauge to another is a "gauge invariance". In the case of the electromagnetic field, this gauge invariance turns expressing the electric charge conservation (as we have just proved it).

	Mathematically, such gauges changes appear to be the result of the action of a symmetry group of infinite dimensions (transforming these into each other gauges) that we name the "gauge group" of the considered interaction  (here the electromagnetic interaction).

	For the gravitational field, for example (\SeeChapter{see section General Relativity page \pageref{general relativity}}), the gravitational interaction is modelled by a symmetric tensor field of rank $2$ with a given signature. This metric field is distributed on a 4D variety  modelling the space-time. This is the gauge of the gravitational interaction. According to General Relativity (the equivalence principle) we do not change anything to the gravitational interaction if we change the system of space-time coordinates in which we express the metric. The portion of an expression of the metric to another by changing coordinate system is also a gauge change. The invariance of General Relativity gauge then expressed the possibility to change to a gauge to another without changing the geodesic followed by the testing particles falling freely in the gravitational field modelled by the metric field.

	The invariance of General Relativity gauge is what we named "diffeomorphism invariance" (bijective change of coordinate system with some degree of regularity) and the General Relativity gauge group is the "diffeomorphisms group" of $\mathbb{R}^2$ (named the "soft group").

	It should be noted also that the potential-vector $\vec{A}$ is perhaps not so virtual as it may seem. Indeed, it is possible to modify the trajectories of charged particles passing outside a cylindrical volume where there is a magnetic field $\vec{B}$ induced by an electric current (travelling in the winding of a solenoid where this field $\vec{B}$ is "trapped" ). It is therefore possible to influence the particle trajectory circulating in an area where the magnetic field $\vec{B}$ is zero but where its vector potential equation is not!!!

	Furthermore, we will use the results we just get here for our study of the Yang-Mills theory in our way to the electroweak unification (see the Standard Model in the section of Field Quantum Theory). 
	\begin{tcolorbox}[title=Remark,colframe=black,arc=10pt]
	The well-known experiment that involves the vector potential is the Aharonov-Bohm experiment. The "\NewTerm{Aharonov–Bohm effect}\index{Aharonov–Bohm effect}", sometimes named the "\NewTerm{Ehrenberg–Siday–Aharonov–Bohm effect}\index{Ehrenberg–Siday–Aharonov–Bohm effect}", is a quantum mechanical phenomenon in which an electrically charged particle is affected by an electromagnetic potential $(U, \vec{A})$, despite being confined to a region in which both the magnetic field $\vec{B}$ and electric field $\vec{E}$ are zero. The underlying mechanism is the coupling of the electromagnetic potential with the complex phase of a charged particle's wave function, and the Aharonov–Bohm effect is accordingly illustrated by interference experiments.
	\end{tcolorbox}
	
	\subsubsection{Electromagnetic field tensor}
	In electromagnetism, the "\NewTerm{electromagnetic tensor}\index{electromagnetic tensor}\label{electromagnetic tensor}" or "\NewTerm{electromagnetic field tensor}\index{electromagnetic field tensor}" (sometimes named the "\NewTerm{field strength tensor}\index{field strength tensor}", "\NewTerm{Faraday tensor}\index{Faraday tensor}\label{faradey tensor}" or "\NewTerm{Maxwell bivector}\index{Maxwell bivector}") is a mathematical object that describes the electromagnetic field in space-time of a physical system. The field tensor was first used after the $4$-dimensional tensor formulation of special relativity was introduced by Hermann Minkowski. The tensor allows some physical laws to be written in a very concise form.
	
	To determine the electromagnetic field tensor let us assume at first that the action (\SeeChapter{see section Analytical Mechanics page \pageref{action integral}}) of an electric charged particle in an electromagnetic field is given by (a priori empirical choice... but you'll see a little further):
	
	\begin{tcolorbox}[title=Remark,colframe=black,arc=10pt]
	The notation $S_0$ is reserved to the action of a free particle (\SeeChapter{see section Special Relativity page \pageref{relativistic lagrangien}}).
	\end{tcolorbox}
	The Lagrangian for an electric charged particle in an electromagnetic field is therefore the sum of the Lagrangian of the particle interacting with the electromagnetic field $L_1$ added to the Lagrangian of the free particle $L_0$ (\SeeChapter{see section Special Relativity page \pageref{relativistic lagrangien}}):
	
	\begin{tcolorbox}[title=Remark,colframe=black,arc=10pt]
	This is therefore the Lagrangian of the interaction of the particle with the field added to the Lagrangian of the mass of the particle. Thus we see that it still missing the Lagrangian of the electromagnetic field itself in the case of absence of electric charges (named: "Lagrangian of the free field") but we will see it further below.
	\end{tcolorbox}
	This is therefore (a priori) the Lagrangian of a charged particle in an electromagnetic field having for source a given quantity of electric charges in relative movement.
	
	We will now prove that this Lagrangian is correct (the previous relation could be stated as a theorem)!
	
	For this remember first that the general momentum is (\SeeChapter{see section Analytical Mechanics page \pageref{general momentum} and Special Relativity page \pageref{relativistic lagrangien}}):
	
	To verify that we made the right choice of Lagrangian initially, we will obtain the equations of motion and ensure that they coincide with the Lorentz force. The Lagrange equations are in this case:
	
	But we have:
	
	and therefore:
	
	But we pointed out in the definition of the scalar potential that $\vec{E}=-\vec{\nabla}\cdot\phi$ hence:
	
	We should necessarily have by analogy with the Lorentz force:
	
	So we need before proceeding to check that:
	
	With:
	
	In components:
	
	Therefore:
	
	That is:
	
	Therefore, distributing the terms:
	
	and as:
	And as:
	
	Therefore we have indeed the equality (yes i know written like this it looks stupid but don't forget the relation above we start from):
	
	These developments thus confirm our initial hypothesis as what the action of the field can be written:
	
	and that it expresses the interaction of a charged particle with a magnetic field (since it contains the Lorentz force!).

	So now we have proved that the "\NewTerm{Lagrangian of the current-field interaction}\index{Lagrangian of the current-field interaction}":
	
	which we assumed empirically the shape earlier and that is finally well correct!

	The action integral is then written:
	
	Let us introduce the velocity $\vec{v}$ of the particle in the form $\vec{v}=\mathrm{d}\vec{r}/\mathrm{d}t$ and the integral is then written:
	
	Let us recall that we have proved in the section of Special Relativity the invariant:
	
	and also:
	
	As the space-time intervals are invariants such as (\SeeChapter{see section Special Relativity page \pageref{interval invariant}}):
	
	Let us introduce the velocity $\vec{v}$ of the particle in the form $\vec{v}=\mathrm{d}\vec{r}/\mathrm{d}t$ and the integral is then written:
	
	Let us recall that we have proved in the section of Special Relativity the invariant:
	
	and also:
	
	As the space-time intervals are invariants such as (\SeeChapter{see section Special Relativity page \pageref{interval invariant}}):
	
	If the repository O' is not in movement ($\mathrm{d}x'=\mathrm{d}y'=\mathrm{d}z'=0$), we have:
	
	hence:
	
	which is also written as\label{action variation}:
	
	Therefore:
	
	Now let us use of the contravariant four-vector potential (see above):
	
	and the contravariant four-vector displacement (\SeeChapter{see section Special Relativity page \pageref{four-vector displacement}}):
	
	The expression of the action of an electric charged particle in an electromagnetic field and in a Minkowski metric $\eta_{ij}$ (\SeeChapter{see sections of Special Relativity page \pageref{minkowski metric} and General Relativity page \pageref{minkowski metric general relativity}}) is finally reduced to the condensed expression:
	
	with therefore:
	
	without forgetting that here we use the metric $(+, -, -, -)$ (\SeeChapter{see sections of Special Relativity page \pageref{minkowski metric} and General Relativity page \pageref{minkowski metric general relativity}}).
	
	Let us notice that the integral action in the absence of magnetic and electric field is the written:
	
	which corresponds well to what we have obtained in the section of Special Relativity for a free particle!

	According to the principle of least action, the action integral has zero variation to the effective motion of the particle, therefore:
	
	\begin{tcolorbox}[title=Remark,colframe=black,arc=10pt]
	By the equality with zero, we can eliminate the minus sign before the integral.
	\end{tcolorbox}
	Using the expression of the curvilinear abscissa (\SeeChapter{see section Special Relativity page \pageref{minkowski metric}}):
	
	for the Minkowski metric, we can write (remember that in the Euclidean metric only the terms of the diagonal where $i=j$ are non-zero):
	
	Therefore:
	
	the preceding integral is then written:
	
	This gives using curvilinear components (\SeeChapter{see section Tensor Calculus page \pageref{curvilinear coordinates}}):
	
	Let us integrate by parts (\SeeChapter{see section Differential and Integral Calculus page \pageref{integration by parts}}) the first integral:
	
	But as:
	
	Therefore:
	
	with:
	
	can be written:
	
	The quantities $\delta x_i$ being arbitrary, the expression in brackets is zero:
	
	Let us put:
	
	The contravariant quantities $F^{ik}$ form the contravariant components of what we named the "\NewTerm{tensor of the electromagnetic field}\index{tensor of the electromagnetic field}" or "\NewTerm{Faraday's tensor}\index{Faraday's tensor}" (hence the $F$...) or more commonly the "\NewTerm{Maxwell tensor}\index{Maxwell tensor}". We say then that $F^{ik}$ the "\NewTerm{curl of (the magnetic) potential}\index{curl of (the magnetic) potential}".

	The "\NewTerm{equations of motion of a particle in an electromagnetic field}\index{equations of motion of a particle in an electromagnetic field}\label{equations of motion of a particle in an electromagnetic field}" then take the form:
	
	that some physicists name the "\NewTerm{geodesic corrected by a Lorentz force}\index{geodesic corrected by a Lorentz force}".	
	
	It is important not notice that the term:
	
	corresponds obviously  to the electric field seen by the observer!
	\begin{tcolorbox}	[title=Remark,colframe=black,arc=10pt]
	The tensor of the electromagnetic field is invariant under the transformations:
	
	Indeed:
	
	\end{tcolorbox}
	In Minkowski metric $\eta_{ij}$ (we will need the tensor of the electromagnetic field in the section Special Relativity, hence the choice of this metric), we have however:
	
	Which gives:
	
	The term $\partial_jA^k-\partial^k A_j$ is often denoted $F^{ik}$ (even if it is not more fully contravariant).

	It remains to us to determine the contravariant components of the tensor $F^{ik}$ (tensor which has the property of being antisymmetric such that $F^{ik}=-F^{ki}$ as we will see further below).

	Let's start with the simplest. We assume as obvious that:
	
	Then, remembering that $\vec{B}=\vec{\nabla}\times\vec{A}$:
	
	Hence (by choosing the Minkowski metric with signature $(+, -, -, -)$):
	
	Which give us so far:
	
	\begin{tcolorbox}[title=Remark,colframe=black,arc=10pt]
	Strictly speaking not to confuse the Faraday's tensor $F^{ik}=\partial^iA^k-\partial^kA^i$ with its matrix form, we should put the first term of the above equality between brackets $[F^{ik}]$ as we have already mention it in the section of Tensor Calculus!
	\end{tcolorbox}
	Now, being known that $x_0=ct$ and $A_0=\phi/c$ the other components of the tensor $F^{ki}$ are written taking into account that:
	
	and therefore:
	
	thus, with the partial contravariant  derivatives according to the Minkowski metric:
	
	Thus we have for the tensor of the electromagnetic field in contravariant components with and always with the Minkowski signature $(+, -, -, -)$:
	
	So that the equation of motion is finally:
	
	But as we will see it in the section of Special Relativity, the real tensor of the electromagnetic field is defined by (still in the metric $+, -, -, -$):	
	
	so that the Lorentz transformations are conform.

	The expression in tensor form of the electromagnetic field clearly shows the unity of the electromagnetic field while generally the electric and magnetic fields are considered separately in classical theory.

	But as in theoretical physics we often work in natural units (this is somewhat the "norm" ...), then we have in advanced graduate books:
	
	Hence the equation of motion in natural units:
	
	By denoting now the components of $1$ to $4$ instead of $0$ to $3$ (it is easier for students to find their way in the matrix) and without forgetting that the partial derivatives are covariant and adopting again the natural units such as $\mu=\varepsilon=1$ (verbatim $c=1$), the two Maxwell equations with source are written:
	
	Using the tensor of the electromagnetic field, it remarkably appears that these two equations can be written as in the condensed tensor equation:
	
	where $j^\nu$ is the "\NewTerm{four-vector current}\index{four-vector current}\label{four vector current}" defined by (in natural units!):
	
	Using the first definition of the Faraday's tensor (the one where the field components are divided by $c$) and taking as known (we will prove it later) that $c=1/\sqrt{\varepsilon_0\mu_0}$ we have in the SI system:
	
	with:
	
	As we shall immediately see, the temporal part of this equation gives the divergence of the electric field and the spatial part the rotational magnetic field.
	
	Indeed:
	
	This is latter equality is the equivalent of:
	
	And we also have:
	
	These three sets of equalities are the equivalent of:
	
	Also the both Maxwell equations:
	
	can be written in tensor form:
	
	Indeed we have first:
	
	that corresponds obviously to:
	
	and:
	
	These three sets of equalities are the equivalent of:
	
	Finally, all the Maxwell equations, adopting the natural units, are implicitly contained into the both relations:
	
	We can also use a the pseudo antisymmetric tensor of rank $4$ that can be seen as a generalization of the Levi-Civita symbol (\SeeChapter{see section Tensor Calculus page \pageref{levi civita symbol}}) such that we can write:
	
	with:
	
	The Lagrangian we have determined earlier above, however, is not complete. Indeed, when we apply the variational principle, we have already seen many times in the various section of this book (Classical Mechanics, Wave  Mechanics, Magnetostatic, Special Relativity, General Relativity, etc.) that we could get the equations of motion ( trajectories) of the subjects (bodies) of interest. The equations obtained also contain the parameters that explained the source of this movement (material properties, speed, field, etc.) as it was the case before!

	Previously, we applied the variational principle on the Lagrangian of the charge-field interaction (electrostatic + magnetic) and got the equation of motion corrected by the Lorentz force.

	When we determined the equations of motion of the charged particle from the principle of least action, we fixed the electromagnetic field (the field is known) and we varied the trajectory. The variational principle, must also allow us to get the field equations from the opposite approach: we fix the path of the particle (known of course) and we vary the electromagnetic field (potential and tensor field).

	We should then obtain the Maxwell equations that, in the same way that we get what makes the motion of the particle when fixing the field in the variational principle, should gives us the information on what is the source of electric and magnetic field when the trajectory is fixed in the variational principle (I hope you have followed that explanation........).

	The desire is then great to just simply start again from the action obtained earlier:
	
	and apply to it a variation on the field.

	As we have:
	
	we can write:
	
	Let us consider electrical charges moving at the speed $v$ and let us write the following quantity (do not forget that we continue to work in natural units such as $\mathrm{d}x_0=c\mathrm{d}t=\mathrm{d}t$!):
	
	with in natural units: equation
	
	Therefore we have:
	
	If we apply the variational principle only on the field (constant in amplitude so that the source of the field is constant and therefore $\delta j_i=0$) and that we therefore consider the movement of electric charges known, it is immediate that the first term above is zero. Then we have:
	
	that this integral to be zero it would require that $j_i$ is zero ... which is pretty annoying if we wish to determine the characteristics of a source which then would not exist ... Therefore, we notice that something is missing in our Lagrangian!

	The idea is the following: we know a tensor equation which involves the current density which is $\partial_\mu F^{\nu\mu}=j^\mu$ and that implicitly contains the only two Maxwell equations that give information on the source of the respective electric and magnetic fields (the other two giving properties of the field, not sources!) that is (always in natural units) for recall:
	
	It is therefore sufficient to get these two equations (thus the corresponding tensor equation) following the variational principle to get the properties of the source of the field.
	
	This simply means that ideally, we should (and expect) to have:
	
	where the integral equation vanishes exactly when $\partial_mu F^{\nu\mu}=j^\nu$!

	It is then tempting to write something of the form (notice that we lowered the index of the potential $A$ and raised that of the current density $j$ in the second integral which does not change anything mathematically speaking to the result):
	
	We can use the following property of the Lagrangian quantities for to determine the missing expression "???": they all are invariant. In other words and for recall their pseudo-norm (scalar) is equal by Galilean basis change (\SeeChapter{see section Mechanics page \pageref{Galilean Relativity Principle} and section Special Relativity page \pageref{special relativity principle}}) such that:
	
	The first relation is quite obvious, we have already proved it many times. The second is perhaps less obvious therefore let us give a small indication (not general) to check whether it's correct: $A_\nu j^\nu$ is the dot product of $j$ and $A$. If we apply the same (four) rotation to the two vectors, since Lorentz transformations are rotations (\SeeChapter{see section Special Relativity page \pageref{hyperbolic rotation matrix}}), the angle between $j$ and $A$ remains unchanged and thus the dot product.

	But we must not forget that need to find the quantity "$???$" as a scalar invariant involving the Faraday tensor in one way or another.

	We can then try directly with the following quantity (knowing in advance thanks to our precursors that this is the right hypothesis):
	
	involving the covariant Faraday's tensor $F_{\mu\nu}$ and contravariant $F^{\mu \nu}$ because we know that:
	\begin{enumerate}
		\item It is a scalar invariant! Indeed, let us write $F_{\mu\nu}F^{\mu\nu}$ in terms of electric and magnetic fields to understand the physical meaning of that latter (still in natural units)\label{electromagnetic tensor invariant proof}:
		
		\begin{tcolorbox}[title=Remark,colframe=black,arc=10pt]
		If we were not in natural units, the calculation result would be of the form:
		
		The quantity:
		
		(or $\|\vec{B}\|^2-\|\vec{E}\|^2$ in natural units) is therefore an invariant of the field.
		\end{tcolorbox}
		\begin{tcolorbox}[colframe=black,colback=white,sharp corners]
		\textbf{{\Large \ding{45}}Example:}\\\\
		In a repository O, let us consider a plane electromagnetic wave. The modules of the electric field and magnetic field are linked by the relation $E=cB$ (see below for proof). The invariant of the considered field is therefore zero. In another reference frame, with the same structure of the field, then we will also have $E'=cB'$.
		\end{tcolorbox}
		
		\item Because a variational on this term gives:
		
		where we guess ... by digging a little bit, that $\delta F_{\mu\nu}$ implicitly contains the term $\delta A^\nu$. We also see that a factor $2$ appears such that we will have to introduce a normalization constant $\alpha$,  that anyway has to be introduced for a matter of homogeneity of the units of the expression of the action.
	\end{enumerate}
	
	So eventually let try with something like:
	
	Now to find the equations of the electromagnetic field, we consider that the movement of electric charges are known and we use the principle of least action by varying only the components of the vector potential and those of the tensor of the electromagnetic field.

	It follows that the variation of the first integral is zero and that it remains:
	
	But we know that $F_{\mu\nu}$ is equal to $-F_{\nu\mu}$ since the Faraday tensor is antisymmetric:
	
	Nothing prevents us to swap the indices $\mu$,$\nu$ in the first term at the right of the equality:
	
	But we know that $F_{\mu\nu}$ is equal to $-F_{\nu\mu}$ since the Faraday tensor is antisymmetric:
	
	Nothing prevents us to swap the indices $\mu$,$\nu$ in the first term at the right of the equality:
	
	So finally:
	
	Let us look at the second integral:
	
	Applying the Fubini theorem (\SeeChapter{see section Differential and Integral Calculus page \pageref{fubini theorem}}) that says that we can integrate in any order the integration variables (under certain conditions) then we can apply the integration by parts (\SeeChapter{see section Differential and Integral Calculus page \pageref{integration by parts}}) to write:
	
	where $\mathrm{d}S$ represents the boundary surface of the hyper-volume $\mathrm{d}\Omega$ which we were initially integrating and that omits the variable considered by the choice of the superscript $\nu$.

	Now according to the superscript $\nu$ concerned, the boundaries of the first term of the equality:
	
	will be on the components of time or components of space. If we focus on the temporal boundaries of integration, these are the initial and final moments of the action at which we apply this variational.
	
	But at the time ends, the variational of the vector potential $\delta A_\mu$ is zero (by definition), therefore the integral over the time component will be zero.

	Now on the spatial components, the (spatial) boundaries are those that permits to integrate the border surface of the hyper-volume at the final time. If this one is taken as being the infinity, the radius of the border area will be infinite and at every point of this surface, the energy carried by the field and the amplitude of the field components will be zero (see proof further below).

	Therefore the variational of action is finally written:
	\\
	The variations of the vector potential being arbitrary, the preceding integral will be zero if the integrand is also zero, hence the relations:
	
	which brings us to:
	
	\begin{tcolorbox}[title=Remark,colframe=black,arc=10pt]
	We return on this Lagrangian with another (very interesting) approach in the section of Quantum Field Theory.
	\end{tcolorbox}
	
	\pagebreak
	\subsection{Electromagnetic wave equation}\label{electromagnetic wave equation}
	Maxwell supposed that the electromagnetic wave was a combination of phenomena that explain the third and fourth equations. If an electromagnetic wave is far from its source, then we can then neglect the surface density of current from the source as having no influence on the wave (we say then that it is the Maxwell equations without sources which we have already mentioned earlier above). So the third and fourth Maxwell equations are written:
	
	Let us first prove that the magnetic field and electric field are perpendicular!\label{perpendicularity electric magnetic field wave}
	\begin{dem}
	A plane wave can be written:
	
	with constant amplitude vectors $\hat{E}$, $\hat{B}$ and where the plane that is perpendicular to a given vector $\vec{k}$.

	Now let's put the plane wave in the third Maxwell equation:
	
	with the Cartesian coordinate vectors $\vec{x}$, $\vec{y}$, $\vec{z}$.
	
	Here we see that the electric and magnetic components have to be in phase ($\varphi_e=\varphi_m$) and hence we can focus on the amplitudes $\hat{E}$, $\hat{B}$. To check for orthogonality, we evaluate the scalar product of the amplitudes and use the representation of $\hat{B}$ in terms of $\hat{E}$ we just found:
	
	\begin{flushright}
		$\blacksquare$  Q.E.D.
	\end{flushright}
	\end{dem} 
	The excitation magnetic field $\vec{H}$ and electric field  $\vec{E}$ being perpendicular (in the special case of electromagnetic waves in vacuum!), let us place ourselves conveniently in a system of orthonormal and Euclidean axes $(\vec{i},\vec{j},\vec{k})$ belonging to $\mathbb{R}^3$ by choosing:
	
	\begin{tcolorbox}[title=Remark,colframe=black,arc=10pt]
	Warning! The reader must remember that in what follows, $H$ is the $z$ component of $\vec{H}$ and $E$ the $y$ component of $\vec{E}$.
	\end{tcolorbox}
	The (simple) calculations of $\vec{\nabla}\times\vec{E}$ and $¨\vec{\nabla}\times\vec{H}$ gives, after simplification:
	
	Before going further, a reader asked us to develop the details that lead to the left equality. So we start from:
	
	But:
	
	because the wave is a plane wave and the component of the electric field being in $y$, it does not vary following $z$. Then we have:
	
	This being done, if we continue, we have:
	
	Identifying similar terms, we get the "\NewTerm{equation of propagation of the electric field}\index{equation of propagation of the electric field}":
	
	and proceeding exactly in a similar way:
	
	relations that are both in the form of a wave equation (\SeeChapter{see section Wave Mechanics page \pageref{wave equation}}) of the type (for recall) of a "Poisson equation" (specifically it is a "d'Alembert's equation"):
	
	where we have:
	
	The propagation speed of an electromagnetic wave in vacuum is:
	
	the units and numerical values agree...
	\begin{center}
		  \begin{tikzpicture}[scale=2,x={(-10:1cm)},y={(90:1cm)},z={(210:1cm)}]
		    % Axes
		    \draw (-1,0,0) node[above] {$x$} -- (5,0,0);
		    \draw (0,0,0) -- (0,2,0) node[above] {$y$};
		    \draw (0,0,0) -- (0,0,2) node[left] {$z$};
		    % Propagation
		    \draw[->,ultra thick] (5,0,0) -- node[above] {$c$} (6,0,0);
		    % Waves
		    \draw[thick] plot[domain=0:4.5,samples=200] (\x,{cos(deg(pi*\x))},0);
		    \draw[gray,thick] plot[domain=0:4.5,samples=200] (\x,0,{cos(deg(pi*\x))});
		    % Arrows
		    \foreach \x in {0.1,0.3,...,4.4} {
		      \draw[->,help lines] (\x,0,0) -- (\x,{cos(deg(pi*\x))},0);
		      \draw[->,help lines] (\x,0,0) -- (\x,0,{cos(deg(pi*\x))});
		    }
		    % Labels
		    \node[above right] at (0,1,0) {$\vec{E}$};
		    \node[below] at (0,0,1) {$\vec{B}$};
		  \end{tikzpicture}
		
		  \begin{minipage}{.5\linewidth}
		    \[
		      c = \frac{E}{B}
		    \]
		    \begin{tabular}{r@{${}={}$}p{.8\linewidth}}
		      $\vec{E}$ & electric field amplitude \\
		      $\vec{B}$ & magnetic field amplitude (instantaneous values) \\
		      $c$ & speed of light ($3\times10^8\mathrm{m/s}$) \\
		    \end{tabular}
		  \end{minipage}%
		  \begin{minipage}{.5\linewidth}
		    \[
		      c = \frac{1}{\sqrt{\mu_0 \varepsilon_0}}
		    \]
		    \begin{tabular}{r@{${}={}$}p{.8\linewidth}}
		      $\mu_0$ & magnetic permeability in a vacuum, $\mu_0 = 1.3\times10^{-6}\,\mathrm{N/A^2}$ \\
		      $\varepsilon_0$ & electric permeability in a vacuum, $\varepsilon_0 = 8.9\times10^{-12}\,\mathrm{C^2/N m^2}$ \\
		    \end{tabular}
		  \end{minipage}
	\end{center}

	The propagation velocity of the electromagnetic wave in the a material is then given by:
	
	We have $v<c$ because experience has show so far that we can not exceed the speed of light, which is one of the postulates of Special and General Relativity.

	So we can finally write:
	
	Therefore using the d'Alembertian in one dimension:
	
	As we failed to get a direct expression of $E(x, t)$ and $B (x, t)$, we have just obtained differential equations containing only one of these fields. We name these equations respectively "\NewTerm{wave equation for the electric field}\index{wave equation for the electric field}" and "\NewTerm{wave equation for the magnetic field of induction}\index{wave equation for the magnetic field of induction}".

	The both equations have the same form and admit a solution of the same type. An obvious and particular solution (we leave it to the reader to make this check) of these differential equations is the sine trigonometric function:
	
	by not forgetting the relation between the pulsation $\omega$, the propagation velocity $c$ and the wave number $k$ that we had proved in the section of Wave Mechanics!

	A more general solution is the sum of the trivial solutions (\SeeChapter{see section Differential and Integral Calculus page \pageref{sum solutions of differential equations}}):
	
	But we have seen in our study of phasors (\SeeChapter{see section Wave Mechanics page \pageref{phasors}}) that this real solution is only a particular case of a more general solution that is in the imaginary number field. So finally we can write:
	
	Which constitutes the "\NewTerm{monochromatic plane wave}\index{monochromatic plane wave}\label{monochromatic plane wave}" which is the simplest type of wave to manipulate in physics.

	In three dimensions, the solution is by extension:
	
	where $\vec{k}$ is named the "\NewTerm{propagation vector}\index{propagation vector}".
	\begin{tcolorbox}[title=Remark,colframe=black,arc=10pt]
	The monochromatic wave can has no real signification as physical reality. Indeed, if we calculate the electrical energy associated with in all space, we obtain for it an infinite energy (as it has neither beginning nor end!) Which is not realistic. We will see in the section of Wave Quantum Physics that in fact the light is enclosed in a quantum of energy.
	\end{tcolorbox}
	But the wave equation is linear (solution is always the sum of other solutions). This implies that a superposition of waves of different frequencies (wave number and pulsation also!) is also a solution. Thus, by varying the wave vector $\vec{k}$ (and implicitly via its norm $\|\vec{k}\|$, pulsation $\omega$, frequency $f$ and period $T$) we also scan all the possible directions of propagation.

	Written mathematically this gives, for the electric field:
	
	And nothing prevents us from extracting a coefficient of the initial amplitude of the field such that:
	
	and we find here a relation very similar to that of an inverse Fourier transform (\SeeChapter{see section Sequences and Series page \pageref{fourier transform}}) which is a remarkable fact! So the trick is now to put $t=0$ because the previous relation is not just a simple analogy with the Fourier transform, it is a Fourier transform!!!!

	We can therefore relate the real field $\vec{E}(\vec{r},0)$ to the field $\vec{E}_0(\vec{k})$:
	
	These two relations are often condensed as following:
	
	The real field is thus at the initial instant the inverse Fourier transform of the field $\vec{E}_0(\vec{k})$. The term $\vec{E}_0(\vec{k})$ therefore represents the spectral component related to the particular wave vector $\vec{k}$ of the real field. This general solution of the wave equation is named "\NewTerm{wave packet}\index{wave packet}"
	\begin{tcolorbox}[title=Remarks,colframe=black,arc=10pt]
	\textbf{R1.} Identically to our study of Wave Mechanics the coefficients $\omega$ (pulsation) and $\vec{k}$ (wave number) coefficients are required to express the variation of the sine by radians and to give it a direction and a pulsation (\SeeChapter{see section Wave Mechanics page \pageref{wave number}}).\\

	\textbf{R2.} The periodicity of the sine function requires (\SeeChapter{see section Trigonometry page \pageref{trigonometric circle}}):
	
	hence the definition of the period of a wave:
	\\

	\textbf{R3.} The periodicity in space makes it possible to define identically the wavelength of the function as:
	
	We thus observe that the plane wave moves along $x$ by travelling a distance $\lambda$ in a time $T$. The velocity of the electromagnetic wave is then:
	
	\end{tcolorbox}
	By introducing:
	
	Into the one dimensional version of:
	
	we get:
	
	to finally obtain the famous result for the plane electromagnetic wave:
	
	Now let us generalize the above two relations as:
	
	We can see that the Maxwell relation impose other constraints in that case like:
	
	This says that $\vec{E}$ is always perpendicular to $\vec{k}$.
	
	Using $\vec{\nabla} \circ \vec{B}=0$ we get the analogous result for $\vec{B}$, namely:
	
	We also have then:
	
	Therefore we get a quite famous relation used in optics\label{wave equation by wave vector}:
	
	
	\subsubsection{Helmholtz equation}
	Now let us examine in detail another solution of the form:
	
	where this time we make explicit mention of the coordinates in order to avoid any confusion.
	\begin{tcolorbox}[title=Remark,colframe=black,arc=10pt]
	The particular solution with the cosine is more appreciated by many teachers than the one with the sine, as it allows, as we will see, a condensed writing using the phasors (\SeeChapter{see section Wave Mechanics page \pageref{phasors}}).
	\end{tcolorbox}
	If we use the concept of phasor, we can rewrite this solution in the form:
	
	Therefore:
	
	into the wave equation:
	
	we get:
	
	which is nothing else that one the "\NewTerm{one-dimensional Helmholtz equation for electrodynamics}\index{Helmholtz equation for electrodynamics}". It is just a special way to write the conventional Helmholtz that is traditional in this field of physics.

	More generally in undergraduate courses it is defined using the prior previous wave equation and rearranging it as following:
	
	and using the property established earlier above:
	
	But we can write it for the three dimension case also:
	
	And using the laplacian operator (\SeeChapter{see section Vector Calculus page \pageref{laplacian of vector fields}}):
	
	Factorized:
	
	And as it is also valid for the magnetic field we can generalize the notation a last time:
	
	to get the "\NewTerm{homogeneous wave equation}\index{homogeneous wave equation}" (as the second term is zero).

	This done, separation of variables (\SeeChapter{see section Differential and Integral Calculus page \pageref{separation vaiables method}}) begins by assuming that the wave function $f(\vec{r}, t)$ is in fact separable:
	
	Substituting this form into the homogeneous wave equation, and then simplifying, we get immediately the following equation:
	
	Notice the expression on the left-hand side depends only on $\vec{r}$, whereas the right-hand expression depends only on $t$. As a result, this equation is valid in the general case if and only if both sides of the equation are equal to a constant value that we will denote by $C$. From this observation, we obtain two equations, one for $A(\vec{r})$, the other for $T(t)$:
	
	Rearranging the first equation, we get the famous "\NewTerm{general homogeneous Helmholtz equation}:
	
	and this is the common general "\NewTerm{Helmholtz equation}\index{Helmholtz equation}" partial differential equation.
	\begin{tcolorbox}[title=Remark,colframe=black,arc=10pt]
	If in the Helmholtz equation we put $C=0$ we get:
	
	More commonly denoted:
	
	and named the "\NewTerm{Laplace's equation}index{Laplace's equation}". It is therefore a special case of the Poisson's equation but where the second member is equal to zero.
	\end{tcolorbox}
	
	
	\subsubsection{Energy flow transportation (Poynting vector)}\label{poynting vector}
	It is relatively intuitive that any electromagnetic wave must carry energy. Let us express the value of this energy in a classical (non-probabilistic and non-quantum) point of view.

	Since the propagation direction of an electromagnetic wave is that of the vector $\vec{E}\times\vec{B}$ as proven earlier above, we can define the "\NewTerm{Poynting vector}\index{Poynting vector}" as:
	
	named sometimes the "\NewTerm{Abraham form}\index{Abraham form}" and whose value is expressed indeed in joules per second and per unit area (radiated power by surface unit) as $[\text{J}\cdot\text{s}^{-1}\cdot\text{m}^{-2}]$.

	The norm of the Poynting vector represents therefore the instantaneous power which is transported by the electromagnetic wave through a unitary perpendicular surface (we insist on the "perpendicular") to its direction of propagation. Therefore, we can also write the Poynting vector in the following form (be careful not to confuse the energy and the electric field which are represented by the same letter):
	
	where $\vec{n}$ is as usual the unit vector perpendicular to $\mathrm{d}S$ (this last relation will be useful to study a small property of the synchrotron radiation further below).

	For a plane electromagnetic wave, the norm of the Poynting vector is:
	
	This quantities varies according to time and place. At a given location, its average value is the mean value of the $\sin^2(\ldots)$ for a period $T$:
	
	Let us recall that (\SeeChapter{see section Differential and Integral Calculus page \pageref{usual primitives}}):
	
	Therefore $\forall x$ we have:
	
	The average value of the Poynting vector of a plane electromagnetic wave is a constant ... which does not depend on position and time.
	
	\begin{tcolorbox}[title=Remark,colframe=black,arc=10pt]
	We can make a daring and fun analogy with electrokinetics by doing a dimensional analysis of the above product $\mu_0c$. We have:
	
	\end{tcolorbox}
	... to demonstrate the energy contained in a unit of volume pragmatic physicists would do a dimensional analysis. Let us avoid this and focus ourselves with the special case of the plane wave!

	For this purpose we base ourselves on the electrical energy of an ideal plane capacitor (\SeeChapter{see section Electrostatics page \pageref{capacitor}}) producing plane electromagnetic waves with a yield of $100\%$:
	
	Therefore:
	
	and that is denoted in most textbooks by:
	
	from which we get:
	
	And the total energy transported by the electromagnetic wave in this particular case is thus:
	
	Therefore, the electrical energy density of an electromagnetic wave is equal to its magnetic energy density. For the power we just multiply by $c$ to get:
	
	\begin{tcolorbox}[title=Remark,colframe=black,arc=10pt]
	Notice that the intensity of an electromagnetic wave is then only related to its amplitude and not its frequency!!!! So if a light source is kept at the same amplitude and the wave length is reduced, its intensity remain the same even though its photons now carry more energy!
	\end{tcolorbox}
	From this result, we are led to define the "\NewTerm{mean intensity $I$ of an electromagnetic wave}" by the average value of its Pointing vector:
	
	It is therefore the average power that carries the wave per unit area. We have already demonstrated the mean expression of the Poynting vector, which leads us to write:
	
	Now, using the relation between energy and linear momentum (\SeeChapter{see section Special Relativity page \pageref{relativistic mass momentum relation}}):
	
	we get the "\NewTerm{linear momentum density}\index{linear moment density}" of the electromagnetic wave:
	
	But if the direction of $\vec{E}\times\vec{B} $ is perpendicular to the wavefront and is therefore confounded with the propagation direction of the wave its module is:
	
	We therefore have for the linear momentum density:
	
	As the linear momentum density must have the direction of propagation, we can write it in vector form:
	
	If an electromagnetic wave has a linear momentum density, it also has a kinetic angular momentum. The angular momentum per unit volume is then:
	
	Thus, an electromagnetic wave carries both linear momentum density and angular momentum density as well as energy density !!!

	This result is not surprising. Electromagnetic interaction between two electrical charges involves an exchange of energy and momentum between loads. This is done through the electromagnetic field which carries a density of energy and momentum density that are exchanged.
		
	\pagebreak
	\subsubsection{Emissions}\label{electromagnetic emissions}
	To predict the shape and properties of the radiation emitted by antennas or other sources, computers and numerical models corresponding to the problem to be studied should be rigorously used. Formally, solving Maxwell's equations in macroscopic systems is quite difficult and takes time. Moreover, this is rather the work of the engineer who seeks practical exploitation from fundamental theories. The theoretical physicist is interested in the foundations of the Universe and with isolated and perfect systems.

	However, we would like to expose the theory of diffraction and for this we must make a theoretical trip on an approximation of the properties of the radiation of a spherical point source in the vacuum.

	The wave in the case of a spherical point source propagates spherically in space (we speak then of "\NewTerm{spherical wave}\index{spherical wave}\label{spherical wave}") and the Poynting vector of is therefore obviously radial to the source.
	\begin{figure}[H]
		\centering
		\includegraphics[scale=0.47]{img/electromagnetism/spherical_wave_front.jpg}
		\caption[Spherical wavefront]{Spherical wavefront (source: ?)}
	\end{figure}
	The vector $\vec{E}(r,t)$ and $\vec{B}(r,t)$ and vectors are locally contained in the plane tangent to the sphere of radius $r$ (logical!) as shown in the figure below:
	\begin{figure}[H]
		\centering
		\includegraphics{img/electromagnetism/spherical_wave_poynting_vector.jpg}
		\caption[]{Representation of the propagation with respect to the plane tangent to the sphere}
	\end{figure}
	For the flow energy to be constant, the intensity of the wave must decrease with distance. Indeed, the conservation of the energy imposes that through a sphere of radius $r_1$ the energy radiated per unit of time (written with an right "E" so as not to be confused with the notation of the electric field) is equal to that which traverses the sphere of radius $r_2$:
	
	This naturally implies:
	
	But using the relation proved earlier above:
	
	and using the property of perpendicularity of the electric and magnetic field for a plane wave:
	
	Therefore:
	
	which implies:
	
	that is to say:
	
	and therefore the magnitude of the electric or magnetic of a electromagnetic plane wave field decrease inversely proportional to the distance $r$:
	
	Thus the intensity density $I$ of a spherical electromagnetic wave propagating in the vacuum decreases in $r^2$ since:
	
	Therefore:
	
	By extension (important information for mobile phones and radio communications), in view of the results demonstrated above, the energy transported thus decreases in also in $1/r^2$.
	
	It is now easy to understand now why physicists systematically use the frequency to characterize a electromagnetic wave because its amplitude is not constant in the vacuum whereas the frequency is a kind of signature of the transmitter that is not lost through a static empty space !!!
	
	\pagebreak
	\subsection{Synchrotron radiation (bremsstrahlung)}\label{bremsstrahlung}
	Broadly speaking, "\NewTerm{Bremsstrahlung}\index{Bremsstrahlung}" or "\NewTerm{braking radiation}\index{braking radiation}" is any radiation produced due to the deceleration (negative acceleration) of a charged particle, which includes synchrotron radiation, cyclotron radiation, and the emission of electrons and positrons during beta decay. However, the term is frequently used in the more narrow sense of radiation from electrons (from whatever source) slowing in matter. 
	
	\begin{tcolorbox}[title=Remark,colframe=black,arc=10pt]
	"\NewTerm{Thermal bremsstrahlung}" is an X-ray emission mechanism that typically takes place in a high temperature, low-density plasma. As a free electron passes close to an ion in the gas it is deflected without being captured. As a result of this acceleration the electron emits a photon while at the same time losing a corresponding amount of kinetic energy and slowing down a little.
	\end{tcolorbox}
	
	A synchrotron light source is a source of electromagnetic radiation (EM) usually produced by a storage ring or more generally any electric charge particle in acceleration. First observed in synchrotrons, synchrotron light is now produced by storage rings and other specialized particle accelerators, typically accelerating electrons. Once the high-energy electron beam has been generated, it is directed into auxiliary components such as bending magnets and insertion devices (ondulators or wigglers) in storage rings and free electron LASERs. These supply the strong magnetic fields perpendicular to the beam which are needed to convert high energy electrons into photons.
	\begin{figure}[H]
		\centering
		\includegraphics[scale=0.25]{img/electromagnetism/synchrotron.jpg}
		\caption[European Synchrotron]{European Synchrotron (source: ESRF)}
	\end{figure}
	The major applications of synchrotron light are in condensed matter physics, materials science, biology and medicine. A large fraction of experiments using synchrotron light involve probing the structure of matter from the sub-nanometer level of electronic structure to the micrometer and millimetre level important in medical imaging. An example of a practical industrial application is the manufacturing of micro-structures by the LIGA process.
	\begin{figure}[H]
		\centering
		\includegraphics[scale=1]{img/electromagnetism/synchrotron_source.jpg}
		\caption[Synchrotron source produces electromagnetic radiation]{Synchrotron source produces electromagnetic radiation, as evident from the visible glow (source: United States Department of Energy)}
	\end{figure}
	It is also an important subject of study because the Rutherford  Model (\SeeChapter{see section Corpuscular Quantum Physics page \pageref{rhuterford model}}) was not able to explain the fact that there was not Bremsstrahlung of the electron orbiting around the nucleus as at it's time the Bremsstrahlung was a well known effect!
	
	To begin, let us consider an electric charge in uniform rectilinear motion. The electrical and magnetic fields of such an electric charge have been studied in the previous sections. We have also shown above that the magnetic field is in this configuration, always perpendicular to the electric field. The first consequence is that the electric field is radial and the magnetic field transverse.

	So if we surround the moving particle of an imaginary closed spherical surface, we then have trivially (see the definition of the Poynting vector):
	
	since at any point on the surface, $\vec{E}$ is perpendicular to it and $\vec{B}$ tangent to it, hence $\vec{E}\times\vec{B}$ is also tangent to the surface and therefore the angle between $\vec{E}\times\vec{B}$ and $\vec{n}$ is equal to a right angle therefore the scalar product is zero.

	Thus, in conclusion, the total flux of radiated energy is zero for an electric charge in uniform rectilinear motion. In other words, a uniformly rectilinear moving electric charge does not radiate electromagnetic energy but carries with it the energy of the electromagnetic field (here we are reassured!). This is confirmed by the experimental observations.

	However, the situation is very different for an accelerated moving electric charge. The electric field of an accelerated charge is no longer radial and no longer possesses the symmetry with respect to the charge which it possesses when the motion is uniform (as will we prove it). Consequence: ... an accelerated electric charge radiates electromagnetic energy and therefore sees its kinetic energy that decrease!

	An important conclusion is that in order to maintain an electric charge in accelerated motion, it is necessary to provide energy to compensate for that lost by radiation. If the particle instead of being accelerated is decelerated (it is typically what we seek to do in radio-protection) again the particle will emit the same radiation in the same way (we will also prove it). This happens, for example, when an electric charge, such as an electron or a proton, strikes a target at high speed. A substantial fraction of its total energy goes in the form of a radiation.

	The equations that we are going to determine remain valid for any type of relativistic accelerated motion or not (but not from a quantum point of view!). For example, a charged particle moving in a circular orbit is subjected to centripetal acceleration and therefore emits radiation. Consequently, when an ion is accelerated in a cyclic accelerator such as a cyclotron, betatron or synchrotron, a fraction of the energy supplied to it is lost as electromagnetic radiation. Cyclic accelerators than in linear accelerators.
	
	When the electric charges reach very high energies, as it happens in synchrotrons where the acceleration is great (fortunately for us because it will allow us to make a very useful approximation ...), the losses due to radiation, named "synchrotron radiation" as we already know, become important and constitute a serious limitation in the construction of cyclic accelerators of very high energy but nevertheless remain infinitely useful to the advanced industry and scientific medicine and archaeology.

	Another important consideration relates to the atomic structure. According to the atomic model of Rutherford (\SeeChapter{see section Corpuscular Quantum Physics page \pageref{rhuterford model}}), we imagine the atom as formed of a positively charged central nucleus, the negatively charged electrons describing closed orbits around it. But this implies that the electrons move with accelerated motion and, if we apply the ideas developed so far, all atoms should continuously radiate energy (even in the absence of an external source of energy like the Sun). As a result of this loss of energy, electronic orbits should contract, resulting in a corresponding reduction in the size of all bodies. Fortunately for us, this is not observable (matter does not collapse on itself), but this leads us to suppose, within the framework of the Rutherford model, that the movements of electrons in atoms are governed by certain principles which were not considered at its time. This is what will lead us to create the Bohr model of the atom (\SeeChapter{see section Corpuscular Quantum Physics page \pageref{bohr model}}) but which, as we shall see, will also have other defects.

	To determine the energy emitted by an electric charge in accelerated motion we will have to make use of mathematical tools that are no longer at the same level as those used previously. It is therefore recommended that the reader have a good mathematical background. Moreover, exceptionally we will make use of CAD softwares for certain points of the development.

	Let us consider first the following figure:
	\begin{figure}[H]
		\centering
		\includegraphics[scale=1]{img/electromagnetism/synchrotron_study_configuration.jpg}
		\caption[]{Scenario to consider for the study of synchrotron radiation}
	\end{figure}
	When the charge distribution $\rho(\vec{r},t')$ and the current distribution $\vec{J}(\vec{r},t')$ are at the point $P_2$, the point $M$ receives the electromagnetic wave emitted by the electric charges and the current when they were at the point $P_1$, ie at the instant $t'$ (because of the speed limit $c$ of the propagation of the field in space). The time delay is the propagation time from the point $P_1$ to the point $M$, ie:
	
	Therefore:
	
	Therefore:
		
	The scalar and vectorial potentials associated respectively with the electric field and the magnetic field at the point of vectorial coordinate $\vec{R}$ at time $t$ have, on the basis of the results obtained in the two preceding sections, the following expressions:
	
	where on the other hand we have to prove immediately in detail that the vector potential associated with the magnetic field is expressed as indicated above!
	\begin{tcolorbox}[title=Remark,colframe=black,arc=10pt]
	We will use these two relations of potential in our study of the radiated field because their similar mathematical form will allow us, at least we hope it..., to simplify the developments.
	\end{tcolorbox}
	These two relations are already partially familiar to us, the first one expressing the (delayed) electrical potential has been proved in the section of Electrostatics in the non-relativistic framework (so our calculations may not be correct if we come across a result that depends on the speed! We will see...).

	Concerning the second relation which expresses the delayed potential-vector, we have seen above that:
	
	was always correct to given gradient of an additive function for $\vec{A}$ (due to the properties of the differential vector operators) such that:
	
	and that $\vec{B}$ is in relativistic form or not, we have:
	
	Let us also recall that we have the Biot-Savart law (\SeeChapter{see section Magnetostatics page \pageref{biot savart law}}) that:
	
	It follows that if we put:
	
	we fall back on the Biot-Savart law since if and only if $\vec{J}$ does not depend on $r$ then (trivial):
	
	We thus get indeed:
	
	Although this form of vector potential gives only the Biot-Savart's law in non-relativistic form, as still always satisfies:
	
	this is still valid in the relativistic framework because this Maxwell equation does not depend on the velocity. Moreover, if our results in the study of synchrotron radiation give us at the end an expression independent of velocity, we shall have once again confirmed this fact.

	\subsubsection{Liénard-Wiechert potentials}\label{linear wiechard potentials}
	Liénard–Wiechert potentials describe the classical electromagnetic effect of a moving electric point charge in terms of a vector potential and a scalar potential in the Lorenz gauge. Built directly from Maxwell's equations, these potentials describe the complete, relativistically correct, time-varying electromagnetic field for a point charge in arbitrary motion, but are not corrected for quantum-mechanical effects. Electromagnetic radiation in the form of waves can be obtained from these potentials. These expressions were developed in part by Alfred-Marie Liénard in 1898 and independently by Emil Wiechert in 1900.
	
	\begin{tcolorbox}[colback=red!5,borderline={1mm}{2mm}{red!5},arc=0mm,boxrule=0pt]
	\bcbombe Caution! The only purpose of the twenty pages of hard and abstract mathematics that follow have for only purpose in this book to derive the Larmor's relativistic equation with for small speeds ($v\ll c$) lead to the Larmor's (classical) equation. And it is that later that classical version that we need for many other section of this book. However there is also a much simpler derivation that leads directly to the classical version of Larmor's equation. If the reader wants he can directly jump to that latter by going at page \pageref{Larmor equation classical derivation}. 
	\end{tcolorbox}
	
	To start this study, let us consider the case where a particle of mass $m$ and of electric charge $q$ travel a path $\Gamma$. Compared to an origin point O, its vector coordinate is $\overrightarrow{\text{O}P}=\vec{h}(t)$, its velocity vector will be denoted:
	
	and its acceleration:
	
	If the point charge $q$ is at the origin O, we saw in the section of Differential and Integral Calculus that the Dirac function gives us:
	
	and if the punctual electric charge $q$ is at an abscissa $x_0$, we have:
	
	What has just been said for a one-dimensional space can also be applied to a three-dimensional space as we saw it and then we write:
	
	If we choose inverse cubic meters ($[\text{m}^{-3}]$) for units for the Dirac function, then we can write:
	
	where $q$ is then the total charge at the point $\vec{r}$.

	For the distribution of the current density, we also have by choosing the same units as for the Dirac function:
	
	Therefore, at point $M$, the potentials at time $t$ have for expression:
	
	This is a very useful formulation (a workaround) that will enable us to solve our problem.

	For this purpose, when the electric charge is at the point $P_1$ at the time $t'$, we put:
	
	We will use a looooong artifice to solve the integral of the electrical potential (which is therefore a multiple integral in Cartesian coordinates)!

	This begins by multiplying the factor under the integrand $U(\vec{R},t)$ by:
	
	this does not modify the integral since:
	
	and that (\SeeChapter{see section Functional Analysis page \pageref{dirac function}}):
	
	We then have the following expression in which the time $t'$ appears:
	
	what we have the right to write because the second integral does not depend explicitly on $t'$.
	
	Good now if we try to solve this integral, we will spend our lives on it... for nothing. We'll have to be more clever!

	Before looking for a solution of this integral, we must first deal with the more general case of the following integral:
	
	Or written in a more condensed form:
	
	which it is easy to approximate with the prior-previous integral:
	
	where we have deal such that $f_1,f_2,f_3,f_4$ depend respectively only (explicitly) on $x$, $y$, $z$ and $t'$.

	We now want now to make the following change of variables:
	
	Let us recall that in changes of variables in multiple integrals (see the Jacobian in the section of Differential and Integral Calculus), we have, passing from Cartesian coordinates to curvilinear coordinates, the following relations:
	
	where for recall:
	
	and where:
	
	is not an absolute value but the determinant of a matrix!
	
	Now, in our case, let us recall that we have all the $f_i$ which are null and therefore:
	
	and in the case where during the developments one of the $f_i$ would no longer be zero for reasons not yet determined, we would have:
	
	The multiple integral then becomes:
	
	Where the term between braces is taken at $f_i=0$ by necessity of the construction of previous developments preparing the mathematical artifice!

	And let us recall once more (!!) the property of Dirac functions:
	
 	We then have immediately the simplification:
	
	where:
	
	is therefore the Jacobian of the transformation of the artifice ...

	It is evident that by construction of the Jacobian we have:
	
	Therefore it comes:
	
	For the integral $I$ we then have:
	
	Let us now calculate our Jacobian ...:
	
	Returning to the our treated case, $\vec{R}-\vec{h}(t')$ therefore has for components:
	
	Therefore, we have the calculation of the elements of the inverse of the Jacobian:
	
	Well... now that we have the components of the Jacobian matrix, we have only to calculate its determinant. So either we use the general relation of the calculus of determinant proved in the section of Linear Algebra, or we use Maple ... So to win a little bit time let's do it with Maple 4.00b:
	
	\texttt{>with(linalg):\\
	>A:=matrix(4,4,[1,0,0,a,0,1,0,b,0,0,1,c,d,e,f,1]);}
	
	where (for people having difficulty to read Maple notations):
	
	with:
	
	Let us continue with Maple 4.00b:

	\texttt{>Det(A);}
	
	Which gives as output:

	\begin{center}
		\texttt{1-cf-eb-da=1-(fc+eb+da)}
	\end{center}
	The inverse of the Jacobian has then for expression:
	
	Where we used the dot product in the prior-previous equality in order to condense the expression.

	Therefore:
	
	The multiple integral:
	
	where for recall:
	
	or written differently:
	
	But as a result of our change of coordinate system we have for recall:	
	
	And let us recall once again that:
		
	Therefore we have to take $g$ on $f_1=f_2=f_3=f_4=0$! It comes:
	
	Which allows us to write:
	
	We have to apply the same process for:
	
	that can then be written as:
	
	Finally, the resolution of the integral $I$ is written:
	
	Finally, we get the expressions of the potentials.
	\begin{itemize}
		\item The scalar potential is then written:
		

		\item The vector potential is written:
		
		Considering that the integral which is almost the same as for the scalar potential except the term $\vec{v}'$, we arrive by making the same developments as before with to the expression (please... don't ask us for to put also the details for this one... plzzzzz!):
	
	\end{itemize}
	In summary, the potentials taken at the moment (time delay of propagation):
	
	have for expression in a very condensed form ($K_e$ in the Coulomb constant for recall as introduced in the section of Electrostatics!):
	
	These potentials are named "\NewTerm{Liénard-Wiechert retarded potentials}\index{Liénard-Wiechert retarded potentials}" with:
	
	
	\subsubsection{Retarded Electric and Magnetic fields}
	Let us recall that we have proved earlier above (see page \pageref{electric field with potential vector}) that:
	
	and by definition and the construction of potential vector we also have:
	
	Therefore a difficulty appears in the sense that the Liénard-Wiechert potentials are at the moment $t'=t-\frac{\|\vec{R}-\vec{r}\|}{c}$ and that the fields are given at the moment $t$. It is therefore advisable to review the expressions of the fields corresponding to the instant $t'$. 
	
	By taking back the two relations:
	
	with for recall:
	
	The notation in the last line may be confusing, but it's just a change of notation.
	
	Let us now determine the retarded electric field. For this, let us first rearrange $t'$ given above:
	
	As $z=z(t')$ and $t=t(t')$ (in other words, $z$ is function of $t'$ and $t$ is function of $t'$), we have the following partial derivative:
	
	We also have $z^2=\|\vec{z}\|^2$. Deriving relatively to $t'$:
	
	By definition we put (!): 
		
	By isolating the derivative with respect to $t'$ in $z\partial_{t'}z=\vec{z}\circ\partial_{t'}\vec{z}$ and taking into account from $\partial_{t'}\vec{z}=-\vec{v}$, we get:
	
	If we derive $z=c(t-t')$ relatively to $t$:
	
	and equating:
	
	We get:
	
	Isolating $\partial_tt'$:
	
	We multiply both side by $\partial_{t'}$:
	
	At constant $t$ we apply the nabla operator (ie gradient... and for reminder that gives then a vector!) to $z$:
	
	But we also have (denoting by $\vec{\nabla}^\prime()$ the nabla operator only on the spatial components and remembering as seen above that $\partial_t z=\partial_{t'}z\partial_tt'$):
	
	where the last equality holds because anyway $t'$ depends only on $t$.
	
	Therefore we can write:
	
	But from:
	
	We can then write:
	
	\begin{tcolorbox}[title=Remark,colframe=black,arc=10pt]
	Now let us denote $u=\|\vec{R}-\vec{r}(t')\|$ instead of $z$ and imagine we explicit that relation in the following way:
	
	Now let us recall that if $f$ and $g$ are both differentiable and $F(x)$ is the composite function defined by $F(x)=u(k(x))$ then $F$ is differentiable and $F'$ is given by  the product chain rule:
	
	Then:
	
	So going back to the notation we have in this section:
	
	\end{tcolorbox}
	Taking into account what we just detailed in the remark above. We can then also write:
	
	Rearranging the following par:
	
	we get:
	
	Therefore the relation we have derived earlier:
	
	can be written:
	
	If we get rid of the $z$ this gives the following useful relation for further calculations (nabla operator expressed form a retarded nabla operator):
	
	Now remember that we have:
	
	Therefore:
	
	Rearranging (and remembering that $\varepsilon\mu_0c^2=1$) gives:
	
	Now we can use the result we get before about the nabla operator because we will compute:
	
	and we can also use the relation derived earlier above:
	
	To calculate:
	
	Therefore:
	
	Now let us open a parenthesis! Remember that by definition we have put earlier above $\partial_{t'} \vec{z}=-\vec{v}$ and that we have just seen that:
	
	So now let us consider the following rectilinear case:
	
	Then:
	
	Therefore:
	
	But is:
	
	valid for any type of movement? The answer is no! Indeed, consider the famous example of the circular motion in a plane (typical of many particles accelerators):
	
	Then obviously:
	
	And therefore:
	
	So that means we can use:
	
	only as a locally linear approximation of the particle path when the movement is collinear to the speed! Therefore all relations that will follow are approximations!
	
	So according to what we have just say above, we can write locally in the particle path that:
	
	Injecting that result in the earlier relation:
	
	we get:
	
	But we also have:
	
	Injecting the result in the previous relation a remembering that we have proved earlier that:
	
	we get:
	
	Rearranging a bit:
	
	Let us now focus on that term:
	
	Rearranging we get:
	
	Working on the first term we have:
	
	Injected into the previous relation and simplifying we get:
	
	The term between square bracket is a triple cross product and we have already proved in the section of Vector Calculus that this can be written (see page \pageref{grassman rule}):
	
	Finally, we find for the retarded electric field, the following expression (we put now the approximate equal sign because we know that some assumptions made earlier make is an approximation):
	
	Now let us do the same work with the magnetic retarded field. We start from:
	
	Therefore:
	
	Rearranging and using $\mu_0\varepsilon_0c^2=1$ we get:
	
	We have proved in the section of Vector Calculus that (see page \pageref{curl quotient chain rule}):
	
	Therefore:
	
	For the next step Larmor used a trick (still debated nowadays) to simplify the expression to be able to fall back later on a simplified expression of radiated electromagnetic field that was known before his theory. The trick (mathematically quite wrong but physically it works...) is to expression the nabla operator on the $\vec{\nabla}\times\vec{v}$ by it's retarded version derived earlier above for the position (...) and given for reminder by:
	
	If we do that we get for the first parenthesis or the prior relation:
	
	After without mathematical reasons or also without physical interpretation, but very likely to fall back on a known result as mentioned above, Larmor put:
	
	claiming (we don't how this claim makes any sense...) that's because $\vec{v}=f(t')$....???
	
	We may think that Larmor did that because if $\vec{v}$ is a vector field, then $\vec{\nabla}^\prime\times\vec{v}=\vec{0}$ assume that the velocity field is irrotational and therefore the velocity field is conservative (such fields have for reminder the property that the line integral around any closed loop, often representing the work done in moving a particle, is zero). But if $\vec{v}$ is a conservative field, in our point of view, then we should also have $\vec{\nabla}\times\vec{v}=\vec{0}$... and then nothing makes sense anymore. That's why it is very likely to be a physicist trick just because we have to it to fall back later on a result that was already known at his time.
	
	Then obviously we get for the first parenthesis:
	
	Now coming back to:
	
	Now let us focus on the second term that we can immediately write as (using again the relation between the nabla operator and retarded nabla operator):
	
	And using again the relation derived earlier above that was:
	
	We get:
	
	But we have proved earlier above that:
	
	Therefore:
	
	Hence:
	
	Let us rearrange and simplify the right hand side:
	
	Therefore:
	
	We now propose to investigate the possibility of bringing together the last two terms inside the square brackets in the form of cross-products. That means:
	
	By inverting the factors of the first cross product and simplifying by reducing to the same denominator in the second term we get:
	
	We multiply this expression by $z/z$ and factorize $1/(zc)$:
	
	We introduce the term $\vec{z}\times\vec{z}$ (that is obviously equal to $\vec{0}$):
	
	But the term between curl braces:
	
	can be recognized as being a triple cross-product (\SeeChapter{see section Vector Calculus page \pageref{grassman rule}}). Indeed, if we put:
	
	We know that then we have the following relation:
	
	and it's corresponding:
	
	Therefore:
	
	After rearranging:
	
	In summary we find for the delayed fields, the following expressions:
	
	Hence explicitly:
	
	Let us analyse these expressions of retarded electric fields and magnetic induction. For this analysis, we will modify somewhat the two expressions above.
	
	For this purpose we put:
	
	and we will denote by $\vec{u}_z$ the unitary vector collinear to $\vec{z}$.
	
	Therefore:
	
	Simplifying a bit more:
	
	So we observe that the two retarded fields $\vec{E}$ and $\vec{B}$ are composed of two terms $\vec{E}_1+\vec{E}_2$ and $\vec{B}_1+\vec{B}_2$.
	\begin{itemize}
		\item The first terms $\vec{E}_1$ and $\vec{B}_1$ are:
		
		They correspond to a charge animated by a speed $\vec{v}$ and whose fields are functions of the speed of displacement of the so said charge and of the inverse of the square of the distance $z$.
		
		These fields decrease rapidly with the distance $z$.
		
		We notice that in the non-relativistic case $\beta \ll 1$, we fall back on the classical expression of the electric field and of magnetic induction:
		
		Likewise, in the particular case where the speed is zero, only the electrostatic field remains as expected!
		
		\item The second terms $\vec{E}_2$ and $\vec{B}_2$ are:
		
		They correspond to charges animated by an accelerated movement $\dot{\vec{\beta}}=\dot{\vec{v}}/c$  but which have a decreasing amplitude with $z$:
		
		These are radiating fields known as the "electromagnetic wave" (E.M).
		
		These waves propagate from the charge in the form of a spherical wave but which at a great distance can be locally considered as plane.
		
		In the non-relativistic $\beta\ll 1$ case we get:
		
		We can also write the fields as following (for the last equality see the explanations just afterwards!):
		
		Indeed, let us analyse how we can rewrite:
		
		With a small change of variable we recognize again the triple cross-product (\SeeChapter{see section Vector Calculus page \pageref{grassman rule}}):
		
		Therefore:
		
		It is immediate that $\vec{u}_z\circ \big(\vec{u}_z\times\dot{\vec{\beta}}\big)=0$, then it remains:
		
		Therefore:
		
	\end{itemize}
	In summary, keeping only the radiated part of the fields (!), we have:
	\begin{itemize}
		\item For the relativistic case:
		
		
		\item And for the non-relativistic case:
		
	\end{itemize}
	From now on, as the reader will have very likely noticed it, we will only work with the radiated fields and we will remove the index $2$ from $\vec{E}_2$ and $\vec{B}_2$.
	
	\subsubsection{Power radiated through an elementary surface}
	Let us seek a relation for the power radiated through an elementary surface as function of $\vec{B}$.
	
	Let us recall that we have(see page \pageref{poynting vector}):
	
	And the relation above for the relativistic case: 
	
	We want to extract $\vec{E}$ as function of $\vec{B}$ from that latter relation to put it in the expression of the power.
	
	For that purpose we use the following trick by calculating:
	
	We recognize here again the triple cross-product (\SeeChapter{see section Vector Calculus page \pageref{grassman rule}}). Therefore if we put:
	
	Therefore:
	
	For an electromagnetic wave far enough from its emission source, we have a plane wave as a first approximation and therefore: $\vec{E}\perp\vec{B}$ and $\{\vec{E},\vec{B}\}\perp\vec{z}$ which leads obviously to $\vec{u}_z\circ\vec{E}=0$, hence:
	
	Therefore under these conditions:
	
	So:
	
	Injecting this into:
	
	we get:
	
	We recognize here again the triple cross-product (\SeeChapter{see section Vector Calculus page \pageref{grassman rule}}). So let us put
	
	Therefore:
	
	Hence:
	
	But we know that for the non-relativistic case:
	
	Injecting it in the relation above we get:
	
	Assuming that the following fraction can be interpreted as the solid angle that is the cone of vision of the surface $\mathrm{d}\vec{S}$ whose vertex coincides with the center of the charge and its axis coincides with the direction of $\vec{u}_z$:
	
	So finally we get:
	
	where for recall, $\vec{a}$ is the acceleration of the electric charge $t'=t-z/c$.
	
	So what we can say so far is that:
	\begin{itemize}
		\item The radiated power per steradian emitted by the accelerated charge is proportional to the square of its acceleration and to the square of its electric charge
	
		\item It is zero in the direction of the acceleration and maximum in the direction
	perpendicular $\theta=\pm \pi/2$.
	
		\item Importantly, it does not depend on the measuring distance (this is an approximation obviously!). This allows to detect it at a great distance.
	\end{itemize}
	We will now calculate the total power radiated by the accelerated electric charge on a sphere having for center the point O (the electric charge itself!) and for radius $z$ (the distance between the electric charge and the observation point). In this case, it is useful to work with spherical coordinates
	
	In spherical coordinates, a surface element $\mathrm{d}\vec{S}$ traced on the surface of the sphere of origin O and of radius $z$ has the expression (\SeeChapter{see section Geometric Shapes page \pageref{infinitesimal element of a surface of a sphere}}):
	
	Injecting it in:
	
	we get (see the usual primitives in the section of Differential and Integral Calculus):
	
	Therefore the expression of the total power radiated by an accelerated charge in the non-relativistic case is given by:
	
	And as we can notice it, the distance $z$ has vanished as we would expect it!
	
	The total power emitted in the form of radiation from a charge moving in an accelerated motion is proportional to the square of the acceleration $a$ and the square of its charge $q$. It is independent of the distance $z$.	
	
	Now we want to calculate the power emitted by the electric charge in the relativistic case!
	
	In the application of special relativity, the following two elements will be taken into account:
	\begin{itemize}
		\item there are invariants in all inertial frames of reference (here: $q$, $\varepsilon_0$, $c$).

		\item in the relativistic expression of power, we must be able to fall back on the previous relation  if we consider the case of low speeds compared to the speed of light!
	\end{itemize}
	If we multiply the norm of the four-vector acceleration derived in the section of Special Relativity (see page \pageref{norm of relativistic acceleration}):
	
	by the mass of the electric charge, we have then what is conventionally named the "\NewTerm{Abraham-Becker radiation damping force}\index{Abraham-Becker radiation damping force}\label{Abraham-Becker radiation damping force}.
	
	The reader must remember that this result is performed in an instantaneous rest frame where the charged particle velocity and acceleration are collinear, and is independent of the internal structure of the particle and shows that the radiation damping force is essentially a relativistic effect.
	
	That is the "\NewTerm{Liénard's generalization}\index{Liénard's generalization}\label{Liénard's generalization}" to the Larmor's equation\footnote{It was obtained by A. Liénard in 1898, only one year after the publication of Larmor’s result, and seven years prior to the invention of special relativity!} we will just see afterwards:
	
	We see obviously that for small speed $v\ll c$, it leads to:
	
	And this is the result named the "\NewTerm{Larmor's equation}\index{Larmor's equation}\label{Larmor equation}". It implies that any charged particle radiates when accelerated and that the total radiated power is proportional to the square of the acceleration. The largest astrophysical accelerations are usually produced by electromagnetic forces, so the acceleration is proportional to the charge/mass ratio of the particle. 
	
	This result is also an interesting and very difficult paradox that a number of physicists have addressed over the last century, including Pauli. If the charge radiates under uniform acceleration, it will violate the principle of equivalence. The famous Richard Feynman was one of those rare physics celebrities who did insist, based on his line of argumentation, that this is false paradigm that an uniformly accelerated charge actually radiates, and that actually it does not, and only oscillating and other non-uniform accelerations can lead to radiation (people interested in a detailed analysis of this point of view can read \cite{singal2018no} and \cite{singal2020discrepancy}). But with this remarkable exception, the overwhelming consensus of the physicists is that a uniformly accelerated charge does radiate! There is a large amount of literature and experience (worldwide) since many decades on synchrotron radiation (leptons but also hadrons for very high energies) and no further fundamental discussion required. Also keep in mind that accelerated charged particles in LINACs radiate a lot.
	
	\begin{tcolorbox}[title=Remark,colframe=black,arc=10pt]
	We will also see during our study or Special Relativity that the covariance of Maxwell's equations under general coordinate transformations, that, despite inertial observers can indeed detect electromagnetic radiation emitted from an accelerated charge, comoving observers will see only a static electric field. Therefore when the charge radiates it doesn't lost momentum!\\
	
	Also the ready should keep in mind that a charge moving along a geodesic doesn't accelerate by definition (!), then it doesn't emit any electromagnetic radiation. Furthermore let us remind that we will prove during our study of General Relativity that gravitation isn't a force!
	\end{tcolorbox}

	Larmor's extremely useful equation will be the basis for our derivations of radiation from a short dipole antenna as well as for synchrotron emission from astrophysical sources. Beware that Larmor's equation is non-relativistic; it is valid only in frames moving at velocities $v\ll c$ with respect to the radiating particle. Also the reader should keep in mind that an electric charge in free fall in a uniform and homogeneous gravitational field does not radiate because it is following a geodesic, i.e. it is moving by inertia!
	
	Also, Larmor's equation does not incorporate the constraints of quantum mechanics, so it should be applied only with great caution to microscopic systems such as atoms. For example, Larmor's equation incorrectly predicts that the electron orbiting in the lowest energy level of a hydrogen atom will quickly radiate away all of its kinetic energy and fall into the nucleus. On the other hand, it correctly predicts the radio power emitted by an electron orbiting in a very high energy level of a hydrogen atom.
	
	Anyway, that's a lot of words and not much to look at, so how about some soothing animations? 
	
	Below is a Flash animation illustrating the effect, made with MATLAB™, that will play in a PDF Reader having Adobe Flash installed and activated (otherwise see here: \url{https://vimeo.com/575746945}):
	\begin{figure}[H]
		\includemedia[activate=pageopen,width=\textwidth,height=500pt,
	]{}{swf/Synchrotron_radiation_gamma1.swf}
		\caption[Synchrotron radiations of a moving charge]{Synchrotron radiation of a circle moving particle with $\gamma=1.2$ (author: Jason Cole)}
	\end{figure}
	The top left represents the Poynting flux $S$, or the radiation which would be observed. The retarded time is plotted beneath, and the two components of the electric field are plotted on the right. You can see how the retarded time is trailing the particle, and spreads out in a circle at the speed of light.

	We can turn up the energy a bit, now $\gamma = 5$ (the animation will play in a PDF Reader having Adobe Flash installed and activated otherwise see here: \url{https://vimeo.com/575747231}):
	\begin{figure}[H]
		\includemedia[activate=pageopen,width=\textwidth,height=500pt,
	]{}{swf/Synchrotron_radiation_gamma5.swf}
		\caption[]{Synchrotron radiation of a circle moving particle with $\gamma=5$ (author: Jason Cole)}
	\end{figure}
	Here the electron is much closer to the speed of light, and is consequently chasing it's own light-cone – notice how there is a much larger jump in retarded time (from yellow to green). This causes the Poynting flux to be squeezed into a smaller time window, it's much more compressed, intense, and energetic. The temporal shortening of the radiation pulse corresponds to a broad bandwidth, and it is precisely this pulse of radiation that corresponds to broadband synchrotron radiation.
	
	What about something less extreme, an electron travelling at constant velocity in a straight line, again at $\gamma = 1.2$ (the animation will play in a PDF Reader having Adobe Flash installed and activated otherwise see here: \url{https://vimeo.com/575745050}):
	\begin{figure}[H]
		\includemedia[activate=pageopen,width=\textwidth,height=500pt,
	]{}{swf/Synchrotron_radiation_gamma_linear.swf}
		\caption[]{Synchrotron radiation of a uniform moving particle with $\gamma=1.2$ (author: Jason Cole)}
	\end{figure}
	We now see that the field is being dragged along with the particle, but otherwise not radiating. Because the electron is travelling significantly slower than the speed of light, the field looks relatively undistorted. If you squint you'll notice that the transverse field $E_y$ is stronger than the longitudinal field, which is a manifestation of special relativity (Lorentz transforming a uniform field back into the lab frame where the charge is moving).

	Back to something more interesting, a simple dipole motion at $\gamma = 1.2$ (the animation will play in a PDF Reader having Adobe Flash installed and activated otherwise see here: \url{https://vimeo.com/575745668}):
	\begin{figure}[H]
		\includemedia[activate=pageopen,width=\textwidth,height=500pt,
	]{}{swf/Synchrotron_radiation_dipole.swf}
		\caption[]{Synchrotron simple dipole motion with $\gamma=1.2$ (author: Jason Cole)}
	\end{figure}
	We see what we hoped for, oscillating radiation perpendicular to the dipole, and zero far-field radiation parallel to the dipole. In the near-field (close to the dipole), there are electric field components everywhere, but it is clear that these must correspond to non-radiative terms and they die off rapidly.

	Finally, let us turn to a use for this radiation: ondulators/wigglers. These are machines which purposefully oscillate an electron bunch to force synchrotron radiation emission, where the electron travels with high velocity in one direction and oscillates in the other (the animation will play in a PDF Reader having Adobe Flash installed and activated otherwise see here: \url{https://vimeo.com/575747509}):
\begin{figure}[H]
		\includemedia[activate=pageopen,width=\textwidth,height=500pt,
	]{}{swf/Synchrotron_radiation_uniform_oscillating.swf}
		\caption[]{Synchrotron uniform oscillating moving particle (author: Jason Cole)}
	\end{figure}
	Notice that:
	
	as:
	
	can be written:
	
	The killer for CERN is that $E^6$ factor because if we double the energy of a particle, it radiates much more times the equivalent received energy. In the case of electrons, this very quickly becomes such a high power that the accelerator can't accelerate any more – the electrons radiate away their energy quicker than it can be pumped in. For protons of the same energy, because of the $m^6$ factor at the denominator the $E^6$ factor is $10^{19}$ times smaller (as $m_p/m_e\cong 1836$), so radiate much less. This is why the LHC is so big, and uses protons rather than electrons. There are plans to build an even bigger LHC, unimaginatively named the "Very Large Hadron Collider", or to accelerate electrons in a straight line so that they radiate much less energy at the proposed International Linear Collider. Both of these propositions are very expensive though, and very large, so there is much interest in smaller accelerators like plasma accelerators.
	
	We can also make appear the influence of the radius of the accelerator. Indeed, starting from:
	
	Remember that we have seen in the section of Classical Mechanics that for a uniform circular motion:
	
	Therefore it follows:
	
	Hence:
	
	We therefore understand that the bigger is the radius, the smaller is the radiated power. This is why linear accelerator (considers as infinite curvature radius\footnote{And obviously for linear accelerator: $\|\vec{a}\times\vec{\beta}\|^2=0$ as the acceleration is collinear to the speed.}) are much more interesting than circular accelerator (but more difficult to build as wee need to find and large enough area to build them...).
	
	\paragraph{Larmor's equation (non-relativistic derivation)}\label{Larmor equation classical derivation}\mbox{}\\\\
	Maxwell's equations imply that all classical electromagnetic radiation is generated by accelerating electrical charges. It is possible to derive the intensity and angular distribution of the radiation from a point charge (a charged particle) subject to an arbitrary but small acceleration $\Delta v / \Delta t$ via Maxwell's equations, but the complicated math we have seen above obscures the physical interpretation that remains clear in J. J. Thomson's illuminating derivation below for the non-relativistic case.

	If a particle with electrical charge $q$ is at rest or moving with a constant velocity, its electric field lines are purely radial: $|\vec{E}|=E_{r}$. Suppose a charged particle initially at rest is accelerated to a small velocity $\Delta v \ll c$ in a short time $\Delta t$. This disturbs the lines of force, and the disturbance travels outward at the speed of light $c$. The figure below shows that at time $t$ after the acceleration, the disturbance will have propagated to $r=c t$ and the perpendicular component of the electric field will have magnitude:
	
	where $\theta$ is the angle between the acceleration vector and the line of sight connecting the charge to the observer.
	\begin{figure}[H]
		\centering
		\includegraphics[scale=0.5]{img/electromagnetism/accelerated_charge.jpg}
	\end{figure} 
	In the above figure the electric field lines from an accelerated electron. The dotted circle shows the initial position of the electron, and the dotted lines are the radial lines of force emanating from that position. At time $t$ after a small acceleration $\Delta v / \Delta t,$ the electron position has moved by $\Delta v t$ and its lines of force have shifted transversely by $\Delta v t \sin (\theta)$.
Coulomb's law for the radial component $E_{r}$ of the electric field (electric force per unit charge) a distance $r$ from a stationary charge $q$ is:
	
	Substituting $r / c$ for $t$ in the prior-previous relation and using Coulomb's law to eliminate $E_{r}$ gives:
	
	and:
	
	where:
	
	The above relation of $E_{\perp}$ valid for any small acceleration, not just a sinusoidal acceleration at a single frequency. The transverse field $E_{\perp}$ at $r=c t$ mimics the acceleration at $t=0 .$ Thus a sinusoidal acceleration would result in a sinusoidal variation of $E_{\perp}$ with the same frequency.

	Notice that $E_{\perp} \propto r^{-1}$ falls more slowly with distance $r$ than $E_{r} \propto r^{-2}$. Far from the charged particle, only $E_{\perp}$ will contribute significantly to the observed electric field. From a distant observer's point of view, only the "visible" component of acceleration perpendicular to the line of sight $(\dot{v} \sin(\theta))$ contributes to the radiated electric field; the "invisible" component of acceleration parallel to the line of sight $\dot{v} \cos (\theta)$) does not appear to radiate! Likewise, the radiated electric field is linearly polarized in the direction parallel to the component of acceleration perpendicular to the line of sight.

	How much power is radiated in each direction? The Poynting flux, or power per unit area is given by (see page \pageref{poynting vector}):
	
	Inserting:
	
	into $\bar{S}$ gives:
	
	The accelerated charge radiates then with a dipolar power pattern $\propto \sin ^{2} (\theta)$ shaped like a doughnut whose axis is parallel to the acceleration $\dot{v}$:
	\begin{figure}[H]
		\centering
		\includegraphics[scale=0.5]{img/electromagnetism/accelerated_charge_radiation_pattern.jpg}
	\end{figure}
	In the figure above the power pattern of Larmor radiation from a charged particle shown for an acceleration vector $\dot{v}$ tilted 60 degrees from the line of sight. The power received in any direction is proportional to the component of $\dot{v}$ perpendicular to the line of sight.
	
	The total power emitted is the integral of $\bar{S}$ on a spherical surface of radius $r$:
	
	The integral $\int_{\theta=0}^{\pi} \sin ^{3} (\theta) \mathrm{d} \theta=4 / 3$, so the total power emitted by the accelerated charged particle is:
	
	So we fall back on the classical Larmor's equation but without the complicated maths!
	
	\paragraph{Abraham-Lorentz force}\index{Abraham-Lorentz force}\mbox{}\\\\
	In the physics of electromagnetism, the "\NewTerm{Abraham–Lorentz force} is an approximated recoil force on an accelerating charged particle caused by the particle emitting\footnote{The formula may also give an approximation when photons are absorbed.} electromagnetic radiation outside of a Quantum Mechanics framework. It is also named the "\NewTerm{radiation reaction force}" or "\NewTerm{radiation damping force}" or even the "\NewTerm{self-force}" or... "\NewTerm{radiation resistance}".
	
	The formula - then we will derive further below in a special case - predates the theory of Special Relativity and is not valid at velocities on the order of the speed of light. Its relativistic generalization is named the "Abraham–Lorentz–Dirac force". Both of these are in the domain of classical physics, not quantum physics, and therefore may not be valid at distances of roughly the Compton wavelength or below. There is, however, an analogue of the formula that is both fully quantum and relativistic, named the "Abraham–Lorentz–Dirac–Langevin equation".

	As we will see it, force is proportional to the square of the object's charge, times the jerk (rate of change of acceleration) that it is experiencing. The force points in the direction of the jerk. For example, in a cyclotron, where the jerk points opposite to the velocity, the radiation reaction is directed opposite to the velocity of the particle, providing a braking action. The Abraham–Lorentz force is also the source of the radiation resistance of a radio antenna radiating radio waves.

	\begin{tcolorbox}[title=Remark,colframe=black,arc=10pt]
	There seem to be pathological solutions of the Abraham–Lorentz–Dirac equation in which a particle accelerates in advance of the application of a force, so-named "pre-acceleration solutions". Since this would represent an effect occurring before its cause (retro-causality), some theories have speculated that the equation allows signals to travel backward in time, thus challenging the physical principle of causality. One resolution of this problem was discussed by Arthur D. Yaghjian and is further discussed by Fritz Rohrlich and Rodrigo Medina.
	\end{tcolorbox}
	
	The simplest non-rigorous derivation (in the engineer way of life...) for the self-force is found for periodic motion from the Larmor equation (\SeeChapter{see section Electrodynamics page \pageref{Larmor equation}}) for the power radiated from a point charge:
	
	If we assume the motion of a charged particle is periodic, then the average work done on the particle by the Abraham–Lorentz force is the  Larmor power integrated over one period from $\tau _{1}$ to $\tau _{2}$:
	
	The above expression can be integrated by parts. If we assume that there is periodic motion, the boundary term in the integral by parts disappears:
	
	Clearly, we can identify:
	
	It should be noted that already in 1905, Abraham clearly stated that this result cannot be always correct! A more rigorous derivation, which does not require periodic motion, was found using an Effective Field Theory formulation. An alternative derivation, finding the fully relativistic expression, was found by Dirac.
	
	Here $\vec{F}_\mathrm{rad}$ is the force, $\vec{\dot{a}}$ is the derivative of acceleration, or the third derivative of displacement, also named "jerk".
	
	Physically, an accelerating charge emits radiation (according to the Larmor formula), which carries momentum away from the charge. Since momentum is conserved, the charge is pushed in the direction opposite to the direction of the emitted radiation.
	
	\pagebreak
	\subsection{Dipole antenna radiation pattern derivation (Hertz Electric Dipole)}\label{Hertz diple}
	Now that we have acquire a quite good knowledge of Maxwell's equations, potential vector, power emission and retarded potential, we can finally study a simple antenna configuration.
	
	Given a current distribution on an antenna, the problem is one of determining the $\vec{E}$ and $\vec{H}$ fields due to this current distribution which satisfies all four of Maxwell’s equations along with the boundary conditions, if any. 
	
	In the vector potential approach (a very clever one and sadly we don't know who was the first who found that very clever trick) we carry out the solution to this problem in second steps by defining intermediate potential functions. In the first step, we determine the potential function due to the current distribution in sinusoidal regime and in the second step we solve the geometric problem under some approximations.
	
	In the analysis that follows, the relationships between the vector potential and the current distribution as well as the $\vec{E}$ and $H$ fields are derived. All four of Maxwell’s equations are embedded in these relationships.
	
	Let us start with the second Maxwell’s equations:
		
	Since the curl of a vector field is divergence-free (vector identity: $\vec{\nabla}\circ(\vec{\nabla}\times\vec{A})=0$), the magnetic intensity $\vec{B}$ can be expressed as a curl of an arbitrary vector field, $\vec{A}$ as we already now it! Thus:
	
	or:
	
	Substituting this into the third Mawell's equation:
		
	 in sinsuoidal regime and phasor notation:
	
	Or:
	
	Since as we know that the curl of a gradient function is zero (vector identity: $\vec{\nabla}\times\vec{\nabla}\cdot U$),the previous relation suggests that the quantity in brackets can be replaced by the gradient of a scalar function. Specifically, a scalar potential function $U$ is defined such that:
	
	Using this we relate the $\vec{E}$ field to the potential functions as:
	
	The following relations:
	
	relate the $\vec{H}$ and $\vec{E}$ fields  to  the  potential functions $\vec{A}$ and $\vec{U}$.
	
	Now, to satisfy the fours Maxwell's equation:
	
	in sinusoidal (harmonic) regime in phasor notation:
	
	we substitute the expression for the $\vec{E}$ and $\vec{H}$ fields in therms of the potentials functions:
	
	which is valid for a homogeneous medium. Expanding the left hand side using the vector identity proved in the section of Vector Calculus:
	
	We have (after rearranging):
	
	So far we have satisfied three of Maxwell’s four equations. Note that only the  curl of $\vec{A}$ is  defined  so  far.  Since  the  curl  and  divergence  are  two independent parts of any vector field, we can now define the divergence of $\vec{A}$. We define $\vec{\nabla}\circ\vec{A}$ so as to relate $\vec{A}$ and $U$ as well as simplify the previous relation by eliminating the second term on the right hand side of the equality. We relate $\vec{A}$ and $U$ by the relation:
	
	This empirical choice (relation) is as we already known, named the "\NewTerm{Lorentz gauge}\index{Lorentz gauge}". With this the magnetic vector potential $\vec{A}$ satisfies the vector wave equation (Helmoltz inhomogenous equation):
	
	where $k=\omega\sqrt{\mu\varepsilon}$ is the "\NewTerm{propagation constant}\index{propagation constant}" (in radian per meter) in the medium.
	
	Now, to satisfy the last remaining Maxwell's equation:
	
	For this purpose we substitute:
	
	in it to get:
	
	Or:
	
	Eliminating $\vec{A}$ from this equation using the Lorentz gauge:
	
	Thus, both $\vec{A}$ and $U$ must satisfy the wave equation, the source function being the current density for the magnetic vector potential, $\vec{A}$, and the charge density for the electric scalar potential function $U$.
	
	Let us seek now a solution to:
	
	Consider a spherically symmetric charge distribution of finite volume $V$, centered on the origin. Our goal is to compute the scalar potential $U(x, y, z)$ (or $U(r, \theta, \phi)$) due to this source, which is the solution of the in-homogeneous wave equation as given by above. Since the charge is spherically symmetric, we will solve the wave equation in the spherical coordinate system.The Laplacian $\vec{\nabla}^2 U$ in the spherical coordinate system is written as proved in the section of Vector Calculus (see page \pageref{scalar laplacian}) by:
	
	The scalar potential $U(r, \theta, \phi)$, produced by a spherically symmetric charge distribution  is  independent of $\theta$ and $\phi$.  Therefore,  the  wave  equation reduces to:
	
	The right hand side of this equation is zero everywhere except at the source itself. Therefore, in the source-free region, $U$ satisfies the homogeneous wave equation:
	
	The solutions for $U$ are the scalar spherical waves given by (or the sum of them!):
	
	where $U_0^{+}$ is a complex amplitude constant and ${e^{-\mathrm{i}kr}}/{r}$ is a spherical wave travelling in the $+r$-direction. $U_0^{-}$ is the complex amplitude of the scalar spherical wave  ${e^{+\mathrm{i}kr}}/{r}$ travelling in the $-r$-direction. By substituting this in the wave equation, it can be shown that it satisfies the homogeneous wave equation!
	
	Consider a static point charge $q$ kept at a point with position vector $\vec{r}'$. The electric potential, $U$, at a point $(r, \theta, \phi)$ (or in cartesian coordinates $P(x,y,z)$), with the position vector $\vec{r}$ is given by:
	
	where $R$ is the distance between the charge and the observation point:

	
	\begin{figure}[H]
		\centering
		\includegraphics[width=0.9\textwidth]{img/electromagnetism/position_vectors_of_source_and_field_points.jpg}
		\caption[]{Position vectors of source and field points}
	\end{figure} 
	If there are more than one point charges, the potential is obtained by the superposition principle, i.e., summing the contributions of all the point charges. If the source is specified as a charge density distribution over a volume, the potential at any field point is obtained by integration over the source volume. To do this, we first consider a small volume $\mathrm{d}V$ centered on $\vec{r}^{\,'}$.
	
	The charge contained in this volume is then $\rho(\vec{r}^{\,'})\mathrm{d}V$, where $\rho(\vec{r}^{\,'})$ is the volume charge density distribution function. So we can compute the potential at any field point $\vec{r}$ due to charge contained in the volume $\mathrm{d}V$ using:
	
	Now let us recall the solution we get in a sinusoidal regime for a spherical symmetry configuration:
	
	So by analogy we may write:
	
	The potential at point $(r,\theta,\phi)$ due to a charge distribution $\rho(x',y',z')$ is obtained by integrating the previous relation over the source distribution:
	
	This must be compared with the couple of relations we had proved earlier during our study of the Bremsstrahlung (\SeeChapter{see section Electrodynamics page \pageref{bremsstrahlung}}):
	
	Then the same developments, lead us to write:
	
	Now, let us consider a dipole antenna centered on the origin and oriented along the $Z$-axis with length $L$. We want to found the radiation pattern equation in polar coordinates (as the configuration is quite symmetrical!).
	
	The current on the antenna will be approximately sinusoidal, with zeros at the ends ($L/2$) of the antenna (hence a "harmonic" oscillator), represented by:
	
	where:
	
	The results we will derive starting from now only in the situation where the antenna is short, i.e., $L\ll \lambda$. That assumption allow us to say that the current in the antenna is independent of position along the antenna, depending only on time!
	
	We use the potential vector expression found earlier above for a sinusoidal regime in a spherical symmetry configuration to find the $Z$-component of the vector potential:
	
	where:
	
	and the norm of $\vec{R}$ can be approximated as (the last row is a Maclaurin series):
	
	Keep in mind that $\theta$ is the angle between the radius vector of the point of interest and the longitudinal axes of the antenna!
	
	Substituting yields:
	
	As $ r \gg z' \cos (\theta) $, the term $z' \cos(\theta) $ can be neglected in the denominator. However, it cannot be neglected in the exponential as it is a phase offset:
	
	Rearranging yields:
	
	We eliminate the absolute value by splitting into two integrals:
	
	Flip limits on the first integral and combine:
	
	Substitute $ e^{\mathrm{i}x} + e^{-\mathrm{i}x} = 2 \cos(x)$:
	
	Now, integrate it, setting aside the constants and limits temporarily:
	
	Integrate by parts:
	
	with:
	
	We have then:
	
	And again integration by part for the second term above:
	
	Let us simplify the result:
	
	Time for some algebra:
	
	So:
	
	Now substitute we substitute the famous trigonometric identity $\cos^2 (\theta( - 1 = -\sin^2 (\theta)$ to get:
	
	and plug in the limits $ z' = 0 $ and $ z' = L/2$ and bring back the constants:
	
	After simplification we get:
	
	In the section of Vector Calculus (see page \pageref{spherical coordinates}) we have proved that the change of coordinates between rectangular and spherical vector components is given by:
	
	As $A_x$ and $A_y$ are zero (vector potential has the same orientation as current!), then we have:
	
	Therefore:
	
	Now we want to focus on the electric field only (as it is curiously the main point of interest of people scared of mobile antennas). For this purpose we know that:
	
	And in the region outside of where lies the electric current, this reduces to:
	
	Hence:
	
	As we have proved in the section of Vector Calculus (see \pageref{rotational in spherical coordinates}), the curl in spherical coordinates is given by:
	
	As in our case as $A_\phi=A_r=0$ and $A_\theta$ doesn't depends on $\phi$ we get:
	
	And again:
	
	Therefore focusing only on the remaining non-null component:
	
	As $k=2\pi/\lambda$ and $\omega=2\pi f=2\pi c/\lambda$:
	
	Therefore we get the "\NewTerm{far-fields for a dipole antenna in harmonic regime of length $L$}":
	
	This equation for $E_\theta$ (that should be denoted $\hat{E}_\theta$ rigorously!) is the general form for the $\theta$ component in spherical coordinates of the far-field $\vec{E}$ field of a dipole antenna of any length oriented along the $z$-axis. The $r$ component is zero due to the far-field assumption and the $\phi$ component is zero due to the electric field's orientation along the $z$-axis!
	
	For the Hertz dipole, the magnetic field has only $\phi$-component and the electric field does not have the $\phi$-component. The electric field lies in the $(r,\theta$) plane.
	
	The electrostatic field is inversely proportional to the frequency. As the frequency of the  current approaches zero, this field diverges to infinity. This field is essentially due to the accumulation of charges on the tip of the antenna. When the current flows in the dipole, the opposite charges get accumulated on the tips of the antenna giving a dipole. With the reversal of the current (every half cycle) dipole reverses its polarity giving an oscillating dipole.  The  electrostatic field is due to this oscillating dipole.  As the frequency decreases, the   accumulated charge for a given current increases and therefore the electrostatic field increases.
	
	Here is a MATLAB™ script to reproduce the 2D electric field radiation pattern of the Hertz dipole antenna:
	\begin{lstlisting}[language=MATLAB, caption={MATLAB code for 2D Hertz dipole antenna}]
		n=377;
		Io=1;
		r=10;
		lambda=0.3;
		k=(2*pi)/lambda;
		L=lambda/2;

		theta=0:0.01:2*pi;
		E=j*n*Io*exp(-j*k*r)*(1/(2*pi*r))*((cos(k*L*cos(theta)/2)
		-cos(k*L/2))./sin(theta));
		polar(theta, abs(E))
	\end{lstlisting}
	Here are the three corresponding plots\footnote{Notice the interesting fact that the maximum of the electric field radiation is not always at $\pi/2$...!} for $L=\lambda/2$ (left), $L=3\lambda$ (right) and $L=6\lambda$ (bottom):
	\begin{figure}[H]
		\centering
		\includegraphics[width=1.0\textwidth]{img/electromagnetism/hertz_2d_dipole_antenna_far_field_harmonic_regime.jpg}
		\caption[Plot of some 2D Hertz dipole radiation pattern]{Plot of some 2D Hertz dipole radiation pattern with MATLAB™ 2019b}
	\end{figure}
	Note that this picture is only a 2D slice of a 3D pattern. So here is the corresponding MATLAB™ script to get a 3D view of the electric field radiation pattern:
	\begin{lstlisting}[language=MATLAB, caption={MATLAB code for 3D Hertz dipole antenna}]
		theta=[0:0.1:2*pi];
		phi=[0:0.1:2*pi];
		n=377;
		Io=1;
		r=10;
		lambda=20;
		k=(2*pi)/lambda;
		L=lambda*1.5;
		E=j*n*Io*exp(-j*k*r)*(1/(2*pi*r))*((cos(k*L*cos(theta)/2)-cos(k*L/2))./sin(theta));
		udb=10*log10(abs(E));
		
		%normalizing in order to make U vector positive
		minu=min(udb);
		u=udb-minu;
		
		%expanding theta,phi,u to span entire space
		u(1,1)=0;
		for n=1:length(phi)
		    theta(n,:)=theta(1,:);
		end
		phi=phi';
		for m=1:length(phi)
		    phi(:,m)=phi(:,1);
		end
		for k=1:length(u);
		    u(k,:)=u(1,:);
		end
		[x,y,z]=sph2cart(phi,theta,u);
		surf(x,y,z);
		title('3D radiation pattern of dipole antenna');
	\end{lstlisting}
	Therefore we get for $L=\lambda/2$ (left) and $L=1.5\lambda$ (right):
	\begin{figure}[H]
		\centering
		\includegraphics[width=1.0\textwidth]{img/electromagnetism/hertz_3d_dipole_antenna_far_field_harmonic_regime.jpg}
		\caption[Plot of some 3D Hertz dipole radiation pattern]{Plot of some 3D Hertz dipole radiation pattern with MATLAB™ 2019b}
	\end{figure}
	
	\subsubsection{The $L=\lambda/2$ (half-wave) far-field resonant antenna}
	As $k=2\pi/\lambda$ and $\omega=2\pi f=2\pi c/\lambda$ and that many textbooks like to take $L=\lambda/2$ (therefore $\cos(kL/2)=\cos(2\pi L/(2\cdot 2L)=0$), and that also $\mu=1/(c^2\varepsilon)$, you can found the last relation:
	
	sometimes under the following form:
	
	
	Numerically it can be shown that:
	
	such that we fall back on a famous relation that we can found in many textbooks:
	
	The radiating pattern of the $\lambda/2$ resonant far-field antenna with this last approximation looks very similar to a normal dipole. AS we can illustrate it again with the previous MATLAB™ script.
	
	Let us calculate the total emitted power by integrating $P/S$ on the surface of sphere of radius $r$. We have then:
	
	And as we have already proved it, the surface element of the sphere is:
	
	Therefore the total emitted power is given by:
	
	Now let's see something for the fun. As we have proved it in the section of Electrical Engineering (see page \pageref{average power}) we have:
	
	Therefore:
	
	Therefore for our $L=\lambda/2$ far-field resonant antenna we get the value of a theoretical resistance that summarize the antenna Joule effect:
	
	
	
	\begin{tcolorbox}[title=Remark,colframe=black,arc=10pt]
	The natural electric field of the Earth refers to the planet Earth having a natural direct current (DC) electric field or potential gradient from the ground upwards to the ionosphere. The static fair-weather electric field in the atmosphere is $\sim 150\;[\text{V}\cdot\text{m}^{-1}]$  near the Earth's surface, but it drops exponentially with height to under $1 \;[\text{V}\cdot\text{m}^{-1}]$ at $30$ [km] altitude, as the conductivity of the atmosphere increases. Keep also in mind that at the Earth's surface, the energy surfacic density is approximately $1,000\;[\text{W}\cdot\text{m}^{-1}]$ for a surface perpendicular to the Sun's rays at sea level on a clear day.
	\end{tcolorbox}
	
	Some readers may ask themselves if we could build antenna that emit in the range of the visible light $400$ to $800$ [THz]? The answer is that with traditional circuitry then answer is likely NO. At such high frequencies there are no good conductors so it would be very hard to drive such an antenna. We probably would need some waveguides rather than wires. And antenna size should be comparable to (or even slightly less than) wavelength, which further complicates things.
	
	But such oscillators do exist. Consider LED diode, or still better - LASER. Both rely on electronic transitions. So we probably can't build oscillator in question, but we can use atoms, quantum wells and dots and many others acting as such oscillator.
	
	Of the total power $P_t$ (transmitted power) supplied to the antenna, a part, $P_\text{rad}$, is radiated out into space, and the remainder, $P_\text{loss}$ is dissipated as heat in the antenna structure. The "\NewTerm{radiation efficiency}" $\xi$ is defined as:
	
	The "\NewTerm{gain of an antenna}\index{gain of an antenna}", in a given direction, is defined as the ratio between the power per unit area radiated in the chosen direction and the power per unit area that would radiate an isotropic (hypothetical) antenna that would radiate the same total power :
	
	We put all these terms conditional because the isotropic antenna does not exist (or at least is quite difficult to create at the day we write these lines). We can, if necessary, build a network of antennas whose radiation is almost isotropic!
	
	When we talk about the gain of an antenna without specifying in which direction, it is the gain in the direction where it is maximum.
	
	Let's calculate the gain of an elementary dipole of antenna note. The best direction is for $\theta=\pi/2$ when $L=\lambda/2$. Then:
	
	And for the isotrope:
	
	Then:
	
	But as:
	
	We then have:
	
	
	\begin{tcolorbox}[title=Remark,colframe=black,arc=10pt]
	For anyone wishing to deepen the subject of antennas, we strongly recommend the lecture of the master piece \textit{Electromagnetic Waves and Antennas} (\cite{orfanidis2002electromagnetic}) that also contains numerous MATLAB™ scripts!
	\end{tcolorbox}

	\begin{flushright}
	\begin{tabular}{l c}
	\circled{70} & \pbox{20cm}{\score{4}{5} \\ {\tiny 29 votes,  75.86\%}} 
	\end{tabular} 
	\end{flushright}

	%to force start on odd page
	\newpage
	\thispagestyle{empty}
	\mbox{}		
	\section{Electrokinetics}\label{electrokinetics}
	\lettrine[lines=4]{\color{BrickRed}T}he development of electrodynamics has enabled a part of the humanity to significantly change its life quality. We know almost all today what we have thanks to it: fridge, radio, TV, computers, scanners,  cars, trams, trains, planes, robots, smartphones, tablets, and other wonderful things and also sometimes less wonderful...
	
	Before you start studying electrokinetics (engineers speak of "electronics" or "electrotechniques"), that is to say the structures involving electric charges in movement, we will define the two fundamental laws (the term "law" is a misnomer, as the first one was proved in the section Electrostatics and the second one in the section of Electrodynamics but anyway...) of the study of electrokinetic and the basic terminology/jargon of electrical circuits or equipment (practical cases being studied in section Electrical Engineering). Even if some elements in the beginning of this section will perhaps not immediately be understood by the reader (especially industrial applications), they will become trivial as the progress of reading of the section.
	
	\textbf{Definitions (\#\mydef):}
	\begin{enumerate}
		\item[D1.] An electrical circuit is constituted by a set of devices named "\NewTerm{dipole}\index{dipole}", interconnected by a conductive wire.
		
		\item[D2.] A "\NewTerm{node}\index{node}" of a circuit is an interconnection where three or more conductor wire are connected together.
		
		\item[D3.] A "\NewTerm{branch}\index{branch}" is a circuit section between two nodes.
		
		\item[D4.] Finally, a "\NewTerm{mesh}\index{mesh}" is a set of branches forming a closed loop.
	\end{enumerate}
	\begin{figure}[H]
		\centering
		\includegraphics[scale=0.9]{img/electromagnetism/circuit_vocabulary.jpg}
		\caption{Circuit basic vocabulary}
	\end{figure}
	\begin{tcolorbox}[title=Remark,colframe=black,arc=10pt]
	It is very important to understand what will follow! Some developments will be reused in the section of Special Relativity, Quantum Field Theory, etc. Furthermore, the reader should also read in parallel the section of Special Relativity to better understand the ins and outs of certain results and the provenance of some mathematical tools.
	\end{tcolorbox}
	The dipole is characterized by the response in a current $I$ by (most of time) a difference of electric potential $U$ between its terminals. That is to say i.e. by the characteristic curve:
	
	Obviously some dipoles reacts on radiations difference, or on rotation difference, on pressure, on humidity and son on...
	
	We shall see that in any conductor, the presence of a resistivity (see below the concept) causes a voltage drop and, strictly speaking, it is the same for the wires (made of conductor material). But these latter being connected in series with other dipoles, we neglect usually in small circuits the resistance of wires relatively to those of the dipoles. Therefore, the wires located between two dipoles of a small circuit are supposed to be equipotentials (the potential is the same on the two terminals).
	
	\subsection{Kirchhoff's laws}
	If you connect lots of passive or active elements together in a complicated network, then currents will flow through all the various elements so as to insure that charge is conserved, energy is conserved, and Ohm's Law is satisfied for each resistor.
	
	Simultaneously satisfying all these conditions will give you exactly one solution. The method for writing down equations to represent these conservation laws is named "\NewTerm{Kirchhoff's Laws}\index{Kirchhoff's Laws}" (not to be confused with those of the thermodynamics and optics bearing the same name!) and express the physical properties of the charge and the electric field and are at the number of $2$ (one law for each property).
	
	Briefly:
	\begin{enumerate}
		\item The total current flowing towards a node is equal to the total current flowing from that node
		
		\item In a closed circuit, the algebraic sum of the products of the current and the resistance of each part of the circuit is equal to the resultant electromotive force in the circuit. 
	\end{enumerate}
	
	They will enable us without using the heavy mathematical artillery implicitly hidden behind just to get highly relevant results.
	
	\subsubsection{Mesh law (Kirchhoff's Loop Law)}
	The mesh law (implicitly it is simply the conservation of energy) expresses the fact that when a charge browse a closed circuit (closed path), the energy it loses in by browsing a part of the circuit is equal to the energy it gains in the other. Thus, the algebraic sum of the potentials along a mesh is zero such that:
	
	For this, we must arbitrarily choose a direction of travel of the mesh and agree that the tensions whose arrow points in the direction of travel are counted as positive and others as negative.
	\begin{tcolorbox}[title=Remark,colframe=black,arc=10pt]
	This law simply expresses the fact that the electric field (Coulomb) is a conservative vector field as we have seen in the section of Electrostatic.
	\end{tcolorbox}
	
	In mesh analysis, Kirchhoff's voltage law $\sum V = 0$ (using the \LaTeX{} Circuitikz package notation) is written for each independent loop, and a system of equation is solved. The number of independent loops can be determined by finding the minimum number of wire cuts needed such that there are no loops.
   \begin{center}
        \begin{tikzpicture}[transform shape, thick, american currents]
        \draw (-4, 0) to [V, i_>=$i$, l_=$V\equals 2\;\text{[V]}$, invert] (4, 0)
                     -- (4,3) -- (3.5, 3) -- (3.5, 4) to
                [R, l_=$R_2\equals 3\;\si{[\Omega]}$, i=$i-i_1$] (0,4) node[above] {$a$} to
                [R, l_=$R_1\equals 2\;\si{[\Omega]}$, i=$i-i_1+i_2$] (-3.5, 4)
                -- (-3.5, 3) -- (-4, 3) to node[ground,rotate=-90]{} (-4,0);
        \draw (-3.5, 3) -- (-3.5, 2) to [R, l_=$R_4\equals 5\;\si{[\Omega]}$, i<=$i_1-i_2$] (0,2) node[below] {$b$} to [R, l_=$R_5\equals 1\;\si{[\Omega]}$, i<=$i_1$] (3.5, 2) -- (3.5, 3);
        \draw (0, 4) to [R, l_=$R_3\equals 4\;\si{[\Omega]}$] (0,2);
        \draw (-2.5,3) node[scale=3]{$\circlearrowleft$} node{$A$};
        \draw (2,3) node[scale=3]{$\circlearrowleft$} node{$B$};
        \draw (0,1) node[scale=3]{$\circlearrowleft$} node{$C$};
        \end{tikzpicture}
    \end{center}
    We have three independent loops, labelled as $A,B,C$, which gives the system of three equations:
    \begin{align}
        0 &= -i_2R_3-(i-i_1+i_2)R_1+(i_1-i_2)R_4 \\ 
        0 &= -(i-i_1)R_2 + i_2R_3 + i_1R_5 \\ 
        0 &= V - i_1R_5 - (i_1-i_2)R_4
    \end{align}
    Solving this system gives:
    
    which can be used to determine the currents in all resistors and the associated voltages.
	
	\subsubsection{Nodes law (Kirchhoff's Point Law)}
	The nodes law (implicitly it is simply the current conservation law) expresses the conservation of charge, which means that the sum of currents leaving a node (a node can be seen as a separator of field lines - in extenso volumes connected by a same surface) is equal to the sum of currents entering the node. In other words, the algebraic sum of the currents is zero at every node of a circuit such that:
	
	This is also often denoted (if we denote by $a$ the charges arriving in the node and by $d$ those departing):
	
	For this, we must choose a sign for incoming currents and the opposite sign for outgoing currents (as we do in Thermodynamics with the mass).
	\begin{tcolorbox}[title=Remark,colframe=black,arc=10pt]
	This law simply expresses the charge conservation equation (or equivalently the law continuity of the charge) that we have proved in the section Electrodynamics.
	\end{tcolorbox}
	\begin{figure}[H]
		\centering
		\includegraphics[scale=0.5]{img/electromagnetism/apply_kirchhoffs_law.jpg}
		\caption[]{Try to apply Kirchhoff's law...}
	\end{figure}
	In nodal analysis, we deal with the potentials at each node and write out Kirchhoff's current law $\sum I = 0$ for the current leaving (or alternatively, entering) each node for each node where the potential is unknown. Let the nodes labelled $a$ and $b$ have potentials $V_a$ and $V_b$ in the figure below:
    
    \begin{center}
        \begin{tikzpicture}[transform shape, thick, american currents]
        \draw (-4, 0) to [V, i_>=$i$, l_=$V\equals 2\;\text{[V]}$, invert] (4, 0)
                     -- (4,3) -- (3.5, 3) -- (3.5, 4) to
                [R, l_=$R_2\equals 3\;\si{[\Omega]}$, i=$i-i_1$] (0,4) node[above] {$a$} to
                [R, l_=$R_1\equals 2\;\si{[\Omega]}$, i=$i-i_1+i_2$] (-3.5, 4)
                -- (-3.5, 3) -- (-4, 3) to node[ground,rotate=-90]{} (-4,0);
        \draw (-3.5, 3) -- (-3.5, 2) to [R, l_=$R_4\equals 5\;\si{[\Omega]}$, i<=$i_1-i_2$] (0,2) node[below] {$b$} to [R, l_=$R_5\equals 1\;\si{[\Omega]}$, i<=$i_1$] (3.5, 2) -- (3.5, 3);
        \draw (0, 4) to [R, l_=$R_3\equals 4\;\si{[\Omega]}$] (0,2);
        \draw (-2.5,3) node[scale=3]{$\circlearrowleft$} node{$A$};
        \draw (2,3) node[scale=3]{$\circlearrowleft$} node{$B$};
        \draw (0,1) node[scale=3]{$\circlearrowleft$} node{$C$};
        \end{tikzpicture}
    \end{center}
     Then we have:
    \begin{align}
        0 &= \frac{V_a-V}{R_2} + \frac{V_a-V_b}{R_3} + \frac{V_a-0}{R_1} \\
        0 &= \frac{V_b-V}{R_5} + \frac{V_b-V_a}{R_3} + \frac{V_b-0}{R_4} 
    \end{align}
    which after solving, gives:
    
    which can be easily confirmed via the mesh analysis done earlier above.
	
	\subsection{Drude model}
	The Drude model of electrical conduction will allow us to introduce the basic concepts of the electrokinetic. First, let us define inf what will follow the concept of "\NewTerm{Electric Current}\index{electric current}", "\NewTerm{electric current density}\index{electric current density}", and "\NewTerm{electric resistance}\index{electric resistance}".
	
	An electrical conductor (we are not talking about semiconductors and superconductors at this level of our discussion) can be seen in a very simplified way as a pipe section $\vec{S}$ containing an electron gas of $n$ elementary electric charges $q$ per unit volume.
	
	In the absence of electric field, each electron has a zero vector average speed because it remains in the vicinity of the atom. Under the action of a constant and homogeneous electric field $\vec{E}$ (the case of direct current therefore!), some electrons are moved in a particular direction, until they collide with another atom (traditional aspect... not quantum one obviously!!!) where they take again an average vectorial zero speed drift and so on.
	
	This is the oldest model and the most basic one of the electric current. The bases were laid by Paul Drude in 1902, shortly after the discovery of the electron by Joseph John Thomson (1897). Hence the name "\NewTerm{Drude model}\index{Drude model}".
	
	Insufficient to conceive and develop most of the active electrical components that exists since the late 20th century, the billiard balls model has nevertheless considerable interests:
	
	\begin{itemize}
		\item This is a useful tool to give to our limited mind a picture of phenomena we do not get any direct perception (at least at this day), since they take place in the infinitesimal world of atoms an elementary particles.

		\item The results, for the engineer, of more accurate modern theories, especially such as energy band theory, allow themselves to be formulated using the same concepts as those appearing in the "Billiard Balls" Drude model. Let us quote among them the "density number" and the "electron mobility" concepts (that we will introduce later rigorously further below in our study of energy band theory).

		\item Even if this approach is quite primitive, this model leads to a phenomenological interesting interpretation of the fundamental laws such as Ohm's law or the Joule's law. It binds together certain microscopic phenomena and observable quantities.
	\end{itemize}
	As it's friendly name suggests it, this model treats the electrons as tiny billiard balls. These particles are therefore considered as classical objects, simply governed by Newton's and Maxwell's laws proved in previous sections of this book. This corpuscular conception of the electron is also not totally opposed to the results of quantum physics (study in the next sections of this book), in which a wave packet can always be interpreted as a particle with its mass and speed (see the Ehrenfest's theorem in the section of Wave Quantum Physics).
	
	In a millimetre copper cube, we assume that the number of electrons is so high that it therefore does not matter then treat individually, which would also be irrelevant. It is the average behaviour of electrons that should be studied! Two types of interactions that determine behaviour are:
	
	\begin{itemize}
		\item The interaction of electrons with the material in which they operate, and to which they belong;
	
		\item The interaction of electrons with the electromagnetic field applied from the outside (all other interaction being neglected).
	\end{itemize}
	The distance $\lambda$ travelled by an electron distance is named "\NewTerm{electron mean free conduction path}\index{electron mean free conduction path}" and if $\tau$ is the time interval between two successive collisions then we have trivially:
	
	where $v$ is obviously the mean electron velocity of the material.
	
	The collision time is a random variable. All physical parameters maintained as constant, this random variable is stationary, its average value is named "\NewTerm{mean collision time}\index{mean collision time}".
	
	We suppose that:
	
	the mean velocity, is therefore created by the acceleration of the electric field (\SeeChapter{see section Electrostatic page \pageref{electric force}}):
	
	We then get the "\NewTerm{mean drift velocity}\index{mean drift velocity}" or simply electrons "\NewTerm{drift velocity}\index{drift velocity}" given by:
	
	This relation is so named because their initial speed is maintained due to the thermal excitation of the external environment and corresponds to the thermal velocity for which we have determined the expression in our study of the Maxwell-Boltzmann distribution in the section of Statistical Mechanics (we will calculate their further below in this section).
	
	We admit, in the context of the billiard balls model, that electrons behave like atoms of an ideal gas. This is a gross approximation but enough satisfying for now!
	
	The mean velocity is assumed to be identical for all the free electrons when a constant homogeneous electric field is applied, stationary, and directed along a single axis. It gives the possibility to define the "\NewTerm{current intensity $I$}\index{current intensity}" of electric current in the conductor.
	
	\textbf{Definition (\#\mydef):} The "\NewTerm{electric current}\index{electric current}" or "\NewTerm{electric intensity}\index{electric intensity}", denoted by $I$,  measures the charge $\mathrm{d}Q=nq$ that passes through the cross section $S$ of a conductor per unit of time $\mathrm{d}t$ and is given according to what has been shown just before by:
	
	A slice of conductor, of volume $\mathrm{d}V=S\mathrm{d}L$ contains therefore the electric charge:
	
	It passes through the section $S$ in a time $\mathrm{d}t$, such as:
	
	The current is therefore written:
	
	\begin{tcolorbox}[title=Remark,colframe=black,arc=10pt]
	Attention to the usage of the latter relation in practice! If we consider a theoretical wire, the potential difference across the wire is zero. So there will be no gain / loss of kinetic energy of the electron and therefore no change in speed. If we now consider a real wire, resistive therefore, the potential difference across its terminals will be low but not zero, and the electron's potential drop will not be won in kinetic energy but dissipated as heat in the wire.
	\end{tcolorbox}
	If $I$ is seen as the flow of a "\NewTerm{current density $J$}\index{current density}\label{current density}" through the surface $S$, then we have:
	
	the current density being assumed constant at each point of the surface.
	
	We have therefore:
	
	and after simplification:
	
	which is therefore the expression of the "current density" in the conductor.

	As we know the expression of the velocity, this lead us to the famous "\NewTerm{drift current}\index{drift current}\label{drift current}" relation:
	 
	\begin{tcolorbox}[title=Remark,colframe=black,arc=10pt]
	That latter relation is often denoted:
	
	where $\rho=\rho_q/m$ is the density of charges (for recall $\rho_q$ was the massic density of charges) and $\mu$ is another notation of the mobility $\tau$. Or this relation is also sometimes denoted:
	
	\end{tcolorbox}
	And we define the "\NewTerm{conductivity}\index{conductivity}" by:
	
	where this time $n$ is not the number of electrons, but the number density of electrons! By definition, the "\NewTerm{resistance}\index{resistance}" is the inverse of the conductivity!
	
	We notice that the conductivity contains the product of the number density of electron by their mobility. Therefore it is necessary that at least one of these variables has a high value for a material has a high conductivity.
	
	The mobility is greater in the semiconductors than in metals. This characteristic however is completely masked by the ratio of the volumetric numbers of electrons: $n$ is between $1,000,000$ to $100,000,000$ times lower in semiconductors than in metals, which explains the higher conductivity of these.
	
	Following the relation:
	
	proved just earlier above, the conductivity depends on the electric field, through the collision time. Indeed, the more the electric field grows, the more the speed of the electrons increases. The distance between the points of possible shocks remaining the same, the collision time, and hence the conductivity should decrease (and thus the resistance increase!).
	
	However, the independence of the conductivity (and respectively the resistance) with the electric field is an experimental fact established precisely with all usual conductors in normal civilian uses.	
	
	The origin of this contradiction lies in the large difference in the magnitudes of the thermal velocity given by the Maxwell-Boltzmann distribution (\SeeChapter{see section Statistical Mechanics page \pageref{maxwell distribution}}):
	
	and of the average speed drift seen above:
	
	with the mean free path time that will be obtained using the expression:
	
	
	We have proved in the section of Statistical Mechanics that for an electron at room temperature:
	
	And let us calculate the drift velocity for copper with for this particular metal the following values:
	
	This allows us to get the value:
	
	and therefore:
	
	Taking $E=0.32 \;[\text{V}\cdot\text{m}^{-1}]$, which is considered as a high value since this field produces a current density of:
	
	we finally have:
	
	Therefore, even in a strong industrial electric field, the drift velocity is negligible compared to the thermal velocity.
	
	As the thermal velocity depends only a little bit of the electric field, it turns out that in practice the electron velocity is independent of the electric field. In other words, establishing a current, even intense, has only a negligible effect on the speed of electrons!
	
	\begin{tcolorbox}[title=Remark,colframe=black,arc=10pt]
	In the vast majority of cases, the conductor sizes are large, compared to the average distance that an electron travel between two consecutive shocks. The behaviour of the surface of the conductive electrons then are of secondary importance. This is why the conductive medium is often implicitly considered as infinite. The FET (Field-Effect Transistor) and MOST (Metal Oxide Semiconductor Transistor) are an important exception to this. The current circulates in a layer thin enough that the electron mobility is affected by electron scattering on surfaces defining this layer.
	\end{tcolorbox}
	However, an important point to notice is the calculation of the mean free path of electrons in the classical Drude model. We have indeed:
	
	which is much higher, at least an order of magnitude (factor of $10$), to the inter-atomic distances. It results from this that successive collisions with atoms of the network is not responsible for Ohm's law (which we will see now) contrary to one of the initial assumptions of the Drude model but that it are the impurities and defects of the material involved in it that generates the collisions! We will see a further below in the theoretical model of energy bands that the mean free path is in fact still much greater!
	
	Warning!!! This relation may suggest that since the mean free path is proportional to the thermal velocity and therefore proportional to the square root of the temperature, that the resistance decreases with temperature. But in fact it is not so! The Drude model is too simplistic because in reality it is the opposite that occurs for conductors (resistance increases with temperature because the time interval $\tau$ between collisions decreases faster than the speed increases). And then there is also an opposite problem ... almost at a temperature equal to the zero Kelvin (and over for some materials) the mean free path should be almost zero but superconductors show us that it is not the case! In short, without explicit relation depending on the temperature we are in total darkness!
	
	The only thing we know how to do is to admit that to a given constant factor $\alpha$ (positive or negative), a temperature change requires a relative change in resistance by (first order Taylor development):
	
	hence:
	
	Then we get the relation know in high-schools:
	
	Finally, let us notice that the fourth Maxwell equation (\SeeChapter{see section Electrodynamics page \pageref{fourth maxwell equation}}) can then be written by  the results obtained just previously:
	
	which then explicitly makes appear the conductivity coefficient.
	
	\pagebreak
	\subsection{Ohm's law}\label{ohm law}
	Ohm's law states that the current through a conductor between two points is directly proportional to the voltage across the two points. Introducing the constant of proportionality, the resistance.
	
	From the relation proved just above:
	
	and taking the definition of "\NewTerm{conductivity}\index{conductivity}" by:
	
	We have finally:
	
	which is the "\NewTerm{local Ohm's law}\index{local Ohm's law}". We have already see it in differential form in the section of Statistical Mechanics and we already know therefore that it belongs in fact to the family of diffusion laws!
	\begin{tcolorbox}[title=Remark,colframe=black,arc=10pt]
	Since the conductivity is necessarily a scalar, the vector notation of Ohm's law implies that the electrostatic field lines also indicate the path taken by the electrical charges. Moreover, as the conductivity is a scalar necessarily positive in the traditional model, this implies that the current has the same direction as the electric field.
	\end{tcolorbox}	
	If we multiply the previous equality under scalar left and right by a length $L$ we get:
	
	Then we have:
	
	or:
	
	We define the inverse of conductivity as the "\NewTerm{electric resistance}\index{electric resistance}" defined by:
	
	\begin{tcolorbox}[title=Remark,colframe=black,arc=10pt]
	It is important to notice that the electrical resistance is proportional to the length of the resistive element and inversely proportional to its sectional area. For example in high voltage cables, the resistance is given in ohms per kilometre, which permits then to calculates the power lost by kilometre and therefore the money lost by Joule loss.
	\end{tcolorbox}	
	Therefore, we can write Ohm's local law in its most commonly known form:
	
	whence (beware !!!) the potential $U$ is the potential difference over the length of the resistive element (also named "\NewTerm{resistive dipole}\index{resistive dipole}") as we see it in the previous developments and not the total outside potential!
	\begin{figure}[H]
		\centering
		\includegraphics[scale=0.6]{img/electromagnetism/resistors.jpg}
		\caption[Some resistive dipoles]{Some resistive dipoles (source: \url{http://www.e-style.ch}, author: Martin Bircher)}
	\end{figure}
	As for the capacitors seen in the section Electrostatics, resistors have also color codes that have as far as we know until know the same definitions:
	\begin{figure}[H]
		\centering
		\includegraphics[scale=0.8]{img/electromagnetism/resistors_color_code.jpg}
		\caption{Resistor color codes}
	\end{figure}
	\begin{tcolorbox}[title=Remark,colframe=black,arc=10pt]
	This relation is valid only for ideal conductors under normal conditions of temperature and pressure and for which the Drude model applies. So semiconductors and superconductors are excluded.
	\end{tcolorbox}
	Since $U$ is the potential of the resistive element, then we often do reference in the field of electrical engineering to the "\NewTerm{voltage drop}\index{voltage drop}" (indeed, beyond the resistive element the potential is not the same that at the point above this same resistive element).
	
	For copper cables in typical non-industrial use, there is a very useful American table in practice giving with a relatively good accuracy the resistivity according to the diameter and the maximum permissible current. Here is a sample of this table:
	\begin{table}[H]
		\begin{center}
			\definecolor{gris}{gray}{0.85}
				\begin{tabular}{|c|c|c|c|c|}
					\hline
					\multicolumn{1}{c}{\cellcolor{black!30}\textbf{AWG}} & 
	  \multicolumn{1}{c}{\cellcolor{black!30}\parbox{3cm}{\textbf{Wire $\varnothing$ in [mm]} \\ \textbf{with isolation}}} & \multicolumn{1}{c}{\cellcolor{black!30}\parbox{2.5cm}{\textbf{Resistance in} \\ \textbf{[$\boldsymbol{\Omega}\cdot\text{m}^{-1}$]}}}  & \multicolumn{1}{c}{\cellcolor{black!30}\parbox{4.7cm}{\textbf{Max. theoretically} \\ \textbf{eligible current in [A]}}} & \multicolumn{1}{c}{\cellcolor{black!30}\parbox{3.5cm}{\textbf{Max. theoretically} \\ \textbf{eligible current} \\ \textbf{in [A] outdoors}}}  \\ \hline
					$1$ & $7.35$ & $0.0040$ & $211$ & $119$ \\ \hline
					$2$ & $6.54$ & $0.0051$ & $181$ & $94$ \\ \hline
					$\ldots$ & $\ldots$ & $\ldots$ & $\ldots$ & $\ldots$ \\ 	\hline
					$12$ & $2.05$ & $0.00521$ & $41$ & $9.3$ \\ \hline
					$13$ & $1.83$ & $0.00657$ & $35$ & $7.4$ \\ \hline
					$14$ & $1.63$ & $0.00829$ & $32$ & $5.9$ \\ \hline
					$15$ & $1.45$ & $0.0104$ & $28$ & $4.7$ \\ \hline
					$16$ & $1.29$ & $0.0132$ & $22$ & $3.7$ \\ \hline
					$\ldots$ & $\ldots$ & $\ldots$ & $\ldots$ & $\ldots$ \\ \hline
			\end{tabular}
		\end{center}
		\caption[AWG Codes]{AWG Codes (source: Wikipedia)}
	\end{table}
	where AWG stands for "\NewTerm{American Wire Gauge}\index{American Wire Gauge}" and corresponds to a small gauge that can be buy easily on Internet to determine the diameter of a cable without having to use a calibrator:
	\begin{figure}[H]
		\centering
		\includegraphics{img/electromagnetism/awg_gauge.jpg}
		\caption[AWG Gauge]{AWG Gauge (source: Wikipedia)}
	\end{figure}
	
	\pagebreak
	\subsubsection{Complex Conductivity}\label{complex conductivity}
	The "\NewTerm{complex conductivity}\index{complex conductivity}" is a simple result of a simple extension of the classical Drude model for the material conduction. In the Drude model it is assumed that in the conduction material there is a concentration $n$ of electrons and that an electron has a probability $1/\tau$, where $\tau$ is named "\NewTerm{scattering time}" or "\NewTerm{relaxation time}", for scattering with the lattice for losing its momentum. The motion of the electrons can be described by Newton's law:
	
	where $m$ is the (effective) electron mass, $v$ the mean electron velocity in field direction, and $E$ is the electric field strength.
	
	This relation is easy to derived. Indeed, consider the motion of an electron in an electric field of strength $E\;[\text{V}\cdot \text{m}^{-1}]$. The force on the electron is $-qE$. In the event of a collision between the electron and a gas molecule, energy is lost in the form of heat. The frictional or retarding force due to the collision is $mv/\tau$, or $mv\nu$, where $\tau$ is the mean time between collision, $\nu$ is the frequency of collision and $m$ is the effective mass of electron. Therefore, net force on the electron is:
	
	Since the electron is moving with a velocity $v$, the force on the electron is equal to the rate of change of the momentum:
	
	
	
	The term $-mv/\tau$ resembles a "friction term" in the equation of motion. Assuming sinusoid variation of the electric field, ie:
	
	Thus from the differential equation above it follows:
	
	Therefore:	
	
	Simplifying:
	
	Hence:
	
	This gives the current density:
	
	and thus the complex material conductivity is defined as:
	
	
	The complex conductivity relates both the magnitude and the phase of the current density to the magnitude and phase of the applied electric field similar to the current and voltage in an alternative current circuit are related by the complex conductance. This can be most easily seen when you remember that only real parts of the complex values of current and electric field are used and you express the complex conductivity by the complex exponential $\sigma=|\hat{\sigma}|\exp( -\mathrm{i}\phi)$. Then you see that $|\hat{\sigma}|$ is the magnitude of the proportionality constant and $\varphi$ is the current-field phase shift.
	
	The real part of $\hat{\sigma}$ thus describes the convective current and the imaginary part corresponds to a displacement current ( imaginary part indicates that the current lags behind the electrical field, which happens because the electrons need roughly a time $\tau$ to accelerate in response to a change in the electrical field).
	
	Like radio waves, the electric field of the optical light acts on the electrons in a material and can cause them to move. But visible light has a much shorter wavelength than radio waves (about $10^6$ shorter) and so it doesn't drive macroscopic currents. Moreover, the conductivity decreases, which we can see in the Drude Model:
	
	where for recall $\omega$ is the light frequency, $\tau$ is the relaxation time (due to scattering and related to the carrier density, temperature, etc.), and $\sigma _{0}=\frac {\rho_q q\tau }{m}$ is the "\NewTerm{effective DC conductivity}\index{effective DC conductivity}" (complex conductivity when $\omega=0$) also named "\NewTerm{nominal conductivity}\index{nominal conductivity}" ($n$ is the number density and $m$ is an effective mass which can be different for electrons and holes depending on the material).

	We can see that when the frequency $\omega$ is large (like for optical waves) the conductivity goes to zero. This happens because at really high frequencies the electrons do not have time to respond to the field and so they no longer (appreciably) move back and forth. Above this frequency the material becomes transparent (identical to the behaviour at the critical frequency in plasmas).
	
	\pagebreak
	\subsubsection{Simple (harmonic oscillator) model of dielectric and conductors (and plasmas)}\label{simple model of dielectric and conductors}
	As we have seen during our study of Rayleigh scattering (\SeeChapter{see section Nuclear Physics page \pageref{Rayleigh scattering}}), a simple model for the dielectric properties of a material may be obtained by considering the motion of a bound electron in the presence of an applied electric field. As the electric field tries to separate the electron from the positively charged nucleus, it creates an electric dipole moment. Averaging this dipole moment over the volume of the material gives rise to a macroscopic dipole moment per unit volume.
	
	A simple model for the dynamics of the displacement $x$ of the bound electron is as follows (with $\dot{x}=\mathrm{d} x / \mathrm{d} t)$:
	
	where we assumed that the electric field is acting in the $x$-direction and that there is a spring-like restoring force due to the binding of the electron to the nucleus, and a friction-type force proportional to the velocity of the electron.

	The spring constant $k$ is related to the resonance frequency of the spring via the relationship $\omega_{0}=\sqrt{k / m_e},$ or, $k=m_e \omega_{0}^{2}$. Therefore, we may rewrite the above relation as:
	
	The limit $\omega_{0}=0$ corresponds to unbound electrons and describes the case of good conductors. The frictional term $\gamma \dot{x}$ arises from collisions that tend to slow down the electron. The parameter $y$ is a measure of the rate of collisions per unit time, and therefore, $\tau=1 / \gamma$ will represent the mean-time between collisions.

	In a typical conductor, $\tau$ is of the order of $10^{-14}$ seconds, for example, for copper, $\tau=2.4 \cdot 10^{-14}\; [\text{s}]$ and $\gamma=4.1 \cdot 10^{13}\; [\text{s}^{-1}]$. The case of a tenuous, collisionless, plasma can be obtained in the limit $\gamma=0$. Thus, the above simple model can describe the following cases:
	\begin{itemize}
		\item[a.] Dielectrics, $\omega_{0} \neq 0, \gamma \neq 0$
		\item[b.] Conductors, $\omega_{0}=0, \gamma \neq 0$
		\item[c.] Collisionless Plasmas, $\omega_{0}=0, \gamma=0$
	\end{itemize}
	The basic idea of this model is that the applied electric field tends to separate positive from negative charges, thus, creating an electric dipole moment. In this sense, the model contains the basic features of other types of polarization in materials, such as Ionic/molecular polarization arising from the separation of positive and negative ions by the applied field, or polar materials that have a permanent dipole moment.
	
	\paragraph{Dielectrics}\mbox{}\\\\
	The applied electric field $E(t)$ above can have any time dependence. In particular, if we will assume it is sinusoidal with frequency $\omega, E(t)=E_0 e^{\mathrm{i}\omega t},$ then the above differential equation will have for solution as proved during our study of Rayleigh scattering (\SeeChapter{see section Nuclear Physics page \pageref{Rayleigh scattering}}):
	
	The corresponding velocity of the electron will also be sinusoidal $v(t)=v e^{\mathrm{i}\omega t}$, where $v=\dot{x}=\mathrm{i} \omega x$. Thus, we have:
	
	From the both relations above we can find the polarization per unit volume $P$. We assume that there are $N$ such elementary dipoles per unit volume. The individual electric dipole moment is $p=q_e x$ (\SeeChapter{see section Electrostatics page \pageref{electric rigid dipole}}). Therefore, the polarization per unit volume will be:
	
	The electric flux density will be then:
	
	where the "\NewTerm{effective permittivity}\index{effective permittivity}" $\varepsilon(\omega)$ is then defined by:
	
	This can be written in a more convenient form, as follows:
	
	where $\omega_{p}^{2}$ is the so-named "\NewTerm{plasma frequency}\index{plasma frequency}" of the material defined by (we have already met this relation during our study of plasmas at page \pageref{plasma frequency}):
	
	The model defined by:
	
	is known as a "\NewTerm{Lorentz dielectric}". The corresponding susceptibility, defined through $\varepsilon(\omega)=\varepsilon_{0}(1+x(\omega)),$ is:
	
	For a dielectric, we may assume $\omega_{0} \neq 0$. Then, the low-frequency limit $(\omega=0)$ gives the "\NewTerm{nominal dielectric constant}":
	
	The real and imaginary parts of $\varepsilon(\omega)$ characterize by convention the refractive and absorptive properties of the material. By convention, we define the imaginary part with the negative sign (because we use $e^{\mathrm{i} \omega t}$ time dependence):
	
	It follows from:
	
	that:
	
	The figure below shows a plot of $\varepsilon^{\prime}(\omega)$ and $\varepsilon^{\prime \prime}(\omega)$ . Around the resonant frequency $\omega_{0}$, the real part $\varepsilon^{\prime}(\omega)$ behaves in an anomalous manner, that is, it drops rapidly with frequency to values less than $\varepsilon_{0}$ and the material exhibits strong absorption. The term "normal dispersion" refers to an $\varepsilon^{\prime}(\omega)$ that is an increasing function of $\omega$, as is the case to the far left and right of the resonant frequency:
	\begin{figure}[H]
		\centering
		\includegraphics[scale=0.7]{img/electromagnetism/effective_permittivity_plot_dielectric.jpg}
		\caption{Real and imaginary parts of the effective permittivity $\varepsilon(\omega)$}
	\end{figure} 
	Real dielectric materials exhibit, of course, several such resonant frequencies corresponding to various vibrational modes and polarization mechanisms (e.g., electronic, ionic, etc.). The permittivity becomes the sum of such terms:
	
	
	\begin{tcolorbox}[title=Remark,colframe=black,arc=10pt]
	A more correct quantum-mechanical treatment leads essentially to the same relation:
	
	where $\omega_{j i}$ are transition frequencies between energy levels, that is, $\omega_{j i}=\left(E_{j}-E_{i}\right) / h$ and $N_{i}, N_{j}$ are the populations of the lower, $E_{i}$, and upper, $E_{j}$, energy levels. The quantities $f_{j i}$ are named "oscillator strengths." For example, for a two-level atom we have:
	
	where we defined:
	
	Normally, lower energy states are more populated, $N_{i}>N_{j},$ and the material behaves as a classical absorbing dielectric. However, if there is population inversion, $N_{i}<N_{j}$ then the corresponding permittivity term changes sign. This leads to a negative imaginary part, $\varepsilon^{\prime \prime}(\omega),$ representing a gain.
	\end{tcolorbox}
	As:
	
	For a material have and relative permeability equal to $1$ (vacuum) then:
	
	Therefore:
	
	can be written in the following form, known as the "\NewTerm{Sellmeier equation}\index{Sellmeier equation}\label{Sellmeier equation}" (where the $B_{i}$ are constants):
	
	In practice the relation above is applied in frequency ranges that are far from any resonance so that one can effectively set $\gamma_{i}=0$ :
	
	or:
	
	where $\lambda, \lambda_{i}$ denote the corresponding free-space wavelengths (e.g., $\lambda=2 \pi c / \omega$ ). In practice, refractive index data are fitted to the relation above using $2-4$ terms over a desired frequency range. For example, fused silica ($\mathrm{SiO}_{2}$ ) is very accurately represented over the range $0.2 \leq \lambda \leq 3.7 \;[\mu \text{m}]$ by the following formula where $\lambda$ and $\lambda_{i}$ are in units of $\mu$:
	
	
	\begin{tcolorbox}[title=Remark,colframe=black,arc=10pt]
	The previous relation:
	
	Can be found in some textbook for the first term ($i=0$) under the following form:
	
	Therefore:
	
	$A$ et $B$ being two constants.When the dispersion is low, $B /\left(\lambda_{0}^{2} A\right) \ll 1,$ we fall back on the well-known "\NewTerm{Cauchy's formula}\index{Cauchy's formula}\label{Cauchy formula}":
	
	This relationship is well verified for dielectrics used in optics. However the constants $n_{0}$ and $C$ are determined experimentally because the simple model adopted here does not take into account the complexity of the electronic properties of materials.
	\end{tcolorbox}
	
	\paragraph{Conductors}\mbox{}\\\\
	We will see now that using the first results above we can fall back on relation of the complex conductivity we get at page \pageref{complex conductivity}.
	
	The conductivity properties of a material are described by Ohm's law. To derive this law from our simple model, we use the relationship $J=\rho v$, where the volume density of the conduction charges is $\rho=N q_e$. It follows from above that:
	
	and therefore, we identify the conductivity $\sigma(\omega)$ :
	
	We note that $\sigma(\omega) / \mathrm{i} \omega$ is essentially the electric susceptibility considered above. Indeed, we have:
	
	 and thus, $P=J / \mathrm{i} \omega=(\sigma(\omega) / \mathrm{i} \omega)E$. It follows that $\varepsilon(\omega)-\varepsilon_{0}=\sigma(\omega) / \mathrm{i}\omega$, and:
	
	since in a metal the conduction charges are unbound, we may take $\omega_{0}=0$. After cancelling a common factor of $\mathrm{i} \omega,$ we get:
	
	So we fall back on the expected result. The nominal conductivity is obtained at the low-frequency limit as we know, $\omega=0$:
	
	
	\begin{tcolorbox}[colframe=black,colback=white,sharp corners]
	\textbf{{\Large \ding{45}}Example:}\\\\
	Copper has a mass density of $8.9 \cdot 10^{6}\;[\text{gr}\cdot \text{m}^{-3}]$ and atomic weight of $63.5 \; [\text{gr}\cdot\text{mole}^{-1}]$. Using Avogadro's number of $6 \cdot 10^{23}$ atoms per mole, and assuming one conduction electron per atom, we find for the volume density $N$ :
	
	It follows that:
	
where we used $q_e=1.6 \cdot 10^{-19}$ [C], $m_e=9.1 \cdot 10^{-31}$ [kg], $\gamma=4.1\cdot 10^{13}\;[\text{s}^{-1}]$. The plasma frequency of copper can be calculated by:
	
	which lies in the ultraviolet range. For frequencies such that $\omega\ll \gamma$, the conductivity may be considered to be independent of frequency and equal to the de value of $\sigma=N q_e^{2}/m_e\gamma$. This frequency range covers most present-day RF applications. For example, assuming $\omega \leq 0.1 \gamma$, we find $f \leq 0.1 \gamma / 2 \pi=653\; [\text{GHz}]$.
	\end{tcolorbox}
	
	\pagebreak
	\subsubsection{Equivalent Resistance}
	We can now study the entire length of a an electric field line collinear with a constant current $I$ supposed to be constant at any point (this is an approximation obviously...) to obtain total resistance if $n$ resistive elements are put  next to each other linearly:
	\begin{figure}[H]
		\centering
		\includegraphics{img/electromagnetism/resistor_series.jpg}
	\end{figure}
	
	The answer is relatively simple since if we denote by $U_{n-1}$ the potential in the first extremity of the resistive element and $U_n$ that of the other extremity. We then have (the reader will have notice that the use of the mesh law in the following relation logically without even necessarily be aware of it existence):
	
	that is to say, a result similar to that obtained by a single resistance whose value is approximately given by (if the electric current is constant throughout the wire) the "\NewTerm{equivalent resistance of resistors in series}\index{equivalent resistance of resistors in series}":
	
	which is the arithmetic sum of the individual resistances.

	Let us now consider $n$ resistance in parallel all at a given voltage $U$ (by law of mesh!) and alimented by an electric current $I$. The current then splits into $n$ streams such that:
	
	in each of the $n$ mesh. Applying the node law, we have:
	
	that is to say that the set of all the resistance put in parallel is analogous to an "\NewTerm{equivalent resistance of resistors in parallel}\index{equivalent resistance of resistors in parallel}":
	
	\begin{figure}[H]
		\centering
		\includegraphics{img/electromagnetism/resistor_parallel.jpg}
	\end{figure}
	Plugging devices in parallel allows to always have the same voltage across them (neglecting the voltage drops). This is the way that the electrical outlets are installed in a domestic installation!
	
	 Therefore for a series circuit, we can calculate the equivalent resistance by adding them. Using the notation of the \LaTeX{} Circuitikz package we have:
    \begin{center}
        \begin{tikzpicture}[transform shape, thick]
            \draw (0, 0) to [V, i_>={$i$},
                                l={$V$}, invert] (0, 4)
                         to [R, l={$R_1$}] (4, 4) node[right] {$A$}
                         to [R, l={$R_2$}] (4, 0)
                         to node[ground]{} (0,0); 
            \draw[fill=black] (4,4) circle (1.5pt);
            \draw[fill=black] (4,0) circle (1.5pt);
        \end{tikzpicture}
    \end{center}
    For this circuit, we have:
    
    and the voltage:
    
     The current is given obviously by $i = \tfrac{V}{R_1+R_2}$ and the voltage at $A$ is given by:
     
    
    For a parallel circuit, the effective resistance is the harmonic sum as proven earlier:
    \begin{center}
        \begin{tikzpicture}[transform shape, thick, american currents]
        \draw (0, 0) to [V, i_>=$i$, l=$V$] (0, 4)
                     -- (4,4)
                     to [R,i_>=$i_1$, l=$R_1$] (4, 0)
                     -- (0,0);
        \draw (4,4) -- (8,4)
                    to [R, i_>=$i_2$, l=$R_2$] (8,0)
                    to node[ground]{} (0,0);
        \end{tikzpicture}
    \end{center}
    In this circuit, we have therefore:
    
    and the current in the branches is given by:
    
    Indeed, as the voltage drop across $R_1$ and $R_2$ must be the same, we  have:
    
    We also have $i_2=i-i_1$, which gives:
    
	
	\subsubsection{Equivalent Capacities}
	Even if this has nothing to do with the Ohm's law we can apply the previous reasoning to capacitor.
	
	Let us recall that we defined in the section Electrostatics, the capacity given by:
	
	Let us consider, as well as resistors, $n$ capacitors of capacity  $C_i$ in series set behind each other:
	\begin{figure}[H]
		\centering
		\includegraphics{img/electromagnetism/capacitors_series.jpg}
	\end{figure}	
	 We put at potential $U_0$ and $U_n$ the two extremities of the chain and we bring the charge $Q$ on the whole system. The potential (voltage) across the total capacitor chain is then written simply:
	
	and therefore corresponds to that of a single capacitor $C_e$ that is the "\NewTerm{equivalent capacitance of capacitors in series}\index{equivalent capacitance of capacitors in series}":
	
	where we also find here a harmonic mean.
	
	Now let us consider $n$ capacitor of  capacity $C_i$ in parallel with the same potential $U$:
	\begin{figure}[H]
		\centering
		\includegraphics{img/electromagnetism/capacitors_parallel.jpg}
	\end{figure}	
	The electric charge of each is then imposed (by the mesh law) by the relation:
	
	The total electrical load is simply:
	
	which corresponds to an "\NewTerm{equivalent capacitance of capacitors in parallel}\index{equivalent capacitance of capacitors in parallel}":
	
	which is the arithmetic sum of the individual capacities.

	Finally to close this subject, let us recall that we have proved in the section of Electrostatics that:
	
	In the case where a capacity is alone in series with an AC sine generator (fairly typical case in the industrial world of the 19th and 20th century), then we have:
	
	And therefore we get:
	
	That we will write in analogy with Ohm's law in the form:
	
	hence:
		
	is named the "\NewTerm{capacitive reactance}\index{capacitive reactance}". We notice that in the continuous case where the pulsation is zero, the capacitive reactance becomes infinite and that we find the known situation where the capacity does not pass current (at least in the ideal case ...).
	
	Therefore, a capacitor connected to a circuit that changes over a given range of frequencies can be said to be "\NewTerm{Frequency dependent}\index{frequency dependent capacitor}".
	
	\pagebreak
	\subsection{Electromotive Force}\label{electromotive force}
	Given $AB$ a portion of an electric circuit travelled by a constant current $I$ from $A$ to $B$. The existence of this current implies that the potential on $A$ is greater (different) in absolute value than in $B$ (in absolute value). This potential difference is reflected in the existence of the electrostatic field $\vec{E}$ producing a Coulomb force:
	
	capable of accelerating a charge $q$.

	Then, given:
	
	the power required to give a speed $v$ to a any particle of charge $q$. Knowing that this in the conductor there is $\rho_q$ electric charge per unit volume, the total power $P$ in the circuit $AB$ travelled by a current $I$ is:
	
	That is to say:
	
	where:
	
	This power is therefore the "\NewTerm{electric power}\index{electric power}\label{electric power}" available between $A$ and $B$, simply because there flows a current $I$.

	If we consider in this electric circuit $\overline{AB}$ a resistive for which we measure a potential difference:
	
	then the power available inside thereof is given by the "\NewTerm{Joule power}\index{Joule power}"
	
	Thus, among this available power, a certain part is dissipated as heat (Joule effect) in a passive dipole such as a resistance. Obviously it is this power that invoice us our power company and to know the corresponding energy consumed, we simply multiply the power of the device which is used by the functioning duration.
	
	Now that we have the latter relation we can finally introduce the famous Ohm Circle that related the most important DC relations:
	\begin{figure}[H]
		\centering
		\includegraphics[scale=0.5]{img/electromagnetism/ohm_circle.jpg}
	\end{figure}
	However..., something is wrong in our previous developments if we look more closely. Indeed, if we apply the reasoning in a closed circuit, that is to say, if we look at the total power supplied between $A$ and $A$ by the Coulomb force, we get (obviously because the Coulomb electrostatic field is conservative):
	
	that is to say null power ?! Eh yes! This means they can not be a steady state current in a closed loop and when there is a current, then this implies that the Coulomb force is not responsible for the global movement of charge carriers in a conductor!!

	Therefore, the current in a conductor can be understood with the analogy of the river flowing in its riverbed. So that there is a flow, it is necessary that the water flows from a higher region to a lower region (a higher gravitational potential to another smaller one). Thus, the movement of water from a highest point towards a lowest point is indeed due to the simple force of gravity. But if we want to form a closed circuit, then we have to provide energy (by a pump) to bring the water to a greater height and the cycle starts again.
	
	This is exactly what happens in an electric circuit. If we want that a permanent current flows, there must be a force other than the electrostatic force that enables the electric charges to close the path (this is a purely mathematical reasoning)! It is for this reason that we must involve an "artificial" external energy source such as an "\NewTerm{electrical generator}\index{electrical generator}" which is then equivalent to the hydraulic pump for water.
	
	The generator involve then as physical property that only when its circuit is open (the current $I$ then being equal to zero) a "\NewTerm{potential difference}\index{potential difference}" is maintained between its terminals necessarily involving the presence of another force compensating the Coulomb attraction of the conductor. Thus, the total force acting on an electric charge $q$ is written thus:
	
	with being $\vec{E}_S$ the electrostatic field and $\vec{E}_M$ the "\NewTerm{electromotive field}\index{electromotive field}". In equilibrium and in the absence of current, we must have:
	
	This means that the difference of potential across an open generator is then:
	
	We name and denote by:
	
	(somewhat abusively) the proper "\NewTerm{electromotive force EMF}\index{electromotive force}" of the generator.
	
	Since, inside the generator, we have:
	
	at open circuit, this means that a generator is a non-conductive equipotential (or with "non-conservative field").

	At equilibrium, but in the presence of a current $I$ (generator set in a closed circuit), the charge carriers responsible for the current undergo additional force due to collisions occurring inside the conductor. For an ideal generator, these collisions are negligible and we get:
	
	However, for a non-ideal generator, such collisions occur and result in the existence of an internal resistance $r$ (very small for generators that just go out of the factory!). Thus, the true electromotive force is given by:
	
	The internal resistance of the generator introduces a voltage drop proportional to the current supplied, so it outputs a potential lower than that given by it electromotive force.
	
	The latter relation is sometimes noted as follows:
	
	and often with the following writing:
	
	What is measured with a voltmeter is however the generator electromotive force (GEF) since the generators have an internal resistance admit as infinite and therefore involve a current $I$ almost equal to zero:
	\begin{figure}[H]
		\centering
		\includegraphics[scale=0.8]{img/electromagnetism/electromotive_force.jpg}	
		\caption[Electromotive force schematic concept]{Electromotive force schematic concept (source: OpenStax)}
	\end{figure}

	Generators vary depending on the energy source used and the method of conversion of the latter into electrical energy (ie, the nature of $\vec{E}_m$). We can then produce electrical energy from a battery (chemical energy), an electrostatic generator (mechanical energy), a dynamo (mechanical energy) a solar cell (radiation energy) or thermocouple (heat energy).
	
	The total power $P$ to be provided in steady state is therefore:
	
	where:
	
	is the total EMF of the circuit\label{total emf of the circuit}. The integral being on the whole circuit, the total EMF is the sum of the EMF present along the circuit (if any). If they are located in dipoles, the expression becomes:
	
	where the $e_k$ are the algebraic values of the different EMF:
	\begin{itemize}
		\item $e_k>0$ corresponds to "\NewTerm{generators}\index{electric generators}" devices (production of electrical energy)

		\item $e_k<0$ corresponds to "\NewTerm{receptors}\index{receptors}" devices (consumption of electrical energy)
	\end{itemize}
	We also have for the electric power:
	
	and for the Joule (resistive) Power:
	
	A motor converts electrical energy into mechanical energy and therefore corresponds to a EMF receptor, we also say that it has a "counter-electromotive force" or CEMF.
	
	\subsubsection{Faraday's law of induction}
	Now that we have proved the necessity of the electromotive force, we will be able to prove the origin of the "\NewTerm{Faraday's law}\index{Faraday's law}" and also of the "\NewTerm{Lenz's law}\index{Lenz's law}" that we had used in the section of Electrodynamics to prove the third Maxwell equation. 

	The Determination of Faraday's law will also allow us to define the concept of inductance and study its properties.

	Let us do the same approach as did Faraday and ask ourselves the following question: How do we create an electric current?

	An electric current is a moving set of electric charges in a conductive material. These electric charges are moved thanks to a difference of potential that is maintained by an electromotive force. Thus, a battery by converting the chemical energy during a time $\mathrm{d}t$ provides a power $P$ modifying the kinetic energy of the $\mathrm{d}Q$ charge carriers producing then an electric current $I$.

	Given $P_q$ the power required to communicate a speed $\vec{v}$ to an electric charge particle $q$. Knowing that in a conductor, there is $n$ charge carriers per unit volume, the total power $P_q$ to be provided by the (ideal) generator is then (see above):
	
	We therefore put that the ideal electromotive force of a circuit is:
	
	However, the Coulomb force is unable to produce an electromotive force as we have proved earlier. To create a direct (continuous) electric current in a closed circuit, we therefore need an electromotive field which circulation along the circuit is not zero. The Faraday's experiment thus shows that it is the existence of the magnetic field that allows the creation of a current (!!!!). This means that the Lorentz force must be responsible for the creation of an electromotive force, that is to say:
	
	Therefore:
	
	The properties of the vector cross product (\SeeChapter{see section Vector Calculus \pageref{cross product}}) giving us:
	
	We can then write:
	
	A small remark is essential at this level. If $\vec{v}$ is the velocity vector of the charges $q$ it cannot be the one which is collinear with $\mathrm{d}\vec{l}$ because otherwise we would have:
	
	and therefore $e$ would be zero and this is not possible because it would contradict all the developments made so far! In fact, $\vec{v}$ is the speed of the whole circuit which brings with it all the charges at the same speed!
	
	Thus, during a time $\mathrm{d}t$, the circuit moves by a distance:
	
	vector that is perpendicular to $\mathrm{d}\vec{l}$. Since then:
	
	is the surface (see the properties of the cross product in the section of Vector Calculus) described by the displacement of the element $\mathrm{d}\vec{l}$ on the distance $\mathrm{d}\vec{r}$ such that:
	
	We then have:
	
	We recognize here the expression of the flux (named sometimes "cutted flux") through the elementary surface $\mathrm{d}^2 \vec{S}$. Which brings us to write (there is a little bit of intuition - common sense - with the manipulation of differentials but hey that's also physics ...):
	
	We have just proved the "\NewTerm{Faraday's law of induction}\index{Faraday's law of induction}\label{Faraday's law of induction}" in the case of a rigid circuit immersed in a variable magnetic field. We have seen the expression of the cut flux appears naturally. In fact, the only thing that matters is the existence of movement of the overall or a part of the circuit (review the proof to be convinced). Thus, the expression of the induced EMF:
	
	remains valid for a circuit deformed and / or displaced in a static magnetic field. This proof was made from the Lorentz force and is therefore a priori independent of the chosen reference system!
	
	Phenomenologically this can be summarized into the following figure:
	\begin{figure}[H]
		\centering
		\includegraphics[scale=0.7]{img/electromagnetism/faraday_induction_law.jpg}
	\end{figure}
	\begin{enumerate}
		\item[(a)] When a magnet is moved toward a loop of wire connected to a sensitive ammeter, the ammeter deflects as shown, indicating that a current is induced in the loop. 
		
		\item[(b)] When the magnet is held stationary, there is no induced current in the loop, even when the magnet is inside the loop
		
		\item[(c)] When the magnet is moved away from the loop, the ammeter deflects in the opposite direction, indicating that the induced current is opposite that shown in part (a). Changing the direction of the magnet's motion changes the direction of the current induced by that motion.
	\end{enumerate}
	
	\paragraph{Lenz's law}\mbox{}\\\\
	The statement of "\NewTerm{Lenz's law}\index{Lenz's law}\label{Lenz's law}" is as follows: Induction produces effects which oppose the causes which gave rise to it.

	This law is, like the rule of maximum flux, already contained in the equations and brings nothing more, except an intuition of the physical phenomena. In this case, Lenz's law is only the expression of the sign "$-$" contained in Faraday's law.
	
	\begin{tcolorbox}[colframe=black,colback=white,sharp corners]
	\textbf{{\Large \ding{45}}Example:}\\\\
	If we approach a circuit of the north pole of a magnet, the flux increases and therefore the induced EMF is negative. The induced current will then be negative and will itself produce an induced magnetic field opposite to that of the magnet. Two consequences:
	\begin{itemize}
		\item The increase in flux through the circuit is lessened.
		
		\item A negative Laplace force appears (\SeeChapter{see section Magnetostatics page \pageref{Laplace law}}) $\vec{F}=I\mathrm{d}\vec{l}\times\vec{B}$, opposing to the approach of the magnet.
	\end{itemize}
	\end{tcolorbox}
	This sign "$-$" in Faraday's law (Lenz's law) describes the fact that under normal conditions, there is no possible runaway (example: current only increasing).
	
	This is the reason why the Lenz's law is often named the "\NewTerm{Lenz-Faraday law}\index{Lenz-Faraday law}".
	
	\subsubsection{Inductance}
	So we have:
	
	But the Biot-Savart's law gives us (\SeeChapter{see section Magnetostatics page \pageref{biot savart law}}):
	
	Therefore:
	
	which we historically write in condensed form as follows:
	
	where:
	
	is named the "\NewTerm{self-induction coefficient}\index{self-induction coefficient}" or "\NewTerm{self-inductance}", or simply "\NewTerm{self}" or commonly "\NewTerm{inductance}\index{inductance}", expressed in "Henry" [H]. It only depends on the geometric properties of the circuit and is necessarily positive.
	
	Explicitly this unit can be written in various forms but the most common one are:
	
	
	With the laws that we have stated so far, we are able to study some variable regimes (also named "transcient regimes"). Indeed, all reasoning based on the notion of a field (electric or magnetic) constant over time can easily be applied to variable physical systems (time-dependent fields), provided that this variability takes place on long times scales  compared to the characteristic adjustment time of the field in the corresponding materials. 

	Here is a concrete case:

	Most laws of magnetostatics assume a permanent current, that is, the same throughout the circuit. When we close a switch, an electromagnetic signal propagates throughout the circuit and this is how a permanent current can be established: it takes a time of the order of $l / c$ where $l$ is the length of the circuit and $c$ the speed of light. If we now have a sinusoidal voltage generator with period $T$ (this is just an example ... taken randomly ...), then we can still use the relations deduced from magnetostatic if:
	
	Thus, although the current is variable, the creation of a magnetic field will obey the Biot-Savart law as long as the above criterion remains satisfied. This type of variable regime is named "\NewTerm{quasi static regime}" in the sense that it is transient.
	
	So, since we have:
	
	we then have if and only if the current is variable in the circuit
	
	$L$ being constant for a rigid circuit. The self therefore creates an electromotive force opposite to that generated by the current at its terminals. This electromotive force therefore has an opposite direction to that of the electric generator.
	
	In the case where an inductor is alone in series with a sinusoidal alternating current generator (fairly typical case in the industrial world of the 19th and 20th century), then the conservation of the field will give:
	
	And therefore explicitly in the typical case of a sine regime:
	
	And therefore:
	
	To get the current, we integrate:
	
	So in this special case, the current is behind the voltage (or the voltage is ahead of the current) by a quarter of a period. It is also customary to write:
	
	Therefore:
	
	where by analogy with pure resistance, the expression:
	
	is named "\NewTerm{inductive reactance}\index{inductive reactance}".
	
	\begin{tcolorbox}[title=Remark,colframe=black,arc=10pt]
	We see well in the relation obtained above that in stationary regime, if the current is constant, then the electromotive force is zero and the self then behaves like a simple equipotential!
	\end{tcolorbox}
	
	We must now give an important and simple example of Lenz's law at the same time by applying it to the calculation of the inductance of a solenoid of radius $r$ (the inductance of a toric solenoid with circular section having already been treated in the section of Magnetostatics). We proved in the section of Magnetostatics that the magnetic field in a solenoid was given by:
	
	where as a reminder $N$ is the number of turns and $l$ is the length of the solenoid. We saw above that Faraday's law of induction was given by:
	
	and in the case of a turn we will travel $N$ times the path of the integral. It then comes:
	
	We saw above that the flux of the magnetic field was given by (if the field is perpendicular to the crossed surface):
	
	Therefore:
	
	\begin{tcolorbox}[title=Remark,colframe=black,arc=10pt]
	Caution!! The flux in a solenoid is not equal to the flux in a turn multiplied by the number of turns of the solenoid!
	\end{tcolorbox}
	The rate of change of the magnetic flux is found by derivation, that is:
	
	Or in the case of circular turns:
	
	The electromotive force generated is then:
	
	and therefore by correspondence:
	
	Now calculate the instantaneous power received by a coil. We have proved above that we still have in our case study and if we modelize the inductance as a non-ideal dipole:
	
	where the lowercase letters indicate as always that we are in a non-constant regime:
	
	At the contrary to the development which we had made in the section of Electrostatics for the same calculation with regard to the capacitor, we have not neglected here the dissipation of energy by Joule effect. But the reader must be aware that in most cases this term is also neglected!

	So by integration in a given time interval from $0$ to $t$ we have for the second term:
	
	This energy is therefore always positive and is stored in electrostatic form in the inductance.
	
	Or explicitly in the case of a circular solenoid:
	
	When $i$ decreases, the coil restores this energy. So we cannot store energy in an isolated coil unlike a capacitor.
	
	Under a sinusoidal regime, the average power will be zero. We can generalize this by admitting that a perfect inductance does not dissipate any power by the Joule effect.
	
	Let us now determine the inductance of a coaxial structure (similarly as we did of the capacitance of such a cable at page \pageref{capacitor}) as shown in the figure below:
	\begin{figure}[H]
		\centering
		\includegraphics[width=0.3\linewidth]{img/electromagnetism/inductance_coaxial_cable.jpg}
		\caption{Section of a long coaxial cable}
	\end{figure}
	where the inner and outer conductors carry equal currents in opposite directions.

 	The inductance of this structure is of interest for a number of reasons – in particular, for determining the characteristic impedance of coaxial transmission line.
 	
 	We must find the magnetic flux through the light gold rectangle in the figure above. Ampère’s law tells us that the magnetic field in the region between the conductors is due to the inner conductor alone and that its magnitude is:
 	
 	where $r$ is measured from the common center of the cylinders. A sample circular field line is shown in the figure above, along with a field vector tangent to the field line.
 	
 	The magnetic field is perpendicular to the light gold rectangle of length $l$, and width $R_2-R_1$, the cross section of interest.

	Because the magnetic field varies with radial position across this rectangle, we must use calculus to find the total magnetic flux.
	
	Divide the light gold rectangle into strips of width $\mathrm{d}r$ such as the darker strip in the figure above. We evaluate the magnetic flux through such a strip:
	
	Substitute for the magnetic field and integrate over the entire light gold rectangle give us:
	
	Using the definition of inductance we then get:
	
	and therefore we can also define the "\NewTerm{coaxial inductance per unit length}\index{coaxial inductance per unit length}\label{coaxial inductance per unit length}":
	
	
	\pagebreak
	\subsection{Skin effect}
	The "\NewTerm{skin effect}\index{skin effect}" or "\NewTerm{pellicular effect}\index{pellicular effect}" (or, more rarely, "\NewTerm{Kelvin effect}\index{Kelvin effect}") is an electromagnetic phenomenon which makes that at high frequency a current tends to circulate only at the surface of the conductors. This phenomenon of electromagnetic origin exists for all the conductors traversed by alternating currents. It causes the current density to decrease as one moves away from the periphery of the conductor. This results in an increase in the resistance of the conductor.

	This effect can be used to lighten the weight of high frequency transmission lines by using tubular conductors, or even pipes, without (too much) current loss. It is also used in the electromagnetic shielding of the coaxial wires by surrounding them with a thin metal case which keeps the currents induced by the high ambient frequencies on the outside of the cable.
	
	What we would now like to determine is the attenuation of the electric field (or an attenuation coefficient) in the material of a solid cylindrical conductive cable of conductivity $\sigma$ and permeability $\varepsilon$:
	\begin{figure}[H]
		\centering
		\includegraphics[scale=1]{img/electromagnetism/skin_effect.jpg}
		\caption[Configuration study for skin effect]{Configuration study for skin effect (source: Introductory Electromagnetics, Z. and B.D. Popovic}
	\end{figure}
	and assuming that the current density vector is parallel to the boundary surface, and that is has a single component, for example, $\vec{J}=J_z\vec{e}_z$, depending on the coordinate $y$ (the distance from the interface) only. We wish to determine the distribution of current in the conducting half-space.
	
	To do this, we take the fourth Maxwell's equation in the form given previously:
	
	and we assume to work with a conductor which has no capacitive effect (ie no displacement current), contrary to the general case demonstrated in the Electrodynamics section, the latter relation then reduces to:
	
	and if we associate it with Maxwell's third equation (\SeeChapter{see section Electrodynamics page \pageref{third maxwell equation}}), and assuming a harmonic excitation, we have for recall:
	
	Therefore, from that latter:
	
	So we have so far:	
	
	We assumed the current density vector has only a $z$ component, which depends only on $y$. From the Biot-Savart law and symmetry it therefore follows that there is only an $x$ component of the vector $\vec{B}$ (the reader can request we put the details of the section of Magnetism here again if necessary). According to the expression for the curl in a rectangular coordinates system, the both previous relations become:
	
	where we use ordinary derivatives (not partial) because $j_z$ and $B_x$ depend only on $y$.
	From the both relations above we can eliminate $B_x$ to get an equation in $j_z$:
	
	Let us place ourselves in the important case of a harmonic regime
	
	and let us use temporarily the phasors notation:
	
	We then have:
	
	By injecting this into the previous differential equation and simplifying, we then get:
	
	Thus:
	
	and therefore:
	
	remembering (\SeeChapter{see section Numbers page \pageref{complex numbers}}) that:
	
	it comes:
	
	Therefore:
	
	Thus:
	
	We must reject for physical reasons (conservation of energy) the solution:
	
	Therefore it remains:
	
	that physicists write:
	
	because the units of $\delta$ are meters (parameter that is zero if the electric field is constant) and is assimilated to the "\NewTerm{attenuation coefficient}" that we had set ourselves to determine at the beginning:
	
	For a copper conductor, we have according to Wikipedia the following values:
	\begin{table}[H]
		\centering
		\begin{tabular}{|l|c|}
		\hline
		\rowcolor[HTML]{9B9B9B} 
		\multicolumn{1}{|c|}{\cellcolor[HTML]{9B9B9B}\textbf{Frequency}} & \textbf{$\pmb{\delta}$ [mm]} \\ \hline
		$50$ [Hz] & $9.38$ \\ \hline
		$60$ [Hz] & $8.57$ \\ \hline
		$10$ [kHz] & $0.66$ \\ \hline
		$100$ [kHz] & $0.21$ \\ \hline
		$1$ [MHz] & $0.066$ \\ \hline
		$10$ [MHz] & $0.021$ \\ \hline
		\end{tabular}
		\caption[Some attenuation skin effect factor for Copper]{Somme attenuation skin effect factor for Copper (source: Wikipedia)}
	\end{table}

	\pagebreak
	\subsection{Semiconductors}\label{semiconductors}
	The main defect of the Drude model seen above is to consider the electron as a classical particle. A set of such particles is obviously not subject to quantum distributions and therefore to an explicit relations of temperature.

	Moreover, if we observe our Drude model, it is difficult to say anything about resistivity as a function of temperature.

	In fact, we generally consider four dates at the source of the development of the semiconductor theory:
	\begin{itemize}
		\item In 1833, Michael Faraday reported the conductivity of a material that increases with temperature.

		\item In 1839, Antoine Becquerel discovers that under illumination an electrical tension appears at the junction of certain materials (and liquids). It is the photovoltaic effect, which will give birth much later (around 1950) to the solar cells.

		\item In 1873, Willoughby Smith shows that the conductivity of certain substances increases when illuminated. This is photo-conductivity.

		\item Finally, in 1874, Karl Ferdinand Braun discovers the phenomenon of electrical straightening when a metallic tip is deposited on certain conductors, that is to say that the electric current passes in one direction when the electric potential applied to the point is positive but not when it is negative!
	\end{itemize}
	Although these discoveries were totally misunderstood and especially not recognized as the different expressions of the same physical phenomenon (semi-conductivity), practical applications were immediate and led to the second industrial revolution, that of microelectronics!
	
	This type of difficulty (among many others...) largely disappears with the model of the free electron in a potential well, as imagined by Arnold Sommerfeld in 1928 (\SeeChapter{see section Wave Quantum Physics page \pageref{quantum potential well}}). In this model the electrons, subjected to the Pauli exclusion principle, follow the Fermi-Dirac energy distribution (\SeeChapter{see section Statistical Mechanics page \pageref{fermi dirac distribution}}), whereas in the Drude model they followed the Maxwell-Boltzmann energy distribution.

	There are two important results:
	\begin{itemize}
		\item Only a fraction of the electrons is likely to see its energy vary under the effect of an external action (temperature, electric field, etc.)

		\item Even at absolute zero, the kinetic energy of the electrons is not zero.
	\end{itemize}
	The Sommerfeld model provides a basis for the construction of more specific theories and is the basis of the field of "\NewTerm{solid physics}\index{solid physics}" according to some sources. It is therefore not a completed model dealing with a specific problem such as electrical conduction or thermoelectronic emission. This base is the energy distribution of the electrons, obtained by the product of two functions: the density of the states and the Fermi-Dirac distribution.
	
	As the years go by, we will complete the developments that will follow to finally try to have the whole detailed approach. Until then, the reader will have to be patient or to go search the information in other sources on the Internet...

	We will make abstraction of the concepts that are not absolutely necessary for the introduction of the model to present here only the essential which is sufficient for the engineer in his daily work.

	To begin the mathematical part of the semiconductor study, we will consider a crystal subjected to a potential difference. A conduction electron of the crystal will therefore be subjected, on the one hand, to an internal force $\vec{F}_i$ resulting from the crystalline field and on the other hand to a force of external origin $\vec{F}_e$ resulting from the electric field applied to the crystal.
	
	The model assumptions (hypothesis) of the model are:
	\begin{enumerate}
		\item[H1.] There is a high enough potential barrier on the metal surface that prevents electrons from leaving the material.

		\item[H2.] Inside the material, electrons are subjected to constant potential!

		\item[H3.] The electrons are independent (no interactions between them).

		\item[H4.] The electrons obey the laws of Quantum Physics and Classical Mechanics.

		\item[H5.] The electrons obey to the Maxwell's electrodynamics laws.

		\item[H6.] The energy bands form a continuous spectrum of energy levels.
	\end{enumerate}
	The first hypothesis is based on the following observation: electrons travelling in a metal do not cross, at room temperature at least, the surfaces limiting the sample.

	The second hypothesis appears rather brutal. It banish from the model the notion of the "structure of matter". It will be replaced in the model of the energy bands by a periodic potential that accounts for the influence of the positively charged nuclei. This hypothesis reflects the fact that electrons are considered as free in the potential well defined by the sample.

	The potential barrier has a finite width but infinite height, that is to say that the passage of the potential inside the material to the potential outside it is done on some inter-atomic distances. However, since the dimensions of the sample are in practice always very great with respect to an inter-atomic distance, the potential barrier can be considered as infinitely abrupt, which simplifies the calculations.
	\begin{tcolorbox}[title=Remark,colframe=black,arc=10pt]
	In order to simplify the calculations, we will assume that the electrons move in only one direction (that of the electric field), thus avoid to work with vectors.
	\end{tcolorbox}
	The equation of the dynamics is then written naturally for this electron:
	
	We then write (nothing prohibits us to do so) that the electron in the crystal responds to the stress of the external force $\vec{F_e}$ as a quasi-particle of mass $m_e$ in a vacuum:
	
	It is the study of the latter term which will interest us. For this purpose let us recall that in the detailed study of the propagation of the free electron in a vacuum, where we neglect the effects of its spin, we have shown that it must be described according to the Schrödinger equation by a wave packet (\SeeChapter{see section Wave Quantum Physics page \pageref{wave packet}}) centered on a state $k_0$, otherwise its energy would be infinite.

	However, we can ask ourselves ... what leads us to consider it as free...? Well it is experience that shows that when we apply a certain threshold potential, a current begins to appear in the semiconductors.

	We have proved (always in the context of the propagation of the free spin-free particle in the section of Wave Quantum Physics) that the wave packet can then be seen in its mathematical solution as a (free) plane wave moving at phase speed:
	
	which we will write for as following in order to simplify the notations of the next developments:
	
	However, in the crystal lattice, the phase velocity can vary, depending on the location of the electron in the lattice due to the geometrical shape of the potential in the crystal. We must therefore use the instantaneous phase velocity:
	
	Let us recall (\SeeChapter{see section Corpuscular Quantum Physics page \pageref{planck einstein relation}}) that we also have in general the total energy given by:
	
	Therefore we get:
	
	The term:
	
	is by no means simple in the case of a crystal (it is even a nightmare ...).

	Obviously for a free particle (\SeeChapter{see section Wave Quantum Physics page \pageref{wave velocity group traveling wave}}), let us recall that it is equal to:
	
	But for a particle in a potential field having a complex geometry, the energy $E$ begins to have an expression dependent of $k$ as a function of the zones which can become very complex (see the examples of the section Wave Quantum Physics). Hence the justification for the use of the derivative.

	The acceleration in the classical sense of this electron is then given by:
	
	We also have (\SeeChapter{see section Classical Mechanics page \pageref{work equation}}):
	
	Therefore:
	
	Hence:
	
	The derivative of $\vec{F}$ with respect to $\vec{k}$ in the preceding relation will be cancelled because the force is derived from the potential applied to the semiconductor only and not from the wave vector of the electron itself! We then have:
	
	Since here $E$ is only the total energy coming from the externally submitted potential, then the force $\vec{F}$ is the external force $\vec{F_e}$ generated by the application of this same potential. We then have:
	
	and:
	
	It then comes by equalization:
	
	Since the energy of the electron can have a complicated mathematical form according to the practical applications cases presented in the section of Wave Quantum Physics, let us express $E(\vec{k})$ in the form of a limited Taylor development (\SeeChapter{see section Sequences and Series page \pageref{usual maclaurin developments}}) of a function of three variables at the second order by dropping the terms of interactions and by not taking the terms of first degree:
	
	In fact, this rough but still acceptable approximation in many practical cases is due to the fact that experience shows that the energy surfaces as a function of $k$ approximate a parabolic shape in certain semiconductor crystals.
	\begin{figure}[H]
		\centering
		\includegraphics[scale=0.7]{img/electromagnetism/band_conduction_parabolic_approximation.jpg}
		\caption[]{Band structure of Silicium following different plane in the crystal. The blue rectangle shows a region of one of the conduction bands which may be described by a parabola approximately}
	\end{figure}
	In the conductors, the approximation of the preceding relation is taken only at the first term.

 	Another way of seeing it is that for a free electron, for recall, in one dimension, the dispersion curve\index{dispersion curve} (\SeeChapter{see section Wave Quantum Physics page \pageref{dispersion curve}}):
	
	which is indeed a parabola as a function of $k$. Indeed, if we take our Taylor development in one dimension it remains:
	
	and as we determined before that:
	
	It comes:
	
	If the electron is free, the dispersion curve imposes us to have (without the presence of a potential):
	
	which is then considered as the "\NewTerm{energy of the minimum}" $E_{\min}$. It then remains:
	
	and by taking $k_{x0}=0$ we fall back on the dispersion curve of a free particle (which therefore justifies the fact of having to choose $E(k_0)=0$ for a free electron):
	
	This shows that the approximation is not too false ... and justifies the fact that in some textbooks the previous relation (Taylor series) describes a so-named "\NewTerm{quasi-free particle}\index{quasi-free particle}".

	But let us come back to:
	
	And since the wave packet is centered around $k_0$, let us normalize it as being equal to $0$ (which is equivalent to centering the wave vector values). We then have:
	
	What is interesting with these developments is that we started from a free electron in the form of a wave packet and thanks to the Taylor development we find ourselves with an extremely simple expression of the energy of a quasi-free electron.

	It emerges that for a quasi-free electron, without interactions and without taking into account the effects of spin we have:
	
	We then notice a very sympathetic thing! This is that our quasi-free electron has a wave number which resembles in all points tot particle stuck in a potential well with rectilinear walls (see the proof in the section of Wave Quantum Physics).

	We now wish to calculate the density of states (in extenso of electrons) in the volume given by the corresponding rectangular well, by means of the expression of $k$ (not having directly that of $E$ as being too complex).

	We have proved in the section of Wave Quantum Physics that for the potential well with rectangular barriers that:
	
	if we imposed an integer half wavelength. If we impose an integer wavelength ("\NewTerm{Born-von Karman conditions}\index{Born-von Karman conditions}" so that after a translation of the periodic lattice of the crystal we fall back on the same properties) so that the solution is physically acceptable, then we have:
	
	Which obviously implies two times fewer states.

	By extension, for space, we then have in the three-dimensional case:
	
	with:
	
	and where $n_{x,y,z}=1,2,3,\ldots$.
	
	The result is very similar to that of the one-dimensional infinite rectangular potential well but now we have special boundary conditions in the purpose to have a correspondence with the experiment and three main quantum numbers instead of just one. Moreover, each combination of these three numbers corresponds to a different wave function (state). Moreover, these numbers are independent (no condition imposed).
	\begin{tcolorbox}[colframe=black,colback=white,sharp corners]
	\textbf{{\Large \ding{45}}Example:}\\\\
	Consider a solid metal cube of edge length $2$ [cm]. We want to know the value of the lowest energy level for an electron within the metal and what is the spacing between this level and the next energy level!\\
	
	The lowest energy level corresponds to the quantum numbers $n_x=n_y=n_z=1$. From the above relation, the energy of this level is:
	
	The next-higher energy level is reached by increasing any one of the three quantum numbers by $1$. Hence, there are actually three quantum states with the same energy. Suppose we increase $n_x$ by $1$. Then the energy becomes:
	\end{tcolorbox}
	\begin{tcolorbox}[colframe=black,colback=white,sharp corners]
	
	The energy spacing between the lowest energy state and the next-highest energy state is therefore:
	
	his is a very small energy difference. Compare this value to the average kinetic energy of a particle, as the product $kT$ is in comparison $1,000$ times greater than the previously calculated energy spacing.
	\end{tcolorbox}
	We then have for the first level where all $n$ are unitary:
	
	If we accept to simplify that the well has edges of equal length (cubic crystal lattice semiconductor), we then have:
	
	Let us represent the space $k$ for such a cubic lattice and for different multiples of $n_x$, $n_y$, $n_z$:
	\begin{figure}[H]
		\centering
		\includegraphics{img/electromagnetism/cubic_crystal_lattice.jpg}
		\caption{$k$-space for a cubic crystal lattice}
	\end{figure}
	Therefore, all quantized states can take only values space from $2\pi/L$ in the space of $k$, which means that by elementary volume there is only one possible wave vector and therefore only one associated state. Indeed, the reader can draw a picture over the figure above if he wants and he will see (!) but do not rely on the big blackheads that are there only to show the ends of the elementary volumes and which do not all correspond to possible states!
	\begin{tcolorbox}[title=Remark,colframe=black,arc=10pt]
	The sphere of radius $k$ containing the levels with one electron is sometimes referred to as the "\NewTerm{Fermi sphere}\index{Fermi sphere}". The radius value is then denoted $k_F$ and named the "\NewTerm{Fermi wave vector}\index{Fermi wave vector}". The surface of the Fermi sphere, which separates the occupied levels from those which are not occupied as we shall see later, is named the "\NewTerm{Fermi surface}".
	\end{tcolorbox}	
	Thus, in a spherical volume with radius $k$ of the $k$-space. We have a precise number (upper limit) of elementary volumes (states), remembering that $V=L^3$, given by:
	
	where in literature it is customary (tradition) to retain only the form of the second equality in the developments. This relation has been a useful to us for recall in the section of Thermodynamics to determine the Debye-Einstein model of the constant-volume (isochoric) thermal capacity of crystalline solids (see page \pageref{calorific capacities})!

	The density of modes in a volume $V$ will then be given by (relation used in the section of Thermodynamics to express the calorific capacity at constant volume of solids):
	
	In facts, due to the presence of potential bonds between the atoms of the crystal, the sphere is not perfect and has like giant pockmarks/photholes on what is otherwise a smooth surface:
	\begin{figure}[H]
		\centering
		\includegraphics[scale=0.8]{img/electromagnetism/fermi_surface_cupper.jpg}
		\caption{Fermi surface of Cupper}
	\end{figure}
	or for real, here is a (supposed) Fermi surface and electron momentum density of Copper in the reduced zone schema measured with 2D Angular Correlation of Electron Positron Annihilation Radiation method ("supposed" as I was not able to find the original article with the picture):
	\begin{figure}[H]
		\centering
		\includegraphics[scale=0.65]{img/electromagnetism/fermi_surface_cupper_acar.jpg}
		\caption[Fermi surface of Cupper with 2D ACAR method]{Fermi surface of Cupper with 2D ACAR method (source: Wikipedia)}
	\end{figure}
	Now considering the spin (yes indeed why not do it...?!) we multiply by $2$ since there are two spin states possible by state for the electron\label{density of states factor} (remembering that $V=L^3$):
	
	(relation that we will see again in the section of Nuclear Physics chapter during our study of the liquid drop nucleus model at page \pageref{liquid drop model}) and by injecting in it:
	
	we then have:
	
	The volume density of (quasi-free) states will be obtained by deriving this last relation by the volume:
	
	And if we want the density of (quasi-)free states (of vibrations) per unit of energy and volume, we will have to derive also with respect to the energy:
	
	which gives:
	
	This result does not depend on the volume, it is unchanged when the latter tends towards infinity! So it is valid for any point of the semiconductor crystal if this one is perfect and without bonds between atoms...
	
	What we also find sometimes in the following (somewhat unfortunate ...) forms in some textbooks:
	
	and there are also those who take into account the spin into account only much more later ... which gives a form identical to that of the previous three relations but divided by a factor $2$.and there are also those who take into account the spin into account only much more later ... which gives a form identical to that of the previous three relations but divided by a factor $2$.
	
	The difference $E-E_0$ as we will see further below is denoted $E_F$, and for obvious reasons related to Statistical Mechanics that we will see further below named the "Fermi Energy". Therefore notice that we have using the previous results:
	
	After rearranging we get immediately:
	
	Therefore:
	
	Fermi energies for selected materials are listed in the following table:
	\begin{table}[H]
		\centering
		\begin{tabular}{|l|c|c|}
		\hline
		\rowcolor[HTML]{9B9B9B} 
		\textbf{Element} & \multicolumn{1}{l|}{\cellcolor[HTML]{9B9B9B}\textbf{Conduction band electron density}} & \multicolumn{1}{l|}{\cellcolor[HTML]{9B9B9B}\textbf{Free-Electron Model Fermi Energy}} \\ \hline
		$\mathrm{Al}$ & $18.1\cdot 10^{28}\;[\text{m}^{-3}]$ & $11.7\;[\text{eV}]$ \\ \hline
		$\mathrm{Ba}$ & $3.15\cdot 10^{28}\;[\text{m}^{-3}]$ & $3.64\;[\text{eV}]$ \\ \hline
		$\mathrm{Cu}$ & $8.47\cdot 10^{28}\;[\text{m}^{-3}]$ & $7.00\;[\text{eV}]$ \\ \hline
		$\mathrm{Au}$ & $5.90\cdot 10^{28}\;[\text{m}^{-3}]$ & $5.53\;[\text{eV}]$ \\ \hline
		$\mathrm{Fe}$ & $17.0\cdot 10^{28}\;[\text{m}^{-3}]$ & $11.1\;[\text{eV}]$ \\ \hline
		$\mathrm{Ag}$ & $5.86\cdot 10^{28}\;[\text{m}^{-3}]$ & $5.49\;[\text{eV}]$ \\ \hline
		\end{tabular}
		\caption{Fermi energy for some materials}
	\end{table}
	We can obviously associate the Fermi temperature (\SeeChapter{see section Statistical Mechanics page \pageref{fermi temperature}}) with the Fermi Energy as $E=kT$. Therefore:
	
	\begin{tcolorbox}[colframe=black,colback=white,sharp corners]
	\textbf{{\Large \ding{45}}Example:}\\\\
	The Fermi temperature associated to the Fermi energy of silver is:
	
	which is much higher than room temperature and also the typical melting point ($\sim 10^3$ [K]) of a metal.
	\end{tcolorbox}
	
	\subsubsection{Non-degenerated statistic density of negative electric charge carriers}\label{non degenerated statistic density of negative electric charge carriers}
	In short, however this relation has an issue (one more...)! Indeed, we have seen in the section of Statistical Mechanics in our study of Quantum Statistics that in a system where even the energy spectrum is considered continuous it is impossible not to take into account the degeneration of the different levels of energy. We have then demonstrated that for a population of fermions, at a given energy (or temperature) the percentage of degenerate levels occupied is given by the Fermi-Dirac function:
	
	and that function therefore returns a value between $0$ and $1$.

This function therefore gives for a fixed temperature T the probability that an electron occupies a state of energy $E$.

Hence our relation $D(E)$ overestimates the real density value of (quasi-)free occupied states for a given energy (or temperature). In order to obtain a better approximation, we logically write the volumetric density of (quasi-free) states per unit of energy by:
	
	However, in practice, we will try to calculate the volume density of (quasi-free) states in a spectrum (interval) of energy. It then comes with the correction added previously:
	
	Thus:
	
	It then immediately follows that the volumetric density of (quasi-free) states at a given temperature (normal temperature conditions for civilian applications) taking into account all possible states (continuous levels) of energy is then given by:
	
	Take $E_0$ as a lower boundary avoids us, as we shall see explicitly a further below, to find ourselves with a negative root ... which would be very a priori troublesome!

	Moreover, we can, without any appreciable error, postpone the limit of the integral to the infinity because $f_{\text{FD}}\rightarrow 0$ when $E$ is large.

	Unfortunately, this (improper) integral is generally not analytically soluble. We will have to use approximations.

	We will start by making the assumption that we are in the classical regime of electron gas. That is, we have:
	
	which implies:
	
	Therefore, we also have the approximation:
	
	In other words, the energy $E$ must be much higher than the chemical potential $\mu$ (assimilated often unfortunately to my knowledge wrongly in the literature on semiconductors to the Fermi level $E_F$). Physicists then note this energy $E_C$ to distinguish it and name it "\NewTerm{minimal energy of the conduction band}" (which corresponds to the minimum energy of a quasi-free electron to satisfy this condition).

	Therefore, we also change the notation for the charge density:
	
	\begin{tcolorbox}[title=Remark,colframe=black,arc=10pt]
	Unfortunately, as mentioned in the previous paragraph (!) in many quality textbooks on semiconductors, the chemical potential $\mu$, which is a purely thermodynamic notion implying a hypothesis of interactions, is replaced by the concept of Fermi energy $E_F$ and however this is not the same thing! The two energies coincide only in the case where the temperature $T$ is equal to zero!\\

	So we must consider the term "\NewTerm{Fermi level}\index{Fermi level}" as being nothing but a synonym for "\NewTerm{chemical potential}\index{chemical potential}" in the context of semiconductors.
	\end{tcolorbox}
	We then have:
	
	where $f_{\text{MB}}$ is the Maxwell-Boltzmann distribution (\SeeChapter{see section Statistical Mechanics page \pageref{maxwell distribution}}) given for recall by:
	
	and thus corresponds well to a non-quantum behaviour (ie a non-degenerate electron gas!) because when:
	
	we have:
	
	and therefore the energy states of are far from being all occupied by electrons (there is therefore no degeneration).

	We are therefore well in a situation where Classical Physics predominates over Quantum Physics. This is why in this approximation (of Maxwell-Boltzmann type) we say that we are dealing with a "\NewTerm{non-degenerate semiconductor}\index{non-degenerate semiconductor}" because the electrons are not all stacked into the lowest available levels.

	To continue, we make a change of variable by putting:
	
	hence:
	
	Therefore it comes:
	
	We do an integration by parts:
	
	We then make a change of variable by putting:
	
	which gives:
	
	We have already calculated this integral in the Statistics section. It comes:
	
	We then have finally\label{maxwell-boltzmann density states}:
	
	Where, for recall, $m_e$ is the mass of the quasi-particle (and not the mass of the electron for recall!). So after integration everything happens as if all the electrons were concentrated on the level of energy $E_C$ with a number of available places corresponding to:
	
	The relation $\rho_C(E)$ is traditionally written (and in a somewhat unfortunate way ... because it is not easy to remember that it is a density):
	
	or also:
	
	where we have approximately at ambient temperature the following values of (quasi-)free states for Silicon:
	
	and for the Germanium:
	
	while there is a density of about $4.5\cdot 10^{22}$ atoms by cubic centimetres and about $10^{24}$ electron by cubic centimetre for these two elements.
	\begin{figure}[H]
		\centering
		\includegraphics[scale=0.6]{img/electromagnetism/silicum_vs_germanium.jpg}
		\caption[Silicium vs Germanium band and crystal structure]{Silicium vs Germanium band and crystal structure (source: ?)}
	\end{figure}
	This means that there is thus a ratio of: $10^{19}/10^{24}=10^{-5}$ between the total electron density and the number of quasi-free electrons.

	We also notice that this theoretical model does not take into account the electronic structure (atomic number) of the studied material.

	Thus, we see that the variations of quasi-free electron densities as a function of temperature (in the temperature validity range ...) are essentially of increasing or decreasing exponential type.

	From the density of the free electrons (caution! it must be remembered that these are only the quasi-free electrons that are wandering in our mathematical equations so far...) in the semiconductor crystal, we can deduce the Fermi energy level (more rigorously it is the chemical potential!):
	
	hence:
	
	and since $N_{n,T}\ge N_n$ we always have because of the logarithm which is negative, the energy of Fermi (more rigorously it is the chemical potential!) which is less than or equal to the energy of the quasi-free electrons:
	
	or in other words, (quasi-)free electrons have an energy higher than the Fermi energy (chemical potential...), which is in line with the non-degenerate gas approximation made earlier above. This gives a condition of major importance for the negative carriers to be the generators of the conduction in the material.

	Thus, when we place ourselves at a temperature different from absolute zero, the electronic states are not all degenerate: there is spreading of occupied states in the neighbourhood of what constitutes by definition the Fermi energy level (\SeeChapter{see sections of Statistical Mechanics page \pageref{fermi energy level semi conductors} and Wave Quantum Physics page \pageref{fermi energy}}), effect that is as increase as the temperature is high.
	
	\subsubsection{Non-degenerated statistic density of positive electric charge carriers}\label{non-degenerated statistic density of positive electric charge carriers}
	First of all, we must know that in the present state of our knowledge, "\NewTerm{holes}\index{holes}" do not emerge from equations, but are an empirical construction that makes it possible to match theory and experience (positive charges of the Hall effect for example). It is therefore an artifice to make a simple theory of a question that seems rigorously nowadays not manageable by Quantum Physics.

	Personally, I consider the holes in the same way as the Lagrange points in astronomy: Even if there are no bodies at these points of Lagrange this does not avoid a satellite from orbiting around them (possibility that we have not proved in the section of Astronomy) as if there was a mass even if the orbit is quasi-stable! Moreover, experiments would have shown in the early 2000s that Lagrange points appear at the level of the atom under certain ideal and simplified conditions!

	That said, it must be remembered that a hole is not a missing electron! It is an idiocy (in my opinion ...) that we see in some specialized works.

	At the risk of being repeated a little often, let us recall that for a fixed temperature $T$ the probability that an electron occupies a state of energy $E$ is given by:
	
	What bring us in order to get a better approximation, we logically had written the volume density of occupied states per unit of energy:
	
	This finally led to the following relation of the volumetric density of negative charge states where the presence of a mass in the relation indicates that the occupied states are by quasi-particles such as:
	
	But what about the probability that an electron does not occupy for a fixed temperature $T$ a state of energy $E$ and trivially given by the difference:
	
	where the $n$ in index is there to indicate that the distribution concerns the "negative" carriers (distribution given as we have rightly proved earlier by the Maxwell-Boltzmann distribution which follows from an approximation of the Fermi-Dirac law).
	
	Well! We shall see that the equations lead us to the possibility to associate also to these unoccupied states a density of states with a given effective mass. We shall also see later that it will be possible to associate to these unoccupied states a positive and equal electric charge to that of the electron, hence the $p$ in index in the preceding relation and signifying "positive".

	We therefore have for these positive carriers:
	
	let us now make an approximation similar to that used for the negative carriers, that is to say:
	
	to impose a semi-classical regime and therefore the states of energy are by far not all occupied by the holes (there is therefore no degeneration).

	This restriction requires:
	
	Either written in the same way as for negative carriers:
	
	Unlike negative carriers, this imposes:
	
	in other words the energy must be much lower than the Fermi level (chemical potential). Physicists then write this energy $E_V$ to distinguish it and name it "\NewTerm{maximum energy of the valence band}" (which corresponds to the maximum energy of a quasi-free hole to satisfy this condition).
	\begin{tcolorbox}[title=Remark,colframe=black,arc=10pt]
	The reader may observe that the conditions mentioned earlier above also impose that $E$ is either very small in absolute value or negative. Which gives us already a track for the integration terminals to come...
	\end{tcolorbox}	
	We then have:
	
	Therefore, we also have the approximation:
	
	We are therefore well in a situation where classical physics predominates over quantum physics. This is why in this approximation we say that we are dealing with a "non-degenerate semiconductor" because the holes are not crammed into the highest levels available.

	We have then:
	
	where the reader will have observed that the integration terminals have been chosen according to the remark we had just made earlier and that the terms in the square root have been swapped so as not to have any negative value.

	To continue, we make a change of variable by putting:
	
	hence:
	
	Therefore it comes:
	
	We do an integration by parts:
	
	we then make a change of variable by putting:
	
	which gives:
	
	We have already calculated this integral in the section Statistics. It comes (since the function is even we use the property demonstrated in the section of Differential and Integral Calculus):
	
	We then have finally:
	
	Where, for recall, $m_e$ is the mass of the quasi-particle (and not the mass of the hole for recall!). So after integration everything happens as if all the holes were concentrated on the level of energy $E_V$ with a number of available places corresponding to:
	
	What we write traditionally (and in a somewhat unfortunate way ... because it is not easy to remember that it is a density):
	
	or:
	
	where we have approximately at ambient temperature the following values of (quasi-)free states for Silicon:
	
	and for Germanium:
	
	
	\subsubsection{Energy bands (conduction band, band gap, valence band)}
	The previous developments for negative and positive carriers have shown us that, in the approximation of a non-degenerate fermion gas, the energy of the negative carriers must be well above the Fermi level (chemical potential) and positive carrier energy well below.

	It is therefore as if there was a forbidden energy interval or neither electrons nor holes have the right to be located! This energy interval is traditionally referred to as "\NewTerm{forbidden energy band}\index{forbidden energy band}" or simply "\NewTerm{band gap}\index{band gap}" and abbreviated B.G. The forbidden energy interval is often named simple "\NewTerm{gap}\index{gap}" and is denoted $E_g$.

	Let us see this in a rough schematic form, taking care that this scheme is somewhat misleading because it gives the impression that the conduction or valence band occupies an entire block (furthermore "flat"...), whereas in reality the valence band is constituted by the last layer completely filled, the band of permitted energy which follows being named "\NewTerm{conduction band}\index{conduction band}".
	\begin{figure}[H]
		\centering
		\includegraphics[scale=1]{img/electromagnetism/band_structures_various_materials.jpg}
		\caption{Representation of band structures in different materials}
	\end{figure}
	Moreover, knowing that molecular chemistry makes it possible to demonstrate that structures are composed of multiple bands (as a function of the first and second quantum number), the following rigorous definitions are obtained:
	\begin{enumerate}
		\item[D1.] The "\NewTerm{conducting band}\index{conduction band}" (denoted sometimes "CB") of a solid structure is the lowest energy partially occupied or empty  band (other bands are above in energy terms but will only fill at high temperatures And exist only by a theoretical description when they are empty).

		\item[D2.] The "\NewTerm{valence band}\index{valence band}" (denoted sometimes "BV") of a solid structure is the band of highest energy that is saturated, that is to say where all the states of which are occupied (knowing that there may be below the valence band other multiple bands in energy terms and all saturated).
	\end{enumerate}
	We also have the traditional schematic association of the conduction and valence bands with the Fermi-Dirac function (which as already mentioned in all rigour should be the chemical potential at non-zero temperature!) represented in simplified form by:
	\begin{figure}[H]
		\centering
		\includegraphics[scale=1]{img/electromagnetism/band_structure_with_fermi_level.jpg}
		\caption{Association of band structure with Fermi-Dirac function}
	\end{figure}
	But in fact this representation, which we find almost everywhere in certain textbooks, is relatively erroneous ... since by making a semi-classical approximation by the Maxwell-Boltzmann distribution there is no longer question of rigorously representing the distribution in the form of a Fermi-Dirac distribution as the attentive reader will have noticed! This shows that we must always be useful with figures since the traditional representation of $f_{\text{FD}}$ in the semi-classical model would indicate that there would be occupied states in the forbidden band, whereas if we represented the Maxwell-Boltzmann distribution we see two distinct functions, above and below the forbidden band!
	
	And it must be remembered (!!!) that the above figure (even if it is quite false) represents conceptually a non-degenerate semiconductor following semi-classical approximations that we have made in our developments using the model of a non-degenerate gas (Maxwell-Boltzmann approximation) and that theoretically imposed:
	
	that many authors write (again it's unfortunate but it is so...):
	 
	So there is another possible definition of the non-degenerate semiconductor: this is where the Fermi level (the chemical potential!) lies in the forbidden band and this case corresponds to how the majority of the microelectronic components work.
	
	\begin{tcolorbox}[title=Remark,colframe=black,arc=10pt]
	Let us recall (\SeeChapter{see section Statistical Mechanics page \pageref{maxwell distribution}}) that the Maxwell-Boltzmann statistic was constructed on the assumption of the absence of interaction between the particles involved. Moreover, this statistic is constructed in the framework of Classical Mechanics and therefore applies only when the quantum effects are negligible, for example at high enough temperatures!
	\end{tcolorbox}
	Here are some experimental values for common semiconductors:
	\begin{table}[H]
		\centering
		\begin{tabular}{|l|c|c|}
		\hline
		\rowcolor[HTML]{9B9B9B} 
		\multicolumn{1}{|c|}{\cellcolor[HTML]{9B9B9B}\textbf{$\pmb{E_g}$ {[}eV{]}}} & \textbf{$\pmb{300}$ {[}K{]}} & \textbf{$\pmb{0}$ {[}K{]}} \\ \hline
		$\mathrm{C}$ & $5.47$ & $5.51$ \\ \hline
		$\mathrm{Ge}$ & $0.66$ & $0.75$ \\ \hline
		$\mathrm{Si}$ & $1.12$ & $1.16$ \\ \hline
		$\mathrm{GeAs}$ & $1.43$ & $1.53$ \\ \hline
		\end{tabular}
		\caption{Values of a few semiconductors gaps}
	\end{table}
	and visually and in 3D here is another figure representing de various idealized bands:
	\begin{figure}[H]
		\centering
		\includegraphics[scale=0.9]{img/electromagnetism/metal_insulator_transition.jpg}
		\caption[Metal–insulator transitions]{Metal–insulator transitions (source: Carbon conductor corrupted, authors: Michael S. Fuhrer and Shaffique Adam)}
	\end{figure}
	We understand then on the basis of these figures why the diamond, with a quasi-identical crystalline and atomic structure, is insulating while the silicon becomes conductive!

	What is interesting for researchers is to combine materials in order to play with the width of the equation according to the needs!

	Moreover, we can also conclude hastily ... that what differentiates insulators and semiconductors is the width of their forbidden band.

	It should also be noted that the energy required for an electron to pass from the valence band to the conduction band can be supplied by radiation. In the case of light absorption, the energy equation of a photon may be sufficient for this as long as:
	
	At low temperatures, such a process is capable of making the material conductive (low-temperature space telescope technology). This property is named "\NewTerm{photo-conductivity}\index{photo-conductivity}".

	Finally, let us recall the two relations obtained above:
	
	The product of these two densities possesses a very interesting property. We can observe that it is independent of the position of the Fermi level and named "\NewTerm{intrinsic density}\index{intrinsic density}":
	
	For example, some values of the square root of the intrinsic density at $300$ [K] are given in the table below:
	\begin{table}[H]
		\centering
		\begin{tabular}{|l|l|}
		\hline
		\rowcolor[HTML]{9B9B9B} 
		\textbf{Material} & $\pmb{\bar{n}\bar{p}}$ \\ \hline
		$\mathrm{Ge}$ & $2.5\cdot 10^{13}\;[\text{cm}^{-3}]$ \\ \hline
		$\mathrm{Si}$ & $1.6\cdot 10^{10}\;[\text{cm}^{-3}]$ \\ \hline
		$\mathrm{GaAs}$ & $1.1\cdot 10^{7}\;[\text{cm}^{-3}]$ \\ \hline
		\end{tabular}
		\caption[]{Some square root intrinsic density values}
	\end{table}
	We also notice that the critical density is strongly temperature dependent. These values of densities are obviously idealized, in reality these values are much lower due to imperfections (residual impurities, defects in crystallization, etc.) which locally disrupt the periodicity of the potential and, therefore, introduce energy levels which can be accessible to electrons. In contrast to the levels corresponding to the pure material, we will speak of "\NewTerm{extrinsic levels}\index{extrinsic levels}".
	
	\subsubsection{Ohm's law of Semiconductors}
	We have proved in the framework of the Drude model that the conductivity was given by:
	
	where $n$ is for recall the density of carriers in the material. We have also proved that the current is inversely proportional to the conductivity according to the relation:
	
	In the framework of the developments made above we have seen that the density $n$ of the carriers was given respectively by the following relations for a constant potential (model hypothesis):
	
	where the relative masses $m_e$ of the quasi-particle (negative carrier or positive carrier) are not necessarily equal! Thus, we have the resistance which can be approximated by a relation of the form:
	
	and we easily verify this dependence by graphically representing:
	
	that is $\ln(R)$ as a function of $1/T$ (the resistance therefore depends only on the theoretical temperature at constant voltage).
	
	The real complexity lies in the fact that many terms are dependent on temperature (Fermi level, mean free travel time, etc.) and the applied potential, which means that in reality the curves obtained are by no means according to the theory....!
	
	A numerical application shows that the carrier densities $N_n$ and $N_p$ therefore increase very quickly already from the ambient temperature! What is consistent with the experience with non-degenerate semiconductors because we will then have the conductivity which increases just as strongly which implies a rapid decrease in resistance!
	
	The high sensitivity of the conductivity of certain solids to temperature variations is the cause of many applications, both for conductive metals and semiconductors. This is what we name "\NewTerm{thermistors}\index{thermistors}".
	
	Finally, let us notice that in the case of Silicium, we have $E_g=1.12$ [eV] whereas the kinetic energy due to the thermal agitation (\SeeChapter{see section Continuum Mechanics page \pageref{virial theorem}}) is given at ambient temperature by:
	
	But, we have already seen that only the electrons whose energy was close to that of the Fermi level could participate in the conduction. Their kinetic energy then being:
	
	where $v_F$ is the "\NewTerm{Fermi Velocity}\index{Fermi Velocity}" (velocity of electron-wave in a conductor).
	
	By equalizing the last two relations:
	
	There is therefore a ratio of a factor of $30$ between the two energies, either by taking the square root, a ratio of $5$ between the velocities. Therefore we have:
	
	 But we have already seen, in our study of the Drude model, that the thermal velocity led us to a superior average distance of one order of magnitude (factor $10$) of the inter-atomic distances. And here we have a factor $5$ more !!!! Thus more than $50$ inter-atomic distances! The mean free path of an electron of conduction is therefore much greater than that which we had determined from the classical Drude model. Thus, the mean free path does not seem to be due to collisions with the ions of the network but it is due to the network's imperfections: structural defects, foreign atoms ...

	A perfect (pure) semiconductor, without imperfections, as we have theoretically treated it so far, is named an "\NewTerm{intrinsic semiconductor}\index{intrinsic semiconductor}": it therefore has no impurity and its electrical behaviour depends only on the Structure of the material. This behaviour corresponds to a perfect semiconductor, that is to say without structural defects or chemical impurities. A real semiconductor is never perfectly intrinsic, but can sometimes be as close as pure mono-crystalline silicon.

	In an intrinsic semiconductor, the charge carriers are created only by thermal excitation. The number of electrons in the conduction band is then equal to the number of holes in the valence band as shown by our theoretical model.

	It must be realized that these semiconductors actually do not, or very little, conduct the current, except if we carry them at high temperature.
	
	
	\pagebreak
	\subsection{Superconductivity}\label{superconductivity}
	"\NewTerm{Superconductivity}" is a phenomenon of exactly zero electrical resistance and expulsion of magnetic flux fields occurring in certain materials, named "superconductors", when cooled below a characteristic critical temperature. It was discovered by Dutch physicist Heike Kamerlingh Onnes on April 8, 1911, in Leiden (he get the Nobel Price for this discovery on with the element Mercury $\mathrm{Hg}$). 
	
	Like ferromagnetism and atomic spectral lines, superconductivity is a quantum mechanical phenomenon. It is characterized by the Meissner effect, the complete ejection of magnetic field lines from the interior of the superconductor as it transitions into the superconducting state. The occurrence of the Meissner effect indicates that superconductivity cannot be understood simply as the idealization of perfect conductivity in classical physics.
	\begin{figure}[H]
		\centering
		\includegraphics[scale=1]{img/electromagnetism/superconductor_mercury.jpg}
	\end{figure}
	Superconductors are also able to maintain a current with no applied voltage whatsoever, a property exploited in superconducting electromagnets such as those found in MRI machines. Experiments have demonstrated that currents in superconducting coils can persist for years without any measurable degradation. Experimental evidence points to a current lifetime of at least 100,000 years. Theoretical estimates for the lifetime of a persistent current can exceed the estimated lifetime of the universe, depending on the wire geometry and the temperature.
	
	\subsubsection{Meissner effect (second London's equation)}
	The "\NewTerm{Meissner effect}\index{Meissner effect}" is the expulsion of a magnetic field from a superconductor during its transition to the superconducting state. The German physicists Walther Meissner and Robert Ochsenfeld discovered this phenomenon in 1933 by measuring the magnetic field distribution outside superconducting tin and lead samples. The samples, in the presence of an applied magnetic field, were cooled below their superconducting transition temperature. Below the transition temperature the samples cancelled nearly all interior magnetic fields. They detected this effect only indirectly because the magnetic flux is conserved by a superconductor: when the interior field decreases, the exterior field increases. The experiment demonstrated for the first time that superconductors were more than just perfect conductors and provided a uniquely defining property of the superconductor state.
	\begin{figure}[H]
		\centering
		\includegraphics[scale=0.57]{img/electromagnetism/superconductor_cern.jpg}
		\caption[Electric cables for accelerators at CERN]{Electric cables for accelerators at CERN (source: Wikipedia)}
	\end{figure}
	A superconductor with little or no magnetic field within it is said to be in the "\NewTerm{Meissner state}". The Meissner state breaks down when the applied magnetic field is too large. Superconductors can be divided into two classes according to how this breakdown occurs:
	\begin{itemize}
		\item In "\NewTerm{Type I superconductors}\index{type I superconductors}", superconductivity (perfect diamagnetism behaviour) is abruptly destroyed when the strength of the applied field rises above a critical value $H_c$. Depending on the geometry of the sample, one may obtain an intermediate state consisting of a baroque pattern of regions of normal material carrying a magnetic field mixed with regions of superconducting material containing no field. 
		
		\item In "\NewTerm{Type II superconductors}\index{type II superconductors}", raising the applied field past a critical value $H_{c1}$ leads to a mixed state (also known as the "\NewTerm{vortex state}") in which an increasing amount of magnetic flux penetrates the material, but there remains no resistance to the flow of electric current as long as the current is not too large. At a second critical field strength $H_{c2}$, superconductivity is destroyed. The mixed state is actually caused by vortices in the electronic superfluid, sometimes named "\NewTerm{fluxons}\index{fluxons}" because the flux carried by these vortices is quantized. Most pure elemental superconductors, except niobium and carbon nanotubes, are Type I, while almost all impure and compound superconductors are Type II.
	\end{itemize}
	
	We will now determine London's second equation. It relate the external magnetic field applied on a sample to the value of this same magnetic field inside the sample.
	
	We consider for this purpose that in the superconductor phase, the electric charge carrier of mass $m$, of electric charge $q$, of density $\rho$, animated of a speed $v$ such that their kinetic energy is written:
	
	where the integral is obviously extended to the whole sample volume of the superconductor. 
	
	We have proved earlier that we also:
	
	and in static regime, it is linked to the magnetic field $\vec{B}$ by the four Maxwell equation and if the conductor is perfect it has no capacity and we can omit the second term such that it remain:
	
	Combining these relations we have:
	
	It is important at this step to notice that we assume that in the superconductor that since $\vec{v}\neq 0$ there is like as permanent current! So if we observe the Meissner effect it means that implicitly that there is something like a current inside the superconductor sample and this without that we need to provide any external potential (voltage)!
	
	The term in parenthesis has the dimension of a length and is named "\NewTerm{London penetration depth}\index{London penetration depth}" and denoted $\lambda_L$ such that the latter equality is written:
	
	Furthermore, we have prove in the section of Electrodynamics that the energy density of a monochromatic wave was given by:
	
	Therefore we assume that to the external magnetic field energy is given by:
	
	The total energy is the written:
	
	The total energy should be, by the Maupertuis principle, minimal for the real magnetic field profile that we will denote $B^{*}(\vec{r})$ and that we are looking for.
	
	The variational $\delta E_\text{tot}$ is then equal to zero if the profile $B(r)$ departs only in an infinitesimal way from the real profile from a quantity $\delta B(\vec{r})$. Therefore:
	
	But we have introduced in the section of vector Calculus the following identity:
	
	Therefore with $\vec{x}=\text{rot}(\vec{B})$ and $\vec{y}=\delta\vec{B}$ we have:
	
	Therefore:
	
	Therefore our variational can be written:
	
	Now let us recall the Ostrogradsky theorem proved in the section of Vector Calculus:
	
	Therefore:
	
	The surface integral is always zero since, by hypothesis, $\vec{B}$ is fixed on the sample boundary ($\delta\vec{B}(r_\text{surface})=0$), therefore it remains:
	
	The real $B^{*}(\vec{r})$ we are looking for is such that:
	
	Therefore we can try the trivial solution:
	
	But we have seen in the section of Vector Calculus that:
	
	Therefore (don't forget that the divergence of the magnetic field is always equal to zero!):
	
	Therefore:
	
	Hence:
	
	Now it must be notice that we cannot have $B_x$ that of $x$ if the sample is put in a magnetic field that is static and uniform otherwise we would not have $\text{div}(B^{*}(\vec{r}))=0$. And as we need $\text{rot}(B^{*}(\vec{r}))=\mu\vec{j}$ then as simple solution is to suppose that each component $B_i$ depends only on a coordinate that is perpendicular to itself. Therefore we will choose $B_z$ as depending on $x$ such that the latter differential equation will be written for this component:
	
	A special simple solution is obviously:
	
	This relation is knows as the "\NewTerm{second London's equation}\index{second London's equation}". The fact that the magnetic field decrease exponential inside the material is interpreted (...) in many textbooks and by many physicists as the fact the superconductor rejects the magnetic field since in laboratories we have the following observation:
	\begin{figure}[H]
		\centering
		\includegraphics[scale=0.6]{img/electromagnetism/messnier_effect.jpg}
		\caption{Observation of a superconductor in levitation in magnetic field due to Meissner effect}
	\end{figure}
	This bring us to the following well known misleading figure below that we can see on many Internet sites and in many textbooks:
	\begin{figure}[H]
		\centering
		\includegraphics[scale=0.6]{img/electromagnetism/superconductor_criticaltemperature.jpg}
	\end{figure}
	
	The derivation above proposed by Pierre-Gilles de Gennes for undergraduate class that don't use Quantum Physics is however a coincidence (as there exists many others in physics)! Obviously we see that this derivation of the London equation is very not appropriate. In this derivation the fact that the metal is in superconducting state is not used at all for example... Therefore, if this derivation is correct, any metal should show the Meissner effect.

	\begin{flushright}
	\begin{tabular}{l c}
	\circled{40} & \pbox{20cm}{\score{4}{5} \\ {\tiny 23 votes,  66.09\%}} 
	\end{tabular} 
	\end{flushright}

	%to force start on odd page
	\newpage
	\thispagestyle{empty}
	\mbox{}	
	\section{Optics (ray optics)}\label{geometrical optics}
	\lettrine[lines=4]{\color{BrickRed}O}ptics is the study of the fraction of radiant energy sensitive to the retina, that is to say the "\NewTerm{light}\index{light}" or said more generally: the "\NewTerm{electromagnetic waves}\index{electromagnetic waves}" (\SeeChapter{see section Electrodynamics page \pageref{electromagnetic wave equation}}) and over a wide frequency band which is not limited (depending on the case studies) to visible light!
	
	We chose in this book to split the study of "\NewTerm{geometrical optics}\index{geometrical optics}" or simply "\NewTerm{optics}\index{optics}" into three parts: the photometry (see below), geometrical optics (this section) and wave optics (next section).
	\begin{enumerate}
		\item The "\NewTerm{photometry}\index{photometry}" is responsible for the study of some of the variables definitions of energy properties of electromagnetic waves with respect to the visual sensitivity.
		
		\item The "\NewTerm{geometrical optics}\index{geometrical optics}" is responsible for the study that describes the propagation of light in transparent media without involving the nature of light. This is a part of physics has the advantage not to require complicated mathematical tools, but a lot of good geometrical sense...
		
		\item The "\NewTerm{wave optics}\index{wave optics}" is responsible for the study where the luminous phenomena are interpreted taking into account of the nature of light. This is considered as an electromagnetic wave of a given wavelength defining its color (subjective concept as we will discussed below). This use much more complicate tools as Fourier transforms and Green theorem for example.
	\end{enumerate}
	In some experiments, however, we must consider light as a corpuscular phenomenon (\SeeChapter{see section Wave Quantum Physics}). We assume then that it is formed by particles, "\NewTerm{photons}\index{photons}" whose energy is proportional to the light frequency according to the Planck's law (not the Planck's law of thermodynamics ... the other one!).
	
	For consistency reasons, as we have already mentioned, we have chosen to put photometry in the section of Geometric Optics (here so...).
	
	Before we start studying the mathematics of geometrical optics, it seemed good to us clarify some blurred areas of optics that are rarely clearly defined or not even treated at all in most physics books. Thus, we will first present what is a source of light or an absence of light and then how colors are seen and treated by humans (and most similar animals).
%	\begin{figure}[H]
%		\centering
%		\includegraphics[scale=0.09]{img/electromagnetism/sun_rays.jpg}
%	\end{figure}

	\begin{tcolorbox}[title=Remark,colframe=black,arc=10pt]
	It would be pretentious to claim to do with this section as well and also complete as the \textit{Optics} textbook of Eugene Hecht that is a priori unrivalled in content and quality to this date. It is therefore strongly recommended to refer to it if the reader wants to drive full information about Optics (\cite{hecht2016optics})!
	\end{tcolorbox}
	
	\pagebreak
	\subsection{Sources and Shadows}
	Experience teaches us that in a homogeneous and transparent medium light travels in a straight line (at least in a flat space) and that the latter always comes from "\NewTerm{light sources}\index{light sources}".
	
	Some objects are illuminated by themselves (Sun, Flames, etc.). Some other objects are not generally visible in the dark in the visible bandwidth (when there is absolutely no source of light) but if they are illuminated they send back all or part of light in some or all directions and therefore behave like light sources but with less intensity as the original one.
	
	\textbf{Definitions (\#\mydef):}
	\begin{enumerate}
		\item[D1.] A "\NewTerm{punctual source}\index{punctual light source}" is a unique "\NewTerm{bright spot}\index{bright spot}".
		
		\item[D2.] An "\NewTerm{extended source}\index{extended source}" is a set being a sum of punctual sources.
		
		\item[D3.] A "\NewTerm{light ray}\index{light ray}" is any straight line (or "geodesic"...) that light will follow.
		
		\item[D4.] A "\NewTerm{reversible light ray}\index{reversible light ray}" is a light ray that will follow the exactly same path backwards if we reverse its direction of propagation.
		
		\item[D5.] A "\NewTerm{light beam}\index{light beam}" is a set of light rays.
		
		\item[D6.] A "\NewTerm{ray diagram}\index{ray diagram}" is used to determine the image location, size, orientation and type of image formed by objects when place at a given location from a mirror or lens. They proved useful information about object-image relations and focal-geometry properties.
		
		\item[D7.] An "\NewTerm{apparent diameter}\index{apparent diameter}" is the angle, usually small, under which we see  one of the dimensions of an object (angle in radians as seen in the section of Trigonometry).
	\end{enumerate}
	Light passes through empty space without undergoing alteration (on small distance at least...). Thus is how Sunlight, before reaching the limit of the earth's atmosphere, travel through huge empty spaces without undergoing major transformations.
	
	On Earth, between a luminous object and the seeing eye of  the object, the light passes through a certain thickness of air. The object remains visible in other gases, or through a sheet of glass, mica, cellophane ..., or even through a layer of water, alcohol, glycerine ... such bodies are "\NewTerm{transparent media}\index{transparent media}" (even if they alter the quality of the image).
	
	Most bodies do not let light pass through. Placed between the eye and a luminous object, they suppress the vision of the object: then we say that they are "\NewTerm{opaque bodies}\index{opaque bodies}".
	
	In fact, no substance is perfectly transparent and propagation in a transparent medium is always accompanied by a weakening. This phenomenon of absorption depends on the nature of the medium and increases with the thickness of material traversed. Thus the water, even very pure, is opaque at a thickness of a few hundred meters. Also the deep seabed never receive sunlight.
	
	Sometimes some body, named "\NewTerm{translucent bodies}\index{translucent bodies}" let light filter  without allowing the eye to identify the object that emits light. These are frosted glass, striped glass, thin porcelain, oiled paper...
	
	In a dark space, the eye located outside the path of light, sees this path thanks to fine solid particles (dust, tobacco smoke, fog, etc.) suspended in the air. These illuminated particles scatter the light they receive, becoming as many bright spots which materialize the volume traversed by the light. The familiar observation shows that these light volumes still seem limited by straight lines of light.
	
	We can therefore apply the theorem of Thales to some light phenomena. So imagine the following experiment:
	
	We create quite small dimensions light sources that we can consider as point sources (i.e. bright spots).
	
	Consider $S$ is such a point source of light. Consider the volume that the source $S$ illuminates through an aperture in a diaphragm located in the path of the light at a distance $d$. If we denote by $AB$ the circular diameter of the opening diaphragm $K$ and that we cut the light path by a screen $E$, parallel to $K$ and at a distance $D$ from the source, we would observe that the irradiated region is limited by a disc of diameter $A'B'$.
	
	\begin{figure}[H]
		\centering
		\includegraphics{img/electromagnetism/thales_theorem_punctual_source.jpg}
		\caption{Application of Thales' theorem on punctual sources}
	\end{figure}
	If we can measure the diameters $AB$ and $A'B'$ of the two discs and their distances $d$ and $D$ to the source, we would find that they satisfy the Thales' theorem and therefore that:
	
	It is also proof that the light volume is effectively limited by straight lines issues from $S$ and leaning on the edge of the aperture.
	
	These observational facts and basic experience suggest the following hypothesis: 
	
	In a homogeneous transparent medium (remember that a medium is homogeneous when all is infinitesimal volumes have the same physical properties) light coming from a light spot propagates along straight lines from this point. These lines are named as we already know "light rays".
	
	If we return to the previous above figure, all the light rays contained in the cone defined by the source $S$ and the diaphragm $K$ is also as we already know a "light beam".
	
	\textbf{Definitions (\#\mydef):}
	\begin{enumerate}
		\item[D1.] Light travelling here from $S$ we say then that the ray "diverges" or that the beam is a "\NewTerm{divergent beam}\index{divergent beam}" (at the opposite of the LASER for those that know what is a LASER).
		
		\item[D2.] When a point source is at infinity (as is almost a Star in the night sky for example), the rays that come from it are considered as parallel and the beams that they form are named "\NewTerm{parallel beam}\index{parallel beam}" or "\NewTerm{cylindrical beams}\index{cylindrical beams}".
		
		\item[D3.] Thanks to a convergent lens (magnifying glass for example), we will see that it is possible to change the direction of rays from a punctual source and make the concentrate on a point $S'$. Such a set of rays then constitutes a "\NewTerm{convergent beam}\index{convergent beam}".
	\end{enumerate}
	
	A very narrow light beam takes the name of "\NewTerm{light pen}\index{light pen}". For example, the rays from a light spot on the eye always form a very slender pencil of light, because the distance from the observed eye is necessarily large compared to the diameter of the pupil.
	
	If we return to our experience with the diaphragm: If we decrease the opening of the diaphragm that will limit  a light beam to a light pen, we observe (when the diameter is reduced to less than a few tenths of a millimetre) that the trace of the brush on a screen $E$, rather than diminish, expands (!) and this is an evidence that the light reaches now points outside of the cone $SA'B'$.
	
	It is as if the very small aperture $AB$ itself became a punctual source: we say that the light is "\NewTerm{diffracted}\index{diffracted}". We will return later to this property of light because it is a fairly elaborate mathematical study (\SeeChapter{see section Wave Optics page \pageref{fraunhofer diffraction}}) and therefore complex to handle but still very very interesting (having huge implications because related to quantum physics!).
	
	Now consider a punctual source of light. Between the source and a screen $E$, let us interpose an opaque body of any shape. Under the assumption of rectilinear propagation, we observe a "\NewTerm{shadow cone}\index{shadow cone}" limited by the rays that are based on the outline of the interposed body.
	\begin{figure}[H]
		\centering
		\includegraphics{img/electromagnetism/shadow_cone.jpg}
		\caption{Shadow cone with proper shadow example}
	\end{figure}
	The non-illuminated region of the opaque body is the "\NewTerm{proper shadow}\index{proper shadow}" or "\NewTerm{umbra}\index{umbra}", one that matches the screen is the "\NewTerm{projected shadow}\index{projected shadow}".
	
	If the light source is extended, the proper shadow and projected shadow have no longer clearly defined contours. Their edges are surrounded by an intermediate area which is named the "\NewTerm{penumbra}\index{penumbra}". A well known example is given by the figure below:
	\begin{figure}[H]
		\centering
		\includegraphics{img/electromagnetism/shadow_penumbra.jpg}
		\caption{Shadow cone with Penumbra example}
	\end{figure}
	
	\subsection{Colors}
	\textbf{Definition (\#\mydef):} We name "\NewTerm{color}\index{color}" the perception of light excitation following neurophotochemical reaction process in the eye following of one or more light wave frequencies with one or more given amplitudes.
	
	It is important never to confuse "\NewTerm{color}\index{color}" that is a perceptive concept  (it is not a physical property!!!) and "\NewTerm{wavelength}\index{wavelength}" that is a physical concept. Thus, the human eye is most often unable to distinguish a theoretical yellow monochromatic (single wavelength) of a corresponding composition of green and red. This illusion will display yellow on our computer screens, and more generally any color.
	
	By the fact that the sensitive part of the retina of the human eye is made up of elements named "\NewTerm{cones}\index{cone!human eyes}" each sensitive to a small bandwidth of wavelength corresponding respectively\footnote{According to some rare studies at this day (year 2019), it seems that a few humans are able to see in some very special conditions (impulse LASER) low infra-red around  $\sim 1000 [\text{nm}]$} to Red (via the erythrolabe molecule sensible to wavelength $\sim 700 [\text{nm}]$ and denoted by the letter $L$) , Green (via the chlorolabe molecule sensible to wavelength $\sim 545 [\text{nm}]$ and denoted by the letter $M$) and Blue (via the molecule cyanolabe sensible to wavelength $\sim 440 [\text{nm}]$ and denoted by the letter $S$): 
	\begin{figure}[H]
		\centering
		\includegraphics[scale=0.5]{img/electromagnetism/eye.jpg}
		\caption{Naive eye structure summary}
	\end{figure}
	We can create any color by adding these three basic colors named "\NewTerm{additive primary colors}\index{additive primary colors}" (or "\NewTerm{additive primary colors}\index{additive primary colors}"). This is named the "\NewTerm{additive synthesis of colors}\index{additive synthesis of colors}".
	
	\begin{tcolorbox}[title=Remark,colframe=black,arc=10pt]
	Some scientists believe (without proof and without being able to reproduce this fact) that the sensitivity to each of theses colors comes from: for the Red for the time where evolution was in water and animals were looking in infra-red, for the Green at the time where animals were on earth with a luxurious and dense vegetation and finally for the Blue because of the blue sky!
	\end{tcolorbox}	
	In what follows, we will denote the red by ($R$), the green by ($G$), the blue by ($B$), the white by ($W$) and the black by ($N$):
	\begin{table}[H]
		\centering
		\begin{tabular}{|l|c|c|}
		\hline
		\rowcolor[HTML]{C0C0C0} 
		\textbf{color} & \textbf{wavelength interval} & \textbf{frequency interval} \\ \hline
		red & \cellcolor[HTML]{FE0000}$\sim 625-740\;[\text{nm}]$ & \cellcolor[HTML]{FE0000}$\sim 480-405\;[\text{THz}]$ \\ \hline
		orange & \cellcolor[HTML]{F8A102}$\sim 590-625\;[\text{nm}]$ & \cellcolor[HTML]{F8A102}$\sim 510-480\;[\text{THz}]$ \\ \hline
		yellow & \cellcolor[HTML]{F8FF00}$\sim 565-590\;[\text{nm}]$ & \cellcolor[HTML]{F8FF00}$\sim 530-510\;[\text{THz}]$ \\ \hline
		green & \cellcolor[HTML]{34FF34}$\sim 500-565\;[\text{nm}]$ & \cellcolor[HTML]{34FF34}$\sim 600-530\;[\text{THz}]$ \\ \hline
		cyan & \cellcolor[HTML]{00D2CB}$\sim 485-500\;[\text{nm}]$ & \cellcolor[HTML]{00D2CB}$\sim 620-600\;[\text{THz}]$ \\ \hline
		blue & \cellcolor[HTML]{3166FF}$\sim 440-485\;[\text{nm}]$ & \cellcolor[HTML]{3166FF}$\sim 680-620\;[\text{THz}]$ \\ \hline
		violet & \cellcolor[HTML]{6200C9}$\sim 380-440\;[\text{nm}]$ & \cellcolor[HTML]{6200C9}$\sim 790-680\;[\text{THz}]$ \\ \hline
		\end{tabular}
		\caption{Visible spectrum summary}
	\end{table}
	It's interesting to compare the human color vision with that of another animal like the mantis shrimp:
	\begin{figure}[H]
		\centering
		\includegraphics[scale=0.4]{img/electromagnetism/human_mantis_shrimp_vision.jpg}
	\end{figure}
	
	The French Association for Standardization (AFNOR) has defined in the 20th century the visual trivariance principle as follows:\textit{ A radiation of any color can be produced visually identically by algebraic mixture, in a uniquely defined proportions, of luminous beam of three rays that can be arbitrarily selected, provided that none of them can be reproduced by a combination of the other two}.
	
	Clearly when we see the frequencies of the visible spectrum it will not be tomorrow with the known material of this early 21st century that we are going to build antennas or parabola able to transmit at such frequencies! Already $120 [\text{GHz}]$ is a challenge by the year 2010 then $500 [\text{THz}]$ will not be for tomorrow...
	
	You should know that until the years 1800 it was not known if the colors were limited or not those visible to the human eye. It was with the advent of thermometers with mercury that were sufficiently sensitive and specific that the astronomer Herschel placed one in front of a light spectrum and found that moving it from a color strip to the other, from violet to red, the temperature raised. To his surprise, it continued to rise when he accidentally let the thermometer to on or two centimetres beyond the red light visible bandwidth. Herschel detected invisible light to the human eye, described later as being infra-red radiation.
	\begin{figure}[H]
		\centering
		\includegraphics[scale=0.5]{img/electromagnetism/whole_spectrum_colors.jpg}
		\caption[Whole standard electromagnetic spectrum]{Whole standard electromagnetic spectrum (source: Wikipedia)}
	\end{figure}
	Or corresponding temperatures:
	\begin{figure}[H]
		\centering
		\includegraphics{img/electromagnetism/temperature_spectrum.jpg}
		\caption{Temperature Spectrum}
	\end{figure}
	
	\begin{tcolorbox}[title=Remark,colframe=black,arc=10pt]
	Experimental measurements of the early 21st century provide us strong evidence that hydrogen bonds ionization ranges between $1000\;[\text{kJ}\cdot \text{mol}^{-1}]$ to $2000\;[\text{kJ}\cdot \text{mol}^{-1}]$ (ie $10$ [eV] to $20$ [keV]) and hydrogen bonds energy ranges between $6\;[\text{kJ}\cdot \text{mol}^{-1}]$ to $1000\;[\text{kJ}\cdot \text{mol}^{-1}]$ (ie $0.06$ [eV] to $10$ [keV]). Keep in mind that to ionize or break such bonds, photons must have at least the same energy (the energy of multiple photons can't be summed easily as it is quantum physics under the hood). For the $0.06$ [eV] ionization bond we need then photons with a frequency of \underline{at least} $1$ [THz]...
	\end{tcolorbox}

	A magistral and pedagogical example of what can be seen in the various spectra is the Crab Nebula that is a supernova remnant resulting from the explosion of a historical supernova (SN 1054) observed by several Chinese astronomers of the dynasty Song from July 1054 to April 1056:
	\begin{figure}[H]
		\centering
		\includegraphics[scale=0.6]{img/electromagnetism/SN1054_v2.jpg}
		\caption{SN 1054 observed under different wavelengths and different telescopes}
	\end{figure}
	Or also these different photos of the Earth under different wavelength:
	\begin{figure}[H]
		\centering
		\includegraphics[scale=1]{img/electromagnetism/earth_photos_different_wavelengths.jpg}
		\caption{Earth views under different wavelengths}
	\end{figure}
	Also for health prevention reasons, let us inform that Ultraviolet is commonly separated in three well known groups (UVC in space is a killer factor...):
	\begin{figure}[H]
		\centering
		\includegraphics[scale=0.5]{img/electromagnetism/ultraviolet_rays.jpg}
		\caption{Various types of Ultraviolet rays}
	\end{figure}
	And also let us give the following figure that is quite instructive (representing the state of knowledge and Earth's atmosphere of the early 21st century):
	\begin{figure}[H]
		\centering
		\includegraphics[width=1.0\textwidth]{img/electromagnetism/electromagnetic_spectrum_and_atmosphere_absorption.jpg}
		\caption[Electromagnetic spectrum energy and atmosphere absorption]{Electromagnetic spectrum energy and atmosphere absorption (source: \cite{fels2015fields})}
	\end{figure}
	Pointing three light beams ($R$, $G$ and $B$) in the same place, we can get (in fact it would be more rigorous to say "perceived" because this is specific only to certain trichromats mammals): white light. We say then that white ($W$) (in the human sense) is the sum of the three additive primary colors (remember that white is rigorously the sum of all the colors of the spectrum - so that white is made of a continuous light spectrum). All imaginable colors are obtained by varying the intensity of each of the three beams. Black is obtained when we do send no light at all.
	
	For example, if we add (in the theoretical sense: with infinitely small and transparent colors component ...) just red ($R$) and green ($G$), we get the yellow ($Y$), if we add the red and blue we get the magenta ($M$), if we add green ($G$) and blue ($B$), we get the Cyan ($C$). So we can sum this up by the following equations:
	
	These three colors ($YMC$) obtained by adding two additive primary colors are named "\NewTerm{secondary additive colors}\index{secondary additive colors}". The notion that blue and yellow form green is only a perception in our eyes!!! If the reader shine blue and yellow light on a chromograph it will show blue and yellow independently. In fact light doesn't interfere with light in our regular experience set at the opposite of sound. And this because as our eye only measure three things at the opposite of ears that measure every sound on the spectrum we can hear (as it do a indirectly a Fourier Transform).
	
	Diagram of additive synthesis:
	\begin{figure}[H]
		\centering
		\includegraphics{img/electromagnetism/rgb.jpg}
		\caption{Additive synthesis (RGB)}
	\end{figure}
	The existence of these three types of pigments in our photoreceptor cones serves as physiological basis to the "\NewTerm{trichromatic model}\index{trichromatic model}" or of "\NewTerm{visual trivariance}\index{visual trivariance}".
	
	\textbf{Definition (\#\mydef):} A color is named "\NewTerm{complementary color}\index{complementary color}" of another if they give white\footnote{Keep in mind that strictly speaking, white nor black are colors. White is the sum of all colors (electromagnetic waves radiation) from all the visible spectrum and black is the absence of colors (the absence of electromagnetic waves in the visible spectrum)} when you add them. For example, yellow ($Y$) is the complementary color of blue ($B$):
	
	In contrast to additive synthesis, there are the "\NewTerm{subtractive colors}\index{subtractive colors}": it is that we are talking about when we take away the color to a base color. This is for example the case of the ink or of color filters (in the sense that there is a base support where color must be treated).
	
	To understand what this is, let us put a red filter on an projector. The projected light will be red. We note that the filter has removed color to white light: $W$ became $R$ but as $W = RGB$, this means that the red color filter removed the $GB$ to the white light of the overhead projector. With the same reasoning, we understand that a $G$ filter subtracts the colors $RB$ and $B$ filter subtracts $RG$.
	
	If we overlay two filters of different primary colors: for example, $R$ and $G$ filter filters, we will not get anything at all, that is, to say $N$. Indeed, the $R$ filter let pass only red light and $G$ filter subtracts this color and nothing remains. This leaves no more color, i.e. $N$.
	
	We notice that filters $R, G$ and $B$ fail to synthesize different colors by subtraction as we get the black ($N$) as soon as we superimpose two different. What is very annoying when the concerned support is paper and the purpose is to print something colourful.
	
	It is therefore more useful to use the yellow, magenta and cyan filters ($Y$, $M$ and $C$) that is to say secondary additive colors. Indeed, a filter $Y$ let pass yellow, that is to say $RG$. It therefore subtracts only the blue $B$ to the original white $W$ light. By the same principle, a magenta $M$ filter subtracts green $G$ and a filter cyan $C$ subtracts red $R$.
	
	We note therefore that the superposition of two filters of these secondary colors gives a new color to an existing support. We can synthesize any color by varying the intensity of each of the three filters $Y$, $M$ and $C$ that we overlay (on the projector or paper for example). We name these three colors "\NewTerm{basic subtractive colors}\index{basic subtractive colors}".
	
	Scheme for subtractive synthesis:
	\begin{figure}[H]
		\centering
		\includegraphics{img/electromagnetism/cmy.jpg}
		\caption{Subtractive synthesis (CMY)}
	\end{figure}
	In reality we do not get a perfect black $N$ this is why printers have a black color ink.
	
	\begin{tcolorbox}[colframe=black,colback=white,sharp corners]
	\textbf{{\Large \ding{45}}Examples:}\\\\
	E1. A television or computer screen works on the principle of additive color mixing. Indeed, looking at the screen through a magnifying glass, one can realize that it is filled with small groups (pixels) of three phosphors (bright area when excited by electricity) $R, G$ and $B$. These phosphors are so close that when they light up together, they seem to get confused (at least on screen have a good "definition": pixels per inch) and most human therefore only sees the additive synthesis of the three phosphors. For example, on a television screen entirely red, only red phosphors glow. By cons, if the screen turns yellow, this means that green phosphors glow along with the red one.\\
	
	E2. At the opposite to television or computer screens, we find the printing processes that operate in subtractive synthesis. Indeed, the sheet is white and you have to remove from it to get the color we are looking for. The technique is the same as that of filter: inks contain pigments which filter certain colors. Using $Y, M$ and $C$ inks we can get all colors of the visible spectrum. However, the pigments are not perfect and black $N$ is very difficult to obtain (ink and color overload rather dark brown). So we use black as a fourth color. This system is named "\NewTerm{quadrichromy printing}\index{quadrichromy printing}". It is used for example in most color printers and in rotary newspapers.
	\end{tcolorbox}
	It is interesting now to focus on the phenomena that overlap the two concepts (if we can say ...). Thus, a system that projects color following a $RGB$ concepts by additive or subtractive system itself may be illuminated by an equivalent system. This results in an effect overlay.
	
	So when we talk about the color of objects, we normally refer to the appearance they have when they are illuminated with white light!
	\begin{tcolorbox}[colframe=black,colback=white,sharp corners]
	\textbf{{\Large \ding{45}}Example:}\\\\
	A red $R$ tomato, absorbs some of the white light $W$ ($GB$ part) and distributes the rest (red $R$) back around. That is why it appears Red $R$ to us when under white light. A yellow lemon, appears yellow to us because it absorbs the Blue $B$ color distributes the rest around ($RG=R+G=Y$) .... But what about a tomato lit by a blue light? What looks like a yellow lemon if we illuminate it by red light?\\
	
	We can answer by reasoning as follows: as tomato absorbs $GB$ and therefore intrinsically Blue ($B$), there remains nothing. It then appears black $N$. For the lemon, as it absorbs the blue $B$ and diffuses back $R + V=Y$ light so if we only illuminate it with red $R$ it will only diffuse the red and therefore appear will appear Red $R$ to us.
	\end{tcolorbox}
	\begin{figure}[H]
		\centering
		\includegraphics[scale=0.55]{img/electromagnetism/additive_subtractive_colors.jpg}
		\caption[Additive and Subtractive experiment]{Additive and Subtractive experiment (source: physicsfun instagram)}
	\end{figure}
	
	\pagebreak
	\subsection{Radiometry/Photometry}
	Material are capable of emitting, transmitting and / or absorb electromagnetic energy. Several factors characterize the radiation such as its spectral range, intensity, direction, and some intrinsic properties to the material. Photometry proposes to seek the magnitudes that are specific to the material and the laws governing them.
	
	We recognize two types of photometry: "\NewTerm{energetic photometry}\index{energetic photometry}" and "\NewTerm{visual photometry}\index{visual photometry}". In what follows, we will stick mainly to energetic photometry which generally relate to the energy carried by electromagnetic radiation, whatever its wavelength.
	
	Beforehand, we have to specify the conditions under which we will define the new variables/quantities. We will assume the following assumptions:
	\begin{enumerate}
		\item[H1.] The radiation propagates in a transparent medium for all intensities, wavelength and polarization.
		
		\item[H2.] The propagation takes place along the solid angles (\SeeChapter{see section Trigonometry page \pageref{solid angle}}). We therefore omit the radiation in parallel rays.
		
		\item[H3.] The elementary surfaces $\mathrm{d}S$ that are study are sufficiently small so that the radiation of their points are identical but not too small to avoid phenomena such as diffraction.
	\end{enumerate}
	
	\subsubsection{Energy flow} 
	\textbf{Definition (\#\mydef):} The "\NewTerm{energy flow}\index{energy flow}" or "\NewTerm{radiant flow}\index{radiant flow}\label{radiant flow}" of a radiation source is the power it radiates. The flow equation is measured in Watts [W] (or joules per second [$\text{J}\cdot \text{s}^{-1}$]), and it follows therefore that for a source that radiates energy (not necessarily constant), we have:
	
	In some professional fields the energy flow is expressed in photometric units as the "\NewTerm{Lumen}\index{Lumen}" denoted [lm] or in photonic units as a number of photons per second: $[\text{s}^{-1}]$. This is why, when buying lamps or displays in some stores, the units are not the same from one brand to another... (ISO norms are not respected...!).
	
	\paragraph{Beer–Lambert law}\label{Beer-Lambert law}\mbox{}\\\\
	The "\NewTerm{Beer–Lambert law}\index{Beer-Lambert law}", also known as "\NewTerm{Beer's law}\index{Beer's law}", the "\NewTerm{Lambert–Beer law}\index{Lambert–Beer law}", or the "\NewTerm{Beer–Lambert–Bouguer law}\index{Beer–Lambert–Bouguer law}" relates the attenuation of light to the properties of the material through which the light is travelling. The law is commonly applied to chemical analysis measurements and used in understanding attenuation in physical optics, for photons, neutrons or rarefied gases.
	
	If the absorption and diffusion of a medium can be considered proportional to the thickness $\mathrm{d}z$ of the crossing matters, the energy flow variation can be written:
	
	in this expression $\Phi_0$ is the incident energy flow and $\mu\;[\text{m}^{-1}]$ is the "\NewTerm{linear attenuation coefficient}\index{linear attenuation coefficient}", which is a function of the radiation frequency depending of the material the radiation pass through.
	
	So we have a simple differential equation (\SeeChapter{see section Differential and Integral Calculus page \pageref{first order lde with constant coefficients}}):
	
	which is the "\NewTerm{Beer-Lambert law}\index{Beer-Lambert law}" (which can also be expressed from the light intensity that we will define further below).
	
	We have typically the following order of amplitude:
	
	\begin{tcolorbox}[title=Remark,colframe=black,arc=10pt]
	The variation of the atmospheric absorption coefficient with the wavelength allows in particular to explain the blue color of the sky and because of water light reflection why our oceans looks like blue seen from space (yes don't forget that water don't have color... it's transparent!).
	\end{tcolorbox}
	There are many other formulations of the Beer-Lambert law with one fairly used in nuclear physics (see section of the same name in this book page \pageref{nuclear physics}) in the framework of radiation protection. Let's see now what it is:
	
	Let us consider a flow $\Phi$ of particles striking perpendicularly the surface of a material of thickness $\mathrm{d}z$ and of atomic density $\rho_N$ (in [$\text{atoms}\cdot \text{m}^3$]). If we consider the particles striking a surface $S$, the latter can theoretically meet $\rho_NS\mathrm{d}z$ targets atoms in this layer. The number of interacting particles will be proportional to the intensity times this number, and we have:
	
	where $\sigma$ is a proportionality constant named "microscopic cross section." Its units are often expressed as "\NewTerm{barn}\index{barn}" ($1\;[\text{barn}]=10^{-26}\;[m]$).
	\begin{tcolorbox}[title=Remark,colframe=black,arc=10pt]
	The atomic density $\rho_N$ is equal to:
	
	where  $\rho$ is the density in $\text{kg}\cdot\text{m}^3$, $N_\text{Av.}$ is the Avogadro's number $6.022\cdot 10^{23}\; [\text{atoms}\cdot \text{mole}^{-1}]$ and $M_\text{mol.}$ the molar mass of the target expressed in $[\text{kg}\cdot \text{mole}^{-1}]$.
	\end{tcolorbox}
	If we now assume that the scattering centers are the electrons and not the target atoms, then we must replace $\rho_N$ by $\rho_{N_e}$ where $N_e=\rho_NZ$ with $Z$ being the number of electrons interacting by target atom. Therefore:
	
	If we now admit that the scattering centers are the electrons, not the target atoms, then we must replace $N$ by $N_e$ where $N_e=NZ$ where $Z$  being the number of electrons interacting with the target atom. From where:
	
	By identifying with the first formulation of the Beer-Lambert law, we see that $\mu$ plays the same role as:
	
	And the hypothesis that the electron is a "\NewTerm{sphere of action}\index{sphere of action}" having a front surface $\pi r_e^2$, where $r_e$ is the radius of this sphere, then:
	
	and we have for the radius of the sphere of action of the electron:
	
	
	\subsubsection{Light Intensity (Radiant Intensity)}\label{solid angle optics}
	To describe the energy flow $\Phi$ of a source, you must first measure it. The used sensor (thermocouple, bolometer, photocell, eye or others) may receive only one part: that which happens in the solid angle  $\mathrm{d}\Sigma$ defined by its section.

	 \textbf{Definition (\#\mydef):} The "\NewTerm{light intensity}\index{light intensity}", "\NewTerm{energy intensity}\index{energy intensity}" or "\NewTerm{radiant intensity}\index{radiant intensity}" denoted $I$ of a punctual source is radiated flow $\Phi$ in the unit solid angle $\Omega$ centered around a direction $\Delta$ of emission:
	
	The light intensity is expressed in certain professional fields in the photometric units (see summary table further below) "\NewTerm{Candela}\index{Candela}" denoted [Cd] or in photonic units in $[\text{Ws}^{-1}]$ (remember that the steradians don't have any units as well as for radians). This is why, when you buy screens or lamps in some stores, the units are not the same from one brand to another.
	\begin{tcolorbox}[title=Remark,colframe=black,arc=10pt]
	A source is named an "\NewTerm{anisotropic source}\index{anisotropic source}" or "\NewTerm{directional source}\index{directional source}" if its intensity varies with the direction of observation.
	\end{tcolorbox}	
	By comparison (as it may help), a Candela unit is equivalent to the intensity of a source in a given direction, which emits a monochromatic radiation of frequency $540.1012$ [Hz] (which roughly corresponds to the frequency at which the eye is the most sensitive), and whose luminous flow(or intensity) in that direction of $1/683$ [W] per steradian.
	
	\subsubsection{Energy Emittance (Radiant Emittance)} \label{radiant emittance}
	\textbf{Definition (\#\mydef):} The "\NewTerm{emittance}\index{emittance}", "\NewTerm{exitance}\index{exitance}" or "\NewTerm{illumination}\index{illumination}" $M$ of a source is the radiated energy flow (power) per unit surface $\mathrm{d}S$ in [W / m${}^2$] in all directions of outer space the source and depends on the physico-chemical properties of the emitting surface:
	
	A common mistake is to make a confusion between the emittance and the energy!!!
	\begin{tcolorbox}[title=Remark,colframe=black,arc=10pt]
	"\NewTerm{Radiant emittance}\index{radiant emittance}" is an old term for this quantity. Radiant exitance is often named "\NewTerm{intensity}\index{intensity}" in branches of physics other than radiometry, but in radiometry this usage leads to confusion with radiant intensity.
	\end{tcolorbox}	
	The emittance is often assimilated in the current vocabulary to the "luminosity" of a light source which sometimes leads to confusion with the concept of light intensity.
	
	The emittance is expressed by many professional of the field in photometric units named "\NewTerm{Lux}\index{Lux}" denoted [lx] or in photonic units $[\text{W}\cdot\text{m}^{-2}]$ or even worst ... in $[\text{lm}\cdot \text{m}^{-2}]$ or even in $[\text{cd}\cdot \text{sr} \text{m}^{-2}]$. For example when you buy a car, the headlights are indicated as being about $\sim 20$ [lx].
	
	If the source is punctual (or considered as!) and its radiation isotropic , its direction has not to be taken into consideration. In the case of the related sphere of radius $r$, the emittance is then expressed as:
	
	In the case of the sphere, one element $\mathrm{d}S$ of the spherical surface receives perpendicularly the radiation. Very generally, an elementary surface can be inclined relative to the direction of radiation with an angle $\theta$.\label{surface projection emittence} So we have to project the surface perpendicular to the radiation using the basic reasoning of trigonometry (\SeeChapter{see section Trigonometry page \pageref{trigonometry}}):
	
	It is this projection that explains the seasons on Earth: the area swept by the emittance more or less constant and isotropic of the Sun (considered as a punctual source) is maximum at the equator (perpendicular surface) and implies a radiation higher compared to that receives at higher or lower latitude for which the perpendicular projection of the surface in question is smaller than at the equator for the same emittance.
	
	\begin{tcolorbox}[title=Remarks,colframe=black,arc=10pt]
	\textbf{R1.} The energy emittance is calculated only in the outer half space before (the point from which we look at the source), because only half of the energy exchanged by the points of the surface $\mathrm{d}$S is emitted as radiation. The other half is exchanged with the atoms located in the body.\\
	
	\textbf{R2.} The emittance is usually also sometimes denoted $F$ or even $E$. The practitioner will however take care not to confuse the emittance $M$ with the magnitude (denoted by the same letter) that we define in the section of Astrophysics.
	\end{tcolorbox}
	
	Now that we have some basics on radiant emittance let us deal with a quite recurrent question on social networks related to the fact that on most actual videos of the NASA we can't see the stars. This question is related to the calculation of mow many visible photons hit a pixel on a CCD sensor? With a few simplifying assumptions let us see that it not too difficult to calculate an approximation result to this question and get a simple answer to why we can't see distant stars on most videos made by CCD cameras.

	Let us recall for this purpose that we have proved in the section of Thermodynamics (see page \pageref{planck law}) that the emittance of the black-body was given by (remember that $\nu=c/\lambda)$:
	
	The power of spectral radiant exitance\index{spectral radiant exitance} contained in light within a specified wavelength range is then obviously obtained by integrating:
	
	over a given range. The result is named "\NewTerm{irradiance}\index{irradiance}" $E_e$ and given by:
	
	this can be expressed in terms of photons as follows using the Planck-Einstein relation ($E=hc/\lambda$);
	
	Such calculation is easily done numerically.  For instance the daylight Sun is considered to be a black-body radiator of around temperature $5,300$ [K].  In units better suited to a photographic sensor context it therefore emits about $4.549\cdot 10^{13}\; [\text{photons}\cdot\text{s}^{-1}\cdot\mu\text{m}^{-2}]$ in the standard visible range (380-780) in all directions at its surface (multiplied by the surface of the Sun it gives a number of approximately $1.5\cdot 10^{41}\;[\text{photons}\cdot\text{s}^{-1}]$). 
	
	As we have proved in the section of Electromagnetism, the intensity of the electromagnetic field decrease in the inverse square of the distance ($4\pi R^{-2}$). Then in reality we receive form the Sun:
	
	Or in units more suited to CCD (in $2019$ the typical civil CCD pitch size is a square of $1.51\;[\mu\text{m}]$):
	
	Keeping in mind that most CCD HD (high definition) civil cameras in the beginning of the 20th century have a capture rate of 20 to 30 frames per second! So the reader can easily understand now why cameras having a too fast frame rate, cannot captures stars (that you are in space or on Earth...).
	
	\begin{tcolorbox}[title=Remark,colframe=black,arc=10pt]
	Notice that for a distant star (the nearest one being at Proxima Centauri at almost $4.37\cdot 10^9$ [km]), the same calculation gives:
	 
	\end{tcolorbox}
	
	All of these photons do not necessarily make it through the atmosphere which absorbs varying amounts of light energy depending on wavelength.  On the other hand a household incandescent bulb at around $3,000$ [K] emits about $1.267 \cdot 10^{12}\;[\text{photons}\cdot\text{s}^{-1}\cdot\mu\text{m}^{-2}]$ most of which make it to our sensors.
	
	To determine what exposure the equivalent quantities above correspond to we simply convert the given irradiance to photometric units by taking into consideration its visible wavelengths only.
	
	The photometric version of irradiance $E_e$ is named  "\NewTerm{illuminance}\index{illuminance}" and defined by:
	
	where $V(\lambda)$ is the photonic eye response function and $K_m$ the luminous efficiency conversion constant $683\;[\text{lm}\cdot\text{W}^{-1}]$.  Remember that $[\text{lumens}\cdot\text{m}^{-2}]$ are named lux [lx]...
	
	Now that we know how to determine how many photons impinge on a sensor we can estimate its Effective Quantum Efficiency (EQE)\index{effective quantum efficiency}, that is the efficiency with which it turns such a photon flux into photoelectrons ($e^-$ ), which will then be converted to raw data to be stored in the captured raw file:
	
	It represents the probability that a photon arriving on the sensing plane from the scene will be converted to a photo-electron by a typical digital camera sensor.  It therefore includes the effect of microlenses, fill factor, CFA and other filters on top of silicon in the pixel.  It is usually expressed as a percentage.  For instance if  an average of 100 photons per pixel within the sensor's passband were incident on a uniformly lit spot of the sensor and on average each pixel produced a signal of 20 photoelectrons we would say that the Effective Quantum Efficiency of the sensor is $20\%$.
	
	\begin{tcolorbox}[title=Remark,colframe=black,arc=10pt]
	It seems that the eye can have a some wavelength an EQE of $100\%$ same as some expensive CCD...
	\end{tcolorbox}
	
	\subsubsection{Radiance and Luminance}\label{radiance and luminance}
	Given a non-punctual source which emittance $M$ is known at any point. An element $\mathrm{d}S$ of the surface of this kind of source will by definition not necessarily by isotropic intensity and therefore brighter (more powerful) when we observe col-linearly to the vector $\mathrm{d}\vec{S}$.
	
	The intensity $I$ that radiates in a direction forming an angle $\theta$ with the normal to the emission surface $S$ is always less than that radiated in the direction of the vector $\mathrm{d}\vec{S}$. So by simple application of trigonometry, we get the definition of "\NewTerm{luminance}\index{luminance}" (or "radiance"):
	
	more often written (to avoid a unit confusion because of the presence of the square on the differential element):
	
	expressed in certain areas, in "Nits"... photometric units  $[\text{Cd}\cdot\text{m}^{-2}]$ or photonic units $[\text{W}\cdot\text{m}^{-2}]$ (without specifying explicitly the steradian).
	
	Radiance is useful because it indicates how much of the power emitted, reflected, transmitted or received by a surface will be received by an optical system looking at that surface from some angle of view. In this case, the solid angle of interest is the solid angle subtended by the optical system's entrance pupil. Since the eye is an optical system, radiance and its cousin luminance are good indicators of how bright an object will appear. For this reason, radiance and luminance are both sometimes named "\NewTerm{brightness}\index{brightness}", even if this usage is now discouraged.
	\begin{tcolorbox}[title=Remark,colframe=black,arc=10pt]
	When we are concerned only with the human visible part of light, the luminance and radiance of a source is named by practitioners "brightness" (note this is not the case when we deal of shining as we will see it in the section of Astrophysics).
	\end{tcolorbox}	
	Luminance is a measure of how much power (energy over time) is emitted in a particular direction (basically how much light your TV will produce), Radiance broadly speaking, measures how much of the luminance you'll actually see. Finally brightness doesn't really measure anything it's broadly a catch-all term describing both luminance and radiance.
	
	We can obviously write by making some elementary algebra that:
	
	that gives us the energy intensity that radiates a source of luminance $L$ in a given direction.

	Johann Heinrich Lambert (1728-1777) has observed that the energy intensity of some anisotropic sources (among all imaginable types of sources ...) decreases as the cosine of the angle $\theta$, around the direction perpendicular to the surface source, such that:
 	
	and that we name "\NewTerm{Lambert's cosine law}\index{Lambert's cosine law}". So  the radiant intensity or luminous intensity observed from an ideal diffusely reflecting surface or ideal diffuse radiator is directly proportional to the cosine of the angle $\theta$ between the direction of the incident light and the surface normal:
	\begin{figure}[H]
		\centering
		\includegraphics{img/electromagnetism/lamberts_cosine_law.jpg}
		\caption[Lambertian Emittor]{Lambertian Emittor (source: Wikipedia)}
	\end{figure}
	A surface which obeys Lambert's law is said to be "\NewTerm{Lambertian surface}\index{Lambertian surface}". The apparent brightness of a Lambertian surface to an observer is the same regardless of the observer's angle of view. More technically, the surface's luminance is isotropic, and the luminous intensity obeys Lambert's cosine law. Unfinished wood exhibits roughly Lambertian reflectance, but wood finished with a glossy coat of polyurethane does not, since the glossy coating creates specular highlights. Not all rough surfaces are Lambertian reflectors, but this is often a good approximation when the characteristics of the surface are unknown.
	\begin{tcolorbox}[title=Remark,colframe=black,arc=10pt]
	We speak of "luminance" of a source of "illumination" of an object (by a source).
	\end{tcolorbox}
	
	\pagebreak
	\paragraph{Lambert's Law}\mbox{}\\\\\
	A light source follow the Lambert's cosine law if its luminance (or radiance) is the same in all direction. Indeed, if we put first:
	
	we have:
	
	and we have:
	
	But we have proved in the section of trigonometry that:
	
	Which brings us to write:
	
	The energy emittance being equal to:
	
	This result is important for the study of black-body radiation, since the luminance value measured by a sensor gives the possibility to deduce the emittance $M$, so the energy flow from the source is equal to: 
	

	Here is a small summary of what we have seen so far as sometimes it can be quite confusing:
	\begin{table}[H]
	\begin{center}
		\begin{tabular}{|c|c|c||c|c|c|}
			\hline
			\multicolumn{3}{|c|}{\cellcolor{black!30}\textbf{Radiometry}} & \multicolumn{3}{|c|}{\cellcolor{black!30}\textbf{Photometry}}\\
			\hline
			\cellcolor{black!30}\textbf{Quantity} & \cellcolor{black!30}\textbf{Symbol} & \cellcolor{black!30}\textbf{Units} & \cellcolor{black!30}\textbf{Units} & \cellcolor{black!30}\textbf{Symbol} & \cellcolor{black!30}\textbf{Quantity}  \\
			\hline
			Radiant Energy & $E_r$ & J & lm s & $E_r$ & Luminous Energy \\
			\hline
			Radiant Flow & $\Phi$ & W & lm s & $\Phi$ & Luminous Radiant \\
			\hline
			Light Intensity & $I$ & W/sr & Cd & $I$ & Radiant Intensity \\
			\hline
			Emittance & $M$ & W/m$^2$ & lm/m$^2$ & $M$ & Radiant Emittance \\
			\hline
			Radiant Intensity & $I$ & W/sr & lm/sr & $I$ & Luminous Intensity\\
			\hline
			Radiance & $L$ & W/(m$^2$sr) & lm/(m$^2$sr) & $L$ & Luminance\\
			\hline
		\end{tabular}
		\caption[]{Radiometric and Photometric quantities}
	\end{center}
	\end{table}
	
	\pagebreak
	\subsubsection{Kirchhoff's law of Radiation}\label{kirchhoff law of radiation}
	Any body irradiated by an energy source sees the incident energy flow incident divided into three intuitive terms:
	
	where:
	\begin{itemize}
		\item $\Phi_r=\rho\Phi$: is the energy flow reflected or diffused.

		\item $\Phi_t=\tau\Phi$: is the energy that flow through the body without interactions (full transparency).

		\item $\Phi_a=\alpha\Phi$: is the energy flow transformed into other forms of energy.
	\end{itemize}
	The three coefficients respectively named "\NewTerm{reflectance factor $\rho$}\index{reflectance factor}" or more simply "\NewTerm{albedo}\index{albedo}", "\NewTerm{transmittance factor $\tau$}\index{transmittance factor}" and "\NewTerm{absorption factor $\alpha$}\index{absorption factor}" depend obviously on the wavelength $\lambda$ (and therefore the frequency) of the incident light and the of the temperature of receiver body.
	
	For each object we have obviously:
	
	which is the expression of the "\NewTerm{simple Kirchhoff law}\index{simple Kirchhoff law}" in photometry (at the opposite of the differential one).
	
	The albedo is an important concept in climatology as we will use it in the corresponding section of the book.
	\begin{figure}[H]
		\centering
		\includegraphics[scale=0.15]{img/electromagnetism/earth_albedo.jpg}
		\caption[2003–2004 mean annual clear-sky and total-sky albedo]{2003–2004 mean annual clear-sky and total-sky albedo (source: Wikipedia)}
	\end{figure}
	
	\pagebreak
	A interesting (funny) but also very useful modern material for space telescopes imagery quality is Vantablack whose albedo is of  $0.00035$...! The result is quite mind blowing as you can see in the photo below:
	\begin{figure}[H]
		\centering
		\includegraphics{img/electromagnetism/ventablack.jpg}
		\caption{Vantablack painted face...}
	\end{figure}
	Looks like quite unreal isn't it? ;-)
	
	\subsubsection{Spectral Decomposition}
	From what has been said above, it follows that all the previously defined quantities can be related to their spectral decomposition in wavelength $\lambda$. This results comes from the superposition principle: any radiation can be treated as the superposition of monochromatic radiation (even if we will prove during our study of quantum physics later that a monochromatic radiation does not really exists).
	
	Thus, we define:
	
	Where we also write to simplify the notations:
	 
	\begin{tcolorbox}[title=Remark,colframe=black,arc=10pt]
	The units AND values of spectral decomposed energy flow, spectral intensity, spectral luminance/radiance, spectral emittance and also the spectral absorption factor, spectral reflectance factor and spectral transmission factor are of course not equivalent to their integrated expression.
	\end{tcolorbox}
	We will need for the density of the emittance in the study of black-body in the section of Thermodynamics of the chapter Mechanics. Remember only that we have in S.I. units based on the spectral decomposition principle (and vice versa superposition):
	
	\begin{tcolorbox}[title=Remark,colframe=black,arc=10pt]
	 We have seen in the section of Thermodynamic that parameters defined above, being dependent on wavelength, they are also dependent on the temperature of the source which emits these waves.
	\end{tcolorbox}
	
	\subsection{Law of Refraction}
	The law of refraction, which is generally known as "\NewTerm{Snell's law}\index{Snell's law}", governs the behaviour of light-rays as they propagate across a sharp interface between two transparent dielectric media.

	Pierre de Fermat proposed that light rays (electromagnetic waves) met a very general principle that the path taken by the light to travel from one point to another was the one for which the travel time was minimum (actually an extremum which may be a minimum or maximum). This proposal, known as "\NewTerm{Fermat's principle of least time}\index{Fermat's principle of least time}", a fundamental principle of geometrical optics is based on the principle of least action (principle we have already introduced in the section of Analytical Mechanics) as we prove it later below.

	Before starting the developments let us give some important definitions and introduce some concepts. some of which are related to the figure below:
	\begin{figure}[H]
		\centering
		\includegraphics{img/electromagnetism/vocabulary_optical_geometry.jpg}
		\caption{Vocabulary for the study of geometrical optics}
	\end{figure}
	
	\pagebreak
	\textbf{Definitions (\#\mydef):} 
	\begin{enumerate}
		\item[D1.] A "\NewTerm{refracting medium}\index{refracting medium}" is a medium that causes the deflection of an incident light ray.
		
		\item[D2.] The "\NewTerm{incident ray}\index{incident ray}" is the ray of light propagating in a medium $1$, pass wholly or partially in a refracting medium $2$, the rest being absorbed or partially reflected.
		
		\item[D3.] The "\NewTerm{angle of incidence}\index{angle of incidence}", sometimes denoted $i$ is the angle by which the incident beam enters the refracting medium.
		
		\item[D4.] The "\NewTerm{totally or reflected ray}\index{totally or reflected ray}" is the part of the light ray that having met the interface between the propagation medium and the refractive medium, continues his path in the propagation medium.
		
		\item[D5.] The "\NewTerm{reflection angle}\index{reflection angle}", sometimes denoted $r_x$ or simply $r$ if there is no possible confusion, is the angle by which the ray is reflected from the plane representing the interface between the propagation medium and itself. We will prove that the incident and reflected angles are equal in absolute values.
		
		\item[D6.] The "\NewTerm{partially or totally refracted ray}\index{partially or totally refracted ray}" is the portion of the light ray having met the interface between the propagation medium and the refractive medium, continues its path in the refracting medium.
		
		\item[D7.] The "\NewTerm{angle of refraction}\index{angle of refraction}", sometimes denoted $r_c$ or sometimes simply $r$ if there is no confusion, is the angle by which the ray is refracted by the plane representing the interface between the propagation medium and refracting medium. The incident and refracted angles are linked by a relation we prove further below.
	\end{enumerate}
	An illustration that could help to understand some of these definitions better is the following:
	\begin{figure}[H]
		\centering
		\includegraphics[scale=0.75]{img/electromagnetism/type_of_reflections.jpg}
		\caption{Type of reflections}
	\end{figure}
	For example with water this can give quite funny observations:
	\begin{figure}[H]
		\centering
		\includegraphics[scale=0.58]{img/electromagnetism/reflection_water_in_real_life.jpg}
		\caption{Funny water reflection effect}
	\end{figure}
	with its corresponding schematic explanation:
	\begin{figure}[H]
		\centering
		\includegraphics[scale=0.65]{img/electromagnetism/reflection_water.jpg}
		\caption{Water reflection ray diagram}
	\end{figure}
	Or a more tricky example that can be reproduced at home:
	\begin{figure}[H]
		\centering
		\includegraphics[scale=0.65]{img/electromagnetism/refraction_air_water_air.jpg}
		\caption[]{Source: Maric Vladimir}
	\end{figure}
	Before we continue, something important must be understand (or recall): "ideal refraction" must be discerned form "diffused refraction"! Since the light strikes different parts of the surface at different angles, it is reflected in many different directions, or diffused. So when we study refraction we in physics, we consider an "ideal refraction" with a "directional source of light"!
	\begin{figure}[H]
		\centering
		\includegraphics[scale=0.7]{img/electromagnetism/diffused_reflection.jpg}
		\caption[Light is diffused when it reflects from a rough surface]{Light is diffused when it reflects from a rough surface (source: OpenStax)}
	\end{figure}
	Diffused light is what allows us to see a sheet of paper from any angle:
	\begin{figure}[H]
		\centering
		\includegraphics[scale=1]{img/electromagnetism/diffused_vs_ideal_reflection.jpg}
	\end{figure}
	
	\pagebreak
	\subsubsection{Refractive index}
	The "\NewTerm{absolute refractive index $n_\lambda$}\index{absolute refractive index}" of a medium at a given wavelength $\lambda$ (and hence at a given frequency $\nu$) measure the reduction factor of the phase velocity of light, denoted $v_n$ in Geometric Optics, in the considered medium relatively to the speed of light in vacuum  and is given quite generally by the "\NewTerm{Cauchy's formula}\index{Cauchy's formula}" (\SeeChapter{see section Elektrokinetics page \pageref{Cauchy formula}}):
	
	where $A$ and $B$ are experimentally determined constants. We can notice through the Cauchy's formula that the absolute refractive index decreases as the wavelength increases (verbatim when the frequency decrease).
	
	Anyway, the theory of light-matter interaction on which Cauchy based this equation was later found to be incorrect. The  "\NewTerm{Sellmeier equation}\index{Sellmeier equation}"  is a later development of Cauchy's work that handles anomalously dispersive regions, and more accurately models a material's refractive index across the ultraviolet, visible, and infra-red spectrum and that is given by (see proof earlier above page \pageref{Sellmeier equation}):
	
	Following what we have seen in the section of Electrodynamics it comes immediately:
	
	Therefore:
	
	\begin{tcolorbox}[title=Remark,colframe=black,arc=10pt]
	Obviously in reality all that stuff is also a function of the temperature and... not only...
	\end{tcolorbox}	
	All pure materials have absolute refractive index of a positive value greater than $1$. More a medium is dense, more the phase velocity of light is slowed down, more the absolute refractive index is high.
	\begin{tcolorbox}[title=Remark,colframe=black,arc=10pt]
	There exist however material with smaller than $1$ refractive index and even negative-index meta-material.
	\end{tcolorbox}	
	The process of describing light transport via the quantum mechanical description isn't trivial to explain the "slow down" of the phase velocity. The use of photons to explain such process involves the understanding of not just the properties of photons, but also the quantum mechanical properties of the material itself (something one learns in Solid State Physics courses).

	A common explanation that has been provided is that a photon moving through the material still moves at the speed of $c$, but when it encounters the atom of the material, it is absorbed by the atom via an atomic transition. After a very slight delay, a photon is then re-emitted. This explanation is incorrect and inconsistent with empirical observations. If this is what actually occurs, then the absorption spectrum will be discrete because atoms have only discrete energy states. Yet, in glass for example, we see almost the whole visible spectrum being transmitted with no discrete disruption in the measured speed. In fact, the index of refraction (which reflects the speed of light through that medium) varies continuously, rather than abruptly, with the frequency of light.

	Secondly, if that assertion is true, then the index of refraction would ONLY depend on the type of atom in the material, and nothing else, since the atom is responsible for the absorption of the photon. Again, if this is true, then we see a problem when we apply this to carbon, let's say. The index of refraction of graphite and diamond are different from each other. Yet, both are made up of carbon atoms. In fact, if we look at graphite alone, the index of refraction is different along different crystal directions. Obviously, materials with identical atoms can have different index of refraction. So it points to the evidence that it may have nothing to do with an "atomic transition".

	When atoms and molecules form a solid, they start to lose most of their individual identity and form a "collective behaviour" with other atoms. It is as the result of this collective behaviour that one obtains a metal, insulator, semiconductor, etc. Almost all of the properties of solids that we are familiar with are the results of the collective properties of the solid as a whole, not the properties of the individual atoms. The same applies to how a photon moves through a solid.

	A solid has a network of ions and electrons fixed in a "lattice". Think of this as a network of balls connected to each other by springs. Because of this, they have what is known as "collective vibrational modes", often named "phonons" as we will see it during our study of semi-conductors in the section Electrokinetics. These are quanta of lattice vibrations, similar to photons being the quanta of EM radiation. It is these vibrational modes that can absorb a photon. So when a photon encounters a solid, and it can interact with an available phonon mode (i.e. something similar to a resonance condition), this photon can be absorbed by the solid and then converted to heat (it is the energy of these vibrations or phonons that we commonly refer to as heat). The solid is then opaque to this particular photon (i.e. at that frequency). Now, unlike the atomic orbitals, the phonon spectrum can be broad and continuous over a large frequency range. That is why all materials have a "bandwidth" of transmission or absorption. The width here depends on how wide the phonon spectrum is.
	
	On the other hand, if a photon has an energy beyond the phonon spectrum, then while it can still cause a disturbance of the lattice ions, the solid cannot sustain this vibration, because the phonon mode isn't available. This is similar to trying to oscillate something at a different frequency than the resonance frequency. So the lattice does not absorb this photon and it is re-emitted but with a very slight delay. This, naively, is the origin of the apparent slowdown of the light speed in the material. The emitted photon may encounter other lattice ions as it makes its way through the material and this accumulate the delay. 
	
	Moral of the story: the properties of a solid that we are familiar with have more to do with the "collective" behaviour of a large number of atoms interacting with each other. In most cases, these do not reflect the properties of the individual, isolated atoms.
	
	Finally, the "\NewTerm{relative refractive index $n_{12}$}\index{relative refractive index}" specifically refers to comparing (ratio) of the absolute refractive index of an optically dense media to another background media such that:
	
	We can also use wavelengths to calculate refractive index. By substituting (notice that by conservation of energy the frequency is assumed to be the constant as $E=h \nu$):
	
	then:
	
	This can be depicted by the following figure:
	\begin{figure}[H]
		\centering
		\includegraphics{img/electromagnetism/snell_law_wavelength.jpg}
	\end{figure}
	
	\pagebreak
	\subsubsection{Snell's law}
	Let us consider (see figure below) two respective medium $M_n$ and $M_m$ of refractive indices $n$ and $m$ (implicitly dependent on the wavelength as we will consider constant temperature) and whose contact surface is flat. Consider two points $A$ and $B$ respectively in the medium of index $n$ (point $A$) and in the medium of index $m$ (point $B$).

	Consider the path of light going from $A$ to $B$. Fermat's principle teaches us that the path taken by the light is such that the time taken to travel is minimum.

	Therefore we propose in a first time to apply a standard method to calculate the path of the light ray and secondly, we will show that Fermat's principle can be stated as a variational principle!!!
	
	Let us choose a reference frame that simplifies the problem: let us make coincide the $x$-axis with the contact plane of both medium and the $y$-axis through point $B$. In such a reference frame, points $A$ and $B$ have the following coordinates: $A(x_A,y_A)$, $B(0,y_B)$.
	
	Let us denote by $M(x,0)$, the point where the light beam passes through the contact surface between the two medium. The time $t$ taken by light to get from $A$ to $B$ is by applying simple kinematics relations:
	
	\begin{figure}[H]
		\centering
		\includegraphics{img/electromagnetism/refraction_law.jpg}
	\end{figure}
	where:
	
	are the phase velocity of light in the medium $M_n$ and $M_m$.
	
	We can observe on the figure above that the incident rays are refracted on the other side of the axis perpendicular to the interface. This is a typically characteristic of material having a positive refraction index. But it is physically possible to build since the 1990s artificial  composites  "meta-materials" having negative refraction  index:
	\begin{figure}[H]
		\centering
		\includegraphics{img/electromagnetism/positive_negative_refraction_index_material.jpg}
		\caption{Positive and Negative refraction index material behaviour}
	\end{figure}
	
	The writing of the two previous relations:
	
	Developing the values of $\overline{AM}$ and $\overline{MB}$ we get the following dependence of $t$ as a function of the position $x$ of $M$:
	
	According to Fermat's principle, the path taken by the light is the one for which $t$ is minimum. The extremum of $t (x)$ is reached when the derivative with respect to $x$ is zero. Therefore:
	
	Notice that:
	
	The condition of an extremum time brought by the light is then expressed by:
	
	It is sufficient that the angles of incidence and refraction meet that condition so that the path travelled by light is actually the one that takes the least time.
	
	Hence we get the relation, known as the "Snell's law" (which is not anymore a law as we just proved it):
	

	We write more frequently the Snell-Descartes law in physics as following:
	
	\begin{tcolorbox}[title=Remarks,colframe=black,arc=10pt]
	\textbf{R1.} We will see during our study of wave optics that we can fall back (prove) the same relation but without the assumptions bases of geometrical optics. Therefore, the latter relation is named "Descartes-Snell relation " or just "Snell's law".\\
	
	\textbf{R2.} When we speak of the relative refractive index of a given medium without reference to another medium, the default environment is the vacuum.\\
	
	\textbf{R3.} Some materials do not have an absolute isotropic index of refraction : it then depends on the direction of propagation and the polarization state of light. This property is named "\NewTerm{birefringence}\index{birefringence}".
	\end{tcolorbox}
	Let us now study the relation between the relative refractive index and the phase velocity of light in the various medium through which it passes.

	A light ray connects two points $A_1$ and $A_2$ located on either side of an interface $S$ between the two mediums. This light ray is not shown in the figure below. Only are drawn the path located on either side of the light ray that satisfy the extremum (we rely now on the study of the maximum path length). By hypothesis, they are extremely close, so the distance $\overline{PQ}$ is very small:
	
	We will admit that they correspond to the same travel time.
	
	\begin{figure}[H]
		\centering
		\includegraphics{img/electromagnetism/phase_speed_refraction_index.jpg}
		\caption{Figure allowing to connect phase velocity and refractive index}
	\end{figure}
	Since both paths are very close, we can assume the equality of the distances $\overline{A_1B_1}$ and $\overline{A_1Q}$ one hand, and of $\overline{PA_2}$ and $\overline{B_2A_2}$ of the other one. Thus, by hypothesis:
	
	But, under the same hypothesis:
	
	such that:
	
	The "\NewTerm{refraction law}\index{refraction law}" is finally stated in general as following:
	
	And about the reflection angle, as we have already stated, it remains equal to the angle of incidence if the reflective surface is perfectly smooth and flat.
	
	If we consider the following writing:
	
	and the case where $n_1>n_2$ (for example passage from water to air). Then, for values close to $1$, that is to say for grazing angles of incidence (incident beam close to the surface), the Snell's law gives a value greater than $1$. We then get out of the validity domain of the law. This corresponds to situations where there is no refraction but only for reflection, then we speak of "\NewTerm{total reflection}\index{total reflection}\label{total reflection}":
	\begin{figure}[H]
		\centering
		\includegraphics[scale=0.5]{img/electromagnetism/critical_angle.jpg}
		\includegraphics[scale=0.6]{img/electromagnetism/critical_angle_water_air.jpg}
		\caption{Critical angle schematic illustration}
	\end{figure}
	To find the critical angle, we find the value for $\theta_i$ when $\theta_r$ is equal to $\pi/2$ and thus $\sin(\theta _r)=1$. The resulting value of $\theta _i$ is equal to the critical angle $\theta_c$.

	Now, we can solve for $\theta_i$, and we get the equation for the critical angle:
	
	In the case of a water-air boundary where $n_2 = 1.33$ and $n_1=1$ then the critical angle is $\theta_c = 48.7^\circ$.
	
	So in the real life we can get such things:
	\begin{figure}[H]
		\centering
		\includegraphics{img/electromagnetism/total_reflection_water_air.jpg}
		\caption[]{Air-Water total reflection example}
	\end{figure}
	\begin{tcolorbox}[title=Remark,colframe=black,arc=10pt]
	Total internal reflection is used to carry light in fiber optics as we will see later below when introducing the mathematics of these information transmitters. The critical angle for a diamond-to-air surface is only $24.4^\circ$, and so when light enters a diamond, it has trouble getting back out. Although light freely enters the diamond, it can exit only if it makes an angle less than $24.4^\circ$ when he enters it.
	\end{tcolorbox}
	Fermat's principle therefore has obvious similarities with the principle of least action in that it consists of a minimum principle. Even if a rigorous description of light requires the introduction of Quantum Mechanics, it is however possible to apprehend it with the tools of Analytical Mechanics and to apply on it, under certain assumptions, the principle of least action. We will prove just now that we then fall back on the Fermat's principle.

	The calculations that we will present, introduce many hazardous assumptions, but this process should be considered as an approximation. It must be known to the reader that Fermat's principle also carries an approximation that we can qualify of "classical limit".
	
	Now let us Imagine for the proof that light is composed of material "grains" . We must admit that these grains obey to rather unusual physical properties: its mass is zero since according to the classical description, the light rays are not deflected by the gravitational field. This lack of mass thus makes them insensitive to the Earth's gravitational field (be careful and don't forget: we are in a "classical" description of the light as it was done in 19th century and before!!!!).

	Let us write the action for one of these grains of light (\SeeChapter{see section Analytical Mechanics page \pageref{euler lagrange}}):
	
	Now by assuming that the only existing potential field $V$ is that one that derives from the gravitational field and that we assume that light as it is insensitive to it (we know in General Relativity that this is wrong but we said just now that we would stay in classical point of view!!!), it follows that the action of light can be written:
	
	But no force applied to the light, so the kinetic energy $T$ is a constant of motion. Applying  the variational principle of least action we then have:
	
	Hence we get:
	
	This equation means that the time taken by the light along its path is minimum (or more generally, is an extremum). We then fall back indeed on Fermat's principle. So we have proved that the under the classical limit  Fermat's principle follows directly from the principle of least action.
	
	\begin{tcolorbox}[colback=red!5,borderline={1mm}{2mm}{red!5},arc=0mm,boxrule=0pt]
	\bcbombe Caution! The Fermat's principle doesn't explain anything, even if it's accurate that light follow the shortest path! That's why most textbooks introducing refraction as we did above geometrically (named sometimes the "marching soldier analogy", but wrong as nothing explain why the solders don't continue you walk straight) or using the variational principal (ie "Fermat's principle") are quite misleading. Using Huygens principle as we will see further below (see page \pageref{Huygens principle}) is also wrong! In fact we should use Maxwell equation to understand better what happens physically!
	\end{tcolorbox}

	\pagebreak
	\subsubsection{Rainbow}
	A rainbow is an optical and meteorological phenomenon that makes the spectrum of light visible  when the Sun shines through the rain and the observer looks at the sky in a direction opposite to that of the Sun. It is a coloured arc with red on the outside and purple inside.
	\begin{figure}[H]
		\centering
		\includegraphics[scale=0.15]{img/electromagnetism/rainbow.jpg}
		\caption[Rainbow after the rain]{Rainbow after the rain, Grodno, Belarus (source: Wikipedia)}
	\end{figure}
	The rainbow is caused by the scattering of sunlight by raindrops approximately spherical. The light is first refracted by penetrating the surface of the drop, then undergoes a partial reflection at the back of the drop and is refracted again on leaving (see figure below). The overall effect is that the incoming light is mainly refracted back at an angle of approximately $40-42^\circ$  (see proof below), regardless of the size of the drop. The precise value of the refraction angle depends on the wavelength (color) of light components as we know. 

	In the case of entry in a refracting medium, the angle of refraction of the blue light is lower than that of red light (phenomenon highlighted in the study of triangular prisms). Thus, after reflection at the water-air interface, the blue light comes out of a drop above the red light (see again figure below). The observer being fixed (at rest), he sees light from different drops of water with different angles with respect to the sunlight. Red therefore appears higher in the sky than the blue.
	
	Sometimes a less bright second rainbow can be seen above the primary rainbow. It is caused by a double reflection of sunlight inside the raindrops and appears in a $50-53^\circ$ angle still in the opposite direction to the Sun. Due to the additional reflection, the colors of the second arc are reversed relative to the primary arc, with the blue outside and red inside, and the arc is less bright. That is why it is more difficult to observe. 
	
	\begin{fquote}[Richard Dawkins]Understanding how a rainbow works doesn't make it less beautiful, it makes it more beautiful. That is Science!
 	\end{fquote}
 	
	\begin{figure}[H]
		\centering
		\includegraphics[scale=0.24]{img/electromagnetism/double_rainbow.jpg}
		\caption[]{Double rainbow and supernumerary rainbows on the inside of the primary arc. The shadow of the photographer's head on the bottom marks the antisolar point (source: Wikipedia, Eric Rolph)}
	\end{figure}
	A third rainbow sky may be present near the second, and inverted with respect thereto (ie, identical to the first)...
	
	However, third rainbow is much less bright and observable only under exceptional conditions. In practice it is not very easy to distinguish supernumerary arcs associated with the secondary arc. It corresponds to the light rays having undergone five reflections in the water drops. 

	Two rainbows inverted with respect to each other can also be observed in the opposite direction, about $45^\circ$ from the Sun (therefore in the direction of it), but this is particularly difficult because of the proximity to the Sun . The few observations of these two arcs mention pieces of rainbows visible intermittently. These two rainbows correspond to light rays having undergone three and four reflections in the water drops. As they are located facing the Sun, it is not the same drops of water that contribute to it. In practice, favourable configurations to their observation are much fewer than those who favour the observation of the secondary arc, especially because of their proximity to the Sun.

	To study the phenomenon let first consider the spherical water drop below with an incident beam of light (one refraction, two reflections) we represented the red and purple component (the angles are approximate) and the refractive indices water and air:
	\begin{figure}[H]
		\centering
		\includegraphics[scale=1]{img/electromagnetism/spherical_wather_drop_generating_rainbow.jpg}
		\caption[]{Spherical water drop generating the rainbow sky}
	\end{figure}
	We seek to determine the angle between the incoming light ray (beam considered to contain all components of visible light) and the outgoing light beam (opposite the Sun: antisolar). Thus, the angle difference for two colors give us the angle by which we must change our view to observe two different colors in the rainbow sky.
	\begin{tcolorbox}[title=Remark,colframe=black,arc=10pt]
	There is no sense to me to calculate the angle we must look at relatively with the ground (assumed plane) to observe a rainbow as some books mention it. Indeed, anyway if we turn our eyes to a rainbow sky will seen anyway over a large angle from the ground. The only thing that really makes sense, is the angle difference between two "monochromatic" colors.
	\end{tcolorbox}
	For this study, we will consider the following approximate figure:
	\begin{figure}[H]
		\centering
		\includegraphics[scale=1]{img/electromagnetism/rainbow_technical_study.jpg}
		\caption[]{Path travelled by light in a spherical water drop}
	\end{figure}
	With the Snell-Descartes law we have initially:
	
	What interests us here is the angle of apparent reflection $2\delta$, which we will denote $D$ (careful! Some teachers choose the convention $D=\pi-2\delta$). To determine this, we start from the following relation of the triangle $ABE$:
	
	Therefore it comes:
	
	Hence:
	
	What we will write finally:
	
	and as:
	
	Therefore it comes:
	
	which is sometimes abusively written:
	
	If we make a practical application we have for red $750$ [nm] with for example an incidence angle of $30^\circ$ in the spherical water drop:
	
	and for purple $400$ [nm] with the same angle of incidence of $30^\circ$ in the spherical water drop:
	
	Thus an angle difference of about $2.4^\circ$.
	
	Finally, we could interest us in the angle $\alpha$ at which the angle $D$ is maximum (which corresponds to the angle of the most visible rainbow sky in term of size in reality). We start then of:
	
	and we seek the solutions of:
	
	with $\alpha\in[0,\pi/2]$ and $n\in]0,1[.$

	Let us recall that we have proved in the section of Differential and Integral Calculus that:
	
	Then we have:
	
	Hence:
	
	From there we get:
	
	that is to say:
	
	as we seek for solutions with $\alpha\in[0,\pi/2]$ and that the cosine is positive on this interval.

	Therefore:
	
	He then comes for $n\cong0.746$ (purple):
	
	This corresponds fairly well to the reality (the angle at which we raise our heads to see the most visible / larger part of the rainbow).
	
	The corresponding deviation is then:
	
	and is named "\NewTerm{rainbow angle}\index{rainbow angle}" for purple:
	\begin{figure}[H]
		\centering
		\includegraphics[scale=1]{img/electromagnetism/rainbow_angle.jpg}
		\caption[Rainbow angle]{Rainbow angle (source: Wikipedia)}
	\end{figure}
	For the blue we get:
	
	So the average gives the famous $42^\circ$ that we can read in many books.
	
	Thus, the light rays received by the observer and in which the red (outer edge of the rainbow) dominates correspond to all the rays coming from the wall of rain and making angle of about $40^\circ$ with the direction of the sunlight (see figure below). The light rays forming each color of the rainbow sky then form the top of the cone observer's eyes and for axis the solar ray passing through the eyes of the observer:
	\begin{figure}[H]
		\centering
		\includegraphics[scale=1]{img/electromagnetism/rainbow_schematic_summary.jpg}
		\caption[]{Figure of the generation of the primary rainbow (source: Culture ENS-Sciences)}
	\end{figure}
	That gives in real life:
	\begin{figure}[H]
		\centering
		\includegraphics[scale=0.9]{img/electromagnetism/full_arc_rainbow.jpg}
		\caption[]{Source: ed g2s, Christophe Afonso}
	\end{figure}
	
	\pagebreak
	\subsubsection{Cherenkov radiation}
	Cherenkov radiation, also known as Vavilov–Cherenkov radiation, is electromagnetic radiation emitted when a charged particle (such as an electron) passes through a dielectric medium at a speed greater than the phase velocity of light in that medium. The characteristic blue glow of an underwater nuclear reactor is due to Cherenkov radiation. It is named after Soviet scientist Pavel Alekseyevich Cherenkov, the 1958 Nobel Prize winner who was the first to detect it experimentally. A theory of emitted corresponding spectrum was later developed within the framework of Einstein's special relativity theory by Igor Tamm and Ilya Frank, who also shared the Nobel Prize. Cherenkov radiation had been qualitatively predicted by the English polymath Oliver Heaviside in papers published in 1888–89.
	
	\begin{figure}[H]
		\centering
		\includegraphics[scale=0.19]{img/electromagnetism/cherenkov_radiation.jpg}
		\caption{Oak Ridge National Laboratory Nuclear Reactor}
	\end{figure}
	\begin{tcolorbox}[title=Remark,colframe=black,arc=10pt]
	Sometimes some wonder why charged particles can go faster than light in a medium other than a vacuum. It is simple to the point: even if the two particles meet roughly the same obstacles and difficulties to spread in a medium the photon cannot be accelerated by a pulse when a charged particle can be accelerated by a given phenomenon in a given medium.
	\end{tcolorbox}
	We saw in the preceding paragraphs the hypothesis (relatively intuitive) than the phase propagation speed of light in an absolute refractive index medium $n$ was not equal to $c$ but still less by writing this:
	
	The Cherenkov effect is (basically) a phenomenon similar to that of an (acoustic) shock wave, but producing a flash of light instead of sound, which takes place on the path of a charged particle moving in a medium with a higher phase velocity that of the speed of light in that medium (rigorous explanation is beyond the scope of study of this book because of its complexity treatment!).

	Indeed, let us first recall that we proved in the section Electrodynamics that any moving charged particle emits electromagnetic radiation. Then we have proved in the preceding paragraphs that the speed of light in a given medium depends on the absolute refractive index $n$ of the medium (hypothesis which is verified by the experimental accuracy of the theoretical developments arising therefrom).
	
	So we have two basic informations:
	\begin{enumerate}
		\item The speed of the charged particle that can be written as follows with the traditional relativistic notations:
		

		\item The phase velocity of light in a medium with absolute refractive index given:
		
	\end{enumerate}
	It is easy to see that to get $v>v_c$ we must have:
	
	Therefore:
	
	Some authors prefer to compare the distance travelled by the light in relation to the distance travelled by the particle. It comes that as:
	
	And therefore for the particle traverses distances equal to those of light at in the same time interval we must have that $\beta=1/n$. Beyond appears the Cherenkov effect.
	
	\pagebreak
	\subsection{Descartes' Formulas}
	We have previously discussed some phenomena that occur when a wavefront passes from one medium to another in which the propagation is different. Not only have we analysed what becomes the wavefront, but we have introduced the concept of "radius" which is particularly useful for geometric constructions. We now propose to deepen the phenomena of refraction and reflection of a geometric point of view using the concept of radius as the tool to describe the processes taking place in the discontinuity surfaces of the propagation. We also assume that the process is limited to reflections and refractions, no other changes affecting the wave surfaces.

	This geometry treatment is correct until surfaces and discontinuities encountered by the wave during its propagation are very large relatively to the wavelength. As long as this condition is met, the treatment applies both to light waves, sound (in particular ultrasound - very high frequency), seismic waves, etc.

	We begin by considering the wave reflection on a spherical surface. For this purpose we must first given some obvious definitions: The center of curvature $C$ (\SeeChapter{see section Differential Geometry page \pageref{center of curvature}}) is the center of the spherical surface of the figure below and the top O is the pole of the spherical cap.
	
	\textbf{Definition (\#\mydef):} The line passing through the pole cap O and the center of curvature $C$ is named the "\NewTerm{optical axis}\index{optical axis}".
	
	If we take O as coordinate origin, all measured quantities to the right of O will be taken as positive, all those on the left as negative !!!
	
	\begin{figure}[H]
		\centering
		\includegraphics{img/electromagnetism/descartes_formulas_optical_axes.jpg}
		\caption{Representation of the optical axis concept}
	\end{figure}
	Let us suppose that the point $P$ is a source of spherical waves. The radius $\overline{PA}$ gives by reflection the radius $\overline{AQ}$ and, as the angles of incidence and reflection are equal with respect to the perpendicular $\overline{AC}$ to the surface (as we have already observed it in our study of refraction), we see in the figure that:
	
	hence:
	
	Assuming that the angles $\alpha_1,\alpha_2$ and $\beta$ are very small, that is to say, the rays are "\NewTerm{para-axial}\index{para-axial}" and the source is very far or that the sensor is very small compared to the source, we can write with a good approximation with a Maclaurin  development (\SeeChapter{see section Sequences and Series page \pageref{usual maclaurin developments}}) for small angles:
	
	Substituting these approximations of $\alpha_1,\alpha_2$ and $\beta$ into $\alpha_1+\alpha_2=2\beta$, we get:
	
	which is the "\NewTerm{Descartes formula for reflection on a concave spherical surface}". It involves, in the approximation used to establish it, that for all incident rays passing through $P$ will pass through $Q$ after reflection on the surface. We can then say that $Q$ is "\NewTerm{the image of the object}\index{image of an object}" $P$.
	
	In the special case where the incident beam is parallel to the optical axis, that is equivalent to place the object at a great distance of the lens, we have $p=+\infty$. The Descartes formula for reflection on a concave spherical surface becomes therefore:
	
	and the image is formed at the point $f$ named the  "\NewTerm{focal}\index{focal}" and its distance from the lens given by:
	
	and is named "\NewTerm{focal length}\index{focal length}". We also get the ratio $r/2$ if we make $q$ tend to infinity (that means the curvature radius is put to an infinite value).
	
	So the bigger is $r$, the bigger is $C$ (smaller curvature = wider angle photo) and therefore the bigger is $q$.

	The relation obtained above is also valid for a convex surface. Indeed, we simply need pull the lines representing the light rays beyond the concave surface to see that the object of study is the same to a given symmetry:
	\begin{figure}[H]
		\centering
		\includegraphics{img/electromagnetism/descartes_formulas_extension.jpg}
	\end{figure}
	The only difference between the concave and convex surface is that in the case of the convex surface, the reflected image of the object appears as if it is behind the surface (at the equivalent of to the point $P$). This leads us to define the following terminology:
	
	\textbf{Definitions (\#\mydef):}
	\begin{enumerate}
		\item[D1.] A "\NewTerm{virtual image}\index{virtual image}" is a term used in optics to refer to any image formed before the exit face of an optical instrument (in the direction of travel of light) and therefore will not be displayed on the projection screen. For a thin converging spherical lens an object placed between the object focus and the optical center of the lens will give a right (not inversed) virtual image. This is particularly the case of an optical system used as a magnifying glass, which provides a magnified image of the object observed through the lens.

		\item[D2.] A "\NewTerm{real picture}\index{real picture}" is a term used in optics to refer to any image formed after the exit face of an optical instrument (in the direction of travel of light). For a thin spherical converging lens an object placed against the object focus of the lens will give a real (inversed) image.
	\end{enumerate}
	\begin{figure}[H]
		\centering
		\includegraphics[scale=0.9]{img/electromagnetism/real_and_virtual_image.jpg}
		\caption{Why image are reverse when passing through a thin spherical lens (for simplification purposes the focal in the drawing above is supposed to be inside the lens...)}
	\end{figure}
	In the context of flat mirrors\footnote{When we stand in front of a mirror, our mirror image flips right with left, left with right. But never top with bottom, bottom with top. The answer is, this is an act of our brain. Our brain always creates a frame of reference with a vertical axis. Hence, we see things flipping vertically. This is because our brain is conditioned and trained to always flip images in the direction which demands less distortion.} the meaning is quite different:
	\begin{figure}[H]
		\centering
		\includegraphics[scale=0.6]{img/electromagnetism/real_and_virtual_image_mirror.jpg}
	\end{figure}
	obviously for flat mirrors the distance $p$ of the real object to the mirror will have virtually the same distance $q$ that has the "virtual object" to this same mirror so that virtually we could write $p=q$. For flat mirror the image is obviously upright and it produces and apparent left-right reversal!
	
	\subsubsection{Stigmatism}
	If the opening (wide angle) of the mirror is large, so it receives steeply inclined rays, the above Descartes formula isn't anymore, we know it by construction, a good approximation. There is in this case no more well defined punctual image of a "point object" but an infinite number of them: consequently the image of a large object appears blurred since the images are superimposed . This effect is named "\NewTerm{spherical aberration}\index{spherical aberration}" and part of the optical axis which contains all the reflected images is then named the "\NewTerm{caustic reflection}\index{caustic reflection}". The spherical aberration can not be eliminated, but a suitable design of the surface allows to remove it for certain positions on the optical axis named "\NewTerm{stigmatic}\index{stigmatism}". For example, in our previous study case, it is clear (by geometrical construction) that if we put $P$ on $C$ then the point C becomes the stigmatic points. We say then that's the point "\NewTerm{strictly stigmatic}\index{strict stigmatism}".
	
	By cons, for the parabolic mirror all the rays converge on the focus of the mirror where is concentrated the light energy received by the mirror. Conversely, we place the filament of a lamp at the focus of a parabolic mirror to get far-reaching headlights (typically lighthouse having no Fresnel lens). We also give a parabolic shape to antennas needing to receive radio waves. For television broadcast by satellite as it works at the centimetre waves (frequency of several GHz) a focal distance of one meter is suitable for the antenna (ie. this applies to telescopes and radio telescopes).
	
	\begin{figure}[H]
		\centering
		\includegraphics{img/electromagnetism/stigmatism.jpg}
		\caption{Representation of the concept of stigmatism}
	\end{figure}
	The idea to prove mathematically that the focus of the parabola is the rigorous stigmatic point is this:

Let us recall the following figure we used in our study of conical in the section of Analytical Geometry:
	\begin{figure}[H]
		\centering
		\includegraphics{img/electromagnetism/stigmatism_figure_proof.jpg}
	\end{figure}
	We have added on the figure the point $\mu$ that is the orthogonal projection of the point $M$ (point of incidence of the light beam) and also the tangent to the parabola at the point $M$. If we can prove that the tangent to $M$ is the mediator of segment $\overline{F\mu}$, then we also prove that the angle of incidence and reflection are equal. Therefore as we already know that incidence and reflection angles are equal, then we prove indirectly that on any point $M$ of the parabola the incident ray comes on $F$.
	\begin{dem}
	Let us consider the equation of the parabola relatively to the focal as proved in the section of Analytical Geometry:
	
	We have also proved in the section of Analytical Geometry that relatively to $\Omega$ the Focal is at the position $F(h/2,0)$ and the direction had for equation:
	
	We get the equation of the tangent at $M(x_0,y_0)$ by the derivative at the same point (caution! ... remember the particular orientation of the parabola!):
	
	Which can also be written:
	
	and knowing that:
	
	So we get the equation of the tangent:
	
	On of the vector of the tangent is therefore as proved in the section of Analytical Geometry during our study of the parametric equation of a the line:
	
	where in our case of the parabola, $p$ is equal to $h$.
	
	On the other hand, we have (this is easily verified by taking $x_0=0$):
	
	So we have the scalar product:
	
	as the vectors $\overrightarrow{MF}$ and $\overrightarrow{M\mu}$ have the same norm by definition of the parabola in Analytical Geometry\footnote{Yes, for recall a "parabola" is the set of all points which are equidistant from a point, called the focus, and a line, named the "directrix"}, we deduce that the vector $\vec{v}$ (giving the direction of the tangent) leads the line bisecting the angle of the $\overrightarrow{MF}$ and $\overrightarrow{M\mu}$ and therefore by extension that the tangent $M$ is the mediator of $\overline{M\mu}$.
	\begin{flushright}
		$\blacksquare$  Q.E.D.
	\end{flushright}
	\end{dem} 
	\begin{figure}[H]
		\centering
		\includegraphics[scale=0.6]{img/electromagnetism/antenna.jpg}	
		\caption[Parabolic antenna for reception or emission]{Parabolic antenna for reception or emission (source: OpenStax)}
	\end{figure}
	Let us also calculate the focus position of a parabolic reflector whose equation is given by (\SeeChapter{see section Analytical Geometry page \pageref{paraboloid}}):
	
	We fix for example $x$ to $0$ then it remains:
	
	After rearranging we get:
	
	As:
	
	we get immediately:
	
	And we have prove in the section of Analytical Geometry that:
	
	A circular paraboloid is theoretically unlimited in size. Any practical reflector uses just a segment of it. Often, the segment includes the vertex of the paraboloid, where its curvature is greatest, and where the axis of symmetry intersects the paraboloid. However, if the reflector is used to focus incoming energy onto a receiver, the shadow of the receiver falls onto the vertex of the paraboloid, which is part of the reflector, so part of the reflector is wasted. This can be avoided by making the reflector from a segment of the paraboloid which is offset from the vertex and the axis of symmetry:
	\begin{figure}[H]
		\centering
		\includegraphics[scale=0.5]{img/electromagnetism/off_axis_satellite_dish_construction.jpg}	
		\caption[Off-axis receiver dish construction principle]{Off-axis receiver dish construction principle (source: Wikipedia)}
	\end{figure}
	The receiver is still placed at the focus of the paraboloid, but it does not cast a shadow onto the reflector. The whole reflector receives energy, which is then focused onto the receiver. This is frequently done, for example, in satellite-TV receiving dishes, and also in some types of astronomical telescope:
	\begin{figure}[H]
		\centering
		\includegraphics[scale=0.7]{img/electromagnetism/off_axis_satellite_dish.jpg}	
		\caption{Off-axis satellite dish}
	\end{figure}
	
	\pagebreak
	\subsubsection{Lenses}
	Let us now do a similar study to that carried out earlier, with the same properties of symmetry and defects, but the "\NewTerm{spherical dioptres boundaries}\index{spherical dioptres boundaries}" that is a spherical boundary between to mediums (interesting results regarding the study of the eye). The results will be helpful to us before tackling the study of spherical lens (the traditional magnifying glass).
	
	So we will now consider the refraction of the light during the passage through a spherical surface separating two medium with absolute refractive indices $n_1$ and $n_2$ (see figure below):
	\begin{figure}[H]
		\centering
		\includegraphics{img/electromagnetism/spherical_dioptre.jpg}
		\caption[A biconvex lens in real life]{A biconvex lens in real life... (source: Wikipedia)}
	\end{figure}
	where for recall the center of curvature $C$ (\SeeChapter{see section Differential Geometry page \pageref{center of curvature}}) is the center of the spherical surface of the figure below and the top O is the pole of the spherical cap.

	The basic geometric elements are the same as those specified for spherical surfaces. We therefore consider initially a concave refractive surface (concave dioptre) and observing that the "\NewTerm{object distance}\index{object distance}" is located at the opposite of the other points, we must adopt a sign convention to put this observation in evidence in the equations. Thus, $q$ will be defined as a negative value.

	An incident ray such as $\overline{PA}$ is refracted as following $\overline{AQ}$ and thus cross the optical axis $Q$. We see from the figure that:
	
	We have the Snell-Descartes' law:
	
	and we will admit as for spherical surfaces that the ray a only little bit inclined rays. Under these conditions the angles $\theta_i,\theta_r,\alpha_1,\alpha_2$ and $\beta$ are very small and we can write using the Maclaurin series expansion (\SeeChapter{see section Sequences and Series page \pageref{usual maclaurin developments}}):
	
	so that the Snell-Descartes' law is now written:
	
	From the figure, we can do the approximations:
	
	so that by substituting in the approximation of the Snell-Descartes' law we find elementary simplification:
	
	hence for a concave surface:
	
	The "\NewTerm{object focal $F_0$}\index{object focal}" also named "\NewTerm{first focal point}\index{first focal point}" of a refractive spherical surface is the position of a punctual object on the optical axis such that the refracted rays are parallel to the optical axis, which is equivalent to form the image of the point at infinity, where $q=\infty$.

	The distance of the object to the spherical surface is then named "\NewTerm{focal distance object}\index{focal distance object}" and we denote it by $f_0$. By putting $p=f_0$ and $q=\infty$, then we have for the concave  case:
	
	The focal length $f_0$ is positive and the system is the convergent when the object focus is real, is placed in front of the spherical surface. When the object focus is virtual, the focal length $f_0$ is negative and the system is then divergent.
	
	Similarly, if the incident rays are parallel to the optical axis, which is equivalent as to have a very distant object form the spherical surface $p=\infty$, the refracted rays pass through a point $F_i$ of the optical axis named "\NewTerm{image focus}\index{image focus}" or "\NewTerm{second focal point}\index{second focal point}" (again with the same problems of stigmatims).
	
	In this case the distance of the spherical surface to the image is named "\NewTerm{focal image distance}\index{focal image distance}" and we denote it by $f_i$. By putting $p=\infty$ and $q=f_i$ then we have for the concave case:
	
	\begin{figure}[H]
		\centering
		\includegraphics{img/electromagnetism/focals_schemes_dioptre.jpg}
		\caption{Position of first and second focal points in a dioptre}
	\end{figure}
	By mixing the two previous relations, we have a useful result in practice:
	
	So if we know the relative refractive index and one of the focal lengths, we can infer the other. Or (and we can of course make also other combinations):
	
	Now let us take look at the type of reflective surfaces and refractions we were waiting for: the lenses!
	
	\textbf{Definition (\#\mydef):} A "\NewTerm{lens}\index{lens}" is a transmissive optical device that focuses or disperses a light beam by means of refraction. A simple lens consists of a single piece of transparent material (generally spherical), while a compound lens consists of several simple lenses (elements), usually arranged along a common axis. Lenses are made from materials such as glass or plastic, and are ground and polished or moulded to a desired shape. A lens can focus light to form an image, unlike a prism, which refracts light without focusing. Devices that similarly focus or disperse radiation other than visible light are also named "lenses", such as microwave lenses, electron lenses or acoustic lenses.
	\begin{figure}[H]
		\centering
		\includegraphics[scale=0.5]{img/electromagnetism/biconvex_lens.jpg}
		\caption{Concept of spherical dioptre}
	\end{figure}
	The study of optical lenses is a huge and subtle subject and our only purpose in this book is to give the mathematical basic properties of some elementary lenses shapes. As we will show it through some illustration, even traditional camera have already some huge complications to get qualitative results for the professional photographer as illustrated in the figure below:
	\begin{figure}[H]
		\centering
		\includegraphics[scale=0.55]{img/electromagnetism/zoom_nikkon_af_s_80_200_mm_f_2dot8D_if_ed.jpg}
		\caption[Zoom-Nikkor AF-S 80-200 mm f/2.8D IF-ED]{Zoom-Nikkor AF-S 80-200 mm f/2.8D IF-ED (source: \url{http://www.pierretoscani.com}, author: Pierre Toscani)}
	\end{figure}
	or even better (optical system also existing for smartphones!):
	\begin{figure}[H]
		\centering
		\includegraphics[scale=0.55]{img/electromagnetism/fish_eye_lens.jpg}
		\caption[Fisheye-Nikkon 6 mm f/2.8 simplified cut]{Fisheye-Nikkon 6 mm f/2.8 simplified cut (source: \url{http://www.pierretoscani.com}, author: Pierre Toscani,)}
	\end{figure}
	An incident ray thus undergoes two refractions at the crossing of the lens. Assume for simplicity that the media of both sides of the lens are identical and their absolute refraction index equal to $1$ (air or vacuum for example). We will also consider only thin lenses in this book, that is to say whose thickness is very small compared to the radii of curvature:
	\begin{figure}[H]
		\centering
		\includegraphics{img/electromagnetism/lens_technical_scheme.jpg}
		\caption{Technical representation of a lens}
	\end{figure}
	The optical axis is the line determined by the two centers $\overline{C_1C_2}$. We seek to establish a relation between the position of $P$ and $Q$ from easily measured physical parameters!

	We will consider for the analysis that the image formed after the refraction on the first surface is the object for the refraction on the second surface.

	Let us consider the incident ray $\overline{PA}$ through $P$. When passing the first surface, the incident ray is refracted following the radius $\overline{AB}$ and continues virtually to $Q'$ according to the behaviour of a convex spherical surface. So we apply the relation prove just earlier above for the convex diopter:	
	
	In $B$ the ray undergoes a second refraction and becomes the radius $\overline{BQ}$ according to the behaviour of a concave spherical surface. We can imagine that the incident beam in $B$ come from a virtual point $P'$ (not show in the figure above) that is on the optical axis and immersed in the material virtually extended to the right of the lens (that is the difficult side of this proof ... you have to imagine that!).

	So we have to apply the relation proved earlier above for the concave refractive surface (dioptre) but by being careful this time to the order of absolute refractive indices (subtle trap!) and by imagining that the point:
	
	As we consider the lens is surrounded by air (unitary refractive index), then we have:
	
	We will now assume that the thickness of the lens tends to zero. In other words his two radii tend to infinity. Then we have:
	
	By identifying term to term and remembering that what is left from the origin O is negative, then we have:
	
	while subtly ignoring the other two terms ... (we understand now easily why this proof is often omitted in the literature ...).

	Therefore, both prior previous relations become:
	
	Summing it comes finally:
	
	and that is often denoted in the following form, name "\NewTerm{Descartes's first formula for thin lenses}\index{Descartes's first formula for thin lenses}" or simply "\NewTerm{thin lens equation}\index{thin lens equation}":
	
	While being careful to rehabilitate the figure above such that it becomes:
	\begin{figure}[H]
		\centering
		\includegraphics{img/electromagnetism/descartes_first_formula_thin_lens_figure.jpg}
		\caption{Thin lens equation corresponding technical figure}
	\end{figure}
	By writing this formula it should be applied to $r_1,r_2$ the sign convention we have set, that is to say, the rays are positive for a concave surface and negative for a convex surface, as seen from the side on which the light hits the lens. Thus, if the two radii are the same, we have:
	
	The right term of the thin lens equation is only a constant depending on the physical characteristics of the lens and it customary to name it "\NewTerm{refractive power}\index{refractive power}" or "\NewTerm{dioptric power}\index{dioptric power}" and whose unity is the "diopter" and denoted:
	
	Converging lenses have positive optical power, while diverging lenses have negative power. When a lens is immersed in a refractive medium, its optical power and focal length change.
	
	The following figure can help to understand (play with the values of $q,p,r$ in the previous relations to understand it completely):
	\begin{figure}[H]
		\centering
		\includegraphics{img/electromagnetism/eye_adaptative_power_vision.jpg}
		\caption{Human eye adaptative refractive power mechanism}
	\end{figure}
	\begin{tcolorbox}[title=Remark,colframe=black,arc=10pt]
	An eye that has too much or too little refractive power to focus light onto the retina has a refractive error. A myopic eye has too much power so light is focused in front of the retina. Conversely, a hyperopic eye has too little power so when the eye is relaxed, light is focused behind the retina. An eye with a refractive power in one meridian that is different from the refractive power of the other meridians has astigmatism. Anisometropia is the condition in which one eye has a different refractive power than the other eye.
	\end{tcolorbox}
	The point O in the previous figure, is selected to coincide with the "optical center" of the lens. As we know, the optical center has the property of being such that any ray passing through it exit parallel to the direction of the incident ray !! This is an important property because any point of an object lying on one side of a lens (whatever which one because of the symmetry of the lens) will emit light from which some rays will pass through the optical center. Thus allowing to have similar triangles on the left and right of the symmetry axis of the lens and apply Thales' theorem (\SeeChapter{see section Euclidean Geometry page \pageref{thales theorem}}) to calculate the concept of "magnification" (see further below).

	To prove that such a point exists let us consider in the lens below (with horizontal and vertical symmetry):
	\begin{figure}[H]
		\centering
		\includegraphics{img/electromagnetism/optical_center_property.jpg}
		\caption[]{Figure representing the property of the optical center of a lens}
	\end{figure}
	 Let us consider two parallel curvatures radius:
	
	generators of the diopters (elements of the thin spherical lens for recall...) chosen such that the corresponding tangent planes $T_1$ and $T_2$ are also parallel.

	For the radius $\overline{R_1A_1}$, which direction is such that it is refracted following $\overline{A_1A_2}$, the emerging ray $\overline{A_2R_2}$ is parallel to $\overline{A_1R_1}$ by the horizontal symmetry of the lens. Thus the triangles $C_1A_1\text{O}$ and $C_2A_2\text{O}$ are similar regardless of the "generators radius", thus we see that the position of the optical center O is satisfied by the relation:
	
	and therefore exists independently of the generating radii.
	
	As in the case of a single diopter, the "\NewTerm{focal object}\index{focal object}" $F_0$, or "\NewTerm{first focal point of a lens}\index{first focal point of a lens}" is the position of the object for which the rays emerge parallel to the optical axis ($q=+\infty$) after have passed through the lens. The distance from the lens to the object focus is named the "\NewTerm{focal distance object}\index{focal distance object}" and we designate it in practice often by the simple letter $f$ for thin lenses.
	\begin{figure}[H]
		\centering
		\includegraphics{img/electromagnetism/thin_lense_first_focal_point.jpg}
		\caption{First focal point (source) is here on the left of the thin lens}
	\end{figure}
	Or for those who cannot see well the figure above, there is another more explicit illustration of what happen with the first focal point of a bi-convex lens:
	\begin{figure}[H]
		\centering
		\includegraphics[scale=1]{img/electromagnetism/first_focal_point_illustration.jpg}
		\caption[First focal point with bi-convex lens behaviour]{First focal point with bi-convex lens behaviour (source: \url{http://www.pierretoscani.com}, author: Pierre Toscani)}
	\end{figure}
	Then by putting $p=f$ and $q=+\infty$ in the equation of spherical thin lenses we get:
	
	We then get the following focal distance object in the form named "\NewTerm{focal length of the lens}\index{focal length of a lens}":
	
	Similarly in the case of a simple diopter, the "\NewTerm{focal image}\index{focal image}" $F_i$, or "\NewTerm{second focal point of a lens}\index{second focal point of a lens}" is where converge the light rays after passing through the lens, but that were before the lens parallel between them and to the optical axis ($p=+\infty$). Thus, given the central spherical symmetry of thin lenses it is simple enough to reverse in our imagination the previous picture to get the concept:
	\begin{figure}[H]
		\centering
		\includegraphics{img/electromagnetism/thin_lense_second_focal_point.jpg}
		\caption{Second focal point (target) is here on the right of the thin lens}
	\end{figure}
	Or for those who cannot see well the figure above, there is another more explicit illustration of what happen with the second focal point of a bi-convex lens:
	\begin{figure}[H]
		\centering
		\includegraphics[scale=0.8]{img/electromagnetism/second_focal_point_illustration.jpg}
		\caption[Second focal point with bi-convex lens behaviour]{Second focal point with bi-convex lens behaviour (source: \url{http://www.pierretoscani.com}, author: Pierre Toscani)}
	\end{figure}
	The distance from the lens to the image focus is then named the "\NewTerm{focal image distance}\index{focal image distance}" and we denote it in practice with the same letter $f$ as by symmetry of the thin lens, by putting $q=f$ and $p=+\infty$ we have:
	
	Therefore, in a thin lens the two foci are located symmetrically on each side of the vertical axes of symmetry of the lens.

	So as in both the inverse of the focal length is equal to the refractive power nothing prevents us to write the equation of thin lenses as we often found in textbooks:
	
	In this form it is then named the "\NewTerm{equation of Opticians}\index{equation of Opticians}" or "\NewTerm{eyewear equation}\index{eyewear equation}" or "\NewTerm{Lens maker equation}\index{lens maker equation}"... In the following simple form that does not make appear the physical properties of the lens (which is often the relation shown in high-school classes and could be the reason why it seems that it has a different name):
	
	we name it "\NewTerm{conjugation equation}\index{conjugation equation}" or "\NewTerm{second Descartes formula for thin lenses}\index{second Descartes formula for thin lenses}" or "\NewTerm{spherical mirror equation}\index{spherical mirror equation}".

	Furthermore, if the focal distance is positive, and thus respectively the dioptric power as well, then the lens is say to be (obviously) a "\NewTerm{convergent lens}\index{convergent lens}":
	\begin{figure}[H]
		\centering
		\includegraphics{img/electromagnetism/thin_convergent_lenses_examples.jpg}
		\caption{Examples of converging lenses (bi-convex, plano-convex, convex meniscus)}
	\end{figure}
	if the focal length is negative, and thus the refractive power respectively also, the lens is say to be a "\NewTerm{diverging lens}\index{diverging lens}":
	\begin{figure}[H]
		\centering
		\includegraphics{img/electromagnetism/thin_divergent_lenses_examples.jpg}
		\caption{Examples of different lenses (bi-convex, plano-convex, concave meniscus)}
	\end{figure}
	Note also the following traditional symbolic representations of lenses (for example in the E-draw software):
	\begin{figure}[H]
		\centering
		\includegraphics{img/electromagnetism/thin_divergent_lenses_examples.jpg}
		\caption{Examples of different lenses (bi-convex, plano-convex, concave meniscus)}
	\end{figure}
	Notice also the following traditional symbolic representations of lenses (for example in the E-draw software):
	\begin{figure}[H]
		\centering
		\includegraphics{img/electromagnetism/lens_technical_softwares_symbols.jpg}
		\caption[]{Examples of different lenses (bi-convex, plano-convex, concave meniscus)}
	\end{figure}
	
	\begin{tcolorbox}[colframe=black,colback=white,sharp corners]
	\textbf{{\Large \ding{45}}Example:}\\\\
	We want to found the radius of a biconcave lens symmetrically ground from a glass with index of refractive $1.55$ so that its focal length in air is $20$ [cm] (for a biconcave lens, both surfaces have the same radius of curvature). To found the answer we use the thin-lens form of the lens maker's equation:
	
	where $r_1<0$ and $r_2>0$. Since we are making a symmetric biconcave lens, we have $|r_1|=|r_2|=r$. Therefore:
	
	Solving for $R$ and inserting $f=-20$ [cm], $n_2=1.55$, and $n_1=1.00$ gives:
	
	\end{tcolorbox}
	
	\paragraph{Optical Magnification}\mbox{}\\\\
	Magnification is the process of enlarging something only in appearance, not in physical size. This enlargement is quantified by a calculated number also named "\NewTerm{magnification}\index{magnification}". When this number is less than one, it refers to a reduction in size, sometimes named "\NewTerm{minification}\index{minification}" or "\NewTerm{de-magnification}\index{de-magnification}".
	\begin{figure}[H]
		\centering
		\includegraphics[scale=0.6]{img/electromagnetism/magnification_lens.jpg}
	\end{figure}
	Typically, magnification is related to scaling up visuals or images to be able to see more detail, increasing resolution, using microscope, printing techniques, or digital processing. In all cases, the magnification of the image does not change the perspective of the image.
	\begin{figure}[H]
		\centering
		\includegraphics[scale=0.3]{img/electromagnetism/magnifying_glass.jpg}
		\caption[]{The stamp appears larger with the use of a magnifying glass (source: Wikipedia)}
	\end{figure}
	Before studying the magnification of spherical convex lenses, let us focus on the general definition of a magnification. For this purpose, consider the following figure:
	\begin{figure}[H]
		\centering
		\includegraphics[scale=1]{img/electromagnetism/magnification_principle.jpg}
		\caption[]{Magnification basic principle}
	\end{figure}
	where for recall the center of curvature $C$ (\SeeChapter{see section Differential Geometry page \pageref{center of curvature}}) is the center of curvature of the spherical surface of the figure above and the top O is the pole of the spherical cap.

	Thus, the "\NewTerm{magnification $M$}\index{magnification}" of any optical system is defined as the ratio of the size of the image $\overline{ab}$ to that of the real object $\overline{AB}$, that is to say:
	
	We see from the figure above that:
	
	We therefore get, taking into account that $\theta_i={\theta'}_r$:
	
	Hence the "\NewTerm{magnification equation}\index{magnification equation}":
	
	So in the case of a thin symmetrical spherical lens whose rays pass through the optical center we obtain rays which are describe similar triangles on each side of the axis of symmetry of the spherical thin lens then applying Thales' theorem we also get:
	
	That is the same result as that obtained already above.
	\begin{tcolorbox}[colframe=black,colback=white,sharp corners]
	\textbf{{\Large \ding{45}}Example:}\\\\
	Two sides of a convex lens have a radius of $3$ [cm]. The index of the lens material is $1.52$. An object of $1.80$ [m] height is set to $14$ [m] distance of the lens (no matter that it is left or right of the lens because it is assumed as biconvex and therefore symmetrical). Then the dioptric power of the lens is first:
	
	Which is indeed a positive value and thus gives a focal length (image or object focal regardless because of symmetry of the spherical thin lens!) of	$f\cong 28.84\;[\text{cm}]$ so it is better to have a camera with a telephoto lens in the present case... We notice by the value of the focal length (home), that the subject is obviously beyond the focus. The position of the image is given by:
	
	Therefore:
	
	So the picture is about $29.42$ centimetres beyond the focus and will by definition be named the "\NewTerm{real (reversed) image}\index{real (reversed) image}". The magnification value is then:
	
	The size of the inverted real image will be in $q$:
	
	\end{tcolorbox}
	Therefore we have:
	
	If we want a magnification (zoom), we must have:
	
	Therefore:
		
	Is trivially:
	
	Finally:
	
	So to do that there is magnification (zoom) it is necessary that the real object is between the center of curvature and the vertical axis of symmetry of the lens.
	\begin{tcolorbox}[title=Remark,colframe=black,arc=10pt]
	The study of this model will allow us to understand partially how the prism works and also of goniometer in astronomy for spectrum analysis as well as X-ray diffraction by a network of atoms (the importance of the latter being quite significant!).
	\end{tcolorbox}
	We haven't so far studies all possibles configuration of biconvex lenses (mainly because I'm not a fan of the subject...) but here is very good summary founded on Instagram of all main configurations:
	\begin{figure}[H]
		\centering
		\includegraphics[width=1.0\textwidth]{img/electromagnetism/biconvex_lenses_configurations_and_applications.jpg}
		\caption[Biconvex lenses typical configurations and applications]{Biconvex lenses typical configurations and applications (source: ?)}
	\end{figure}	
	The concave mirror bends light inwards, towards ourselves. If we imagine the mirror as a part of a larger sphere, then if we view the mirror from a point closer than the center of this sphere, we see an enlarged (magnified) upright image. If look further away fro the center we will see reversed image as illustrated by the photo below:
	\begin{figure}[H]
		\centering
		\includegraphics[scale=0.947]{img/electromagnetism/concave_mirror.jpg}
		\caption[Concave mirror]{Concave mirror (source: ?)}
	\end{figure}
	Notice above now the image of the girl with the striped shirt is right side up and magnified. She is inside the focal point. The person in the plaid shirt is outside the focal point, so it is upside down and about the same size. But the image from the back of the room is both upside down and smaller than normal, because it is much further away from the focal point!
	
	The same rules applies with a famous case know by all young children:
	\begin{figure}[H]
		\centering
		\includegraphics[scale=0.90]{img/electromagnetism/spoon_mirror.jpg}
		\caption[Left concave spoon-mirror, Right convex spoon-mirror]{Left concave spoon-mirror, Right convex spoon-mirror (source: ?)}
	\end{figure}
	\begin{tcolorbox}[title=Remark,colframe=black,arc=10pt]
	Depending on the manufacturer a makeup mirror is concave or convex as they have the same structure as a spoon but at the difference that they have a very polish surface. But most of time the visible part is a concave mirror build in such a way that a person at $25$ [cm] away from the surface see its image doubled.
	\end{tcolorbox}
	For practical purposes, let us indicate the following figure (left global view, right zoom on the curve part):
	\begin{figure}[H]
		\centering
		\includegraphics[scale=1]{img/electromagnetism/full_planed_curved_lens.jpg}
		\caption[Full plane-curved lens]{Full plane-curved lens (source: ?)}
	\end{figure}
	and therefore a full plane-curved lens, as a parabolic mirror, has the property of making parallel the rays having for source its focal. It produced by refraction the effect that the parabolic mirror produced by reflection...!

	Fresnel invented a lens that is seen in many lighthouses and that achieves the same result with less material:	
	\begin{figure}[H]
		\centering
		\includegraphics[scale=1]{img/electromagnetism/fresnel_lense.jpg}
		\caption[Fresnel lens with lighthouse example in the left bottom corner]{Fresnel lens with lighthouse example in the left bottom corner (source: ?)}
	\end{figure}
	\pagebreak
	So far it could be important to make a summary of the sign convention for the equation of opticians (or "mirror equation") and the magnification equation:
	
	therefore:
	\begin{itemize}
		\item Focal length $f$:
			\begin{itemize}
				\item Positive for concave mirrors or concave lenses
				\item Negative for convex mirrors or convex lenses
			\end{itemize}
		\item Magnification $M$:
			\begin{itemize}
				\item Positive for upright images
				\item Negative for inverted images
				\item Enlarged when $|M|>1$
				\item Reduced when $|M|<1$
			\end{itemize}
		\item Image distance $q$:
			\begin{itemize}
				\item Positive for real images
				\item Negative for virtual images
			\end{itemize}
	\end{itemize}
	
	\begin{tcolorbox}[colframe=black,colback=white,sharp corners]
	\textbf{{\Large \ding{45}}Example:}\\\\
	Consider that the Sun is the object, so the object distance is essentially infinity: $q=+\infty$. The desired image distance
is $q=40.0$ [cm] . We use the mirror equation to find the focal length of the mirror:
	
	But we have proved earlier that in the approximation of small angles:
	
	Thus, the radius of the mirror if considered as spheric, will be equal to:
	
	\end{tcolorbox}
	
	\begin{figure}[H]
		\centering
		\includegraphics[width=1.0\textwidth]{img/electromagnetism/parabolic_heat_collector.jpg}
		\caption[Parabolic trough collectors used to generate electricity in southern California]{Parabolic trough collectors are used to generate electricity in southern California (source: Wikipedia)}
	\end{figure}
	
	
	\pagebreak
	\paragraph{Human eyes}\mbox{}\\\\
	Let us do some human biology to close this topic about lenses...

	The lens of the eye which can be deformed under the effect of certain muscles, constitutes a variable focus lens to accommodate variable distance vision of objects. The distance from the optical center to the retina $r$ being fixed (see figure below), the only way to clearly see objects at different distances $d$ is to change the focal length $f$. In its normal state, the lens has a relatively flat configuration, with a large radius of curvature (he was then a long focal length).
	\begin{figure}[H]
		\centering
		\includegraphics[scale=1]{img/electromagnetism/eye_optics.jpg}
		\caption[Human eye optics]{Human eye optics (source: ?)}
	\end{figure}
	The human eye has the function of focusing light from an object at infinity on the retina. But all eyes are not doing this correctly and the far point (maximum distance of distinct vision without accommodation) or "\NewTerm{punctum remotum}\index{punctum remotum}" is sometimes at a finite distance, sometimes even less than $5$ meters (probably causing eyestrain) as practitioners consider $6$ meters already as infinity relatively to the retina modification over this distance being most of time not significant.
	
	If the object approaches the eye, the muscles of the retina contract, the lens swells and its focal length decreases so that the image is always formed on the retina. The nearest point which can be seen clearly with maximum accommodation is named the "near point" or "\NewTerm{punctum proximum}\index{punctum proximum}". This distance is changing significantly with age: it is ten centimetres for a ten years old child, a hundred centimetres for a person of sixty years old (that is "presbyopia").
	
	\subparagraph{Eye magnification}\mbox{}\\\\
	When we want to examine the details of an object, we approach it to our eye. By doing so, we get a bigger picture on the retina. Indeed, the distance from the focal point O of the eye to the retina can be considered to be $15$ [mm], $\overline{ab}$ image of $\overline{AB}$ is the intersection of the rays $\overline{A\text{O}}$ and $\overline{AB}$ with the retina (the rays being considered as a first approximation in a straight line):
	\begin{figure}[H]
		\centering
		\includegraphics[width=1.0\textwidth]{img/electromagnetism/eye_magnification.jpg}
		\caption[]{Eye as a magnifier tool (source: OpenStax)}
	\end{figure}	
	We then using Thales theorem (ie intercept theorem) once again and assuming that everything is measured in millimetres (and not forgetting that we consider the rays as a first approximation in a straight line from the focus to the original object):
	
	or:
	
	However, the distance $D$ cannot become smaller than that for which the accommodation is maximum. This minimum distinct vision distance can be considered to be $250$ [mm]; it allows the fairly prolonged examination of objects without too much "exhaustion" of the eye.
	
	We have shown that a converging lens gives an enlarged virtual image of a real object provided that the object is placed between the center of curvature and the vertical axis of symmetry of the converging lens. So let's look under what configuration this device is really advantageous for human vision (considering that $I$ is a symbolic $1$D representation of converging lens).
	
	Let us place the eye at the focal point $F'$; instead of the object $\overline{AB}$, he sees the image $\overline{A'B'}$ which is at distance $\overline{A'F'}$:
	\begin{figure}[H]
		\centering
		\includegraphics[scale=0.8]{img/electromagnetism/magnification_eye_lens.jpg}
		\caption[]{Eye as a magnifier tool (source: OpenStax)}
	\end{figure}
	The size of the corresponding retinal image is then according to the previous image:
	
	Then from the similarity of the triangles, we can write (application of Thales theorem again):
	
	since the distance $\overline{\text{O}F'}$ corresponds to the focal length $f$ of the converging lens. We then have:
	
	However, as we mentioned, if the object were seen with the naked eye, the retinal image would have under the aforementioned hypotheses an optimum of:
	
	So that we have an advantage to observe through a covering lens, which in this case we name a "magnifying glass", $\overline{a'b'}$ must be larger than $\overline{ab}$. Either mathematically:
	
	which give the fundamental condition of the converging lens:
	
	Thus, a converging lens can only be used as a magnifying glass if its focal length is less than $250$ [mm] under the aforementioned assumptions. By definition, we named "magnification" of the magnifying glass the ratio:
	
	of the retinal images obtained through the magnifying glass and with the naked eye in the best conditions.
	
	\pagebreak
	\subsubsection{Triangular Prism}
	 a prism is a transparent optical element with flat, polished surfaces that refract light. At least two of the flat surfaces must have an angle between them. The exact angles between the surfaces depend on the application. The traditional geometrical shape is that of a triangular prism with a triangular base and rectangular sides, and in colloquial use "prism" usually refers to this type. Some types of optical prism are not in fact in the shape of geometric prisms. Prisms can be made from any material that is transparent to the wavelengths for which they are designed. Typical materials include glass, plastic and fluorite.

	A dispersive prism can be used to break light up into its constituent spectral colors (the colors of the rainbow). Furthermore, prisms can be used to reflect light, or to split light into components with different polarizations.
	\begin{figure}[H]
		\centering
		\includegraphics[scale=1.5]{img/electromagnetism/prism_david_parker.jpg}
		\caption[Triangular prism dispersing white light ray]{Triangular prism dispersing white light ray (source: Wikipedia, David Parker)}
	\end{figure}
	In optics, the triangular prism is one of the most important components of optics. It can be found in chemistry, condensed matter physics, astrophysics, optoelectronics and still many other popular devices of everyday life (such as lentils). This is probably the first tool shaped by human to make "spectroscopy" (spectrum analysis) after the rainbow sky obviously... which is a natural phenomenon spectroscopy.
	
	We will in the following paragraphs identify the most important relations in relation to know about triangular prisms and useful to the engineer and physicist.

	We will focus on light rays entering through one side and exiting by another having undergone two refractions (we do not study the internal reflections).

	Here is the typical schematic representation of a triangular prism in optical geometry with the incident light ray $S$ and outgoing one $S'$ and the two normals $N$, $N'$, to the edges of the top of opening angle $\alpha$. More the different angles of incidence and refraction:
	\begin{figure}[H]
		\centering
		\includegraphics[scale=1]{img/electromagnetism/triangular_prism.jpg}
		\caption[]{Schematic triangular prism}
	\end{figure}
	We know that the sum of the angles of a quadrilateral (always decomposable into two triangles whose the sum of the angles is $\pi$) is equals $2\pi$ (\SeeChapter{see section Geometric Shapes page \pageref{angle sum theorem}}). So in the quadrilateral defined by the nodes $1234$. We have the sum of:
	
	Now that the situation has know let us focus on the optical part ...

	We have $4$ fundamental relations to prove for to the triangular prism.

	First, we have at the point of incidence $I$ and $I'$ the Descartes' law that allows us to write:
	
	As the absolute refractive index of air is $1$ we simply have in $I$:
	
	In the same idea in $I'$ we have:
	
	and therefore:
	
	We also have the relation:
	
	Therefore:
	
	The deflection angle $D$ is easy to determine. We just have to take the central quadrilateral:
	
	Therefore:
	
	So we have the $4$ fundamental relations of the triangular prism:
	
	Knowing $i$ and $i'$ and the relative refractive index $m$ then we can determine all parameters.

	The ideal would still be able to get rid of the experimental knowledge of $i'$.

	So we have:
	
	But as:
	
	Therefore it comes:
	
	Therefore:
	
	Since it is obvious that the index $n$ of a material varies with the wavelength according to Cauchy law introduce at the beginning of this section, it is easy to understand that the triangular prism is capable of dispersing white light.

	Finally if $i$ is small:
	
	and if $i$ and $\alpha$ are small, we have to first order in Maclaurin development (\SeeChapter{see section Sequences and Series \pageref{usual maclaurin developments}}):
	
	Therefore:
	
	either by explicitly introducing Cauchy's law as introduced earlier in this section:
	
	The fact that the index of refraction in glass depends on wavelength (Cauchy's law) is then the reason that prisms can spread the colors of the rainbow!
	
	\pagebreak
	\subsubsection{Pentaprism}
	The pentaprism roof that is enclosed  in most viewfinder of single-lens reflex cameras is an adaptation of the pentaprism, or "\NewTerm{optical square}\index{optical square}", invented in 1864 by Charles Moÿse Goulier.
	\begin{figure}[H]
		\centering
		\includegraphics[scale=0.66]{img/electromagnetism/pentaprism.jpg}
		\caption[Pentaprism roof]{Pentaprism roof (source: \url{http://www.pierretoscani.com}, author: Pierre Toscani)}
	\end{figure}
	A pentaprism is a five-sided reflecting prism used to deviate a beam of light by a constant $90^\circ$, even if the entry beam is not at $90^\circ$ to the prism (see proof below). The beam reflects inside the prism twice, allowing the transmission of an image through a right angle without inverting it vertically (that is, without changing the image's handedness) as an ordinary right-angle prism or mirror would.
	
	
	The reflections inside the prism are not caused by total internal reflection, since the beams are incident at an angle less than the critical angle (the minimum angle for total internal reflection). Instead, the two faces are coated to provide mirror surfaces. The two opposite transmitting faces are often coated with an anti-reflection coating to reduce spurious reflections. The fifth face of the prism is not used optically but truncates what would otherwise be an awkward angle joining the two mirrored faces.
	
	As we have show it, the camera lens renders an image that is both vertically and laterally reversed, and the reflex mirror re-inverts it leaving an image laterally reversed. In this case, the image needs to be reflected left-to-right as the prism transmits the image formed on the camera's focusing screen. This lateral inversion is done by replacing one of the reflective faces of a normal pentaprism with a "roof" section, with two additional surfaces angled towards each other and meeting at $90^\circ$, which laterally reverses the image back to normal. 
	
	In a precision optical device, when the path of a light beam must be deflected at $90^\circ$, the simplest solution is to use a plane mirror oriented at $45^\circ$ relatively the direction of the incident beam (see figure below). The mirror must be accurately positioned and stable because any angular positioning error induces a double error in the direction of the reflected beam: the reflected beam rotates twice as fast as the mirror (this is obvious as when the mirror in oriented at $\alpha=45^\circ$ the output beam is at an output angle of $2\alpha=90^\circ$ and therefore in general any modification $\delta$ implies a deflection of $2\delta$):
	\begin{figure}[H]
		\centering
		\includegraphics[scale=0.6]{img/electromagnetism/mirror_deflection.jpg}
		\caption[Mirror double deflection property]{Mirror double deflection property (source: \url{http://www.pierretoscani.com}, author: Pierre Toscani)}
	\end{figure}
	But for the pentaprism it is different... It behave like a set square:
	\begin{figure}[H]
		\centering
		\includegraphics[scale=0.6]{img/electromagnetism/pentaprism_set_square_property.jpg}
		\caption[Mirror double deflection property]{Mirror double deflection property (source: \url{http://www.pierretoscani.com}, author: Pierre Toscani)}
	\end{figure}
	Before we deal with the maths, the reader should be aware that this pentaprism has the following properties:
	 \begin{itemize}
		\item Two transparent faces (input or output) forming an angle of $90^\circ$
		 \item Two reflecting faces forming an angle of $45^\circ$ , separated by a chamfer (fifth face, no function)
		\item The both reflective surfaces have to be metallized to provide good reflection coefficient.
	\end{itemize} 
	\begin{figure}[H]
		\centering
		\includegraphics[scale=0.75]{img/electromagnetism/pentaprism_detailed_design.jpg}
		\caption[Pentaprism detailed design]{Pentaprism detailed design (source: \url{http://www.pierretoscani.com}, author: Pierre Toscani)}
	\end{figure}
	Now to prove that property consider the following figure:
	\begin{figure}[H]
		\centering
		\includegraphics[scale=0.75]{img/electromagnetism/pentaprism_set_square_property_proof.jpg}
	\end{figure}
	We have then looking this figure:
	
	That we will rearrange and write:
	
	Now in the above triangle we have:
	
	Therefore:
	
	On the left triangle we have:
	
	Therefore:
	
	That is:
	
	Finally in the last triangle we have:
	
	And therefore:
	
	That is:
	
	Finally:
	
	And now at the exit of the pentaprism we have:
	
	But as $\alpha_5=\alpha_2$ we have:
	
	Therefore it follows that:
	
	And this finish our proof!
	
	
	
	\begin{flushright}
	\begin{tabular}{l c}
	\circled{90} & \pbox{20cm}{\score{4}{5} \\ {\tiny 48 votes,  76.25\%}} 
	\end{tabular} 
	\end{flushright}

	%to make section start on odd page
	\newpage
	\thispagestyle{empty}
	\mbox{}	
	\section{Wave Optics}\label{wave optics}
	\lettrine[lines=4]{\color{BrickRed}I}n this section will study some elements that led to the development of quantum mechanics. Indeed, quantum mechanics was born, first, by a careful study of the nature of light. Although this new science was developed in the early 20th century, the considerations which have guided to it are undoubtedly the result of 25 centuries of maturation. Basically, it is a long history of the light full of controversies to which quantum mechanics in the 20th century finally brings a masterful conclusion.
	
	Electron microscopes can make images of individual atoms, but why will a visible-light microscope never be able to? Stereo speakers create the illusion of music that comes from a band arranged in your living room, but why doesn't the stereo illusion work with bass notes? Why are computer chip manufacturers investing billions of dollars in equipment to etch chips with X-rays instead of visible light?

	The answers to all of these questions have to do with the subject of wave optics. So far this book has discussed the interaction of light waves with matter, and its practical applications to optical devices like mirrors, but we have used the ray model of light almost exclusively. Hardly ever have we explicitly made use of the fact that light is an electromagnetic wave. We were able to get away with the simple ray model because the chunks of matter we were discussing, such as lenses and mirrors, were thousands of times larger than a wavelength of light. We now turn to phenomena and devices that can only be understood using the wave model of light.

	\subsection{Huygens' principle}\label{Huygens principle}
	Huygens visualized the propagation of light as a result of the process of a generation of spherical wavelets in each point reached by a wavefront, wavelet whose sum gave the propagation field. By drawing the tangent to the wavefronts of the wavelet at a given time, one obtained the wavefront of the total wave at the same moment.
	
	\begin{tcolorbox}[colback=red!5,borderline={1mm}{2mm}{red!5},arc=0mm,boxrule=0pt]
	\bcbombe Caution! As we have already mentioned it earlier, the Fermat's principle doesn't explain anything, even if it's accurate that light follow the shortest path! That's why most textbooks introducing refraction as we did above geometrically (named sometimes the "marching soldier analogy", but wrong as nothing explain why the solders don't continue you walk straight) or using the variational principal (ie "Fermat's principle") are quite misleading. Using Huygens principle as we will do it now isn't also accurate as we will explain it (even if we fall back on the same result as using Fermat's principle or the marching soldier analogy)! In fact we should use Maxwell equation to understand better what happens physically!
	\end{tcolorbox}
	
	\begin{figure}[H]
		\centering
		\includegraphics{img/electromagnetism/huyghens_schema.jpg}
		\caption{Illustration of Huy­gens' construction of a prop­a­gat­ing wave}
	\end{figure}
	\textbf{Definition (\#\mydef):} A "\NewTerm{wavefront surface}\index{wavefront surface}" is the localization of points of the medium reach by the wave motion at the same instant. The disturbance has therefore the same phase at any point of a wavefront. For a plane wave, for example, the disturbance is expressed by (we have prove this in the section of Wave Mechanics):
	
	or in a more general way:
	
	which therefore gives the expression of the propagation of the disturbance for which the "\NewTerm{wavefront}\index{wavefront}" is the locus of points where the phase $\vec{k}\circ\vec{r}-vt$ has the same value at a given instant. The wave surface is given accordingly by the equation:
	
	Huygens gave a pictorial representation method of the passage of a wave surface to another in the case where the wave is assumed to result from the movement of the particles constituting the material medium. Thus, if we consider the wave surface $S$ below:
	\begin{figure}[H]
		\centering
		\includegraphics{img/electromagnetism/huygens_wave_propagation.jpg}
		\caption{Representation of a wavefront according to Huygens}
	\end{figure}
	When the wave motion reaches this surface, each particle $a, b, c, ...$ of the surface becomes in turn a wave source, emitting secondary waves (indicated by the small half-circles in the figure above) that reach the next layer of particles of the medium. These particles are put in motion and form the new wave surface $S'$ and so on... So, Huygens had a wave conception of light, but he did not consider the periodic nature of the wave, which not allow him to introduce the concept of light of color (frequency); moreover, according to its principle, a wave propagating in the opposite direction to that of the incident wave should also occur, which is not the case in a homogeneous material...
	
	The intuition of Huygens is however close to reality as will show if Fresnel in its diffraction theory (see further below). However, it was Kirchhoff, which will introduce a tilt factor (obliquity) in the theory, to gen an explanation of the absence of wave propagating backward (when the time will come we will write related mathematical developments).
	
	As on the previous figure all the "corresponding points" $(a,b,c,\ldots),(a',b',c',\ldots)$ are  equidistant, by Huygens' principle, the time interval between corresponding points of two wavefronts is the same for any two corresponding points!
	
	The consequences are (refer simultaneously to the figure below):
	\begin{figure}[H]
		\centering
		\includegraphics{img/electromagnetism/huygens_time_interval_principle.jpg}
		\caption[]{Schematic diagram}
	\end{figure}
	If we denote the propagation velocities of the incident rays $R_1$, $R_2$ by $v_1$ and $v_2$ respectively, we have:
	
	\begin{itemize}
		\item When the wave propagates in a homogeneous medium, the light rays must be straight and the wave surfaces remain parallel.

		\item When the wave change of medium, the distance between two pairs of corresponding points vary from one medium to another, if the propagation velocities are different.
	\end{itemize}
	\begin{theorem}
	Let us prove that this principle allows to find the Descartes-Snellius law that we have already proven in the section of Geometric Optics, which ensures a priori that the Huygens principle is still valid in the context of Geometric Optics.
	\end{theorem}
	\begin{dem}
	Following the figure above, we have:
	
	by dividing each term by $\overline{a_1b_2}$, we get:
	
	so we fall back well on the Descartes-Snellius law as we had obtained it in the section of Geometric Optics:
	
	noting on the way that on the diagram, we also have $\theta_1={\theta'}_1$.
	\begin{flushright}
		$\blacksquare$  Q.E.D.
	\end{flushright}
	\end{dem}
	
	So to derive Snell's law we have seen so far three commonly suggested explanations that are wrong, or at least incomplete. They are for Fermat's principle, the analogy of soldiers marching and Huygens' principle. Fermat's principle says that light will travel from one point to the other, taking the minimum amount of time. It's often explained in terms of a lifeguard in a drowning swimmer in order to maximize the chances that the swimmer will be saved, the lifeguard needs to get to them as soon as possible. But while that's true for light, it doesn't explain anything. It just says what it does, not why it does it. So that's not an explanation and we need to come up with a proper reason. Let's look at the second option on the list. A very common explanation used to teach why light bends involves soldiers marching over firm ground, but then who encounter a wall, although not all soldiers hit mud at the same time. The idea is simple. The first soldier who encounters the mud cannot move as quickly as the rest of the soldiers, so he slows down. Each soldier hits the mud and turn, resulting in a direction change. Eventually the lines of soldiers are moving in a different direction at a different speed. This is how it's often taught. The problem is that this doesn't work. In fact, the soldiers will slow down, but their direction doesn't change. They will continue to walk in the same direction, but with marching lines that are slanted compared to how they started. This explanation gets the angle of the lines of soldiers right but not the direction of travel. In fact, the only way that this explanation can be right is if the lines of soldiers are rigid, then the first soldier hitting the mud puts a torque on the entire line and that would work. The problem is that this torque means that the soldiers on the top of the screen move faster, which is to say that if they were light, this part of the beam of light would have to move faster than light, so this explanation doesn't work either.

	What about Huygens' principle? Remember in the figure on the left below that the lines denote the top of the peaks of the waves. We can see that there are many places where the peaks line up. We can overlay a couple of lines that show the direction that the wave is travelling and then remove the circular waves and we see that this explanation seems to be doing a good job of predicting what happens when light goes from air to glass. Looks good, right? But not so fast! What happens when we look at all of the waves coming through the glass? Things get a lot more complicated. We do see that the waves add together to predict the direction that light moves in glass. On the other hand, that's not the only alignment! For instance, we see other geometry's where the waves line up a visible in the figure on the right below. 
	\begin{figure}[H]
		\centering
		\begin{subfigure}{0.5\textwidth}
			\includegraphics[width=\textwidth]{img/electromagnetism/huygens_success.jpg}
			\caption[]{What most textbooks show}
		\end{subfigure}
		\begin{subfigure}{0.485\textwidth}
			\includegraphics[width=\textwidth]{img/electromagnetism/huygens_failure.jpg}
			\caption[]{What only a few textbooks admit...}
		\end{subfigure}		
		\caption{Failure of Huygens' principle}		
	\end{figure}
	So it seems that this approach doesn't give a unique prediction. OK, we've ruled out Fermat's principle, the soldier analogy, and Huygens' principle! 
	
	These are all often seen as explanations of the cause of refraction given by people who really do know some physics. So what's the real answer? It turns out that the only way to really answer the question of why light bends when it goes from air to glass is to get serious about the nature of light and to embrace the fact that it is made of oscillating electromagnetic fields, and that means we need Maxwell's equations! 
	
	We will focus here only on the electric fields. So we start with Maxwell's equations:
	
	In fact, we're going to need only the bottom two equations:
		
	So let's see what's going on. We start with light going from air to glass, hitting the surface at an angle. In our figure, we can replace the waves with the direction of motion. Now it turns out that the electric field of light is perpendicular to the direction that light is travelling, and we can add that field direction to the diagram. And it's very important to remember that this field has components. One parallel to the surface of the glass and one that is perpendicular to the glass:
	\begin{figure}[H]
		\centering
		\includegraphics[scale=0.75]{img/electromagnetism/snell_law_maxwell_derivation_01.jpg}
	\end{figure}
	And here is where Maxwell's equations come into play! 

	Two copies of the equations are written here, one that covers when light is travelling in air:
	
	 and one where it is travelling in glass:
	
	So here's the key point! The surface belongs to both the air region and the glass region. This means that at the surface, the equations on the top and the equations on the bottom have to apply, and with that little bit of calculus we can find two important restrictions. Let us see which are these restrictions?
	
	For that purpose let us see what are the most general boundary conditions satisfied by the electric field at the interface between two media: e.g., the interface between a vacuum and a conductor? Consider an interface $P$ between two media $A$ (for example "air") and $B$ (for example "glass"):
	\begin{figure}[H]
		\centering
		\includegraphics{img/electromagnetism/refraction_gauss_law.jpg}
	\end{figure}
	Let us, first of all, apply Gauss' law (\SeeChapter{see section Electrodynamics page \pageref{maxwell equations}}):
	
	to a Gaussian pill-box $S$ of cross-sectional area $A$ whose two ends are locally parallel to the interface (see figure above). The ends of the box can be made arbitrarily close together. In this limit, the flux of the electric field out of the sides of the box is obviously negligible. The only contribution to the flux comes from the two ends. In fact:
		
	where $E_{\perp A}$ is the perpendicular (to the interface) electric field in medium $A$ at the interface, etc. The charge enclosed by the pill-box is simply $\sigma_A$, where $\sigma$ is the sheet charge density on the interface. Note that any volume distribution of charge gives rise to a negligible contribution to the right-hand side of the above equation, in the limit where the two ends of the pill-box are very closely spaced. Thus, Gauss' law yields:
		
	at the interface: i.e., the presence of a charge sheet on an interface causes a discontinuity in the perpendicular component of the electric field. 
	
	The way electric permittivity works is that it lets us use Gauss' law as if there is no charge, provided we integrate $\vec{D}=\varepsilon\vec{E}$ over the surface rather than $\vec{E}$. Therefore the boundary condition is $D_{\perp A}=D_{\perp  B}$, which implies:
	
	What about the parallel electric field? Let us apply Faraday's law (\SeeChapter{see section Electrokinetics page \pageref{Faraday's law of induction}}) to a rectangular loop $C$ whose long sides, length $l$, run parallel to the interface:
		
	(see figure above). The length of the short sides is assumed to be arbitrarily small. The dominant contribution to the loop integral comes from the long sides:
	
	where $E_{\parallel A}$ is the parallel (to the interface) electric field in medium $A$ at the interface, etc. The flux of the magnetic field through the loop is approximately $B_{\perp A}$, where $B_\perp$ is the component of the magnetic field which is normal to the loop, and $A$ is the area of the loop. But, $A\rightarrow 0$ as the short sides of the loop are shrunk to zero. So, unless the magnetic field becomes infinite (we shall assume that it does not), the flux also tends to zero. Thus:
	
	i.e., there can be no discontinuity in the parallel component of the electric field across an interface.
	
	So, coming back to our glass and air case, this leads us to\label{boundary condition for refraction}:
	
	
	\begin{tcolorbox}[title=Remark,colframe=black,arc=10pt]
	For the magnetic field, the boundary conditions will be affected due to the magnetic moment of the charges in the medium, as encoded in the permeability $\mu$. Since a magnetic field induces in a material a current perpendicular to the field, only $B_\parallel$ will be affected (because of the change of material perpendicular to the surface in the plane where the induced current should happen), not $B_\perp$.\\

	Thus:
	
	Since the part of the magnetic field which is sensitive to an accumulated current in a material is
	treated using $\vec{H}=\vec{B}/\mu$ the condition is then that:
	
	\end{tcolorbox}
	
	In transparent materials, the electric permittivity can vary significantly in comparison to air, but so does not the magnetic permeability. So some textbooks assume that $\mu_A\cong \mu_B$. In this case, the four boundary conditions:
	
	reduce to:
	
	The first is that the electric field parallel to the surface of the glass has to be the same in the air in the glass and similarly  perpendicular to the surface:
	
	Therefore:
	
	Because $\varepsilon$ is bigger in glass than in air, that means that the perpendicular electric field in glass has to be smaller than it is in air. 

	Now we remember that the direction the late travels is perpendicular to the electric field, so we can put in an arrow to show the direction light must travel in the glass. And finally, we can see what light does when it enters glass or water or any transparent medium, it bends. And the reason that it bends is because the epsilon in glass is bigger than an air. 

	OK, what we just seen is an equation thing. The reader may probably asking himself what that epsilon actually physically means. It's there because how the electric fields from the light interacts with the matter in the glass. We start out with glass with no electric field in it. The glasses charges in it, but they're arranged in a random way. But when we send light in, we impose an electric field on it. That field makes the charges move around, which sets up a counterbalancing electric field from the charges. The result is that the electric field in the glass is lower than it is in air because of how the electric field from the glass is in the opposite direction. And this is the reason that the perpendicular electric field is lower in glass, and that is why light bends when it goes from here to glass. It's not because of many of the ordinary explanations, it's because of how light interacts with glass and changes the glasses properties, and it's because the electric field inside the glass is affected by the arrangements of atoms and molecules in the glass when light hits it. It's the same reason why. Light slows down in matter inside matter. The electric field due to the light and the electric field due to the matter both have to be taken into account. Outside they don't. And now you know some serious physics! We imagine that some of you will may ask about the quantum explanation, which is similar to the one we just told you here using Maxwell equations except using energy and momentum of the photons instead.
	
	\pagebreak
	\subsubsection{Normal refraction}
	Let us start with the case of normal incidence (perpendicular to the interface). To find out how much light is reflected, we need to work out the impedance, which determines the reflection and transmission coefficients.
	
	For normal incidence the incident angle to the normal is $\theta_{1}=0$ (see the figure \ref{fig:fresnel coefficients} below). Thus $E_{\perp}=$ $B_{\perp}=0$. That is, the electric and magnetic fields are both polarized in the plane of the interface so there simply is no $\perp$ component. 
	
	Without loss of generality, let's take a plane wave of frequency $\omega$ moving in the $z$ direction with $\vec{E}$ in the $y$ direction. Then since (see page \pageref{wave equation by wave vector}):
	
	the magnetic field points in the $x$ direction. So the incident fields are:
	
	The transmitted and reflected waves are also moving normal to the surface (by Snell's law), so we can write:
	
	with $E_{T}, E_{R}, B_{T}$ and $B_{R}$ the transmission and reflection coefficients to be determined. Note that we have flipped the sign on the $z$ term in the phase since the reflected wave is moving in the $-\vec{u}_z$ direction (that is, $\vec{k}$ flips).
	
	 Since $\omega \vec{B}=\vec{k} \times \vec{E}$, this means that $\vec{B}$ flips sign too, which explains the minus sign in $\vec{B}_R$. 
	 
	 Since there is no $\perp$ component and that we have the boundary condition $E_{\|}^{(1)}=E_{\|}^{(2)}$, we will assume that this implies:
	
	Similarly we will assume that:
	
	implies:
	
	Since $||\vec{B}||=\frac{1}{v}||\vec{E}||$ for any plane wave, we have obviously:
	
	and the above relation becomes:
	
	Combining:
	
	and solving gives:
	
	where:
	
	For most materials, $\mu_i$ is pretty constant, so the form $Z=\mu_i c /n_i$, which says $Z \propto n_i^{-1}$, is most useful. 
	
	What is the power reflected and transmitted assuming the above assumptions correct? The incident power in a medium is given by the Poynting vector (see page \pageref{poynting vector}):
	
	times area $A$. So:
	
	Thus the reflected power is:
	
	and the transmitted power is:
	
	These satisfy $P_{T}+P_{R}=P_{I}$ as expected. Note that these equations hold for normal incidence only and the others assumptions!
	
	\subsubsection{Thin film interference}
	Using the relations we get from the above developments:
	
	and that since $Z \propto n^{-1}$, the first relation becomes:
	
	Thus for $n_{2}>n_{1}$ the reflected electric field has a phase flip compared to the incoming field, and for $n_{1}<n_{2}$ there is no phase flip.

	So what happens when light hits a thin film? 
	
	The film will generally have $n_{2}=n_{\text {film }}>n_{\text {air }}=$ $n_{1}$. Thus we get a phase flip at the top surface, but no phase flip at the bottom surface. The picture looks like this:
	\begin{figure}[H]
		\centering
		\includegraphics[width=1.0\textwidth]{img/electromagnetism/thin_film_interference.jpg}
		\caption[]{There is a $\pi$ phase flip from reflections off the top surface, where $n_{2}>n_{1}$, but not off the bottom surface, where $n_{2}<n_{1}$.}
	\end{figure}
	So what happens if light of wavelength $\lambda$ hits a film of thickness $d ?$ 
	
	There will be two reflected waves, one off the top surface $(A)$ of the film and one off the bottom $(B)$ of the film which will interfere!
	
	Say the incoming electric field on $(A)$ is:
	
	pointing to the right. Then the wave which reflects off the top surface $(z=0)$ will be:
	
	with $R$ being the reflection coefficient. This now points to the left! 
	
	The wave which gets through the surface $(A)$ from the film will be:
	
	with $T$ the transmission coefficient. This wave then goes to the bottom surface, reflects with no phase flip, and comes back and exits with no more phase flips. So when it exits, it is back to $z=0$ after having traversed a distance $\Delta z=2 d$. Thus it is:
	
	The total wave at $t=0$ and $z=0$ is therefore:
	
	where $\lambda$ is the wavelength in the second medium (inside the film).
	
	In the limit that $d \ll \lambda$ the two reflections (taken at $t=0$ and $z=0$):
	
	are exactly out of phase. For $T \approx 1$, there will be complete destructive interference and no reflection. But there will also be complete destructive interference whenever $\cos \left(\frac{4 \pi d}{\lambda}\right)=1$, which is when:
	
	On the other hand if the two waves are completely in phase there will be constructive interference. This happens when $\cos \left(\frac{4 \pi d}{\lambda}\right)=-1$ which is when:
	
	If the material is much thicker than the wavelength of light, and not of completely uniform thickness, then there will be some constructive and some destructive interference and we won't see much interesting. However, if the material has a well-defined thickness which is of the same order of magnitude as the wavelength of visible light, we will see different wavelengths with different intensities. This happens in a soap film.

	If we put a soap film vertically, then gravity will make it denser at the bottom. The result is the following:
	\begin{figure}[H]
		\centering
		\includegraphics[scale=0.5]{img/electromagnetism/thin_film_interference_gravity.jpg}
	\end{figure}
	Note at the very top, the film is black because there is complete destructive interference for all wavelengths. Similar patterns can be seen in soap bubbles:
	\begin{figure}[H]
		\centering
		\includegraphics[scale=0.5]{img/electromagnetism/thin_film_interference_bubble.jpg}
	\end{figure}
	Colour due to thin-film interference is known as "\NewTerm{iridescence}\index{iridescence}\label{iridescence}". The color of many butterflies and the gorgets of hummingbirds is due to iridescence:
	\begin{figure}[H]
		\centering
		\includegraphics[width=1.0\textwidth]{img/electromagnetism/iridescence.jpg}
	\end{figure}
	
	\subsubsection{Fresnel coefficients}
	Now let's consider the more general case. Suppose we have a plane wave moving in the $\vec{k}$ direction towards a surface with normal vector $\vec{n}$. Thus the angle $\theta_{1}$ that the wave is coming in at satisfies:
	
	The two vectors $\vec{k}$ and $\vec{n}$ form a plane. 
	
	There are two linearly independent possibilities for the polarization: the electric field $\vec{E}$ can be polarized in the $\vec{k}-\vec{n}$ plane of the boundary, or transverse to that plane. 
	
	We know that these are named "vertical polarizations" and "horizontal polarization" respectively. Snell's law then tells us the directions of the transmitted and reflected electric fields. The magnetic field is always determined by the electric field through (see page \pageref{wave equation by wave vector}):
	
	What we need to solve for is the amplitude of the transmitted and reflected fields! The two cases look like (crosses indicate vectors point into the page, and dots that the vectors come out):
	\begin{figure}[H]
		\centering
		\includegraphics[scale=0.7]{img/electromagnetism/fresnel_coefficients.jpg}
		\caption[]{Two linearly independent linear polarizations are vertical and horizontal.}
		\label{fig:fresnel coefficients}
	\end{figure}
	Let's start with vertical polarization. From the left figure above we see that $\vec{B}$ is always parallel to the surface, so $B_{\perp}=0$. The electric field has for components project on the surface of the interface between the two medium (ie the parallel component):
	
	as well as:
	
	The reader can check that when $\theta_{1}=\theta_{2}=0$, this reduces to the normal incidence case see earlier above, so we have the sines and cosines right. 
	
	
	Note the relative sign between $E_{\perp}^{I}$ and $E_{\perp}^{R}$. You can check this sign because when $\theta=\pi/2$, then $E_{I}$ points up and $E_{R}$ points down, i.e. they have opposite signs. There is similar sign flip between $B_{I}$ and $B_{R}$, as can be seen in the figure, since $\vec{B}$ always lags behind $\vec{E}$ by $\pi/2$. 
	
	Now we simply plug these relations into our boundary conditions and solve as seen earlier above (see page \pageref{boundary condition for refraction}):
	
	imply:
	
	Hence:
	
	The relation:
	
	is trivially satisfied since $B_{\perp}=0$. Finally, the relation:
	
	implies:
	
	Using $B=E/v$ this last relation becomes:
	
	Injecting it in:
	
	we get:
	
	Using:
	
	We have for example
	
	Hence:
	
	we then get by dividing both sides by $\sqrt{\varepsilon_{0}\mu_{0}}$:
	
	which is Snell's law. It is reassuring that our derivation here reproduces Snell's law...
	
	To simplify the solution it is helpful to define one parameter which depends on the angles but not on the materials:
	
	and another which depends only on the materials:
	
	Then the equations:
	
	reduce to:
	
	with solutions (keep in mind that's for the vertical polarization!):
	
	The solution for horizontal polarization would lead to a similar result:
	
	These are known as "\NewTerm{Fresnel coefficients}\index{Fresnel coefficients}\label{Fresnel coefficients}". The Fresnel coefficients tell us how much of each polarization is reflected and transmitted. Since any polarization vector can be written as a linear combination of vertical and horizontal polarizations, we can use the Fresnel coefficients to understand how any polarizations reflect.
	
	To interpret the Fresnel coefficients, we would like to know not just the amplitude of the wave transmitted, but the intensity of the light that passes through, or equivalently the power. Recall that the power in a plane wave in the vacuum is (\SeeChapter{see section Electrodynamics page \pageref{poynting vector}}):
	
	 For a plane wave in a medium, $\varepsilon$ changes and only the component of the velocity moving into the medium is relevant, so this becomes (if we don't multiply the $\cos(\theta)$ we wouldn't have conservation of energy as we will check it further below!):
	
	 Thus the fraction of power reflected for vertical and horizontal polarizations is by using the Fresnel coefficients relations:
	
	For the fraction of power transmitted, and:
	
	and:
	
	One can check that for either polarization (just put the above relation inside this identity) energy conservation is respected:
	
	If the reader remember, for the normal case we get earlier for the reflected power:
	
	and the for transmitted power :
	
	And obviously we get the equality with the previous expression only if the conditions of the normal incidence are met. That is:
	
	And that makes completely sense! This is another evidence that our previous results are not so wrong...

	For example, the air-glass interface has $\beta=1.5$. Then the transmitted and reflected power in vertical and horizontal polarizations are:
	\begin{figure}[H]
		\centering
		\includegraphics[width=1.0\textwidth]{img/electromagnetism/transmitted_power_refraction.jpg}
		\caption{Transmitted and reflected power as a function of incident angle for the two polarizations}
	\end{figure}
	We can also plot the ratio of power reflected in vertical to horizontal polarizations:
	\begin{figure}[H]
		\centering
		\includegraphics[scale=0.7]{img/electromagnetism/reflected_power.jpg}
		\caption{Ratio $P_{R}^{\text {vert }} / P_{R}^{\text {horiz }}$ of power reflected vertically polarized to horizontally polarized light}
	\end{figure}
	From these plots we see there is an angle where the vertical polarization exactly vanishes. This angle is named the "\NewTerm{Brewster's angle}\index{Brewster's angle}", $\theta_{B}$. We can see from the plots that for the glass-air interface $\theta_{B} \approx 56^{\circ}$. At this angle, the reflected light is completely (uniquely) polarized.
	
	We can also see that Brewster's angle is an angle of incidence at which light with a particular polarization is perfectly transmitted through a transparent dielectric surface, with no reflection. 
	
	What is the general formula for $\theta_{B} ?$ From the fraction of power reflected for vertical and horizontal polarizations is:
	
	we see that $P_{R}^{\mathrm{vert}}=0$ when $\alpha=\beta$. That is:
	
	For most materials $\mu_{1} \sim \mu_{2} \approx \mu_{0}$. Then we can use $n=\sqrt{\mu \varepsilon}$ to get:
	
	Using also:
	
	we can solve for $\theta_{B}$ giving:
	
	Hence the Brewster's angle:
	
	For the air-water interface, $\theta_{B}=\tan^{-1} (1.5)=56.3^{\circ}$. One can see from the previous figure that one does not have to be exactly at this angle to have little reflected vertical polarization. Angles close to $\theta_{B}$ work almost as well.

	What is going on physically at Brewster's angle? We know that Snell's law:
	
	is always satisfied. At Brewster's angle, when $\theta_{1}=\theta_{B}$, we found also $n_{1} \cos (\theta_{2})=n_{2} \cos(\theta_{1})$. 
	
	Dividing these two relations gives:
	
	or equivalently:
	
	Now, by definition $\theta_{1}$ and $\theta_{2}$ are both between $0$ and $\frac{\pi}{2}$, so $2 \theta_{1}$ and $2 \theta_{2}$ are between $0$ and $\pi$. Thus:
	
	has two solutions. Either $\theta_{1}=\theta_{2}$, which corresponds to $n_{1}=n_{2}$ so the light just passes through, or:
	
	which simplifies to:
	
	Using simple geometry we see that the transmitted (ie refracted) and reflected waves are perpendicular at Brewster's angle:
	\begin{figure}[H]
		\centering
		\includegraphics[scale=0.8]{img/electromagnetism/brewster_angle_properpty.jpg}
	\end{figure}
	Since the reflected wave has to be produced by the motion of particles in the surface we can interpret why there may be no reflection of the vertical polarization component at Brewster’s angle: at this angle particles moving in the surface can only produce light polarized transverse to their direction of motion and a Brewster's angle the vertical motion is hard or even impossible.
		
	\pagebreak
	\subsection{Fraunhofer Diffraction}\label{fraunhofer diffraction}
	From the point of view of geometrical optics, a light beam is a cylinder of section $\Sigma$ which brings together a large number of parallel rays. It is therefore assumed to be straight when it is set in a homogeneous medium.
	
	The "\NewTerm{energy emittance}\index{energy emittance}" $M \; [\text{Wm}^{-2}]$ of the beam varies only if a lens (or another device) make vary its section $\Sigma$ or if the medium absorbs energy.
	
	The light beam "bursts" when an obstacle allows only a portion of its section $\Sigma$.
	
	Huygens' principle shows that it is the edges of the obstacle that generate this diffraction.
	
	The phenomenon is general but is well observed only if the ratio $L/\Sigma$ is very large. $L$ being the length of the edges. This condition is necessary so that the intensity of the undiffracted portion of the beam does not mask the effect.
	
	\begin{enumerate}
		\item[D1.] We speak of "\NewTerm{Fraunhofer diffraction}\index{Fraunhofer diffraction}" when, as assumed above, the incident light rays are parallel and the phenomenon observed at relatively large distance from the screen.

		\item[D2.] We speak of "\NewTerm{Fresnel diffraction}\index{Fresnel diffraction}" when the incident rays form a divergent beam from a point source or if we observe the phenomenon at close range.
	\end{enumerate}
	
	Let us consider a generic case and in fact the most widespread case in the physics laboratories which is the diffraction by a narrow rectangular aperture!
	
	For this, we consider that the incident beam, perpendicular to the aperture, has a plane electromagnetic periodic wavefront and is given by (\SeeChapter{see section Wave Mechanics page \pageref{harmonic waves}}):
	
	where for recall, its wavelength is given by:
	
	
	\subsubsection{Case of a small rectangular aperture (slit)}
	The span width $e$ of the aperture is oriented along the $y$ axis, the height $h$ (parameter that can not be represented in the figure because it is a view from above) is assumed as very big so we can neglected the border effects.

	Following the Huygens principle , the front of the plane wave, delimited by the aperture, is a multitude of sources $\mathrm{d}f(x,t)$, of width $\mathrm{d}y$, which emit in phase, spherical wavelet described by their associated field vector:
	
	Now consider an observation point $P$, at a distance $R$ from the source (assimilated to an aperture). We have proved during our study of the emission sources of spherical type  (\SeeChapter{see section Electrodynamics page \pageref{spherical wave}}) that their amplitude decreases inversely proportional to the distance as:
	
	
	\begin{tcolorbox}[title=Remark,colframe=black,arc=10pt]
	Using modern phasor notation the above relation can be found in many textbooks in the following form:
	
	or some textbooks replace the source strength per unit length $\hat{f}/e$ (assumed uniform!) by a new term like:
	
	\end{tcolorbox}
	
	However, the wavelets, each depending on the point of the aperture to which it is assimilated, will not all travel the same distance $R$ but a distance proper distance $r$. However, if $R$ is sufficiently distant from the aperture, we will allow ourselves to approximate:
	
	remains still the periodic term $\sin(kx-\omega t)$ where we put $x=r$. But we have for extremal values:
	
	These extremal values correspond respectively, to the  advance and the delay of the wave function describing the propagation of the wavelet at the borders of the aperture.
	
	Indeed, it is enough to see the figure below, considering $R \gg e$ and therefore:
	
	\begin{figure}[H]
		\centering
		\includegraphics{img/electromagnetism/fraunhofer_rectangular_slot.jpg}
	\end{figure}
	Thus, by placing the origin of the $y$ coordinate in the middle of the aperture, we have:
	
	So the different wavelets are out of phase and produce interferences.
	
	\textbf{Definition (\#\mydef):} In wave mechanics, we speak of "\NewTerm{interference}\index{interference}" to describe a phenomenon in which two waves superpose to form a resultant wave of greater, lower, or the same amplitude. Interference usually refers to the interaction of waves that are correlated or coherent with each other, either because they come from the same source or because they have the same or nearly the same frequency. Interference effects can be observed with all types of waves, for example, light, radio, acoustic, surface water waves or matter waves.
	
	The diffracted wave in the direction of $\theta$, is given by the sum of all contributions:
	
	Knowing that (trigonometric relation are explicited on request of a reader):
	
	We have therefore:
	
	We have proved in the section Electrodynamics that the energy (i.e. the intensity) of an electromagnetic wave was given (in vacuum) by the average scalar value of the Poynting vector:
	
	We therefore have by considering that the magnetic and electric field are proportional to the term:
	
	the following result:
	
	which is the luminous emittance emitted in the direction $\theta$ and where have put:
	
	If we introduce the sinus cardinal sinc that we have meet in our study of Fourier transforms in the section of Sequences and Series and that was defined in the section Trigonometry then we can write the previous relation in a more clearly condensed way:
	
	For which can have a generic form plotted with Maple 4.00b using:
	
	\texttt{>Gamma:=3;plot((sin(Gamma*x)/(Gamma*x))\string^2, x=-Pi..Pi);}
	
	So we can get the same result by taking the squared modulus of the Fourier transform of a monochromatic signal through a rectangular window. Thus, it seems possible to study the phenomena of diffraction by using the Fourier transform, and this area is named "\NewTerm{Fourier optics}\index{Fourier optics}" that we will study later.
	
	Here is a graphical representation of the relation $\bar{S}/\overline{S_0}$ for different values of the ratio $\Gamma=e/\lambda$:
	\begin{figure}[H]
		\centering
		\includegraphics{img/electromagnetism/fraunhofer_rectangular_slot_diffraction_plot.jpg}
		\caption{Representation of diffraction fringes}
	\end{figure}
	On either side of the central fringe, there are other fringes, more narrow and arranged symmetrically. Their intensity decreases very rapidly as the preponderant  term at the denominator:
	
	\begin{figure}[H]
		\centering
		\includegraphics{img/electromagnetism/fraunhofer_rectangular_slot_diffraction_plot_zoom.jpg}
	\end{figure}
	from which we have a real image here:
	\begin{figure}[H]
		\centering
		\includegraphics{img/electromagnetism/fraunhofer_rectangular_slot_real_photo.jpg}
		\caption{Photo of a real Fraunhofer diffraction fringe of a rectangular aperture}
	\end{figure}
	Between the fringes, there are dark areas that are the seat of destructive interference. Their position is given by the condition\label{destructive interference pattern}:
	
	excepted for $n=0\Rightarrow \theta=0$ where we observe a maximum!
	
	So we see black bands in the directions:
	
	Thus the angular width of the central band  is twice the angular value obtained for the first minimum:
	
	We get the width of the following peaks as follows:

	Two successive minima satisfy the conditions:
	
	Therefore:
	
	Put putting:
	
	Therefore it comes:
	
	Since the emittance decreases very rapidly, only the first bands (for which $\cos(\theta)\cong 1$) are observable. Therefore it remains:
	
	The positions of the maxima are then given by the condition:
	
	Let us put:
	
	The numerical resolution of:
	
	gives:
	
	The positions of the successive maxima are then:
	
	etc.
	We could have also easily get a decent approximation of this result, considering that the intensity is maximum when:
	
	Which bring us to write:
	
	with $n\in \mathbb{N}^{*}$.
	
	A remarkable result of the Fraunhofer experience is that it challenges the corpuscular view of light as we had in the 19th century.

	Indeed, many experiments such at the projection of the shadow of an object on a wall seemed to show that the light was such a particle that does not pass through materials and being net stopped by any obstacle either in its center or by its edges (you must draw your attention to the "edges" in particular).

	But, Fraunhofer experience and in particular that of Fresnel regarding the edges (we will see further because it is mathematically more difficult to study), show that the light appears to behave not as a single particle but as a wave (from the Huygens principle that we used for our developments) as we showed to us the earlier developments that perfectly explain the experimental results of Fraunhofer diffraction.

	But then why keep the corpuscular model of light? Simply because of other experimental and theoretical results of which the best known are the photoelectric effect or Compton scattering (\SeeChapter{see section Nuclear Physics page \pageref{compton scattering}}), which can be explained theoretically quite well if it is not perfectly with a corpuscular  model light (and also with some other particle of different size, charge, spin, etc.).
	
	By the early 1800s, there were two theories about the nature of light. One of them, going back to Newton, is that light is a ray. But there was another idea – going back even farther (to Christiaan Huygens) – that light might also be a wave. And this gained a lot of support in 1799, when Thomas Young first passed light through two thin, nearby slits. So if you're a good theorist, and you're interested in studying light, what do you do?
	
	Well, if you're famed French mathematician and physicist Simeon Poisson, you would think of the most ridiculous configuration you could imagine in the hopes of disproving the light-is-a-wave theory. And that's exactly what he did in 1818.

	He imagined that you took a wave source of light, and had it shine on and around a completely black, spherical obstacle, setting up a screen behind it. Obviously, he reasoned, you would see some light on the screen indicating the outside of the sphere, and darkness, or a shadow, on the inside.

	But, he calculated, if the wave theory of light were correct, you would get something completely absurd!

	Sure, you'd get light on the outside, and shadow on the inside, but what's that at the very center? Poisson predicted, using the wave theory of light, that you'd actually get a bright spot of light at the very center of this shadow! How absurd was that! And therefore, he reasoned, the wave theory of light was absolutely crazy, and had to be wrong.

	Shortly after Poisson's prediction, Francois Arago decided to put the theory to the test, and actually performed the experiment to look for the "theoretically absurd" spot.

	And what happens if, in fact, you perform this experiment yourself? :
	\begin{figure}[H]
		\centering
		\includegraphics{img/electromagnetism/arago_spot.jpg}
		\caption[Arago Spot]{Arago Spot (source: Thomas Bauer at Wellesley)}
	\end{figure}

	Amazingly, the spot is real! If your theory is any good, scientifically, this is exactly what it will do. It will not only explain what's already been observed, it will allow you to apply it to new situations, and make testable predictions about what you can expect to find. The crazier the prediction, and the more successful the experiment, the more compelling the theory becomes.
	
	\paragraph{Optical resolution in the far field region for small rectangular aperture}\mbox{}\\\\\
	According to the criterion of the English physicist Lord Rayleigh: the "\NewTerm{optical resolution}\index{optical resolution}" (or "\NewTerm{power angular resolution}\index{power angular resolution}") of a circular aperture, is the angle $\alpha_{\min}$ between two light rays of wavelength $\lambda$, emitted from two point sources $S_1$,$S_2$, far away, whose diffraction figures are separated such that the first zero of the diffraction pattern is found at the place of the maximum of the other as illustrated below:
	\begin{figure}[H]
		\centering
		\includegraphics{img/electromagnetism/optical_resolution.jpg}
		\caption{Optical resolution as defined by Lord Rayleigh}
	\end{figure}
	Or with a photo of experiment perhaps it's better:
	\begin{figure}[H]
		\centering
		\includegraphics{img/electromagnetism/optical_resolution_photo.jpg}
	\end{figure}
	This concept is used extensively in photography, astronomy, radio astronomy, etc. So the reader should pay a special attention to it!
	
	Now we have proved that the minima for a rectangular aperture were given by:
	
	And if we take the case where $n=1$, we fall back on the relation available in many books without proof:
	
	named "\NewTerm{Rayleigh criterion of a small rectangular aperture (slit)}\index{Rayleigh criterion of a rectangular aperture}".
	
	Therefore the angular resolution of a rectangular aperture is proportional to the ratio of the wavelength $\lambda$ to the thickness $e$ of the slot. Obviously in practice the goal is to have either:
	
	\begin{enumerate}
		\item A large as possible value of the angle $\alpha_{\min}$ so that the objects (both source in this case) does not coincide and that their images are therefore distinct.

		\item A small as possible value  so that the sole observed object (only one source) do not have a blurry image. This is one reason why many popular books say you have to have a smaller wavelength than the observed object so that it is observable (so not too be blurred!).
	\end{enumerate}
	To increase the resolving power to decrease the blur effect, so you have to be working with shorter wavelengths or increase the thickness of the slot of the instrument, as the wavelength is often imposed by the observed object, it is natural to make vary $e$.
	
	Therefore if the light that passes through a slot (rectangular or circular!) forms an image on a screen, and that this image is observed under a microscope, for example, it is impossible, whatever the magnification of the microscope ,to observe more details in the image than it is allowed by the resolving power of the slot. We must take these considerations into account when we design of optical instruments.
	
	\subsubsection{Case of a network of small rectangular apertures (slits)}
	Let us now consider an array of $N$ apertures of width $e\cong \lambda$, of height $H \gg e$ and distant each other by $d$. A single incident beam illuminates all apertures.
	\begin{tcolorbox}[title=Remark,colframe=black,arc=10pt]
	The study of this model will allow us to understand partially how the prism works and also of goniometer in astronomy for spectrum analysis as well as X-ray diffraction by a network of atoms (the importance of the latter being quite significant!).
	\end{tcolorbox}
	Given the following figure:
	\begin{figure}[H]
		\centering
		\includegraphics{img/electromagnetism/network_of_rectangulare_apertures.jpg}
		\caption{Basic principle of the rectangular aperture network}
	\end{figure}
	We see in the diagram above that for some directions $\theta$, the distance $d\sin(\theta)$ is such that constructive or destructive interference are occurring.
	
	Let us consider that this network is placed in the $YZ$ plane and the beam direction is along the $X$ axis. We put ourselves in an observation point $P$ in the $XY$ plane. Following the properties of electromagnetic waves (\SeeChapter{see section Electrodynamics page \pageref{electromagnetic wave equation}}), the electric field vector $\vec{E}_i$ of the wave emitted by the $i$-th slot is perpendicular to the direction of observation and can be expressed as:
	
	and we also proved earlier above that:
	
	where by analogy between these two relations, we recognize that:
	
	and therefore it comes:
	
	In any direction $\theta$, the waves issued from the two adjacent apertures are out of phase of $d\sin(\theta)$ (where $d$ is the step value between two apertures) and at the point $P$ of observation, the resulting electric field is given by the sum of the contributions of each aperture with its own $i$-th shift. Therefore:
	
	So we see that each wave is out of phase of:
	
	We can now represent $E(\theta,R,t)$ using phasors (\SeeChapter{see section Wave Mechanics page \pageref{phasors}}) in the phase space such that:
	
	Which gives graphically for the second term containing the summation variable $j$ for a fixed distance $R$:
	\begin{figure}[H]
		\centering
		\includegraphics{img/electromagnetism/phasor_fresnel_apertures.jpg}
		\caption[]{Representation of the summation of terms}
	\end{figure}
	We see that the $\vec{E}_i$ put together form a regular polygon inscribed in a circle of radius:
	
	The norm of the resulting electric field being equal to the chord defined by the angle:
	
 	we will have:
	
	The light energy (ie. intensity) emitted in the direction $\theta$ is proportional to the square of the electric field (\SeeChapter{see section Electrodynamics page \pageref{poynting vector}}), then we have for destructive or constructive interference:
	
	We now substitute $\overline{S}_i(\theta)$ by the expression found in our earlier study of diffraction by a single aperture:
	
	Thus we get to the addition of interference and diffraction effects:
	
	Although this relation seems complicated, its parameters do not have the same practical significance. Indeed, consider the function:
	
	The term $A$ has maxima when:
	
	and is equal to zero if:
	
	with $m\in \mathbb{N}$.
	
	Although the term $B$ makes the relation diverge for $x=m\pi$, the Hospital's rule (\SeeChapter{see section Differential and Integral Calculus page \pageref{Hospital rule}}) provides that:
	
	It follows that for $x=m\pi$ and therefore for zero values of $A$ and $B$, the function $\psi$ presents enormous peak height:
	
	Since the huge amplitude, the main peaks are those we observe experimentally the more easy. Thus, the angular position of the maxima of the function $\psi$ is given by:
	
	with $m\in\mathbb{Z}$.

	The value $m$, designate the "\NewTerm{order number of the maximum's interference}\index{order number of the maximum's interference}".

	Let us apply these results to the interference relation:
	
	The peak of order $m$ is center on the equivalent value $\theta$ that cancel the numerator and the denominator of this fraction such that:
	
	with $m\in\mathbb{Z}$, hence:
	
	Thus an aperture network that we know the step value $d$ can be used to measure the wavelength $\lambda$ of an incident unknown light beam.
	
	However, if the incident light is polychromatic (typically for astronomical observations), the above relation gives us for a given wavelength given the position of the interference fringes. Thus, an astronomer passing polychromatic light of its telescope through a diffraction network can make a spectroscopic analysis of the light.

	This relation also gives us that for fixed values of $m$ and $d$ that the greater is $\lambda$ the greater is the angle $\theta$ is in a range $[0,2\pi]$. Thus, the spectral lines arising from the impact of a polychromatic beam show a spectrum from violet (shorter wavelength so small angle) to red (high angle so large wavelength).

	Using a goniometer, we measure the angles $\theta_m$ of the main peaks of order $m$ for the greatest possible number of $m$ values. We deduce from these measurement the slope $\lambda$ of the function plot:
	
	The peak bottom is located at an angle $\theta_m+\Delta\theta_m$ where the numerator:
	
	cancel for the first time after passage of the peak.
	
	Since the argument of this function increases of $m\pi$ between two successive peaks (among all main and secondary peaks), it is equal to $N(m\pi)$ at the place of the peak of order $m$ (main peak therefore) and has to travel $\pi$ radians further to reach the bottom of the peak.

	The numerator is then:
	
	The angular distance $\Delta \theta_m$ between the top and the bottom of the main peak is the given by:
	
	But from the first order, we have $\theta_m \ll \Delta\theta_m$. The difference of two sinus gives (\SeeChapter{see section Trigonometry page \pageref{remarkable trigonometric identities}}):
	
	A Maclaurin development (\SeeChapter{see section Sequences and Series page \pageref{usual maclaurin developments}}) gives when taking the first term of the development $\sin(\alpha)\cong \alpha$:
	
	But we also have the identity (\SeeChapter{see section Trigonometry page \pageref{remarkable trigonometric identities}}):
	
	Hence the angular width of a peak of order $m$:
	
	But as:
	
	Therefore:
	
	It is clear that two superposed light lines will be seen as distinct if they are separated by an angular distance equal to their angular width. The expression:
	
	establishes that to two angular positions correspond two wavelengths. So we can give the separation of two light lines by $\Delta \lambda=\lambda_2-\lambda_1$ instead of $\Delta\theta=\theta_2-\theta_1$.
	
	Therefore from:
	
	we get:
	
	But:
	
	When $\Delta \theta_m$ and $\theta_m$ are small, we have:
	
	Which brings us to write by substitution:
	
	The resolving power $R$ of a network is its ability to separate two spectral lines of neighbouring wavelengths $\lambda$ and $\lambda+\Delta\lambda$ such as:
	
	We see that the resolving power increases proportionally with the diffraction order. This relation is very important in spectroscopy for astronomy. We can also calculate what should be the minimum number $N$ of lines that must have a network capable of separating two wavelengths in a spectrum of given order $m$.
	
	\subsubsection{Young's interference experiment}\label{young interference experiment}
	According to the wave-particle duality principle, light behaves both as a wave and as a particle (material particle). It is the solving of problems like those of the black-body (\SeeChapter{see section Thermodynamics page \pageref{black body}}), of the photoelectric effect (\SeeChapter{see section Nuclear Physics page \pageref{photoelectric effect}}) or that of the Compton effect (\SeeChapter{see section Nuclear Physics page \pageref{compton scattering}}) , which revealed the existence of this duality.

	But we will now study the most mind blowing proof the duality aspect of the matter at the atomic scale using the Young's experiment. We will address this in a simplified manner as a special case of the rectangular slots network that has the advantage of being able to be easy to put in practice in an experiment showing the dual and probabilistic behaviour of matter at the atomic scale.

	So to study this subject, let us consider a light source $S$, which radiates monochromatic wave $\Psi$ of wavelength $\lambda$ through two slots $F_1$ and $F_2$ perforated  in an opaque barrier to light, as shown in the figure below:
	\begin{figure}[H]
		\centering
		\includegraphics{img/electromagnetism/young_experiment.jpg}
		\caption{Implementation of Young's experience-slit}
	\end{figure}
	
	\begin{tcolorbox}[title=Remark,colframe=black,arc=10pt]
	The advantage of that device is that it can produce two sources of coherent light. That is to say two sources whose phase difference is constant throughout the experiment.
	\end{tcolorbox}
	We have on a viewing (projection) screen $E$ at a point $H$ such that the distance is:
	
	where $a$ would typically be of the order of the millimetres and $D$ of the meter.
	
	The wave $\Psi$ will after passing through the slots $F_1$ and $F_2$, as we have already seen it just previously, generates sub-waves $\Psi_1(r_1,t)$ and $\Psi_2(r_1,t)$ of same pulsation $\omega$ that will travel respectively the paths $r_1$ and $r_2$ and that will go interfere at the point $M$ of the screen $E$ that we are interested for.
	
	If the interference is constructive in $M$, this point will then be located on a bright fringe and if the interference is destructive $M$, it will be on a dark fringe. To observe this, let us first write the resulting wave at $M$:
	
	in which we have in terms of phasor (\SeeChapter{see section Wave Mechanics page \pageref{phasors}}):
	
	where $A$ is the amplitude, $k$ is the wave vector and $t$ is the time variable as we have already discussed in detail in the section of Wave Mechanics.

	Now let make a change of variable (just to not have to manage long exponential functions):
	
	\begin{tcolorbox}[title=Remark,colframe=black,arc=10pt]
	We will see later that in fact $D_1=r_1$ and $D_2=r_2$.
	\end{tcolorbox}
	To calculate the intensity at the point $M$, we will take the complex norm (module) of $\Psi_M$ that is thus written as the product of the complex variable and its conjugate (\SeeChapter{see section Numbers page \pageref{complex conjugate}}):
	
	\begin{tcolorbox}[title=Remark,colframe=black,arc=10pt]
	This calculation is very important because the analogy with the Wave Quantum Physics is very strong at this level and similar to the calculation of the probability amplitude (\SeeChapter{see section Wave Quantum Physics page \pageref{amplitude of probability}}).
	\end{tcolorbox}
	Therefore:
	
	The intensity is then maximum if and only if:
	
	So that:
	
	with $n\in\mathbb{N}$. Which gives:
	
	\begin{tcolorbox}[title=Remark,colframe=black,arc=10pt]
	It is here that we see obviously that $D_1=r_1$ and $D_2=r_2$
	\end{tcolorbox}
	The intensity is zero if and only if:
	
	So that:
	
	with $n\in\mathbb{N}$. Which gives:
	
	Now we must calculate $r_1-r_2$ in function of $z$ to find out what we see on the projection screen $E$.
	
	For this let us consider the following figure:
	\begin{figure}[H]
		\centering
		\includegraphics{img/electromagnetism/magnification_and_special_case_of_young_experiment.jpg}
		\caption[]{Magnification and special case of young experiment}
	\end{figure}
	where $\overline{F_1P}$ and $\overline{F_2P}=r_2$.
	
	We have on our figure:
	
	But:
	
	So we have:
	
	As $z$ and $a$ are small relatively to $D$ and using the approximation:
	
	if $\varepsilon$ is small compared to $1$. Then we have:
	
	Also:
	
	So subtracting those two relations we get:
	
	So finally using the relation:
	
	It comes:
	
	Thus, the distance between two consecutive maxima is:
		
	and is named "\NewTerm{interfringe}\index{interfringe}".

	For fringes with zero intensity it comes immediately:
	
	This relation indicates that the intensity $I$ has minima (dark fringes) and maxima (bright fringes) distributed along the $z$ direction periodically. This does not surprise us more than that for now because it derives from the more general case studied above.
	\begin{figure}[H]
		\centering
		\includegraphics[angle=90,origin=c,scale=0.35]{img/electromagnetism/interference_fringe.jpg}
		\caption[]{Interfinge simplified schema for the Young experiment}
	\end{figure}
	It should be notice that the above calculations show that the intensity of the fringes is equal everywhere. But we observe experimentally (see figure above) that their intensity decreases with distance from the projection screen center. As we have already seen it, two are at the origin of this observation:
	\begin{enumerate}
		\item The slots have a given width, which implies a diffraction phenomenon. Indeed, a light sent on a small hole does not go out isotropically. This results in that the light is predominantly directed forward. This effect is reflected in the figure observed after Young's slots: the intensity of fringes decreases gradually as we moves away from the center of the projection screen.

		\item The fact that the waves emitted in $F_1$ and $F_2$ are spherical waves, that is to say, their amplitude decreases gradually as we go away from the screen projection center. Thus the amplitude of $F_1$ and $F_2$ will not be the same at the point $M$.
	\end{enumerate}
	So our calculations are remains approximate relatively to the study we made of the rectangular slots network but that is how Young's slots experiment is presented in most schools and this is enough to highlight the main result.
	
	The original experiment of Thomas Young can be interpreted using the simple Fresnel's laws as we have done with the network of rectangular of slots. Which highlights the wave nature of light. But this experience has subsequently be refined, particularly by ensuring that the source $S$ emits a quantum at a time. For example, it is possible in the third quarter of the 20th century to emit photons or electrons or atoms one by one. These are detected one by one placed on the screen after the double-slit. We see then that these impacts form gradually the interference pattern. By conventional laws concerning the trajectories of these particles, it is impossible to interpret this phenomenon!!! Hence the importance of the theoretical study of the Young's experiment!!!
	\begin{figure}[H]
		\centering
		\includegraphics[scale=0.55]{img/electromagnetism/young_experiment_perspective.jpg}
		\caption[Young's interference experiment]{Young's interference experiment (source: ?)}
	\end{figure}
	From left to right, top to bottom, here are the patterns obtained by accumulating $10$, $300$, $2,000$ and $6,000$ electrons with a flow of $10$ electrons / second (the same experiment with the same result has been reproduced with neutrons and atoms!). The accumulation of electrons eventually form interference fringes which is quite confusing a priori!
	\begin{figure}[H]
		\centering
		\includegraphics{img/electromagnetism/young_real_experiment.jpg}
		\caption[Photos of patterns obtained in a real Young's experience]{Photos of patterns obtained in a real Young's experience (source: ?)}
	\end{figure}
	We will come back on this crucial phenomenon in the section of Wave Quantum Physics to say a little bit more about it.
	
	For summary here is another more detailed schema of what we haven seen so far:
	\begin{figure}[H]
		\centering
		\includegraphics[scale=0.6]{img/electromagnetism/young_real_experiment_detailed_summary.jpg}	
		\caption[Young experiment summary]{Young experiment summary (source: OpenStax)}
	\end{figure}
	
	\pagebreak
	\subsection{Fresnel diffraction}\index{Fresnel diffraction}
	In optics, the "\NewTerm{Fresnel diffraction}" is diffraction mathematical analysis method that is applied to the propagation of waves in the near field. In other words it is used to calculate the diffraction pattern created by waves passing through an aperture or around an object, when viewed from \underline{relatively close} to the object. In contrast the diffraction pattern in the \underline{far field region} is given by the "Fraunhofer diffraction" methods used earlier above.
	
	\begin{tcolorbox}[title=Remarks,colframe=black,arc=10pt]
	\textbf{R1.} The formula we used earlier for the Fraunhofer diffraction and the one we will use now for the Fresnel diffraction are in fact both an approximation of a formula named the "Fresnel-Kirchhoff diffraction" who takes into account secondary waves that are propagating in various directions as per Huygens' principle (for a nice derivation see \cite{ware2015physics}).\\
	
	\textbf{R2.} Also the reader should notice that our previous study of Fraunhofer diffraction and the one that will follow on Fresnel diffraction belongs to the field of "scalar diffraction theory", which ignores polarization effects. In some situations, ignoring polarization is benign, but in other situations ignoring polarization effects produces significant errors!
	\end{tcolorbox}
	
	Disclaimer! The huge majority of what will follow, including the figures is inspired and copied from the very excellent textbooks on Optics from Eugene Hecht \cite{hecht2016optics}.
	
	\subsubsection{Case of a rectangular aperture}\label{fresnel rectangular aperture}
	Consider the configuration depicted in the figure below:
	\begin{figure}[H]
		\centering
		\includegraphics[width=1.0\textwidth]{img/electromagnetism/fraunhofer_diffraction_aperture.jpg}
		\caption[]{Fraunhofer diffraction from an arbitrary aperture, where $r$ and $R$ are very large compared to the size of the hole (source: \cite{hecht2016optics})}
	\end{figure}
	 A monochromatic plane wave propagating in the $x$-direction is incident on the opaque diffracting screen $\Sigma$ with an aperture of area $A$. We wish to find the consequent (far-field) flux-density distribution in space or equivalently at some arbitrary distant point $P$. According to the Huygens principle, a differential area $\mathrm{d}S$, within the aperture, may be envisioned as being covered with coherent secondary point sources. But $\mathrm{d} S$ is much smaller in extent than is $\lambda,$ so that all the contributions at $P$ remain in-phase and interfere constructively. This is true regardless of $\theta$; that is, $\mathrm{d}S$ emits a spherical wave. If $\rho_{A}$ is the source strength per unit area, assumed to be constant over the entire aperture, then the optical disturbance at $P$ due to $\mathrm{d}S$  is either the real or imaginary part of:
	 
	The choice is yours and depends only on whether you like sine or cosine waves, there being no difference except for a phase shift. The distance from $\mathrm{d}S$ (of coordinate $(0,y,z)$) to $P$ (of coordinate $(X,Y,Z)$) is (see figure above):
	
	and as we have seen, the Fraunhofer condition occurs when this distance approaches infinity. As before, it will suffice to replace $r$ by the distance $\overline{OP}$, that is, $R$, in the amplitude term, as long as the aperture is relatively small. But the approximation for the phase needs to be treated a bit more carefully; $k=2 \pi / \lambda$ is a large number. To that end we expand out the above relation, by making use of:
	
	we get:
	
	In the far-field case $R$ is very large in comparison to the dimensions of the aperture, and the $\left(y^{2}+z^{2}\right) / R^{2}$ term is certainly negligible. Since $P$ is very far from $\Sigma, \theta$ can still be kept small, even though $Y$ and $Z$ are fairly large, and this mitigates any concern about the directionality of the emitters (the obliquity factor). Now:
	
	and dropping all but the first two terms in the binomial expansion (\SeeChapter{see section Algebra page \pageref{binomial expansion}}), we have:
	
	The total disturbance arriving at $P$ is:
	
	Consider now the specific configuration shown in  the figure below:
	\begin{figure}[H]
		\centering
		\includegraphics[width=1.0\textwidth]{img/electromagnetism/fraunhofer_diffraction_rectangular_aperture.jpg}
		\caption[]{A rectangular aperture for near-field analysis (source: \cite{hecht2016optics})}
	\end{figure}
	The previous relation can now be written as:
	
	where $\mathrm{d} S=\mathrm{d} y \mathrm{d} z$. With $\beta^{\prime} \equiv k b Y / 2 R$ and $\alpha^{\prime} \equiv k a Z / 2 R,$ we have (we used Euler formula as proved at page \pageref{euler formula}):
	
	and similarly:
	
	so that:
	
	where $A$ is the area of the aperture. Let us denote that relation as following:
	
	where $I(0)$ is the irradiance at $P_{0}$; that is, at $Y=0, Z=0$. At values of $Y$ and $Z$ such that $\alpha^{\prime}=0$ or $\beta^{\prime}=0, I(Y, Z)$ assumes the following familiar shape (source: \cite{hecht2016optics}):
	\begin{figure}[H]
		\centering
		\includegraphics[scale=0.6]{img/electromagnetism/fraunhofer_diffraction_rectangular_aperture_1d_diffraction_pattern.jpg}
	\end{figure}
	When $\beta^{\prime}$ and $\alpha^{\prime}$ are non-zero integer multiples of $\pi$ or, equivalently, when $Y$ and $Z$ are non-zero integer multiples of $\lambda R / b$ and $\lambda R / a$, respectively, $I(Y, Z)=0,$ and we have a rectangular grid of nodal lines, as indicated in Fig. $10.31 .$ Notice that the pattern in the $Y$ $Z$ -directions varies inversely with the $y-, z$-aperture dimensions. A horizontal, rectangular opening will produce a pattern with a vertical rectangle at its center and vice versa:
	\begin{figure}[H]
		\centering
		\includegraphics[scale=0.8]{img/electromagnetism/fraunhofer_diffraction_rectangular_aperture_pattern_photo.jpg}
		\caption[Fraunhofer pattern of a square aperture ]{Fraunhofer pattern of a square aperture (source: \cite{hecht2016optics})}
	\end{figure}
	An illustration or 3D computer simulation would give the following figures (source: \cite{hecht2016optics}):
	\begin{figure}[H]
		\centering
		\includegraphics[width=1.0\textwidth]{img/electromagnetism/fraunhofer_diffraction_rectangular_aperture_3d_pattern_illustration.jpg}
		\caption[]{(a) The irradiance distribution for a square aperture. (b) The irradiance produced by Fraunhofer diffraction at a square aperture. (c) The electric-field distribution produced by Fraunhofer diffraction via a square aperture.\\ (source: R.G. Wilson, Illinois Wesleyan University)}
	\end{figure}
	
	\subsubsection{Case of a circular aperture}\label{fresnel circular aperture}
	Most experimental situations in optics (e.g. telescopes, microscopes) have circular apertures, so the study of diffractions from such apertures is of particular interest. It is interesting to know that in 1818 Augustin Fresnel entered a competition sponsored by the French Academy. His paper on the theory of diffraction ultimately won first prize, but not until it had provided the basis for a rather interesting story. The judging committee consisted of Pierre Laplace, Jean. B. Biot, Siméon D. Poisson, Dominique F. Arago and Joseph L. Gay-Lussac... a quite formidable group indeed. Poisson, who was an ardent critic of the wave description of light, deduced a remarkable and seemingly untenable conclusion from Fresnel's theory. He showed that a bright spot would be visible at the center of the shadow of a circular opaque obstacle, a result that then proved the absurdity of Fresnel's treatment. This surprising prediction, fashioned by Poisson as the death blow to the wave theory, was almost immediately verified experimentally by Arago; the spot actually existed! Amusingly enough, "\NewTerm{Poisson's spot}\index{Poisson's spot}" or "\NewTerm{Arago spot}\index{Arago spot}", as it is now called, had been observed many years earlier (1723), but this work had long gone unnoticed.
	
	We start with a plane wave incident normally on a circular hole with radius $a$:
	\begin{figure}[H]
		\centering
		\includegraphics[scale=0.7]{img/electromagnetism/spherical_wave_circular_aperture.jpg}
	\end{figure}
	in an otherwise opaque screen, and ask ourselves what is the distribution of the intensity of the light on a screen on distance $R \gg a$ away?
	
	We can use the same expression for the $f$ field that we had for the rectangular aperture as for any possible aperture, as long as the limits of integration are appropriate. So we can write for the optical disturbance at point $P$ in the far-field case:
	
	For a circular aperture this integration is most easily done with cylindrical coordinates. Look at the figure (source: \cite{hecht2016optics}):
	\begin{figure}[H]
		\centering
		\includegraphics[width=1.0\textwidth]{img/electromagnetism/circular_aperture_fresnel_diffraction.jpg}
	\end{figure}
	Then we have:
	
	The differential area is now:
	
	Then:
	
	or:
	
	and the integral becomes:
	
	In order to do this integral we need to learn a little about Bessel functions (\SeeChapter{see section Sequences and Series page \pageref{bessel functions}}) and especially the following first kind of order $0$ in its integral representation (\SeeChapter{see section Differential and Integral Calculus page \pageref{integral representation of first kind Bessel's function}}):
	
	that is a special case (for reminder...) of a more general definition of Bessel functions :
	
	They have a number of interesting properties such as the recurrence relations (\SeeChapter{see section Sequences and Series page \pageref{bessel differential reccurence relation}}):
	
	so that for example when $m=1$:
	
	In order to numerically calculate the value of a Bessel function one uses the expansion (\SeeChapter{see section Sequences and Series page \pageref{condensed expression of Bessel series}}): 
	
	Now we want to evaluate the integral:
	
	which we can do at any value of $\Phi$ since the problem is symmetric about $\Phi$. So we can simplify things greatly if we do the integral at $\Phi=0$:
	
	which becomes:
	
	Now $J_0$ is an even function so we can drop the minus sign and rewrite the expression as:
	
	To do this integral we change variables:
	
	So that:
	
	So finally we have the result:
	
	Or recognizing that $\pi a^2$ is the area of the aperture $A$ and squaring to get the intensity we write:
	
	If we want to write this in terms of the angle $\theta$ then one uses the fact that $q/R=\sin(\theta)$:
	
	Below is a plot made with Mathematica of the function $J_1(x)/x$. Notice how it peaks at $1/2$ which is why there is the factor of two in the expression for the irradiance:
	\begin{figure}[H]
		\centering
		\includegraphics[scale=0.8]{img/electromagnetism/plot_fresnel_J1_bessel_function_2d.jpg}
	\end{figure}
	We easily do the plot of $(J_1(x)/x)^2$ with Maple 4.00b for example:
	
	\texttt{>with(plots):}\\
	\texttt{>i:=(BesselJ(1,Pi*theta)/(Pi*theta))\string^2;}\\
	\texttt{>plot(i,theta=-4..4);}
	
	\begin{figure}[H]
		\centering
		\includegraphics[scale=0.8]{img/electromagnetism/fresnel_diffraction_pattern_1d.jpg}
		\caption{Fresnel circular aperture diffraction pattern 2D plot with Maple 4.00b}
	\end{figure}
	
	Below is a 3D plot also made with Mathematica of the same thing (ie. $J_1(r)/r$). Notice the rings:
	\begin{figure}[H]
		\centering
		\includegraphics[scale=0.6]{img/electromagnetism/plot_fresnel_J1_bessel_function_3d_mathemetica.jpg}
	\end{figure}
	Below is a 3D plot of $(J_1(r)/r)^2$ which corresponds to the irradiance one sees. The central peak out to the first ring of zero is named the "\NewTerm{Airy disk}\index{Airy disk}" (which contains $84\%$ of the total light intensity) - named after Sir George Biddell Airy, an English astronomer who described this pattern mathematically in 1834. We can plot that with Maple 4.00b:
	
	\texttt{>with(plots):}\\
	\texttt{>i:=(BesselJ(1,Pi*sqrt(x\string^2+y\string^2))/(Pi*sqrt(x\string^2+y\string^2)))\string^2;}\\
	\texttt{>contourplot(i,x=-3..3,y=-3..3,filled=true,coloring=[white,blue],}	
	\texttt{contours=[0.001,0.003,0.008,0.012,0.016,0.02],grid=[100,100]);}
	
	\begin{figure}[H]
		\centering
		\includegraphics{img/electromagnetism/fresnel_airy_disk_maple.jpg}
		\caption{Fresnel Airy disk 2D plot with Maple 4.00b}
	\end{figure}
	
	or the same thing in 3D still with Maple 4.00b:
	
	\texttt{>with(plots):}\\
	\texttt{>plot3d((BesselJ(1,Pi*sqrt(x\string^2+y\string^2))/(Pi*sqrt(x\string^2+y\string^2)))\string^2,x=-4..4,}\\
	\texttt{y=-4..4,orientation=[30,70],axes=box,style=patchcontour,contours=40,}
	\texttt{view=0..0.01,grid=[100,100]);}
	
	\begin{figure}[H]
		\centering
		\includegraphics[scale=0.5]{img/electromagnetism/fresnel_airy_disk_3d_maple.jpg}
		\caption{Fresnel circular aperture diffraction pattern 3D plot with Maple 4.00b}
	\end{figure}
	
	\paragraph{Optical resolution in the near field region for circular aperture}\mbox{}\\\\\
	Notice that the destructive part of the pattern where $J_1(r)/r=0$ can be numerically evaluated to give $r\cong 3.83$ for the first ring.
	
	For our circular aperture above this means the first zero occurs at:
	
	or:
	
	Therefore:
	
	In our case $a$ is the radius of the aperture and we can rewrite the expression using the diameter $D=2a$:
	
	The factor $1.22$ is more precisely $1.21966989\ldots$
	
	Light passing through any circular aperture is going to be diffracted in this manner and this sets the limit of resolution on an optical device such as a telescope. Say one is trying resolve two sources, we can say the limit of resolution is when the central spot of one Airy disk is on the zero of the other Airy pattern (say differently: the central maximum of the Airy pattern of one point emitter is directly overlapping with the first minimum of the Airy pattern of the other). This is known as the "\NewTerm{Rayleigh criterion}\footnote{There are other existing criterion in practice like the "Sparrow limit", the "Abbe Limit", the "FWHM", etc. Astronomical studies of equal brightness stars have shown that Sparrow’s criterion is by far more useful..}\label{Rayleigh criterion}".
	
	While it is possible to define other criteria, this is the most commonly used. See:
	\begin{figure}[H]
		\centering
		\includegraphics[scale=0.7]{img/electromagnetism/resolution_fresnel_diffraction_circular_aperture.jpg}
		\caption{Overlap of PSFs (point spread function) and resolution}
	\end{figure}
	In the above figure the two sources can clearly be resolved (according to Rayleigh criterion). On the right the two sources are going to be difficult to resolve...
	
	So we say that the limit of our resolution occurs when the distance $\Delta q$ between two sources is:
	
	Or in the small angle limit $\Delta \theta=\Delta q/R$:
	
	By convention (as an approximate condition) we declare that two particles are too close together to resolve (identify as two separate particles rather than one) if their central diffraction peaks overlap. since the angle spanned by the central diffraction peak is $1.22 \lambda / D$ ,this means that we say we can't resolve objects unless the angle between them (as measured from the microscope lens) is greater than $1.22 \lambda / D$.
	
	\begin{tcolorbox}[colframe=black,colback=white,sharp corners]
	\textbf{{\Large \ding{45}}Example:}\\\\
	A positive lens having a $40$ [mm] diameter is used to form the image of two stars on a CCD in a camera. If the stars are $1,000$ light-years from Earth, how far apart are they if they are just resolvable according to the Rayleigh criterion assuming $\lambda=550 [\mathrm{nm}]$?\\
	
	The solution is obviously given by:
	$$
	\Delta \theta\cong \dfrac{1.22\lambda}{D}=\dfrac{1.22\left(550 \cdot 10^{-9}\right)}{40 \cdot 10^{-3}}\cong 1.6775 \times 10^{-5} [\mathrm{rad}]
	$$
	The stars' separation, $L$, is then:
	$$
	L\cong R\Delta \theta\cong 1000\left(1.6775 \cdot 10^{-5}\right)
	$$
	and $L\cong 0.0168$ light-year.
	\end{tcolorbox}
	If the smallest resolvable separation between images is to be reduced (i.e., if the resolving power is to be increased), the wavelength, for instance, might be made smaller. Using ultraviolet rather than visible light in microscopy allows for the perception of finer detail. The electron microscope utilizes equivalent wavelengths of about $10^{-4}$ to $10^{-5}$ that of light. This makes it possible to examine objects that would otherwise be completely obscured by diffraction effects in the visible spectrum. On the other hand, the resolving power of a telescope can be increased by increasing the diameter of the objective lens or mirror. Besides collecting more of the incident radiation, this will also result in a smaller Airy disk and therefore a sharper, brighter image.
	
	So now in the case of a telescope the circular opening of the telescope creates a circular interference pattern.

	Because of this interference pattern, when we make an image of a star, it does NOT focus to a perfect point. Rather, it focuses to a disk, and if we set our telescope for high magnification and examine the image carefully, we can see that there is a disk with faint rings around it - this is the interference pattern that is caused by the circular aperture of our telescope:
	\begin{figure}[H]
		\centering
		\includegraphics{img/electromagnetism/simple_airy_disk.jpg}
		\caption[]{Idealized picture of Airy disk}
	\end{figure}
	\begin{figure}[H]
		\centering
		\includegraphics{img/electromagnetism/double_airy_disk.jpg}
		\caption[]{When close stars appear as two Airy disks}
	\end{figure}
	\begin{figure}[H]
		\centering
		\includegraphics{img/electromagnetism/double_narrow_airy_disk.jpg}
		\caption[]{Too close Airy disk that starts to make distinction difficult}
	\end{figure}
	\begin{figure}[H]
		\centering
		\includegraphics{img/electromagnetism/too_close_airy_disks.jpg}
		\caption[]{When closer together than the radius of the Airy\\ disk, we can no longer tell them apart}
	\end{figure}
	The angular resolution may be converted into a spatial resolution, $\Delta L$, by multiplication of the angle (in radians) with the distance to the object. For a microscope, that distance is close to the focal length $f$ of the objective. For this case, the Rayleigh criterion reads:
	
	This is the size, in the imaging plane, of smallest object that the lens can resolve, and also the radius of the smallest spot to which a collimated beam of light can be focused.
	
	With regard to telescopes, what is important in this equation is that the size of the Airy disk depends on the diameter of the objective ONLY, and as the diameter of the objective gets bigger, the Airy disk gets smaller.

	This means as the diameter of our scope gets bigger, we can see smaller and smaller detail --or equivalently, we can split stars that are closer together. Conversely, this means that there is a limit to the detail we can see with our scope. That limit is dictated ENTIRELY by the diameter of the scope, and it is due to the diffraction effects caused by the scope opening.

	For this reason, the radius of the Airy disk, as calculated above, is known as the "\NewTerm{diffraction limit}"\index{diffraction limit}. Note then that when we can see the rings of the Airy Disk, this signals to us that we are operating at the limit of the scope's power to resolve detail.

	The diffraction limit equation gives an answer in radians, whereas most dimensions in astronomy are given in degrees, minutes, or seconds of arc. Also because of various other factors, the actual limit of detail that an observer can see does not match exactly to the diffraction limit. Interestingly, a skilled observer can do better than the diffraction formula would suggest.
	
	In 1867, William Rutter Dawes determined the practical limit on resolving power for a telescope, known as the "\NewTerm{Dawes limit}"\index{Dawes limit}. Dawes expressed this as the closest that two stars could be together in the sky and still be seen as two stars. The Dawes Limit is $4.56$ seconds of arc, divided by the telescope aperture in inches. We can multiply the Dawes limit by $25.4$ to convert to the metric system (in [mm]), which gets us to $115.8$, and then round to a number that is more convenient when doing the math in our head, and we get the resolving power of the scope, $P_R$ in seconds of arc:
	
	In meter we can often found that later rounder to $11.6$ such that:
	
	\begin{tcolorbox}[colframe=black,colback=white,sharp corners]
	\textbf{{\Large \ding{45}}Example:}\\\\
	Hubble's mirror is $2.4\;[\text{m}] = 240\;[\text{cm}]$ across. Plugging that into the above relation, we see that Hubble’s resolution is:
	
	That's an incredibly small size; a human would have to be nearly $8,000$ kilometres away to be $0.05$ arcseconds in size!
	\end{tcolorbox}
	
	\begin{tcolorbox}[title=Remark,colframe=black,arc=10pt]
	With the example above it is quit easy to understand why the Hubble space telescope cannot resolve object size of the Apollo mission on the moon as they're are not more much bigger than human size and especially that they are (at nearest point) $356,700$ [km] away! In fact Hubble's space telescope resolution on the Moon is about $200$ meters! In other words, even a football stadium on the Moon would look like a dot to Hubble (that's also point out the difference between "seeing" a candle on the moon, and "resolve" it into a disc!).\\
	
	That's a pretty big surprise to most people. They're used to seeing magnificent detail in Hubble images, stars in galaxies and wisps of gas in beautiful nebulae. But those objects are far, far larger than the Moon lander. Hubble's resolution is 0.1'' no matter how far away an object is. Those wisps of gas appear to be finely resolved, but they're billions of kilometres across. That's a bit roomier than the lunar landers were.
	\end{tcolorbox}
	
	An angle is a fraction of a circle. We can consider the angular size of an object to be equal to its actual (linear) size $d$, which in turn is roughly the fraction of a circle with a radius equal to the distance of the object, $D$ (actually, $d$ is the "chord" of an arc, and the arc is the fraction of the circumference, thus the cosine factor mentioned in other comments. But we can ignore that for small angles.)

	So: Fraction of circle (our angle) $=\alpha/360^\circ$ (because there are $360$ degrees in a circle).
	
	Fraction of circle (size of object) $= d /(2\pi D)$ (because the circumference of the circle is $2\pi$ times the radius, and in this case we're using a circle of radius $D$).
	
	So:
	
	Rearrange:
	
	But remember, we want our number to be in arcseconds, not degrees, so
	
	Rearrange ($60$ arcminutes per degree, $60$ arcseconds per arcminute):
	
	Calculating all that out, we get:
	
	Round the constant up to $206,265$ and we're done.
	
	So let's look at our lunar descent stage. It's $4$ meters across, but $350,000,000$ meters away. That gives it an angular size of:
	
	Hey, wait a sec! This is far below Hubble's resolution! So the lander is way too small to be seen as anything more than a dot, even by Hubble. 
	\begin{tcolorbox}[title=Remark,colframe=black,arc=10pt]
	At then end of the 20th century, it seems that the KH-11 reconnaissance satellite series, first launched by the American National Reconnaissance Office, with their supposed $2.4$ meter mirror (quite similar to Hubble!) observing in the visual (i.e. at a wavelength of $500$ [nm]) have the best resolution power in the World with a diffraction limited resolution of around $0.05$ arcsec, which from an orbital altitude of $250$ [km] corresponds to a ground sample distance of $0.06$ [m]... 
	\end{tcolorbox}

	\subsubsection{Case of a circular obstacle}\label{fresnel circular obstacle}
	We can handle the case of diffraction by a circular obstacle quite easily now, using the result just obtained. For the field,
	
	Therefore the intensity is proportion to:
	
	This result has the curious property of not being zero inside the shadow of the obstacle (when $q\rightarrow 0$). To prove that fact let see first a special limit of the first order Bessel function.
	
	Take the recurrence relation at $m=1$ and use the chain rule:
	
	Therefore:
	
	Note that $J_{0}(0)=1$ and $J_{1}(0)=0$: 
	
	or:
	
	In fact, there's a sharp peak exactly in the center, with peak intensity:
	
	as $a/R\ll 1$. 
	\begin{figure}[H]
		\centering
		\includegraphics[width=1\textwidth]{img/electromagnetism/poisson_arago_spot.jpg}
		\caption[Spot of Poisson-Arago]{Spot of Poisson-Arago\index{Poisson-Arago spot} Intensity distribution behind a coin ($a= 1$ [cm]) on a detector placed at $R=5$ [m].}
	\end{figure}

		
	\pagebreak	
	\subsection{Light polarization}\label{light polarization}
	It was not before the 19th century that it was discovered the polarization of light (we will immediately explain what it is). However, at the time of Newton, we already knew a phenomenon due to polarization: the existence of crystals named "\NewTerm{birefringent crystals}\index{birefringent crystals}" (as the "Iceland spar") which have the property of refracting a single beam into two separate beams
	\begin{figure}[H]
		\centering
		\includegraphics[scale=0.6]{img/electromagnetism/icelan_spar.jpg}
		\caption{Icelan spar (crystallized calcite)}
	\end{figure}
	Now we know that the two rays refracted by such a crystal are polarize:
	\begin{figure}[H]
		\centering
		\includegraphics[scale=0.5]{img/electromagnetism/birefringent_crystal_polarization_principle.jpg}
	\end{figure}
	To understand the "polarization of light" let us come back to the case of a wave propagating on a string (\SeeChapter{see section Wave Mechanics page \pageref{wave mechanics}}). Such a wave can do it in a vertical plane (right) as well as in a horizontal plane (left) or in all intermediate levels:
	\begin{figure}[H]
		\centering
		\includegraphics{img/electromagnetism/polarized_wave_illustration.jpg}
		\caption[Pictorial representation of the light polarization concept]{Pictorial representation of the light polarization concept (source: ?)}
	\end{figure}
	In both cases, we say that the wave is "\NewTerm{linearly polarized}\index{linearly polarized}", which means that the oscillations are only and always in the same plane, named "\NewTerm{polarization plane}\index{polarization plane}". Such a wave can pass through a vertical slot if it is polarized vertically, a horizontally polarized wave can not.

	Let us recall that we have seen in the section of Electrodynamics that for electromagnetic waves, the electric $\vec{E}$ field oscillates (at least for the standard solution of Maxwell's equations) and is orthogonal to the direction of propagation.
	
	The electric field vector $\vec{E}$ of a wave can be decomposed into two perpendicular components to each other, $(\vec{E}_x,\vec{E}_y)$ if the wave propagates in the $z$ direction and each carrying half of the intensity of the wave. These two components change at any time when $\vec{E}$ varies. The result at any time is a total horizontal field and a total vertical field.
	\begin{figure}[H]
		\centering
		\includegraphics{img/electromagnetism/electric_field_wave_propragation_decomposition.jpg}
		\caption[Illustration of the decomposition of the electric field of a propagating wave]{Illustration of the decomposition of the electric field of a propagating wave (source: ?)}
	\end{figure}
	If $\vec{E}$ revolves around the propagation direction with its end describing circle, then we say that the wave is "\NewTerm{circularly polarized}\index{circularly polarized}":
	\begin{figure}[H]
		\centering
		\includegraphics{img/electromagnetism/circular_polarized_wave.jpg}
		\caption[Pictorial representation of a circular polarized wave]{Pictorial representation of a circular polarized wave (source: ?)}
	\end{figure}
	Then $\vec{E}$ remains of constant modulus but turns while moving, making one complete turn for each distance equal to one wavelength $\lambda$.
	\begin{tcolorbox}[title=Remark,colframe=black,arc=10pt]
	The light is not necessarily polarized! Every atom emits a wave train that lasts less than one hundred-millionth of a second (these wave trains are perfectly explained by the spread of a the free particle in quantum physics with the Fourier transforms as proved in the section of Wave Quantum Physics) and all these waves have no phase correlation or orientation. The resultant field in a given position in space, is the geometric sum of all these wave trains: it is constantly changing.
	\end{tcolorbox}
	Thus, the natural light is a random and quickly variable mixture of linearly polarized waves in all directions. Looking to the source we observe a resulting field $\vec{E}$ oscillating in a certain direction for a fraction of time and then suddenly jumps to a new random direction while remaining perpendicular to the direction of propagation:
	\begin{figure}[H]
		\centering
		\includegraphics{img/electromagnetism/natural_light.jpg}
		\caption[Pictorial representation of "natural" light wave]{Pictorial representation of "natural" light wave (source: ?)}
	\end{figure}
	This introduction done, let us move now to something a bit more formal:

	We therefore saw in the section Electrodynamics that a hypothetical progressive monochromatic plane wave (even if physically it cannot exist ...) propagating in the vacuum consisted of an electric field $\vec{E}$ and a magnetic field $\vec{E}$ and was characterized by its pulsation $\omega$ (or corresponding wavelength $\lambda$), its electric field amplitude $\hat{E}$ and magnetic field amplitude $\hat{B}$ and its propagation direction given by a unit vector $\vec{u}_x$, $\vec{u}_y$, $\vec{u}_z$ at choice depending on the orientation of the selected reference frame.
	
	We have also proved that these waves have remarkable structural properties, in particular:
	\begin{itemize}
		\item $\vec{E}$ and $\vec{B}$ are transverse, that is to say, their direction is at any point and at any time orthogonal to the direction of propagation (Malus theorem). This, allow us to define a wave plane, plane generated by the two directions of $\vec{E}$ and $\vec{B}$.

		\item The norms of these two vectors are connected by the relations $\hat{B}=\hat{E}/c$ where $c$ is the speed of light in vacuum (this is this huge ratio between the magnetic field and electric field of an electromagnetic wave that makes developments presented further below are done preferably compared with the electric field component of the wave).

		\item Finally, these two vectors are orthogonal to each other, and the trihedron $(\vec{E},\vec{B},\vec{u}_i)$ is an orthogonal trihedron .
	\end{itemize}
	These three properties can be summarized as we have prove it by the relation:
	
	where we have chose the reference frame such as the wave propagates in the direction $\vec{u}_z$. Furthermore, we proved that the electric field is a trigonometric wave function given by an the arbitrary phase by:
	
	Let us now position ourselves in a base $(x, y, z)$. The most general expression of the electric field of a progressive monochromatic plane wave progressing in the $\vec{u}_z$ direction can be decomposed into two components:
	
	\begin{figure}[H]
		\centering
		\includegraphics[scale=0.8]{img/electromagnetism/wave_decomposition.jpg}
	\end{figure}
	The norm of the field is therefore given by:
	
	If $\varphi_x=\varphi_y$ (which is most often the case) we have then:
	
	By choosing another beginning of time, we are lead to write:
	
	with:
	
	\begin{tcolorbox}[title=Remark,colframe=black,arc=10pt]
	The choice of writing $-\varphi$ instead of $+\varphi$ will be useful later for the use of remarkable trigonometric relations and will allow us to find the equation of an ellipse (...patience the proof is not far anymore).
	\end{tcolorbox}
	Using the phasors (\SeeChapter{see section Wave Mechanics page \pageref{phasors}}) the last relations can be reduced to:
	
	However, to describe this field, and therefore the whole wave, it is convenient to put ourselves in the plane $z=0$ and describe the evolution of the vector in $\vec{E}$ in this plane. That's what we'll do next. This is equivalent as to choose a coordinate origin following the $z$-axis. In this case, we can write:
	
	But the most general polarization is described by a complex vector normalized to unity in a two-dimensional space of components:
	
	with:
	
	such that:
	
	
	
	\subsubsection{Linear polarization}\label{linear polarization}
	\textbf{Definition (\#\mydef):} We say that a wave is "\NewTerm{linearly polarized}\index{linearly polarized}" when $\varphi=0$ or $\varphi=\pi$.

	In the first case ($\varphi=0\Rightarrow \varphi_x=\varphi_y$), we have:
	
	
	Therefore, we have $\hat{E}_{0x}$,$\vec{E}_{0y}$ which have respectively values between:
	
	
	With respect to a diagram that we will see below it should be taken into account that when a component is positive the other is also positive, and vice versa.

	We then have at every instant:
	
	which means that the field keeps a fixed direction. Hence the fact that we speak of linearly polarized wave.

	If $\varphi=\pi$ we then have:
	
	and that can therefore be written:
	
	Therefore, we have $\hat{E}_{0x}$,$\vec{E}_{0y}$ which have respectively values between:
	
	We then have at every instant:
	
	which means that the field keeps a fixed direction. Hence the fact that we speak of linearly polarized wave.
	
	\subsubsection{Elliptical polarization}
	If $\varphi$ is arbitrary, and placing ourselves at $h=0$, we always have by starting from:
	
	The first relation:
	
	as well as:
	
	hence:
	
	Moreover, we can write:
	
	In squaring the two previous relations:
	
	and by summing, we eliminate the time and get:
	
	
	We notice that if $\varphi=0$, $\varphi=\pi$ we fall back on:
	
	By the way, in the prior previous relation we recognize the equation of an ellipse (\SeeChapter{see section Analytical Geometry page \pageref{analytical expression ellipse}}):
	
	similar in all respects to the general equation of the conical we have proved in the section of Analytical Geometry that was for recall:
	
	In this case, the end of the vector $\vec{E}$ thus describes an ellipse and we talk therefore naturally of "\NewTerm{elliptical polarization}\index{elliptical polarization}".

	According to the value of $\varphi$, the ellipse may be traversed in one direction or the other. To determine this direction, let us derivate the expression of the field:
	
	with respect to time and let us put ourselves at $t=0$ still in the same wave plane in $z=0$:
	
	Therefore:
	\begin{itemize}
		\item If $0<\varphi <\pi/2$ the ellipse is travelled in the forward direction (counter clock wise) as shown in the figure below. We say then that the polarization is "\NewTerm{elliptical direct left}\index{elliptical direct left}".

		\item If $\pi/2<\varphi <\pi$ the ellipse is travelled also in the forward direction (counter clock wise) as shown in the figure below. We say then that the polarization is "\NewTerm{elliptical direct right}\index{elliptical direct right}".

		\item If $\pi<\varphi<3\pi/2$ the ellipse is travelled clockwise. We say then that the polarization is "\NewTerm{elliptical indirect right}\index{elliptical indirect right}".

		\item If $3\pi/2<\varphi<2\pi$ the ellipse is travelled clockwise as shown in the figure below. We say when the polarization is "\NewTerm{elliptical indirect left}\index{elliptical indirect left}".
	\end{itemize}
	
	\subsubsection{Circular polarization}
	If:
	
	and:
	
	we have then the equation of the ellipse which reduces to:
	
	which is the equation of a circle of radius $E_{0x,y}$, the sign being always given by the sign of the sine:
	\begin{itemize}
		\item If $\varphi=\pi/2$ it is a "\NewTerm{left circular polarization}\index{left circular polarization}"

		\item If $\varphi=3\pi/2$it is a "\NewTerm{right circular polarization}\index{right circular polarization}
	\end{itemize}
	....see figure below for a visual summary.
		
	\subsubsection{Natural polarization}
	We can consider the emission of a source as a succession of progressive theoretical monochromatic plane waves whose expression will be:
	
	These wave trains are in a particular polarization state. However, this states varies randomly from one wave train to another, and this in a very short time relative to the integration time of detectors. They therefore will see no particular bias and the field $\vec{E}$ will have no particular direction.

	We speak therefore of "\NewTerm{unpolarized light}\index{unpolarized light}". If we overlay this light to a polarized wave, we get what we name a "\NewTerm{partial polarization}\index{partial polarization}".
	
	Finally, we can summarize what we have seen so far in the following figure where we have:
	\begin{itemize}
		\item The linear polarization: $\varphi=0,\varphi=\pi,\varphi=2\pi$

		\item The linear partial polarization (not represented below)

		\item The direct elliptical polarization left $0<\varphi<\pi/2$ or right $\pi/2<\varphi<\pi$

		\item The indirect elliptical polarization left $\pi<\varphi<3\pi/2$ or $3\pi/2<\varphi <2\pi$

		\item The partial elliptical polarization (not represented below)

		\item The circular polarization left $\varphi=\pi/2$ or right $\varphi=3\pi/2$

		\item The partial circular polarization (not represented below)
	\end{itemize}
	\begin{figure}[H]
		\centering
		\includegraphics[scale=1]{img/electromagnetism/various_polarizations.jpg}
		\caption{Representations of different polarizations}
	\end{figure}
	
	We can represent this with an animated plot in Maple 4.00b (we have not put the *.gif below as we don't wanted to overload the PDF too much...) using the following commands:
	
	\texttt{> with (plots):\\
	> Ex:=1;Ey:=1;phi:=Pi/4;k:=1;omega:=1;\\
	> animate3d([x,a*Ex*cos(omega*t-k*x),a*Ey*cos(omega*t-k*x-phi)],\\
	a=0..1,x=-10..10,t=0..2*Pi,frames=15,grid=[35,35],style=patchnogrid,axes=boxed);}
	\begin{figure}[H]
		\centering
		\includegraphics[scale=1]{img/electromagnetism/wavefront_polarization_animation_maple.jpg}
		\caption{Animation of a polarized wave with Maple 4.00b}
	\end{figure}
	It is of course possible to change the parameters. For example, $\phi=\pi/2$ gives circular polarization, $\pi=\pi$ gives a linear polarization as we have prove above.

	We can also put the figure of Wikipedia which also summarizes very well the subject:
	\begin{figure}[H]
		\centering
		\includegraphics[scale=1]{img/electromagnetism/polarized_wave_illustration_wikipedia.jpg}
		\caption{Visual polarized wave principle summary as proposed by Wikipedia}
	\end{figure}
	
	\pagebreak
	\subsubsection{Malus' law}
	To polarize light, the physicist will use polarisers. We will not enter (because it is not part of the field of Wave Optics) in the details of atomic or molecular properties of matter that are the cause of the polarization of the transmitted light.

	For our purposes, we will restrict ourselves to a polariser that polarizes incident light linearly along the $x$ axis (the component $E_y$ being therefore zero). We have therefore:
	
	But, we have proved in the section of Electrodynamics during our study of Maxwell equations that:
	
	Therefore it comes for the maximum intensity (such that $e^{\mathrm{i}(\omega t-kz)}=1$):
	
	relation that is the famous "\NewTerm{Malus' law}\index{Malus' law}".

	To investigate quantitatively the polarization, we will use a polariser / analyser set . We first let light pass through a polariser whose axis makes an angle $\theta$ with the $x$-axis, then pass through a second polariser, referred to as "analyser", whose axis makes an angle $\alpha$ with the same axis (see figure below) with:
	
	whose norm is equal to unity!
	\begin{figure}[H]
		\centering
		\includegraphics[scale=0.6]{img/electromagnetism/polarizer_analyzer_set_principle.jpg}
		\caption{Simplified example of a polariser / analyser set}
	\end{figure}
	At the exit of the analyser, the electric field $\vec{E}'$ is obtained by projecting the linear polarized light ($\varphi=0$) obtained at the exit of polariser:
	
	with:
	
	on $\vec{n}$ (which means: projection = dot product, to get a vector we multiply for recall by the vector on which we do the projection):
	
	We the deduce the Malus law for the intensity\label{malus law}:
	
	for the case of a linear polarization of course. We will reuse that result for our study of quantum cryptography (\SeeChapter{see section Cryptography page \pageref{quantum computing}}).
	
	
	\subsection{Coherence and interference}
	We will now see what are the conditions such that a plane waves interfere with each other. These developments help to explain  well much about the vision of the world around us through our eye (especially why all the waves received by our retinas do not mix and so the colors either!).

	Let us consider two plane waves $\Sigma_1$ and $\Sigma_2$ of pulsation $\omega_1$ and $\omega_2$ and of wave vectors $\vec{k}_1$ and $\vec{k}_2$ propagating both parallel to the $z$-axis.

	We denote by $\Psi_1$ and $\Psi_2$the complex amplitudes of the two waves and we are interested in the average intensity observed at a point O taken as the coordinate origin:
	\begin{figure}[H]
		\centering
		\includegraphics[scale=1]{img/electromagnetism/plane_wave_coherence_incoherence.jpg}
		\caption{Representation of plane waves}
	\end{figure}
	We put:
	
	and we will assume:
	
	At the point O the complex amplitudes are written:
	
	where $\theta_1$ and $\theta_2$ represent the phases of $\Psi_1$ and $\Psi_2$.
	
	Let us now calculate the instantaneous intensity at the point O will be denoted $J(t)$. As the average intensity $I$ is proportional to the square of the amplitude, we assume it will be the same for the instantaneous intensity. Which brings us to calculate the sum of the real parts of the amplitudes of the two waves:
	
	What is written keeping in mind that (\SeeChapter{see section Numbers page \pageref{complex numbers}}):
	
	Therefore:
	
	And then we have:
	
	It follows the sum of four terms:
	
	To calculate the average intensity, we will choose an experimental approach:

	The average intensity of the exposure time $\tau$ of the detector (electronic or biological) will be given by:
	
	$I$ is then the sum of the averages of the four terms intervening in $J(t)$ given just above. For visible light (in the case of our eye), the frequencies are of the order of $10^{15}$ [Hz] and the exposure time of the detectors vary between the millisecond and the second. Then $\tau$ typically contains $10^{12}$ periods of $\Psi_1$ and $\Psi_2$!!

	Let us examinate the effect on the average value of each of the term of $J (t)$ by first recalling the relation (\SeeChapter{see section Wave Mechanics page \pageref{pulsation frequency period wave number}}):
	
	\begin{enumerate}
		\item We have using the usual integral proved in the section of Differential and Integral Calculus, the following valuation of the integrand on an important number of periods:
		
		Calculation that is traditionally written and very abusively under the following condensed form:
		
		We can estimate that a large number of periods (opening time of the detector), it is this average that will be measured (in fact its really that latter!).
		
		\item  We have identically:
		
		with the same remarks as above regarding to the detector!
	
		\item For the third term it is a bit different:
		
		However, the average of a cosine and a sine on a period is zero. So if the detector makes a measure on an exposure time above $1/2\omega_0$, that is on a large number of periods, we will have:
		
		
		\item For the fourth term it is still different in the experimental approximation. Indeed:
		
		But, $\delta\omega \ll 2\omega_0$. Therefore the detector does not have time to measure the average intensity over a whole period in first approximation since:
		
		and that this value is much much greater in the visible spectrum that the opening / sampling time of the eye that it is $0.1$ [s].
	
		Thus, we will notice the average of the fourth term by:
		
	\end{enumerate}
	The average intensity is therefore in an experimental context:
	
	or:
	
	If the pulsations $\omega_1$, $\omega_2$ are equal (or almost equal), then it is the interference between two monochromatic plane waves. The average intensity is then written:
	
	When we know that the eye interprets the intensity to form perceptions of the objects we understand why two objects of two different colors are not a perception corresponding to a mixture of the two colors because even if in the visible spectrum, the pulses are almost equal, their phase is rarely at a given point in space zero such that:
	
	There is therefore no interference and we have in reality:
	
	\begin{tcolorbox}[title=Remark,colframe=black,arc=10pt]
	During the composition of several waves, we can always consider that there is interference. However, we name "\NewTerm{interference conditions}\index{interference conditions}" the conditions of observation of these interferences, ie conditions for the result of their composition to be stable enough to be observed. It is customary to speak of "visibility", which restricts to the single observation by the (human) eye.
	\end{tcolorbox}
	We have seen for the eye that the sampling time frequency is $10\;[\text{s}^{-1}]$. Knowing that the visible light has a frequency of $f\cong 10^{14}\;[s^{-1}]$, the frequency must then be stabilized by the source during:
	
	which is materially impossible except that the source is the same. We conclude that for interference to be visible to the eye, the sources must be synchronized at best at $10^{-13}$ which in practice leads to consider only sources absolutely synchronized on a single source.

	In the previous model, we also have neglected the fact that a real wave is limited in time. A photon is represented by a limited wave packet. Given $T$, it will have a length $l_c=cT$ in a vacuum or in the air that we name "\NewTerm{temporal coherence length}\index{interference conditions}".

	A given radiation is thus a superposition of succession of wave trains whose average length is $l_c$, the successive wave trains have no phase relation between them: they can not interfere!
	
	\pagebreak
	\subsection{LASER}
	The conceptual contribution of Albert Einstein to light-matter interaction is essential. His approach, which was to introduce the concept of absorption and stimulated emission of radiation, is the origin of the LASER emission process (Light Amplification by Stimulated Emission of Radiation) used in many areas such as: read / write optical discs, satellite data transmission, conferences pointers, precision measuring devices (telemetry, anemometry), surgery (eye, hair removal, gyrometry, cutting or industrial welding, micromachining, priming nuclear reactions (megajoule  LASER) prototyping (3D printers), cleaning surfaces (ablation / LASER etching), structure analysis, cooling, etc.

	By analysing the thermal equilibrium conditions in the interaction of electromagnetic radiation with matter, Albert Einstein understood that taking into account the spontaneous emission permits only to find the Wien's distribution law (\SeeChapter{see section Thermodynamics page \pageref{wien displacement law}}). However Planck's law (\SeeChapter{see section Thermodynamics page \pageref{planck law}}) can not be obtained if we assume the existence of a stimulated emission process. This is where Einstein's work germ contain the development of coherent electromagnetic radiation sources.

	Indeed, the first proofs of a coherent emission of radiation have involved specialists in atomic physics to those of electromagnetism. This gave birth to the MASER (Microwave Amplification by Stimulated Emission of Radiation).

	It is the evolution of this multifaceted tool and the response that it has be able to bring to problems that contributed to the development of optoelectronics, optronics and biophotonics.
	
	I have hesitated during a long time to put the basic developments of the LASER in the section of Statistical Mechanics, or Thermodynamics or Electrodynamics. As in my private circle the majority of people (not scientific) associated the LASER with optics, I thought it was more appropriate to present the below developments in this section of Wave Optics.
	
	We will, as did Albert Einstein, show by contradiction that consider only stimulated absorption and spontaneous emission are insufficient to fall back on the law Planck proved in the section Thermodynamics (law which describes the radiation density at equilibrium in a cavity of finite dimension):
	
	Let us represent schematically what are stimulated absorption and spontaneous emission disregarding the notations of energy levels as used in the sections of Corpuscular Quantum Physics, Wave Quantum Physics and Quantum Chemistry:
	\begin{figure}[H]
		\centering
		\includegraphics[scale=1]{img/electromagnetism/laser_absorption_emission.jpg}
	\end{figure}
	In our gedankenexperiment, $N_\text{tot}$ atoms are divided into the populations $N_1$ and $N_2$ respectively on two energy levels $E_1$ and $E_2$.

	We observe a resonant absorption of electromagnetic radiation when the frequency $\nu$ of the radiation is equal to the energy difference between the two considered levels. The energy emitted or absorbed is then linked to the relation:
	
	The absorption rate obviously depends on the spectral energy density of the incident electromagnetic field $R(\nu,T)$, of the population $N_1$ of the lower level and finally it is probably proportional to a factor $B_{12}$ that would give the properties of the atomic system. Under these assumptions, we are led to write naturally:
	
	The phenomenon of stimulated resonant absorption is superimposed the fact that an excited system returns to its initial state with a characteristic time: the lifetime of the excited state. The de-excitation rate is proportional to the population of the upper level of the transition such as this brings us to write:
	
	The conservation of the total number of atoms:
	
	interacting with the radiation is reflected by the kinetic condition (it is just the derivative with respect to time of the preceding expression):
	
	By injecting previous relations wisely, we get:
	
	Therefore (if we fall back on Planck's law, our assumptions for the gendankenexperiment will then be checked!):
	
	The population of levels at thermodynamic equilibrium following a Maxwell-Boltzmann distribution law (\SeeChapter{see section Statistical Mechanics page \pageref{maxwell distribution}}):
	
	We therefore have:
	
	So we see well that we will never be able to fall back on Planck's law like this:
	
	with our assumptions. Of course, by choosing the constant well, we fall back on the Wien's law (\SeeChapter{see section Thermodynamics page \pageref{wien displacement law}}) but it has since the problems we already know.

	So to get the Planck's law we must have missed something or the approach is totally false! To keep it simple ... if we observe a long time the result we just get, we find that we could possibly reach our goal if the relation arising from the conservation of the number of atoms had one more term (actually, simply from the Planck's law and to developments in reverse: reverse engineering). This is the brilliant idea that had Albert Einstein and allowed the creation of the theoretical concept of LASER.

	Let us assume therefore that in addition to the spontaneous  emission and absorption, we have the concept of "\NewTerm{stimulated emission}\index{stimulated emission}" that is really not intuitive (but that appears when the developments are done in reverse):
	\begin{figure}[H]
		\centering
		\includegraphics[scale=1]{img/electromagnetism/laser_stimulated_emission.jpg}
	\end{figure}
	Therefore, in the spontaneous emission, the photon can be emitted in any direction. In the stimulated emission, we recognize (figure) two photons in the same direction, the incident photon causing the emission and the emitted photon. By filtering processes, it is then possible to obtain a monochromatic beam which can be used for power transmission, information or measurement.

	Therefore in the presence of radiation of the same frequency as that of the transition, the system has a certain probability of being desexcited to regain its fundamental or initial state:
		
	We therefore have the following kinetic equation:
	
	becomes:
	
	Therefore:
	
	Which brings us to:
	
	and as:
	
	then it comes:
	
	and then we see that to get the Planck's law:
	
	We just have to put:
	
	Bingo! The coefficients are named by tribute: "\NewTerm{Einstein coefficients}\index{Einstein coefficients}".
	
	For a system in thermal equilibrium, the lowest level of a transition is always more populated than the upper level. Accordingly, the medium behaves as an absorbent in the presence of incident radiation of the same frequency as that of the transition between the two states. However, if this equilibrium is changed so that the upper level is significantly more populated than the lower level, the system facilitates the stimulated emission process. We thus obtain an optical amplifying mechanism, which is associated with the population inversion produced through a technique named "\NewTerm{optical pumping}\index{optical pumping}".

	It must be indicate that the LASER may be made of gas mixtures, liquids doped with rare earths, semiconductor materials (laser diodes), crystals or glasses doped with active ions.

	Also let us mention that through this model, Albert Einstein was the first to introduce the probabilities in quantum physics in 1916. This led Max Born ten years later to advance a probabilistic interpretation of the Schrödinger's wave function (\SeeChapter{see section Wave Quantum Physics page \pageref{schrodinger wave equation}}).
	
	\pagebreak
	\subsection{Fiber Optics}\label{fiber optics}\index{fiber optics}
	The concept of channelling light within a long, narrow dielectric (via total internal reflection) has been around for quite a while. John Tyndall (1870) showed that light could be contained within and guided along a thin stream of water. Soon after that, glass "light pipes" and, later, threads of fused quartz were used to further demonstrate the effect. But it wasn't until the early 1950s that serious work was done to transport images along bundles of short glass fibers\footnote{All the text that will follow on this naive introduction to fiber optic is copied from \cite{hecht2016optics} and the reader must keep in mind that there are entire books (more than $1,000$ pages) on that topic.}.

	After the advent of the LASER (1960), there was an immediate appreciation of the potential benefits of sending information from one place to another using light, as opposed to electric currents or even microwaves. At those high optical frequencies (of the order of $10^{15}$ [Hz] ), one hundred thousand times more information can be carried than with microwaves. Theoretically, that's the equivalent of sending tens of millions of television programs all at once on a beam of light. It wasn't long (1966) before the possibility of coupling lasers with fiberoptics for long-distance communications was pointed out. Thus began a tremendous technological transformation that's still roaring along today.

	In 1970 researchers at the Corning Glass Works produced a silica fiber with a signal-power transmission of better than $1 \%$ over a distance of $1$ [km] (i.e., an attenuation of $20\;[\text{dB}\cdot \text{km}^{-1}]$), which was comparable to existing copper electrical systems. During the next two decades, the transmission rose to about $96 \%$ over $1$ [km] (i.e., an attenuation of only $0.16\; [\text{~dB}\cdot \text{km}^{-1}]$).

	Because of its low-loss transmission, high-information-carrying capacity, small size and weight, immunity to electromagnetic interference, unparalleled signal security, and the abundant availability of the required raw materials (i.e., ordinary sand), ultrapure glass fibers have become the premier communications medium of the end of 20th century and early 21st century.

	As long as the diameter of these fibers is large compared with the wavelength of the radiant energy, the inherent wave nature of the propagation is of little importance, and the process obeys the familiar laws of Geometrical Optics. On the other hand, if the diameter is of the order of $\lambda$, the transmission closely resembles the manner in which microwaves advance along waveguides. 
	
	the thin-film variety, are of increasing interest, this discussion will be limited to the case of relatively large-diameter fibers, those about the thickness of a human hair.

	Consider the straight glass cylinder of the figure below:
	\begin{figure}[H]
		\centering
		\includegraphics[width=0.7\textwidth]{img/electromagnetism/fiber_optic_dielectric_cylinder.jpg}
		\caption[Rays reflected within a dielectric cylinder]{Rays reflected within a dielectric cylinder (source: \cite{hecht2016optics})}
	\end{figure} 	
	surrounded by an incident medium of index $n_{i}$ - let it be air, $n_{i}=n_{a}$. Light striking its walls from within will be totally internally reflected (see page \pageref{total reflection}), provided that the incident angle at each reflection is greater than:
	
	where $n_{f}$ is the index of the cylinder or fiber. As we will show, a meridional ray (i.e., one that is coplanar with the central or optical axis) might undergo several thousand reflections as it bounces back and forth along a fiber, until it emerges at the far end. If the fiber has a diameter $D$ and a length $L$, the path length $\ell$ traversed by the ray will be:
	
	But we know that:
	
	and using Snell's law:
	
	We have then:
	
	Therefore:
	
	The number of reflections $N_{r}$ is then given by:
	
	or:
	
	rounded off to the nearest whole number. The $\pm 1,$ which depends on where the ray strikes the end face, is of no significance when $N_{r}$ is large, as it is in practice. Thus, if $D$ is $50\; [\mu \text{m}]$ (a hair from the head of a human is roughly $50\;[\mu \text{m}]$ in diameter), and if $n_{f}\cong 1.6$, $n_a\cong 1$ (therefore $n_{21}\cong 1.6$) and $\theta_{i}=30^{\circ}$, $N$ turns out to be approximately $6,000$ reflections per meter. Fibers are available in diameters as small as $2\; [\mu \text{m}]$ or so but are seldom used in sizes much less than about $10\; [\mu \text{m}]$. Extremely thin glass (or plastic) filaments are quite flexible and can even be woven into fabric.

	The smooth surface of a single fiber must be kept clean (of moisture, dust, oil, etc.), if there is to be no leakage of light (via frustrated total internal reflection). Similarly, if large numbers of fibers are packed in close proximity, light may leak from one fiber to another in what is known as cross-talk. For these reasons, it is customary to enshroud each fiber in a transparent sheath of lower index called a cladding. This layer need only be thick enough to provide the desired isolation, but for other reasons it generally occupies about one tenth of the cross-sectional area. Although references in the literature to simple light pipes go back 100 years, the modern era of fiberoptics began with the introduction of clad fibers in 1953.
	
	Typically, a fiber core might have an index $\left(n_{f}\right)$ of $1.62$, and the cladding an index $\left(n_{c}\right)$ of $1.52$, although a range of values is available. A clad fiber is shown in the figure below:
	\begin{figure}[H]
		\centering
		\includegraphics[width=0.7\textwidth]{img/electromagnetism/fiber_optic_rays_in_a_clad_optical_fiber.jpg}
		\caption[Rays in a clad optical fiber]{Rays in a clad optical fiber (source: \cite{hecht2016optics})}
	\end{figure} 	
	Notice that there is a maximum value $\theta_{\max }$ of $\theta_{i}$, for which the internal ray will impinge at the critical angle, $\theta_{c}$. Rays incident on the face at angles greater than $\theta_{\max}$ will strike the interior wall at angles less than $\theta_{c}$. They will be only partially reflected at each such encounter with the core-cladding interface and will quickly leak out of the fiber. Accordingly, $\theta_{\max}$, which is known as the "acceptance angle", defines the half-angle of the acceptance cone of the fiber. To determine it, we start the Snell's law at the medium–core interface that gives (look closely at the $n_i$ in the figure above!):
	
	From the geometry of the above figure we have:
	
	where:
	
	is the critical angle for total internal reflection.
	
	Substituting $\cos(\theta_c)$ for $\sin(\theta_r)$ in Snell's law we get:
	
	By squaring both sides:
	
	Solving, we find:
	
	More commonly we find the light entering the fiber will be guided if it falls within the acceptance cone of the fiber, that is if it makes an angle with the fiber axis that is less than the "\NewTerm{acceptance angle}\index{acceptance angle}" under the following form:
	
	The quantity $n_{i} \sin (\theta_{\max })$ is defined as the "\NewTerm{numerical aperture}\index{numerical aperture}", or NA. Its square is a measure of the light-gathering power of the system. The term originates in microscopy, where the equivalent expression describes the corresponding capabilities of the objective lens. The acceptance angle $\left(2 \theta_{\max }\right)$ corresponds to the vertex angle of the largest cone of rays that can enter the core of the fiber. It should clearly relate to the speed of the system, and, in fact:
	
	Thus for a fiber:
	
	The left-hand side of:
	
	cannot exceed $1$, and in air $\left(n_{a}=1.00028 \approx 1\right)$ that means that the largest value of $\mathrm{NA}$ is $1$. In this case, the half-angle $\theta_{\max }$ equals $90^{\circ}$, and the fiber totally internally reflects all light entering its face. Fibers with a wide variety of numerical apertures, from about $0.2$ up to and including $1.0$, are commercially obtainable.
	
	\begin{tcolorbox}[colframe=black,colback=white,sharp corners]
	\textbf{{\Large \ding{45}}Example:}\\\\
	A fiber has a core index of $1.499$ and a cladding index of $1.479$. When surrounded by air what will be its numerical aperture will be:
	$$
	\mathrm{NA}=\sqrt{n_{f}^{2}-n_{c}^{2}} =\sqrt{1.499^{2}-1.479^{2}}=0.244
	$$
	which is a typical value.\\
	
	It's maximum angle will be:
	$$
	\sin (\theta_{\max })=\dfrac{1}{n_{i}} \mathrm{NA}=\mathrm{NA} \Rightarrow \theta_{\max }=\sin ^{-1}(0.244)=14.1^{\circ}
	$$
	Hence
	$$
	2 \theta_{\max }=28.2^{\circ}
	$$
	The critical angle follows from:
	$$
	\sin(\theta_{c})=\dfrac{n_{r}}{n_{i}}=\frac{n_{c}}{n_{f}}=\frac{1.479}{1.499}
	$$
	Notice that $\sin(\theta_{c})$ must be equal to or less than $1$L
	$$
	\theta_{c}=\sin ^{-1} (0.9866)=80.6^{\circ}
	$$
	\end{tcolorbox}
	Bundles of free fibers whose ends are bound together (e.g., with epoxy), ground, and polished form flexible light-guides. If no attempt is made to align the fibers in an ordered array, they form an incoherent bundle. This unfortunate use of the term incoherent (which should not be confused with coherence theory) just means, for example, that the first fiber in the top row at the entrance face may have its terminus anywhere in the bundle at the exit face. These flexible light carriers are, for that reason, relatively easy to make and inexpensive. Their primary function is simply to conduct light from one region to another. Conversely, when the fibers are carefully arranged so that their terminations occupy the same relative positions in both of the bound ends of the bundle, it is said to be coherent. Such an arrangement is capable of transmitting images and is consequently known as a flexible image carrier.

	Coherent bundles are frequently fashioned by winding fibers on a drum to make ribbons, which are then carefully layered. When one end of such a device is placed face down flat on an illuminated surface, a point-by-point image of whatever is beneath it will appear at the other end. These bundles can be tipped off with a small lens, so that they need not be in contact with the object under examination. Nowadays it is common to use fiberoptic instruments to poke into all sorts of unlikely places, from nuclear reactor cores and jet engines to stomachs and reproductive organs. When a device is used to examine internal body cavities, it's called an endoscope. This category includes bronchoscopes, colonoscopes, gastroscopes, and so forth, all of which are generally less than about $200$ [cm] in length. Similar industrial instruments are usually two or three times as long and often contain from $5,000$ to $50,000$ fibers, depending on the required image resolution and the overall diameter that can be accommodated. An additional incoherent bundle incorporated into the device usually supplies the illumination.
	
	The total time delay between the arrival of the axial ray and the slowest ray, the one travelling the longest distance, is $\Delta t=$ $t_{\max }-t_{\min } .$ Here, referring back to the first figure above the minimum time of travel is just the axial length $L$ divided by the speed of light in the fiber:
	
	The non-axial route $(\ell)$, given by $\ell=L / \cos (\theta_{t})$, is longest when the ray is incident at the critical angle, whereupon $n_{c} / n_{f}=\cos (\theta_{t})$. Combining these two, we get $\ell=Ln_{f} / n_{c}$, and so:
	
	Thus it follows that, subtracting both previous relation that:
	
	As an example, suppose $n_{f}=1.500$ and $n_{c}=1.489$. The delay, $\Delta t / L$, then turns out to be $37\; [\text{ns}\cdot \text{km}^{-1}]$. In other words, a sharp pulse of light entering the system may be spread out in time some $37$ [ns] for each kilometre of fiber traversed. Moreover, travelling at a speed $v_{f}=c / n_{f}=2.0 \cdot 10^{8}\; [\text{m}\cdot \text{s}^{-1}]$, it may spread in space over a length of $7.4\;[\text{m}\cdot \text{km}^{-1}]$. To make sure that the transmitted signal will still be easily readable, we might require that the spatial (or temporal) separation be at least twice the spreadout width. Now imagine a line to be $1.0$ [km] long.

	In that case, the output pulses are $7.4$ [m] wide on emerging from the fiber and so should be separated by $14.8$ [m]. This means that the input pulses must be at least $14.8$ [m] apart; they must then be separated in time by $74$ [ns] and so cannot come any faster than one every $74$ [ns], which is a rate of 13.5 million pulses per second. In this way the inter-modal dispersion (which is typically $15$ to $30\; [\text{ns}\cdot \text{km}^{-1}]$ ) limits the frequency of the input signal, thereby dictating the rate at which information can be fed through the system. Stepped-index multimode fibers are used for low-speed, short-distance lines.

These large-core fibers are used mainly in image transmission and illumination bundles. They're also useful for carrying high-power laser beams where the energy is distributed over a larger volume, thereby avoiding damage to the fiber.

	For more details the reader can refer again to \cite{hecht2016optics} where twelve pages are dedicate to the subject. Otherwise there are plenty of textbooks or handbooks of a few hundreds or even thousands of pages dedicated only to that topic.
	
	\begin{flushright}
	\begin{tabular}{l c}
	\circled{90} & \pbox{20cm}{\score{3}{5} \\ {\tiny 30 votes,  64.00\%}} 
	\end{tabular} 
	\end{flushright}

\chapter{Atomistic}

	\textit{\textbf{The atomic physics is the part of physics that deals with quantified energy states of corpuscular and wave particles and of exchange of energies within the atom}}. (Larousse)
	\minitoc
	\pagebreak
	\input{Chapter_Atomistic.tex}
	

\chapter{Cosmology}

	\textit{\textbf{Cosmology is the science that studies the structure, evolution and the general laws of the universe as a whole}}. (Larousse)
	\minitoc
	\pagebreak
	\input{Chapter_Cosmology.tex}

	
\chapter{Chemistry}

	\textit{\textbf{Chemistry is the science that studies the nature and properties of simple substances, the molecular action of these bodies on each other and combinations due to this action.}}(Larousse)
	\minitoc
	\pagebreak 
	\input{Chapter_Chemistry.tex}
	
	
\chapter{Theoretical Computing}
	\label{theoretical computing}
	\textit{\textbf{The theoretical computer science is the branch of science that deals with the development of algorithms and theoretical tools that can be apply to IT to solve formal problems related to the simulation of physical phenomena or data treatments and the exchange of information.}}(Larousse)
	\minitoc
	\pagebreak 
	\input{Chapter_Computing.tex}
	
	
\chapter{Social Sciences}

	\textit{\textbf{Social mathematics are the analysis and formal modeling tools of the behavior and management of a population and its trade in goods in all its activities, in interaction with its environment.}} (Sciences.ch)
	\minitoc
	\pagebreak
	\input{Chapter_SocialSciences.tex}
	
		
\chapter{Engineering}

	\textit{\textbf{Engineering is the set of practices consisting to apply the results of the exact sciences and basic research to practical industrial or daily problems.}} (Larousse)
	\minitoc
	\pagebreak
	\input{Chapter_Engineering.tex}

	
	
 	\chapter{Epilogue}
	You have traveled a long distance with us through this book. We have now reached the epilogue, where by tradition, the main redactor is allowed to give voice its own personal opinions. 

	Indeed, I want to leave you with some of my thoughts on theory versus practice, business and engineering education, applied mathematics research, and what I hope you will take with you after having read this book.
	
	By nature, academic engineering is very closely related to its practice. Theory and practice are ruled by the same ideas. As an academic myself teaching since 2001 in Fortune 500 and SME companies, I am proud to claim that the majority of engineering ideas were either invented or developed in academia first before they crossed over into practice.
	
	But engineering research is not just for aspiring academics: As i know it very well management and economics consultants are basically researchers even if most of them have a very low analytic level. Firms like McKinsey, Ernst \& Young, KPMG and Accenture may have different audiences, production speeds, team systems, and publication and evaluation processes, but they research the same issues that academics do and with the same methods at the exception that the level is significantly lower.

	There is also much cross-fertilization: Many professors work regularly with major consulting in USA or asset-investment firms and some have even quit academia altogether to quadruple their pay (or increase their global hapiness...).

	Because engineering is by nature such an applied discipline, after reading this book, you should probably not need anything else to understand engineering research today. In an ideal World, you should be able to read the current state-of-the-art research right now.
	
	So, do we really understand engineering? Certainly not fully! We haven seen through this book that finance is as much an art as it is a science. Given our deficiencies, given that all our methods have their errors, what should we do? My best advice to you is to use common sense, to employ a number of different techniques to come up with a range of possible answers, avoid cognitive biases, and to then make a judgment at the end of the day as to what estimate appears most reasonable in light of different models and your peer-reviewers.
	
	Our book has covered the principles of engineering in some depth and breadth. You should be very well prepared now for the next steps in your lifetime engineering education.
	
	I have enjoyed writing this book in the same way that I enjoy writing my training books, and pretty much for the same reason: It has been like solving an intriguing puzzle that no one else has figured out in quite the same way a particular way to see and explain finance. Of course, writing it has taken me quite a long times (20 years, without translation: 15 years!).
	
	But it will all have been worth it if you have learned from it. If you have studied the book, you should now know about $99\%$ of what I know about engineering. Interestingly, there were a number of topics that I thought I had understood, but had not - and it was only my having to explain them to you that clarified them for me, too. And this brings me to a key point that I want to leave you with - never be afraid to ask questions, even about first principles. To do so is not a sign of stupidity on the contrary, it is often a sign of deepening awareness and understanding.

	I have no illusions: You will not remember all the fine details in this book as time passes - I know I won't. But more than the details, I hope that I will have left you with an appreciation for the big ideas, an arsenal of tools, a method for approaching novel problems, and a new perspective. You can now think like an Engineer or even better: as a Scientist! 
	\begin{figure}[H]
		\centering
		\includegraphics[scale=0.1]{img/knowledge_is_power.jpg}	
	\end{figure}
	
	\chapter{Biographies}
	\input{Chapter_Biographies.tex}

	\chapter{Chronology}
	When arriving at the three thousandth A4 page of writing this book and at the occasion of the 3rd edition, it seemed appropriate to us to try to give a rough timeline of the majority of subjects mentioned in this book\footnote{Even if for some dates it is not possible to check if they are legends or real facts...}. This gives a better perspective on the tools used and also to pay tribute to our illustrious predecessors whom we owe our quality of life, our longevity, our mastery of the environment (not necessarily its respect...) and of its understanding.

If there are important dates missing (but only on subjects near of those presented in the various sections of this book!) or that you identify errors, do not hesitate to let us know, it is a first draft, and therefore the chronology can only be improved.

For more information the reader may refer to \href{http://www.wikipedia.com}{{\color{blue} Wikipedia}}, which has reached the top level in the number of available historical dates and in quality (with verification of sources!).

\begin{center}
\textit{This is the story of how history made science and how science entered in history, and how the ideas which emerged made the modern World.}
\end{center}

\includegraphics[width=\textwidth]{img/raphael_school_of_athens.jpg}

\textbf{+2016}\\
The first observation of gravitational waves was made on 14 September 2015 and was announced by the LIGO and Virgo collaborations on 11 February 2016.

\textbf{+2013}
On 14 March 2013 CERN confirmed that CMS and ATLAS have compared a number of options for the spin-parity of this particle, and these all prefer no spin and even parity. This, coupled with the measured interactions of the new particle with other particles, strongly indicates that it is a Higgs boson. This also makes this particle the first elementary scalar particle to be discovered in nature. In July 2017, CERN confirmed that all measurements still agree with the predictions of the Standard Model.

\textbf{+2006}
The  cognitive psychologist and computer scientist Geoffrey Everest Hinton publish \textit{A fast learning algorithm for deep belief nets}, which rejuvinates interest in Deep Learning. 

\textbf{+2001}\\
First release of Opera Magistris but under the name "Sciences.ch". A compendium on Applied Mathematics that has for purpose to merge all the modern knowledge on STEM (science, technology, engineering, and mathematics) and beyond at the Bachelor/Master level with a maximum of details in mathematical developments.

\textbf{+1995}\\
Michel Mayor and Didier Queloz definitively observe the first extrasolar planet around a main sequence star. The same year the first gaseous Bose-Einstein condensate is produced by the physicists Eric Cornell and physicists Carl Wieman at the University of Colorado at Boulder NIST–JILA lab, in a gas of rubidium atoms cooled to $170$ nanokelvins.

\textbf{+1994}\\
The works of the mathematician Andrew Wiles (more than 10 years of research!) give a solution to the Fermat's last theorem. First algorithm using Quantum computer for Prime factorization by Peter Shor.

\textbf{+1992}\\
Support vector machines (SVMs) are invented.

\textbf{+1986}\\
David E. Rumelhart, Geoffrey E. Hinton and Ronald J. Williams invent backpropagation, a new learning procedure for networks of neurone-like units. The engineers Bill Smith and Mikel J. Harry while working at Motorola introduces Six Sigma ($6\sigma$), a set of techniques and tools for scientific process improvement.

\textbf{+1983}\\
The physicists Carlo Rubbia, Simon van der Meer, and the CERN UA-1 team discovered the W and Z bosons which confirm the unification of the weak nuclear and the electromagnetic forces.

\textbf{+1982}\\
The astrophyicist Werner Becker discover the first millisecond pulsar. The same year, Alain Aspect team observe the violation of Bell inequalities.

\textbf{+1978}\\
The cryptologists mathematicians Ronald Rivest, Adi Shamir and Leonard Adelman propose a public key encryption procedure named "RSA" based on the difficulty of factorization into primes numbers.

\textbf{+1976}\\
Development of a method by a Peter Mansfield and Andrew Maudsley to makes possible nuclear magnetic resonance scanners (NMR).

\textbf{+1975}\\
The computer scientist John Holl invents genetic algorithms.

\textbf{+1973}\\
The mathematician Fischer Black and economist Myron S. Scholes publish a financial asset pricing model.

\textbf{+1969}\\
In his \textit{Control charts for measurements with varying sample sizes} the quality engineer Irving Wingate Burr introduce the famous unbiased constants $c_4$ and $d_2$ that bear his name.

\textbf{+1968}\\
Hidden Markov Model (HMM) invented

\textbf{+1967}\\
Jocelyn Bell Burnell and Antony Hewish discover the first pulsar.

\textbf{+1965}\\
Detection by the astrophysicists Arno Penzias and Robert Wilson of the cosmic microwave due to sky background radiation predicted by the theory of the physicist Robert Dicke. Development of the Fast Fourier Transform algorithm by James W. Cooley and John W. Tukey that has huge application in sciences. The physicist John Stewart Bell discover the Bell inequalities.

\textbf{+1964}\\
The astrophysicist Irwin Shapiro predicts a gravitational time delay of radiation travel as a test of General Relativity. The same year, the physicist John Stewart Bell shows that all local hidden variable theories must satisfy Bell's inequality.

\textbf{+1963}\\
The mathematician and meteorologist Edward Lorenz found what is probably the first strange attractor and opens the way to chaos theory.

\textbf{+1962}\\
The mathematician Benoît Mandelbrot discovered fractals by chance in the analysis of signals located at Bell Laboratories in the United States where he will use computers for repeat graphic patterns endlessly and whose principle is the basis of the theory of fractals. The same year, the economist William Forsyth Sharpe publishes the CAPM (Capital Asset Pricing Model). 

\textbf{+1960}\\
The physicist Abdus Salam postulates the existence of W and Z bosons to explain beta decay and the emergence of a new Z boson, which had never been seen before. The same year, the pysicists Ali Javan and Theodore Maiman invented each particular type of LASER. The mathematicians and engineers Irving Reed and Gustave Solomon present the Reed-Solomon error-correcting code.

\textbf{+1959}\\
The physicists Yakir Aharonov and David Bohm predict the Aharonov-Bohm effect (particle turning around a magnetic field but in a region where the magnetic field is zero is sensible to the field through the vector potential) and the following year the physicist Robert G. Chambers confirmed the effect experimentally. 

\textbf{+1958}\\
The perceptron is developed at Cornell University by Frank Rosenblatt. It is the first program able to learn by trial and errors.

\textbf{+1957}\\
The physicists John Bardeen, Leon Neil Cooper and John Robert Schrieffer propose and theory of supraconductivity.

\textbf{+1956}\\
The physicists Clyde Cowan and Fred Reines observe the neutrino hypothesized 25 years ago by the physicist Wolfgang Pauli. The same year, the  linguist, philosopher, cognitive scientist, historian, social critic, and political activist Avram Noam Chomsky publish Three Models for the Description of Language where he introduces the classification of formal grammars named today the "Chomsky hierarchy", which contains the class of out-of-context grammars, playing an important role in computer science to create programming languages. In the Logic Theorist by the cognitive computer scientist Allen Newell and the political scientist, economist, sociologist, psychologist, and computer scientist Herbert Simon, the first artificial intelligence computer program is proposed, which produced evidence in Russell and Whitehead's Principia Mathematica system. For one of the theorems, the program produces a simpler proof than that presented in the Principia!

\textbf{+1955}\\
The physicists Owen Chamberlain, Emilio Gino Segrè, Clyde Wiegand and Thomas Ypsilantis discover the antiproton.

\textbf{+1954}\\
The economist Harry Markowitz published his thesis on the efficient diversification model of financial assets portfolios. The same year, the physicists John Bell and the duo Wolfgang Pauli and Gerhart Lüders develop the CPT theory analyzing the symmetry of physical laws for transformations involving simultaneously the charge, parity and time. The physicist Charles Hard Townes developed the MASER. The biologist Milner Baily Schaefer publishes his equilibrium populations model.

\textbf{+1953}\\
Monte Carlo Markov Chain (MCMC) invented. Bayesian inference finally becomes tractable on real problems.


\textbf{+1952}\\
The mathematician George Bernard Dantzig developed the simplex algorithm for operational research.

\textbf{+1951}\\
The CPT theorem appears for the first time implicitly in the work of the physicist Julian Schwinger to prove the correlation between spin and statistics.

\textbf{+1950}\\
The physicists Johannes Hans Daniel Jensen and Maria Goeppert-Mayer developed the shell model of the nuclear core. The same year, the economist and mathematician John Forbes Nash developed the concept of non-cooperative games and generalizes the notion of minimax for zero-sum games; the engineer David Huffman found the algorithm used to compress any type of series symbols. In his \textit{Error-detecting and error-correcting codes} the mathematician Richard Wesley Hamming presents the Hamming family of code, linear codes used to o detect and correct transmission errors.

\textbf{+1949}\\
The mathematician and electrical engineer Claude Shannon published an article containing the information theory which will became the foundation of a number of physical theories, statistics and numerical methods. The physicist Richeard Feynman proposed the interpretation of the positron as an electron moving backward in time in his paper \textit{The Theory of Positrons}.

\textbf{+1948}\\
The physicist Maria Göpper-Meyer develops with success a theoretical model for the structure of the atomic nucleus and textile engineer and statistician Genichi Taguchi developed the experimental designs (DOE) that bear his name. The physicist Richard Feynman introduce the diagrams that bear his name and also the path integral formulation. The physicist Polykarp Kusch measure the anomalous magnetic moment of the electron (deviation of theoretical prediction of Dirac theory) thus leading to reconsideration of and innovations in quantum electrodynamics.

\textbf{+1947}\\
The physicists Cecil Powell, Cesare Mansueto Giulio Latte, and Giuseppe Occhialini discover the pion in the study of cosmic rays. The same year the physicists John Bardeen, Walter H. Brattain and William Schockley invent semiconductor transistors in the laboratories of the Bell telephone company in the United States that will cause the computer revolution. Measurement of the Lamb shift by Willis Eugene Lamb (shift of energy spectrum non-predicted by Dirac theory but explain in the framework of Quantum Electrodynamics).

\textbf{+1946}\\
The physicist and chemist Willard Frank Libby develops and discovers the possibility of Carbon 14 dating. The same year, the physicists Walter Houser Brattain, John Bardeen and William Bradford Shockley discover the transistor effect.

\textbf{+1945}\\
The Trinity Test, the first successful detonation of a nuclear weapon by the physicist Robert Oppenheimer and his team in New Mexico.

\textbf{+1944}\\
The mathematician John von Neumann developed the foundations of the mathematical theory of games.

\textbf{+1943}\\
The physicist Tomonaga Sin-Itiro published an article posing the physical basis of quantum electrodynamics.

\textbf{+1942}\\
The physicist Enrico Fermi and his team conducted the first controlled chain reaction in the purpose to build the first atomic bomb.

\textbf{+1941}\\
The physicist Ernst Stueckelberg interprets positrons as electrons with positive energy traveling back in time. The physicist Lev Davidovich Landau publish a theory of superfluidity.

\textbf{+1940}\\
The physicist Edward Teller sees the possibility of using the enormous amount of heat generated by the explosion of a fission bomb to trigger the nuclear fusion process. This is the approach considered as the discovery of nuclear fusion. The same year, the physicist William Donald Kerst developed the first betatron. The physicist John Wheeler in a telephone call to Richard Feynman hypothesizes that all electrons and positrons are actually manifestations of a single entity moving backwards and forwards in time (the "one-electron universe postulate").

\textbf{+1939}\\
The chemists Otto Hahn and Fritz Strassmann bombard uranium with neutrons and discovered that barium is produced by the experience (discovery of nuclear fission). The same year, the physicists Lise Meitner and Otto Robert Frisch determine that nuclear fission occurred during the Hahn-Strassman experiment. The physicists Wolfgang Pauli, Markus Fierz and Frederik Jozef Belinfante prove that the properties of permutation of identical particles, bosons or fermions are controlled by their spin.

\textbf{+1938}\\
The chemist and physicist Isidor Isaac Rabi and his colleagues studied the effects of placing beams of molecules in strong external magnetic fields, leading to the development of nuclear magnetic resonance (NMR). The same year, the physicists Hans Bethe and Carl von Weizsäcker propose a nuclear theory of stars and the mathematician and electrical engineer Claude Shannon publish what is probably the most famous master's thesis of the 20th century (\textit{A symbolic analysis of relay and switching circuits}), and prove that it is possible to simplify the design of logic circuits by using the Boolean algebra. This master's thesis has played an important role in the design of electronic computers. The physicist Piotr Leonidovich Kapitza observed the phenomenon of superfluidity.

\textbf{+1937}\\
The physicists Seth Neddermeyer, Carl Anderson, Jabez Curry Street and E.C. Stevenson discover muons in the traces left by cosmic rays in a bubble chamber. The same year the mathematician John von Neumann developed the Monte Carlo methods for various numerical methods and the physicist Niels Bohr developed the liquid drop model of the nucleus.

\textbf{+1936}\\
The physicists George Gamow and Edward Teller work together to formulate the theory of beta radioactive emissions. The same year, in his \textit{On computable numbers}, the computer scientist, mathematician, logician, cryptanalyst, philosopher and theoretical biologist Alan Mathison Turing analyzes the concept of computability using the Turing machine concept, one of the foundations of theoretical computing. The engineer, statistician, professor, author, lecturer, and management consultant publish his works on the CUSUM, UWMA and EWMA control charts.

\textbf{+1935}\\
The physicist Hideki Yukawa present the strong interaction theory and predicted the existence of mesons. The same year, the astrophysicist and mathematician Subrahmanyan Chandrasekhar reports the results of his researches on the collapse of stars into white dwarfs and beyond 1.44 solar masses into neutron stars. The article by Albert Einstein, Boris Podolsky and Nathan Rosen on the EPR paradox is published in Physical Review and questioned the nonlocality of the Copenhagen interpretation.

\textbf{+1934}\\
The physicist Pavel Cherenkov Alekseyevich study the emission of light when relativistic particles pass through an amorphous medium. The same year the physicist Enrico Fermi suggested to bombard uranium atoms with neutrons to obtain an element with 93 protons, formulated the theory of beta decay and the physicist Leó Szilárd realizes that a nuclear chain reaction is possible. The physicists Irène Joliot-Curie and Frédéric Joliot bombard aluminum atoms with alpha particles and create artificially radioactive phosphorus-30.

\textbf{+1933}\\
The mathematician Andrei Nikolaevich Kolmogorov published a book containing a solid base of axioms of probability. The physicist Ernst August Friedrich Ruska realize the first electronic microscope by transmission using electrons instead than photons.

\textbf{+1932}\\
The physicist Carl David Anderson discovered the positron. The same year the physicist Werner Heisenberg present the theoretical model of the proton-neutron nuclear core and uses it to explain isotopes. The physicist James Chadwick discovered the neutron and the physicists John Cockcroft and Ernest Walton break the nuclear core of lithium and boron by proton bombardment.

\textbf{+1931}\\
The physicist Wolfgang Pauli puts forward the hypothesis of neutrino to explain the apparent violation of the principle of conservation of energy in beta decay. The same year the mathematician and logician Kurt Gödel showed that a system can be both consistent and complete (incompleteness theorem) and that if the system is coherent, then the coherence of the axioms can not be proved within the system. The physicist Ernest Lawrence invents the first cyclotron and the physicist, engineer and statistician Walter Andrew Shewhart publish his book \textit{Economic Control of Quality of Manufactured Product} where he introduces the main control charts.

\textbf{+1930}\\
The physicist Fritz London explains that Van der Waals forces are due to the interaction of the dipole moments of molecules. The same year the physicist Paul Dirac present his electron-hole theory and the economist John Maynard Keynes publish his A Treatise on Money.

\textbf{+1929}\\
The astronomer Edwin Hubble by studying the redshift hypothesizes that the Universe is not static. The same year, the physicist Robert Van de Graaff invents the first particle accelerator, known today as the "Van de Graaff accelerator".

\textbf{+1928}\\
The physicist Paul Dirac established his relativistic wave equation for the electron, which generalizes and improves the spinless relativistic equation of Klein-Gordon. The same year, the physicists Friedrich Hund and Robert S. Mulliken introduces the concept of molecular orbital and the physicist and cosmologist George Gamow develops the theoretical model of quantum alpha decay by tunneling. The physicist Félix Bloch studies the problem of a particle submitted to a periodic potential and develops the band theory that will become a main foundation of solid state theory, particularly to understand their transport or optical properties.

\textbf{+1927}\\
The physicist Werner Heisenberg establish the uncertainty principle, by which the position and momentum of a particle can not be simultaneously known with accuracy, indirectly by developing a new theoretical basis for quantum mechanics. The same year, the physicists Walter Heitler and Fritz London present the quantum theory of the chemical bond established from the hydrogen molecule and the physicist Max Born interprets the wave function of Schrödinger as probabilities and with the help of the physicist Robert Oppenheimer presents the Born-Oppenheimer approximation. The physicists Clinton Joseph Davisson, Lester Germer and George Paget Thomson confirm the wavelike nature of the electrons by diffraction. The physicist Paul Ehrenfest prove the famous quantum physics theorem that bears his name. The physicist Paul Dirac introduces the quantification of the electromagnetic field.

\textbf{+1926}\\
The physicist Erwin Schrödinger establish his wave equation that defines quantum mechanics in an analytical form by developing the ideas of De Broglie on theory of wave mechanics and he proves that the wave and matrix formulations of quantum theory are mathematically equivalent. The same year the physicists Oskar Klein and Walter Gordon establish the equation of relativistic quantum mechanics for spinless particles and Paul Dirac defines the Fermi-Dirac statistics. In the field of population dynamics the physicist and mathematician Vito Volterra published the nonlinear differential equation modeling the predator/prey equilibrium.

\textbf{+1925}\\
The physicist Pierre Auger discovered the Auger effect (2 years after Lise Meitner) and the same year the physicists George Uhlenbeck and Samuel Goudsmit postulate and reveal the existence of electron spin. Also the same year, the physicist Wolfgang Pauli established by necessity the principle of quantum exclusion. The physicists Werner Heisenberg, Max Born and Pascual Jordan formulate quantum matrix mechanics.

\textbf{+1924}\\
The physicist John Lennard-Jones proposed a semi-empirical description of inter-atomic interaction forces and the same year, the physicists Satyendranath Bose and Albert Einstein define the Bose-Einstein statistics. In the field of statistics, the statistician Ronald Fisher defines major modern concepts in statistics.

\textbf{+1923}\\
The astronomer Edwin Hubble estimates the distance between the Earth and the spiral galaxies, showing that they are far from the Milky Way and in the same year the physicist Louis de Broglie suggested the wave-particle duality from quantum theory and from the mass-energy equivalence and the physicist Lise Meitner discovers the Auger effect. The mathematician Norbert Wiener introduces Brownian movement theory.

\textbf{+1922}\\
The physicist Arthur Compton studied the scattering of X photons by electrons and the astrophysicist Alexander Friedmann develops non static universe models.

\textbf{+1921}\\
The physicist Alfred Landé defines the gyromagnetic ratio and introduces also half integer quantum numbers. The same year physicists Otto Stern and Walter Gerlach show experimentally that the intrinsic moment of the electron is quantized. The mathematician and physicist Theodor Franz Eduard Kaluza proved that a five-dimensional version of Einstein's equations unifies gravitation and electromagnetism.

\textbf{+1920}\\
The astronomer Vesto Melvin Slipher highlights the phenomenon of red shift in the spectrum of galaxies. The same year, the physicist Arnold Sommerfeld introduces a fourth quantum number to the original model of the Bohr's atom. the physicist Niels Bohr introduce the correspondance principle assuring the transition quantum physics $\mapsto$ classical physics when $\hbar\rightarrow 0$. The astronomer and astrophysicist Ernst Julius Öpik confirms that the "Andromeda nebula" is located outside the Milky Way-this confirms that our universe is much larger than we thought because it is not limited to the Milky Way.

\textbf{+1919}\\
The physicist Ernest Rutherford performed the first artificial disintegration of an atom by bombarding nitrogen with alpha particles. The same year, the physicist and mathematician Amali Emmy Noether develops his theorem on differential invariants in the calculus of variations, one of the most important mathematical theorems ever proved in guiding the development of modern physics.

\textbf{+1918}\\
The astronomer Harlow Shapley made the first accurate estimate of the size of our galaxy and the Sun's position in it. The same year, the physicist Hermann Weyl introduces the notion of gauge, the first step in what will become the gauge theory.

\textbf{+1917}\\
The physicist Albert Einstein introduces the idea of stimulated emission of radiation used in the manufacturing base of LASER. The same year, the physicist Arnold Sommerfeld introduces a third quantum number to the original model of the Bohr's atom.

\textbf{+1916}\\
The physicist Albert Einstein developed his theory of General Relativity and how matter plays on the space-time to produce gravitational effects. This is the first theory named "background independent". The same year, the physicists Gilbert Lewis and Irving Langmuir present the electronic shell model to explain chemical bonds and the physicist Arnold Sommerfeld introduces relativity in his model of 1915 and this relativistic correction explained the observed values by high resolution spectrographs and therefore the splitting of spectral lines named "fine structure" and he introduces at the same time a second quantum number describing elliptical orbits. The physicist Karl Schwarzschild found a mathematical solution of Einstein's equations, that he applies to neutron stars and black holes.

\textbf{+1915}\\
The physicist Arnold Sommerfeld refines the atomic model of the physicist Niels Bohr by introducing elliptical orbits to explain the fine structure lines of the hydrogen atom. This new model does not, however, explain the range of the observed spectra of the hydrogen atom. The same year the physicist Albert Einstein calculates the trajectory of Mercury with General Relativity. He finds that his theory to calculate the perihelion advance of Mercury with high accuracy and the bending of light rays in the gravitational field of the Sun. The end of that year he submitted the article that describes the field equations of gravitation, these equations will be the basis of the theory of General Relativity.

\textbf{+1914}\\
The physicist Ernest Rutherford showed that the positively charged atomic nuclei contain protons. The same year the physicist Albert Einstein and the mathematician Marcel Grossmann published an article on tensor calculus, and more particularly on the Riemann-Christoffel and Ricci tensor (more generally on tensor analysis and differential geometry) and the physicist Peter Debye develops a model of the behavior of the thermal capacity of the solids as a function of temperature. The mathematician Felix Hausdorff introduces the concepts of Hausdorff distance and Hausdorff dimension.

\textbf{+1913}\\
The physicist Niels Bohr present the quantum model in circular layers of the atom and the same year the physicist Robert Millikan measures the fundamental electric charge. The same year, the physicists William Henry Bragg and William Lawrence Bragg find the Bragg condition for strengths X-ray reflection and the physicist Henry Moseley showed that the atomic number is the true criterion of elements discrimination. In mathematics, the mathematician Elie Cartan announces his discovery of spinors. The pysicist Johannes Stark demonstrates that strong electric fields will split the Balmer spectral line series of hydrogen.

\textbf{+1912}\\
The physicist Max von Laue proposes the use of crystal lattices to diffract X-rays and in the same year the physicists Walter Friedrich and Paul Knipping diffract X-rays using zinc sulfide. The same year, the physicist Ernest Rutherford proposes the use of radioactivity as a means of dating. The chemists Otto Sackur and physicist Hugo Tetrode prove a formula to calculate the entropy of a mono-atomic gas (first apparition of Planck constant in thermodynamics).

\textbf{+1911}\\
The physicist Ernest Rutherford discovered the atomic nucleus by bombarding a thin gold foil with alpha particles. Some particles bounce on the core of gold atoms. The same year, the physicist and chemist Jean Perrin proves the existence of atoms and molecules and the physicist Heike Kammerlingh Onnes discovers superconductivity.

\textbf{+1911}\\
The physicist Robert Andrews Millikan determine the electric charge carried by a single electron with his famous oil-drop experiment in which he replaced water (which tended to evaporate too quickly) with oil.

\textbf{+1909}\\
The physicists Hans Geiger and Ernest Marsden discover that alpha particles can be strongly deflected by thin metal foils and in the same year the physicists Ernest Rutherford and Thomas Royds demonstrated that alpha particles are helium atoms ionized twice. In the field of Applied Mathematics, the mathematician Agner Krarup Erlang published the first paper on the theory of queues. 

\textbf{+1908}\\
The statistician William Sealy Gosset published an article proposing a new statistical distribution and a new statistical test named respectively the" Student law" and "Student's t-test". The same year, the mathematician Ernst Friedrich Ferdinand Zermelo proposed an improvement of the axioms of set theory. In his \textit{Mendelian proportions in a mixed population}, the mathematician Godfrey Harold Hardy exposes what is now known as the "Hardy-Weinberg principle" in genetics that establishes how dominant and recessive genetic traits spread in a large population. Hardy is known as Fundamental mathematician, specialist in number theory, but has contributed significantly to the study of population genetics by the results presented in this article, found independently by the obstetrician-gynecologist Wilhelm Weinberg.

\textbf{+1907}\\
The physicist Albert Einstein deduced the expression of the famous equivalence between mass and energy, and the same year he established the expression of the heat capacity of crystalline solids and calculates the gravitational redshift. The mathematician and physicist Hermann Minkowski unified space-time together in a unified mathematical structure. The mathematician Guido Fubini proves the multiple integral theorem that bears his name.

\textbf{+1906}\\
The physicist Walther Nernst presents a formulation of the third law of thermodynamics. The same year, the mathematician Andrei Markov published the first work on the chains of events that will later bare his name and that have occupied an important place in the quantum physics of his time. 

\textbf{+1905}\\
The physicist Albert Einstein explained the photoelectric effect by the existence of quantums and the same year he explained mathematically the Brownian motion as a result of the random motion of molecules and published his research on his theory of Special Relativity that proves the equivalence of mass and energy. The physicist Paul Langevin publish his theory on the susceptibility of paramagnetic materials.

\textbf{+1904}\\
The physicist Antoon Lorentz discovers the contraction of time in the direction of movement of the body relatively to the constant speed of light and proposed the transformation equations of the electromagnetic forces. The same year, the physicist Hantaro Nagaoka proposed a theoretical Saturnian model of the atom where electrons revolve around a massive positive nucleus like the rings of Saturn.

\textbf{+1903}\\
Radioactivity is explained in terms of fission of atoms by the physicist Ernest Rutherford and by the radiochemist Frederick Soddy.

\textbf{+1902}\\
The physicist Philipp Lenard observed that the photoelectric effect does not depend on the power of the light beam but its frequency. The same year, the chemist Theodor Svedberg suggests that fluctuations of molecular bombardment creates Brownian motion and the logician Bertrand Russell propose his "ultimate" paradox undermining the naive set theory. The physicist James Jeans describes the gravitational collapse phenomenon that can occur, for example in a cloud of gaseous material based on a critical mass or radius. The mathematician Henri Léon Lebesgue put the basis of measure theory and introduce the Lebesgue integral.

\textbf{+1901}\\
The mathematicians Gregorio Ricci-Curbastro and his assistant Tullio Levi-Civita developed tensor calculus.

\textbf{+1900}\\
The physicist Max Planck suggested that light can be emitted in discrete frequency generalizing the law of black body radiation. The same year, the physicist Johannes Rydberg refines the mathematical expression for the wavelengths of the Balmer's lines of the hydrogen and the physicist Paul Villard discovered gamma rays by studying the decay of uranium. In the field of Applied Mathematics, the mathematician Louis Bachelier developed the Brownian model motion applied to game and speculation theory that will be the mainstay of quantitative financial tools of the 20th century. The same year, the mathematician Karl Pearson defines the statistical distribution of the Chi-2 and explores important properties of this distribution for statistical inference. The physicist Paul Drude adaptated the kinetic theory of gases to electrons in metals and gets a model that still bears his name.

\textbf{+1899}\\
The physicist Ernest Rutherford discovered that the radiation emitted by uranium compounds are positively charged alpha particles and beta particles negatively charged. The same year, the mathematician David Hilbert replaces the five usual axioms of Euclidean geometry axioms by 21 to eliminate the weaknesses of Euclidean geometry. 

\textbf{+1898}\\
The mathematician David Hilbert gives a first approach of the class field. The same year, the physicists Marie and Pierre Curie isolate and study the radium and polonium and the physicist Wilhelm Wien Carl Werner identifies a new particle with a positive charge approximately equal to the mass of hydrogen that he will name the "proton". The engineer Alfred-Marie Liénard calculates the electromagnetic field produced by a point charge, animated with any movement. The mathematical expressions that have been independently established, but 2 years later by the physicist Emil Wiechert.

\textbf{+1897}\\
The physicist Joseph John Thomson measure the charge/mass ratio of certain negative particles created by cathodic rays. He measures their charge, and he concludes that their mass is about 2'000 times smaller than hydrogen. These particles were later named "electrons", a term suggested by the physicist George Johnstone Stoney. Televisions and other CRTs are improved versions of the Thomson's device.

\textbf{+1896}\\
The physicist Henri Becquerel discovered radioactivity of uranium and the same year the physicist Pieter Zeeman studied the decomposition of the sodium D lines when it is heated in a strong magnetic field and he discovered that the spectral lines of a light source subject to a magnetic field has many components, each having a certain polarization. To explain this phenomenon, one must add additional quantum number named "magnetic quantum number". The same year, the physicist Wilhelm Wien Carl Werner establishes the law that bears his name for the energy emitted by the black body.

\textbf{+1895}\\
The physicist Wilhelm Röntgen discovered X-rays and the same year the physicist and inventor Guglielmo Marconi carried out in the Swiss Alps at Salvan the first wireless "long distance" of 1.5 kilometers.

\textbf{+1893}\\
The mathematician, physicist, engineer and philosopher Henri Poincaré published his studies on the three-body problem and introduces the qualitative study of differential equations and chaos theory. The same year the mathematician Georg Cantor developed the theory of transfinite sets and the proposal of the engineer Nikola Tesla use AC instead of DC current is adopted by the first U.S. state. In his \textit{Uniplanar Algebra}, the mathematician Irving Stringham uses the symbol $\ln$ for the natural logarithm instead of the traditional $\log_e$.

\textbf{+1892}\\
The autodidact physicist Oliver Heaviside reduced the 8 equations of electrodynamics of Maxwell to $4$ differential equations. 

\textbf{+1891}\\
In his \textit{Arithmetices Principia, ova methodo exposita} the mathematician Giuseppe Peano introduces the axioms to build the Natural numbers set $\mathbb{N}$ and the symbol of appartenance and a first version of the quantifiers symbolic. Their final form will be given by the mathematician David Hilbert. He provides more than $40,000$ definitions in a language that he wants as universal.

\textbf{+1890}\\
The systematic study of groups is growing with the mathematician Sophus Lie, Issai Schur and Élie Cartan. This last one introduces the notion of algebraic group and continuous groups. 

\textbf{+1889}\\
The mathematician Giuseppe Peano postulates $5$ properties of integers as axioms in the idea to do with integers what Euclid did for geometry. He defines also the axiomatic of vectorial space in $\mathbb{R}$ and introduce the concept of linear application. He introduce also the notation $\cup$ and $\cap$ for the union and intersection of sets.

\textbf{+1888}\\
The anthropologist, explorer, geographer, inventor, meteorologist, proto-geneticist, psychometrician, and statistician ... Francis Galton defines the concept of statistical correlation coefficient. The same year, the mathematician Richard Dedekind proposes the definition of a finite set. 

\textbf{+1887}\\
The physicists Albert Michelson and Edward Morley measured the speed of light to test the hypothesis of ether that their experimental results reject and the same year the physicist Heinrich Hertz discovered the photoelectric effect and conducted experiments on electromagnetic waves (production and reception).

\textbf{+1886}\\
The mathematician and physicist Oliver Heaviside introduces the handling differential operators as algebraic entities which will bring up later to the Laplace transforms.

\textbf{+1885}\\
The chemist Johan Balmer found the mathematical expression which gives the wavelength of the different lines of the spectrum of hydrogen.

\textbf{+1884}\\
The physicist and chemist Willard Gibbs defines the notation still in use in the early 21st century for the scalar and vector product as well as vector differential operators in its books about to vector calculus. The same year, the physicist Ludwig Boltzmann derives the Stefan-Boltzmann black body radiant flux law from thermodynamic considerations and the physicist John Henry Poynting introduces the vector that still today bear his name. The mathematician Karl Hermann Amandus Schwarz proves that the sphere is the solid with the minimal surface for a given volume (this result explains many shapes visible in the Universe).

\textbf{+1882}\\
The mathematician Ferdinand von Lindemann proved the transcendence of $\pi$. The same year, the astronomer, mathematician, economist and statisticien Simon Newcomb observes a $43''$ per century excess precession of Mercury's orbit. Where Isaac Newton mechanics explains correctly the precession period of other planets, it failed to explains that of Mercury.

\textbf{+1880}\\
The mathematician and physicist Oliver Heaviside reduced the $8$ Maxwell equations to 4 equations and introduces the step function that always bear his name today (Heaviside step function).

\textbf{+1879}\\
The mathematician and physicist Joseph Stefan publishes Stefan's law which states that the power transmitted across the whole spectral range is proportional to the fourth power of the absolute temperature of a star and the surface of it. For the same surface temperature, a star is also brighter the more it is big. The same year, the physicist Edwin Herbert Hall discovered that an electric current through a material immersed in a magnetic field generates a voltage perpendicular to the initial direction of the electric current. The philosopher, logician, and mathematician Gottlob Frege publish his \textit{Begriffsschrift, eine der arithmetischen nachgebildete Formelsprache des reinen Denkens} where he introduces the axiomatic predicate logic, the quantifiers (for all $\forall$, it exists $\exists$), the theory of quantified variable, the rigorous concept of formula/function and of variable.

\textbf{+1878}\\
The mathematician and philosopher William Kingdon Clifford introduces the divergence operator.

\textbf{+1877}\\
The physicist and chemist Willard Gibbs defines for chemical reactions two useful quantities, namely the enthalpy that represents the heat of reaction at constant pressure and the free energy that determines whether a reaction can proceed as spontaneous at constant temperature and pressure. The physicist John William Strutt (3rd Baron Rayleigh)introduces the foundations of modern sound theory. 

\textbf{+1876}\\
The mathematician and philosopher William Kingdon Clifford suggests that the motion of matter may be due to changes in the geometry of space.

\textbf{+1874}\\
The physicist Lord Kelvin formally states the second law of thermodynamics. The same year, the mathematical economist Léon Walras published his Elements of Pure Economics.

\textbf{+1873}\\
The mathematician Georg Cantor laid the foundations of the theory of sets and cardinals and shows that the algebraic numbers are in fact countable and defines rigorously the real numbers $\mathbb{R}$, and introduce the his famous diagonal method. The same year, the physicist James Clerk Maxwell showed that light is an electromagnetic phenomenon and reduces the equations of electrodynamics to $8$ instead of $20$ equations (at the same times he defines the curl operator) and the physicist Johannes van der Waals introduces the idea that there are weak attractive forces between molecules. In his \textit{Sur la fonction exponentielle} the mathematician Charles Hermite proves the transcendence of $e$ (two proofs, one of $11$ pages and the other of $20$ pages).

\textbf{+1872}\\
The mathematician Karl Weierstrass presented at the Royal Academy of Sciences in Berlin an example of a function continuous everywhere but differentiable nowhere.

\textbf{+1871}\\
The chemist Dmitri Mendeleev examines his periodic table and predicted the existence of gallium, scandium and germanium. The same year the physicist James Clerk Maxwell established thermodynamic Maxwell relations. 

\textbf{+1870}\\
The physicist Rudolph Clausius proves the (scalar) Virial theorem.

\textbf{+1869}\\
The chemist Dmitri Mendeleev proposed the periodic table of elements which still bears his name.

\textbf{+1867}\\
The historian, journalist, philosopher, economist, sociologist Karl Marx published Das Kapital. 

\textbf{+1866}\\
The physicist James Clerk Maxwell elaborates, independently of the physicist Ludwig Boltzmann, the kinetic theory of gases of Maxwell-Boltzmann. The same year, the monk and botanist Gregor Johann Mendel formulated the laws of statistics hybridization (experiment on $29,000$ peas...) and the mathematician Leopold Kronecker used for the first time the symbol that bear always today his name.

\textbf{+1865}\\
The physicist James Clerk Maxwell publishes for the first time the equations of electrodynamics in the form of equations 20 with 20 unknowns using quaternions. 

\textbf{+1862}\\
The physicist Gustav Kirchhoff developed the concept of Black body that can absorb and emit radiation at all frequencies and that the energy emitted depends only on the frequency of the emitted radiation and the temperature of the black body itself.

\textbf{+1859}\\
The physicist James Clerk Maxwell discovered the law of distribution of molecular velocities. The same year the astronomer Urbain Le Verrier reported an anomaly in the motion of Mercury not predictable by Newton's law and the physicist Gustav Kirchhoff with the chemist Robert Wilhelm Bunsen developed prism spectroscopy. 

\textbf{+1858}\\
The lawyer and mathematician Arthur Cayley emerges the notion of vector space, the notion of matrix and exposes the utility by using the multiplication of matrices and determinants; he rewrites the system of linear equations in matrix form. His works are often seen as the emergence of linear algebra.

\textbf{+1855}\\
The astronomer and physicist Léon Foucault discovered that the force required for the rotation of a copper disc increases when must rotate with its rim between the poles of a magnet, the disk heating at the same time because of the "Foucault's currents" induced in the metal. 

\textbf{+1854}\\
The mathematician George Boole published his system of symbolic logic, now known as Boolean algebra. The same year, the mathematician Arthur Cayley shows that quaternions can be used to represent rotations in four-dimensional space and the mathematician Georg Friedrich Bernhard Riemann gave a new definition of the integral and lays the foundations of differential geometry.  The mathematician Charles Hermite defines the concept of orthogonal matrices and prove that their eigenvalues are real numbers.

\textbf{+1852}\\
The physicists James Joule and William Thomson Kelvin show that gas in expansion cools quickly.

\textbf{+1851}\\
The astronomer and physicist Léon Foucault made a spectacular proof of the rotation of the Earth by suspending a pendulum with a long cable attached to the dome of the Pantheon in Paris. The same year, the mathematician Georg Friedrich Bernhard Riemann published the first work on functions with a complex variable. In his \textit{Paradoxien des Unendlichen}, the  mathematician, logician, philosopher and theologian Bernardus Placidus Johann Nepomuk Bolzano discusses of the issues related to the manipulation of infinites in mathematics. 

\textbf{+1850}\\
The mathematicians Arthur Cayley and James Joseph Sylvester introduces the term matrix and the same year, the mathematician Richard Dedekind introduces the term field. The physicist Rudolf Clausius developed the mechanical theory of heat and formulated the second principle of thermodynamics. The physicist George Stokes proves the famous theorem that bears his name.

\textbf{+1849}\\
The mathematicien and astronome Edouard Roche finds the limiting radius of tidal destruction and tidal creation for a body held together only by its self gravity and uses it to explain why Saturn's rings do not condense into a satellite.

\textbf{+1848}\\
The physicist William Thomson Kelvin discovers the absolute 0 point temperature and sets its own unit of measurement. The same year the physicist and astronomer Hippolyte Fizeau transposes the results of Christian Doppler to the light that like sound has a wave nature (Doppler effect) and highlights the redshift and towards the blueshift.

\textbf{+1847}\\
The physicist Jame Joule founds experimentally the mechanical equivalent of heat and the same year, the physiologist and physicist Hermann Helmholtz formally states the law of conservation of energy.

\textbf{+1845}\\
The physicist Gustav Kirchoff defines the concept of electric potential and sets the laws of networks that bear his name (node law, mesh law). The same year, the physicist George Stokes publishes what will be the basis of the Navier-Stokes fluid mechanics and the physicist Michael Faraday discovers that light propagation in a material can be influenced by external magnetic fields.

\textbf{+1844}\\
The mathematician Joseph Liouville proves the existence of an infinite number of transcendental numbers.

\textbf{+1843}\\
The physicist, mathematician and astronomer William Rowan Hamilton defines sets of complex vector spaces (quaternions). The concept of vector space will be clearly defined by the mathematician and astronomer August Ferdinand Möbius and the mathematician and linguist Giuseppe Peano $40$ years later. The same year, hhe mathematician Laurent Pierre Alphonse publishes his memoir on what will later became the series that bear his name in complex analysis.

\textbf{+1842}\\
The principle of conservation of energy is expressed by the physicist Julius von Mayer who calculated the amount of work that can be obtained by converting heat energy, this means the mechanical equivalent of calories. The same year, the physicist Christian Doppler discovered the acoustic effect that bears his name (change in frequency with the relative movement).

\textbf{+1841}\\
The mathematician Karl Weierstrass discovers but does not publish the Laurent expansion theorem. The same year, the mathematician Carl Gustav Jacob Jacobi introduces the Jacobian matrices and reintroduced the partial derivative notation originally proposed by the mathematician André-Marie Legendre.

\textbf{+1838}\\
The astronomer and mathematician Friedrich Bessel measure that the distance that separates us from the star 61 Cygni is about 96 trillion kilometers.

\textbf{+1835}\\
The mathematician Carl Friedrich Gauss gives a rigorous construction of the complex numbers and the mathematician Augustin Louis Cauchy establishes a first theory of determinants. The same year, the mathematician and engineer Gaspard Coriolis proves that the acceleration of a mobile in a rotating frame is subjected to a complementary force perpendicular to the direction of movement of the mobile in this reference frame.

\textbf{+1834}\\
The engineer and physicist Émile Clapeyron presents a formulation of the second law of thermodynamics. The same year, the physicist Heinrich Lenz establishes the law of electromagnetic induction.

\textbf{+1832}\\
The physicist Michael Faraday established the basic theory of electrolysis.

\textbf{+1831}\\
The physicist Michael Faraday discovered electromagnetic induction, namely the obtention of an electric current from the change of a magnetic field (principle of the dynamo). The same year the mathematician and physicist Carl Friedrich Gauss provides two of the four Maxwell equations.

\textbf{+1829}\\
The mathematician Evariste Galois presents the first draft of his work on solvable equations which will cause the set-approach for solving algebraic equations by radicals. The mathematician Augustin Louis Cauchy proves that the eigenvalues of a symmetric matrices are all real. The mathematician Johann Peter Gustav Lejeune Dirichlet studies the convergence of Fourier series.

\textbf{+1828}\\
The physicist George Green proves Green's theorem.

\textbf{+1827}\\
The botanist Robert Brown discovered the Brownian motion of pollen particles and dye in water; the same year the physicist Georg Ohm establishes the law of electrical resistance and the physicist and mathematician André Ampère discovered the laws that bind the magnetic forces to electric current. The physicist, mathematician and astronomer William Rowan Hamilton presents the theory of a single function that unifies mechanics, optics and mathematics and helped to establish the wave theory of light.

\textbf{+1826}\\
The mathematician Niels Henrik Abel proves that it is impossible to solve general quintic equation ($5$th order polynomials) in radicals. In his \textit{On a Method of Expressing by Signs the Action of Machinery}, the mathematician, philosopher, inventor and mechanical engineer Charles Babbage describes of a symbolic language that will help him to design its analytical machine, the first universal mechanical machine. The "Babbage machine" is the first complete programmable computer (having the same computational capabilities as a Turing machine or a modern computer) that has been designed. 

\textbf{+1825}\\
The mathematician Augustin-Louis Cauchy presents the Cauchy integral theorem for general integration paths and introduces the theory of residues. The scientist and inventor William Sturgeon invented the first electromagnet.

\textbf{+1824}\\
The mathematician Augustin-Louis Cauchy discovers the characteristic polynomial of a matrix and proves that it is invariant by linear transformation and calculates for the first time eigenvalues and eigenvectors.

\textbf{+1824}\\
The physicist and engineer Sadi Carnot scientifically analyzes the efficiency of steam engines (Carnot cycle), showing that their performance is limited and also defines the second principle of thermodynamics.

\textbf{+1823}\\
The physicist and chemist Michael Faraday presents of a series of papers on the liquefaction of gases.  The mathematician Pierre Frédéric Sarrus introduces the vertical bar symbol for integral: $\int_a^bf(x)\mathrm{d}x=F(x)|_a^b$.

\textbf{+1822}\\
The mathematician Jean-Victor Poncelet found the projective geometry. The same year, the physicist and mathematician Joseph Fourier formally presents the use of dimensions (units) for physical quantities and introduce Fourier series and also the notation $\int_a^bf(x)\mathrm{d}x$.

\textbf{+1821}\\
The principle of the dynamo is described for the first time by the physicist and chemist Michael Faraday. The same year the physicist John Herapath proposes that heat is in reality the effect of agitation and therefore the movement of elementary bodies.

\textbf{+1820}\\
The physicist Hans Oersted discovers and proves the magnetic effects of electric current. The same year, physicists Jean-Baptiste Biot and Félix Savart determine in the field of magnetism the famous law that bears their name.

\textbf{+1819}\\
The physicist and chemist Hans Christian Örsted shows that electric current deflected a magnetized needle, thus demonstrating electromagnetism and announcing an industrial revolution.

\textbf{+1818}\\
The mathematician, geometer and physicist Simeon Poisson calculates the Poisson bright spot at the center of the shadow of a circular opaque obstacle.

\textbf{+1817}\\
By studying the polarization of light, the physicist Augustin Fresnel shows that it is a transversal wave motion and not longitudinal and also shows that the diffraction and interference can be explained if we consider light as a wave. The same year, the astronomer Friedrich Bessel publishes the works making use of the famous functions that bear his name.

\textbf{+1816}\\
The mathematician Joseph Diaz Gergonne introduces the symbol marking the inclusion in the set theory.

\textbf{+1814}\\
The physicist and optician Joseph von Fraunhofer studied for the first time the absorption lines of the solar spectrum and this using the spectroscope which he was the inventor. The mathematician, astronomer and physicist Pierre-Simon Laplace makes the assumption that a perfect knowledge of the present state of the universe would enable one to determine perfectly all its future states.

\textbf{+1812}\\
The mathematician, astronomer and physicist Pierre-Simon Laplace published a major work on probability theory (including also the method of least squares) for which he is considered as one of the founders.

\textbf{+1811}\\
The chemist Amaedo Avogadro hypothesizes that equal volumes of different gases contain the same number of molecules under the same conditions of temperature and pressure.

\textbf{+1810}\\
The mathematician, astronomer and physicist Carl Friedrich Gauss discovered the basic concepts of non-Euclidean geometry but never published his work on the subject. The same year, the physicist and mathematician Joseph Fourier models the evolution of the temperature with trigonometric series.

\textbf{+1809}\\
The mathematician, astronomer and physicist Carl Friedrich Gauss developed the method of least squares independently of Legendre. The same year, the mathematician, astronomer and physicist Pierre-Simon Laplace proved the general form of the central limit theorem. The engineer, physicist and mathematician Etienne Malus publishes the law of Malus.

\textbf{+1808}\\
The physicist and chemist John Dalton proposes what is considered as the first theory of the atom. The mathematician Christian Kramp introduces in his \textit{Éléments d'arithmétique universelle} the notation $n!$ for the factorial.

\textbf{+1806}\\
The mathematician Jean Robert Argand published the first plane representation of complex numbers and algebraic measures are used. The mathematician Johann Carl Friedrich Gauss develops the idea of adding vectors in a geometric form and introduces the notation $\ overrightarrow{ab}$ for a vector.

\textbf{+1805}\\
The mathematician André-Marie Legendre developed the method of least squares.

\textbf{+1803}\\
The physicist and chemist John Dalton has the original idea to assume that each chemical element is composed of different atoms. A chemical combination was then explained by the union of these atoms in fixed proportions and relative atomic masses became calculable from experimental facts. The same year, the economist, journalist and industrialist Jean-Baptiste Say published his Treatise on Political Economy.

\textbf{+1802}\\
The physicist Thomas Young showed the wave nature of light by an important experience that shows the interference of waves. The same year, the chemist and physicist Joseph Louis Gay-Lussac discovered the famous law connecting gas volume and temperature of a real gas, the law that bears his name (Gay-Lussac's law).

\textbf{+1801}\\
The chemist and physicist John Dalton discovered the law of the sum of the partial pressures which still bears his name.

\textbf{+1800}\\
The chemist William Nicholson and the surgeon Anthony Carlisle used electrolysis to separate water into hydrogen and oxygen. The same year, the astronomer William Herschel discovers infrared radiation and the physicist Alessandro Volta and invented the first electric battery.

\textbf{+1799}\\
The mathematician Gaspard Monge published his book of descriptive geometry. He is considered as the inventor of this field. First satisfactory but incomplete proofs of the fundamental theorem of algebra by the mathematician Johann Carl Friedrich Gauss.

\textbf{+1798}\\
An Essay on the Principle of Population by the cleric and scholar Thomas Robert Malthus (one of the founding books of the study of population dynamics) proposes one of the first models of population growth, the exponential model, which will be the basis for future work.

\textbf{+1798}\\
The mathematician Carl Friedrich Gauss gives a rigorous proof of the theorem of d'Alembert (fundamental theorem of algebra). The same year, the physicist Benjamin Thompson had the idea that heat is a form of energy and the physicist and chemist Henry Cavendish measures the gravitational constant. The economist Thomas Malthus stated his law of population.

\textbf{+1797}\\
The mathematician Caspar Wessel associates vectors with complex numbers and studies complex number operations in geometrical terms.

\textbf{+1793}\\
The National Assembly of the French Republic established the metric system.

\textbf{+1789}\\
The physicists and chemists Antoine Lavoisier states the law of conservation of mass.

\textbf{+1788}\\
In his \textit{Méchanique Analitique}, the mathematician and Joseph-Louis Lagrange formulates a new way to study classical mechanics (by Isaac Newton for recall) using the principle of least action. The same year, the Academy of Science approves the creation of a universal measurement system, the future metric system. This project will also be approved by the French National Assembly in 1790, which will give the first definition of the meter.

\textbf{+1787}\\
The physicist, chemist and inventor Jacques Alexandre César Charles experimentally determined that the volume of a fixed mass of gas at constant pressure is proportional to temperature.

\textbf{+1786}\\
The astronomer William Herschel made a detailed description of our galaxy.

\textbf{+1785}\\
The physicist Charles Augustin Coulomb proves that the forces between electric charges and between magnets applied at the inverse square of the distance.

\textbf{+1783}\\
The clergyman and natural philosopher John Michell in a paper for the Philosophical Transactions of the Royal Society of London, read on 27 November 1783, first proposed the idea that there were such things as black holes, which he called "dark stars". A few years after Michell came up with the concept of black holes, the French mathematician Pierre-Simon Laplace suggested essentially the same idea in his 1796 book, \textit{Exposition du Système du Monde}. 

\textbf{+1782}\\
The mathematician, physicist and astronomer Pierre-Simon de Laplace introduces the "Laplace transform", a transformation that solves several differential equations in physics.

\textbf{+1781}\\
The chemist and physicist Joseph Priestley creates water by combustion of hydrogen and oxygen which shows that water is not an essential element as we thought since Aristotle.

\textbf{+1778}\\
The physicists and chemists Carl Scheele and Antoine Lavoisier discovered that air is composed mainly of nitrogen and oxygen.

\textbf{+1777}\\
The physicist and mathematician Leonhard Euler re-introduces the letter $\mathrm{i}$ for the imaginary part of the complex numbers.

\textbf{+1776}\\
The moral philosopher and a pioneer of political economy Adam Smith publishes his An Inquiry into the Nature and Causes of the Wealth of Nations.

\textbf{+1774}\\
The mathematician, astronomer and physicist Pierre-Simon Laplace explicitls Euler's integral. The same year, the theologian, dissenting clergyman, natural philosopher, educator and political theorist Joseph Priestley made his major discovery, that of oxygen.

\textbf{+1772}\\
The mathematician, engineer and astronomer Joseph-Louis Lagrange studied the three-body problem and discovered the libration points today named "Lagrange points".

\textbf{+1770}\\
In his \textit{Mémoire sur les éequations aux différences partielles}, the philosopher, mathematician, and early political scientis Marie Jean Antoine Nicolas de Caritat, Marquis of Condorcet introduces the symbol of partial derivatives $\partial$.

\textbf{+1769}\\
The physicist and chemist Henry Cavendish discovered and studied hydrogen. The same year, in his \textit{Institutiones calculi integralis}, the mathematician Leonhard Euler studies for the first time the double integrals, calculates them by successive integration and changes of variable. These methods will be generalized to the triple integrals by the mathematician, engineer and astronomer Joseph-Louis Lagrange, which also gives the general formula for the change of variables (determinant of the Jacobian).

\textbf{+1766}\\
The physicist and chemist Henry Cavendish discovered and studied hydrogen.

\textbf{+1764}\\
The latent and specific heat are described for the first time by the physicist and chemist Joseph Black. It is also the first to clearly distinguish temperature and momentum. The same year, the physicist and mathematician Leonhard Euler examines the partial differential equation for the vibration of a circular drum and finds one of the Bessel function solutions.

\textbf{+1763}\\
A posthumous article of the mathematician and clergyman Thomas Bayes reveals that he discovered what is named today the "Bayes' theorem".

\textbf{+1757}
The physicist and mathematician Leonhard Euler founds modern hydrodynamics.

\textbf{+1756}\\
The mathematician, engineer and astronomer Joseph-Louis Lagrange develops analytical mechanics based on his invention of the calculus of variations independently of Leonhard Euler.

\textbf{+1755}\\
The physicist and mathematician Leonhard Euler introduces the uppercase Greek letter sigma ($\sum$) for the symbol of the sum.

\textbf{+1753}\\
In his 1749 study of the motions of the earth Leonhard Euler obtained differential equations for the orbital elements and in 1753 he applied the method of variation of constants to his study of the motions of the moon.

\textbf{+1750}\\
The mathematician Gabriel Cramer define the Cramer's rule for solving linear systems.

\textbf{+1749}\\
The astronomer and physicist Jean le Rond D'Alembert developed the first model of precession based on the theory of gravitation of Newton and gives a possible solution to the problem of three bodies.

\textbf{+1748}\\
In his \textit{Introductio in Analysin Infinitorum} the mathematician Leonhard Euler introduces the concept of function (defined as any composition of algebraic and analytic expression), defines the concepts of even and odd functions, popularize the use of the symbols $e$ and $\pi$, proves Euler's identity and defines the $\Gamma$ function that generalized the factorial.

\textbf{+1746}\\
The encyclopedist Jean le Rond D'Alembert gives the first evidence (acceptable but will be corrected later) of the fundamental theorem of algebra. The following year (1747) he published the equation of vibrating strings, which was the first example of the wave equation. This makes D'Alembert, one of the founders of mathematical physics.

\textbf{+1744}\\
The philosopher, mathematician, physicist, astronomer and naturalist Pierre Louis Moreau de Maupertuis states the principle of least action which will be formalized mathematically 22 years later by the mathematician, engineer and astronomer Joseph-Louis Lagrange. The same year, the physicist and mathematician Leonhard Euler shows the existence of transcendental numbers and introduces variations calculus.

\textbf{+1742}\\
The astronomer Anders Celsius defines its own unit of measurement for temperature.

\textbf{+1739}\\
The physicist and mathematician Leonhard Euler solves the general homogeneous linear ordinary differential equation with constant coefficients.

\textbf{+1738}\\
The physician, physicist and mathematician Daniel Bernoulli published a book on hydrodynamics introducing the kinetic theory of gases and the famous Bernoulli theorem (pressure balance). The same year in his \textit{Doctrine of chance}, the mathematician Abraham De Moivre introduces the Gaussian distribution as a means of approximating the binomial law for large number of experiments and demonstrates a partial version of the central limit theorem.

\textbf{+1737}\\
The physicist and mathematician Leonhard Euler solves the problem of graph theory on the bridges of Königsberg. The resolution of this problem is considered as the first theorem of graph theory. He establishes the same time the "Euler's formula" linking the number of vertices, edges and faces of a convex polyhedron, and hence of a planar graph.

\textbf{+1736}\\
The inventor Jonathan Hulls puts the first patent for a boat propelled by a steam engine.

\textbf{+1734}\\
The physicist and mathematician Leonhard Euler introduces the notation $f(x)$ for a function applied to the argument $x$.

\textbf{+1733}\\
The mathematicien Geralamo Saccheri studies what geometry would be like if Euclid's fifth postulate were false.

\textbf{+1729}\\
The dyer and amateur of physics and astronomy Stephen Gray was the first to discover the transmission of electricity in materials that he named "conductors".

\textbf{+1727}\\
The physicist and mathematician Leonhard Euler introduces the modern notation for the trigonometric functions and the letter $e$ for the base of the natural logarithm (occasionally also known as the "Euler number").

\textbf{+1724}\\
The mathematician Abraham De Moivre studies mortality statistics and the foundation of the theory of annuities in Annuities on lives.

\textbf{+1715}\\
The mathematician Brook Taylor publishes the tools that gives the possibility to make integration by parts and series expansions of functions (the famous Taylor series).

\textbf{+1714}\\
The mathematician Brook Taylor derives the fundamental frequency of a stretched vibrating string in terms of its tension and mass per unit length by solving an ordinary differential equation.

\textbf{+1713}\\
The mathematician and physicist Jacques Bernoulli publishes the rigorous principles of basic probabilities and statistics.

\textbf{+1705}\\
The astronomer Edmund (or Edmond) Halley predicted with an almost negligible calculation error with that the comet passed near the Earth in 1682 will return in 1758.

\textbf{+1704}\\
The physicist and mathematician Isaac Newton found experimentally that white light is composed of many colors. It also assumes that a light ray is composed of particles.

\textbf{+1701}\\
In his \textit{Explication de l'Arithmétique Binaire} Gottfried Wilhelm Leibniz introduces binary arithmetic (Leibniz may have been the first computer scientist and information theorist). He anticipated Lagrangian interpolation and algorithmic information theory. His \textit{Calculus ratiocinator} anticipated aspects of the universal Turing machine. In 1961, the mathematician and philosopher Norbert Wiener suggested that Leibniz should be considered the patron saint of cybernetics.

\textbf{+1698}\\
The mathematician and physicist Jacques Bernoulli clearly poses the problem of the brachistochrone curve (which belongs to the family of cycloid curves) and proposes a solution. The same year, the mathematician Guillaume de L'Hopital states his rule for the examination of indeterminate forms. Gottfried Leibniz proposes the use of the point $\cdot$ to denote multiplication, instead of the cross $\times$, which is too easily confused with the variable $x$ in the equations.

\textbf{+1696}\\
The mathematician and physicist Jacques Bernoulli clearly poses the problem of the brachistochrone curve (which belongs to the family of cycloid curves) and proposes a solution. The same year, the mathematician Guillaume de L'Hopital states his rule for the examination of indeterminate forms.

\textbf{+1693}\\
The astronomer and engineer Edmund Halley discovered the relation between the focal length of a lens with the distance of the image to its axis and the real object to its axis. The same year, he prepares the first mortality tables statistically relating death rate to age.

\textbf{+1691}\\
The philosopher and mathematician Gottfried Leibniz discovers the technique of separation of variables for ordinary differential equations.

\textbf{+1690}\\
The wave theory of light is put forward by the physicist and astronomer Christiaan Huygens. The same year the physicist and mathematician Johann Bernoulli developed the exponential calculus and find the equation of the catenary. The same year, the mathematician and physicist Jacques Bernoulli (brother of Jean Bernoulli) develops integral calculus.

\textbf{+1687}\\
The physicist and mathematician Isaac Newton published \textit{Naturalis Principia Mathematica} in which he explains the force of gravity and planetary orbits. He also describes the three laws of dynamics and derivates Kepler laws for his gravitational law. This is the first scientific revolution (before Special/General Relativity and quantum physics).

\textbf{+1685}\\
The philosopher and mathematician Gottfried Leibniz solves linear systems using without theoretical justification matrices and determinants.

\textbf{+1682}\\
The physicist and mathematician Isaac Newton establishes the law of gravitation, which now bears his name.

\textbf{+1679}\\
The philosopher and mathematician Gottfried Leibniz introduces binary arithmetic and develops a calculating machine that performs 4 operations. The same year, the physicist, mathematician and inventor Denis Papin shows experimentally the influence of atmospheric pressure on the boiling point of water.

\textbf{+1678}\\
The mathematicien, astronomer and physicist Christiaan Huygens states his principle of wavefront sources.

\textbf{+1676}\\
The physicist Robert Hooke states that stretching a spring is proportional to the voltage.

\textbf{+1675}\\
The astronomer Olaus Roemer makes accurate measurements of the speed of light. The same year, tThe apothecary chemist Nicolas Lemery writes \textit{Cours de chymie} which is considered as the first great chemistry treatise where the mixtures are defined, the first theory of bases and acids, and so on are introduced.

\textbf{+1673}\\
The philosopher and mathematician Gottfried Leibniz invents his differential calculus, introduces the symbol $\int$ and use the term "convergence test" for alternated series and use the definition of the convergence for the first time. Leibniz also use the notation $\mathrm{d}y/\mathrm{d}x$ and $\mathrm{d}x$, $\mathrm{d}x$.

\textbf{+1671}\\
First attempt to calculate life annuities (comparable to life insurance) by the politician Johan de Witt in collaboration with the mathematician Christian Huygens and first calculations of life expectancy.

\textbf{+1670}\\
The mathematician John Wallis introduces the symbols $\le$ and $\ge$.

\textbf{+1669}\\
The mathematician, astronomer and physicist Christian Huygens published results on the observation of the conservation of kinetic energy becoming verbatim the discoverer of the concept of kinetic energy. The same year, in his manuscript \textit{De analysi per aequationes numero terminorum Infinitas}, the physicist and astronomer Isaac Newton gives the first description of the Newton's method which makes it possible to find approximations of function roots by iterated process. The description of Newton only applies to polynomials and does not use the notion of derivative (the manuscript will be published in 1711).

\textbf{+1668}\\
The physicist and astronomer Isaac Newton made the first reflecting telescope and the same year the mathematician John Wallis suggests the law of conservation of momentum.

\textbf{+1667}\\
In his \textit{Vera Circuli and Hyperbolae Quadratura} the mathematician and astronomer James Gregory gives the first proof of the fundamental theorem of calculus and independently discovers Taylor series. Draft of the concept of transcendental number elaborated in relation to the problem of the quadrature of the circle.

\textbf{+1665}\\
The physicist Isaac Newton formulated the three laws of mechanics. He lays the foundations of differential calculus, these techniques allows him starting from the expression of a force inverse of square of the distance to find the general form of Kepler's laws.

\textbf{+1664}\\
The physicist Isaac Newton begins to work on differential and integral calculus.

\textbf{+1661}\\
The founder of statistical demography John Graunt published the first mortality table, the same year the physicist and chemist Robert Boyle determines the laws of compressibility of gas bearing his name and sometimes attached to that of the physicist Edme Mariotte who rediscovered a few years after the same laws.

\textbf{+1659}\\
The mathematician, astronomer and physicist Christian Huygens discovered the rigorous isochronism formula (when the end of the pendulum travels an arc of cycloid, the period of oscillation is constant regardless of the amplitude).

\textbf{+1658}\\
The mathematician, un astronome and physicist Christian Huygens experimentally discovers that balls placed anywhere inside an inverted cycloid reach the lowest point of the cycloid in the same time and thereby experimentally shows that the cycloid is the isochrone.

\textbf{+1657}\\
The lawyer and mathematician Pierre de Fermat states his "Fermat's principle" in optics as how the light propagates from one point to another on trajectories such that the duration of the propagation is locally minimal.

\textbf{+1655}\\
The mathematician, astronomer and physicist Christian Huygens was the first to use the concept of expected mean in probabilities. The mathematician John Wallis introduces the symbol $\infty$ in his \textit{Mathesis Universalis}.

\textbf{+1654}\\
The mathematician, physicist, inventor, philosopher, moralist and theologian Blaise Pascal and the lawyer and mathematician Pierre de Fermat create the theory of probability.

\textbf{+1644}\\
The physicist and mathematician Evangelista Torricelli has the idea of substituting water by mercury in the so named Torricelli's experiment to highlight the "grosso-vido"; later will the works of the mathematician, physicist, inventor, philosopher, moralist and theologian Pascal Blaise follow (experience Puy de Dôme experiment).

\textbf{+1638}\\
The mathematician, geometer, astronomer and physicist Galileo Galilei publishes the mathematical relationship that defines the period of the simple pendulum.

\textbf{+1637}\\
The philosopher and mathematician René Descartes renames the unknowns $x$, $y$, $z$ and the parameters $a$, $b$, $c$ and extends the use of algebra to the lengths and plane, creating analytical geometry with Pierre de Fermat. The same year, always René Descartes, quantitatively derives the angles at which primary and secondary rainbows are seen with respect to the angle of the Sun's elevation.

\textbf{+1631}\\
The mathematician Thomas Harriot introduces, in a posthumous publication, the symbols $>$ and $<$. The same year the theologian and mathematician William Oughtred provides for the first time the multiplication symbol $\times$ and the symbol $\pm$.

\textbf{+1629}\\
The lawyer and mathematician Pierre de Fermat develops a rudimentary differential calculus. The same year, in his Invention nouvelle en algèbre Albert Girard states, without proof, for the first time the fundamental theorem of algebra (a polynomial of degree $n$ has $n$ complex distinct or non-distinct roots) using complex numbers.

\textbf{+1626}\\
Tables of sine, tangent and secant of the engineer Albert Girard with the use of abbreviations sin, cos and tan.

\textbf{+1624}\\
Invention of the first thermometer (whose graduations are obviously not standardized...) by the physician Santorio Santorio.

\textbf{+1621}\\
The astronomer and physicist Willebrord Snell discovered that the angle of refraction of light is determined by the sine of the angle of the incident light with the normal of the dioptre.

\textbf{+1620}\\
The engineer Francis Thomas Bacon defends the experimental method and leads numerous observations on the heat. He suggested that the heat is related to the movement.

\textbf{+1619}\\
Johannes Kepler finished to publish the three laws of planetary motion.

\textbf{+1614}\\
The mathematician John Napier (John Napier in French) invented logarithms, which bring the operations of multiplication and division to simple additions or subtractions.

\textbf{+1611}\\
Johannes Kepler discovers total internal reflection, a small angle refraction law and thin lens optics.

\textbf{+1610}\\ In his \textit{Sidereus Nuncius}, Galileo Galileo reports the first observations at the telescope: the discovery of the satellites of Jupiter, the confirmation that the milky way is constituted of stars, the discovery of the rings of Saturn. These
Observations will have an important effect because they contradict some of the ideas of the Universe models of the time and the Bible writings.

\textbf{+1609}\\
In his \textit{Astronomia Nova}, the astronomer and mathematician Johannes Kepler explains Kepler's first two laws for the motion of the planets.

\textbf{+1608}\\
The optician Hans Lippershey invented the telescope to will be used and improved (with a random quality) the following year by the mathematician, geometer, astronomer and physicist Galileo Galilei to confirm the theories of Copernicus.

\textbf{+1604}\\
In a letter to Paolo Sarpi, Galileo states the law of the fall of bodies: the distance traveled is proportional to the square of the time of fall.

\textbf{+1603}\\
The mathematician and astronomer Thomas Harriot determines how to calculate qualitatively the surface of a spherical triangle.

\textbf{+1591}\\
The mathematician François Vieta opens a new period int algebra by making calculations with letters, using vowels for unknowns and consonants for parameters. Moreover, he gives the development of the Newton binomial theorem.

\textbf{+1590}\\
The astronomer Galileo Galilei demonstrated experimentally that all falling bodies have the same acceleration. The same year, the opticians Hans and Zacheraius Janssen created the first microscope by combining several lenses that define the beginnings of scientific medicine and biology.

\textbf{+1586}\\
The engineer and physicist Simon Stevin proved the method of the parallelogram of forces and discovered that the pressure of a liquid on the bottom of a container is independent of its shape, and also of the bottom surface and depends only on the height water in the container. He also gave the pressure measurement on any portion of the side of a container.

\textbf{+1576}\\
The astronomer Tycho Brahe observed a new star in the constellation of Cassiopeia and built an observatory on the island of Hveen.

\textbf{+1572}\\
The mathematician Rafaelle Bombelli gives a formulation of complex numbers and the rules of actual calculations. He introduces the terms più di meno (pdm) and meno di meno (mdm) to represent $+\mathrm{i}$ and $-\mathrm{i}$.

\textbf{+1548}\\
The mathematician and physicist Simon Stevin wrote the tenth powers identified with an exponent. He also gives the first writing of vectors. 

\textbf{+1545}\\
The mathematician Ludovico Ferrari gives the solution of equations of degree $4$ (know under "Cardan formula").

\textbf{+1543}\\
The work of the astronomer Nicolas Copernicus summarizing 26 years of research and observations is published and clearly highlights that the heliocentric system of Ptolemy is not valid.

\textbf{+1536}\\
The mathematician Niccolò Fontana launched the new science of ballistics.

\textbf{+1530}\\
The mathematician and physicist Robert Recorde introduces the "$=$" sign and the mathematician Michael Stifel developed an early form of algebraic notation.

\textbf{+1525}\\
The mathematician Christoff Rudolff introduces the notation for square roots $\sqrt{\phantom{a}}$.

\textbf{+1515}\\
Publication of Scipione del Ferro's works where he find a formula giving the general solution of the polynomial equations of degree $3$. These solutions involve the implicit manipulation of imaginary numbers

\textbf{+1510}\\
The painter, engraver and mathematician Albrecht Dürer develops the basics of descriptive geometry and perspective.

\textbf{+1500}\\
The Italian mathematician Scipione del Ferro succeeds for the first time to solve a large algebraic type of cubic equations.

\textbf{+1490}\\
The painter, sculptor, architect, musician, mathematician, engineer, inventor, anatomist, geologist, cartographer, botanist, and writer... Leonardo da Vinci describes capillary action.

\textbf{+1489}\\
\textit{Behende und hupsche Rechnung auf allen kauffmanschafft} of Johannes Widmannwe the symbols "$+$" and "$-$" are introduced for the first time (before plus was denoted "P" and minus "M").

\textbf{+1464}\\
\textit{Triparty in the science of numbers} by  Nicolas Chuquet we we can find the first use of negative powers and zero power - states the property of the exponents we still use $x^{n+m}=x^nx^m$.

\textbf{+1420}\\
The mathematician and astronomer Jamshid al-Kashi computed and observed the solar eclipses of 1406, 1407 and 1408. He is also the first to use decimal notation in arithmetic and in arabic numerals.

\textbf{+1400}\\
The Mathematician and astronomer Jamshid al-Kashi developed an early form of Newton's Regula falsi method.

\textbf{+1200-1400}\\
Madhava and the Kerala School (India) discover several infinite series for numbers like $\pi$ and specific values of trigonometric functions - these works ahead of those of the Europeans on the differential and integral calculus and the series of powers.

\textbf{+1350}\\
\textit{Tractatus of configurationibus qualitatum and motuum} of Oresme is draft of geometry using coordinates, and use axes for different sizes, which is an important step in the transition from qualitative science based mainly on Aristotle to quantitative science. Proof of the mean velocity theorem, which anticipates the results of Gallileo on the uniform rectilinear motion and the bodies in free fall by linking area under the curve of the velocity to the position in a graph.

\textbf{+1303}\\
\textit{Siyuan Yujian }(translation:\textit{ Precious mirror of for elements}) of Zhu Sjijie describes the elimination method for solving systems of equations containing up to four unknowns and up to degree 14 for some form of equations. We also find there the definition of the Pascal's triangle and the formulas of summations for some series.

\textbf{+1300}\\
Raymon Lulle developed a geometric (useless) machine to automate theist logic. This idea will influence Leibniz in his research of a universal language for reasoning, research that will lead him to take an interest in Chinese writing and binary arithmetic. Lulle's ideas anticipate modern ideas of formal deduction systems.

\textbf{+1269}\\
The scholar Pierre de Maricourt coined the expressions of the magnetic "north" and "south poles" and he was the first who wrote that opposite poles attract each other

\textbf{+1268}\\
The philosopher, scientist and alchemist Roger Bacon publishes proposals to reform school, arguing that to study nature, the use of observations of the measures is the only basis of rigorous testing and verification while affirming at the same time the need of mathematics for this purpose.

\textbf{+1200}\\
The mathematician Jordan Nemore introduces the notation for unknowns with symbols.

\textbf{+1150}\\
Creation of modern notation for fractions (horizontal bar) by Al-Hassãr. At the same period, we have the Latin translation of the 820 treaty by Al-Khwarizmi on the Indian calculation which allows the decimal system and the use of zero to spread in Europe and also Gérard de Cremona publishes a translation in Latin of the Arabic version of the Almagest of Ptolemée, the name "sinus" comes from this translation...

\textbf{+1121}\\
The astronomer, physicist, biologist, chemist, mathematician and philosopher Abu al-Fath Khazini published a book in which he proposed that gravity and gravitational potential energy vary with distance from the center of the Earth. He also makes a distinction between force, mass and weight. He also invented several scientific instruments, including a steelyard and hydrostatic balance. He also introduces experimental scientific methods to static and dynamic, unifies them in the science of mechanics and combines hydrostatics with the dynamic to create hydrodynamics.

\textbf{+1114}\\
The mathematician Bhaskara provides a comprehensive summary of Hindu mathematics, as developed from the 5th to the 7th century AD. He also recognizes the negative square root, solves quadratic equations with several unknowns, equations of higher order such as Fermat and the general quadratic equations. He was also a pioneer in the principle of differential calculus nearly 500 years before Newton and Leibniz.

\textbf{+1100}\\
The philosopher and physicist Allah Abu'l-Barakat Hibat al-Baghdaadi is the first to deny Aristotle's idea that a constant force produces uniform motion what prepares the Newton's second law of motion. Like Newton, he described acceleration as the variation of speed.

\textbf{+1037}\\
The mathematician, physicist and philosopher Ibn al-Haytham is aware of the magnitude of the acceleration due to gravity. He discovers the law of inertia, known today as the first law of motion Newton.

\textbf{+1030}\\
The philosopher, writer, physician and scientist Abu Ali al-Husayn ibn Abd Allah Ibn Sina (known in Occident as Avicenna) note that if the perception of light is due to the emission of some sort of particles by a light source, the speed of light has to be finished. He also provided a sophisticated explanation for the phenomenon of rainbow. The mathematician, astronomer, physicist, scholar, encyclopedist, philosopher, astrologer, traveler, historian, pharmacologist Ab? al-Rayhan Muhammad ibn Ahmad al-Biruni, and later the astronomer Abu al'Fath Khazini, were the first to apply scientific methods in experimental mechanics, especially in the fields of statics and dynamics, to determine the specific weight, such as those based on the theory of balances and weighting.

\textbf{+1021}\\
The philosopher, mathematician and physicist Ibn al-Haytham is considered the father of optics and a pioneer of the scientific method describes correctly the light and vision, and introduces the experimental scientific method, laying the foundations of experimental physics. He also discusses experimental psychology and describes various optical instruments such as the darkroom. He wwas able to estimate atmosphere width with an accuracy of $1$ [km], he defined the inertia principia (first Newton's law), the linear momentum and he calculated $\sum_{k=1}^n k^4=\frac{n(2n+1)(n+1)(3n^2+2n-1)}{30}$.

\textbf{+1019}\\
The mathematician, astronomer and physicist Abu Rayhan Al-Biruni observed and described the solar eclipse of April 8, 1019, and the lunar eclipse of September 17, 1019, in detail; he gave the exact location of the stars during the lunar eclipse. He invented the astrolabe and the planisphere.

\textbf{+1010}\\
The mathematician Al-Sijistani Zuraqi invented a astrolabe designed for a single heliocentric planetary model in which the Earth is moving, rather than the sky.

\textbf{+1000}\\
The mathematician, physicist and astronomer Abu Sahl al-Qouhi discovers that the weight of bodies varies with their distance from the center of the Earth, and solves equations higher than the second degree. During the same decade, the mathematician and engineer Al'Karkhi wrote a book containing the first known proof by mathematical induction. He uses it to prove the binomial theorem, the Pascal's triangle, and the sum of the cubes integrals.

\textbf{+996}\\
The mechanical oriented astrolabe, with 8 gears is invented by the mathematician, astronomer and physicist Abū Rayhān Al-Biruni who is also the author of works on the summation of series and combinatorics.

\textbf{+980}\\
Abitu al-Wafaa do the first calculation of values of trigonometric functions and publish the sinus law adapted for triangles on sphere. He also proved by induction $\sum_{k=0}^n k^3=\frac{n^2(n+1)^2}{4}$.

\textbf{+964}\\
The mathematician, physicist and philosopher Abd al-Rahman al-Sufi explains the magnifying power of lenses and was the first to use a scientific method of analysis that will greatly influence future scientists.

\textbf{+953}\\
The engineer and mathematician Al-Karkhi defines different monomials and gives rules for products of any two of them. He also discovered the binomial theorem for integer exponents.

\textbf{+952}\\
The mathematician Abu'l-Hasan al-Uqlidisi modifies the calculation methods for the numerical Indian system to make it possible for feathers and paper usage. Until then, do calculations with Indian numerals necessitated the use of a board.

\textbf{+900}\\
The first reference to a viewing tube can be found in the work of the astronomer and mathematician Al-Battani, and the first accurate description of the observation tube was given by the mathematician, astronomer, physicist, scholar, encyclopedist, philosopher, astrologer, traveler, historian, pharmacologist Al-Biruni, in a section of his work dedicated to verify the presence of the new crescent moon at the horizon. Although these preliminary observations tubes do not have lenses, they allow an observer to focus on a part of the sky by eliminating light interference. These observation tubes were later adopted in Europe, where they influenced the development of the telescope.

\textbf{+880}\\
The astronomer and mathematician Al-Battani discovered the motion of the apogee of the Sun, calculate the values of the precession of the equinoxes and the inclination of the Earth's axis. He is at the origin of the definition of the tangent and cotangent trigonometric functions.

\textbf{+820}\\
The word "algebra" appears. The mathematician, geographer, astrologer and astronomer Muhammad ibn Musa Al'Khwarizmi is often regarded as the father of medieval algebra because he releases it from geometry. He is also the origin of the quadrant, the mural instrument, the sinus quadrant that was used to solve trigonometric problems and make astronomical observations.

\textbf{+800}\\
Astronomers invent the universal sundial and universal time dial in Baghdad.

\textbf{+780}\\
The alchemist Jabir Ibn Hayyan introduces the experimental scientific method for chemistry and also laboratory equipment such as still and processes such as pure distillation, liquefaction, crystallisation and filtration. He also invented more than 20 types of laboratory equipment, which resulted in the discovery of several chemicals. He also developed recipes for colored glass.

\textbf{+773}\\
Arabic numerals (adapted from India) made their first apparition in Europe.

\textbf{+628}\\
The mathematician Brahmagupta gives rules for solving linear and quadratic equations. He discovers that the quadratic equations have two roots: the negative one and the irrational and give the modern form of the solution we know today. He also give the rules to calculates with negative signs (arithmetic of negative numbers).

\textbf{+550}\\
Hindu mathematicians give zero a numeral representation in a positional notation system.

\textbf{+499}\\
The mathematician Âryabhat gets the full number of solutions of a system of linear equations by methods equivalent to modern methods, and describes the general solution of such equations. He also provides solutions of differential equations. He also asserts that the moon and celestial objects moon other than the stars reflect sunlight, he correctly explains the causes of lunar and solar eclipses, gives the length of the sidereal year to a few minutes, approximate $\pi$ by $62832/20000$, calculates the Earth's diameter with more precision than Erathostene, described the calculation with the Indian numbering system, gives the oldest sinus table for $24$ angles.

\textbf{+275}\\
The mathematician Diophantus of Alexandria considered as the father of algebra equations studied equations with rational variables (thus including quadratic equations) and Diophantine equations.

\textbf{+195}\\
\textit{Suàn shù shu}(translation: \textit{book on numbers and calculation}) in China is one of the oldest known Chinese mathematical texts that contains among other things, calculations of sums of Geometric progression for interest.

\textbf{+130}\\
\textit{Mathematical Composition} (also know under the name of \textit{Almagest's}) of Ptolemy in Alexandria presents a geometric model of the solar system that attempts to describe the motion of the planets the model is inspired by a geometric idea of Apollonius and uses circles whose center move in circular orbits this model puts the Earth at the center of the solar system but gives a fairly good description of the observed movements of the different stars it will be the dominant model until Copernicus some ideas prefigure the Fourier series that will be introduced in the 19th century.

\textbf{+125}\\
The \textit{Yale Music Papyrus} and the \textit{Michigan Instrumental Papyrus} seems to contain the oldest known examples of musical notation.

\textbf{+121}\\
Year corresponding to the oldest document mentioning the magnetic stone.

\textbf{+120}\\
Star catalog of Zhang Heng which contains 2500 stars. Zhang Heng correctly describes the cause of the eclipses and demonstrates that the moon is spherical.

\textbf{+100}\\
The engineer, mechanician and mathematician Hero of Alexandria rediscovered (after the Chinese) the concept of force. He also invented a system of gears to lift weights using steam power. He gives the first description of the sextant (but did not, however, invented it). His contemporary astronomer Claudius Ptolemy invented the sextant and described the astrolabe (perhaps invented by the astronomer, geographer and mathematician Hipparchus) and studied the refraction and reflection. During the same century the mathematician and philosopher Nicomachus of Gerase defines the even and odd numbers, prime and composite numbers, and perfect numbers. Also during the same century the final version of the \textit{Nine Chapters on Mathematical Art} (almost $1180$ pages), written over ten years by several anonymous Chinese authors contains the first use of negative numbers, chapter 9 uses the Pythagorean theorem, chapter 8 uses matrices and the elimination of Gauss to solve systems of equations (at least 1700 years before Gauss!!!)

\textbf{+80}\\
The scholar Wang Ch'ung made the first magnetic compass on a plate of brass.

\textbf{-87}\\
Year corresponding to the dating of the Antikythera mechanism, considered as the first calculator and the first antique analog gear (thirty!) machine to calculate complexes astronomical positions. The care and skill with which this machine was made, as well as the necessary mechanical and astronomy capacities question the historical knowledge of Greek science before its discovery. Indeed, any object of the same age and same complexity was known in the world and it takes nearly a millennium to see similar mechanisms appear! The physicist, mathematician and engineer Archimedes of Syracuse is the hypothetical creator.

\textbf{-100}\\
The indian text \textit{Anuyoga Dwara Sutra} contains several identities involving square roots and squares that seem to imply some knowledge of the laws of exponents or logarithms. An identity taken from this text in modern notation: $\sqrt{a}\sqrt{\sqrt{a}}=(\sqrt{\sqrt{a}})^3$.

\textbf{-134}\\
The astronomer, geographer, and mathematician Hipparchus of Nicaea discovers the precession of the equinoxes.

\textbf{-150}\\
The astronomer, geographer and mathematician Hipparchus is often referred as the founder of trigonometry and corresponding numerical tables. He calculates the first period of revolution of the Sun around the Earth (but the numerical results are in fact those of the rotation of the Earth around the Sun) and develops the theory of eccentrics and epicycles.

\textbf{-200}\\
During the century, the Chinese had invented the tide gate, the rudder, the principle of the steam engine several hundred years before the occident! During this century, the scientist and engineer Philo of Byzantium wrote treatises on levers, pneumatics, automation, traction and water clocks.

\textbf{-225}\\
The astronomer and mathematician Apollonius of Perga published the first study on conics giving to the ellipse, the parabola and the hyperbola the names we know today. He is also credited for the hypothesis of eccentric orbits to explain the apparent motion of the planets and the speed variation of the Moon.

\textbf{-250}\\
The mathematician, physicist and engineer Archimedes of Syracuse study simple machines such as the lever, the famous screw for pumping water ("Archimedes screw") and discovered the Archimedes's principle explaining buoyancy. In the same decade, the astronomer, geographer, philosopher and mathematician Eratosthenes of Cyrene calculated the diameter of the Earth using a gnomon and its shadow and demonstrated the inclination of the ecliptic, the distance Earth-Moon and Earth-Sun and also created a method to determine prime numbers (Sieve of Eratosthenes).

\textbf{-260}\\
The mathematician, physicist and engineer Archimedes of Syracuse computes to two decimal places using inscribed and cirumscribed polygons and computes the area under a parabolic segment.

\textbf{-281}\\
The astronomer and mathematician Aristarchus of Samos assumed that the Sun is the center of the solar system and uses trigonometry to estimate the radius of the Moon and its distance from the Earth using the Earth's shadow during a lunar eclipse.

\textbf{-300}\\
The mathematician and geometer Euclid published his \textit{Elements}, where he reorganized the entire knowledge of the geometry including logical proofs, the construction of the $5$ Platonic solids. In his \textit{Optica} he noted that the light goes in a straight line and describes the law of reflection. We have also that the conical, elliptical, parabolic and hyperbolic concepts appears in the works of two mathematicians of the ancient Greece, namely Menaech-mus and Appolonius de Perge that also introduces the concept of tangent. The same period Bhagabati Sutra calculates the permutations and combinaisons of order $1$, $2$ and $3$.

\textbf{-310}\\
The scholarly Autolycus of Pitane defines uniform motion as an object that travels an equal distance in a equal amount of time.

\textbf{-370}\\
The philosopher Aristotle develops the logic with a theory of naive proposals, quantities and inferential reasoning.

\textbf{-388}\\
The philosopher and astronomer Heraclides of Pontus assumes the rotation of the Earth itself to explain the apparent motion of stars in the night (but still in a geocentric context) and suggests that each planet is a body like the Earth.

\textbf{-400}\\
The Stoic School develops the composed proposals and the logical connectors: "implies", "and", "or" and the inferences "Modus ponens" and "Modus tollens".

\textbf{-430}\\
The philosopher Democritus of Abdera advance the idea that matter is composed of tiny and indentical particles he name "atoms". In reality it is rather an extension of the ideas of his teacher, the philosopher Leucippus of Miletus developed ten years before. Hippasus, a disciple of Pythagoras, would have given what is probably the first rigorous proof of the irrationality of $\sqrt{2}$. The proof uses a Reductio ad absurdum reasoning to show that the sides of a square are incommensurable with its diagonal. 

\textbf{-500}\\
The philosophers Leucippus and Democritus are the founders of atomism.

\textbf{-540}\\
The philosopher, mathematician Pythagoras studies propositional geometry and vibrating lyre strings. 

\textbf{-600}\\
The philosopher Empedocles of Acragas calls for decomposition of the world into four fundamentals elements: water, earth, air and fire. The same century, the mathematician Thales of Miletus highlights electrostatic by rubbing a piece of amber, predicted an eclipse and develops the geometry of the triangle.

\textbf{-750}\\
Manava Sulbar Sutras of Manava finds irrationality of $\sqrt{2}$ and $\sqrt{61}$ and agrees to use irrational numbers in his calculations.

\textbf{-800}\\
Assyrians use water-clocks and Chinese plot planetary movements for their calendar

\textbf{-1500}\\
Indians develops theory of the $4$ elements (fire, air, water, earth)

\textbf{-1400}\\
The Neolithic peoples of Scotland nowadays build stone models of Plato's five solids (regular polyhedra).

\textbf{-1700}\\
The mathematician Apastamba solves general linear equations and uses Diophantine systems of equations with up to five unknowns. The same century, egyptian mathematicians employ primitive fractions.

\textbf{-1800}\\
Babylonian scribes seek the solution of a quadratic equation.

\textbf{-2000}\\
Babylonian priests do the first records of celestial observations.

\textbf{-2300}\\
Chinese astronomers make the first observations of the sky.

\textbf{-2500}\\
The Mesopotamians imagined a position numbering system composed of symbols whose value is based on their rank within a number.

\textbf{-2600}\\
Oldest known mathematical table where a multiplication table is engraved to calculate areas.

\textbf{-3000}\\
Chinese and Babylonians invented the abacus as first adding machine. Geometric concepts are developed for land surveying (hypotenuse calculus). It is also the period corresponding to the oldest know tool used by Incas to record numbers thanks to knots on a string and also the wall representation of wheels.

\textbf{-3500}\\
Oldest Weather Report Found on Stone in Egypt. The unusual weather patterns described on the slab were the result of a massive volcano explosion at Thera, the present day island of Santorini in the Mediterranean Sea

\textbf{-3700}\\
A Babylonian tablet seems to contain first remarkable trigonometric angles and the evidence that Pythagoras' theorem was already known by Babylonians.

\textbf{-4900}\\
The Goseck circle (German: Sonnenobservatorium Goseck) may be one of the oldest Solar observatories in the world.

\textbf{-5000}\\
The decimal system is used in Ancien Egypt (it seems that the consensus is between -6000 and -3000).

\textbf{-8000}\\
Warren Field is the location of a mesolithic calendar. It includes $12$ pits believed to correlate with phases of the Moon and used as a lunar calendar. It is considered to be the oldest lunar calendar yet found.

\textbf{-5200}\\
Radiocarbon dating of the oldest founded wheels (Ljubljana Marshes wooden Wheel).

\textbf{-8000}\\
Marks one bones or woods are slowly replaced by tokens of various shapes to count.

\textbf{-20000}\\
The Ishango bone is a bone tool, dated to the Upper Paleolithic era. It is a dark brown length of bone, the fibula of a baboon, with a sharp piece of quartz affixed to one end, perhaps for engraving. It was first thought to be a tally stick, as it has a series of what has been interpreted as tally marks carved in three columns running the length of the tool. It has also been suggested that the scratches might have been to create a better grip on the handle or for some other non-mathematical reason.

\textbf{-200000}\\
Claims for the earliest definitive evidence of control of fire by a member of Homo range from 0.2 to 1.7 million years ago. Evidence for the controlled use of fire by Homo erectus, beginning some 400,000 years ago, has wide scholarly support.

	\chapter{Humor}
	\minitoc
	\pagebreak
	\input{Chapter_Humor.tex}

 	\chapter{Links}
	\input{Chapter_Links.tex}

			
	\chapter{Quotes}
	\begin{figure}[H]
		\centering
		\includegraphics{img/be_greater_than_average.jpg}	
	\end{figure}
	\begin{fquote}[John Aubrey]Without faith, God is nothing. Without science, man is nothing.
 	\end{fquote}
 	
	\begin{fquote}[Leonardo da Vinci]He who loves practice without theory is like the sailor who boards ship without a rudder and compass and never knows where he may cast.
 	\end{fquote}
 	
	\begin{fquote}[Richard Feynman]Physics is like sex. Sure it may have some practical results, but that's not why we do it.
 	\end{fquote}
 	
	\begin{fquote}[Martin Gardner]Mathematics is not only real, but it is the only reality. That is that entire universe is made of matter, obviously. And matter is made of particles. It's made of electrons and neutrons and protons. So the entire universe is made out of particles. Now what are the particles made out of? They're not made out of anything. The only thing you can say about the reality of an electron is to cite its mathematical properties. So there's a sense in which matter has completely dissolved and what is left is just a mathematical structure.
 	\end{fquote}
 	
 	\begin{fquote}[Owen Chamberlain]Each generation of scientists stands upon the shoulders of those who have gone before.
 	\end{fquote}
 	
 	\begin{fquote}[Kiyoshi Ito]In precisely built mathematical structures, mathematicians find the same sort of beauty others find in enchanting pieces of music, or in magnificent architecture. There is however, one great difference between the beauty of mathematical structures and that of great art. Music by Mozart, for instance, impress greatly even those who do not know musical theory; the cathedral in Cologne overwhelms spectators even if they know nothing about Christianity. The beauty in mathematical structures, however, cannot be appreciated without understanding of a group of numerical formulae that express laws of logic. Only mathematicians can read "musical scores" containing many numerical formulae, and play that "music" in their hearts.
 	\end{fquote}
 	
 	\begin{fquote}[Ray Bradbury]You don't have to burn books to destroy a culture. Just get people to stop reading them.
 	\end{fquote}
 	
 	 \begin{fquote}[Edward Teller]The science of today is the technology of tomorrow.
 	\end{fquote}
 	
 	 \begin{fquote}[Brian Cox]Science is different to all the other systems of thought... because you don't need faith in it, you can check that it works!
 	\end{fquote}
 	
 	\begin{fquote}[George Bernard Shaw]The fact that a believer is happier than a skeptic is no more to the point than the fact that a drunken man is happier than a sober one.
 	\end{fquote}
 	
 	\begin{fquote}[Carl Sagan]I don't want to believe, I want to know.
 	\end{fquote}
 	
 	 \begin{fquote}[Richard Feynman]Science is just imagination in straitjacket.
 	\end{fquote}
 	
 	 \begin{fquote}[Stephen Hawking]The greatest enemy of knowledge is not ignorance, is the illusion of knowledge.
 	\end{fquote}
 	
 	 \begin{fquote}[?]Without data you are juste another person without opinions, but without mathematics you are just a data collector.
 	\end{fquote}
 	
 	\begin{fquote}[Albert Einstein]I have no special talents, I am only passionately curious.
 	\end{fquote}

	\begin{fquote}[Dara Briain]People keep saying "Science doesn't know everything!". Well, science know it doesn't know everything; otherwise it would stop.
 	\end{fquote}
 	
 	\begin{fquote}[?]Without faith, God is nothing. Without science, man is nothing!
 	\end{fquote}

 	 \begin{fquote}[Stephen Hawking]One can't prove that God doesn't exists, but Science makes God unnecessary.
 	\end{fquote}

 	 \begin{fquote}[Voltaire]The power of numbers was much more respected among us when we knew nothing about them..
 	\end{fquote}

	\begin{fquote}[Jean Rostand]To reflect is to disturb one's thoughts.
 	\end{fquote} 
 	
 	\begin{fquote}[John Cleese]If you are stupid how can you possibly realize that you are stupide?.
 	\end{fquote} 	
 	
 	\begin{fquote}[Frank Wilczek]If you don't make mistakes, you're not working on hard enough problems. And that's a big mistake.
 	\end{fquote}

	\begin{fquote}[?]Success relies on the success of those who have preceded us.
 	\end{fquote}
 	
 	\begin{fquote}[Brigitte Le Roux, Henry Rouanet]Between quantity and quality there is geometry.
 	\end{fquote}

	\begin{fquote}[Leonardo da Vinci]Practice should always be based upon a sound knowledge of theory.
 	\end{fquote}
 	
 	\begin{fquote}[Richard Feynman]It doesn't matter how beautiful your theory is, it doesn't matter how smart you are. If it doesn't agree with experiment, it's wrong.
 	\end{fquote}
 	
 	\begin{fquote}[Ronald Fisher]To call in the statistician after the experiment is done may be no more than asking him to perform a post-mortem - he may be able to say what the experiment died of.
 	\end{fquote}
 	
 	\begin{fquote}[Charles Babbage]Errors using inadequate data are much less than those using no data at all
 	\end{fquote}
 	
 	\begin{fquote}[Leonardo da Vinci]Details make perfection, and perfection is not a detail.
 	\end{fquote}
 	
 	\begin{fquote}[LNeil Degrasse Tyson]Ignorance is a virus. Once it starts spreading, it can only be cured by reason. For the sake of humanity, we must be that cure!
 	\end{fquote}
 	
 	\begin{fquote}[David Hilbert]The art of doing mathematics consists in finding that special case which contains all the germs of generality. 
 	\end{fquote}

 	\begin{fquote}[Mark Twain]There are three kinds of lies: lies, damned lies and statistics. 
 	\end{fquote}

 	\begin{fquote}[Derek Bok]If you think education is expensive, try ignorance.
 	\end{fquote}
 
  	\begin{fquote}[David Hilbert]The art of doing mathematics consists in finding that special case which contains all the germs of generality.
 	\end{fquote}
 	
 	 \begin{fquote}[Benjamin Franklin]Tell me and I forget, teach me and I remember, involve me and I learn.
 	\end{fquote}
  	
 	 \begin{fquote}[Albert Einstein]Not everything that can be counted counts, and not everything that counts can be counted.
 	\end{fquote}

 	 \begin{fquote}[Niels Bohr]An expert is a person who has found out by his own experience all the mistakes that one can make in a very narrow field.
 	\end{fquote}
 	
 	 \begin{fquote}[Daniel C. Denett]What you can imagine depends on what you know.
 	\end{fquote}
 	
 	\begin{fquote}[Arthur C. Clarke]Any sufficiently advanced technology is indistinguishable from divinity.
 	\end{fquote}
 	
 	\begin{fquote}[Nikola Tesla]What one man calls God, another call it the Laws of Physics.
 	\end{fquote}
 	
 	\begin{fquote}[?]The only person you need to be better thhan is the person you were yesterday.
 	\end{fquote}
 	
 	\begin{fquote}[Eleanor Roosevelt]Great minds discuss ideas. Average minds discuss events. Small minds discuss people.
 	\end{fquote}
 	
 	\begin{fquote}[Marie Curie]Be less curious about people and more curious about ideas.
 	\end{fquote}
 	
 	\begin{fquote}[Albert Einstein]Creativity is intelligence having fun.
 	\end{fquote}
 	
 	\begin{fquote}[Aaron Swartz]With enough of us, around the world, we'll not just send a strong message opposing the privatization of knowledge - we'll make it a thing of the past.
 	\end{fquote}
 	
 	\begin{fquote}[Buddha]When you move your focus from competition to contribution life becomes a celebration. Never try to defeat people, just win the hearts.
 	\end{fquote}
 	
 	\begin{fquote}[Bertrand Russell]The whole trouble with the world is that the stupid are cocksure, and the intelligent are full of doubt.
 	\end{fquote}
 	
 	\begin{fquote}[Gerge Bernard Show]Some men see things as they are and ask why. Others dream things that never were and ask why not.
 	\end{fquote}
 	
 	\begin{fquote}[Albert Einstein]If you can't explain it to a six year old, you don't understand it well enough yourself.
 	\end{fquote}
 	
 	\begin{fquote}[Hubert Reeves]Man is the most insane species. He worships an invisible God and destroys a visible Nature. Unaware that the Nature he is destroying is this God he is worshiping.
 	\end{fquote}
 	
 	\begin{fquote}[Willima Thomson (Lord Kelvin)]If you can't measure it, you can't improve it!
 	\end{fquote}
 	
 	\begin{fquote}[Albert Einstein]Once you stop learning, you start dying.
 	\end{fquote}
 	
 	\begin{fquote}[Socrates]There is only one good, KNOWLEDGE, and one evil, IGNORANCE.
 	\end{fquote}
 	
 	\begin{fquote}[?]The purpose of argument, should not be victory, but progress.
 	\end{fquote}
 	
 	\begin{fquote}[Richard Dawkins]Science is the poetry of reality!
 	\end{fquote}
 	
 	\begin{fquote}[Sthepen Hawking]Real science can be far stranger than science fiction and much more satisfying.
 	\end{fquote}
 	
 	\begin{fquote}[Alan Watts] You don't understand the basic assumptions of your own culture if your own culture is the only culture you know.
 	\end{fquote}
 	
 	\begin{fquote}[Elon Musk]The ones who are crazy enough to think that they can change the world are the ones who do!
 	\end{fquote}
 	
 	\begin{fquote}[Leonardo da Vinci]The noblest pleasure is the joy of understanding!
 	\end{fquote}
 	
 	\begin{fquote}[Galileo Galilei]Doubt is the father of invention!
 	\end{fquote}
 	
 	\begin{fquote}[Fred Mosteller]It is easy to lie with statistics, but easier to lie without them.
 	\end{fquote}

\chapter{Change Log}

This is a detailed change log of the book for people interested to see how this book has evolved (to keep in touch with new releases send us an e-mail to subscribe to the newsletter):
	\begin{itemize}
		\item \textbf{May 2002}
		\begin{itemize}[noitemsep]
			\item Definitions (science, law, theorem, postulate, axiom, corollary, ...)
			\item Operations of addition, subtraction, multiplication, division, power
			\item Concept of Numbers (integers, relative real, fractional, complex algebra, abstract,...)
			\item Domain of definition of variables
			\item Arithmetic polynomials
			\item Absolute value
			\item Binary relations, order relations
			\item Gamma Function Euler, Euler's constant
			\item Electromagnetic wave equation, the wave speed, speed of the light, transported energy
			\item Introduction to Optics
			\item Rule of three, percentages
			\item Net quantities, the cost price of purchase, indices, sale prices, brut prices, net profit, net price, assets
			\item Simple and compound interest, late and early payments
			\item Portfolio management (decision criteria, goodwill, return on investisment)
			\item MarkowitzModel  (utility function, selection criteria)
			\item Vocabulary about Stock Exchange
			\item Plasma Frequency
			\item Probability (universe, events, axioms)
			\item Combinatorial Analysis
			\item Statistics (discrete variables, continuous variable, standard deviation, variance and covariance)
			\item Distribution Functions (discrete uniform law, Bernoulli's law, binomial law, hypergeometric law, multinomial law, Poisson's law, Gauss-Laplace law, Cauchy's law, beta law, gamma law, chi-square law, Student's law)
			\item Estimator, correlation
			\item Matrix of covariance
			\item Statistical adequation tests
		\end{itemize}
		\item \textbf{June 2002}
			\begin{itemize}[noitemsep]
				\item Introduction Set Theory (Zermelo-Fraenkel theory, inclusion, complementarity, intersection, union, product, empty set
				\item Arithmetic, harmonic, geometric, quadratic averages
				\item Univocity
				\item Logarithmic and exponential functions
				\item Golden ratio
				\item Introduction to Fourier series
				\item Coulomb's law, electrostatic field, electrostatic potential
				\item Ampere's Theorem
				\item Biot-Savartlaw, magnetic field, magnetic induction
				\item Maxwell's equations
				\item Dimensions in euclidiean and fractal geometry
				\item Geometric shapes unidimensionelles, bidimensionelles, tridimensionelles
				\item Pythagorean theorem
				\item Physical units systems
				\item The principle of least action and energy conservation
				\item Position, velocity and acceleration
				\item Continuity equation
				\item Bernoulli Equation
				\item The Doppler effect
				\item Restrained Relativity, invariance principle, Lorentz transformations (time, length, velocity addition, increase in mass, Minkowski space-time)
				\item General relativity (world line, punctual event, inertial particles, null cones, space-time vectors spaces and curved planes, metric tensor)
				\item Heisenberg principles of uncertainty, Schrödinger equation, Wave probability density
				\item Introduction to superstrings
			\end{itemize}
		\item \textbf{July 2002}
			\begin{itemize}[noitemsep]
				\item Creating of Table of Contents
				\item Visual representations of functions
				\item 2nd degree polynomials and roots
				\item Operator of vector and scalar fields (gradient, nabla, divergence, curl, laplacien)
				\item Vector Analysis (concept of arrow, set of vectors, scalar multiplication, vector space, linear combinations, generating families, bases of a vector space)
				\item Tensor calculus (Einstein convention, Kronecker symbol, anti-symmetry symbol)
				\item Notations for grouped multiplications (Big Sigma) and sums (Big Pi)
				\item Axioms for set of real number 
				\item Definition of various Inequalities 
				\item Circular and related movements
				\item Inertial forces (Coriolis force)
				\item Drake Equation
			\end{itemize}
		\item \textbf{August 2002}
			\begin{itemize}[noitemsep]
				\item New biographies (Hilbert, Riemann, Legendre)
				\item New section "Humor"
				\item Introduction to Topology
				\item Euclid postulates
				\item Gauss plane (complex numbers)
				\item De Moivre's formula
				\item Transformations in the complex plane
				\item Base change and tensor scalar product
				\item Covariant and contravariant components
				\item Relativistic transformations of the momentum
				\item Prefixes of multiples and submultiples of SI units
				\item Quality control (probabilities), efficiency curve, level value of acceptable quality (LQA)
				\item Kepler's laws
				\item Proof of classical gravitational force from Kepler's Law
				\item Proof of area law (second Kepler's Law)
				\item Moment of force
				\item Angular momentum
				\item Theorem of angular momentum
				\item Newton-Poisson Equation
				\item Thomson's, Borh's and Sommerfeld-Bohr's atom model
				\item Hydrogen spectrum
				\item Assumption of Neutron
				\item Quantum numbers
				\item Pauli exclusion Principles
				\item Classical Analysis of the Schrödinger equation for the ideal rectangular potential
				\item Newton's Laws
				\item Proof of Newton's third law from the principle of least action
				\item Release speed (for rockets or others)
				\item Definition of "Scientism"
				\item Variation of the gravitational acceleration in and out of a homogeneous spherical body
				\item The principle of least action and quantum physics (semi-classical limit)
				\item Ballistics (maximum range, safety parabola )
				\item Introduction to nuclear physics
				\item Atomic number, mass number
				\item Radioactivity, activity, filiation, isotopes, isotones, nuclide dating
				\item Atomic mass system (UMA)
				\item Masse defaults
				\item Fusion and fission
				\item Alpha and beta (minus and plus) desintegration
				\item Electron capture
				\item Gamma emission
			\end{itemize}
		\item \textbf{September 2002}	
			\begin{itemize}[noitemsep]
				\item New Biographies (Dalton, Boltzmann, De Broglie, etc.)
				\item Principle of good order
				\item Archimedean Property
				\item Induction Principle
				\item Divisability (Euclidean division)
				\item Congruent numbers
				\item Proof by nine
				\item Numbers basis
				\item Definition utility of thousands separators
				\item Priorities of parenthesis, brackets, braces and arithmetic operators
				\item Proof Theory (introduction)
				\item Definition of terms, formulas and demonstrations
				\item Definition of languages, symbols, relations and functions
				\item Definition of periodic, compound, basic, rational, fractional, irrational, algebraic and transcendental numbers
				\item Generalization of elementary algebra
				\item Dimensions of a vector space, extension of a free family, rank of a finite family
				\item Vector Hyperplane, direct sums
				\item Almosrt rigorous definition of the concepts of line, surface (plane) and volume
				\item Straight line intersection, half-lines, segments, aliquot part
				\item Continuity axiom of the line and of the plane
				\item Translation and rotation of plan
				\item Angles, units, measurements, sides of the angle, sharp edges, flat angles, equal angles ,straight angles, acute angles, obtuse angles, supplementary angles, complementary angles, 
				\item Perpendicular lines, bissecting angle
				\item Definition of work and energy: kinetic and potential energy theorem (motor work and resisting work)
				\item Concept of conservative vector field
				\item Conservation of energy and momentum
				\item Mass center theorem
				\item Relativistic force transformation
				\item Relativistic transformations of electric and magnetic fields
				\item Chandreskhar limit weights (collapse limit of white dwarfs)
				\item Definitions of optics, generalization of the law of refraction
				\item Broglie normalization condition, linked and non-linked states
				\item Harmonic oscillator
				\item Quantum chemistry and molecular vibrations
			\end{itemize}
		\item \textbf{Octobre 2002}
			\begin{itemize}[noitemsep]
				\item New biographies (Cauchy, Neumann, Bessel, Archimedes)
				\item New section on "Theoretical Computing"
				\item Greatest common divisor, least common multiple
				\item Rule signs (...)
				\item Proof of the irrationality of a number
				\item Introduction to arithmetic, harmonic and geometric sequences
				\item Limits and continuity of functions
				\item Definition of affine spaces
				\item Introduction to the Euclidean tensors and their properties
				\item Triangles and properties of triangles
				\item Definition of thermodynamic systems
				\item Definition of the reduced mass
				\item Bessel's functions
				\item Definition of the moment of inertia
				\item Introduction to Quantum Field Theory
				\item Introduction to radiation protection
				\item Proof of Bethe-Bloch formula
				\item Tunnel effect in quantum physics
				\item Introduction Dirac's formalism
				\item Heron and Archimedes algorithm's
				\item Introduction to fractal sets
				\item Introduction to game theory (cooperative games, earnings, payoff matrix, extensive forms, Pareto optimums, Nash equilibrium, evolutionary games)
			\end{itemize}
			\item \textbf{November 2002}
				\begin{itemize}[noitemsep]
				\item Fundamental theorem of arithmetic
				\item Introduction to crypthography (RSA, DES, MD5, SHA-1)
				\item Euler phi function
				\item Small Fermat theorem
				\item Introduction to topological dimensions and scaling 
				\item Cosine directors
			\end{itemize}
			\item \textbf{December 2002}
				\begin{itemize}[noitemsep]
				\item New Biographies (Nash, Cartan, Lucas, Lie)
				\item Detailed development of the "Integer" function
				\item Superposition of linear quantum states (quantum coherence) 
				\item Definition of Lipschitz functions and contracting functions 
				\item Definition of a convergent Cauchy sequence 
				\item Fixed point theorem (used in fractals, Newton methods and many others)
				\item Definitions of reality, the problem of the theory and trial on reality
				\item Definition of Euclidean space and Euclidean affine space
				\item Definition of property concepts in the field of chemistry
			\end{itemize}
			\item \textbf{January 2003}
			\begin{itemize}[noitemsep]
				\item Pascal's Theorem
				\item Buoyancy (Archimedes' principle)
				\item Simple Introduction to different symmetries in physics (temporal, spatial)
				\item Simple Introduction to different transformations in the plane (translation, scaling, reflection, isometry, rotation)
				\item Definition of an inverse and composed function/application
				\item Trigonometry (introduction, remarkable relations/identities, spherical trigonometry)
				\item Signature of a vector space
				\item Schmidt orthogonalization methods, base changes, Fourier associated spaces
				\item Everything (almost) on the plane and spherical trigonometry
				\item Keplerian orbital trajectories
				\item Introduction to the neoclassical monetary model (Say/Walras laws, homogeneity assumption)
				\item Boolean algebra (simple properties and theorems)
				\item Redesigned of quantum physics section (order of the subjects)
				\item Proof of evolutionnary Schrödinger  equation
				\item Proof of the relativistic evolutionnary Schrödinger equation
				\item Introduction to Antimatter theory
			\end{itemize}
		\item \textbf{February 2003}
				\begin{itemize}[noitemsep]
				\item Proof of gradient, divergence, rotational and Laplacian in cartesian, polary, cylindrical and spherical coordinates
				\item Complete mathematical developments of speed and acceleration expressions in cartesian, polar, cylindrical and spherical coordinates 
				\item Proof  of the relativistic invariance of the electric charge (charge conservation equation) 
				\item Proof of the existence of anti-particle  with opposite charge 
				\item Introduction to Gauge Theory (quadripotentiel Lorenz gauge, Coulomb gauge, Alembertian) 
				\item Introduction to Lagrangian and Hamiltonian formalism (generalized coordinates, configuration spaces, Euler-Lagrange equation, canonical formalism, Legendre transformation, Poisson brackets) 
				\item Rigorous definition of the principle of least action
				\item Definition of Tensor spaces
				\end{itemize}
		\item \textbf{March 2003}
			\begin{itemize}[noitemsep]
				\item New Biographies (Lorentz, Minkowski Hermann, Ricci-Curbastro, Levi-Civita)
				\item Definition of Cartesian product and extension of the scope of application of Cardinals
				\item Proof of Cauchy-Schwarz inequality
				\item Proof of the triangle inequality 
				\item Definitions of the vector product and mixed product 
				\item Proof of condensed form the sum of the first $n$ integers 
				\item Proof of the validity of the integration by parts 
				\item Definition of the algebraic vectorial structure and of an algebra 
				\item Definition of a homorphismes, isomorphism, endomorphism, automorphism
			\end{itemize}
		\item \textbf{April 2003}
			\begin{itemize}[noitemsep]
				\item New Biographies (Göpper-Meyer, Yukawa, Nöther, Cournot)
				\item Cournot Theory equilibrium (competition)
				\item Wilson's model (inventory management)
				\item Mathematics of phasers
				\item Relativistic model of the Sommerfeld's atom
				\item Analytical resolution of the Schrödinger equation
				\item Heisenberg's principles of quantum  uncertainty
				\item Lagrangian formalism of quantum physics fields
				\item Improvement of website forum (add mathematical symbols and external files)
				\item 10 new links to interesting web pages (associations + math stuff)
			\end{itemize}
		\item \textbf{Mai 2003}
			\begin{itemize}[noitemsep]
				\item New Biographies (Bell, Ramanujan, Landau)
				\item Proof of the precession of the perihelion of coupled orbits of stars or electric charges
				\item Definition and developments related to the Virial theorem
				\item Calculation of potential energy of a material sphere (internal temperature of Stars)
				\item Definition of prime numbers and proof that they are in infinite in number
				\item Definition of a fully enclosed ring
				\item Proof that a rational number is an algebraic number if and only if it is a relative integer.
				\item Definition of a multi-linear application/function (or morphism of vector space)
				\item Definition of a partition of a set and an an equivalence class.
				\item Descriptions of set operations of absorption and idempotence.
				\item Definitions and examples of sagittals diagrams.
				\item Definition of a magma and a monoid
				\item Pseudo-proofs of algebraic structures of fundamental sets of arithmetic
				\item Development of the theory of angular momentum in wave quantum physics
				\item Definition of a Diophantine equation and sets of Fermat's Last Theorem
			\end{itemize}
		\item \textbf{October 2003}
			\begin{itemize}[noitemsep]
				\item New Biographies (Abel, Banach, Boole, Bose, Brouwer, Clausius, Cayley, Curie,  Connes, Dirichlet, Frege, Gibbs, Picard, Erdos, Grothendieck, Hamilton, Hausdorf,  Heaviside, Helmholtz, Hermite, Hoyle, Jacobi, Klein, Kronecker, Langevin, Lee, Lobachevsky , Möbius, Monge, Fish, Schwartz, Shannon, Thom, van der Waals, Vieta, Weinberg, Witten, Gamow, Sturm, Liouville,Clairaut, Teller)
				\item Definition of puncutal spaces in classical mechanics, of writing conventions and changes in referentials
				\item Mathematical theory of projective perspective with vanishing points and definition of the projective and isometric perspectives
				\item Simplistic definition of the concept of "derivative" in functional analysis.
				\item Proof of some common derivatives (polynomials, composite functions, inverse functions, cosine, sine, arc sine, arc cosine, quotient of two functions, etc.).
				\item Numerical Methods: mathematical explanation of the complexity of an algorithm and  elementary optimization research
				\item Method of calculating the number $e$
				\item Resolution Method of $n$ linear equations systems with $n$ unknow using the pivot method
				\item Search of function roots by the methods of the proportional parts, of the bisection method, of the secant methods (regula falsi) and Newton's method
				\item Calculations of areas and integrals using the Riemann sums method
				\item Presentation of the Monte Carlo calculation for integrals, or Pi
				\item Mathematical development of simplex algorithm used in the operations research (linear programming)
				\item Contraction of indices in tensor calculus
				\item Definition and detailed properties of some speical tensor (symmetric tensor, antisymmetric tensor, fundamental tensor, etc.)
				\item Curvilinear coordinates (determination of the metric and linear element of a spherical and cartesian space and of the plane in polar coordinates)
Christoffel symbols (first and second type).
				\item Proof of the Cantor-Bernstein theorem
				\item Determination of the free generalized Lagrangian in General Relativity
			\end{itemize}
		\pagebreak
		\item \textbf{December 2003}
			\begin{itemize}[noitemsep]
				\item New Biographies (Kirchhoff biographers, Markowitz, Cox Merton, Scholes, Sharpe, Ferdinan von Lindemann, Bachelier, Stefan)
				\item Proof of one of the Stirling formulas
				\item Detailed calculation of a simple model of a Star surface temperature
				\item Detailed calculations of temporal properties of the securities values
				\item Proof of Taylor series and Maclaurin (limited and unlimited)
				\item Defining the Lagrange rest and of Alembert's, Cauchy, integral test and absolute convergence criterias
				\item Developments relative to the definitions of brightness, luminosity, apparent and absolute magnitude of Stars and calclulation of the distance to Cepheids
				\item Definitions of the solid angle, the solid angle of revolution and the elementary solid angle
				\item Definition of photometric and photonic and international system quantities
				\item Definitions and developments related to the light intensity, energy flow (with proof of Beer-Lambert' law), emittance, radiance (with Lambert's law), Kirchhoff's law
				\item Proofs of Stefan's law and Stefan-Boltzmann law
				\item Resolutions of third degree polyonials by radicals (Cardan's method) and development about solving quadratic polynomials in the complex set
				\item Definition of the concept of equations and inequalities
				\item Determination of the Cartesian equation of the plane, line (in space),cone and sphere
				\item Definition of market efficiency
				\item Determination of the Black \& Scholes equation 
				\item Presentation of the mathematical aspectts of Wiener process
				\item Ito lemma and Brownian motion (random walk)
			\end{itemize}
		\item \textbf{January 2004}
			\begin{itemize}[noitemsep]
				\item Definition of the indefinite integral
				\item Newtonian cosmological model (without the cosmological constant)
				\item Statement of the Peano's
				\item Introduction to Linear Algebra (Gauss reduction method, elementary operations between matrices)
				\item Statement of 5 Euclid's axioms and 5 groups of axioms in geometry
				\item Proof of relations for calculating the perimeter of the surface and the center of gravity of the square, the rectangle and triangle
				\item Proof of relations for calculating the volumes and surfaces of the torus, the sphere, the ellipsoid, cylinder and cone
				\item Definition of the centroid (center of gravity) and demonstration of four properties related thereto
				\item Demonstration of decomposition odd function and pair of any function
				\item Definition of hyperbolic trigonometric functions and enumeration of relations and proof of related remarkable properties
				\item Introduction to differential geometry (definition of a Riemannian geometry, Frenet triad, parametrised surface, etc.)
				\item DIntroduction to graph theory (Königsberg's bridges proof)
				\item Definition of a topological space and Hausdorff definition of a metric / ultra-metric space and associated distances (hölder, discreet, equivalent ...)
				\item Definition of a Lipschitz function (and related isometrics)
				\item Definition of open and closed set (open/closed balls, spheres, adherence, Hausdorff's excess) and diameters
				\item Definition of set distances (gap) and of a variety / map / atlas and differential homeomorphism
			\end{itemize}
		\item \textbf{April 2004}
			\begin{itemize}[noitemsep]
				\item Proof of the Guldin's theorem
				\item Proof of König's theorem of kinetic energy and angular momentum
				\item Presentation and proofs of various techniques for calculating the moments of inertia: Huygens-Steiner theorem, polar inertia moment, inertia tensor, generalized Huygens-Steiner
				\item Definition of "power" and "performance" and proof of the calculation of the power of a rotating machine
				\item Proof of Boltzmann thermodynamic entropy law and the following statistical distributions: Maxwell speeds, Maxwell-Boltzmann, Fermi-Dirac, Bose-Einstein
				\item Proof of some principal moments of inertia of the following bodies: torus, spheres, cones, rectangular plate, tube
				\item Introduction to wave optics: Huygens principle, proof of Malus law, development of Fraunhofer model in the case of a rectangular slot. 
				\item Definition and proof of the resolving power of a simple rectangular slot.
				\item Proof of the origin and the solution of the no less famous Bessel differential equation of order $n$
				\item Proof of the wave function of a stretched rope and a stretched circular membrane
				\item Proof of Planck's law and of known approximations (first Wien's Law, Rayleigh-Jeans law).
				\item Proof of the displacement law (second Wien's law) and the Stefan-Boltzmann law through Planck's law and determination of the Stefan-Boltzmann constant
				\item Study of the origin of Planck dimensions: Planck length, Planck mass, Planck density, Planck time, Planck energy
			\end{itemize}
		\item \textbf{July 2004}
			\begin{itemize}[noitemsep]
				\item Proof of the physical origin of heat
				\item Proof of Torricelli's theorem
				\item Proof of Venturi effect
				\item Proof of Poiseuille's law
				\item Proof of relativistic Compton effect
				\item Proof of the existence of the fossil radiation in the Universe
				\item Introductions to quotients sets (in this case $\mathbb{Z}/n$
				\item Proof of the physical origin of heat
				\item Proof of Torricelli's theorem
				\item Proof of Venturi's effect
				\item Proof of Poiseuille's law
				\item Proof of relativistic Compton effect
				\item Proof of the existence of the fossil radiation in the Universe
				\item Introductions to quotients sets (in this case $\mathbb{Z}/nZ$)
				\item Proof of Lorentz transformations for speed and acceleration
				\item Determination of the relativistic Lagrangian of a free system
				\item Proof and definition of Ricci theorem, of the covariant derivative, of the Ricci identity, of the Riemann-Christoffel tensor, of the Ricci tensor, of the Ricci scalar, identities of both Bianchi and finally Einstein tensor
				\item Mathematical definition of the capacity and determination of the expression of a  parallel plane capacity
				\item Determination of the electrostatic potential energy
				\item Proof of the value of the field of and electric potential of a straight infinite wire.
				\item Determination of the basic properties of electric dipoles as the rigid dipole moment, the presentation of the induced dipole moment, hydrogen bonds, Van der Waals forces, etc.
				\item Definition of the Curie symmetry principle and Statement of 6 resulting properties
				\item Definition of pseudo vectors
				\item Proof of the relations for the volume of the pyramid, the paraboloid, the tetrahedron,  ocatahedron, cube and parallelepiped
				\item Determination of the magnetic field produced by a toroidal solenoid, a rectilinear solenoid infinity and a current loop
				\item Determination of the magnetic field produced by a magnetic dipole and basic definitions of the properties of magnetic materials
				\item Calculation of the Larmor radius and cyclotron pulsation in a non-relativistic framework
				\item Determination of the Lagrangian of the electromagnetic field and by extension in the non-relativistic approximation of the tensor of the electromagnetic field
				\item Introduction to the calculation of the radiation emitted by an accelerated charge (synchrotron radiation, Lienard-Wiechert's retarted potentials)
				\item Calculation of the values of resistors and capacitors in series
				\item Difference between electrical potential and electromotive potential
				\item Proof of Faraday's law and definition of "self inductance"
				\item Proof of Descartes formulas for the concave and convex spherical surfaces and refracting/not refracting as well as for refractive lenses.
				\item Definition of stigma and proof that the parable is strictly stigmatic
				\item Proof of Descartes formulas for thin lenses and conjugation law
				\item Definition of diopter and explanation of various visual disabilities
			\end{itemize}
		\item \textbf{July 2004}
			\begin{itemize}[noitemsep]
			\item Statement of Mach's principle
			\item Presentation of the photoelectric effect and proof of the physical law governing it.
			\item Proof by example that the light can be seen as both a particle or as a wave
			\item Proof of the theorem of monotone class
			\item Detailed proof of generalized Klein-Gordon equation (relativistic particle in a magnetic field) 
			\item Detailed proof of free Dirac equation with explicit Paulis solutions (particles, antiparticles)
			\item Determination of the radius of the atom using the Rutherford-Coulomb scattering(verbatim: determining the cross section for Rutherford)
			\item Presentation of macroscopic and microscopic interactions of X and gamma-rays with matter (which includes details study of the materialization of a photon in electron-positron pair)
			\item Introduction to spinors
			\item Definition of operating properties of matrices, remarkable matrices, determinants, eigenvectors and eignevalues
			\item Statements of the postulates of wave quantum physics
			\item Determination of the orbitals of the hydrogen-atom
			\end{itemize}
		\item \textbf{November 2004}
			\begin{itemize}[noitemsep]
				\item New Biographies (Smith, Say, Malthus, Keynes, Walras, Pareto)
			\item Presentation and proof of Noether's theorem
			\item Enumeration of some major physical chemical, astronomical constants
			\item Introduction to the theory of speculation: (predictive expectation of a financial asset)
			\item Introduction to the preference theory (Arrow-Debreu model)
			\item Presentation of solutions of the Black \& Scholes equation and remarks on the delta - Proof of the Call-Put parity equation
			\item Determination of initial stock (optimum) within the framework of the supply chain management
			\end{itemize}
		\item \textbf{January 2005}
			\begin{itemize}[noitemsep]
				\item New Biographies (Penrose, Hawking, Turing, Marx) 
				\item Development of "dual" version of Maxwell equations and proof of the origin of the expression (hypothesis) of magnetic monopoles 
				\item Definition of P, NP and NPC class algorithms
				\item Proof of the fundamental theorem of calculus also named "fundamental theorem of integral and differential calculus" 
				\item Presentation of the cGH cube and the Copenhagen interpretation 
				\item Introduction to the mathematical concepts of artificial neural networks 
				\item Introduction to the mathematical concepts of Genetic Algorithms
				\item Detailed mathematical introduction to fractals
				\item Basics maths stuffs on Quantum Computing
				\item Introduction to fuzzy logic
				\item Proof of Shannon theorem, Morgan theorems, expansion theorem, Karnaugh maps, complete adder, full substractor
			\end{itemize}
		\item \textbf{April 2005}
			\begin{itemize}[noitemsep]
				\item Basic definitions on block codes, linear codes, systematic codes in error correcting codes
				\item Proof of the relation of the relativistic change in mass
				\item Introduction to codes and prefix codes
				\item Proof of the formula for the calculation of the number of days between two given dates 
				\item Rounding calculations techniques
				\item Definition of rigid or non-rigid post and praenumerando annuities with or without constant rate (certain future)
				\item Definition and study of the properties of loans repayment or constant annuity
			\end{itemize}
		\item \textbf{April 2006}
			\begin{itemize}[noitemsep]
				\item Presentation of Schild's criterion via the Einstein effect (gravitational redshift)
				\item Development of the Newtonian approximation of the geodesic equation
				\item Definitions and proofs of developed forms of four-vectors of displacement, velocity, current, acceleration and energy-momentum
				\item Proof of the provenance of electromagnetic field tensor and referential calculation changes
				\item Definition of a tautology and principle of non-tautology
				\item Descriptions, definitions and many proofs on quaternions
				\item Proof of Euler's number irrationality
				\item Definition of the log-normal, triangular and Weibull distribution and proof of their mean and standard deviation
				\item Introduction to error calculation (absolute and relative uncertainties, error propagation, significant numbers, etc.)
				\item Proof of the deflection of light near a Star with the Newtonian gravitational model
				\item Definition of a rotation matrix (and developments related thereto)
				\item Proof of the existence of the Euclidean division in the ring of polynomials
				\item Definitions of MWRR (Time of Money Weighted Return) and TWRR (Time Weighted Rate of Return)
				\item Proof of Gauss-Ostrogradsky theorem 
			\end{itemize}
		\item \textbf{July 2006}
			\begin{itemize}[noitemsep]
				\item Definition of the concept of dual space
				\item Definition of Pareto law and proof of mean and standard deviation
				\item Definition of quantile (quartile, percentile)
				\item Proof that the mode is the value that minimizes the absolute dispersion
				\item Proof of Minkowski and Bienaymé-Tchebychev inequalities
				\item Proof of the weak law of large numbers 
				\item Proof of Euler's formula for planar graphs
				\item Introduction to the mathematical method of Six Sigma process controls
				\item Proof of the variational calculus theorem
				\item Proof of the calculation of the apparent superluminal speed of a high Redshift Star
				\item Zero-sum game resolution method using operational research
				\item Definition of investment funds
				\item Proof of the beta expression of a simple linear regression model (Sharpe)				
			\end{itemize}
		\item \textbf{October 2006}
			\begin{itemize}[noitemsep]
				\item Proof Binomial and Poisson likelihood estimators
				\item Presentation of the concept of "color" and subtractive and additive synthesis
				\item Proof of Einstein equation fields (approach using weak fields approximation)
				\item Differentiation of the equivalence principle, weak equivalence principle and Einstein's equivalence principle
				\item Proof of the duration of the Daytime arc of planets in the approximation of the null precession and nutation
				\item Digital and formal study (approximative) of Lagrange points of a binary system
				\item Method of resolution of 4th degree polynomial (Ferrari's method)
				\item Presentation of the Gram determinant via the Euclidean volume represented by the joint product of the vectors of a canonical basis
				\item Definition of monotonic, strictly monotone functions, etc .. without pure formal approach
				\item Approximative determination via the Yukawa potential (mass fields) of the mass of mesons of the weak interaction and the strong nuclear interaction.
				\item Introduction to first order linear differential equations			
			\end{itemize}
		\item \textbf{December 2006}
			\begin{itemize}[noitemsep]
				\item New Biographies (Heckman, McFadden, Tesla)
				\item Bernoulli numbers and polynomials
				\item Roche Limit
				\item Flattening of celestial bodies
				\item Pressure and kinetic temperature
				\item Proof of a special case of magnet or electromagnet force
				\item Fundamental introduction to Hermitian and Hilbert vector spaces 
				\item Schwarzschild solution in General Relativity
				\item Precession of the perihelion of Mercury in General Relativity
				\item Light Deflection in General Relativity
				\item Shapiro delay in General Relativity
				\item Evaluation Model of Financial Assets
				\item The Black Hole Universe
				\item Coupling constants of fundamental interactions
				\item Hamiltonian of the Schrödinger equation for a charged particle in an electromagnetic field
				\item Markov chains
				\item Theory of queues
				\item Introduction to weather and marine engineering mathematics
				\item Practical example in Microsoft Excel of the efficiency Markowitz model
				\item Practical example in Microsoft Excel of the Sharp efficiency model
				\item Simplified proof of Green(-Riemann)'s and Stokes theorem
				\item Detailed proofs on principal component analysis (PCA)
				\item Proof of the spectral theorem in the real numbers case
				\item Detailed proofs on logistic regression
				\item Proof of Rolle's theorem and mean value
				\item Proof of the Hopital theorem and generalized finite increments (generalized mean value)
				\item Introduction of Geometric law and proof of its variance and mean
				\item Proof of calculation of the area and volume of the five regular Platonic polyhedra
				\item Introduction to field algebra and geometry
				\item Detailed calculations of the collapsing critical mass (Jeans' Mass) and critical radius (Jeans' Radius) of an interstellar cloud or stellar nurseries
				\item Detailed calculations of the collapse time of an interstellar cloud
				\item Detailed calculations of a Star nuclear life
				\item Detailed mathematical introduction to Fourier transforms
				\item Detailed resolution of homogeneous linear differential equations with constant coefficients
				\item Presentation of the Cornu's spiral for civil engineering
				\item Mathematics of biometric functions
				\item Determination and simple resolution of the Pauli equation
				\item General solution (Fourier transform) of the electromagnetic wave equation
				\item Introduction to strength of materials
				\item Visual Horizon
				\item Growth rate of a population depending on the temperature
				\item Pitot Tube and Pressure drop
				\item $U(1)$ gauge theories in quantum field physics
				\item Bezier curves
				\item Differential equation system with matrix exponentiation
			\end{itemize}
		\item \textbf{September 2008}
			\begin{itemize}[noitemsep]
				\item New Biographies (Pearson, Gosset, Fisher)
				\item New important usual primitives for civil engineering and analytical mechanics
				\item Detailed proofs of the origin of arcsinh and arccosh functions
				\item Proof of Laplace equation and Mayer relation
				\item Proof of the propagation of pressure waves
				\item Historical introduction for nuclear physics
				\item Proof of Maxwell relations in thermodynamics and introduction to free energy and enthalpy
				\item Adiabatic atmosphere model
				\item Proof of Lorenz equations and the butterfly effect
				\item Calculations of some optical properties of the prism
				\item Conditions of decoherence/interferences of electromagnetic waves
				\item Bloch Sphere
				\item Proof of the minimum surface of revolution volume
				\item Treatment of the free particle
				\item Treatment of the polarized spin qubit and the qubit of spin 1/2
				\item Characteristic function and central limit theorem
				\item Some proofs on the inequalities in triangles
				\item Proof of the volume of a barrel with circular section 
				\item Proof of the origin of the mean and standard deviation ot the Student and Fisher-Snedecor laws
				\item Introduction to the marginal cost
				\item Statistical test of the one-way ANOVA
				\item Statistical Pearson's Chi-square adjustment test  
				\item Introduction to the analysis of variance of regression
				\item Correspondence Factorial Analysis
				\item Development of linear free or forced RC, RL, RLC circuits
			\end{itemize}
		\item \textbf{September 2008}
			\begin{itemize}[noitemsep]
				\item Simple or complex topology systems analysis for preventive maintenance
				\item Introduction to finite and infinite continued fractions
				\item Detailed Solutions of a rectangular tunneling barrier
				\item Mathematical model of alpha decay via tunneling
				\item Introduction to Brownian motion model according to Langevin model
				\item Introduction to tribology/friction
				\item Introduction to time series analysis
				\item Additional proofs on long term and short term process capability indices and measuring devices in Statistical Process
				\item Proof of the calculation of the PPM in SPC centered or non-centered process
				\item Proof of Taguchi quality cost relation
				\item Proof of Lienard-Wiechert expression of electric and magnetic potentials 
				\item Introduction to complex analysis
				\item Proof of the second Friedmann equation in cosmology
				\item Proof of "slowing" of light near a Black Hole
				\item Proof of the expression of the Taylor expansion of a function of two real variables
				\item Introduction to experimental design (DoE)
				\item Proof of Eherenfest's theorem
				\item Proof of the likelihoos estimators of the Weibull distribution with two parameters
				\item Example of application of decision theory
				\item Bands theory (parabolic and semi-classical approximation) within semiconductors
				\item Theorem of residues and Laurent series
				\item Proofs of Lean Six Sigma values for business processes/workflows
			\end{itemize}
		\item \textbf{October 2011}
			\begin{itemize}[noitemsep]
				\item New Biography (Erlang, Hotelling)
				\item Detailed calculation of the geostationary orbit
				\item Calculations on PVC meteorological probe balloons
				\item Developments on the symmetrical gyroscope and weighing router
				\item Detail calculations and proof on Gini index
				\item Secular balance, transient and non-equilibrium in radioactive filiation
				\item Detailed calculations on the approximate radius of rapidly rotating Stars
				\item Proofs on historical delta-normal and variance-covariance Value At Risk
				\item Two new jokes in the Humor section
				\item Empirical statistical model of wage control
				\item Simplified proof of the possible absence of arbitrage opportunity in Finance
				\item Calculations on self-financing portfolio on the underlying risk
				\item Introduction to mathematical techniques in Insurance
				\item Mathematical developments on the power and intensity of a longitudinal sound wave
				\item Statistical bilateral Z-test on the difference in the two means
				\item Statistical Student T-test on two paired sample means
				\item Proof of statistical confidence interval relation of sample large proportions
				\item Statistical test for equal proportions of two large samples
				\item Application of Shannon's theorem to calculate a statistical index of diversity
				\item Proof of the determination of coefficients of a multiple linear regression
				\item Proof of the determination of the coefficients of a simple linear regression through the origin
				\item Introduction to sensitivity analysis
				\item Introduction and some proof on rank/order statistics
				\item Demonstration of the provenance, hope and variance of the negative binomial distribution
				\item Control charts with detailed mathematical proofs
				\item Mathematical approach of first Google Page Rank algorithm
				\item Proof of Beltrami's identity to simplify the Euler-Lagrange equation
				\item Exact binomial statistical test for the balance of a population with two characteristics
				\item Developments and study of gravity waves in a fluid
				\item Some simple developments on the gears/gear shafts
				\item Proof of skin's effect
				\item Theory of the rainbow
				\item Theory of double pendulum
				\item Boltzmann distribution law
				\item Dalton's and Amagat laws
				\item Heat Flow
				\item Average power in alternative current
				\item Presentation of some detailed calculations on the betatron
			\end{itemize}
		\item \textbf{May 2013}
			\begin{itemize}[noitemsep]
				\item New Biographies (Napier, Wilcoxon, Born, Heisenberg, Jordan, Kolmogorov, Stokes, Ostrogradsky, Zeeman, Joseph, Faraday, Meitner, Curie, Biot, Debye, Drude, Ohm)
				\item Detailed example of construction of a particular neural network with Microsoft Excel
				\item Resolution of homogeneous linear differential equations of order 1 with non constant coefficients
				\item Example of a Fourier transform of a Gaussian function and proof of the property of the Fourier transform of a derivative
				\item Introduction to interactions in two-factor ANOVA
				\item Confidence interval and prediction interval of a linear regression
				\item Statistical Test signs (median test)
				\item Introduction of conditional probability and the conditional mean of Pareto law
				\item Determination of the estimators of the gamma distribution using the method of moments
				\item Second mathematical approach to the identification of the tides
				\item Statistical Kolmogorov-Smirnov adjustment test with Lilliefors approach
				\item Proof of the Scheafer's quota model
				\item Proof of the calculation of the synodical period of the planets and of demotion of time
				\item Proof of the construction of Brownian bridges
				\item Proof of the origin of the rare events control chart
				\item Calculation of the discount factor of a retiring insurance based on inflation and life expectancy
				\item Proof of two-factor ANOVA relations without repetition and repetition
				\item Mathematical proof and physics of the LASER
				\item Theorem of Taylor series with integral rest
				\item Stability of the Poisson distribution
				\item Statistical Poisson test for 1 or 2 samples
				\item Statistical non-parametric Kruskal-Wallis and Friedman tests
				\item Statistical normality test of Ryan-Joiner
				\item Statistical C Cochran test
				\item Usual Taylor-Maclaurin Series
				\item Polynomial regression by least squares method
				\item Spatiotemporal Finite Difference Method with Maxwell's equations
				\item Break-even Analysis
				\item Mechanical Harmonic Oscillator
				\item Acoustic Doppler Effect
				\item Periodic waves overlays
				\item Statistical Tukey test of range
				\item Forecasting models by moving average, seasonal coefficients, simple smoothing, Brown double smoothing, double Holt (additive) smoothing, double Holt and Winter (multiplicative) smoothing
				\item Introduction to elementary  AR, AM, ARMA and ARIMA autoregressive models
				\item Proof of the expression of the correction factor on finite population
				\item Statistical Fisher exact test
				\item Exponential Laplace Smoothing
				\item Cohen approval Kappa and McNemar statistical test
				\item Kaplan-Meier survival analysis model
				\item Cramer's V
				\item Clustering $K$-Means algorithm
				\item Clustering Dendrograms algorithm
				\item Statistical Wilcoxon signed rank test for one sample or two paired samples
				\item Quantitative study of the effective potential energy (harmonic model of the atomic bonding) of the hydrogen-atom
				\item Proof of beams equation (Euler-Bernoulli equation)
				\item Calculation of the failure rate of a system using the technique of maximum likelihood
				\item We added a chronology of Sciences
				\item Proof of Einstein model (Dulong-Petit law) of the heat capacity of crystalline solids and derivation of the Debye model
				\item Langenvin Model of diamagnetism and paramagnetism
				\item Introduction to line integrals calculation
				\item Naive determination of the energy of a magnetic dipole
				\item "Liquid drop" nuclear Model 
				\item Magnetic model of spin resonance
				\item Integrating factor method for solving differential equations
				\item Constant variation method for solving differential equations
				\item Mendel's law
				\item Temporary and Deferred Life Annuity
				\item Carnot Cycle
				\item Durand and Gordon-Shapiro equity valuation model 
				\item Statistical Anderson-Darling adequation test
				\item Non-linear optimization by the Newton-Quadratic and Gauss-Newton methods
				\item Lagrange Polynomial interpolation method
				\item Statistical Cochran-Mantel Heanzel test
			\end{itemize}
		\item \textbf{November 2016}
			\begin{itemize}[noitemsep]
				\item Singular value decomposition theorem
			\end{itemize}
		\item \textbf{December 2016}
			\begin{itemize}[noitemsep]
				\item Fieller's test (ratio of two means)
			\end{itemize}
		\item \textbf{January 2017}
			\begin{itemize}[noitemsep]
				\item Falling Chimney problem
				\item Levey-Jennings control charts
				\item Simple lattice mixture design of experiment with process variables
				\item Average Failure Rate (reliability)
				\item Markov Chain Reliability Model
				\item Design of reliability tests (Chi-squared time of test, Binomial sampling size, Beta-binomial sampling size)
				\item Weibull distribution linearization
				\item Inverted pendulum
				\item Durbin-Watson autocorrelation test
				\item Fisher's method for multiple $p$-values
				\item Magnifying glass
				\item Laney's control chart
				\item Classification of conical by the determinant	
				\item Classification of partial differential equations
			\end{itemize}
		\item \textbf{February 2017}
			\begin{itemize}[noitemsep]
				\item Folded Normal distribution basics CDF and PDF
				\item Half-Normal distribution CDF, PDF, variance, mean and median
				\item Telescopic and Gandi's series
				\item Césaro's sum
				\item Implicit Differentiation
				\item Bivariate chain rule
			\end{itemize}
		\item \textbf{March 2017}
			\begin{itemize}[noitemsep]
				\item Laplace Integration method
				\item Lenth's PSE Pareto Margin Error for unreplicated factorial designs
				\item Pareto Margin Error for replicated factorial designs
				\item Desing of Experiments desirability
			\end{itemize}
		\item \textbf{April 2017}
			\begin{itemize}[noitemsep]
				\item Friedmann–Lemaître–Robertson–Walker metric
				\item Jensen inequality
			\end{itemize}
		\item \textbf{May 2017}
			\begin{itemize}[noitemsep]
				\item Introducing weak field gravitational waves equation
				\item A mathematical approach of "Divide and rule?" in management
				\item Three new jokes in the Humor section
			\end{itemize}
		\item \textbf{July 2017} (v3.7 $\rightarrow$ v3.8)
			\begin{itemize}[noitemsep]
				\item Correction of many typing errors (almost 50)
				\item Relation 11.3.221 had a integral symbol  (that should not be there)
				\item Relation 7.1.72 had a wrong variable for the angle
				\item Relation 7.5.56 there was a $|a|$ instead of a $|b|$ in the second equality
				\item Relation 11.1.35 had a $4$ that has nothing to do there at the numerator
				\item Relation 15.7.905 there was an alignement issue
				\item Relation 15.7.888 had a $\Lambda$ instead of a $\Gamma$ at the denominator
				\item Relation 11.1.20 had a missing $T$
				\item Relations 11.3.169 must be $r_E$ instead of $r-E$ (LaTeX typing error...)
				\item Table 8.10 and 8.11 had wrong units in the header row
				\item New photo about a quasi vertical circular rainbow and a photo a nuclear magnetic resonance machine
				\item There was a typing error in the period (T) relation of the simple pendulum
				\item Details of the backpropagation method for neural network
				\item New mathematical details on the simple linear regression leverage
				\item Numerical applications of some General Relativity experimental tests
				\item Derivation of the geodesic of the sphere (as example in the section of Analytical Mechanics)
				\item ZeroR classification method
				\item K nearest neighbors classification method
				\item Definition of Convexity/Concavity of a function (for Jensen inequality proof)
				\item Proof of  Hermite polynomial Orthogonality
				\item Derivation of the Bachelier Option Pricing Model
				\item Derivation Forward/Future Cox-Ross-Ingersoll Equality
				\item Derivation of the Stress-Energy tensor for a non-relativistic perfect fluid
				\item Derivation of the orthodromic distance
				\item Derivation of the area of a section of an ellipse
				\item There was twice a "Invariance by translation in space" Noether theorem... one of them was obviously "Invariance by translation in time"
				\item Schwarzschild innermost stable orbit
				\item Hafele–Keating experiment with General Relativity treatment
				\item Introducing hyperparameters in Machine Learning
				\item Kernel smoothing
				\item Credit Default risk
				\item We added a lot of new entries in the Chronology section
			\end{itemize}
		\item \textbf{November 2017} (v3.8 $\rightarrow$ v3.9)
			\begin{itemize}[noitemsep]
				\item Section numbering was not correct anymore because of a MiKTeX update.
				\item We created all simple cross references
				\item Multiple errors in PPI (proton-proton Sun chain fuction reaction) in relations 11.2.27, 11.2.29 and 11.2.30
				\item We added one point to the scientific publication rules (cite equivalent studies for meta-analysis)
				\item Correction of a few hundred minor typing errors
				\item Typing error in the relation 5.4.231, the first line should be a $\cos(x)$ and not a $\sin(x)$
				\item Read "multiplication" instead of "addition" in the remark below the relation 4.2.164.
				\item A square was missing on the $2$ in 4.4.13.
				\item A square was missing on the $r$ in 12.1.57
				\item There was a "$+$" instead of a $\cdot$ in 10.2.21
				\item Square was missing in relation 4.7.1479, 4.7.1480, 4.7.1481, 4.7.1482. 4.7.1484
				\item There was a $c^3$ instead of a $c^2$ in the relation 11.1.212
				\item At 8.3.450 three paragraphs and two equations were duplicates
				\item Proof of the derivative of $f(g)^{g(x)}$
				\item OneR (One Rule) in Data Mining with confusion matrix rule selection
				\item Apriori Association Rule
				\item Kurtosis and Skewness
			\end{itemize}
		\item \textbf{January 2018} (v3.9 $\rightarrow$ v3.10)
			\begin{itemize}[noitemsep]
				\item Typing error in relation 8.2.160 (a $\partial$ missing)
				\item Typing error in relation 9.3.305 (a $\mu$ was missing)
				\item There was an error sign in all inverse Fourier transfom (error made during the translation of the book into English).
				\item Dirac continuity equation
				\item CRS model of Data Envelopment Analysis (DEA)
				\item Kullback-Leiber divergence
				\item Continuous and Discrete Linear Convolution
				\item Dirichlet Integral
				\item Permutation tests
				\item Binomial and Gaussian naive bayes
				\item More cross-references added
			\end{itemize}
			\item \textbf{February 2018} (v3.10 $\rightarrow$ v3.11)
			\begin{itemize}[noitemsep]
				\item Typing error for the return angle range of the function $\mathrm{atan()}$ ($-$ instead of a $+$).
				\item There was a typing error in relations 4.7.264 and 4.7.265 (power of $3$ instead of $2$).
				\item There was a typing error in relation 4.7.512 a $\text{E}(X)$ instead of a $\text{V}(X)$.
			\end{itemize}
	\end{itemize}

\chapter{Nomenclature}

This chapter contains a summary with simple description of all symbols used in this book.

	\begin{table}[H]
	\centering
	\begin{tabular}{*8l}
	$\alpha$ \verb?alpha? &$\theta$ \verb?theta? & o o &$\tau$ \verb?tau? \\
	$\beta$ \verb?beta? &$\vartheta$ \verb?vartheta? &$\pi$ \verb?pi?         &$\upsilon$ \verb?upsilon? \\
	$\gamma$ \verb?upsilon? &$\xi$ \verb?xi?  &$\varpi$ \verb?varpi? &$\phi$ \verb?phi?  \\
	$\delta$ \verb?delta? &$\kappa$ \verb?kappa? &$\rho$ \verb?rho? &$\varphi$ \verb?varphi? \\
	$\varepsilon$ \verb?epsilon? &$\lambda$ \verb?lambda? &$\varrho$ \verb?varrho? &$\chi$ \verb?chi?  \\
	$\varepsilon$ \verb?varepsilon? &$\mu$ \verb?mu? &$\sigma$ \verb?sigma? &$\psi$ \verb?psi? \\
	$\zeta$ \verb?zeta? &$\nu$ \verb?nu? &$\varsigma$ \verb?varsigma? &$\omega$ \verb?omega? \\
	$\eta$ \verb?eta?\\
    \\
	$\Gamma$ \verb?Gamma? &$\Lambda$ \verb?Lambda? &$\Sigma$ \verb?Sigma? &$\Psi$ \verb?Psi? \\
	$\Delta$ \verb?Delta? &$\Xi$ \verb?Xi? &$\Upsilon$ \verb?Upsilon? &$\Omega$ \verb?Omega?\\
	$\Theta$ \verb?Theta? &$\Pi$ \verb?Pi? &$\Phi$ \verb?Phi?
	\end{tabular}
	\caption{Greek letters}\label{greek}
	\end{table}

	and the mathematical operators and objects used in the book:
	\begin{itemize}[label={},leftmargin=0.5cm]
		\setlength{\itemsep}{1pt}
  		\item $($ Open parenthesis
  		\item $)$ Close parenthesis
  		\item $[$ Open bracket
  		\item $]$ Close bracket
  		\item $\therefore$ Therefore
  		\item $\because$ Because
	 	\item $\varnothing$ Empty Set
	 	\item $=$ Equality symbol
	 	\item $:=$, $\triangleq$ By definition symbols
	 	\item $>$ greater than
	 	\item $<$ less than
	 	\item $\gg$ much greater than
	 	\item $\ll$ much smaller than
	 	\item $\leq$ less than or equal to
	 	\item $\geq$ bigger or equal to
	 	\item $\succ$ preferred to (for utility in econometry)
	 	\item $\prec$ not preferred to (for utility in econometry)
	 	\item $\succeq$ preferred or equal to (for utility in econometry)
	 	\item $\preceq$ not preferred or equal to (for utility in econometry)
	 	\item $\sim$ equivalent to (for utility in econometry)
	 	\item $\mathbb{N}$ Natural Numbers set (positive integers)
	 	\item $\mathbb{Z}$ Relative Numbers set (all integers)
	 	\item $\mathbb{Q}$ Rational Numbers set (ratio of relative numbers
	 	\item $\mathbb{R}$ Real Numbers set
	 	\item $\mathbb{C}$ Complex Numbers set
	 	\item $\Re$ Real part of a complex number
	 	\item $\Im$ Imaginary par of a complex number
	 	\item $\aleph$ Transfinite Cardinal symbol
	 	\item $\wedge$ AND operator, noted \& in computer science and corresponding to multiplication in maths
	 	\item $\equiv$ Identity symbol (left term is assumed to be equal to the right one and vice-versa)
	 	\item $\cong$ Approximately equal symbol
	 	\item $\propto$ Linear proportional symbol 
	 	\item $\in$ Symbol that means left term belongs to the right term
	 	\item $\not\in$ Symbol that means left term does not belong to the right term
	 	\item $\subset$ Symbol that means left term (that is a ) is a subset of the set on the right 
	 	\item $\not\subset$ Symbol that means left term (that is a set) is not a subset of the set on the right
	 	\item $\subseteq$ Symbol that means left term (that is a set) is a subset or a set equal to the set on the right
	 	\item $\not\subseteq$ Symbol that means left term (that is a set) is not a subset or even a set equal to the set on the right
	 	\item $\cup$ Symbol that means left term (that is a set) is merged (union) with right term that is also a set
	 	\item $\sqcup$ Symbol that means left term (that is an interval) is merged (union) with right term that is also an interval
	 	\item $\cap$ Symbol that means we take only intersection (equal) items of left and right terms that are sets
	 	\item $\displaystyle \bigcup_{i=1}^n$ Union of multiple sets
	 	\item $\displaystyle \bigcap_{i=1}^n$ Intersection of multiple sets
	 	\item $\mid$ Such that...
	 	\item $\forall$ For all...
	 	\item $\exists$ It exists...
	 	\item $+$ Addition symbol of two terms
	 	\item $\displaystyle \sum_{i=1}^n$ Summation of multiple terms
	 	\item $-$ Subtraction symbol of two terms
	 	\item $\times, \cdot$ Multiplication (product) symbol of two terms
	 	\item $\times$ If left and right terms are vectors, this is the cross product (vectorial product)
	 	\item $\circ$ dot product, also named "inner product" a scalar product
	 	\item $\otimes$ tensor product
	 	\item $\displaystyle \prod_{i=1}^n$ Multiplication (product) symbol of multiple terms
	 	\item $\displaystyle\int$ Riemann primitive
	 	\item $\displaystyle\int\limits_a^b$ Riemann integral in range $[a,b]$
	 	\item $\displaystyle\oint$ Closed non-oriented curvilinear integral (line integral)
	 	\item $\displaystyle\ointclockwise$ Clockwise path integral
	 	\item $\displaystyle\ointctrclockwise$ Counterclockwise path integral
	 	\item $\div, \backslash$, $:$ Symbols for division for two terms depending on school level
	 	\item $:$ When the left term is a matrix at the right one a vectros, this is "Frobenius (matrix) dot product" or "Frobenius inner product"
	 	\item $P$ Depending on the context this is a Probability, Cumulated Probability or Part of a set
	 	\item $\text{E}$ In statistics the expected mean
	 	\item $\text{V},\sigma^2$ In statistics the variance
	 	\item $\hat{x}$ In physics the amplitude of $x$, in statistics an estimator of $x$
	 	\item $C_k^n,\begin{pmatrix}n\\k\end{pmatrix}$ is the binomial coefficient define by $n!/(k!(n-k)!)$
	 	\item $\mathds{1}$ Unity matrix (diagonal with $1$, $0$ everywhere else)
	 	\item $\ln$ Natural logarithm (base $e$) of a number
	 	\item $\log$ Base $10$ (if not indicated) logarithm of a number
	 	\item $\earth$ Symbol in astronomy and astrophysics to refer to the Earth
	 	\item $\odot$ Symbol in astronomy and astrophysics to refer to the Sun
	\end{itemize}


	%\chapter{About the Redactor}
	%\input{Chapter_AboutRedactor.tex}


	\cleardoublepage
	\phantomsection
	\addcontentsline{toc}{chapter}{List of Figures}
	\listoffigures

	\newpage\null\thispagestyle{empty}\newpage %création d'une nouvelle page en forcant la disparition du numéro de page	
	\phantomsection
	\addcontentsline{toc}{chapter}{List of Tables}
	\listoftables
	
	\newpage\null\thispagestyle{empty}\newpage %création d'une nouvelle page en forcant la disparition du numéro de page	
	\phantomsection
	\addcontentsline{toc}{chapter}{List of Algorithms}
	\listofalgorithms

	\newpage\null\thispagestyle{empty}\newpage %création d'une nouvelle page en forcant la disparition du numéro de page	
	\phantomsection
	\addcontentsline{toc}{chapter}{Bibliography}
	\defbibnote{myprenote}{Since this work is a reference, we have opted to exclude any reference in the text. So we simply gathered a final bibliography list according to a ranking of use and unassuming it's completeness.

The books and documents listed below are what I consider as the best references (thus excluding the free PDFs) which have been consulted for the preparation of this book and who is indebted to many high quality borrows. Their reading can be as profitable for all who wish to improve, deepen and broaden their knowledge. 

Numerous references, however, are no longer available on the market and it is also necessary that the reader remembers that each entry in the table by itself refers to dozens of other books, it is also impossible to have an exhaustive and objective list (that would be a vicious cycle) of all high-quality books existing in the areas covered by Sciences.ch.

And here is the complete list:}

	\nocite{*}
	\printbibliography[prenote=myprenote]
	
	+ a thousand of low quality and incomplete scientific papers.
	
	\phantomsection
	\cleardoublepage
	\addcontentsline{toc}{chapter}{Index}
	\printindex  
	
	\chapter{Donate}
	The writing of this book took until now $15$ years and huge effort and sacrifices, so if you find this book useful all donations are greatly appreciated and for this purpose you can use the below link depending on your preferences:
	
	\begin{center}
	\href{http://www.sciences.ch/htmlfr/donate.php}{\includegraphics[scale=0.14]{img/paypal.jpg}} $\qquad$ \href{https://www.patreon.com/sciences}{\includegraphics[scale=0.20]{img/patreon.jpg}} $\qquad$ \href{https://www.tipeee.com/elements-of-applied-mathematics}{\includegraphics[scale=0.023]{img/tipeee.jpg}} $\qquad$ \href{http://ko-fi.com/operamagistris}{\includegraphics[scale=0.34]{img/kofi.jpg}}
	\end{center}
	\begin{center}
		{\large \faBitcoin} 4248d58b-90f0-493e-8114-8dc9f8e5b492
	\end{center}
	The donations will be used mainly for the following purposes:
	\begin{itemize}
		\item Pay professionals to finish the translation into English
		\item Pay professionals for the proof reading
		\item Hire designers to draw high definition copyleft vector illustrations
		\item Hire \LaTeX{} professionals to improve all tables and headers design
		\item Hire photographs to make photo of technical installations
		\item Have more time to write the remaining $2,400$ pages and do continuous improvements
		\item To keep the book freely accessible worldwide and translate it in other languages
		\item Give \LaTeX{} sources for free
		\item Redistribute $1\%$ to OpenStax, $5\%$ to Wikipedia, and $10\%$ to TeXMaker and MiKTeX developers
	\end{itemize}
	There are also other ways of supporting this book to ensure it is maintained and well supported in the future! Linking/spreading the word, and submitting proofs contributions will all help.
	
	Thank you for your kind attention and support.
	
	\newpage\null\thispagestyle{empty}\newpage %Creation of a new empty page and force removal of page number
	\pagestyle{empty}
	\pagecolor{gray}
	\newgeometry{margin=2.5cm}
	{\Huge \textbf{Opera Magistris}}
	
	{\LARGE Elementary Applied Mathematics for Engineers}

	{\color{white}The purpose of this book and its associated PDF is to present to people who approach the study of Applied Mathematics, their basic concepts and to do so with some level of rigour (demonstrations are complete or at least pushed to the point where there can reasonably be judged to be), details and consistency in respect of writing conventions and with a maximum of pedagogy. This book is also born from the desire to present some scientific ideas and understand how they modestly affect our lifestyle, our way of thinking, working and their impact on our ecosystem and the unreasonable effectiveness of Applied Mathematics. Indeed, issues related to science are becoming increasingly important in our contemporary culture and represent a major challenge in terms of ethics, citizenship and development.

	This book is not intended to be a novel to read from start to the end. It is intended as a reference book (after all errors have been corrected and texts completed...) which, when simple questions arise, to find answers quickly using computer technology and for free. This book cannot (and does not) also claim to replace a formal school education with a teacher and numerous practical exercises. It can however be seen as a formulas reference (with proofs) or a relatively good theoretical complement to the preparation of various examinations.

	The view point that is adopted is that of the pragmatic engineer, eager to study mathematics, classical physics, econometrics, numerical analysis, statistics, relativistic mechanics, quantum physics, social mathematics, computer science, chemistry, etc. in a concise way and without wasting time in a formal and extravagant vocabulary and unusable in modern industry. From this point of view the concepts and methods presented are just some of the typical tools mathematical-physical (absolute minimum in the field). The specialist expert will find nothing new and the student who would be interested in a particular theory should know that every subject is unfortunately much larger than anything that can be discussed here so far... The ambition is not one we can have for students of a course of mathematics for whom the acquisition of tools is important. This makes possible a little formal style where one wishes to give evidence less complete than to intuitively understand the objects presented, to see them and own them.

	We dedicate this book to all those for which each answer is a question.

\begin{flushright}
\textit{Giving free access to the source of living water to the thirsty since 14 years}\end{flushright}}


\begin{flushright}
\includegraphics{img/ISBN.eps} 
\end{flushright}

\end{document}
