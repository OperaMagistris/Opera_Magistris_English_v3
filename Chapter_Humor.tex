If you have any scientific humorous stories do not hesitate to let us know. In all cases, we wish you a good time (some stories are in French as they lose their meaning in English).
	
\begin{center}
\textbf{This book is transmitted with 100\% recycled electrons}
\end{center}

	\section{Situations}
	
	An engineer, a physicist, and a mathematician are shown a pasture with a herd of sheep, and told to put them inside the smallest possible amount of fence. 

\begin{itemize}	 
	\item[$-$] The engineer is first. He herds the sheep into a circle and then puts the fence around them, declaring:"A circle will use the least fence for a given area, so this is the best solution." 

	\item[$-$] The physicist is next. She creates a circular fence of infinite radius around the sheep, and then draws the fence tight around the herd, declaring, "This will give the smallest circular fence around the herd."

	\item[$-$] The mathematician is last: After giving the problem a little thought, he puts a small fence around himself and then declares, "I define myself to be on the outside!" 
\end{itemize}
\begin{center}\underline{\hspace{5 cm}}\end{center}

Two men are sitting in the basket of a balloon. For hours, they have been drifting through a thick layer of clouds, and they have lost orientation completely. Suddenly, the clouds part, and the two men see the top of a mountain with a man standing on it.

\begin{itemize}	 
	\item[$-$] "Hey! Can you tell us where we are?!"
\end{itemize}

The man doesn't reply. The minutes pass as the balloon drifts past the mountain. When the balloon is about to be swallowed again by the clouds, the man on the mountain shouts: 

\begin{itemize}	 
	\item[$-$]  "You're in a balloon!"

	\item[$-$] "That must have been a mathematician."
\end{itemize}

The man astonished asks:

\begin{itemize}	 
	\item[$-$] "Why?"

	\item[$-$]  "Firt, he thought long and thoroughly about what to say. Second, what he eventually said was irrefutably correct. And last but not least... it was of no use whatsoever..."
\end{itemize}	
\begin{center}\underline{\hspace{5 cm}}\end{center}
	
	\begin{center}
		\includegraphics[scale=0.7]{img/humour/research_in_peace.jpg}	
	\end{center}
	
\begin{center}\underline{\hspace{5 cm}}\end{center}

An engineer, a mathematician, and a physicist are staying for the night in a hotel. Fortunately for this joke, a small fire out in each room.

\begin{itemize}	 
	\item[$-$] The physicist awakes, sees the fire, makes some careful observations, and on the back of the hotel's wine list does some quick calculations. Grabbing the fire extinguisher, he puts out the fire with one, short, well placed burst, and the crawls back into bed and goes back to sleep.

	\item[$-$] The engineer awakes, sees the fire, makes some careful obsrevations, and on the back of the hotel's romme service list does some quick calculations. Grabbin the fire extinguisher (and adding a factor safety of 5), he puts out the fire by hosing down the entire room server times over, and then crawls into his soggy bed and goes back to sleep.

	\item[$-$] The mathematician awakes, sees the fire, makes some careful observations, and on a blackboard installed in the room, does some quick calculations. Jubliant, he exclaims "A solution exists!", and crawls into his dry bed and goes back to sleep.
\end{itemize}	
\begin{center}\underline{\hspace{5 cm}}\end{center}

A doctor, a lawyer and a mathematician were discussing the relative merits of having a wife or a mistress.

The lawyer says: "For sure a mistress is better. If you have a wife and want a divorce, it causes all sorts of legal problems.

The doctor says: "It's better to have a wife because the sense of security lowers your stress and is good for your health.

The mathematician says: "You're both wrong. It's best to have both so that when the wife thinks you're with the mistress and the mistress thinks you're with your wife... you can do some mathematics."
\begin{center}\underline{\hspace{5 cm}}\end{center}

A prominent businessman hires a mathematician, a physicist and a computer scientist in order to win every horses competition.

The mathematician go on the task, it computes matrices without end, define axioms at any headland and after long weeks of lemmas, theorems and conjectures, he concludes that the problem is formally unhearable.

Then the computer scientist happy to see that the mathematician failed, is approaching its Cray III and after writing volumes of algorithms in C++ and introduced all the parameters and initial conditions joyfully announces that it will take just a few hundred years to calculate the result of each competition ...

The physicist, has a smile, he informs his distinguished colleagues that he has the solution. He approaches a blackboard and while drawing a sphere begins by saying: "approximate the horse by a perfect sphere..."
\begin{center}\underline{\hspace{5 cm}}\end{center}

During a job interview, an entrepreneur receives four engineers: one who followed the Military Polytechnic School of Paris, the second HEC, the third is computer engineer, and the last followed University. It explains the four candidates in the end, to run a business, you just need to count.

He therefore addresses the first of them, the polytechnician, and said, "go ahead, count ..."

\begin{itemize}	 
	\item[$-$] The polytechnician: "one... two... one... two..."\end{itemize}


The man surprised then asked the engineer who followed HEC: "To you! Count ..."

\begin{itemize}	 
	\item[$-$] The engineer of HEC: "one KiloDollar... two KiloDollar, three KD..."\end{itemize} 


He then turns anxiously toward the computer scientist: "To you! Count ..."

\begin{itemize}	 
	\item[$-$] The computer scientist: "0... 1... 0... 1..." \end{itemize}

Desesperated, the entrepreneur ask the engineer who followed University: "Go ahead, count ..."

\begin{itemize}	 
	\item[$-$] The young man begins: "1... 2... 3... 4... 5... 6... 7..." \end{itemize}

The entrepreneur feeling reassured: "Continue, continue ..."

\begin{itemize}	 
	\item[$-$] "8... 9... 10... valet... lady.. king... " !!\end{itemize}
	
	\begin{center}\underline{\hspace{5 cm}}\end{center}

Several people were asked to solve the following problem: "Prove that all odd integers are prime."

\begin{itemize}	 
	\item[$-$] Mathematician: 3 is a prime, 5 is a prime, 7 is a prime, 9 is not a prime - counter-example - claim is false.

	\item[$-$] Physicist: 3 is a prime, 5 is a prime, 7 is a prime, 9 is an experimental error, 11 is a prime ...

	\item[$-$] Engineer: 3 is a prime, 5 is a prime, 7 is a prime, 9 is a prime, 11 is a prime ...

	\item[$-$] Computer Scientist: 3's a prime, 5's a prime, 7's a prime ... segmentation fault

	\item[$-$] Lawyers: one is prime, three is prime, five is prime, seven is prime, although there appears to be prima facie evidence that nine is not prime, there exists substantial precedent to indicate that nine should be considered prime. The following brief presents the case for nine's primeness...

	\item[$-$] Liberals: The fact that nine is not prime indicates a deprived cultural environment which can only be remedied by a federally funded cultural enrichment program.

	\item[$-$] Computer programmers: one is prime, three is prime, five is prime, five is prime, five is prime, five is prime five is prime, five is prime, five is prime...

	\item[$-$] Professor: 3 is prime, 5 is prime, 7 is prime, and the rest are left as an exercise for the student.

	\item[$-$] Linguist: 3 is an odd prime, 5 is an odd prime, 7 is an odd prime, 9 is a very odd prime,...

	\item[$-$] Computer Scientist: 10 prime, 11 prime, 101 prime...

	\item[$-$] Chemist: 1 prime, 3 prime, 5 prime... hey, let's publish!

	\item[$-$] New Yorker: 3 is prime, 5 is prime, 7 is prime, 9 is... NONE OF YOUR DAMN BUSINESS!

	\item[$-$] Programmer: 3 is prime, 5 is prime, 7 is prime, 9 will be fixed in the next release,...

	\item[$-$] Salesperson: 3 is a prime, 5 is a prime, 7 is a prime, 9 -- let me make you a deal...

	\item[$-$] Advertiser: 3 is a prime, 5 is a prime, 7 is a prime, 11 is a prime,...

	\item[$-$] Accountant: 3 is prime, 5 is prime, 7 is prime, 9 is prime, deducting 10% tax and 5% other obligations.

	\item[$-$] Statistician: Let's try several randomly chosen numbers: 17 is a prime, 23 is a prime, 11 is a prime... Looks good to me.

	\item[$-$] Psychologist: 3 is a prime, 5 is a prime, 7 is a prime, 9 is a prime but tries to suppress it... 
\end{itemize}
	\begin{center}\underline{\hspace{5 cm}}\end{center}
	
A mathematician, an engineer and a physicist are given a red rubber ball to determine its volume.

\begin{itemize}	 
	\item[$-$] The mathematician: Measure the diameter and evaluates the triple integral.

	\item[$-$] The physicist: Fill a tub of water, places the ball in water and measure the total displacement volume.

	\item[$-$] The engineer: Search the model and serial number in its "red rubber balls" table.
\end{itemize}		
	\begin{center}\underline{\hspace{5 cm}}\end{center}

The following concerns a question in a physics degree exam at the University of Copenhagen: "Describe how to determine the height of a skyscraper with a barometer." 

One student replied: 

"You tie a long piece of string to the neck of the barometer, then lower the barometer from the roof of the skyscraper to the ground. The length of the string plus the length of the barometer will equal the height of the building." 

This highly original answer so incensed the examiner that the student was failed immediately. The student appealed on the grounds that his answer was indisputably correct, and the university appointed an independent arbiter to decide the case. 

The arbiter judged that the answer was indeed correct, but did not display any noticeable knowledge of physics. To resolve the problem it was decided to call the student in and allow him six minutes in which to provide a verbal answer that showed at least a minimal familiarity with the basic principles of physics. 

For five minutes the student sat in silence, forehead creased in thought. The arbiter reminded him that time was running out, to which the student replied that he had several extremely relevant answers, but couldn't make up his mind which to use. On being advised to hurry up the student replied as follows: 

"Firstly, you could take the barometer up to the roof of the skyscraper, drop it over the edge, and measure the time it takes to reach the ground. The height of the building can then be worked out from the formula H = 0.5g x t squared. But bad luck on the barometer." 

"Or if the sun is shining you could measure the height of the barometer, then set it on end and measure the length of its shadow. Then you measure the length of the skyscraper's shadow, and thereafter it is a simple matter of proportional arithmetic to work out the height of the skyscraper." 

"But if you wanted to be highly scientific about it, you could tie a short piece of string to the barometer and swing it like a pendulum, first at ground level and then on the roof of the skyscraper. The height is worked out by the difference in the gravitational restoring force T =2 pi sqr root (l /g)." 

"Or if the skyscraper has an outside emergency staircase, it would be easier to walk up it and mark off the height of the skyscraper in barometer lengths, then add them up." 

"If you merely wanted to be boring and orthodox about it, of course, you could use the barometer to measure the air pressure on the roof of the skyscraper and on the ground, and convert the difference in millibars into feet to give the height of the building." 

"But since we are constantly being exhorted to exercise independence of mind and apply scientific methods, undoubtedly the best way would be to knock on the janitor's door and say to him 'If you would like a nice new barometer, I will give you this one if you tell me the height of this skyscraper'." 

The student was Niels Bohr (Nobel Price 1923), the only Dane to win the Nobel Prize for physics. 
\begin{center}\underline{\hspace{5 cm}}\end{center}
	
The story goes that Bertrand Russell, in a lecture logic, mentioned that in the sense of material implication, a false proposition implies any proposition. A student raised his hand and said:

\begin{itemize}	 
	\item[$-$] "In that case, given 2+2=5, you can prove that you are the Pope"

	\item[$-$] "Yes", answered Russell

	\item[$-$] "And you could prove it now?!", asked the student sceptical

	\item[$-$] "For sure!", said Russel who proposed immediately the following proof:

	\begin{enumerate}
		\item Suppose that 2 + 2 = 5

		\item Substract 2 from the both side of the equality, whe have 2 = 3

		\item By symetry, 3 = 2

		\item Substract 1 each side, we obtain, 2 =1
	\end{enumerate}

	\item[$-$] Now the set containing just me and the Pope has 2 members. But 2=1, so it has only 1 member; therefore, I am the Pope...
\end{itemize}
	\begin{center}\underline{\hspace{5 cm}}\end{center}
	
What is "pi"?

\begin{itemize}	 
	\item[$-$] Mathematician: "Pi is the ratio of the circumference of a circle to its diameter."

	\item[$-$] Engineer: "Pi is about 22/7."

	\item[$-$] Physicist: "Pi is 3.14159 plus or minus 0.000005"

	\item[$-$] Computer Programmer: "Pi is 3.141592653589 in double precision."

	\item[$-$] Nutritionist: "You're one track math-minded fellows, Pie is a healthy and delicious dessert!"
\end{itemize}
	\begin{center}\underline{\hspace{5 cm}}\end{center}
	
An astronomer, a physicist, a mathematician and a computer scientist (it is said) were holidaying in Scotland. Glancing from a train window, they observed a black sheep in the middle of a field. 

\begin{itemize}	 
	\item[$-$] "How interesting", observed the astronomer, "all scottish sheep are black!" 

	\item[$-$] To which the physicist responded: "No, no! Some Scottish sheep are black!" 

	\item[$-$] The mathematician gazed heavenward in supplication, and then intoned: "In Scotland there exists at least one field, containing at least one sheep, at least one side of which is black."

	\item[$-$] The computer scientist: "Oh, no! A special case!"	
	
\end{itemize}
	\begin{center}\underline{\hspace{5 cm}}\end{center}	
	
A mathematician, a biologist and a physicist are sitting in a street cafe watching people going in and coming out of the house on the other side of the street. 

First they see two people going into the house. Time passes. After a while they notice three persons coming out of the house. 

\begin{itemize}	 
	\item[$-$] The physicist: "The measurement wasn't accurate."

	\item[$-$] The biologists: "They have reproduced".

	\item[$-$] The mathematician: "If now exactly one person enters the house then it will be empty again."
\end{itemize}
	\begin{center}\underline{\hspace{5 cm}}\end{center}

A mathematician and an engineer attend a lecture by a physicist. The topic concerns Kaluza-Klein theories involving physical processes that occur in spaces with dimensions of 9, 12 and even higher. The mathematician is sitting, clearly enjoying the lecture, while the engineer is frowning and looking generally confused and puzzled. By the end the engineer has a terrible headache. At the end, the mathematician comments about the wonderful lecture. 

\begin{itemize}	 
	\item[$-$] The engineer says: "How do you understand this stuff?"

	\item[$-$] Mathematician: "I just visualize the process."

	\item[$-$] Engineer: "How can you visualize something that occurs in 9-dimensional space?"

	\item[$-$] Mathematician: "Easy, first visualize it in N-dimensional space, then let N go to 9."
\end{itemize}
	\begin{center}\underline{\hspace{5 cm}}\end{center}

Two mathematicians are in a bar. The first one says to the second that the average person knows very little about basic mathematics. The second one disagrees, and claims that most people can cope with a reasonable amount of math. 

The first mathematician goes off to the wash-room, and in his absence the second calls over the waitress. He tells her that in a few minutes, after his friend has returned, he will call her over and ask her a question. All she has to do is answer "one third x cubed." 

She repeats "one thir -- dex cue"?

He repeats "one third x cubed".

She asks, "one thir dex cuebd?"

"Yes, that's right," he says.

So she agrees, and goes off mumbling to herself, "one thir dex cuebd...". 

The first guy returns and the second proposes a bet to prove his point, that most people do know something about basic math. He says he will ask the blonde waitress an integral, and the first laughingly agrees. The second man calls over the waitress and asks "what is the integral of x squared?".

The waitress says "one third x cubed" and while walking away, turns back and says over her shoulder "plus a constant!"
	\begin{center}\underline{\hspace{5 cm}}\end{center}

Teachers words:

\begin{itemize}	 
	\item[$-$] There are still pieces of the argument
	\item[$-$] The '-' sign in front of the potential disturbs you? And if I put a '+' do you feel better? Yes! Then put a '+' ....
	\item[$-$] This is a phase curve? No, this is a dog leg! 
	\item[$-$] Let's kill him and we will believe in suicide
	\item[$-$] You can say it's predictable because it is unpredictable!
	\item[$-$] I count on you to understand a little more thoroughly what will come next
	\item[$-$] We will now study this a little later
	\item[$-$] I will present you now some results dependent on Maxwell's equations that you have not seen yet ... anyway, at the point where we are ... 
	\item[$-$] If you do not understand, it's normal ... the contrary would be surprising indeed 
	\item[$-$] General relativity is useless! It is not with how we put satellites into orbit! 
	\item[$-$] Fifty cents question
	\item[$-$] Occasionally, we have to be absurd
	\item[$-$] When you have nothing to do, you take the Gauss theorem
	\item[$-$] By which virtue of the Holy Spirit would it become neutral?! 
	\item[$-$] You will understand once you grow up...
	\item[$-$] Take the example of a bank: there is a Mr. who makes a deposit and a Mrs. who makes withdrawal ... Finally, as usual what! 
	\item[$-$] ... more the parallelism will be parallel...
	\item[$-$] Remove me these residues from earlier calculations 
	\item[$-$] Delete the left side. I said LEFT! Where is your right? This is ... Well the left is the other side! 
	\item[$-$] Ok, who will be the next victim?
	\item[$-$] The '+' is recognized, and the electrons do not deceive themselves, for its beautiful red color! 
	\item[$-$] The goal is to verify that the test does not say shit 
	\item[$-$] If you can not do that, I assure you: it's all over for the exam
	\item[$-$] If you can not do that, do it!
	\item[$-$] You are very good at finding wrong things 
	\item[$-$] I like this kind of demonstration. You, it makes you have nightmares
	\item[$-$] I won't will resolve this, as this may offend your sensibility! 
	\item[$-$] You must be careful with all what the teachers love!
	\item[$-$] You add potatoes and pigs, it is dimensionless ...
	\item[$-$] You can find the derivative of the Dirac pulse, it is not that we will jump to the head
	\item[$-$] It's silly but Riemann, that's how
	\item[$-$] Look at the equation that I just deleted
	\item[$-$] Legally speaking, the current is in this sense
	\item[$-$] There is no question that I waste my time to solve you this low fly joke! 
	\item[$-$] If the girl has understood then you must have all understood
	\item[$-$] Oh yeah, you're laughing at me. In fact, you are right to make fun of us, because we do not hesitate to make fun of you 
	\item[$-$] Attention very important: I propose to dream at it on night
	\item[$-$] I have the feeling of playing the Stradivarius before cows
	\item[$-$] The hand of God spreads the field perpendicular to the surface
	\item[$-$] The miracle of the disappearance of the harmonic will not happen today
	\item[$-$] If you wake up at night, remember that sampling in the time domain is periodized in the frequency domain ... then you go back to sleep 
	\item[$-$] The left curves are not right
	\item[$-$] You see, sometimes my results are correct
	\item[$-$] We denote "Q" the output, it's obvious ...
	\item[$-$] We'll have a trick to successfully recover this variable
	\item[$-$] I am a 68000; Mr. Director comes in the corridor, knocking on the door: there is no better interruption! 
	\item[$-$] Warning: one, two, three, take your brains!
	\item[$-$] In the steppes of automatic, we reach the arid part of mathematics 
	\item[$-$] What interest? Well no. This is a figure of speech Teaching 
	\item[$-$] If you have some memories of the signal flow graph that once we plotted ...
	\item[$-$] The electrons in a hot metal, it's the New York main street at 5 in the evening, so it does not lead
	\item[$-$] The circles, is what has the least corners!
	\item[$-$] If you put your finger in a socket, it's not a complex number that comes
	\item[$-$] See the proof in your daily readings ... So I waiver
	\item[$-$] And then comes Carnot : he brings his mind that does not mean much 
	\item[$-$] Sometimes people who have a bit of culture that know that... it missed today! 
	\item[$-$] We will not call it "performance" because it can reach a value higher than 1, that might disturb some weak minds
	\item[$-$] Me, if I had been sent to the blackboard, I would not have written this 
	\item[$-$] You will find the answer on www.archimedes.com
	\item[$-$] I use a method that dates back several centuries ... as I do!
	\item[$-$] Ask me questions ... I would like to have questions! ... other issues? I'll have to take the list and say: "But you have a question"!
	\item[$-$] The ordering implemented depends on the instantiation of the garbage collector if it is synchronized to the preemptive multi-threading on the OS.
	\item[$-$] You are only boxes receiving inputs and spitting outputs
\end{itemize}
	\begin{center}\underline{\hspace{5 cm}}\end{center}

Sherlock Holmes and Dr. John Watson went on a camping trip. After sharing a good meal and a bottle of Petri wine, they retire to their tent for the night.

At about 3 AM, Holmes nudges Watson and asks, "Watson, look up into the sky and tell me what you see?"

Watson said, "I see millions of stars."

Holmes asks, "And, what does that tell you?"

Watson replies, "Astronomically, it tells me there are millions of galaxies and potentially billions of planets. Astrologically, it tells me that Saturn is in Leo. Theologically, it tells me that God is great and we are small and insignificant. Horologically, it tells me that it's about 3 AM. Meteorologically, it tells me that we will have a beautiful day tomorrow. What does it tell you, Holmes?"

Holmes retorts, "Someone stole our tent." 
	\begin{center}\underline{\hspace{5 cm}}\end{center}

An engineer, a mathematician, and a physicist are staying for the night in a hotel. Fortunately for this joke, a small fire breaks out in each room.

The physicist awakes, sees the fire, makes some careful observations, and on the back of the hotel's wine list does some quick calculations. Grabbing the fire extinguisher, he puts out the fire with one, short, well placed burst, and then crawls back into bed and goes back to sleep.

The engineer awakes, sees the fire, makes some careful observations, and on the back of the hotel's room service list (pizza menu) does some quick calculations. Grabbing the fire extinguisher (and adding a factor of safety of 5), he puts out the fire by hosing down the entire room several times over, and then crawls into his soggy bed and goes back to sleep.

The mathematician awakes, sees the fire, makes some careful observations, and on a blackboard installed in the room, does some quick calculations. Jubliant, he exclaims "A solution exists!", and crawls into his dry bed and goes back to sleep.
	\begin{center}\underline{\hspace{5 cm}}\end{center}

How do you know that the driver driving toward you is a physicist?

He has a red sticker on his bumper, saying: "If this sticker is blue, you are driving too fast."
	\begin{center}\underline{\hspace{5 cm}}\end{center}

A Princeton plasma physicist is at the beach when he discovers a ancient looking oil lantern sticking out of the sand. He rubs the sand off with a towel and a genie pops out. The genie offers to grant him one wish. The physicist retrieves a map of the world from his car an circles the Middle East and tells the genie, "I wish you to bring peace in this region".

After 10 long minutes of deliberation, the genie replies, "Gee, there are lots of problems there with Lebanon, Iraq, Israel, and all those other places. This is awfully embarrassing. I've never had to do this before, but I'm just going to have to ask you for another wish. This one is just too much for me".

Taken aback, the physicist thinks a bit and asks, "I wish that the Princeton tokamak would achieve scientific fusion energy break-even."

After another deliberation the genie asks, "Could I see that map again?"	
	\begin{center}\underline{\hspace{5 cm}}\end{center}

In the beginning, there were two species of apple trees: those whose apples fell, and those whose apples were rising. The apples that fell could reach the ground, germinate and so generating a new tree whose apples fell. But the rising apples never reached the ground, and species of apple trees with apples rising quickly disappeared because they could not reproduce. As they were not adapted, nature has thus eliminated. That's natural selection. If apples fall, it is thanks to natural selection!

But there are clever people who will object that the stones fall too. But they are not living beings, so they are not subject to natural selection, it is obvious. So the above explanation does not explain what happens to stones. In fact, for the stones, it is even simpler:

In the beginning, there were indeed two kinds of stones: those who fell, and who were rising. But those who rised up gone very far. That's why all the stones that remain on earth do the same thing: they fall.

	\pagebreak
	\section{Mathematics}

5 out of 4 people don't understand fractions...
	\begin{center}\underline{\hspace{5 cm}}\end{center}

Love is like $\pi$, natural, irrational, transcendent and very real.
	\begin{center}\underline{\hspace{5 cm}}\end{center}

A math teacher explains to a blonde the limits. He made it with the following exercise:
	\begin{gather*}
	\lim_{x \rightarrow 8} \dfrac{1}{x-8}=+\infty
	\end{gather*}
At the end of the exercise, he asked the blonde if she understood. "Oh yes sir I understood everything". Believing the answer only half, the teacher asked the following exercise:

Calculate:
	\begin{gather*}
	\lim_{x \rightarrow 5} \dfrac{1}{x-5}
	\end{gather*}
The blonde writted: 
	\begin{gather*}
	\lim_{x \rightarrow 8} \dfrac{1}{x-8}=+\infty \quad  \text{then} \quad \lim_{x \rightarrow 5} \dfrac{1}{x-5}= \rotatebox[origin=c]{90}{5}  
	\end{gather*}
	\begin{center}\underline{\hspace{5 cm}}\end{center}
	
	\begin{center}
		\includegraphics{img/humour/self_complementary_graph.jpg}	
	\end{center}
	
	\pagebreak
Evolution of the teaching of Mathematics (...)

\begin{itemize}	 
	\item[$-$] Education 1960: A farmer sells a bag of potatoes for 100\$. Its production costs amounted to 4/5 of the selling price. What is his profit?

	\item[$-$] Traditional Education 1970: A farmer sells a bag of potatoes for 100\$. Its production costs amounted to 4/5 of the selling price, that is to say 80\$. What is his profit?

	\item[$-$] Modern Education 1970: A farmer exchanges a set P of potatoes against a set M of money. The cardinality of the set M is 100, and each PFM element is 1\$. Draw 100 large dots elements of the set M. The set F of the production costs are 20 big points less than the set M. Represent the set F as a subset of the set M and give the answer to the question: What is the cardinality of the set B benefits? (draw it in red)

	\item[$-$] Renovated Education in 1980: A farmer sells a bag of potatoes for 100\$. Production costs amount to 80\$ and the benefit is 20\$. Homework: underline the words "Potatoes" and discussed this with your neighbour.

	\item[$-$] Start-up Education 1999: A wired producer of agricultural space consults a data bank which
display the day-rate of the potato. It load its reliable software computation and determines the cash flow on bit-map screen (under config WMil with 40GB HDD and floppy). Draw with your mouse the integrated 3D contour of the bag of potatoes. Then log yourself to the network by www.blue-potatoe.com and follow the instructions of the menu.

	\item[$-$] Education 2010: What is a farmer?

\end{itemize}
	\begin{center}\underline{\hspace{5 cm}}\end{center}
	\begin{center}
		\includegraphics[scale=0.9]{img/humour/homework.jpg}	
	\end{center}

	\begin{table}[H]
	\begin{center}
		\definecolor{gris}{gray}{0.85}
			\begin{tabular}{|p{7.5cm}|p{7.5cm}|}
				\hline
				\multicolumn{1}{c}{\cellcolor{black!30}\textbf{When you read or listen}} & 
  \multicolumn{1}{c}{\cellcolor{black!30}\textbf{What you have to understand}} \\ \hline
				this is trivial (or obvious) & I can not say why this is true \\ \hline
				we get automatically & idem \\ \hline
				a calculation shows that & a calculation that I did not will certainly show that\\ \hline
				the reader will easily control that & it bothers me to show that\\ \hline
				we strongly recommend the reader to make the indicated exercises & as I have not made them, you could correct me\\ \hline
				i showed this result in a previous paper & i have absolutely no idea how I did to prove that thing			
				\\ \hline
				is easily generalized to & the generalization is beyond my level			
				\\ \hline
				according to a well known property & known by maximum 10 people in the world
				\\ \hline
				the proof is in two lines & yes, but through five lemmas
				\\ \hline
				it's algebra & it is not interesting (in the mouth of an analyst)
				\\ \hline
				it's analysis & it is not interesting (in the mouth of an algebrist)
				\\ \hline
				it's elementary (or classical) & in bornitziens space theory bornitziens of the second kind
				\\ \hline
				i did not understand this step in your demonstration & you're stuck in your demo
				\\ \hline
				This conference was very interesting & i did not understand anything
				\\ \hline
		\end{tabular}
	\end{center}
	\end{table}	
	\begin{center}\underline{\hspace{5 cm}}\end{center}
	
	The number you requested is imaginary, please turn your phone to a quarter turn right and renumber...
	\begin{center}\underline{\hspace{5 cm}}\end{center}
	
	\begin{center}
		\includegraphics[scale=0.6]{img/humour/pizza.eps}	
	\end{center}
	\begin{center}\underline{\hspace{5 cm}}\end{center}
	
	A mathematician to his friend:

\begin{itemize}	 
	\item[$-$] "Are you faithful?"

	\item[$-$] "Yes, up to isomorphism"	
\end{itemize}
	\begin{center}\underline{\hspace{5 cm}}\end{center}
	
How mathematicians do it...

\begin{itemize}	 
	\item[$-$] Algebraists do it by symbolic manipulation.

	\item[$-$] Algebraists do it in a ring, in fields, in groups.

	\item[$-$] Analysts do it continuously and smoothly.

	\item[$-$] Applied mathematicians do it by computer simulation.

	\item[$-$] Banach spacers do it completely.

	\item[$-$] Bayesians do it with improper priors.

	\item[$-$] Catastrophe theorists do it falling off part of a sheet.

	\item[$-$] Combinatorists do it as many ways as they can.

	\item[$-$] Complex analysts do it between the sheets

	\item[$-$] Computer scientists do it depth-first.

	\item[$-$] Cosmologists do it in the first three minutes.

	\item[$-$] Decision theorists do it optimally.

	\item[$-$] Functional analysts do it with compact support.

	\item[$-$] Galois theorists do it in a field.

	\item[$-$] Game theorists do it by dominance or saddle points.

	\item[$-$] Geometers do it with involutions.

	\item[$-$] Geometers do it symmetrically.

	\item[$-$] Graph theorists do it in four colors.

	\item[$-$] Hilbert spacers do it orthogonally.

	\item[$-$] Large cardinals do it inaccessibly.

	\item[$-$] Linear programmers do it with nearest neighbors.

	\item[$-$] Logicians do it by choice, consistently and completely.

	\item[$-$] Logicians do it incompletely or inconsistently.

	\item[$-$] (Logicians do it) or [not (logicians do it)].

	\item[$-$] Number theorists do it perfectly and rationally.

	\item[$-$] Mathematical physicists understand the theory of how to do it, but have difficulty obtaining practical results.

	\item[$-$] Pure mathematicians do it rigorously.

	\item[$-$] Quantum physicists can either know how fast they do it, or where they do it, but not both.

	\item[$-$] Real analysts do it almost everywhere

	\item[$-$] Ring theorists do it non-commutatively.

	\item[$-$] Set theorists do it with cardinals.

	\item[$-$] Statisticians probably do it.

	\item[$-$] Topologists do it openly, in multiply connected domains

	\item[$-$]  Variationists do it locally and globally.

	\item[$-$] Cantor did it diagonally.

	\item[$-$] Fermat tried to do it in the margin, but couldn't fit it in.

	\item[$-$] Galois did it the night before.

	\item[$-$] Mðbius always does it on the same side.

	\item[$-$] Markov does it in chains.

	\item[$-$] Newton did it standing on the shoulders of giants.

	\item[$-$] Turing did it but couldn't decide if he'd finished.
 \end{itemize}
	\begin{center}\underline{\hspace{5 cm}}\end{center}
	 
What will a complex French guy said to a real woman?

Answer: "viens danser !" (you have to read "come in C", the complex set...)
	\begin{center}\underline{\hspace{5 cm}}\end{center}
	
	\begin{center}
		\includegraphics{img/humour/rotation_matrix.jpg}	
	\end{center}
	
	\begin{center}\underline{\hspace{5 cm}}\end{center}

Two-way sentences in French and Fnglish:

\begin{itemize}	 
	\item[$-$] We solve now this problem without complex

	\item[$-$] Un repère d'origine O (un repaire d'originaux...) 

	\item[$-$] Une partie de $\mathbb{Q}$ (a fucking party...)

	\item[$-$] Ne confondez pas un $\rho$ avec un $p$... (don't confuse between a fart and a burp)

	\item[$-$] Une variété de Poisson... (a variety of fish)
\end{itemize}
	\begin{center}\underline{\hspace{5 cm}}\end{center}

Logarithm and exponential functions are at the restaurant. When comes the addition, which will pay?

Answer: Exponential, because logarithme né paie rien (in English "neperien" phonetically means: pay nothing)

Later in the evening, Logarithm and Exponential go home a little bit drunked. Logarithm asks: Do I drive?

Exponential answers: I'd rather it be me who leads. In the case you derivate...
	\begin{center}\underline{\hspace{5 cm}}\end{center}

Two Cauchy sequences want to go out to dance. They arrive at a club where the "No Limit" evening event takes place. They decide to enter, but the guard stops them, saying, "Sorry, we're complete" (in a complete space, a Cauchy sequence is convergent by definition, so it has a limit).	
	\begin{center}\underline{\hspace{5 cm}}\end{center}

	\begin{center}
		\includegraphics{img/humour/socks.eps}	
	\end{center}
	\begin{center}\underline{\hspace{5 cm}}\end{center}	

A mathematician went fool and believed that he was the differentiation operator. His friends had placed him in a mental hospital until he got better. All day he would go around frightening the other patients by staring at them and saying: "I differentiate you!" 

One day he met a new patient; and true to form he stared at him and said "I differentiate you!", but for once, his victim's expression didn't change. Surprised, the mathematician marshalled his energies, stared fiercely at the new patient and said loudly "I differentiate you!", but still the other man had no reaction. Finally, in frustration, the mathematician screamed out "I DIFFERENTIATE YOU!"

The new patient calmly looked up and said: "You can differentiate me all you like: I'm $e$ to the $x$."
	\begin{center}\underline{\hspace{5 cm}}\end{center}	

What mathematicians say and what you have to understand:

\begin{itemize}	 
	\item[$-$] Trivial: If I have to show you this, you're in the wrong class

	\item[$-$] One can trivially show: We do not need more than 4 hours to write the proof

	\item[$-$] Check yourself: This is the hard part of the demonstration so you can do on your spare time

	\item[$-$] Similarly: At least one line of the proof is identical to the previous

	\item[$-$] Proceed formally: We will manipulate symbols with many predefined rules without understanding the real meaning of the result.

	\item[$-$] We will provide us the proof: Trust me, it's true!

	\item[$-$] The reader will easily show: I get tired to show that...

	\item[$-$] We strongly suggest the reader to make the indicated exercises: As I have not made them, you could correct me

	\item[$-$] I showed this result in a previous paper: I do not know how the devil did to prove this thing

	\item[$-$] We generalize easily: The generalization is beyond my level

	\item[$-$] According to a well known property: For 10 people in the world ...
\end{itemize}
\begin{center}\underline{\hspace{5 cm}}\end{center}
	
	\begin{center}
		\includegraphics[scale=0.9]{img/humour/fresh_men.jpg}	
	\end{center}
	
	\begin{center}\underline{\hspace{5 cm}}\end{center}	

There are 3 kinds of people: those who can count and those who can not count...
	\begin{center}\underline{\hspace{5 cm}}\end{center}	
	
Everyone knows the "Salary Theorem" which states that engineers and scientists can NEVER earn as much as businessmen and commercial. This theorem can then be demonstrated by solving a simple math equation.

Our equation is based on two well known postulates:

P1. The Knowledge is Power

P2. Time is Money:

All engineers know that:

\begin{center}
$\text{Power}=\dfrac{\text{Work}}{\text{Time}}$
\end{center}

and:

\begin{center}
$\text{Knowledge}=\text{Power}$
\end{center}

and also:

\begin{center}
$\text{Time}=\text{Money}$
\end{center}

We then obtain by substitution: 

\begin{center}
$\text{Knowledge}=\dfrac{\text{Work}}{\text{Money}}$
\end{center}

And we finally get the following result: 

\begin{center}
$\text{Money}=\dfrac{\text{Work}}{\text{Knowledge}}$
\end{center}

So when the Knowledge approaches zero, Money approaches infinity regardless of the value attributed to work, this value may be very low. Conversely when Knowledge goes to infinity, the Silver tends to zero, even if the work is high value.

Hence the obvious conclusion follows: The less you know, the more money you make.

PS: Those of you who have had some difficulty understanding this should be the highest paid.

	\begin{flushright}
		$\square$  Q.E.D.
	\end{flushright}
	\begin{center}\underline{\hspace{5 cm}}\end{center}
	
	\begin{center}
		\includegraphics{img/humour/proof_trivial.jpg}	
	\end{center}
	
	\begin{center}\underline{\hspace{5 cm}}\end{center}	
	
In the same kind here is the "misogynistic theorem":

First, we state that girls factorable variables in amounts of time and money such as:

\begin{center}
$\text{Girls}=\text{Time}\times\text{Time}$
\end{center}

As we all know "time is money". So:

\begin{center}
$\text{Time}=\text{Money}$
\end{center}

and because "money is the root of evil ...":

\begin{center}
$\text{Money}=\sqrt{\text{Evil}}$
\end{center}

Then we have by substitution:

\begin{center}
$\text{Girls}=\left(\sqrt{\text{Evil}}\right)^2$
\end{center}

We are therefore forced to conclude:

\begin{center}
$\text{Girls}=\text{Evil}$
\end{center}

	\begin{flushright}
		$\square$  Q.E.D.
	\end{flushright}
	
	\begin{center}\underline{\hspace{5 cm}}\end{center}
	\begin{center}
		\includegraphics{img/humour/professor_xi.jpg}	
	\end{center}
	
	The top ten excuses for not doing your math homework:
	
	\begin{itemize}
	
		\item[$\text{\#}10.$] Galileo didn't know calculus; what do I need it for?
	
		\item[$\text{\#}09.$] A math addict stole my homework.
	
		\item[$\text{\#}08.$] I'm taking physics and the homework in there seemed to involve math, so I thought I could just do that instead.
	
		\item[$\text{\#}07.$] I have the proof, but there isn't room to write it in the margin.
	
		\item[$\text{\#}06.$] I have a solar powered calculator and it was cloudy.
	
		\item[$\text{\#}05.$] I was watching the World Series and got tied up trying to prove that it converged.
	
		\item[$\text{\#}04.$] I could only get arbitrarily close to my textbook. (I reached half way, and then half of that, and then ...)
	
		\item[$\text{\#}03.$] I couldn't figure out whether i am the square root of negative one or i is the square root of negative one.
	
		\item[$\text{\#}02.$] It was Einstein's birthday and pi day and we had this big celebration! (This only works for March 14)
	
		\item[$\text{\#}01.$] I accidentally divided by zero and my paper burst into flames.	
	\end{itemize}

	\begin{center}\underline{\hspace{5 cm}}\end{center}
	\begin{center}
		\includegraphics{img/humour/close.jpg}	
	\end{center}
	\begin{center}\underline{\hspace{5 cm}}\end{center}
	
	\pagebreak
	What is the result of:
	\begin{center}
		 $\dfrac{2ab}{2Fr.16}$\\
		(read "2 abbés sur 2 françaises")
	\end{center} 

	Answer: 
	\begin{center}
		$2bb \dfrac{a}{e}$\\
		(read "2 bébés assurés")
	\end{center} 
	\begin{center}\underline{\hspace{5 cm}}\end{center}
	
	What is the result of: 
	\begin{center}
	 $\dfrac{\text{cheval}}{\text{oiseau}}$\\
	(read "horse on bird")
	 \end{center} 
	
	As we have: 
	\begin{center}
	 $\dfrac{\text{cheval}}{\text{oiseau}}=\dfrac{\text{vache} \cdot \text{l}}{\beta \cdot \text{l}}$\\
	(read "vache + l" (contains all letters of the word "cheval") divided by "bête à ailes" meaning animal with wing) 
	 \end{center} 
	
	But: 
	\begin{center}
	 $\dfrac{\text{vache} \cdot \text{l}}{\beta \cdot \text{l}}=\dfrac{\beta \cdot \pi \cdot \text{l}}{\beta \cdot \text{l}}$\\
	(read "bête à pie + l" divided by "bête à ailes")  
	 \end{center}
	
	We simplify to get: 
	\begin{center}
	 $\dfrac{\cancel{\beta} \cdot \pi \cdot \cancel{\text{l}}}{\cancel{\beta} \cdot \cancel{\text{l}}}=\pi$  
	 \end{center}
	
	This prove that $\pi$ is irrational because there is no rational comparison between a "cheval" (horse) and an "oiseau" (bird)... 
		\begin{center}\underline{\hspace{5 cm}}\end{center}
	\pagebreak
	We have to prove that:
	\begin{center}
	$\dfrac{\text{ROSSINI}}{\text{SOLSIDO}}=1$  
	\end{center} 
	
	We can write this as following:
	\begin{center}
	$\dfrac{\text{ROS SI NI}}{\text{SOL SI DO}}=\dfrac{\text{ROS NI}}{\text{SOL DO}}$  
	\end{center}
	
	but "NI vaut Do" (in French this means "niveau d'eau" or in English "water level") thus:
	\begin{center}
	$\dfrac{\text{ROS}}{\text{SOL}}$  
	\end{center}
	
	But "SOL fait RINO" (Solferino: it's during this battle that Henri Dunant had the idea to create the Red Cross) then: 
	\begin{center}
	$\dfrac{\text{ROS}}{\text{RI NO}}$  
	\end{center}
	
	because "RINO c'est ROS" (rhinocéros) then RINO = ROS and we finally have: 
	\begin{center}
	$\dfrac{\text{ROS}}{\text{ROS}}=1$  
	\end{center}

	\begin{flushright}
		$\square$  Q.E.D.
	\end{flushright}	

	\begin{center}\underline{\hspace{5 cm}}\end{center}
	\begin{center}
		\includegraphics[scale=0.6]{img/humour/day_of_an_eigenvector.jpg}	
	\end{center}
	Each future engineer learns to write the sum of two rational numbers, for example:
	\begin{center}
	$1+1=2$  
	\end{center}
	
	This form is however rather banal and indicates gaps in your education.
	
	In the first semester, we learn that:
	\begin{center}
	$1=\ln(e)$  
	\end{center}
	
	and:
	\begin{center}
	$1=\sin^2(p)+\cos^2(q)$  
	\end{center}
	
	Also everybody know that:
	\begin{gather*}
	2=\sum_{n=0}^{+\infty} \left( \dfrac{1}{2} \right)^n
	\end{gather*}
	
	and that therefore the equation:
	\begin{gather*}
	1+1=2
	\end{gather*}
	
	can be written more simply:
	\begin{gather*}
	\ln(e)+\sin^2(p)+\cos^2(q)=\sum_{n=0}^{+\infty} \left( \dfrac{1}{2} \right)^n
	\end{gather*}
	
	we must admit that the look is much clearer and more scientific...
	
	On the other hand, it is clear that:
	\begin{gather*}
	1=\cosh(q)\sqrt{1-\tanh^2(q)}
	\end{gather*}
	
	and also:
	\begin{gather*}
	e=\lim_{z \rightarrow +\infty}\left(1+\dfrac{1}{z} \right) 
	\end{gather*}
	
	it follows that:
	\begin{gather*}
	\ln(e)+\sin^2(p)+\cos^2(q)=\sum_{n=0}^{+\infty} \left( \dfrac{1}{2} \right)^n
	\end{gather*}
	
	can be rewritten as follows:
	\begin{gather*}
	\ln\left( \lim_{z \rightarrow +\infty}\left(1+\dfrac{1}{z} \right)\right)+\sin^2(p)+\cos^2(q)=\sum_{n=0}^{\infty} \left( \dfrac{\cosh(q)\sqrt{1-\tanh^2(q)}}{2} \right)^n
	\end{gather*}
	
	We must also remember that:
	\begin{gather*}
	0!=1
	\end{gather*}
	
	and the inverse exponent of the exponent is opposite equal to the exponent of the exponent opposite. Assuming an $n$-dimensional space, we know that:
	\begin{gather*}
	\left( X^T\right) ^{-1}-\left( X^{-1}\right) ^{-T}=0
	\end{gather*}
	
	Taking the matrix as the metric of an oriented and orthogonal canonical space:
	\begin{gather*}
	\left( g_{ij}^T\right) ^{-1}-\left( g_{ij}^{-1}\right) ^{-T}=0
	\end{gather*}
	
	logically we obtain:
	\begin{gather*}
	\left(\left( g_{ij}^T\right) ^{-1}-\left( g_{ij}^{-1}\right) ^{-T}\right)!=1
	\end{gather*}
	
	we obtain a simple and clear expression of $1+1=2$ for everyone:
	\begin{gather*}
	\ln\left( \lim_{z \rightarrow +\infty}\left(\left(\left( g_{ij}^T\right) ^{-1}-\left( g_{ij}^{-1}\right) ^{-T}\right)!+\dfrac{1}{z} \right)\right)+\sin^2(p)+\cos^2(q)=\sum_{n=0}^{+\infty} \left( \dfrac{\cosh(q)\sqrt{1-\tanh^2(q)}}{2} \right)^n
	\end{gather*}
	
	It is therefore obvious that this equation is much more understandable than:
	\begin{gather*}
	1+1=2
	\end{gather*}
	
	It would be possible to show several other developments of this simple expression and we will do from the moment you begin to understand the simple principles of the previous method.
	\begin{center}\underline{\hspace{5 cm}}\end{center}
	\begin{center}
		\includegraphics{img/humour/coloring_problem.jpg}	
	\end{center}

	A wrestler, a physicist and a mathematician are subject to an experience: they are locked in a room each with a box of spinach, closed, and no can opener. After 24 hours, we'll see what they have become.
	
	\begin{itemize}	 
		\item[$-$] The wrestler was able to open her box, "Well, I just flung violently the box against the wall. The impact was such that it is open", he explains.
	
		\item[$-$] The physicist also managed to open her box: "I watched the box, and distinguished his break points. I then performed a pressure to exert maximum force on them, and the box was naturally open."
	
		\item[$-$] The mathematician, finally, is found prostrate in a corner of the room, the sweat streaming down his face, and his box, closed, between the feet: "We admit that the box is opened ... We admit that... ".
	\end{itemize}
	\begin{center}\underline{\hspace{5 cm}}\end{center}
	\begin{center}
		\includegraphics[scale=0.9]{img/humour/math_useful.jpg}	
	\end{center}

	\begin{center}\underline{\hspace{5 cm}}\end{center}
	Mathematics of life:
	
	Therefore:
	
	Developping:
	
	That's real Life. Enjoy it!
	
	\begin{center}\underline{\hspace{5 cm}}\end{center}
	An opinion without $3.14$ is an onion. You'll understand!
	
	\begin{center}\underline{\hspace{5 cm}}\end{center}

	\begin{center}
		\includegraphics[scale=0.4]{img/humour/math_man_sex.jpg}	
	\end{center}
	
	\begin{center}\underline{\hspace{5 cm}}\end{center}
	
	A logician's wife is having a baby. The doctor immediately hands the newborn to the dad.
	
	His wife asks impatiently: "So, is it a boy or a girl"?
	
	The logician replies: "Yes".

	\begin{center}\underline{\hspace{5 cm}}\end{center}
	Question: What does the "B" in Benoit B. Mandelbrot stand for?	
	
	Answer: Benoit B. Mandelbrot
	
	\begin{center}\underline{\hspace{5 cm}}\end{center}
	
	This is not the way we do this derivative...:
	
	
	\begin{center}\underline{\hspace{5 cm}}\end{center}

	\begin{center}
		\includegraphics[scale=0.7]{img/humour/asymptote.jpg}	
	\end{center}

	\pagebreak
	\section{Physics}
	
	\begin{center}
	\includegraphics{img/humour/heisenberg.eps}
	\end{center}
	
	\begin{center}\underline{\hspace{5 cm}}\end{center}	
	
	\begin{itemize}	 
		\item[$-$] In theory, there is no difference between theory and practice. In practice, there is a difference!
	
		\item[$-$] The theory is when we know everything but nothing works. The practice is when everything works, but we do not know why. In computer science, theory and practice are met: nothing works and you do not know why!
	\end{itemize}

	\begin{center}\underline{\hspace{5 cm}}\end{center}

	Matter is fundamentally lazy - It always takes the path of least effort
	
	Matter is fundamentally stupid - It tries every other path first.
	
	That is the heart of physics - The rest is details.
	
	\begin{center}\underline{\hspace{5 cm}}\end{center}
	
	Two atoms meets together. 
	
	One says to the other: "Shit, I lost an electron!"
	
	The other: "Are you sure?"
	
	And the first replies, "POSITIVELY!!"
	
	\begin{center}\underline{\hspace{5 cm}}\end{center}

	\begin{center}
	\includegraphics{img/humour/einstein.eps}
	\end{center}
	
	\begin{center}\underline{\hspace{5 cm}}\end{center}	
	
	You enter the laboratory and see an experiment. How will you know which class is it?
	
	\begin{itemize}	 
		\item[$-$] If it's green and wiggles, it's biology.
	
		\item[$-$] If it stinks, it's chemistry.
	
		\item[$-$] If it doesn't work, it's physics.
	\end{itemize}
	
	\begin{center}\underline{\hspace{5 cm}}\end{center}	

	Theorem: A cat has nine tails.
	
	Proof: No cat has eight tails. A cat has one tail more than no cat. Therefore, a cat has nine tails.
	
	\begin{center}\underline{\hspace{5 cm}}\end{center}

	\begin{center}
	\includegraphics[scale=0.75]{img/humour/schrodinger_cat.eps}
	\end{center}
	
	\begin{center}\underline{\hspace{5 cm}}\end{center}

	A physicist studying quantum physics, is someone who does not see very well, looking in a dark room for a black cat, who probably does not exist.
	
	\begin{center}\underline{\hspace{5 cm}}\end{center}

	An engineer, a physicist, a mathematician, and a mystic were asked to name the greatest invention of all times. 

\begin{itemize}	 
	\item[$-$] The engineer chose fire, which gave humanity power over matter.

	\item[$-$] The physicist chose the wheel, which gave humanity the power over space.

	\item[$-$] The mathematician chose the alphabet, which gave humanity power over symbols.

	\item[$-$] The mystic chose the thermos bottle.

"Why a thermos bottle?" the others asked.

	\item[$-$] The mystic: "Yes, because the thermos keeps hot liquids hot in winter and cold liquids cold in summer."

	\item[$-$] "Yes... so what?" the others asked.

	\item[$-$] The mystic: "Think about it." said the mystic reverently. "That little bottle.. how does it know?"
	\end{itemize}
\begin{center}\underline{\hspace{5 cm}}\end{center}

	\begin{center}
	\includegraphics[scale=0.4]{img/humour/milkiway.jpg}
	\end{center}
\begin{center}\underline{\hspace{5 cm}}\end{center}

Physics professor has been doing an experiment, and has worked out an empirical equation that seems to explain his data. He asks the math professor to look at it. 

A week later, the math professor says the equation is invalid. By then, the physics professor has used his equation to predict the results of further experiments, and he is getting excellent results, so he asks the math professor to look again. 

Another week goes by, and they meet once more. The math professor tells the physics professor the equation does work, "But only in the trivial case where the numbers are real and positive".
\begin{center}\underline{\hspace{5 cm}}\end{center}

Practical nuclear fusion power plants are juste 30 years away - and always will be.
	
	\pagebreak
	\begin{center}
	\includegraphics{img/humour/howscientistseeworld.eps}
	\end{center}
	
	\pagebreak
	
	Heisenberg was driving down the highway whereupon he was pulled over by a policeman. 

	The policeman asked:
	
	\begin{itemize}	 
		\item[$-$] "Do you know how fast you were going back there?"
	\end{itemize}
	
	Heisenberg replied: 
	
	\begin{itemize}	 
		\item[$-$] "No, but I know where I am."
	\end{itemize}
	\begin{center}\underline{\hspace{5 cm}}\end{center}
	
	What's the difference between an auto mechanic and a quantum mechanic?
	
	The quantum mechanic can sometimes get the car inside the garage without opening the door.
	
	\begin{center}\underline{\hspace{5 cm}}\end{center}
	
	\begin{center}
	It's not the:
	
	\includegraphics{img/humour/kill_fall.jpg}
	\end{center}
	\begin{gather*}
		v_f=v_0+at
	\end{gather*}
	\begin{center}
	that kills you, it's the:
	\end{center}
	\begin{gather*}
		F=m\dfrac{\Delta v}{\Delta t}
	\end{gather*}
	\begin{center}
	\includegraphics{img/humour/kill_final.jpg}
	\end{center}
	

	\begin{center}
	\includegraphics{img/humour/superstring.eps}
	\end{center}
\begin{center}\underline{\hspace{5 cm}}\end{center}
	
Why God Never Received Tenure at any University:
\begin{enumerate}
	\item He had only one major publication.

	\item It was in Hebrew. 

	\item It had no references. 

	\item It wasn't published in a refereed journal.

	\item Some even doubt he wrote it himself. 

	\item It may be true that he created the world, but what has he done since then? 

	\item His cooperative efforts have been quite limited. 

	\item The scientific community has had a hard time replicating his results. 

	\item He never applied to the Ethics Board for permission to use human subjects.

	\item When one experiment went awry he tried to cover it up by drowning the subjects. 

	\item When subjects didn't behave as predicted, he deleted them from the sample . 

	\item He rarely came to class, just told students to read the Book. 

	\item Some say he had his son teach the class. 

	\item He expelled his first two students for learning. 

	\item Although there were only ten requirements, most students failed his tests. 

	\item His office hours were infrequent and usually held on a mountaintop.
\end{enumerate}

\begin{center}\underline{\hspace{5 cm}}\end{center}

	\begin{center}
		\includegraphics{img/humour/physics_gang_sign.jpg}
	\end{center}
	\pagebreak

\begin{center}\underline{\hspace{5 cm}}\end{center}

In the beginning there was Aristotle:
\begin{itemize}
	\item And objects at rest tended to remain at rest
	\item And objects in motion tended to come to rest
	\item And God saw that it was boring, although very restful.
\end{itemize}

Then God created Newton:
\begin{itemize}
	\item And objects at rest tended to remain at rest
	\item And objects in motion tended to remain in motion
	\item And energy was conserved, and momentum was conserved,
	\item And matter was conserved
	\item And God saw that it was conservative.
\end{itemize}

Then God created Einstein:
\begin{itemize}
	\item And everything was relative
	\item And fast things became short
	\item And straight things became curved
	\item And the universe was filled with inertial frames
	\item And God saw that it was relatively general
but some of it was especially relative.
\end{itemize}

Then God created Bohr:
\begin{itemize}
	\item And there was the principle
	\item And the principle was quantum
	\item And all things were quantified
	\item But some things were still relative
	\item And God saw that it was confusing.
\end{itemize}

Then God was going to create Furgeson:
\begin{itemize}
	\item And Furgeson would have unified
	\item And he would have fielded a theory
	\item And all would have been one.
	\item But it was the seventh day
	\item And God rested
	\item And objects at rest tend to remain at rest.
\end{itemize}
\begin{center}\underline{\hspace{5 cm}}\end{center}

	\begin{center}
		\includegraphics[scale=0.6]{img/humour/schrodinger_survey.jpg}	
	\end{center}

\begin{center}\underline{\hspace{5 cm}}\end{center}
The Physicist's Bill of Rights

We hold these postulates to be intuitively obvious, that all physicists are born equal, to a first approximation, and are endowed by their creator with certain discrete privileges, among them a mean rest life, n degrees of freedom, and the following rights which are invariant under all linear transformations:
\begin{enumerate}
	\item To approximate all problems to ideal cases.

	\item To use order of magnitude calculations whenever deemed necessary (i.e. whenever one can get away with it).

	\item To use the rigorous method of "squinting" for solving problems more complex than the addition of positive real integers.

	\item To dismiss all functions which diverge as "nasty" and "unphysical".

	\item To invoke the uncertainty principle when confronted by confused mathematicians, chemists, engineers, psychologists, dramatists, and other lower scientists.

	\item hen pressed by non-physicists for an explanation of (4) to mumble in a sneering tone of voice something about physically naive mathematicians.

	\item To equate two sides of an equation which are dimensionally inconsistent, with a suitable comment to the effect of, "Well, we are interested in the order of magnitude anyway".

	\item To the extensive use of "bastard notations" where conventional mathematics will not work.

	\item To invent fictitious forces to delude the general public. 

	\item To justify shaky reasoning on the basis that it gives the right answer.

	\item To cleverly choose convenient initial conditions, using the principle of general triviality.

	\item To use plausible arguments in place of proofs, and thenceforth refer to these arguments as proofs.

	\item To take on faith any principle which seems right but cannot be proved.
\end{enumerate}

	\begin{center}\underline{\hspace{5 cm}}\end{center}
		\begin{center}
		\includegraphics[scale=0.6]{img/humour/travelling_light.jpg}	
	\end{center}

\pagebreak
You can at least take this simple "Real Scientist Quiz" to find out if you're cut out for the life of a true scientist:
\begin{enumerate}
	\item At Christmas time, you: 

a. Take a couple of days off to spend time with your family.
b. Leave early on Christmas eve so you can pick up a few presents for the family. 
c. Only work half a day, spending the rest of the day at home working on your grant application. 

	\item Your spouse wants to discuss plans for the family vacation with your kids. You: 

a. Propose to go camping so you can explain the wonders of nature to your kids 
b. Propose to go to another city so you can spend the day in your friend's lab while your spouse takes the kids sightseeing.
c.	Ask your spouse, "We have kids?"

	\item At a scientific meeting on an island in the South Pacific, no talks are scheduled in the afternoon. During this free time, you: 

a. Follow the local custom and sunbathe on the beach in the nude.
b. Sit on the beach fully clothed, unaware of the nude sunbathers, and discuss science with your colleagues.
c. Sit in your hotel room with the drapes closed, and work on your manuscript.

	\item The nurse at school calls to tell you that your second-grade child has chicken pox. You: 

a. Immediately drop what you're doing and rush to school to pick up your sick kid. 
b. Immediately drop what you're doing and begin trying to find a cure for chicken pox.
c. Ask the nurse for directions to the school, and the names of your kids.

	\item Beings from outer space visit Earth, and you are the first human they meet. To show their friendship, they present you with a highly advanced device that is capable of prolonging life, ending human suffering, and curing disease. You: 

a. Present it to the United Nations.
b.	Apply for a patent.
c. Break it open to see how it works.

	\item What is the longest amount of time that you have worked without a vacation (excluding scientific meetings)? 

a.	Six months.
b.	Two years.
c. I took a weekend off about 10 years ago.

	\item What are your hobbies? 

a. Sports, music, and dance, because they allow the analytical parts of my brain to relax. 
b. Cooking, because it's quite a lot like science.
c. Reading back issues of scientific journals cover to cover. 

	\item Your best friend is: 

a. A member of your college fraternity.
b. A member of your immediate family.
c. A member of a gene family. 
\end{enumerate}

Score:

Give yourself 1 point for every question you answered with an "a", 5 points for every "b" and 50 points for every "c". If you took the test three times and averaged your score, give yourself 100 extra points. If you calculated the standard error of the mean, give yourself 500 points. 

If you scored less than 10, you are normal. Scores of 11-50 indicate an obsessed scientist. If you scored more than 50, you are in need of help and should consider joining Scientists Anonymous; if you scored greater than 500, you should forget Scientists Anonymous and get back to work since you are beyond help, and may actually succeed as a scientist.
\begin{center}\underline{\hspace{5 cm}}\end{center}

	\begin{center}
	\includegraphics{img/humour/cow.jpg}
	\end{center}
	
\pagebreak

How you must understand certain sentences in physicists publications of:

\begin{itemize}
	\item It is well know that...: I did not read the references, but..

	\item This is of great theoretical importance: This is important for me.

	\item Though it was not possible to give a definitive answer: The experiment failed, but it seems me valuable enough to write a publication.

	\item The used technique was particularly appropriate ...: The lab next door friend had already developed the technique.

	\item 3 samples were chosen for an exhaustive study: The results obtained from other samples yielded nothing coherent.

	\item Handled with extreme caution throughout the experiment: Was not thrown down the trash.

	\item The agreement with theory is excellent: it is passable.

	\item The agreement with theory is good: it is weak.

	\item The agreement with theory is satisfactory: it is doubtful.

	\item The agreement with theory is fair: it is totally imaginary.

	\item It is generally accepted that...: two colleagues agree with me

	\item It is recognized that: I think.

	\item It is clear that further work will be useful: I did not understand anything.

	\item Here are some typical results: Here are the best results.

	\item Significant in a confidence interval of...: not significant.

	\item The reagents used were synthesized in the laboratory according to standardized techniques: The reagents were purchased from...

	\item Unfortunately, quantitative basis to exploit the results have not yet been made: Nobody was able to understand anything of what was observed.

	\item We are grateful to X for his valuable collaboration and Y for the fruitful discussions: X and Y did the work and told me what the results meant.
\end{itemize}

	\begin{center}\underline{\hspace{5 cm}}\end{center}

	\begin{center}
	\includegraphics{img/humour/duality.eps}
	\end{center}
	
	\begin{center}\underline{\hspace{5 cm}}\end{center}
	An experimental physicist performs an experiment involving tow cats, and an inclined tin roof.
	
	The two cats are very nearly identical; same sex, age, weight, breed, eye and hair colour.
	
	The physicist places both cats on the roof at the same height and lets them both go at the same time.
	
	One of the cats fall off the roof first so obviously there is some difference between the two cats.
	
	What is the difference?
	
	One cat has a greater mew!
	
	\begin{center}\underline{\hspace{5 cm}}\end{center}
	
	\begin{center}
	\includegraphics{img/humour/three_body_problem.jpg}
	\end{center}
	
	\begin{center}\underline{\hspace{5 cm}}\end{center}
	
	\begin{center}
	\includegraphics[scale=0.85]{img/humour/bus_stop_physicists.jpg}
	\end{center}
	
	\begin{center}\underline{\hspace{5 cm}}\end{center}
	
	\begin{center}
		This is how physicists see the Pokemon:
		\includegraphics[scale=0.8]{img/humour/pokemon.jpg}
	\end{center}
	
	\begin{center}
	\includegraphics[scale=0.55]{img/humour/feynman_diagrams.jpg}
	\end{center}
	
	\begin{center}\underline{\hspace{5 cm}}\end{center}
	
	\begin{center}
	\includegraphics[scale=0.55]{img/humour/cat_physics.jpg}
	\end{center}
	
	\begin{center}\underline{\hspace{5 cm}}\end{center}
	
	\begin{center}
	\includegraphics[scale=0.7]{img/humour/sun_sohn.jpg}
	\end{center}
	

	\pagebreak
	\section{Statistics}

Three statisticians go out for a static target shooting. The first statistician fired and shoots to the left, the second shot, but symmetrically on the right. The last does not shoot, but say triumphantly: "On average we got it!"
\begin{center}\underline{\hspace{5 cm}}\end{center}
	\begin{center}
	\includegraphics[scale=0.7]{img/humour/correlation.jpg}
	\end{center}
\begin{center}\underline{\hspace{5 cm}}\end{center}

Patient: "Will I survive to this delicate operation?"

Surgeon: "Yes, I am absolutely sure that you will survive."

Patient: "How can you be so sure?"

Surgeon: "9 out of 10 patients die during this operation and my ninth patient died yesterday."
\begin{center}\underline{\hspace{5 cm}}\end{center}
	\begin{center}
	\includegraphics[scale=0.6]{img/humour/gauss.eps}
	\end{center}

\pagebreak
10 reasons to work in statistics field:
\begin{enumerate}
	\item Estimate parameters is easier than fighting in real life

	\item Statisticians are recognized people

	\item You will learn the Greek alphabet entirely

	\item The probability that you get a job in this area is> 0.9999

	\item If you are fired, you can always convert yourself to engineering

	\item You make this work in the confidence, with regularity and variability

	\item You are normal and the rest of the world is wrong

	\item The regression line seems better than the unemployment line

	\item You never have to be exact - only approximate

	\item Nobody understands what you do, then you are always right
\end{enumerate} 

	\begin{center}\underline{\hspace{5 cm}}\end{center}
	\begin{center}
	\includegraphics{img/humour/statistician.eps}
	\end{center}
	
	\begin{center}\underline{\hspace{5 cm}}\end{center}		
	\begin{center}
	\includegraphics[scale=0.9]{img/humour/bayesian_inference.jpg}
	\end{center}
	
	\begin{center}\underline{\hspace{5 cm}}\end{center}
	\begin{center}
	\includegraphics[scale=0.5]{img/humour/normal_distribution.jpg}
	\end{center}
	
	\begin{center}\underline{\hspace{5 cm}}\end{center}
	\begin{center}
	\includegraphics{img/humour/bedtime_stories.jpg}
	\end{center}
		
	\pagebreak
	\section{Chemistry}

We've just discovered a new element:

\begin{itemize}
	\item[$\bullet$] ELEMENT NUMBER: 115

	\item[$\bullet$]NAME: Woman

	\item[$\bullet$] SYMBOL: Wo

	\item[$\bullet$] ATOMIC MASS: Accepted as 60 kg; isotopes may vary from 40-200 kg.

	\item[$\bullet$] OCCURRENCE: Copious quantities in all urban areas

	\item[$\bullet$] PHYSICAL PROPERTIES:

- Boils at room temperature

- Freezes without any known reason

- Melts if given special treatment

- Bitter, if incorrectly used

- Sweet as Honey if given a proper treatment.

	\item[$\bullet$] MOLECULAR STRUCTURE:

Perfect? 90/60/90, growing int the U.S. with 60/90/120 and in the nordic countries as so-called "flat" 50/50/50

	\item[$\bullet$] CHEMICAL PROPERTIES:

- Has great affinity for Gold, Silver and a range of precious stones.Absorbs in general great quantities of expensive substances.

- May explode spontaneously without prior warning and for no known reason.

- Insoluble in liquids but activity greatly increased by saturation in alcohol

- Reactivity varies depending on the time of the day

- Great ability to change mood and jealousy

- Sensitive to certain constraints which sometimes transmit migraine

	\item[$\bullet$] COMMON USES: 

- Highly ornamental, especially in sports cars

- Powerful cleaning agent

- Can be great aid to relaxation.

	\item[$\bullet$] TEST: 

- Pure specimen turns rosy pink when happy

- Turns green when placed behind a better specimen

	\item[$\bullet$] PRECAUTIONS:

- Highly dangerous if placed between the non-expert hands

- Illegal to possess more than one, although several can be maintained at different locations as long as specimens do not come in direct contact with each other.

\end{itemize}

CAUTION: 

Some South American researchers have discovered a way to produce them artificially, usually presented under the marks "transvestite" or "Drag-queen". Consume only the generic!
	\begin{center}\underline{\hspace{5 cm}}\end{center}

	\begin{center}
	\includegraphics[scale=0.5]{img/humour/thorium.jpg}
	\end{center}
	
	\begin{center}\underline{\hspace{5 cm}}\end{center}

We've just discovered a new element:

\begin{itemize}
	\item[$\bullet$] ELEMENT NUMBER: 116

	\item[$\bullet$] NAME: Man

	\item[$\bullet$] SYMBOLE: Hm

	\item[$\bullet$] QUANTITATIVE ANALYSIS:	

Measured at 17 cm, although some isotopes exist in 25, 20, 13 and even 10 cm

	\item[$\bullet$] EXTRACTION LOCATION:	

Cand befound in large quantities in the presence of a deposit of very pure Fm

	\item[$\bullet$] PHYSICAL PROPERTIES:

- Surface covered with hair, steep in places, soft in others

- Boil when it is shaken, will ice when placed in the presence of logic and common sense, liquefies when treated like a god

- Becomes execrable when mixed with any alcohol

- Can cause headaches (other body parts pains); handle with care

- Decreases its entropy directly after its reaction with the element Fm (condition manifested by snoring ... zzzzz)

- Its mass increases significantly with age, losing its reaction capacity

- Dehydrates quickly in dry weather.

- Rarely found in pure form after 14 years old

- Often has an inexplicable attachment to its mother rock, making extraction difficult

- If you put it under pressure it becomes too hard and unproductive; is productive only when used subtlety, with subterfuge and flattery

	\item[$\bullet$] CHEMICAL PROPERTIES:

- Very strong tendency to react with the Fm element, although the reaction is sometimes endothermic

- Deemed to be the best catalyst for the transformation reactions of the Fm element

- Has the ability to react with almost anything

- If the case of an important reaction the aspect of the element changes to dark red

- If it is saturated with alcohol, it becomes inert and repulsive for most elements

- Not suitable for household chores and cleaning operations 

- Not suitable either for family duties

- Is neutral with respect to the courtesy and impartiality 

	\item[$\bullet$] COMMON USES:

- Transporting heavy things, driver, free dinners at the restaurant...

- Possible use for sexual activity

	\item[$\bullet$] TESTS:

The purest specimens are not synonymous with purity, and those who have already served, are less pure

	\item[$\bullet$] HAZARDS:

The reaction with another element Hm is extremely violent if the item Fm is the catalyst
\end{itemize}
\begin{center}\underline{\hspace{5 cm}}\end{center}

The following is an actual question given on a University of Washington engineering mid term. The answer was so profound that the Professor shared it with colleagues, which is why we now have the pleasure of enjoying it as well.

Bonus Question: Is Hell exothermic (gives off heat) or Endothermic (absorbs heat)?

Most of the students wrote proofs of their beliefs using Boyle's Law, (gas cools off when it expands and heats up when it is compressed) or some variant. One student, however, wrote the following:

"First, we need to know how the mass of Hell is changing in time. So we need to know the rate that souls are moving into Hell and the rate they are leaving. I think that we can safely assume that once a soul gets to Hell, it will not leave. Therefore, no souls are leaving. As for how many souls are entering Hell, let's look at the different religions that exist in the world today. Some of these religions state that if you are not a member of their religion, you will go to Hell. Since there are more than one of these religions and since people do not belong to more than one religion, we can project that all souls go to Hell. With birth and death rates as they are, we can expect the number of souls in Hell to increase exponentially.

Now, we look at the rate of change of the volume in Hell because Boyle's Law states that in order for the temperature and pressure in Hell to stay the same, the volume of Hell has to expand as souls are added. This gives two possibilities:

\begin{enumerate}
	\item If Hell is expanding at a slower rate than the rate at which souls enter Hell, then the temperature and pressure in Hell will increase until all Hell breaks loose.

	\item Of course, if Hell is expanding at a rate faster than the increase of souls in Hell, then the temperature and pressure will drop until Hell freezes over.
\end{enumerate}

So which is it? If we accept the postulate given to me by Teresa Banyan during my Freshman year, "...that it will be a cold day in Hell before I sleep with you." and take into account the fact that I still have not succeeded in having sexual relations with her, then, \#2 cannot be true, and thus I am sure that Hell is exothermic and will not freeze."

This student received the only A.
\begin{center}\underline{\hspace{5 cm}}\end{center}

A chemist walks into a pharmacy and asks the pharmacist: "Do you have any acetylsalicylic acid?"

\begin{itemize}
	\item[$-$] "You mean aspirin?" asked the pharmacist.

	\item[$-$] "That's it, I can never remember that word."
\end{itemize}
\begin{center}\underline{\hspace{5 cm}}\end{center}

A physicist, biologist and a chemist were going to the ocean for the first time. 

\begin{itemize}
	\item The physicist saw the ocean and was fascinated by the waves. He said he wanted to do some research on the fluid dynamics of the waves and walked into the ocean. Obviously he was drowned and never returned. 

	\item The biologist said he wanted to do research on the flora and fauna inside the ocean and walked inside the ocean. He too, never returned. 

	\item The chemist waited for a long time and afterwards, wrote the observation: "The physicist and the biologist are soluble in ocean water".
\end{itemize}
\begin{center}\underline{\hspace{5 cm}}\end{center}

CLASSIFICATION OF CHEMISTRY 

\begin{itemize}
	\item \textit{Physical Chemistry}: The pitiful attempt to apply $y = mx+b$ to everything in the universe.

	\item \textit{Organic Chemistry}: The practice of transmuting vile substances into publications.

	\item \textit{Inorganic Chemistry}: That which is left over after the organic, analytical, and physical chemists get through picking over the periodic table.

	\item \textit{Chemical Engineering}: The practice of doing for a profit what an organic chemist only does for fun.
\end{itemize}
\begin{center}\underline{\hspace{5 cm}}\end{center}

\begin{center}
\includegraphics[scale=0.7]{img/humour/cute.jpg}
\end{center}

\begin{center}\underline{\hspace{5 cm}}\end{center}

Free radicals have revolutionized chemistry.
\begin{center}\underline{\hspace{5 cm}}\end{center}

Chemist's last words: 

\begin{itemize}
	\item And now the tasting test... 

	\item And now shake it a bit... 

	\item In which glass was my mineral water? 

	\item Why does that stuff burn with a green flame?!? 

	\item And now the detonating gas problem. 

	\item This is a completely safe experimental setup. 

	\item Now you can take the protection window away... 

	\item Where do all those holes in my kettle come from? 

	\item And now a cigarette... 
\end{itemize}

	\pagebreak
	\section{Engineering}

	Scientists at NASA built a device to launch dead chickens at the windshields of airliners, military jets, the space shuttle, etc. The idea being to simulate collisions with airborne fowl to test the strength of the windshields. 
	
	British engineers heard about the device and were eager to test it on the windshields of their new high speed trains. Arrangements were made and a device was sent to the British engineers.
	
	When device was fired, the British engineers were shocked... the chicken hurled out of the barrel, crashed into the shatterproof shield, smashed it to smithereens, blasted through the control console, snapped the engineer's back-rest in two and embedded itself in the back wall of the cabin.
	
	The horrified Brits sent NASA the disastrous results of the experiment, along with the designs of the windshield and begged the US scientists for suggestions.
	
	NASA responded with a one-line memo: "Defrost the chicken."

	\begin{center}\underline{\hspace{5 cm}}\end{center}
	
	\begin{center}
	\includegraphics[scale=0.3]{img/humour/great_power_great_bills.jpg}
	\end{center}

	\begin{center}\underline{\hspace{5 cm}}\end{center}

Deux ingénieurs et un ami non-ingénieurs se rencontrent à un bar un vendredi soir pour raconter leur semaine de travail.

	\begin{itemize}
		\item Le premier ingénieur: "J'ai passé un semaine horrible à faire des plans un à la fois chaque jour."
	
		\item Le deuxième ingénieur: "J'ai fait un peu moins pire. J'ai au moins pu faire des plans complets plusieurs fois par jour."
	
		\item Le troisième ami non-ingénieur: "Ben les gars vous en avez de la chance! Moi je me limite à des plans culs qu'une fois par mois".
	\end{itemize}
	\begin{center}\underline{\hspace{5 cm}}\end{center}

	\begin{center}
	\includegraphics{img/humour/acdc.jpg}
	\end{center}

	\begin{center}\underline{\hspace{5 cm}}\end{center}

Trying to understand engineers:

\begin{itemize}

	\item Trial N\degree 1

An engineer tell to a friend:"Well, yesterday I was walking home, minding my own business, when a beautiful woman rode up to me on this bike. She threw the bike to the ground, took off all her clothes and said, 'Take what you want!'"

The friend (also engineer) approvingly, "Good choice. The clothes probably wouldn't have fit."

	\item Trial N\degree 2 

For an optimist, the glass is half full.
For a pessimistic person, it is half empty.
To the engineer, it is twice as large as needed.

	\item Trial N\degree 3 

A pastor, a doctor and an engineer were waiting one morning for a particularly slow group of golfers. After a moment the engineer cry: "What's with these guys? We must have been waiting for 15 minutes!". The doctor also desperate says: " I don't know, but I've never seen such ineptitude!". The pastor the says: "Hey, here comes the greens keeper. Let's have a word with him. [dramatic pause] Hi George. Say, what's with that group ahead of us? They're rather slow, aren't they?". The greens keeper answers: " Oh, yes, that's a group of blind fire fighters. They lost their sight saving our clubhouse from a fire last year, so we always let them play for free anytime.". The group is silent for a moment... The pastor then says: "That's so sad. I think I will say a special prayer for them tonight.". And the doctor:" Good idea. And I'm going to contact my ophthalmologist buddy and see if there's anything he can do for them.". Finally the engineer says " Why the fuck can't these guys play at night?".

	\item Trial N\degree 4 

A engineer was crossing a road one day when a frog called out to him and said, "If you kiss me, I'll turn into a beautiful princess." He bent over, picked up the frog, and put it in his pocket. The frog spoke up again and said, "If you kiss me and turn me back into a beautiful princess, I will tell everyone how smart and brave you are and how you are my hero" The man took the frog out of his pocket, smiled at it, and returned it to his pocket. The frog spoke up again and said, "If you kiss me and turn me back into a beautiful princess, I will be your loving companion for an entire week." The man took the frog out of his pocket, smiled at it, and returned it to his pocket. The frog then cried out, "If you kiss me and turn me back into a princess, I'll stay with you for a year and do ANYTHING you want." Again the man took the frog out, smiled at it, and put it back into his pocket. Finally, the frog asked, "What is the matter? I've told you I'm a beautiful princess, that I'll stay with you for a year and do anything you want. Why won't you kiss me?". The man said, "Look, I'm a an engineer. I don't have time for a girlfriend, but a talking frog is cool.".

	\item Trial N\degree 5 

A reporter interviews a Corsican farmer: "Tell me, how do you draw the roads here in your country?". The farmer replies, "beh, we launch a donkey and look where it goes into the mountains .... and that's where we passed the road". The journalist then retorts: "and if you do not have a donkey?". The farmer replies: "ah beh... we take an engineer...".
\end{itemize}

	\begin{center}\underline{\hspace{5 cm}}\end{center}
	
	\begin{center}
	\includegraphics[scale=0.25]{img/humour/weather_forecast.jpg}
	\end{center}
		
	During the heat of the space race in the 1960's, NASA decided it needed a ball point pen to write in the zero gravity confines of its space capsules.
	
	After considerable research and development, the Astronaut Pen was developed at a cost of \$1 million U.S. The pen worked and also enjoyed some modest success as a novelty item back here on earth.
	
	The Soviet Union, faced with the same problem, used a pencil.
	\begin{center}\underline{\hspace{5 cm}}\end{center}
	
	The great mathematician John Von Neumann was consulted by a group who was building a rocket ship to send into outer space. When he saw the incomplete structure, he asked, "Where did you get the plans for this ship?"
	
	He was told, "We have our own staff of engineers."
	
	He disdainfully replied: "Engineers! Why, I have complete sewn up the whole mathematical theory of rocketry. See my paper of 1952."
	
	Well, the group consulted the 1952 paper, completely scrapped their \$10 million structure, and rebuilt the rocket exactly according to Von Neumann's plans. The minute they launched it, the entire structure blew up. They angrily called Von Neumann back and said: "We followed your instructions to the letter. Yet when we started it, it blew up! Why?"
	
	Von Neumann replied, "Ah, yes! That is technically known as the blow-up problem - I treated that in my paper of 1954."

	\begin{center}\underline{\hspace{5 cm}}\end{center}
	
	In an electronic laboratory:
	
	\begin{itemize}
		\item Say, what's the trailer in the parking?
	
		\item The trailer?
	
		\item Yeah, the driver says you're aware ...
	
		\item Oh yes !! I ordered a 1 Farad capacitor .
	\end{itemize}

	\begin{center}\underline{\hspace{5 cm}}\end{center}
	
	\begin{figure}[H]
		\begin{center}
		\includegraphics[scale=0.2]{img/humour/iso.jpg}
		\end{center}	
	\end{figure}
	
	\begin{center}\underline{\hspace{5 cm}}\end{center}

	What engineers say and what they mean by it:

	\begin{itemize} 
		\item Major Technological Breakthrough: Back to the drawing board. 
	
		\item Developed after years of intensive research: It was discovered by accident. 
	
		\item The designs are well within allowable limits: We just made it, stretching a point or two. 
	
		\item Test results were extremely gratifying: It works, and are we surprised! 
	
		\item Customer satisfaction is believed assured: We are so far behind schedule that the customer was happy to get anything at all. 
	
		\item Close project coordination: We should have asked someone else. 
	
		\item Project slightly behind original schedule due to unforeseen difficulties: We are working on something else. 
	
		\item The design will be finalized in the next reporting period: We haven't started this job yet, but we've got to say something. 
	
		\item A number of different approaches are being tried: We don't know where we're going, but we're moving.
	
		\item Extensive effort is being applied on a fresh approach to the problem: We just hired three new guys; we'll let them kick it around for a while.
	
		\item Preliminary operational tests are inconclusive: The darn thing blew up when we threw the switch. 
	
		\item The entire concept will have to be abandoned: The only guy who understood the thing quit. 
	
		\item Modifications are underway to correct certain minor difficulties: We threw the whole thing out and are starting from scratch. 
	
		\item Essentially complete: Half done. 
	
		\item We predict...: We hope to God! 
	
		\item Drawing release is lagging: Not a single drawing exists. 
	
		\item Risk is high, but acceptable: 100 to 1 odds, or with 10 times the budget and 10 times the manpower, we may have a 50/50 chance. 
	
		\item Serious, but not insurmountable, problems: It will take a miracle. God should be the program manager. 
	
		\item Not well defined: Nobody has thought about it. 
	
		\item Requires further analysis and management attention: Totally out of control. 
	
		\item The project is designed for high availability: Malfunctions will be blamed on the operators mistakes. 
	
		\item This project has low maintenance requirements: We wouldn't let the technicians change a light bulb, much less fool around with our baby. 
	
		\item The software is being developed without excessive process overhead: The documentation will be written in clear and lucid Chinese. 
	
		\item The delivery is scheduled for the last quarter of next year: This leaves us plenty of time to decide who to blame for it being late. 
	\end{itemize}
	
	\begin{center}\underline{\hspace{5 cm}}\end{center}

\begin{itemize} 
	\item How many first year engineering students does it take to change a light bulb?: None. That's a second year subject.

	\item How many second year engineering students does it take to change a light bulb?: One, but the rest of the class copies the report

	\item How many third year engineering students does it take to change a light bulb?: Will this question be in the final examination?

	\item How many civil engineers does it take to change a light bulb?: Two. One to do it and one to steady the chandelier

	\item How many electrical engineers does it take to change a light bulb?: None. They simply redefine darkness as the industry standard

	\item How many computer engineers does it take to change a light bulb?: Why bother? The socket will be obsolete in six months anyway

	\item How many mechanical engineers does it take to change a light bulb?: Five. One to decide which way the bulb ought to turn, one to calculate the force required, one to design a tool with which to turn the bulb, one to design a comfortable - but functional - hand grip, and one to use all this equipment. 

	\item How many nuclear engineers does it take to change a light bulb?: Seven. One to install the new bulb and six to figure out what to do with the old one for the next 10,000 years. 
\end{itemize}

\begin{center}\underline{\hspace{5 cm}}\end{center}

	\begin{figure}[H]
		\centering
		\includegraphics[scale=1]{img/humour/2bornot2b.jpg}
	\end{figure}
	
\begin{center}\underline{\hspace{5 cm}}\end{center}

This happens in Moscow: a couple of tourists ask for it's way, on a bridge to a Russian engineer.

The guy answer: "You are going through, and within 50 meters, turn right..."

Thanks from the tourists... And they leave. So the guy runs behind them:

"Wait, wait! I just remembered that the bridge is 70 meters. If you turn right after 50 meters, as I have told you, you will fall into the water".
\begin{center}\underline{\hspace{5 cm}}\end{center}

A quality engineers team is working on the FMECA for a new chemical factory. After several weeks, during the debriefing meeting: 

Quality engineers: "Our conclusion is: there's a 1/10,000 rate for the plant to blow up, killing many people and inducing terrible ecological crisis, it's ethically unacceptable !" 

Boss: "Standards are talking about an acceptable risk for a 1/7,000 rate" 

The team regroups and talk, then: 

Quality engineers: "Our conclusion is: You have an over-quality problem, it's ethically unacceptable!"

\begin{center}\underline{\hspace{5 cm}}\end{center}
	\begin{figure}[H]
		\centering
		\includegraphics[scale=1]{img/humour/airplane_magic.jpg}
	\end{figure}
\begin{center}\underline{\hspace{5 cm}}\end{center}

A guy was seated next to a 10-year-old girl on an airplane. Being bored, he turned to the girl and said, "Let's talk. I've heard that flights go quicker if you strike up a conversation with your fellow passenger."

The girl, who was reading a book, closed it slowly and said to the guy, "What would you like to talk about?"

"Oh, I don't know" said the guy. "How about nuclear physics?"

"OK" she said. "That could be an interesting topic. But let me ask you a question first. A horse, a cow and a deer all eat the same stuff... grass. Yet a deer excretes little pellets, while a cow turns out a flat patty, and a horse produces clumps of dried grass. Why do you suppose that is?"

The guy thought about it and said, "Hmmm, I have no idea."

To which the girl replied, "Do you really feel qualified to discuss nuclear power when you don't know shit?".

\begin{center}\underline{\hspace{5 cm}}\end{center}

Newton asked: How write $4$ in between $5$?

\begin{enumerate}
	\item Medicine students said: Joke!
	
	\item Science students said: Impossible!
	
	\item Management students said: Not found on the internet!
	
	\item Engineering student said: "F(IV)E"
\end{enumerate}

	\pagebreak
	\section{Computing}

	\begin{center}
	\includegraphics{img/humour/meaning_life.jpg}
	\end{center}
\begin{center}\underline{\hspace{5 cm}}\end{center}	

If you want to be a hacker, you will have to use Linux.

Here are 2 solutions :
\begin{enumerate}
	\item You are a capitalistic bourgeois and you buy it at Fry's for \$150.
	\item You are an asshole, and then you download it on the net.
\end{enumerate}

Of course you belong to the second category, so you have to use your FTP client and wait a few hours while your are downloading a Slack or a Debian. Try not to use Mandriva, this is for the public. You must not forget that you are an uNdERgrOuNd guy now, it's normal, you're a Hacker.

O.K., now you have got Linux, you can forget it. You do not need to lose your time learning how this new Operating System works and that you will never use because Xwing vs Tie Fighter doesn't run on it. The best way is to delete lilo, like that you will be sure to boot on Windows Vista. This elegant solution is practiced by many guys like you. The easiest way is to invoke fdisk /mbr in a DOS session, it will delete lilo which was installed on your hard drive's MBR. Good, you do not need to care about Linux anymore.

The goal is to have it, not to know how using it.

Okay, but then how can I show to everybody that I have Linux and that I am a rebel ?". 

That is a natural issue. Hopefully, I thought about you little looser, here comes sentences that you have to tell everybody about Linux:

\begin{itemize}
	\item "Linux is really powerful, you are free to do what ever you want with it compare to these fascists systems like Winblows. Anyway, M\$ is to crappy."

	\item "Well, if you are a beginner, you better not use Linux, this thing is for eLiteS. You, you are better using WinFuck."

	\item "Hey ! Where could I find the libc5.4.36 ? Because the 5.4.35 is not compatible with the modifications have done on my kernel."

	\item "Netscape sucks, it creates core dumps over 50 mo when it launches !I prefer using Lynx, text mode is much easier."

	\item "Wooooww what a fool ! He installed a Mandriva !! Only Debian is good, at least you know that you are the master of your system. No really, Mandriva is really to crappy."
\end{itemize}

With these sentences, you will quickly belong to the "okay, he's an asshole, but an asshole using Linux" category, this is the first step to become a real hacker. Now everybody knows you have Linux, you must move to the next stage, become the guy who knows everything about networks, who masters ICMP like a god. This is the second step of your long journey.

Now, you have to put your hands on the money. Go to Fry's and buy any books about Unix and networks. The main thing is a complicated title. A "rlogin protocol on ethernet sub-address" would have the best effect. Buy them even if you do not understand the titles, you just need to have impressive books : You are not suppose to read them, it is just to impress your friends, whom are assholes just like you.

The best way is to buy a book like "TCP/IP Volume 43" and learn by heart random words: socket, sub-address, FDDI, telnet, for example. Then you will use them in your sentences, even out of context, nobody will check what you are saying. For example, for feel to swing sentences like : "How many packets over FDDI networks does a telnet transmits ?". God, I swear it on a good cowboy channel, that will always impress and nobody will tell you that what you have just said has no meaning. Don't worry.

Then, put these books in your bedroom, the most complicated titles in the most visible areas. Damage a few pages' corners to make it credible. Take some paper and draw network diagrams, or add things like 123.44.5.34 root / lydia to make others believe you spend your days cracking passwords like a mad. Do not hesitated scanning Mitnick's photos and hang them above your bed, or put stickers of skulls on your computer to show that now, you are a thug, a dangerous guy.
To complete your new identity and truly become a hacker, you mustn't hesitate to say great things like : "I am thirsty for knowledge". Okay, you are in high school since 10 ten years, but it does not matter, you love learning anyway, hacking is a passion and you have a lot of willpower. Specify that you never do any damages to all the computers you hack into. Say you are doing that for "Intellectual Challenge". Yes, this time, you will have to force yourself not to laugh hard, so train yourself in front of the mirror before.

When people are dangerous like you, they must meet with other crooks in order to jeopardize the State's security. For this, there is THE thugs' rendez-vous, called the "Meet 2600". Every month, you will go to a MacDo in Paris, place of Italy, and there you will meet very important guys, who rebooted the entire Internet with a Visual Basic program and have special hair cuts like rebels of the society.

Okay, you will not learn much in this meetings, losers who go over there masturbate each other thinking "Yeah, we are hAcKeRz, we are ruthless, real men. Oh shit, it is already 6 pm, I have to go home otherwise my mum will kill me." But you will still feel real thrill thinking that the MacDo is full of cameras and microphones, and that the employees are agents from the DST who are listening to dangerous conversations such as :

\begin{itemize}
	\item[$-$] Asshole1: How much is the Whooper ?
	\item[$-$] Asshole2: Uh, MacDo does Whoopers now ?
	\item[$-$] Asshole1: I thought they always did, no 
\end{itemize}

The hAcKeRz' community also goes to raves. It is a part of the message "rebel no future fuck da society, we take ecstasy and listen to rubbish but we don't care, it is great because it is prohibited". Feel free to go in these places, it's a part of the lost culture to go in these hot parties.

You, you are a real hacker and you hear well to spread your knowledge in order to educate others like you. For this reason, there is e-zines. They include the best known as NoWay or NoRoute where the worse alongside the best (which is unfortunate for the best …) but there is also big shits who deserve to be more famous like the excellent Core-Dump who talks in an English that even my cat understands better than me.

Obviously, you have never read any books about Unix, you never hacked a machine in your life so you do not know what to write. Don't worry, you are not the only one to be in such a situation. The best thing to do is to write a rap article talking about your last rave or plunder anarchist magazines without understanding what you are talking about. If you decide plunder Phrack, do not hesitate to correct the guy or add more complicated stuff, nobody will check. Come on, free yourself, you are thirsty for knowledge, do not forget.

Now, for sure, you have truly become a hacker, a IRC rabble, an Internet thug, you are scaring government agencies and IBM wants to hire you to secure their network because this stupid Henry created a new Internet virus. So, you will have to, on daily basis, behave like a hacker, a real, a true, which means having a hacker's spirit and talking like a hacker.

A hacker primarily lives on IRC. Once your friends and your family will notice that you have changed, that you are not the same man anymore, you will have to spread the news on IRC in order to make new friends whom are assholes like you. Say goodbye to either \#flours nor \#friendship, now you will go down to the bottom of the IRC, the cyber-bronx, nuke-city, where only the real bruisers can be respected in this world of violence. To succeed, you will have to go from the asshole hacker status to an asshole on IRC who pretends being Mitnick, I mean a c0wb0y.
\begin{center}\underline{\hspace{5 cm}}\end{center}	

	\begin{center}
	\includegraphics[scale=0.8]{img/humour/quantum_computing.eps}
	\end{center}
\begin{center}\underline{\hspace{5 cm}}\end{center}

The Top 20 replies by programmers when their programs do not work:
	\begin{enumerate}[nolistsep]
		\item[20.] "That's weird..."
		\item[19.] "It's never done that before."
		\item[18.] "It worked yesterday."
		\item[17.] "How is that possible?"
		\item[16.] "It must be a hardware problem."
		\item[15.] "What did you type in wrong to get it to crash?"
		\item[14.] "There is something funky in your data."
		\item[13.] "I haven't touched that module in weeks!"
		\item[12.] "You must have the wrong version."
		\item[11.] "It's just some unlucky coincidence."
		\item[10.] "I can't test everything!"
		\item[9.] "THIS can't be the source of THAT."
		\item[8.] "It works, but it hasn't been tested."
		\item[7.] "Somebody must have changed my code."
		\item[6.] "Did you check for a virus on your system?"
		\item[5.] "Even though it doesn't work, how does it feel?
		\item[4.] "You can't use that version on your system."
		\item[3.] "Why do you want to do it that way?"
		\item[2.] "Where were you when the program blew up?"
	\end{enumerate}

	And the Number One reply by programmers when their programs don't work:
	\begin{enumerate}
		\item "It works on my machine."		
	\end{enumerate}


	\begin{center}\underline{\hspace{5 cm}}\end{center}	
	
	\begin{center}
	\includegraphics[scale=0.45]{img/humour/programing_languages.jpg}
	\end{center}
	
How the way people code "Hello World" varies depending on their age and job:

\begin{itemize}
	\item High School/Jr. High

 \texttt{10 PRINT "HELLO WORLD"\\
20 END}

	\item First year in College

\texttt{program Hello(input, output)\\
begin\\
writeln('Hello World')\\
end.}

	\item Senior year in College
	
\texttt{(defun hello\\
(print\\
(cons 'Hello (list 'World))))}

	\item New professional

\texttt{\#include <stdio.h>\\
void main(void)\\
\{\\
char *message[] = \{"Hello ", "World"\};\\
int i;\\
for(i = 0; i < 2; ++i)\\
printf("\%s", message[i]);\\
printf("\\n");\\
\}}

	\item Seasoned professional
	
\texttt{\#include <iostream.h>\\
\#include <string.h>\\
class string\{\\
private:\\
int size;\\
char *ptr;\\
public:\\
string() : size(0), ptr(new char('\textbackslash 0')) \{\}\\
string(const string \&s) : size(s.size)\\
\{\\
ptr = new char[size + 1];\\
strcpy(ptr, s.ptr);\\
\}\\
~string()\\
\{\\
delete [] ptr;\\
\}\\
friend ostream \& operator <<(ostream \& , const string \& );\\
string \& operator=(const char *);\\
\};\\
ostream \& operator<<(ostream \& stream, const string \& s)\\
\{\\
return(stream << s.ptr);\\
\}\\
string \&string::operator=(const char *chrs)\\
\{\\
if (this != \& chrs)\\
\{\\
delete [] ptr;\\
size = strlen(chrs);\\
ptr = new char[size + 1];\\
strcpy(ptr, chrs);\\
\}\\
return(*this);\\
\}\\
int main()\\
\{\\
string str;\\
str = "Hello World";\\
cout << str << endl;\\
return(0);
\}
}

	\item System Administrator

\texttt{\#include <stdio.h>\\
\#include <stdlib.h>\\
main()\\
\{\\
char *tmp;\\
int i=0;\\
tmp=(char *)malloc(1024*sizeof(char));\\
while (tmp[i]="Hello Wolrd"[i++]);\\
i=(int)tmp[8];\\
tmp[8]=tmp[9];\\
tmp[9]=(char)i;\\
printf("\%s \textbackslash n",tmp);\\
\}\\
}

	\item Apprentice Hacker

\texttt{\#!/usr/local/bin/perl\\
\$msg="Hello, world.\textbackslash n";\\
if (\$\#ARGV >= 0) \{
while(defined(\$arg=shift(@ARGV))) \{\\
\$outfilename = \$arg;\\
open(FILE, ">" . \$outfilename) || die "Can't write \$arg: \$! \textbackslash n";\\
print (FILE \$msg);\\
close(FILE) || die "Can't close \$arg: \$!\textbackslash n";\\
\}
\} else \{\\
print (\$msg);\\
\}\\
1;}

	\item Experienced Hacker

\texttt{\#include <stdio.h>\\
\#include <string.h>\\
\#define S "Hello, World\\n"\\
main()\{exit(printf(S) == strlen(S) ? 0 : 1);\}}

	\item Seasoned Hacker

\texttt{\%cc -o a.out ~/src/misc/hw/hw.c\\
\%a.out\\
Hello, world.\\
Guru Hacker\\
\%cat
Hello, world.}

	\item New Manager (do you remember?)

\texttt{10 PRINT "HELLO WORLD"\\
20 END}

	\item Middle Manager

\texttt{mail -s "Hello, world." bob@b12\\
Bob, could you please write me a program that prints "Hello, world."?\\
I need it by tomorrow.\\
\^D\\}

	\item Senior Manager

\texttt{\%zmail jim\\
I need a "Hello, world." program by this afternoon.}

	\item Chief Executive

\texttt{\%letter\\
letter: Command not found.
\%mail\\
To: \^X \^F \^C\\
\%help mail\\
help: Command not found.\\
\%damn!\\
!:Event unrecognized\\
\% logout\\}

	\item Research Scientist

\texttt{PROGRAM HELLO\\
PRINT *, 'Hello World'\\
END}

	\item Older research Scientist

\texttt{WRITE (6, 100)\\
100 FORMAT (1H ,11HHELLO WORLD)\\
CALL EXIT\\
END}

\end{itemize}
\begin{center}\underline{\hspace{5 cm}}\end{center}	

	\begin{center}
	\includegraphics[scale=0.8]{img/humour/punition.jpg}
	\end{center}
	
\begin{center}\underline{\hspace{5 cm}}\end{center}

What the software engineers says... and what must be understood:
\begin{itemize}
	\item We will put this project on schedule: We will take care of it if we have nothing else to do

	\item This is a completely new program!: It's absolutely not compatible with the old version

	\item This program requires no maintenance: It is impossible to debug

	\item This program requires little maintenance: It's almost impossible to debug

	\item We will respect the standards: It has always been like that and it's not going to change now that

	\item We want to respect the standards: You're not going to control everything that we do

	\item The new version of this program is 100\% compatible with the previous: We did not touch anything

	\item Different approaches have been tried: We still trying to guess what happens.

	\item We approach a solution: We met for coffee...

	\item The preliminary tests were not satisfying: This damn program crashed as soon as we launch it

	\item We'll have to abandon the entire concept: The only person who understood something just resigned

	\item We prepare a comprehensive report, according to an entirely new approach: We just hired three newbie who left school

	\item This is a major breakthrough: We still can not understand why it does not work

	\item This is the result of years of development: We were finally able to operate a piece of the program ...

	\item We are working on it: We are so in trouble that it's hopeless

	\item Tell us what you think: We will listen to what you have to say as it does not undermine what is already done, or what we have decided to

	\item We'll take a look: Forget it! We have enough problems like that ...

	\item I have not received your e-mail: It's been ages that I have not checked my email...
\end{itemize}

	\begin{center}
	\includegraphics[scale=0.3]{img/humour/tesla_ipaddress.jpg}
	\end{center}

	\pagebreak
	\section{Social Sciences}
	
To understand the marketing vocabulary..... and avoid appearing ridiculous in an evening party with colleagues:
\begin{enumerate}
	\item Michael is at a party and sees a very attractive girl. He approaches her and says: "I am a very good shot". This is named the "direct marketing".

	\item Michael is at a party with a group of friends and he sees a very attractive girl. A friend approached her and said, "You see that boy there, it's a very good shot". This is named the "advertising".

	\item Michael is at a party and sees a very attractive girl. He asks her phone number. The next day he called and said, "I am a very good shot". This is what we name "telemarketing".

	\item Michael is at a party and sees a very attractive girl. He recognized and approaches her, he refreshes her memory by saying: "You remember that I am a very good shot?". This is named the "Customer Relationship Management (CRM)".

	\item Michael is at a party and sees a very attractive girl. He stands up, arranges a little, approaches her and serves as a glass. He opens the door when she leaves, picks up his bag when he falls, offers her a cigarette and said, "I am a very good shot". This is what we name "public relations" or "public relations" (PR).

	\item Michael is at a party and sees a very attractive girl. He invites all her girlfriends to dance, offer them a drink and laugh withostensibly very spiritual jokes. The beautiful girl approach and say, "I feel that you are a very good shot". This is named "lobbying".

	\item Michael is at a party and sees a very attractive girl. She approach him and said: "I heard you're a very good shot". This is named the "brand power".

	\item Michael is at a party and see a super beautiful girl. He looks at her with his friends, doing very fine reflections, completely drunkes, does nothing and return alone at home. This is named the "market reality"...
\end{enumerate}

\begin{center}\underline{\hspace{5 cm}}\end{center}
	\begin{center}
	\includegraphics{img/humour/fluid_dynamics.jpg}
	\end{center}

	\begin{center}
	\includegraphics{img/humour/stupid.eps}
	\end{center}
\begin{center}\underline{\hspace{5 cm}}\end{center}

A man is flying in a hot air balloon and realizes he is lost. He reduces height and spots a man down below. He lowers the balloon further and shouts: "Excuse me, can you tell me where I am?" 
\begin{itemize}
	\item[$-$] The man below says: "Yes, you're in a hot air balloon, hovering 30 feet above this field." 

	\item[$-$] "You must be an engineer" says the balloonist. 

	\item[$-$] "I am" replies the man, "How did you know?" 

	\item[$-$] "Well" says the balloonist, "everything you have told me is technically correct, but it's no use to anyone". 

	\item[$-$] The man below says "you must be in management". 

	\item[$-$] "I am" replies the balloonist, "but how did you know?" 

	\item[$-$] "Well," says the man, "you don't know where you are, or where you're going, but you expect me to be able to help. You're in the same position you were before we met, but now it's my fault." 
\end{itemize}
\begin{center}\underline{\hspace{5 cm}}\end{center}

A man goes on a Saturday at a wedding in a Corsica small village. He's late and he drives as fast as possible on winding roads. Suddenly, after a turn, it must stop short, a flock of sheep occupies the entire road. The shepherd is there and slowly move his flock. The driver uses the horn of his vehicle several times without any effect. After a few minutes, the driver apostrophe the shepherd and says:
\begin{itemize}
	\item[$-$] "I'm late, I'm going to a wedding that will be followed by a barbecue, if I tell you how many sheep you have, you will you give me one of your sheep?"

	\item[$-$] "For sure" said the shepherd, "I am not close to one". 

	\item[$-$] The driver takes his calculator and after a minute announce: "1233".

	\item[$-$] "You won" say the shepherd, "Choose your pet".
\end{itemize}
The driver designate one. The shepherd then ask:
\begin{itemize}
	\item[$-$] If I find what is your profession, you give me back my beast?

	\item[$-$] "Of course" say the driver, "I'm listening".

	\item[$-$] "You are a high level official public servant and you've done and high level administration school or another big school like this."

	\item[$-$] "You are right" say the driver, "But how did you guess?".

	\item[$-$] The shepherd: "Please give my dog back"
\end{itemize}
\begin{center}\underline{\hspace{5 cm}}\end{center}

	\begin{center}
	\includegraphics[scale=0.7]{img/humour/meeting_girls.jpg}
	\end{center}

	\begin{center}
	\includegraphics{img/humour/fibonaughty_sexquence.jpg}
	\end{center}

	\begin{table}[H]
		\centering
			\begin{tabular}{c m{0.1cm} c m{0.1cm} c}
		    \begin{minipage}{.3\textwidth}
    		\center \includegraphics{img/humour/worker.eps}\\
		    \center Worker
		    \end{minipage}
	    	&
			+
			& 
		    \begin{minipage}{.3\textwidth}
    		\center \includegraphics{img/humour/process.eps}\\
		    \center Processs
		    \end{minipage}
		    &
		    =
		    &
		   	\begin{minipage}{.3\textwidth}
    		\center \includegraphics{img/humour/engineer.eps}\\
		    \center Engineer
		    \end{minipage}
	    \\
		    \begin{minipage}{.3\textwidth}
    		\center \includegraphics{img/humour/engineer.eps}\\
		    \center Engineer
		    \end{minipage}
	    	&
			+
			& 
		    \begin{minipage}{.3\textwidth}
    		\center \includegraphics{img/humour/sociability.eps}\\
		    \center Sociability
		    \end{minipage}
		    &
		    =
		    &
		   	\begin{minipage}{.3\textwidth}
    		\center \includegraphics{img/humour/marketing.eps}\\
		    \center Marketing
		    \end{minipage}
	    \\
		    \begin{minipage}{.3\textwidth}
    		\center \includegraphics{img/humour/marketing.eps}\\
		    \center Marketing
		    \end{minipage}
	    	&
			-
			& 
		    \begin{minipage}{.3\textwidth}
    		\center \includegraphics{img/humour/truth.eps}\\
		    \center Truth
		    \end{minipage}
		    &
		    =
		    &
		   	\begin{minipage}{.3\textwidth}
    		\center \includegraphics{img/humour/commercial.eps}\\
		    \center Commercial
		    \end{minipage}
	    \\
		    \begin{minipage}{.3\textwidth}
    		\center \includegraphics{img/humour/commercial.eps}\\
		    \center Commercial
		    \end{minipage}
	    	&
			-
			& 
		    \begin{minipage}{.3\textwidth}
    		\center \includegraphics{img/humour/brain.eps}\\
		    \center Brain
		    \end{minipage}
		    &
		    =
		    &
		   	\begin{minipage}{.3\textwidth}
    		\center \includegraphics{img/humour/manager.eps}\\
		    \center Manager
		    \end{minipage}
	    \\
		    \begin{minipage}{.3\textwidth}
    		\center \includegraphics{img/humour/manager.eps}\\
		    \center Manager
		    \end{minipage}
	    	&
			+
			& 
		    \begin{minipage}{.3\textwidth}
    		\center \includegraphics{img/humour/ego.eps}\\
		    \center Ego
		    \end{minipage}
		    &
		    =
		    &
		   	\begin{minipage}{.3\textwidth}
    		\center \includegraphics{img/humour/project_manager.eps}\\
		    \center Project Manager
		    \end{minipage}
	    \\
	   		\begin{minipage}{.3\textwidth}
    		\center \includegraphics{img/humour/project_manager.eps}\\
		    \center Project Manager
		    \end{minipage}
	    	&
			-
			& 
		    \begin{minipage}{.3\textwidth}
    		\center \includegraphics{img/humour/humour.eps}\\
		    \center Humour
		    \end{minipage}
		    &
		    =
		    &
		   	\begin{minipage}{.3\textwidth}
    		\center \includegraphics{img/humour/hr.eps}\\
		    \center Human Ressources
		    \end{minipage}
	    \\	    
		\end{tabular}
	\end{table}

	\begin{center}
	\includegraphics[scale=0.7]{img/humour/xmas.jpg}
	\end{center}
	
	\begin{center}\underline{\hspace{5 cm}}\end{center}	
	\begin{center}
		\includegraphics[scale=0.8]{img/humour/evolution.jpg}
	\end{center}