	%to force start on odd page
	\newpage
	\thispagestyle{empty}
	\mbox{}
	\section{Astronomy (Celestial Mechanics)}
	\lettrine[lines=4]{\color{BrickRed}C}elestial mechanics is the consequence of the universal Newton's law of attraction  and of the fundamental principle of mechanics (\SeeChapter{see section Classical Mechanics}). Its main objective is the description of the motion of astronomical objects such as stars and planets using physical and mathematical theories.
	
	In this section we will approach the subject as always on this book, in the most elementary  possible way (to this day the topics in this section are not technically beyond the level of what was done in the beginning of 20th century in the field of astronomy).
	
	First we will make a warm up with a funny law on the living in the Universe ... (the Drake equation). Once completed this warm up, we will begin to "enumerate" Kepler's laws (often referring to the section of Classical Mechanics) and then study in detail the properties of Keplerian orbits thanks to our knowledge on classical mechanics and then to using Special Relativity, which will lead us to find a theoretical precessions of studied orbitals. Then we will have fun to model approximately the variation of the duration of the day (or night) on the Earth based on the month and latitude. Finally, to finish in style, we will launch the detailed calculation of the five Lagrangian points!
	
	\subsection{Drake Equation}
	
	This equation was invented (...) by F. Drake in the 1960s with the intention to discuss the number of extraterrestrial civilizations in our galaxy with which we might come in contact in the context of the SETI program (Search for ExtraTerrestrial Intelligence). The main purpose of this equation for scientists is to determine its factors, in order to know the likely number and (very) estimated extraterrestrial civilizations.
	
	This empirical equation which remains more something funny and provocative than something else... and  whose principle can be applied to a lot of different areas of physics and life is written:
	
	The terms of this formula (because it is a formula and not a relation!) are defined as follows:
	\begin{itemize}
		\item $N^{*}$ represents the number of stars in a single galaxy
		\item $f_p$ is the fraction of stars that would have an orbiting planet (between $0$ and $1$)
		\item $n_e$ is the number of planets per star that fulfill the conditions for the development of life
		\item $f_l$ is the fraction of planets whose life has emerged (between $0$ and $1$)
		\item $f_i$ is the fraction of those where an intelligent life has emerged (between $0$ and $1$)
		\item $f_c$ is the fraction of $f_i$ which has implemented radio communication technology (between $0$ and $1$)
		\item $f_l$ is the fraction of time during which the fraction $f_i$ civilizations will live (between $0$ and $1$)
	\end{itemize}
	In practice, it should be noted that this formula purpose is to try to determine an unknown amount from other amounts that are also unknown ... But it's a nice and funny formula to evaluate when you discuss with friends at the restaurant...
	
	There is therefore no guarantee that we are more knowledgeable after the estimate of this formula than before (method sometimes named in the literature "garbage in, garbage out"...).
	
	The resulting value can motivate the fact that following mathematical developments are not only applicable to only one solar (star) system in the universe... maybe... (it would make a lot of useless empty space otherwise...).
	
	Let us talk now about the "\NewTerm{Fermi paradox}\index{Fermi paradox}"  named after physicist Enrico Fermi, is the apparent contradiction between the lack of evidence and high probability estimates, e.g., those given by the Drake equation, for the existence of extraterrestrial civilizations. The basic points of the argument, made by physicists Enrico Fermi (1901–1954) and Michael H. Hart (born 1932), are:
	\begin{itemize}
		\item There are billions of stars in the galaxy that are similar to the Sun, many of which are billions of years older than Earth.
		\item With high probability, some of these stars will have Earth-like planets, and if the Earth is typical, some might develop intelligent life.
		\item Some of these civilizations might develop interstellar travel, a step the Earth is investigating now.
		\item Even at the slow pace of currently envisioned interstellar travel, the Milky Way galaxy could be completely traversed in a few million years.
	\end{itemize}
	According to this line of reasoning, the Earth should have already been visited by extraterrestrial aliens. In an informal conversation, Fermi noted no convincing evidence of this, leading him to ask, "Where is everybody?" There have been many attempts to explain the Fermi paradox, primarily either suggesting that:
	\begin{itemize}
		\item Extraterrestrial life is rare or non-existent
		\item No other intelligent species have arisen
		\item Civilizations lack advanced technology
		\item It is the nature of intelligent life to destroy itself
		\item It is the nature of intelligent life to destroy others
		\item There is periodic extinction by natural events
		\item Intelligent civilizations are too far apart in space or time
		\item It is too expensive to spread physically throughout the galaxy
		\item Human beings have not existed long enough
		\item Humans are not listening properly
		\item Civilizations broadcast detectable radio signals only for a brief period of time
		\item Civilizations tend to isolate themselves
		\item Everyone is listening, no one is transmitting	
		\item Earth is deliberately not contacted
		\item It is dangerous to communicate
		\item ...
	\end{itemize}
	
	\subsection{Kepler's Laws}
	
	In astronomy, Kepler's laws describe the main properties of the motion of planets around a main star, without explaining the reason (at least at the time these laws were developed!). They were discovered by Johannes Kepler based on the observations and measurements (in phenomenal amount) of the position of the planets made by Tycho Brahe, measures that were very accurate for its time.
	
	The first two Kepler's law seems to were published in 1609 and the third in 1618. The elliptical orbits, as set out in its first two laws can explain the complexity of the apparent motion of the planets.
	
	Soon after, in 1687 Isaac Newton discovered the law of gravitational attraction, deducting from it, by calculation, the three Kepler's laws.
	
	We will now try to present these laws in the most relevant possible way:
	
	\subsubsection{First Kepler's Law (conicity law)}
	
	The "\NewTerm{first law of Kepler}\index{first law of Kepler}", sometimes also named "\NewTerm{conicity law}\index{conicity law}" or "\NewTerm{law of orbits}\index{law of orbits}" is stated most of time as follow: The orbits of the planets are conics (ellipses) which the Sun (central star) occupies one of the focals.
	
	In fact, it should be noted that this is not really a "law" in the proper sense, since further below you will see that we can prove that:
	
	
	\begin{tcolorbox}[title=Remark,colframe=black,arc=10pt]
	The reader who has already read the section of Analytic Geometry will not be surprised by this relation...
	\end{tcolorbox}
	
	\subsubsection{Second Kepler's Law (area law)}
	The "\NewTerm{Kepler's second law}\index{Kepler's second law}", sometimes also named "\NewTerm{area law}\index{are law}" tells us that the line joining a planet to the Sun (central start) sweeps out equal areas in equal times (constant areal velocity) as:
	
	
	It is a relation that arises from the conservation of angular momentum as we have already shown it in the section of Classical Mechanics where we got:
	
	
	So again, the status of "law" is questionable in the language of modern physics!
	
	Now let us express this law in another form more conventional in the field of astronomy. Consider for this the movement in the plan in cylindrical coordinates by:
	
	Therefore:
	
	
	It comes therefore from the property of linearity of the vector product:
	
	Therefore taking the norm:
	
	And since it is equal to a constant, it is often customary to write this last equality in a condensed form (and putting the mass in the constant):
	
	Also, remember that we also got the result that the movement is and remains in a plane without any outside action!
	
	We note also that this law gives us the speed of the planet is variable. It is larger than at perihelion than a the aphelion:
	\begin{figure}[H]
		\begin{center}
		\includegraphics{img/cosmology/focus_aphelion_perihelion.jpg}
		\end{center}	
		\caption{Representation of surfaces conservation}
	\end{figure}
	This is true for the Earth for example. Indeed, this latter is closer to the sun in winter (for northern hemisphere) and then has a trajectory speed slightly higher than in summer; the travel time is therefore lower (winter has fewer days than the other seasons).
	
	\paragraph{Time of flight}\mbox{}\\\\
	We propose now to apply the second Kepler's law to determine the time $t$ from the passage to the perihelion as a function of the "\NewTerm{eccentric anomaly}\index{eccentric anomaly}" $\varphi$ in the case of an elliptic orbit  (thus special case!) using its two foci (one of them being assimilable for example to the position of the Sun) and the origin of the "\NewTerm{auxiliary circle}\index{auxiliary circle}" (also named "\NewTerm{apsidal circle}\index{apsidal circle}"):
	\begin{figure}[H]
		\begin{center}
		\includegraphics{img/cosmology/excentricity_anomaly.jpg}
		\end{center}	
		\caption{Schema for the study of eccentric anomaly angle}
	\end{figure}
	To determine the time $t$ between the passage at the perihelion $A$ and the point $P$ of a body following the trajectory of the ellipse of surface $\pi a b$ (see the Geometric Forms section for the proof of the calculation of the surface of the ellipse) in function of the eccentric anomaly $\varphi$, we will use the areas law just proved earlier above (second second Kepler's law) that give us the right to write:
	
	But, the surface of the ellipse is an affine transformation of the surface of the auxiliary circle such that:
	
 	We have then:
	
 	If $\varphi$ is, as it should, expressed in gradients, we have of course:
	
 	Therefore:
	
 	For $S_{F\text{O}Q}$, we have $\overline{F\text{O}}=a$ therefore equal to the radius of the auxiliary circle. The surface of the triangle $S_{F\text{O}Q}$, knowing its height given by $h=a\sin(\varphi)$ is then obtained by:
	
	Therefore we have:
	
	But, by definition of the eccentricity (\SeeChapter{see section Analytical Geometry}), we can write:
	
	Finally, we have:
	
	Therefore:
	
 	where the angle taken at the center of the ellipse is for recall named the "eccentric anomaly".

	\textbf{Definition (\#\mydef):} In the description of the Keplerian orbit of a celestial object, the "\NewTerm{eccentric anomaly}\index{eccentric anomaly}" is the angle between the direction of the periapse and the current position of an object in its orbit, projected on the circle extinct perpendicular to the major axis of the ellipse

	What would interest us now would be to find a relation of passage between this eccentric anomaly and the angle named "\NewTerm{true anomaly}\index{true anomaly}" $\theta$ as sometimes it is often more advantageous to use this last angle.

	For a relation between $\varphi$ and $\theta$, we will reuse our above schema but modified a bit:
	\begin{figure}[H]
		\begin{center}
		\includegraphics{img/cosmology/true_cosmology.jpg}
		\end{center}	
		\caption{Schema for the study of true-eccentric anomaly angle relation}
	\end{figure}
	We have obviously the $4$ below relations which a deduce from the above figure:
	
	We then have already in a first time (relation which will be useful to us a little later):
	
 	We have also proved in the section of Analytical Geometry that:
	
	this relation being valid at any border point of the ellipse. Thus, we also have:
	
 	Which leads us to write:
	
	Ideally, we could get rid of the radius in the denominator. For this, we will use the fact that (relations that we have just proved earlier above):
	
	We have:
	
 	The terms to the left of the equality are simplified immediately:
	
	hence:
	
 	therefore:
	
	thus finally:
	
	Finally notice that in the special case of an elliptical orbit, we deduce thanks to the second Kepler's law (see the proof of the calculation of an area of an ellipse in the section Geometric Shapes) the:
	
	\pagebreak
	\subsubsection{Third Kepler's Law (periods' law)} 
	The "\NewTerm{Kepler's third law}\index{Kepler's third law}", sometimes also named  "\NewTerm{Periods' law}\index{Periods' law}" or "\NewTerm{Kepler's harmonic law}\index{Kepler's harmonic law}", is stated as follow: The squares of the periods of revolution $T$ are proportional to the cube of the semi-major axes of the orbits $D$:
	
	The last ratio is then a constant in practice for all planets (in physics we also speak of "invariant") and the reason for this "harmony" was difficult to explain before Newton's theory.
	
	Again, we will see later that the status of "law" is no longer justified in our time as it is possible to prove that this relation, whose expression will be detailed, is in reality:
	
	and therefore we understand better when we see the term on the right why we had the previous constant ($m$ is the central body mass!).
	
	The last relation is more often written as following in the field of astronomy:
	
	where as we have seen in the section of Analytical Geometry, $a$ is the traditional notation for the apogee radius (semi-major axes).
	
	The most commonly used rearrangement of this la relation is obviously:
	
		
	Of course, Kepler did not immediately published his three laws in this provocative simplicity. Their current presentation order is also not the original one ... They are  in reality to find among  a profusion of physical speculations and reflections on world's harmony.
	
	\begin{tcolorbox}[title=Remark,colframe=black,arc=10pt]
	The Kepler's law are not limite to the gravitation force. They also apply for all acceleration (or force) of the type $1/r^2$. And this is also the case of the Coulomb's law (\SeeChapter{see section Electrostatic}). Kepler's law can therefore also be applied to an electron in orbit around a nucleus. The Borh-Sommerfeld model (\SeeChapter{see section Quantum Corpuscular Physics}) based also on Kepler's law gives also elliptic trajectories for electrons!
	\end{tcolorbox}	
	
	The three Kepler's law can be resume by the following small figure:
	\begin{figure}[H]
		\begin{center}
		\includegraphics{img/cosmology/keplerslaw.jpg}
		\end{center}	
		\caption{Summary of Kepler's law in image (source: ???)}
	\end{figure}
	
	The reader must take precautions with the image above because:
	\begin{enumerate}
		\item The planets are most of time in a movement that is not in the same plane. For the Solar system it is the tradition  at high school level to represent the planets in the "\NewTerm{ecliptic}\index{ecliptic}" plane that is the average plane described by the movement of Jupiter around the Sun:
		\begin{figure}[H]
			\begin{center}
			\includegraphics{img/cosmology/ecliptic_solar_system.jpg}
			\end{center}	
			\caption{Planets spin and angle relatively to the ecliptic plane}
		\end{figure}
		
		\item The orbits precess around the Sun as shown in the following figure (see further below for the mathematical proof in the cas of a 2-body system):
		\begin{figure}[H]
			\begin{center}
			\includegraphics{img/cosmology/orbit_precession.jpg}
			\end{center}	
			\caption{Orbit precession example}
		\end{figure}
		
		\item The planets have an helicity trajectory "behind" the movement of the Sun around the center of our Galaxy and the ecliptic has an angle of approximately $60^\circ$ ($\pi/3$ [rad]) relatively to the perpendicular of the Sun movement as visible in the figure below:
		\begin{figure}[H]
			\begin{center}
			\includegraphics{img/cosmology/solar_system_vortex.jpg}
			\end{center}	
			\caption{Planets with orbits following the Sun in its movement}
		\end{figure}
		and the planets are therefore sometimes in front of the Sun and sometimes behind.
		
		\item On the very very long term the orbits in a $n$-body system is a chaotic deterministic system that has period of quasi-stability but that sometimes diverges completely. This is great opportunity at the date we write these lines to be in such a period of quasi-stability.
	\end{enumerate}	
		
	\pagebreak
	\subsection{Newton Gravitational Law}
	To check the accuracy of his hypothesis, Newton (relatively long after Kepler) found Kepler's laws from the law of gravity, giving the explanation of the general movement of the planets.
	
	Newton considered to determine the law of gravitation a theoretical planet orbiting around the Sun in a circular orbit at a constant speed $v$. During a complete orbit the planet travels a distance equal to the circumference of the circle of radius $R$, or $2\pi R$, in a time (the period) equal to the distance divided by its velocity, either:
	
	Newton then relies on the third Kepler's law with always the assumption of a circular orbit.
	
	We therefore have:
	
	but as:
	
	Then we get by substitution:
	
	By comparing:
	
	and:
	
	and now assuming that $4\pi^2$ is divided by the constant is a new constant (which will be denoted in the same manner as the first although it it is not equal to...) we obtain:
	
	Therefore:
	
	Then, if we reverse the terms, this expression becomes (while noting that the inverse of the original is constant is, also, a constant):
	
	By another calculation, we have already established in the section of Classical Mechanics the expression of centrifugal force:
	
	by comparing this expression with the previous one:
	
	we get:
	
	There should therefore exist a force opposed to the centrifugal force that keeps the orbital cohesion and which can be written:
	
	remains to determine the value of the constant!
	
	It is trivial that the central mass $M$ of the orbital system has to intervene in one way or another in this constant. If the mass of the secondary body intervenes proportionally in the centrifugal force, the desire is great to do the same with the mass of the central body. So:
	
	Now there would be a priori more parameters to take into account. The remaining constant is here to meet the dimensional analysis so that we have "Newtons" (name given to the unit of force) on both sides of the equality. Scientists have determined with precision this "\NewTerm{gravitational constant}\index{gravitational constant}" denoted by $G$ that a priori seems universal and which in SI units, the 2014 CODATA-recommended value of the (with standard uncertainty in parentheses) is:
	
	Which brings us to write the "\NewTerm{Newton Gravitational law}\index{Newton Gravitational law}":
	
	Obviously it is not a true rigorous proof because based on experimental Kepler's observations. By cons, from General Relativity it is possible to prove it (under some given assumptions...)!
	\begin{tcolorbox}[colframe=black,colback=white,sharp corners]
	\textbf{{\Large \ding{45}}Example:}\\\\
	At the Earth's equator the radius is of $6378$ [km] and at the poles of $6357$ [km]. Therefore we have:
	
	Then the acceleration at the equator is equal to $9.800/9.865\cong 99.34 \%$ to that at the poles.
	\end{tcolorbox}
	It is very important to notice that the mutual forces of attraction acting on two mass spheres are always of equal size!
	
	Using Maple 17.00, we can simulate the plane trajectory of a satellite relative to $n$ number of fixed mass (thanks to Forhad Ahmed for his script). Here below is given the basic script that you can customize to your tastes and... feel free to give us your personal work if you have brought significant improvement to this script:
	
	\texttt{>restart; with(plots); with(DEtools)\\
	>G:=1; \#normalized gravitational constant to simplify\\
	>poles:=2; \#number of bodies/masses that we can play with...\\
	>M[1]:=10;M[2]:=1; \#mass of the first and second body (in relative values)\\
	>h[1]:=1;h[2]:=-1; \#X position of the first and second body X (in astronomical units)\\
	>k[1] := 1;k[2] := 1; \#Y position of the first and second bodies (in astronomical units)\\}
	
	\texttt{>\#differential equation of the satellite acceleration in X\\
	>Xeq := diff(x(t), t, t) = sum(-G*M[j]*(x(t)-h[j])/((x(t)-h[j])\string^2\\+(y(t)-k[j])\string^2)\string^(3/2), j = 1 .. poles);\\
	>\#differential equation of the satellite acceleration in X\\
	>Yeq := diff(y(t), t, t) = sum(-G*M[j]*(y(t)-k[j])/((x(t)-h[j])\string^2\\+(y(t)-k[j])\string^2)\string^(3/2), j = 1 .. poles);\\
	>\#position and initial velocity of the satellite\\
	>inits := x(0) = -2, y(0) = 0, (D(x))(0) = 0, (D(y))(0) = 2\\
	>\#numerical solution of the differential equation (you can play with the precision of the error as needed!)\\
	>g:=dsolve({Xeq,Yeq,inits},{x(t),y(t)},type=numeric,method=dverk78,abserr=0.1e-3, output= procedurelist);\\}
	
	\texttt{>n:=50; \#step of iterations\\
	>iter:=300; \#step of iterations\\}
	
	\texttt{>\#loop that resolves the differential equation at each new iteration\\
	>for i from 0 to iter do \\
	px[i]:=rhs(g(i/n)[2]);\\
	py[i]:=rhs(g(i/n)[4]);\\
	KE[i]:=1/2*(rhs(g(i/n)[3])\string^2+rhs(g(i/n)[5])\string^2);\\
	temp:=(rhs(g(i/n)[2])-h[j])\string^2+(rhs(g(i/n)[4])-k[j])\string^2;\\
	PE[i]:=sum(-G*M[j]/sqrt(temp), j = 1 .. poles);\\
	TE[i]:=KE[i]+PE[i]\\
	end do:\\}
	
	\texttt{>data:=seq(pointplot([px[i], py[i]], color = red), i = 0 .. iter):\\
	>\#mettre insequence à true pour avoir une animation\\
	>Anim:=display(data,insequence=false,scaling=constrained,axes=boxed):\\
	>stars:=display(seq(pointplot([h[i], k[i]], color = black), i = 1 .. poles))\\
	>display({Anim,stars},title=`Satellite orbiting a multipolar gravity field`);\\}

	\begin{figure}[H]
		\begin{center}
		\includegraphics{img/cosmology/trajectory_of_a_body_influenced_by_massive_body.jpg}
		\end{center}	
		\caption{Configuration for the study of relativistic effects}
	\end{figure}

	\texttt{>\#it is verified that the total energy of the satellite is always constant\\
	>print(`[Time] -- [Kinetic Energy] - [Potential Energy] - [Net Energy]`);\\
	>print(`======================================`);\\
	> for i by 3 to iter do\\
	print(evalf(i/n, 6), ` `, KE[i], ` `, PE[i], ` `, TE[i]);\\
	end do:\\
	>\#the last column of the table must always have normally an equal value...}
	
	\begin{tcolorbox}[title=Remark,colframe=black,arc=10pt]
	Equalizing the centrifugal force and gravitational force, it is quite easy to get an approximation of the speed of rotation of the planets in their orbits. The reader that will do the calculation will see that the value for the planets of our solar system is around a speed of about $100,000$ [km/h].
	\end{tcolorbox}	
	
	From this last relation, let us come back briefly on our third Kepler's law and detail it a little bit to show that it is valid for any type of conical orbit and to determine the expression of its constant.
	
	Expressed in the Frenet coordinate system (\SeeChapter{see section Differential Geometry}), and decomposed into its normal (centripetal) and tangential acceleration, the acceleration in respect to a geocentric reference frame (in the case of a referential located at the mass center of the system the expression change a little bit!) is written:
	
	From in previous developments (3rd Kepler's Law):
	
	and:
	
	the constant of Kepler's third law takes for value (it is a formulation sometimes used in practice but not a strictly necessary step in this development):
	
	but as we also have:
	
	then:
	
	Therefore:
	
	Finally, the third Kepler law can be found frequently in the literature as follows:
	
	But now let us consider again our figure of the center of mass study (\SeeChapter{see section Classical Mechanics}):
	\begin{figure}[H]
		\centering
		\includegraphics{img/atomistic/hydrogenoid_center_of_mass.jpg}
		\caption[]{Binary System Center of Mass (profile view)}
	\end{figure}
	And let us have a look at circular orbits in the center of mass frame! 
	
	First we look at the forces acting on body $M$:
	
	The forces balance, so:
	
	We know that in the simple case of a circular orbit (circular kinematics):
	
	We insert this into the previous equation and we get after some algebra:
	
	What is $r_M$? From the definition of center of mass we know that (\SeeChapter{see section Classical Mechanics}):
	
	We plug $r_M$ into our equation and now we get:
	
	or after rearranging terms:
	
	So we can use this 3rd Kepler's law to determine the total mass a binary pair if we know the period $T$!!! As the star masses are well estimated using the HR diagram we better understand how astronomers estimate orbiting planet mass knowing the period (in fact they also use the luminosity variation). In fact it is not as simple as there is no reason why we should be looking directly onto the orbital plane of the binary system. In other words, the apparent orbit is almost never the true orbit (which is what we need to do the calculation).
	
	This interlude performed, let us come back on our Newton's gravitation law:
	
	From the law of gravitation, we can find back Kepler's law. Besides, we have already done it for the second and third law of Kepler, since it is these that we used to get this latter relation (however it's a little bit the snake eating its tail...).
	
	In vector notation we have therefore:
	
	Identically to the electric field (\SeeChapter{see section Electrostatic}), we can develop:
	
	As the electric field is derived from an electric potential, identically, the gravitational field derived from a gravitational potential. By performing exactly the same development as in  our study of electromagnetism for the first Maxwell equation (\SeeChapter{see section Electrodynamics}), we prove that:
	
	where $\varphi$ is the "\NewTerm{gravitational potential}\index{gravitational potential}" that varies inversely with the relative distance of the body (this confirms what we had proved in our study of Noether's theorem in the section on Principles) and is therefore equal to:
	
	\begin{tcolorbox}[title=Remark,colframe=black,arc=10pt]
	We often encounter this potential in the section of General Relativity. It is therefore appropriate to remember it if possible!
	\end{tcolorbox}	
	Notation which obviously implies the following relation:
	
	\begin{tcolorbox}[title=Remark,colframe=black,arc=10pt]
	Obviously in the absence of field, we have $\varphi=c^{te}$ and therefore $\vec{a}$  will be zero.
	\end{tcolorbox}
	As in the section of Electromagnetism, again, we prove as we did for the first Maxwell equation:
	
	If we express this equation in terms of a gravitational potential $\varphi$ (also often denoted by the letter $U$ as in Electrostatic...), we get:
	
	that we write more aesthetically with the scalar Laplacian operator (\SeeChapter{see section Vector Calculus}):
	
	which is nothing else than the "\NewTerm{Newton-Poisson equation}\index{Newton-Poisson equation}" that we will  also meet again in our study of General Relativity (it has an important place for validation reasons of Einstein's Gravitation theory)!
	
	This equation means that the Newtonian gravitational theory can be resume to say that the gravitational field is described by a single potential $\varphi$ generated by the volume mass density and determining the acceleration of a test particle immersed in the outfield $\varphi$.
	
	Let's have a little bit fun now with the Newton's gravitation equation to get some interesting and curious results:
	
	\subsubsection{Gaussian Formulation of Newtonian Gravity}
	As we have just mentioned it and proved in the section of Electrodynamics, an alternative formulation of Newtonian gravity is: Gauss’s Law for gravity. It states that the acceleration $\vec{a}$ due to gravity of a mass $m$ (not necessarily a point mass) is given by:
	
	where the $-$ sign we have it's purpose of guarantee a positive scalar acceleration.
	
	For example, let's use Gauss's law to find the acceleration $a$ due to the gravity of a point mass $m$.

	We begin with a point mass m sitting in space. We now need to construct an imaginary closed surface $S$ surrounding $m$. While in theory any surface would do, we should pick a surface that will make the integral easy to evaluate. Such a surface should have these properties:
	\begin{enumerate}
		\item[P1.] The gravitational acceleration $vec{a}$ should be either perpendicular or parallel to $S$ everywhere.

		\item[P2.] The gravitational acceleration $\vec{a}$ should have the same value everywhere on $S$. (Or it may be zero on some parts of $S$).

		\item[P3.] The surface $S$ should pass through the point at which you wish to calculate the acceleration due to gravity.
	\end{enumerate}
	If we can find a surface $S$ that has these properties, the integral will be very simple to evaluate. For the point mass, we will choose $S$ to be a sphere of radius $r$ centered on mass $m$. Since we know $\vec{a}$ points radially
inward toward mass $m$, it is clear that $g$ will be perpendicular to $S$ everywhere. Also, by symmetry, it is not hard to see that $\vec{a}$ will have the same value everywhere on $S$. 

	Having chosen a surface $S$, let us now apply Gauss’s law for gravity. The law states that for recall that:
	
	Now everywhere on the sphere $S$, we have $\vec{a}\circ\vec{n}=-g$ (since $\vec{a}$ and $n$ are anti-parallel $g$ points inward, and $n$ points
outward). Since $g$ is a constant for a perfect sphere, the previous relation becomes:
	
	Now the integral is very simple: it is just $\mathrm{d}S$ integrated over the surface of a sphere, so it’s just the area of a sphere (\SeeChapter{see section Geometric Shapes}):
	
	or (canceling $-4\pi$ on both sides):
	
	and it's in agreement with the Gravitational Newton law!
	
	So the Flat-Earthers and some believers (following some holy books that we will not mention her) have to explain why everywhere in the world they can measure a falling object which acceleration corresponding to a spheric Earth if that latter is recall flat...

	As Flat-Earther sometimes challenge physicists to prove that the Newton law is not the same for a flat Earth (I was also challenged once... and this was a very bad idea from my opponent) here is the proof!
	\begin{figure}[H]
		\centering
		\includegraphics[scale=0.7]{img/cosmology/flat_earth.jpg}
		\caption[]{Schematic idea of a flat planet like... Earth......}
	\end{figure}
	In this case, the appropriate Gaussian surface $S$ is a "pillbox" shape - a short cylinder whose flat faces - (of area $A$) are parallel to the plane of mass. In this case, everywhere along the curved surface of $S$, the gravitational acceleration $\vec{a}$ is perpendicular to the outward normal unit vector $\vec{n}$, so the curved sides of $S$ contribute nothing to the integral. Only the flat ends of the pillbox-shaped surface S contribute to the integral. On each end, $\vec{a}$ is anti-parallel to $\vec{n}$, so $\vec{a}\circ\vec{n}=-g$ on the ends.
	
	Now apply Gauss's law to this situation:
	
	Here the integral needs only to be evaluated over the two flat ends of $S$. Since $\vec{a}\circ\vec{n}=-g$, we can bring $-g$ outside the integral to get:
	
	The integral in this case is just the area of the two ends of the cylinder, $2A$ (one circle of area $A$ from each end). This gives:
	
	Now let’s look at the right-hand side of this equation. The mass $m$ is the total amount of mass enclosed by surface $S$. Surface $S$ is sort of a "cookie cutter" that punches a circle of area $A$ out of the plane. The mass
enclosed by $S$ is a circle of area $A$ and surfacic density $\sigma$, so it has mass $\sigma A$. Then the previous relation becomes:
	
	Note that this is a constant: the acceleration due to gravity of an infinite plane of mass is independent of the distance from the plane...!

	So Flat-Earth have difficulties to only difficulties to explain this but also are not able to found the corresponding value of $g$ in their laboratory or home garage...

	\subsubsection{Shell Theorem}
	The shell theorem gives gravitational simplifications that can be applied to objects inside or outside a spherically symmetrical body. This theorem has particular application to astronomy.

	Isaac Newton proved the shell theorem and stated that:
	\begin{enumerate}
		\item A spherically symmetric body affects external objects gravitationally as though all of its mass were concentrated at a point at its center.
		\item If the body is a spherically symmetric shell (i.e., a hollow ball), no net gravitational force is exerted by the shell on any object inside, regardless of the object's location within the shell.
	\end{enumerate}
	A corollary is that, and we will prove it, that inside a solid sphere of constant density, the gravitational force varies linearly with distance from the center, becoming zero by symmetry at the center of mass.
	
	Given an object located outside of the Earth and $r$ is the distance of the object to the center of the Earth, we have:
	
	it comes:
	
	If the object is placed at the surface of the Earth or radius $R$, we have ($r=R$):
	
	From the two previous relations it comes therefore:
	
	At the surface we have then well (we expected this result...):
	
	Now, if the object is located inside the Earth by denoting the distance from the center by the letter $r$ and the central mass by $M'$, we have:
	
	Let us introduce the density $\rho$ that we will assume equal everywhere:
	
	By combining these last four relations, we get:
	
	\begin{figure}[H]
		\begin{center}
		\includegraphics{img/cosmology/gravity_profile.jpg}
		\end{center}	
		\caption{Internal/External gravitational acceleration profile of a mass body}
	\end{figure}

	For many people this result is quite counterintuitive (do a little survey around you, you'll see).
	
	\begin{tcolorbox}[title=Remark,colframe=black,arc=10pt]
	In addition to gravity, the shell theorem can also be used to describe the electric field generated by a static spherically symmetric charge density, or similarly for any other phenomenon that follows an inverse square law. The derivations below focus on gravity, but the results can easily be generalized to the electrostatic force. 
	\end{tcolorbox}
	
	\subsubsection{Orbital speed}
	We will prove now an obvious property of orbits that will be useful to us later to study of what seem to be an anomaly with structure of the size of galaxies.
	
	The orbital speed of a body, generally a planet, a natural satellite, an artificial satellite, or a multiple star, is the speed at which it orbits around the barycenter of a system, usually around a more massive body. It can be used to refer to either the mean orbital speed, i.e. the average speed as it completes an orbit, or the speed at a particular point in its orbit such as perihelia.
	
	We have proved above the origin of Newton's law. For planets thus considered as physical points in stable circular orbit, so there is balance between centrifugal and gravitational force. So we have:
	
	where we easily deduce:
	
	Which is approximately in good agreement with the experimental measurements as shown in the figure below:
	\begin{figure}[H]
		\begin{center}
		\includegraphics[scale=0.55]{img/cosmology/orbital_speed.jpg}
		\end{center}	
		\caption{Orbital speed characteristic curve}
	\end{figure}
	But as we will proved it in the section of Aerospace Engineering (Vis-Viva theorem) in a more general case the orbital speed is given by:
	
	
	\subsubsection{Asteroids/Meteors impact velocity}
	We have proved in the section of Classical Mechanics that the escape velocity was given by:
	
	and was therefore independent of the mass $m$ of the ejected object. Obviously the same relation can be applied for an object of mass $m$ coming from an infinite far distance.
	
	For the entry velocity of an asteroid in Earth's atmosphere we can assume that the minimum speed of a colliding asteroid is given by the above escape velocity relation for an asteroid returning to the zero potential of the Earth's gravitational field (a numerical application gives ${11 \;[\text{km}\cdot \text{s}^{-1}]}$.

	In fact, their real entering velocity depends on their direction that will determine their relative speed to Earth (which is ${30  \;[\text{km}\cdot \text{s}^{-1}]}$) plus eventually that of the Sun (which is around ${200  \;[\text{km}\cdot \text{s}^{-1}]}$. We can also take into account the escape veloctiy of the Solar System that is around  ${200  \;[\text{km}\cdot \text{s}^{-1}]}$.

	The sum gives therefore a speed between ${11 \;[\text{km}\cdot \text{s}^{-1}]}$ (for the optimistic case....) and ${300  \;[\text{km}\cdot \text{s}^{-1}]}$ (for the pessimistic case....) with a statistical peak that gives most observed entry at ${30 \;[\text{km}\cdot \text{s}^{-1}]}$.

	As we  will prove it in the section of Weather and Marine Engineering that at a height of $600$ [m] we can see at a distance of almost $80$ [km] we better understand why it is a joke in some movie to see huge asteroids entering the Earth atmosphere with people looking at it during $10$-$15$ seconds... and waiting almost $1$ minute before it hits the ground... (this is type of observation available for meteors but not for asteroids coming from very far!!!).

	A good example is to see all the YouTube videos about the small Chelyabinsk meteor (having a diameter of only $20$ meters) that entered Earth's atmosphere over Russia on 15 February 2013 and that had a speed of only almost  $30 \;[\text{km}\cdot \text{s}^{-1}]$ and which trajectory was visible during almost $20$ seconds only with the human eyes (so imagine with a speed $10$ times faster...).

	\subsubsection{Spherisation of Celestial Bodies}
	Thanks to Newton's law, we could answer to a lot of relevant questions in an approximated wayand giving us results quite convincing.

	A first example is to ask ourself at what scale there is a transition in the domain of irregular shapes (comets, asteroids, moons, etc.) to the field of spheres (moons, planets and stars)? Why the moons of Mars, Phobos and Deimos, have a potato shape like while our moon is roughly spherical. We will see below that this is due to the mass that is greater in the case of our moon. Indeed, from a certain mass, arbitrary geometric shapes are not possible anymore.

	To address this study, we will first estimate the maximum height of a mountain on a planet. Mount Everest has an altitude of $8.8$ [km] while Mount Olympus on Mars has a height of $27$ [km]. Why such mountains can not exist on Earth?

	To take a simplistic approach, we will assume that a mountain must be in hydrostatic equilibrium. We know experimentally the pressure limit in such a rocks lattice beyond which the rocks begin to "melt" (given in tables): $P_{\text{lim}}\cong 3 \cdot 10^8\;\text{[Pa]}$.

	We know from our study of continuum mechanics (\SeeChapter{see section Continuum Mechanics}) the pressure at the base of a mountain of height $h$ will be given in the hydrostatic approximation by:
	
	For the mountain to be stable, it is necessary that:
	For the mountain to be stable, it is necessary that:
	
	and therefore:
	
	Therefore:
	
	Assuming an average density of $rho=3,000\;[\text{kg}\cdot \text{m}^{-3}]$ (continental crust of the Earth) we get:
	\begin{enumerate}
		\item Earth: $h_0\cong 10\; [\text{km}]$
		\item Mars: $h_0\cong 27\; [\text{km}]$
	\end{enumerate}
	What is remarkable as approximate result!!!

	To estimate the minimum size $r_m$ of a body, starting the spherical shape becomes predominant compared to the surface deformation (that is to say where gravity has taken over the interatomic forces), we will require the size $r_m$ is greater than the maximum height of a mountain $h_0$. We also assume that the density $\rho$ remains constant through the body. Taking again the relation:
	
	we have:
	
	hence:
	
	The limit $r_m$ can after be estimated by fixing $r=r_m=h_0$ therefore:
	
	obviously for $r\gg r_m$ we will be even closer to the spherical shape.

	\paragraph{Flattening of Celestial Bodies (rotational flattening)}\mbox{}\\\\\
	Because of the symmetry of the gravitational potential a star or a planet should have a perfectly spherical form starting a given size, as we have just prove it. Now, the fact is... that it is not so for and especially for tellurice bodies.
	
	Because of the own rotation of the star or planet, a centrifugal term transforms potential. This term depends on the latitude which explains the ellipsoidal shape of most observed celestial bodies.
	
	Let us recall that:
	
	where $R$ is the equatorial radius of the star or planet, acceleration to which we have to add the centrifugal acceleration at a given latitude radius $r$ (\SeeChapter{see section Classical Mechanics}):
	
	Therefore the total acceleration given by:
	
	explains why the Earth is flattened at the poles (or depending of the point of view stretched to the equator ...) and that more one empty planet rotates, the more it will be flattened at the poles.
	
	On Earth, the equatorial radius is of $6,379$ [km] while the polar radius is of $6.357$ [km]. The difference is $22$ [km]. The "\NewTerm{flattening}\index{flattening}" of as star or planet is sometimes defines as:
	
	thus the difference between the equatorial radius and polar radius divided by the equatorial radius.
	
	Although an ellipsoid of revolution is the best description for the form of a planet:
	\begin{figure}[H]
		\begin{center}
		\includegraphics{img/cosmology/earth_with_atmosphere.jpg}
		\end{center}	
		\caption{Earth with its atmosphere and oceans}
	\end{figure}
	there are obviously imperfections between the model and the reality for some planets (in particular the terrestrial planets, satellites, and small rocky bodies). The geopotential of real body can be shaped much more complicated because of influences of the visible inhomogeneities on the surface as evidenced by this satellite image of the Earth omitting the liquid parts of it (the deformations are amplified by a factor $100,000$ in the image below!):
	\begin{figure}[H]
		\begin{center}
		\includegraphics{img/cosmology/earth_without_atmosphere.jpg}
		\end{center}	
		\caption{Earth without its atmosphere and oceans}
	\end{figure}
	The specialist of geodesics take into account these inhomogeneities. They measure and describe the shape of the planets that they name "\NewTerm{geoid}\index{geoid}".
	\begin{figure}[H]
		\begin{center}
		\includegraphics[scale=0.8]{img/cosmology/asteroid_spherisation.jpg}
		\end{center}	
		\caption{Evolution of asteroids shape in function of the radius}
	\end{figure}
	
	\subsubsection{Stability of Atmospheres}
	Comparing the liberation velocity and the velocities of various gases, we can explain the stability of certain atmospheres and the absence of others. We have proved in the section of Classical Mechanics that the liberation velocity of a spherical star was given by the following relation (on which we will come back in the section of General Relativity):
	
	For the Earth, a numerical application gives $v_L=11.2\cdot 10^3\;[\text{ms}^{-1}]$ and for the moon $v_L=2.37\cdot 10^3\;[\text{ms}^{-1}]$.

	Let us recall that we have proved in the section of Continuum Mechanics during our study of the kinetic temperature the following relation (Viriel's theorem):
	
	Using the molar mass (\SeeChapter{see section Thermochemistry}):
	
	A numerical application gives for nitrogen $v_\text{Az}=517\;[\text{ms}^{-1}]$ and for hydrogen $v_\text{Az}=1,934\;[\text{ms}^{-1}]$with an arbitrary temperature of $300 [\text{K}]$.
	
	So nitrogen is obviously trapped in the Earth's atmosphere. Hydrogen, light gas, more fast is less trapped. The two gases are even less trapped by the Moon.
	\begin{tcolorbox}[title=Remark,colframe=black,arc=10pt]
	In fact, the mean square speed is not the only speed of molecules. There is a distribution of velocities. We have indeed study the Maxwell-Boltzmann distribution of a gas at equilibrium in the section of Statistical Mechanics.
	\end{tcolorbox}
	

	\pagebreak	
	\subsection{Roche's Limit}
	The Roche limit is the theoretical distance below which a satellite would begin to break down under the action of tidal forces caused by the celestial body around which it orbit, these forces exceeding the satellite internal cohesion.
	
	We can simplify the problem by considering the satellite liquid, not rotating on itself (no spin), and decomposing it into two small masses $m$ of radius $r$ and volumic density $\rho_S$.
	\begin{figure}[H]
		\begin{center}
		\includegraphics{img/cosmology/roche_limit.jpg}
		\end{center}	
		\caption{Configuration for the study of the Roche limit}
	\end{figure}
	The planet is a sphere of radius $R$, mass $M$,  volume density $\rho_P$, located at a distance $D$ of the satellite axis.
	
	The planet exerts on the satellite the gravitational attraction:
	
	The difference of forces between the two masses is:
	
	We can consider that $r \ll D$, giving:
	
	So the difference in force is
	
	If the satellite cohesive force result in the gravitational attraction between the two masses:
	
	The satellite is destroyed if the difference in strength between the two masses is greater than the cohesive force:
	
	But we have the relations:
	
	Therefore we get:
	
	and we deduce of it the "\NewTerm{Roche limit}\index{Roche limit}":
	
	Depending on the approach and the approximations they can be a factor $3$ between some results.
	\begin{tcolorbox}[title=Remark,colframe=black,arc=10pt]
	For example the calculations given on Wikipedia consider only the difference in the primary's gravitational pull on the center of the satellite and on the edge of the satellite closest to the primary. This means that the main mass only apply one force momentum. But in fact this is not accurate as what interest us is the difference between the two extremities. This is why there is a factor $2$ between the Wikipedia calculations and ours (with our result the satellite will break twice the distance of that given by Wikipedia).
	\end{tcolorbox}
	
	Since in this calculation, we considered a satellite as a two point masses without rotation, and again we have assumed that the satellite's cohesion was provided exclusively by gravitational interactions, this value is an order of magnitude.
	
	\pagebreak
	\subsection{Keplerian Orbitals}
	Observation (main tool of the physicist and engineer for recall...) suggests at first glance, that the trajectories of celestial bodies in orbit around stars are indeed conical type (whew!) in the heliocentric reference frame. Knowing this, we can, in order to facilitate the calculation, anticipate the complexity of calculations and express the dynamics directly from a material point in polar coordinates.

	As we saw it in the section Vector Calculus, the speed in polar coordinate is expressed by the relation (we changed the angle Greek letter notation to adapt it to the tradition in astronomy):
	
	where to recall the first term is the radial velocity component and the second component the tangential (angular) velocity!

	For acceleration (the proof is still in the section of Vector Calculus):
	
	Now that we have the tools, let us get to the case of Keplerian orbits in the case of a static Newtonian field.

	There is to our knowledge the two main ways of doing the necessary mathematical developments but that do not gives (to our knowledge) the same level of detail results. The first approach provides finer results but is sometimes a bit do-it-yourself sometimes... is based on the use of the radial velocity and an important relation in astronomy, named the "first Binet formula". The second approach is simpler and most elegant, it uses the radial acceleration to approach the problem and a special relation named the "second Binet formula".
	
	\subsubsection{First Binet Formula}
	To start with this first approach to the problem, recall that we have already shown prove earlier that:
	
	However, it is unlikely that the main body is a perfect and homogeneous sphere ... so Astrophysicists have the habit of noting Newtonian potential $U$ under the form:
	
	where $\mu=GM$ is named "\NewTerm{gravitational constant of the star}\index{gravitational constant of a star}" (even if it is not always a star...) and where $f$ is a function representing the heterogeneity of the star.
	
	If there is one place in the universe where the laws of mechanics are perfectly verifiable, it is space, because the friction or causes of dissipation are extremely small. Within the field of a single force deriving from a potential, the movement satisfies the conservation of mechanical energy.

	Thus we end in the so-named "\NewTerm{energy equation}\index{energy equation}", wherein $E$ denotes the "\NewTerm{specific energy}\index{specific energy}" per unit weight (kilogram):
	
	Therefore:
	
	The Newtonian gravitational force is central, thus of having a null torque force at the center O of the main body. This results in the conservation of angular momentum in norm and direction, either:
	
	The vector $\vec{W}$ is the normalized vector of $\vec{b}$ or of $\vec{h}$ nalled the "\NewTerm{reduced momentum}\index{reduced momentum}". $K$ is the constant of areas (\SeeChapter{see section Classical Mechanics}) such that:
	
	We recall to the reader that the norm of the speed expressed in polar coordinates is given by the relation (remember that the both vectors of the polar base are orthogonal and that we can therefore apply the Pythagorean theorem to calculate the norm as it has been proved in the section of Vector Calculus):
	
	Which gives us the possibility to write the area constant $K$ as:
	
	Let us now put ourselves in the orbital plane, in polar coordinates.
	
	Given the relation already proved and known:
	
	and its squared norm:
	
	Or in the case of a central force (conservation of angular momentum):
	
	Let's put this in the prior-previous expression of $v^2$, then we have:
	Let us put this in the expression prior-previous expression of $v^2$, then we have:
	
	The relation:
	
	is named "\NewTerm{Binet's first formula}\index{Binet's first formula}".
	
	By equating with the expression of $v^2$ resulting from the conservation of energy that we get earlier above, we have:
	
	This gives us a rather complicated differential equation:
	
	And then we wonder how we can get out of such a situtation? After some hours of reflection ... we realize that takes to make a substitution. After another hour of neural chaos this ultimately leads to an end.... We decide to put (we have the right to do it!), knowing that $r$ is a function of $u$ and $\theta$:
	
	Let us derivate merrily relatively to $\theta$:
	
	Substituting in the differential equation:
	
	After simplification we get:
	
	We separate the variables to integrate:
	
	We have two solutions according to the sign we choose. However, at the end of the resolution, we notice that the only physically interesting choice is the negative sign. We have proved in the section of Differential and Integral Calculus And in our common derivatives that:
	
	We will chose the primitive in cosine and therefore we have:
	
	We leave, by approximation, the constant of integration that would involve very small oscillations in the orbit's path (if you do a study or a homework on this topic, you can transfer me your plots with or without the constant, it would interest me as I don't  have time to do it myself).

	This allows us to obtain:
	
	Now we see that our choice of the sign for the integration is fully justified because now, if we do a little recall on conics (\SeeChapter{see section Analytical Geometry}), we see that after rearrangement:
	
	So finally we have a relation of the form if we choose $\theta_0=0$:
	
	where by analogy with the section of Analytical Geometry $e$ is the eccentricity (let us recall that $e=c/a<1$ with $a$ the semi big axes and $c$ the distance to the center of the ellipse to the focal) and $p$ the focal parameter ($p=b^2/a$) of an ellipse. This corresponds well to the trajectories that follow celestial bodies in orbit.
	
	We thus fall back on our the first  Kepler "law"... so as we can see it, it can be proven!

	In our case, we have after simplification to resume:
	
	where (for recall) $K$ is the areas constant :
	
	and $\mu$ is the gravitation constant of the celestial body:
	
	and finally $E$ the specific energy:
	
	The reader could be able to check himslef as we have seen in the section of Analytical Geometry in our study conical that if:
	\begin{itemize}
		\item If $E=0$ such that $e=1$ we have an opened orbit in the form of a parabola

		\item If $E>0$ such that $e>1$ we have an opened orbit in the form of an hyperbola

		\item If $E\leq 0$ such that $0\geq e <1$ we have a closed orbit in the form of an ellipse or a circle
	\end{itemize}
	\begin{figure}[H]
		\begin{center}
		\includegraphics{img/cosmology/orbits.jpg}
		\end{center}	
		\caption[]{Reminders of conical but "orbit" oriented}
	\end{figure}
	Finally, if we inject:
	
	in the first Binet formula:
	
	then we get the velocity in any point of the ellipse based on the primary variable parameter which is therefore the angle.
	
	\subsubsection{Second Binet Formula}
	Let us now see the approach based on the radial acceleration which, while being more elegant, allows us to get a result less fine-tuned on the ellipse parameters.

	So we start from the expression of the acceleration in polar coordinates (\SeeChapter{see section Vector Calculus}):
	
	We can simplify the writing of the second term:
	
	Now we have seen just above that:
	
	and so:
	
	Then the acceleration is reduced to:
	
	We can eliminate the time by writing:
	
	and:
	
	Then we get:
	
	And so it comes to the standard "\NewTerm{Binet's second formula}\index{Binet's second formula}"
	
	But according to Newton's second law and his law of gravitation, we have:
		
	We then another form of the second Binet formula:
	
	Or after simplification and choosing the sign of the acceleration at our convenience to get rid of the "-" sign, we have:
	
	By isolating the constants, we get:
	
	After a change of variables we recognize the particular case of a differential equation of the second order we have already met several times so far in the various sections of this book and we will meet again:
	
	As it is customary, however, we will shoe the details of the resolution. The equation without second member is (\SeeChapter{see section of Differential and Integral Calculus}):
	
	We then have the discriminant that is negative since:
	
	We then saw in the section of Differential and Integral Calculus, that in this situation the solution of the homogeneous equation was of the form:
	
	Thus in the situation we are concerned, we have:
	
	We inject the solution into the homogeneous differential equation with second member:
	
	and we see immediately see that that for the equality to be satisfied, the general solution is:
	
	Or after rearrangement:
	
	And choosing the initial angle as zero, so we find well:
	
	at the difference with the first method of resolution that the value of the constant $A$ is unknown.
	
	Let us now come back on:
	
	By expliciting:
	
	And as (\SeeChapter{see section Classical Mechanics}):
	
	So if we choose a particular point of reference of the path (not necessarily circular path), we have:
	
	Then we have:
	
	If we put the phase shift as zero relatively to the reference radius choosed earlier above, the expression simplifies to:
	
	To determine the constant $A$, we place ourselves in the case where $\theta=0$ and imposes that the radius $r$ is the initial radius measured when this angle is zero. Then we have:
	
	This implies immediately:
	
	Thus after elementary rearrangements and simplifications:
	
	And therefore we have a direct correspondence:
	
	And as the eccentricity $e$ is known for a circular, parabolic, elliptical or another trajectory of the conical type... it gets us very easy to deduce the velocity at the particular point of the initial radius $r$ of the studied object.

	The closest distance of the object orbiting around its central star (focus), will be given by the value that can takes $r$ in the relation:
	
	if we impose $\theta=0$.

	In the case of an elliptical orbit it is the "\NewTerm{perigee}\index{perigee}" to be assimilated to the initial radius as the point where the measurement of the radial velocity was the most accurate.

	The farthest distance from the focus will be given by putting the angle as being $180^\circ$ ($\pi$) and then we name it "\NewTerm{apogee}\index{apogee}".
	
	\pagebreak
	\subsubsection{Keplerian orbital period}
	The Kepler's law of equal areas allows, as we already know, to calculate the Keplerian orbital period $T$. In fact, the area $S$ of the ellipse being equal to $S=\pi a b$ (\SeeChapter{see section Geometric Shapes}) and having already determined during our definition of the angular momentum that (\SeeChapter{see section Classical Mechanics}):
	 
	It comes naturally:
	
	Moreover, the study of conics (\SeeChapter{see section Analytic Geometry}) has showed us that:
	
	and we have defined above:
	
	So we have the relation:
	
	and then we fall back again on the third Kepler's law:
	
	which validates our previous calculations.
	
	Obviously the latter relation is only available for $T$ where the corresponding speed is non relativistic otherwise we have to use General Relativity tools.

	\pagebreak
	\subsubsection{Classical deflection of light}
	The calculations done previosuly can be applied to an interesting case: the deflection of light by a star in the Newtonian interpretation (of course!).

	Warning!!! Newton did not know at its time that the photon was massless. The following developments are therefore a wrong approach in our time and should be taken with precaution but are still taught today because it allows students that do not yet studied General Relativity or that will never study it  (in the section on General Relativity, the reader will find the contemporary detailed proof of the deflection of light that is a whole other level) to have a first approach... it's like everything in physics! Until we have reached the level of the university degree, we learn many things "wrong" because oversimplified. Then at the Master or PhD level, we learn a little more realistic and valid theories.

	Well this being recalled (following the remark of one of our reader), so we have proved above for recall that:
	
	In the case of a photon, we tend to put that $r\rightarrow +\infty$ (thus a hyperbolic trajectory) and therefore this requires that in the previous relation we have (which is equivalent to saying that $e$ is strictly greater than the unit as required by the hyperbolic trajectory):
	
	by putting $\varphi=2\theta-\pi$ the elementary trigonometric relations (\SeeChapter{see section Trigonometry}) give us:
	
	and therefore still using trigonometric identities:
	
	Therefore:
	
	And we know that:
	
	Hence:
	
	neglecting the potential energy of the photon since $r\rightarrow +\infty$ (caution !!! recall that according to what we saw in the section of Special Relativity, the photon has no mass strictly speaking but Newton knew nothing about this at his time!):
	
	Therefore:
	
	Hence:
	
	After simplification:
	
	and as $\theta$ is assumed to be small, we have using the Taylor expansion (\SeeChapter{see section Sequences and Series}) of the tangent function:
	
	So it finally comes:
	
	But, we have by definition:
	
	and we know that $v=\omega r=\dot{\theta}r$ (\SeeChapter{see section Classical Mechanics}). Thus we have:
	
	If the particle is a photon passing flush with to surface of the Sun then:
	
	a numerical application gives:
	
	Newtonian theory thus provides a $0.87$ arc seconds deviation for a ray of light passing flush to the Sun's surface. Which is twice less than what can be observed experimentally and that gives the theory of General Relativity (\SeeChapter{see section General Relativity})!
	
	\subsubsection{Classical precession of perihelia}
	Before studying the precession of the orbits, we would recall that the gravitational field is a conservative and center field. This implies that the angular momentum (\SeeChapter{see section of Classical Mechanics}) is constant and that the path is held in a plane whose normal vector to the surface always maintains the same direction (the angular momentum vector is constant in norm and direction for recall!).

	We will address here the analysis of the precession of the perihelion taking into account the results of the theory of special relativity (allowing it to be more accurate in the results and be able to apply these results to the orbiting electrons around the nucleus of the atom in the corpuscular model).
	
	First let us recall that:
	\begin{itemize}
		\item The "perihelion" is the point of the orbit of a celestial body (planet, comet, etc.) that is closest to the star around which it rotates.

		\item The "aphelion" is the point in the orbit of an object (planet, comet, etc.) where it is farthest from the star around which it rotates.

		\item The "equinox" is the moment (time) when the central star crosses the plane of the equator of the object that is in orbit around it.
	\end{itemize}
	\begin{tcolorbox}[title=Remark,colframe=black,arc=10pt]
	When the Sun passes from the southern hemisphere to the northern hemisphere of the Earth (in other words when the Sun is at the Zenith at midday at the equator), it is the spring equinox (20 or 21 March) in the opposite direction, this is the autumn equinox (22 or 23 September). At these dates, there is equality of day and night all over the Earth.
	\end{tcolorbox}
	Obviously, the result we get will here is not complete, since, as we know, we had to wait the development of General Relativity to give the exact value of the perihelion  precession of Mercury (see will come back on this subject further below).

	To calculate the effect of precession, we will seek the equivalent of the Binet formulas seen above in relativistic form (we will see the classical form in the section of General Relativity). For this we proceed as follows:

	The relativistic Lagrangian of the system is (\SeeChapter{see section Special Relativity}):
	
	\begin{tcolorbox}[title=Remark,colframe=black,arc=10pt]
	We subtract then energy at rest because only interest us here the study of the kinetic and potential energy. The potential energy is summed in the Lagrangian above (which is not consistent with the practice) but we will reverse the sign later below during the developments.
	\end{tcolorbox}
	With (see section Special Relativity and Vector Calculus):
	
	and the reduce mass for recall:
	
	The angular moment:
	
	in relativistic form and applied to our study is:
	
	Taking the norm, we have without forgetting that in our study $\vec{\omega}\bot\vec{r}$ and therefore $(\vec{\omega}\bot\vec{r})\bot\vec{r}$:
	and let us recall that we have adopted the notation $\omega=\dot{\theta}$ (in case you forget the definition...). Which finally gives us:
	
	To establish the relativistic equivalent of the Binet formulas:
	\begin{itemize}
		\item We deduce the expression of the angular momentum:
		

		\item We seek for a relation of the type $\dot{r}=\dot{r}(\theta)$ (as the trajectory is a conic):
		
		Indeed let us recall that in polar coordinates the speed is given by the following expression (\SeeChapter{see section Vector Calculus}):
		
		That is to say, $\dot{r}=f(r,\theta)$. The latter exprresion gives us the possibility to write that:
		
		
		\item We seek a relation of the type $\ddot{r}=\ddot{r}(\theta)$:
		
	\end{itemize}
	From the equations obtained previously, we have successively:
	
	Let us recall that we have defined in special relativity $\beta$ and that by using the speed in polar coordinates:
	
	With the previous relations, this gives us:
	With the above relations, this gives us:
	
	On the other hand:
	
	By introducing in the prior previous relationship in the latter:
	
	By putting $u=1/r$ and as:
	
	The prior-previous relationship becomes with this expression:
	
	Equating this relation with that of the Lagrangian:
	
	Differentiating the latter relation relatively to $\theta$:
	
	Indeed, the Lagrangian is constant over time (the system is assumed to be conservative), we then have:
	
	and also:
	
	But if we continue:
	
	By refering to:
	
	So we get:
	
	That gives after a few simplifications:
		
	By multiplying the latter by $\mu^2c^2/b^2$:
	
	In a gravitational potential:
	
	The Binet equation in special relativity is then:
	
	To find a solution to this differential equation, we will group the variable $u$ in the left side:
	
	We put:
	
	The differential equation then can be written:
	
	We put:
	
	By taking the second derivative:
	
	We then get a simple differential equation:
	
	whose solution is well known to us (\SeeChapter{see section Differential and Integral Calculus}):
	
	What can still be written as $\Omega^2$ is a constant:
	
	with $k_1,k_2=c^{te}$.

	To determine the constants $k_1,k_2$, we place ourselves irst in the situation for which $\theta=0$, where $r$ is minimal and therefore by $u$ is maximum by definition.
	
	We derivate relatively to $\theta$:
	
	Therefore $k_2=0$ which makes that the relation:
	
	becomes:
	
	Written differently (trying to return to a similar notation to that of the study of conic) then:
	
	And the interest to write this in this way is to notice that we fall ultimately on the equation of an ellipse with $p$ being the focal parameter of the conic, focal parameter given for recall by (\SeeChapter{see section Analytical Geometry}):
	
	where $a$ is the half major axes of the ellipse.
	
	Now let us put:
	
	In the first passage through the perihelion $\theta=0$ where:
	
	we therefore have:
	
	Now let us put:
	
	At the first passage through the perihelion $\theta=0$ where:
	
	we have:therefore
	
	At the second passage through the perihelion $\theta=2\pi$, we have:
	
	we also have:
	
	The trajectory is stille an ellipse but the angle $\Omega\theta_0$ that was zero initially has become $\Omega\theta_1=2\pi$.

	That is, if we have:
	
	Therefore:
	
	Which gives us:
	
	Since $G^2\ll c^2$, a development in Taylor series give us (\SeeChapter{see section Sequences And Series}):
	
	By limiting at the order $2$:
	
	So in conclusion, there is an advancement of the perihelion taking place in the satellite's direction of rotation. For a repository located in the satellite's rotation plane, the trajectory is always an ellipse.

	This advance is of:
	
	by period. Either by expliciting the momentum given for reminder by:
	
	It comes after simplification:
	
	We will now allow us a rough approximation (mixture of relativistic and non-relativistic). If we consider the last relation, we have obtained during our developments of the Keplerian orbital trajectories the relation:
	
	Therefore, injecting this into the relation of $\Delta \alpha$	we have:
	
	Unfortunately, the numerical values for Mercury precession give only a precessions of an angle of $7''$ per century and not the $43''$ as expected (...) there is therefore a lack of a factor $6$ that only the General 
Relativity (\SeeChapter{see section of General Relativity}) makes possible to found. It is nevertheless interesting that Special Relativity already gives an orbit that precesses where Newton sees stable ellipse and that this approximation works for all the planets except Mercury (the planet closest to the Sun and undergoing the brunt of curvature of space-time).
	\begin{tcolorbox}[title=Remark,colframe=black,arc=10pt]
	By applying exactly the same reasoning to corpuscular quantum physics (electrical potential) but with the ad hoc constants seen in the section of Electrostatics, we find:
	
	with $\vec{b}=\mu\vec{r}\times\vec{v}$ being the momentum and in the case of the atom, we will take (\SeeChapter{see section Corpuscular Quantum Physics}):
	
	with reduced mass equal to:
	
	\end{tcolorbox}
	If the positions of the perihelion (and therefore the aphelion) of the Earth-Moon center of gravity  were constant over time, the duration of the different seasons would be constant. But the orbit of the center of gravity Earth-Moon also rotates in its plane in the forward direction at about 12 '' per year (a revolution is about $108,000$ years).

	The precession of the equinox occurs in the opposite direction (retrograde direction) at about $50''$ per year (then a "\NewTerm{precession equinox}\index{precession equinox}" revolution is about $26,000$ years). The combination of these two movements permits to calculate the period of the passage of the perihelion of the Earth by the direction of the vernal equinox, this period of about $21,000$ years and is named the "\NewTerm{climatic precession}\index{climatic precession}.
	\begin{figure}[H]
		\begin{center}
		\includegraphics[scale=0.9]{img/cosmology/precession_orbit_earth.jpg}
		\end{center}	
		\caption{Effects of precession on the seasons using the Northern Hemisphere terms (source: Wikipedia)}
	\end{figure}
	Indeed, every $10,500$ years (half period of climatic precession) aphelion changes from summer to winter. But even if the Earth-Sun distance is by far not the predominant factor in the nature of the seasons, the combination of the passage of the Earth in the winter in aphelion gives winters a little bit mor harsh. Earth-Sun distance also depends on the variation in the eccentricity of Earth's orbit (due to external and inner planets). Thus, the ice ages are correlated with the minimum eccentricity of Earth's orbit.
	\begin{figure}[H]
		\begin{center}
		\includegraphics[scale=0.9]{img/cosmology/precession_orbit_earth_perspective.jpg}
		\end{center}	
		\caption[]{Simplified and perspective point of view of the previous figure (source: Latsis foundation (2001))}
	\end{figure}
	The work of the Celestial Mechanics Institute (France), since the 1970s, would have to definitively confirm the theoretical predictions as what the eccentricity of Earth's orbit undergoes wide variations formed numerous periodicals under which the most important one have periods near $100,000$ years, and for one of them, a period of $400,000$ years. These results confirm the climatic variations of the Earth during the Quaternary (\SeeChapter{see section Weather \& Marine Engineering}). The paleoclimatology models indeed show the correlation between changes in the Earth's orbit elements and large quaternary glaciation.
	\begin{tcolorbox}[title=Remark,colframe=black,arc=10pt]
	In the case of the hydrogen atom (\SeeChapter{see section of Corpuscular Quantum Physics}), for the case dealing with relativistic model of Sommerfeld, with $n=1,n_\theta=1,Z=1$ and and the fine structure constant approximately equal to $1/137$, we get to the precession of the perihelion of the orbit of the electron given by:
	
	according to a corpuscular point view of matter!
	\end{tcolorbox}
	
	\subsection{Duration of the diurnal arc}
	A diurnal arc is the time, as expressed in right ascension, it takes a planet, point, or degree to move from its rising point to its setting point. This takes place in many celestial bodies such as the sun and moon.
	
	So we will study here at the time length of the day, more exactly to the portion of day where we are illuminated by the Sun, as compared to the night when we are in the shade\footnote{Thanks to Xavier Hubaut for these very friendly developments}.
	
	In reality, the Earth revolves around the Sun and describes an almost circular orbit at the same time it turns on itself around its axis that is tilted (actually) by about $23^\circ 27'$ relatively to its orbital plane (the ecliptic):
	\begin{figure}[H]
		\begin{center}
		\includegraphics{img/cosmology/equinox_solstice.jpg}
		\end{center}	
		\caption{Representation of the rotation of the Earth on its orbit with its major phases}
	\end{figure}
	\begin{tcolorbox}[title=Remark,colframe=black,arc=10pt]
	It is obvious that, given the complexity of the problem, we will simplify it by considering a circular orbit without variations (precession, nutation) of the axis of rotation of the Earth. We will assume that the Sun is reduced to a point (no dawn or twilight, etc.).
	\end{tcolorbox}
	Let us first recall that the precession is the gradual change in direction of the axis of rotation of an object when a torque (force) is applied to it while the nutation is a periodic balancing of the axis of rotation of the Earth around its mean position in addition to the precession (\SeeChapter{see section Classical Mechanics}).
	\begin{figure}[H]
		\begin{center}
		\includegraphics{img/cosmology/nutation_precession_earth.jpg}
		\end{center}
	\end{figure}
	Let us represent the Earth with its vertical axis of rotation. Accordingly the equator will be located in a horizontal plane.

	Suppose that that day, the Earth is in such a position that the Sun's rays form an angle $\alpha$ with the equatorial plane (or conversely that the axis of the Earth form an angle with the equatorial plane). Notice that this angle $\alpha$ will always be according to actual measurements  between $-23^\circ 27'$ and $+23^\circ 27'$ at least... at a human life time scale...
	
	For our example we have chosen to focus our analysis on a day when $\alpha$ is positive. Thus, in the northern hemisphere, we are close to the summer solstice!

	We are looking for the day length at a place located at latitude $\lambda$! To fix ideas, we place ourselves around Brussels at $50^\circ$ north latitude.
	
	Let us now consider the following figures where the first is a view of the side of Earth at a time $t$ of its orbit when $\alpha>0$ and the second in to a cylindrical cutting of diameter $\overline{NJ}$ (corresponding to the diameter of the parallel of Brussels) of Earth's volume Earth at this same moment:
	\begin{figure}[H]
		\begin{center}
		\includegraphics{img/cosmology/diurnal_arc_duration.jpg}
		\end{center}
	\end{figure}
	On the figures above, $C$ denotes the center of the Earth, and O the center of the parallel of Brussels.

	Let us fix a time $t$ and denote by $M$ (morning) and $S$ (evening) the two points of the parallel of Brussels where the Sun rises and sets (these points will be considered fixed whatever the moment $t$, which is obviously wrong relatively to the reality), while $J$ (day) and $N$ (night) will be the points where it is noon and midnight respectively.

	$P$ will be the point on the disc corresponding to the Brussels parallel where the meridian noon plane  (the plan which of the sides is $\overline{NJ}$) cut the line $\overline{MS}$.

	Finally, $\gamma$ designate the angle $\widehat{M\text{O}S}$ (where O is the center of the disk generated by the parallel of Brussels) behind the illuminated part by the Sun and $r$ designate the radius $S\text{O}=M\text{O}$.

	To simplify the problem, let us also assume... that during $24$ hours the Earth rotates on itself without changing the position of its axis of rotation relative to the Sun....

	The angle $\gamma$ can be calculated by noting that $\overline{\text{O}P}$ is, in absolute value equal to:
	
	where $r$ represents for recall the radius of the parallel of Brussels.
	
	Using the properties of trigonometric functions (\SeeChapter{see section Trigonometry}), we have:
	
	But we still need to inject the parameter $\alpha$. Knowing the latitude $\lambda$ of Brussels, we have:
	
	where $R$ is the radius of the Earth.

	We have also:
	
	and in the triangle $C\text{O}P$:
	
	Finally, by comparing the values obtained for $\overline{P\text{O}}$, we get:
	
	and as:
	
	We finally get:
	
	and therefore:
	
	At the equinoxes (that is to say when the equator coincides with the ecliptic plane for recall...) we have $\alpha=0$ and therefore:
	
	However, as we have specified it at the beginning, we must take the absolute value thus:
	
	In other words, whatever the latitude we take, the angle formed by the night area is equal to the angle formed by the day area at equinoxes (both being equal to $\pi$).

	Let us now consider the summer solstice, when $\alpha=23^\circ 27'$ still considering the latitude of Brussels $\lambda=50^\circ$, we have:
	
	This translated into hours by:
	
	So the 24-hour day loses $7.9$ hours. Which is equivalent to a day light of approximately $16$ hours.
	
	In summary to calculate the duration of a "day", it is enough to know two things: the latitude and the angle at which the Sun falls on the plane of the equator to the chosen date. The value of this angle is well known at the equinoxes (it is $0^\circ$) and to the solstices (it is $+23^\circ 27'$ and $-23^\circ 27'$).

	But what about the other dates?
	
	The answer is quite simple. Let us imagine, sitting on the Sun watching throughout the year towards the center of the Earth.

	During its rotation around the Sun (the binome centroid in fact), the axis of rotation of the Earth maintains its inclination to the ecliptic. Seen from the Sun, this axis revolve aroundnormal to the plane of the ecliptic and therefore describe a cone whose half apex angle is $23^\circ27'$ (see figure below).

	The angle of attack $\alpha$ of the sunlight on the equator therefore vary according to the date $\delta$ (we associate to the date, the angle $\delta$ traveled by the Earth on its orbit, from its position the spring equinox)

	Therefore, the angle $\alpha$ vary according to the date $\delta$ sinusoidally.

	For those who are perhaps not convinced by this semi-intuitive reasoning, here's another approach:

	For readability of the diagram, we have greatly exaggerated the angle of the axis of rotation of the Earth with the ecliptic:
	\begin{figure}[H]
		\begin{center}
		\includegraphics{img/cosmology/cone_generated_by_earth_rotation.jpg}
		\end{center}
	\end{figure}
	Given $C$ the Earth's center, $A$ the end of a unitary vector $\overrightarrow{CA}$ oriented directed along the axis of rotation of the Earth (ie perpendicular to the plane of the equator) and another unit vector $\overrightarrow{CS}$ directed toward the Sun. Given now $\alpha$ the angle of the radius $\overline{CS}$  with the plane of the equator and $\beta$ the angle between the unit vectors  $\overrightarrow{CS}$ and  $\overrightarrow{CS}$. Then we have:
	
	Indeed, the vector $\overrightarrow{CA}$ being perpendicular to the plane of the equator, it forms a right angle with it. Therefore since the angle $\beta$ is the angle between this vector and the ecliptic, the angle $\alpha$ is then the complementary angle.
	
	Therefore we have:
	
	Let us decompose now $\overrightarrow{CA}$ in the sum of $\overrightarrow{CA'}$ directed perpendicular to the ecliptic plane and of $\overrightarrow{CA''}$ located in the ecliptic plane:
	
	Therefore:
	
	But:
	
	So finally:
	
	and as we have demonstrated that:
	
	We finally get:
	
	Now the problem is solved and the daylight time duration will depend on two variables: the date $\delta$ and the latitude $\lambda$.

	We just have so now to take again the relation:
	
	and to inject in it the new result to get a first simple version of "\NewTerm{equation of time}\index{equation of time}":
	
	With computer tools at our disposal, we can easily calculate the value $\gamma$. For example, we have below the variations in the length of the day over a year at latitudes of $0$ to $90^\circ$ spread by $10$ by $10^\circ$:
	\begin{figure}[H]
		\begin{center}
		\includegraphics{img/cosmology/equation_of_time.jpg}
		\caption{Equation of time plot}
		\end{center}
	\end{figure}
	From the latitude of the Arctic Circle, we see, in summer, periods with uninterrupted sun (midnight sun) and in winter whole days of night.

	For Brussels (latitude = $50^\circ$) we see from the figure that the length of the day varies approximately between the values of $16$ [h] (summer solstice) and $8$ (winter solstice).
	
	\subsubsection{Trigonometric parallax}
	Measuring distances to objects within our Galaxy is not always a straightforward task – we cannot simply stretch out a measuring tape between two objects and read off the distance. Instead, a number of techniques have been developed that enable us to measure distances to stars without needing to leave the Solar System. One such method is "\NewTerm{trigonometric parallax}\index{trigonometric parallax}", which depends on the apparent motion of nearby stars compared to more distant stars, using observations made $6$ months apart (corresponding to the measurement of diameter of the apparent approximated circular motion they have in the sky).

	A nearby object viewed from two different positions will appear to move with respect to a more distant background. This change is named a "\NewTerm{parallax}"\index{parallax}. A simple demonstration is to hold your finger up in front of your face and look at it with your left eye closed and then your right eye. The position of your finger will appear move compared to more distant objects.

	By measuring the amount of the shift of the object's position (relative to a fixed background, such as the very distant stars) with observations made from the ends of a known baseline, the distance to the object can be calculated.
	
	The trigonometric parallax method is very simple (but difficult to implement on the surface of our planet for very distant stars). Any amateur astronomer observed the flight of the star it observes with his eye. This movement is named as we have just seen the "\NewTerm{diurnal movement}\index{diurnal movement}" It is due to the rotation of the Earth itself. The star is also driven in an elliptical motion much less easily detectable: the "\NewTerm{parallactic motion}\index{parallactic motion}".

	It is due, as suggested in the figure below, to the rotation of the Earth around the Sun. So we measure the angle $p$ and we have obviously:
	
	If the angle is small (which is very often the case given the distance of stars...) we can take the first term of the Taylor expansion (\SeeChapter{see section Sequences And Series}) of the tangent function:
	
	Which allow us to write:
	
	\begin{figure}[H]
		\begin{center}
		\includegraphics{img/cosmology/parallax.jpg}
		\caption{Trigonometric parallax principle}
		\end{center}
	\end{figure}
	If the parallax angle, $p$ is measured in arcseconds (arcsec), then the distance to the star, $d$ in parsecs (pc) is given by:
	
	It is important to notice that in this example we assume that both the Sun and star are not moving with a transverse velocity with respect to each other. If they were this would complicate the picture as presented here. In practice stars with significant proper motions require at least three epochs of observation to accurately separate their proper motions from their parallax. Stars that are members of binaries further complicate the picture.

	The only star with a parallax greater than $1$ [arcsec] as seen from the Earth is the Sun - all other known stars are at distances greater than $1$ [pc] and parallax angles less than $1$ [arcsec] ($1/3600$ of degree... we understand better why this what impossible to measure before the 19th century). When measuring the parallax of a star, it is important to "account for the star’s proper motion, and the parallax of any of the fixed" stars used as references.

	Over a $4$ year period from 1989 to 1993, the Hipparcos Space Astrometry Mission measured the trigonometric parallax of nearly $120,000$ stars with an accuracy of $0.002$ [arcsec]. The GAIA mission, to be launched in 2010, will be able to measure parallaxes to an accuracy of $10^{-6}$ arcsec, allowing distances to be determined for more than $200$ million stars.
	
	In practice when we measure the parallax we must obviously take in account the obliquity of Earth on it's orbit that is in this beginning of the 21st century equal to $23^\circ 26'13.3$ otherwise me may think that stars have a huge parallax and therefore and are therefore at a small distance of us as illustrated be the figure below:
	\begin{figure}[H]
		\centering
		\includegraphics[scale=0.5]{img/cosmology/parallax_big_dipper_north_star.jpg}
		\caption[]{Shift angle of the Big Dipper that seems huge if we do not subtract the obliquity angle}
	\end{figure}
	In the figure above the "North star" may be any fixed star close to either celestial pole of any given planetary body. It might refer to any such star in the Earths remote history or future, situated along the path of the celestial poles in the course of the procession of the Earth's axis.
	
	The identity of the pole stars gradually changes over time because the celestial poles exhibit a slow continuous drift through the star field. The primary reason for this is the precession of the Earth's rotational axis, which causes its orientation to change over time. Precession causes the celestial poles to trace out circles on the celestial sphere approximately once every $26,000$ years, passing close to different stars at different times (with an additional slight shift due to the proper motion of the stars).
	\begin{figure}[H]
		\centering
		\includegraphics[scale=0.5]{img/cosmology/north_star_precession.jpg}
		\caption[]{The path of the north celestial pole amongst the stars due to the effect of precession, with dates shown (source: Wikipedia)}
	\end{figure}
		
	\subsection{Planets' Motion}
	We will briefly turn our attention to the movements of the planets in ideal and situations simplified in the point of view on an observer on Earth. We consider that all the movements will be in the same plane (coplanar) perfectly circular and constant...

	\textbf{Definition (\#\mydef):} The planets that are closer to the Sun than the Earth (whose radius is less than one astronomical unit AU\footnote{defined as an average of $149,597,870,700$ [m] ((about $150$ million kilometers). In ISO 80000-3, the symbol of the astronomical unit is "ua".}), that is to say the planets Mercury and Venus are "\NewTerm{inferior planets}\index{inferior planets}", the other planets (Mars and beyond) are named the "\NewTerm{outer planets}\index{outer planets}".
	
	\subsubsection{Synodic and Sidereal period}
	One of the many tools used in Astronomy are the formulas used to determine Orbital Motion. There are two basic forms of orbit periods:
	\begin{itemize}
		\item Sidereal Period
		\item Synodic Period
	\end{itemize}
	A "\NewTerm{sidereal period}\index{sidereal period}" is an actual measure of a complete orbit relative to the stars (since the stars are unmoving - or at least moving very slowly). A "\NewTerm{synodic period}\index{synodic period}" is a rotation of a planet so that it appears to be in the same place in the night sky.
	
	The synodic period of a planet (or satellite )is the time needed by this planet to return to the same configuration Earth-Planet-Sun (if we consider this particular case), that is to say in the same place in the sky relatively to the Sun, as seen from Earth. This period differs from the sidereal rotation period of the planet because the Earth itself moves around the Sun. Accordingly, it is the period of apparent revolution, the duration between two conjunctions Planet-Sun as viewed from Earth.

	The term generally refers to the time between two identical aspects of the object (opposition, conjunction, etc.) and thus depends on the three bodies involved.
	
	To mathematically study the problem in question, let us consider the following diagram with two planets describing a perfectly circular orbit at a constant angular velocity and in the same plane and in the same direction and where we have $\omega_1>\omega_2$ (thus the inner planet is faster than the outer planet):
	\begin{figure}[H]
		\begin{center}
		\includegraphics[scale=1]{img/cosmology/synodic_period_schema.jpg}
		\end{center}	
		\caption{Basic scheme for determining the synodical period}
	\end{figure}
	where $P_1$ and $P_2$ are two planets which we will denote the respective sidereal  periods by $T_1$, $T_2$ and for which we deduce the angular velocities:
	
	If we take as zero time, the time when the two planets are both aligned with the $X$ axis and at the same side of this axis (so in "inferior conjunction"), then the angle between this axis and each of the planets is:
	
	We have respectively:
	
	We seek therefore all the instants $t$ where the following relation is satisfied for a fixed $\alpha_{12}$:
	
	Therefore it comes:
	
	If we look from time zero the first (next) conjunction ("superior conjonction"), this is equivalent to put that $\alpha_{12}=\pi$ and therefore that:
	
	If we look from time zero the first (next) conjection ("inferior conjonction"),  this is equivalent to put that $\alpha_{12}=2\pi$ and therefore that:
	
	In the case where $\omega_2>\omega_1$ (typically Earth and one of its outer planets), the same reasoning leads us to:
	
	Here are some periods synodic and sidereal planets of the solar system relatively to Earth:
		
	As we can see from this table, we can make some of empirical observations:
	\begin{enumerate}
		\item For the inner planets: The closer we get to the Sun, the more the synodical period is short, indeed in the proved relation above, the more $T_1$ is small more $T$ decreases. So if there was a rotating planet very near the Sun, both sidereal and synodic periods are substantially equal.

		\item When we approach the Earth, the period increases. If there was a planet near to Earth, we would then have $T_1$ value close to $T_2$ value and the synodical period would be very large.

		\item For the outer planets: The synodic period decreases when the planet is farther from the Earth and approaches terrestrial sidereal period of $365$ days. We see well for Neptune, if we discovered a planet even further its synodic period would approach even more the $365$ days.
	\end{enumerate}
	
	\pagebreak
	\subsubsection{Planet's apparent retrograde motion}
	The "\NewTerm{retrograde motion}\index{retrograde motion}" of a planet is an apparent motion of this planet which gives the impression that it stop in his path in the "direct movement" to start reversing. This phenomenon is the result of the difference between the revolution speed of the planet and the Earth around the Sun.

	The example below shows roughly what a terrestrial observer (yellow dot) can be observed by monitoring month after month, the apparent motion of Mars (cyan point):
	\begin{figure}[H]
		\begin{center}
		\includegraphics[scale=0.6]{img/cosmology/retrograde_motion.jpg}
		\end{center}	
		\caption{Retrograde motion principle (source: Wikipedia)}
	\end{figure}
	Or more explicitly:
	\begin{figure}[H]
		\begin{center}
		\includegraphics[scale=0.8]{img/cosmology/retrograde_motion_mars.jpg}
		\end{center}	
		\caption{Apparent retrograde motion of Mars in 2003 as seen from Earth (source: Wikipedia, author: Eugene Alvin Villar)}
	\end{figure}
	To study this phenomenon mathematically, we will consider the following figure:
	\begin{figure}[H]
		\begin{center}
		\includegraphics[scale=0.8]{img/cosmology/retrograde_motion_study_figure.jpg}
		\end{center}	
		\caption{Basic scheme for the study of planet's retrograde motion}
	\end{figure}
	with two planets describing a perfectly circular orbit at a constant angular velocity and in the same plane and in the same direction and where we have. It is clear that the inner planet will therefore caught up the outer planet and it will seem to have a retrograde motion as shown in the figure below:
	\begin{figure}[H]
		\begin{center}
		\includegraphics[scale=0.8]{img/cosmology/retrograde_motion_explicative_sheme_for_time_zero_choice.jpg}
		\end{center}	
		\caption[]{Figure to illustrate the choice of zero time}
	\end{figure}
	As the reader can check it in the figure above we see that the retrograde motion with respect to the fixed stars begins when the angle between the two planets is zero and it ends when the angle between the two planets pass through a maximum.

	Therefore, in the prior previous figure, we have:
	
	So to know the time between when the moment where the angle is zero between the two planets, reaches a maximum and decreases again, we simply need determine when occurs the sign of change in the previous function. To do this we just search when the derivative is zero:
	
	By applying the derivation rules seen in the section of Differential and Integral Calculus:
	
	Hence after simplification:
	
	We develop all this:
	
	and we simplify a first time:
	
	and second:
	
	and finally a third one:
	
	and after rearrangement:
	
	We simplify using trigonometric identities proved in the section  Trigonometry:
	
	The values of $t$ that satisfy this relation gives us the sign change we were looking for.

	If $t_0$ is the first value of $t$ that satisfies the equation, we have:
	
	The next value of $t$ will be such that:
	
	and therefore:
	
	If we introduce the rotation periods, we have:
	
	To come back to:
	
	it may be more convenient to write it in the traditional following form:
	
	So far, we have only do geometry. No law of gravitation intervened in the calculations. As the radius are unknown or little known (at least historically), we will use the Kepler's third law (periods law) that is for recall:
	
	where for recall $D$ is the semi-major axis of the orbit, and if it is circular, it becomes a simple radius. So we have:	
	
	Therefore:
	
	hence:
	
	A numerical application with for Mercury $T_1\cong 87.95$ [d] and for Earth $T_2\cong 365.25$ [d] the value:
	
	Value we have represented in the diagram below:
	\begin{figure}[H]
		\begin{center}
		\includegraphics[scale=1]{img/cosmology/retrograde_motion_first_value.jpg}
		\end{center}	
	\end{figure}
	and therefore:
	\begin{figure}[H]
		\begin{center}
		\includegraphics[scale=1]{img/cosmology/retrograde_motion_second_value.jpg}
		\end{center}	
	\end{figure}
	and therefore we have:
	
	then a new cycle:
	
	etc. What gives in schematic form:
	\begin{figure}[H]
		\begin{center}
		\includegraphics[scale=1]{img/cosmology/retrogradiation_cycle_diagram_principle.jpg}
		\caption[]{Retrogradiation cycle diagram principle}
		\end{center}	
	\end{figure}
	\begin{tcolorbox}[title=Remark,colframe=black,arc=10pt]
	At specific points on Mercury's surface, an observer would be able to see the Sun rise part way, then reverse and set before rising again, all within the same Mercurian day. This apparent retrograde motion of the Sun occurs because, from approximately four Earth days before perihelion until approximately four Earth days after it, Mercury's angular orbital speed exceeds its angular rotational velocity. Mercury's elliptical orbit is farther from circular than that of any other planet in the Solar System, resulting in a substantially higher orbital speed near perihelion.
	\end{tcolorbox}
	\begin{figure}[H]
		\centering
		\includegraphics[scale=0.45]{img/cosmology/retrograde_motion_mars_saturn.jpg}
		\caption{Real sequence of exposures showing Mars and Saturn retrograde motion (author: Tunç Tezel)}
	\end{figure}
	
	\pagebreak
	\subsection{Lagrange Points}
	A "\NewTerm{Lagrange point}\index{Lagrange point}" (denoted by L), or "\NewTerm{libration point}\index{libration point}" is a position in space where the gravitational fields of two bodies in orbit around each other, and of substantial masses, combine to provide an equilibrium point to a third body of negligible mass, such that the relative positions of three bodies are fixed.

	We will in the developments that follow take time prove at best that such points are at the number of $5$ rated L1 to L5 respectively.

	It may be helpful to make a presentation of these points and their properties before going through the mathematical part. This may help in understanding the subject.

	We will immediately consider the following diagram:
	\begin{figure}[H]
		\begin{center}
		\includegraphics[scale=1]{img/cosmology/lagrange_points.jpg}
		\caption[]{Representation of the five Lagrangian point in the Sun-Earth system}
		\end{center}	
	\end{figure}
	There are five Lagrange points:
	\begin{enumerate}
		\item[L1:] On the line defined by the both masses between them (this is the most easy point to interpret intuitively: it is for example the point where the gravitational attraction of the Sun is compensated by that of the Earth).

		\begin{tcolorbox}[colframe=black,colback=white,sharp corners]
		\textbf{{\Large \ding{45}}Example:}\\\\
		We consider an object orbiting around the Sun, closer to the latter than the Earth but on the same line. This object undergoes a solar gravity greater than that of Earth, and therefore spins faster around the Sun than does the Earth. But Earth's gravity partially counteracts that of the Sun, which slows it down. The more we approaches this object of the Earth the more this counteract effect is important. At some point, the point L1, the angular speed of the object becomes exactly equal to that of the Earth.
		\end{tcolorbox}
		

		\item[L2:] On the line defined by the both masses, beyond the smaller (a bit less intuitive as is the point where the cumulative effect of the Sun and Earth will compensate the centrifugal force).

		\begin{tcolorbox}[colframe=black,colback=white,sharp corners]
		\textbf{{\Large \ding{45}}Example:}\\\\
		The principle is similar to the previous case, but on the other side of the Earth. The object should rotate more slowly than Earth because the solar gravity is lower, but the extra gravitational field due to the Earth tends to accelerate it. At some point, the point L2, the object rotates at exactly the same angular velocity as the Earth around the Sun.
		\end{tcolorbox}
		
		\item[L3:] On the line defined by the two masses, beyond the larger (intuitive based on physical considerations: it is clear that an object diametrically opposite to the Earth relatively to the Sun would have the same orbital period as the Earth and therefore would be fixed relative to the Earth-Sun system).

		\begin{tcolorbox}[colframe=black,colback=white,sharp corners]
		\textbf{{\Large \ding{45}}Example:}\\\\
		Identically to the L2 point, there exists a point a little further away than the Earth relatively the Sun, where a negligible mass object would be in equilibrium.
		\end{tcolorbox}
		

		\item[L4 \& L5:] On the apexes of two equilateral triangles whose base is formed by the two masses.
		
		\begin{tcolorbox}[colframe=black,colback=white,sharp corners]
		\textbf{{\Large \ding{45}}Example:}\\\\
		This is a subtle balance between the centripetal force exerted by the two main masses and the centrifugal force of the masses considered at the points of interest. L4 is ahead of the smaller mass in its orbit around the large one, and L5 is late. These two points are sometimes named "\NewTerm{triangular Lagrange points}\index{triangular Lagrange points}" or "\NewTerm{Trojans point}\index{Trojans point}".\\

	Remarkably, the last two points do not depend on the relative masses of the two bodies as we will prove it.
		\end{tcolorbox}

	\end{enumerate}
	For the first three Lagrangian points, stability appears only in the plane perpendicular to the line occupied by the two masses. For example, for the L1 point, if we move an object perpendicular to the line between the two masses, the two gravitational forces will play to bring it back to the starting position. The equilibrium is stable. However, if we move it near to one the two masses, then the field of that latter will prevail over the other and the object will tend to get closer. The equilibrium is unstable. For L4 and L5 points, stability is obtained due to Coriolis forces acting on the objects moving away from the point.
	
	Given the stability issues given above, we have no natural object around point L1, L2 and L3 at least in the solar system. However, they still represent an interest in scientific achievements because they allow savings of fuel for orbit control and attitude. This is not valid for point L3, due to its distance from Earth which only application what that utopic one made by Sci-Fi and comic books authors that place an Anti-Earth twin-planet but which mass was too high in relation to the theory stated above. However, space missions use L1 and L2: the case of the probe SOHO since 1995 (Solar and Heliospheric Observatory) a Sun observation station located at L1 (1.5 million kilometers from Earth) or WMAP (Wilkinson Microwave Anisotropy Probe) satellite or Planck satellite (to study the cosmic microwave background at $2.7$ [K]) close to the point L2 as will be the James Webb telescope in 2018 (the radiation of Earth there are relatively low and those of the Sun attenuated by the Earth which do a screening effect).
	
	The points L4 and L5 being stable, we find many natural celestial objects. In the Sun-Jupiter system, hundreds of asteroids, known as "Trojan asteroids", clump there together (around $1,800$ identified in April 2005). We count also  some in the Neptune-Sun systems and Mars-Sun system. Curiously, it seems that the Saturn-Sun system is not be able to accumulate such celestial objects because of the Jovian disruption. We also find objects to these points in the Saturn planetary system of Saturn: Saturn-Tethys with Telesto and Calypso with the L4 and L5 points and Saturn-Dione with Helen to the point L4 and Pollux at the L5 point. In the Sun-Earth system, there is no known large object to the Trojans points, but it was discovered a slight overabundance of dust in 1950. Slight dust clouds are also present for the system Earth-Moon; this make scientific abandon the idea to place there a space telescope as it was envisaged once.
	
	Strictly speaking, these 5 points exist only for two bodies in circular rotation one around the other. Once the orbit of the two bodies is elliptical, these points are no longer equilibrium points. In practice, if the orbit is slightly elliptical, as is the case for real planets, we can find stable orbits oscillating not departing too much of the regions corresponding to the Lagrangian points and this is well named a "\NewTerm{halo orbit}\index{halo orbit}". So halo orbit is a periodic, three-dimensional orbit near the L1, L2 or L3 Lagrange points in the three-body problem of orbital mechanics. Although the Lagrange point is just a point in empty space, its peculiar characteristic is that it can be orbited. The first mission to use a halo orbit was ISEE-3, launched in 1978. It traveled as we already mention to the Sun–Earth L1 point and remained there for several years. The next mission to use a halo orbit was in fact Solar and Heliospheric Observatory (SOHO), a joint ESA and NASA mission to study the Sun, which arrived at Sun–Earth L1 in 1996. It used an orbit similar to ISEE-3.
	
	So we will consider in space an isolated system of two bodies $A$ and $B$, of mass $M_A$ and $M_B$ in gravitational interaction. These two bodies are assumed to be in circular orbit (for simplicity!) one around the other, in the manner of a two-star system (binary system) or a planet-satellite like system (Saturn-Titan example) . We seek to determine if there are relative equilibrium point to the system of the two rotating body for  a third body also in circular motion in the same plane (of sufficiently low mass to avoid disturbing the motion of the system of the two main bodies).
	\begin{figure}[H]
		\begin{center}
		\includegraphics[scale=1]{img/cosmology/lagrange_points_configuraton_study.jpg}
		\end{center}	
	\end{figure}
	Let O be the centroid (\SeeChapter{see section Classical Mechanics}) of these two stars (or celestial objects in general). Let us consider a Galilean reference frame (in rectilinear and uniform motion therefore !) or origin O. Compared to this reference frame, we assume that the axis $AB$ rotates at a constant angular velocity $\omega$ of fixed axis $\vec{k}$ (perpendicular to the page in the figure and directed towards the reader) and that the distances $r_A=\overline{A\text{O}}$ and $r_B=\overline{B\text{O}}$ also remain constant.
 
 We know from our study of Classical Mechanics that a circular motion the centrifugal force is given by:
	
	So we have (equilibrium between centrifugal and centripetal forces) to guarantee the equilibrium:
	
	By simplifying and summing these two relations:
	By simplifying and summing these two relations:
	
	with in what will follow $AB=r_A+r_B=R$.
	
	Let us onsider a rotating reference frame $R'$ linked to our stars as shown in figure above: $\vec{i}$ will be a collinear unit vector to $AB$, $\vec{j}$ a unit vector perpendicular to $\vec{i}$ and in the rotating plane of the planets and finally $\vec{k}=\vec{i}\times\vec{j}$ co-linear to $\vec{omega}$.

	We consider in this rotating frame (with stars) a third star $S$ of mass $m$ negligible relatively ot $M_A$ and $M_B$, subject to the gravitational attraction of $A$ and $B$.

	Now let us denote by $\vec{a}_{R'}$ the acceleration of $S$ with respect to $R'$, $\vec{v}_{R'}$ its speed and $\vec{e}_r$ the collinear  unit vector to $\overrightarrow{\text{O}S'}$ where $S'$ is the projection of $S$ in the plane O$xy$, and $r=\overline{\text{O}S'}$ (in the figure above, we assumed $S$ in the plane O$xy$, so $S$ and $S'$ are indistinguishable).
	
	$S$ is thus subjected to two forces, one $\vec{F}_A$ directed along $A$ and the other $\vec{F}_B$ directed along $B$, forces of respective intensities :
	
	In a Galilean reference frame, these two forces apply to $S$ an acceleration given by the law of composition of accelerations in a circular reference frame (\SeeChapter{see section of Classical Mechanics}):
	
	But, in our configuration the pulsation (radial velocity) is assumed constant and the drive acceleration is zero since we assumed $R'$ as the main repository. Therefore it comes:
	
	We also have:
	
	where according to the figure all components are positive. The calculation of the cross product then gives (\SeeChapter{see section Vector Calculus}):
	
	So finally:
	
	Let us rather write this relationship into the form:
	
	We then obtain, by projecting on the three axes $x$, $y$ and $z$, the derivatives taken with respect to time $t$ the following system:
	
	with:
	
	so that the coordinates $(x,y,z)$ of the point $S$ are those of an equilibrium point, then it is trivial that in the rotating frame with the stars $A$ and $B$ that:
	
	We then get the following system:
	
	It is also immediately that the third equation has for only solution $z=0$ and thus ultimately the system reduces to:
	
	The third equation simply means that the equilibrium positions are in the plane O$xy$ (we could suspect it a bit ...). The other two, we will see it later, lead us to consider five solutions that are our five Lagrangian points L1, ..., L5.

	If we draw plot with an appropriate software the acceleration (respectively force) with the isoclines highlighted (curves on which the acceleration is equal) we get:
	\begin{figure}[H]
		\begin{center}
		\includegraphics[scale=1]{img/cosmology/lagrange_points_two_bodies_isoclines_plot_3d.jpg}
		\end{center}	
		\caption{Isoclines of the two-body system}
	\end{figure}
	where we see that a short distance of the bodies the gravitational potential energy dominates, but that a large distances the centrifugal potential predominates and the shape of the surface is similar to that of a paraboloid.

	By requesting the software to plot only the isoclines projected on a plane we get:
	\begin{figure}[H]
		\begin{center}
		\includegraphics[scale=1]{img/cosmology/lagrange_points_two_bodies_isoclines_plot_2d.jpg}
		\end{center}	
		\caption{Projected isoclines of the two body system on a plane}
	\end{figure}
	where we have highlighted the five Lagrange points and where the stars (or celestial objects) are represented by blue dots and the centroid of the system by a green dot. It seems that isoclines are named in the astronomy field "\NewTerm{Roche equipotentials lobes}\index{Roche equipotentials lobes}". Otherwise seen:
	\begin{figure}[H]
		\begin{center}
		\includegraphics[scale=1]{img/cosmology/lagrange_points_two_bodies_isoclines_plot_2d.jpg}
		\end{center}	
		\caption{Projected isoclines of the two body system on a plane}
	\end{figure}
	where we have highlighted the five Lagrange points and where the stars (or celestial objects) are represented by blue dots and the centroid of the system by a green dot. It seems that isoclines are named in the astronomy field "\NewTerm{Roche equipotentials lobes}". Otherwise seen:
	\begin{figure}[H]
		\begin{center}
		\includegraphics[scale=1]{img/cosmology/lagrange_points_two_bodies_isoclines_plot_2d_and_3d.jpg}
		\end{center}	
		\caption{Projected isoclines of the two body system on a plane}
	\end{figure}
	For those wishing to reproduce these figures with MATLAB here is how first proceed mathematically. From what we got previously, we have explicitly and in writing in a more academic form, the following relation:
	
	As the point $S$ is supposedly in equilibrium the last term vanishes (its speeds are zero in the rotating frame!). It then remains:
	
	We have proved for recall in the section of Classical Mechanics that:
	
	Then we have:
	
	Thus taken by unit mass of the satellite:
	
	The application of this relations in MATLAB 2013a then gives (sorry it's a bit long and its probably possible to do better ...):
	\begin{lstlisting}[language=MATLAB]
		%We build the grid plot that we will by anticipation densifiate where are the objects of interest
		x1=linspace(-7E8,-8E5,150);
		x2=linspace(8E5,1.2E8,150);
		x3=linspace(1.6E8,7E8,150);
		x=x1+x2+x3;
		y=linspace(-7E8,7E8,450);
		[X,Y]=meshgrid(x,y);
		%These masses and G are real but the rest is fictitious so that the plot is readable 
		f=@(x,y) -(1.3346E20)./(sqrt((x-450).^2+y.^2))-(1.0038E19)./(sqrt((x-449999550).^2+y.^2))-(6.9E-7.*(x.^2+y.^2));
		z=f(X,Y);
		%We eliminate the values that are too big on Z to have an esthetical result to see
		for i=1:450;
		   for j=1:150;
		      if (z(i,j)<-0.8E12) %to do with meshc a nice plot, limit to  -8E11
		         z(i,j)=-0.8E12;
		      end; 
		   end; 
		end; 
		contour(X,Y,z,100); mesh(X,Y,z); meshc(X,Y,z); 
		az = 100; el = 25; view(az, el);
		axis([-7E8 7E8 -0.8E9 0.8E9 -8E11 -4E11]);
		colorbar; light; camlight('right');
	\end{lstlisting}
	That gives:
	\begin{figure}[H]
		\begin{center}
		\includegraphics[scale=0.8]{img/cosmology/lagrange_points_two_bodies_3d_matlab.jpg}
		\end{center}	
		\caption{Lagrange plot and isoclines with MATLAB 2013a}
	\end{figure}
	The reader will notice that it is difficult to intuitively this configuration of the potential. In the rotating frame with the centroid of the two solid bodies, the potential resulting from the combination of rotational and gravitational potentials present 3 extrema L1, L2 and L3 on the right containing the both bodies. One of these maxima is between the two bodies as expected intuitively. The other two maxima are on the line connecting the two objects, but both pn either side... which is more surprising. They come from the contribution to the potential of the rotating frame which can be difficult to model intuitively.
	
	\subsubsection{Equilibrium points of the first type}
	What we mean by equilibrium positions of the first type are simply solutions located on the line $\overline{AB}$ such that $y=0$ which is equivalent to study only:
	
	with therefore:
	
	To this situation, we will consider three possible corresponding sub-cases respectively L1, L2 and L3 as we will immediately see it.
	
	\paragraph{L1 Lagrange point}\mbox{}\\\\\
	In this first sub-case, we consider:
	
	What is also equivalent to have:
	
	This allows us to write:
	
	in the following simplified form:
	
	Now to be able to say something about the possible solutions to this equation derive the left hand side. We then get:
	
	This term is strictly increasing from $-\infty$ to $+\infty$ when $x$ describes $]-r_A,r_B[$. So there is a unique solution and equilibrium point denoted L1 (first Lagrange point) between $A$ and $B$.

	If we typically consider the Sun-Earth case where $M_A>M_B$ and therefore $r_A<r_B$ then on $x=0$ we have:
	
	what is immediately negative. The equilibrium position will be obtained for a positive value of $x$ we will have to determine.

	This value can be obtained by considering a limit case: when $M_B$ tends to $0$ (corresponding to a massive celestial object $A$ turning around a mass $B$ of a much much smaller celestial object), then $A$ tends to O, $r_A$ tends to $0$ and therefore:
	
	with $R=\overline{AB}$. Therefore, in this limit case:
	
	becomes in approximation:
	
	and therefore:
	
	So the only value of $x$ satisfying this relation will be $x=R$.

	In other words, the equilibrium point L1 we are looking after here is between $A$ and $B$ moves near $B$, that is near the less massive celestial object (which corresponds well to the first figure that we used to show the location of the five Lagrange points).

	By this observation we can make the following calculations: 
	\begin{figure}[H]
		\begin{center}
		\includegraphics[scale=1]{img/cosmology/l1_point_configuration_study.jpg}
		\end{center}	
		\caption[]{Configuration to mathematically determine the position of point L1}
	\end{figure}
	We have from the definition of center of gravity (\SeeChapter{see section Classical Mechanics}):
	
	As our study is done relatively to the centroid we have $\vec{r}_G=\vec{0}$ and therefore:
	
	From the above relation by taking the norm, we have obviously:
	
	The distance between the two celestial objects $A$ and $B$ remaining constant and being equal to $r=r_A+r_B$ we write:
	
	We deduce trivially two relations (the second being obtained by exactly the same reasoning as the first):
	
	But since $M_A \gg M_B$ we can write roughly the first relation in the following approximate form (Taylor series):
	
	and since:
	
	we have also:
	
	So with $M_A\gg M_B$:
	
	According to the limiting case studied previously, we can assume $L$ at the neighborhood of $B$ such that it is possible to write:
	
	with $\varepsilon\ll 1$.
	
	Either using:
	
	Then we have:
	
	by neglecting the infinitely small therm of order $2$.

	Hence:
	
	Now in the mentioned  configuration the equilibrium is given by:
	
	Therefore:
	

	Now the third Kepler's law  gives us:
	
	Therefore:
	
	After simplification:
	
	Therefore:
	
	Hence:
	
	Since $1/\varepsilon^2$ is much greater than $1$ and assuming that $3M_A/M_B$ then we have also:
	
	Thus finally:
	
	and therefore:
	
	If we take the $A$ for the Sun and $B$ for the Earth, then:
	
	We find that the distance $\overline{LB}$ is then equal approximately to:
	
	which is the L1 point where was placed the satellite SOHO (since the latter will thus never  have its field of view obscured by the shadow of the Earth or the Moon).

	A special case of the L1 point to consider is when $M_A=M_B=M$, then $r_A=r_B=r$, then O is the midpoint of $\overline{AB}$. Then we have:
	
	Therefore:
	
	becomes:
	
	
	\paragraph{L2 Lagrange point}\mbox{}\\\\\
	In this second sub-case, we consider:
	
	Therefore we are looking for the equlibrium points beyond $B$.

	Thus we have:
	
	which becomes simply:
	
	The left side is a strictly increasing function of $x$ from $-\infty$ to $+\infty$ when $x$ describes $[r_B,+\infty]$. So there is a unique solution, and an equilibrium point beyond $B$. This point is denoted: L2.

	This value can be obtained by considering a limit case: when $M_B$ tends to $0$ (corresponding to a massive celestial object on $A$ around which a much smaller mass object $B$ turn around), then $A$ tends to O, $r_A$ to $0$ and therefore:
	
	with $R=\overline{AB}$. Therefore, in this limit case:
	
	becomes approximately:
	
	and so:
	
	So the only value of $x$ satisfying this relation will be $x=R$. The L2 point therefore ends up being being merged with $B$.

	Knowing this limit case, let us do a more detailed study. Consider the following diagram relatively to our previous limit case:
	\begin{figure}[H]
		\begin{center}
		\includegraphics[scale=1]{img/cosmology/l2_point_configuration_study.jpg}
		\end{center}	
		\caption[]{Configuration to mathematically determine the position of point L2}
	\end{figure}
	and let us consider $M_A\gg M_B$ without forgetting that in this scenario $x>r_B$.

	We then have almost the same developments as for L1 but with the difference that:
	
	becomes:
	
	and that instead of having:
	
	We have:
	
	and therefore:
	
	Still with:
	
	and therefore:
	
	which corresponds to the Lagrange point L2.

	A special case again about L2 is when $M_A=M_B=M$, then $r_A=r_B=r$, then O is at the midpoint of $\overline{AB}$. Then we have:
	
	Therefore:
	
	becomes:
	
	It is no longer possible to extract the roots here (at least to my knowledge). It must be done through a numerical approximation. In Maple 4.00b, we simply put:

	\texttt{>solve(-1/(r+x)\string^2-1/(x-r)\string^2=x/(8*r\string^3),x);allvalues(");}
	
	and the only feasible solution in $\mathbb{R}$ is then $x\cong 2.8r$ and the others being in $\mathbb{C}$.
	
	\pagebreak
	\paragraph{L3 Lagrange point}\mbox{}\\\\\
	In this third sub-case, we consider:
	
	So we look for the equilibrium points beyond $A$.

	Thus we have:
	
	which becomes simply:
	
	The left side is an increasing function of $x$ from $-\infty$ to $+\infty$ when $x$ describes $]-r_A,-\infty]$. So there is a unique solution, and one equilibrium point beyond $A$. This point is denoted: L3.

	This value can be obtained by considering a limit case: when $M_B$ tends to $0$ (corresponding to a massive celestial object $A$ turning around much smaller object $B$), then $A$ tends to O, $r_A$ to 0 and therefore:
	
	with $R=\overline{AB}$. Therefore, in this limit case:
	
	becomes approximately:
	
	and therefore:
	
	So the only value of $x$ satisfying this relation will be $x=R$. The point L3 will finish to merge with the position diametrically opposite to that of $B$.

	Knowing this limit case, let us do a more detailed study now. Consider the following diagram relative to our previous limit situation:
	\begin{figure}[H]
		\begin{center}
		\includegraphics[scale=1]{img/cosmology/l3_point_configuration_study.jpg}
		\end{center}	
		\caption[]{Configuration to mathematically determine the position of point L3}
	\end{figure}
	and let us still consider $M_A \gg M_B$ without forgetting that in this scenario $x<-r_A$.

	We will first consider the following approximation:
	
	and this one also (since $\overline{\text{O}A}$ tends to zero as the celestial object $A$ becomes very massive):
	
	Since then:
	
	We have also (...):
	
	when at the limit where the celestial object $A$ is really massive, we fall back on the first term ...

	With the last two relations, we have:
	
	if we neglect the terms of the second order.

	Furthermore, we have also:
	
	Let us recall the equilibrium condition:
	
	And let us put everything we got until now inside it:
	
	What becomes after simplifications:
	
	after a small approximation:
	
	after simplification:
	
	Hence:
	
	and finally:
	
	\begin{tcolorbox}[title=Remark,colframe=black,arc=10pt]
	Form some Si-Fi authors, for recall... this point L3 opposite to the Earth relatively to the Sun would hide us a hypothetical planet that we would be forever hidden to us by the Sun.
	\end{tcolorbox}
	
	\subsubsection{Equilibrium points of the second type}
	The equilibrium positions of the second type are those for which $y\neq 0$. In other words the points outside of the line $\overline{AB}$, but still in the plane $\text{O}xy$.

	Thus, our system of equations remains:
	
	
	\paragraph{L4, L5 Lagrange points}\mbox{}\\\\\
	To determine the remaining equilibrium points, we can divide the second equation of the system such that the system becomes:
	
	\begin{figure}[H]
		\begin{center}
		\includegraphics[scale=1]{img/cosmology/l4_l5_point_configuration_study.jpg}
		\end{center}	
		\caption[]{Configuration to mathematically determine the position of points L4,L5}
	\end{figure}
	where $\overline{AB}$ is obviously the distance between $A$ and $B$ and $D$ is the centroid of the system given by (\SeeChapter{see section Classical Mechanics}):
	
	which are the radii of gyration of the bodies $A$ and $B$.

	It is easy to verify that the sum of both previous distances is equal to $\overline{AB}$ and their proportion $M_B/M_A$. Another form of $\overline{DB}$ (which will be useful) is obtained by dividing the numerator and denominator by $M_A$:
	
	We know according to our previous calculations that $\overline{AS}=\overline{BS}$ but this is insufficient. We still want to know the angles of the vertices $A$, $B$, $S$, and this is what we will look for now.

	In this context, if a satellite $S$ is in equilibrium, there will always remain at the same distance of $A$ or $B$. The center of rotation of the $3$ points is the point $D$, the mass $A$ itself revolves around it. If the satellite, $S$, remains stable, the three bodies have the same orbital period $T$. If $S$ is immobile in this frame in rotation it will not be subject to the Coriolis force but only to centrifugal force of $A$ and of $B$.
	
	Let us denote by $v_B$ the roation speed of $B$ and $v_S$ the rotation speed of $S$. Then we have:
	
	and:
	
	From whose we get that:
	
	and:
	
	So we can equate these two expressions:
	
	This merely expresses the well known fact that if two objects rotate together, the furthest one from the centroid is the fastest. Speeds are proportional to the distances from the centroid.

	The centrifugal force on $B$ is in equilibrium with the gravitational force of $A$ and it is expressed by:
	
	Thus by simplifying:
	
	Similarly, the centrifugal force applied on $S$ is:
	
	It is balanced by the forces of attraction $\vec{F}_A$,$\vec{F}_B$ of the objects $A$ and $B$. However, only the components of these forces located on the line $R$ oppose efficiently to this centrifugal force. Hence:
	
	
	and as:
	
	We then have:
	
	In addition, the forces applied to $S$ and perpendicular to $R$ must vanish. If not, the object $S$ would follow the largest mass and would not remain in position and would therefore no longer be in equilibrium. We must then have:
	
	Or, after substitution and simplification:
	
	Of all the equations obtained up to now the only that bother us are those containing both speeds and angles $\alpha$,$\beta$. This requires that we must arrive to eliminate what is convenient to have only the last two parameters (that is to say: the angles).

	For this, we take the square:
	
	We multiply both sides by $\overline{AB}^2$ and we divide by $1+\dfrac{M_B}{M_A}$:
	
	which is similar to:
	
	Thus equaling:
	
	So we removed the speed of $B$. Now, let us multiply both sides by $\left(1+\dfrac{M_B}{M_A}\right)R$ and divide by $\overline{AB}^2$:
	
	which is similar to:
	
	Therefore:
	
	By dividing by the whole by $GM_A$ we find:
	
	And as we have proved at the beginning $\overline{AS}=\overline{BS}$ that we will denote by $R'$, then we have:
	
	and let us recall that we have:
	
	Therefore:
	
	This allows us to write:
	
	And multiplying by $\sin(\beta)$:
	We can now notice a thing (not easy to see...). If $R'=\overline{AB}$ (that is that the triangle $ABS $is equilateral) the previous relations simplifies to:
	
	But, if the triangle is really equilateral, then we have (\SeeChapter{see section Trigonometry}):
	
	Hence:
	
	What can finally write:
	
	Which is just the sine theorem for the triangle $SDB$ (\SeeChapter{see section Trigonometry}) and is therefore certain. Returning back, we can now prove that all previous equations are satisfied if and only if $ABS$ is equilateral. If we had not put $ABS$ as equilateral, we would have gotten a different relation of the sine theorem, without possible verification, and the set of equations required for equilibrium at the point $S$ could not be met.

	Conclusion of the thing ... the system gives as a solution:
	
	$ABS$ (or $ABL$ regardless the notation), then forms an equilateral triangle. The two equilibrium points are denoted L4 and L5. L4 is located in advance with respect to the less massive celestial object and and L5 is late relatively to it.
	\begin{figure}[H]
		\begin{center}
		\includegraphics[scale=1]{img/cosmology/l4_l5_final_point_configuration_study.jpg}
		\end{center}	
		\caption[]{L4 and L5 equilateral triangle}
	\end{figure}
	In 2000, $385$ asteroids in the L4 point and $188$ asteroids in the L5 point were counted on the orbit of Jupiter, but located precisely in an equilateral triangle with the Sun and Jupiter either side of Jupiter: these are the Trojan planets. It was also observed two objects at the point $L5$ of Mars discovered in 1990 and 1998.
	
	\pagebreak
	\subsection{Relativistic Doppler-Fizeau Effect}
	The Doppler effect is the difference between the frequency of the transmitted wave and the received wave when the transmitter and receiver are moving relative to each other (\SeeChapter{see section Music Mathematics}). This is an effect that must be take into account astronomy to calculate the distance of the body assuming its known (or estimated)  emission wavelength and measuring its received wavelength or for measuring the speed of rotation (radial velocity) of stars by observing very precisely and successively their opposite edges and measuring the shift of the spectrum obtained.

	In the early 21st century the precision and finesse of the spectra of measures has reached a level that allows to observe even minimal changes in the distance of stars and so speculate on possible planetary satellites (this may work if the plane of the orbit passes through the Earth):
	\begin{figure}[H]
		\begin{center}
		\includegraphics[scale=0.8]{img/cosmology/doppler_effect_radial_velocity.jpg}
		\end{center}	
		\caption{Doppler-Effect radial velocity method (source: ESO Press Photo 22e/05 2007-04-11)}
	\end{figure}
	The Doppler effect of electromagnetic waves must be discussed independently of the acoustic Doppler effect (also named "Galilean Doppler effect") study in the section of Music Mathematics. First, because the electromagnetic waves do not consist of a material movement and therefore the speed of the source relative to the medium does not enter into the discussion, then because their velocity is $c$ (the speed of light) and remains the same for all observers independently of their relative movements. The Doppler effect for electromagnetic waves is thus calculated necessarily using the principle of relativity and is symmetrical with respect to relative movement of the source and the observer (as opposed to acoustic cases).
	
	For an observer in an inertial reference frame, a plane and harmonic electromagnetic wave can be described by a function of the form:
	
	multiplied by an appropriate amplitude factor. For an observer attached to another inertial frame, the components $x$ and $t$ should be replaced with $x'$ and $t'$, obtained by the Lorentz transformation (\SeeChapter{see section Special Relativity}), and that latter will therefore write to describing plane wave:
	
	where $k'$ and $\omega'$ are not necessarily the same as that of the another observer (precisely this is what we want to determine). Moreover, the principle of relativity has allowed us to demonstrate in the section of Special Relativity that:
	
	This assumes that the expression:
	
	remains invariant when we move from one inertial observer to the other. We then have:
	
	Using the Lorentz transformation relations (\SeeChapter{see section Special Relativity}), we have immediately:
	
	By identification, it comes immediately:
	
	If we consider that:
	
	in the case of electromagnetic waves, we can write each of these relation in the form:
	
	The ratio visible in the both expression above is named the "\NewTerm{red shift}\index{red shift}" and is denoted by:
	
	for a movement of the observer relative to the source in the direction of propagation.
	\begin{figure}[H]
		\centering
		\includegraphics[scale=0.19]{img/cosmology/hubble_deep_space_redshift.jpg}
		\caption{High-redshift galaxy candidates in the Hubble Ultra Deep Field 2012 (source: NASA, ESA, R. Ellis (Caltech), and the HUDF Team)}
	\end{figure}
	Obviously if the source or observer don't move away but approach then we not have a red shift but a "\NewTerm{blue shift}\index{blue shift}".
	
	Furthermore, the last relation with the pulsations is most often written in the literature as follows:
	
	Which is written most often in the following form:
	
	Therefore if we measure the both frequencies (supposing that we know what should be the source), then we can also obviously determine the speed $v$ of the observed object.
	
	When a spectrum can be obtained, determining the red shift is rather straight-forward: if you can localize the spectral fingerprint of a common element, such as hydrogen, then the red shift can be computed using simple arithmetic. But similarly to the case of Star/Quasar classification, the task becomes much more difficult when only photometric observations are available:
	\begin{figure}[H]
		\centering
		\includegraphics[scale=0.8]{img/cosmology/red_shift_spectrum.jpg}
	\end{figure}
	It must be recalled that the pulsation offset (and therefore frequency offset) that takes place here is due to a relative motion of the observer with respect to the source and not to something else (or respectively of the source relatively to the observer). Indeed, in our study of General Relativity (\SeeChapter{see section General Relativity}), we will prove that there is a superposition of a shift because of the gravitational field surrounding the emitter that will be considered as caused by the spacetime curvature .

	Finally, for skeptics who want to check in another way that the Doppler phenomenon is well symmetric unlike the acoustic Doppler effect proved in the section of Music Mathematics, here's another approach:

	First, consider that it is the source moving away. If we calculated by the classical relation proved in the section of Music Mathematics the frequency of the signal at the reception would be:
	
	and we must take into account the time dilation for $f$ with (\SeeChapter{see section Special Relativity}):
	
	because the time interval of the fixed observer is longer than that of the source (time is faster for observer at rest).

	It comes then:
	
	and if it is the observer who moves away from the source we provec in the section of Music Mathematics that:
	
	both relationships are indeed symmetric in the special relativistic case (as expected for electrodynamics)!
	\begin{tcolorbox}[title=Remark,colframe=black,arc=10pt]
	Currently, astronomers have courses in astrophysics and their observations are generally studied in an astrophysical context, so there is less distinction between the two disciplines than before.
	\end{tcolorbox}
	A very good example of the application of the Doppler effect is to explore the limits given by measuring the apparent speed. Let's see what it is:
	
	\subsubsection{Apparent speed}
	By measuring the apparent speed of movement of very fast objects in the sky (plasma jets, etc.), astrophysicists have obtained apparent displacement speeds exceeding the speed of light in vacuum!

	In fact, it is an illusion that can occur if the speed of the object is very close to that of light it emits, so close enough to $c$.
	\begin{figure}[H]
		\centering
		\includegraphics[scale=1]{img/cosmology/apparent_speed.jpg}
		\caption{Schematic idea behing the apparent speed}
	\end{figure}
	The object emits light at time $t_0$, it does not instantly reach us but must travel a distance to get to us. We get it after the time:
	
	The object itself, moves with velocity $v$ at an angle $\theta$ with the viewing direction, so at time $t$, the object moved of a distance $vt$. The light emitted by the object at time $t$ must travel the distance (application of Pythagoras thereom)
	
	to reach us (the object has move of a distance $vt\cos(\theta)$ in the direction of observation but moved away from the axis of observation of the distance $vt\sin(\theta)$), so we receive light that was emitted by the object at time $t$ after a time $t_2$:
	
	Between the two positions of the object, it has elapsed the time $t$ but, viewed from the observer, the time interval between the reception of images of these two positions is:
	
	different from $t$!
	
	For a small time interval $t$, we have, by doing a limited Taylor development:
	
	During this time interval, always from the point of view of the observer, the object appears to have moved on the sky plane by a distance of $vt\sin(\theta)$.

	Thus, the apparent speed of the object is:
	
	If we set the angle $\theta$ as being very close to a right angle, then we have the second term of the denominator that is very small which allows us with a Taylor expansion to write a relation that can be found quit often in high-school textbooks :
	
	Let us seek the maximum of this function to understand how such observation is possible by deriving relatively to $\theta$ and by seeking for what value the derivative is zero:
	
	and this vanishes after simplification of the denominator for:
	
	Hence:
	
	The apparent velocity is then:
	
	and is equal to or greater than $c$ if:
	
	Therefore:
	
	Thus we see that it is possible to observe apparent movements faster than light, even though the subject is very fast indeed, but slower than $c$. As it is only an "illusion", there is no contradiction with the theory of relativity.

	Knowing the speed of movement of a celestial object obtained using the Doppler effect and the apparent speed with the observations, it is easy for astrophysicists to determine the angle  $\theta$ by doing a little bit elementary algebra from the following relation:
	
	
	\begin{flushright}
	\begin{tabular}{l c}
	\circled{80} & \pbox{20cm}{\score{3}{5} \\ {\tiny 47 votes,  64.68\%}} 
	\end{tabular} 
	\end{flushright}
	
	%to make section start on odd page
	\newpage
	\thispagestyle{empty}
	\mbox{}
	\section{Astrophysics}
	\lettrine[lines=4]{\color{BrickRed}A}strophysics is an interdisciplinary branch of astronomy which mainly concernes physics and the study of the properties of objects in the Universe (stars, planets, galaxies, interstellar medium for examples) as their luminosity, density, temperature and their chemical composition. The first scientific approaches in this area date from the early 19th century.
	
	\begin{tcolorbox}[title=Remark,colframe=black,arc=10pt]
	Currently, astronomers have courses in astrophysics and their observations are generally studied in an astrophysical context, so there is less distinction between the two disciplines than before.
	\end{tcolorbox}
	
	\subsection{Stars}
	Before addressing the mathematical formalism on the dynamics of stars, we wanted following readers requests, write a small popularized introduction to complete the general knowledge on this field.
	
	The stars are gaseous celestial body whose mass goes from $0.05$ solar mass to more than $100$ solar masses. The brightness of a star (its power radiation) ranges from $10^{-6}$ to $10^6$ times that of the Sun. Roughly, when the mass doubles, brightness is multiplied by. Most of the stars visible to the naked eye in our skies are blue giants of $10^4$-$10^5$ times more luminous than the Sun; they represent only $10\%$ of stars that inhabit our galaxy, the remaining $90\%$ being less luminous than the Sun.
	
	The Astronomers (of Harvard between 1918-1928) have developed a method of classification of stars based on their position in the spectrum, of the spectral absorption lines (spectroscopy). Formerly classified from A to Q, the evoluton of the spectrometry allowed their grouping and organization. Classes are now defined by the letters OBAFGKM, and each is divided into $10$ subclasses, rated from $0$ to $9$. The spectral classification (taken from a continuous spectrum which summarizes only certain lines of the spectrum after passing of light in a given medium) can be crossed with the lighting classes so that we can infer the temperature at the surface of the star (we will prove later how to get this information):
	
	\begin{figure}[H]
		\begin{center}
		\includegraphics[scale=0.9]{img/cosmology/hertzprung_russel_diagram.jpg}
		\end{center}	
		\caption{Hertzsprung-Russel Diagram example (source: Wikipedia)}
	\end{figure}
	And corresponding hypothesized path evolution of our Sun on this same diagram:
	\begin{figure}[H]
		\begin{center}
		\includegraphics[scale=0.6]{img/cosmology/hertzprung_russel_diagram_sunpath.jpg}
		\end{center}	
		\caption{Sun path on Hertzsprung-Russel}
	\end{figure}
	As it evolves, each star describes a particular curve on the HR diagram: it begins by following the "\NewTerm{Hayashi-path}\index{Hayashi-path}" (proto-star and after one of the existing tar) until it reaches the main sequence in which it operates as its core burns hydrogen. When beginning the burning of helium, it goes back up where red giants are concentrated and remains there until nuclear fusion stops; it then collapses on itself to join the white dwarfs or in the case of a certain value of solar masses, neutron stars, black holes, or if its mass is very high, exploding as supernovae.
	
	The O stars were discovered in the late 19th century. They are hot and their spectra look like nebulae. The B are helium stars, A hydrogen stars. The predominant component of F is calcium. G are of the same type as the Sun and K differ very little bit. M are characterized by titanium oxide and S of zirconium oxide, while R and N contain hydrocarbons and cyanogen.
	
	Therefore a star of the mass of the Sun after a stint on the main sequence, becomes a red giant, eventually a planetary nebula (ejection of fuel of the star at long distances), before ending his life as a white dwarf. The end as supernovae cannot be shown in this diagram because of their Luminosity that is to high. Neutron star and Black holes are in the same path than white dwarfs (lower on the right than Procyon B).
	
	A star is initially in hydrostatic equilibrium. Gravitational forces due to its mass are compensated by the internal pressure forces due to the elevated temperature maintained by thermonuclear reactions at low density and to the degeneracy pressure of electrons: 
	\begin{figure}[H]
		\begin{center}
		\includegraphics[scale=0.9]{img/cosmology/star_pressure.jpg}
		\end{center}	
	\end{figure}	
	A star spends almost $90\%$ of his life to fuse hydrogen into helium that builds up in the center. During this phase, it evolves into the "main sequence" of the Hertzsprung-Russian diagram.
	
	For a low mass Main Sequence star, hydrogen fusion is the first energy source that provides radiation pressure to maintain the hydrostatic equilibrium. When hydrogen fusion ends, the star begins to undergo structural changes, and it begins to become a red giant (helium fusion) through what we name a "\NewTerm{helium flash}\index{helium flash}". At the end of the life of a low mass star, the core collapses until the electrons provide a source of pressure to withstand the collapse, and at this stage the star is a white dwarf. For higher mass stars, the early stages of life are the same, but the core of these stars reach higher temperatures, so they can burn more massive species, like Carbon, Oxygen, Neon, Magnesium, and Silicon. Towards the end of its life, a high mass star's core will look like the layers of an onion.
	\begin{figure}[H]
		\begin{center}
		\includegraphics[scale=0.6]{img/cosmology/star_structure.jpg}
		\end{center}	
	\end{figure}
	The core of a high mass star will eventually create iron, but when the core tries to fuse iron, it will die in a catastrophic explosion. The problem is that unlike Hydrogen, Helium, Carbon, etc., the fusion of iron does not release energy (\SeeChapter{see section Nuclear Physics}). When the core contains enough iron, the star implodes in seconds, and all of the mass of the outer part of the star hits the core and rebounds, and the rebound sends a shockwave outward pushing all of the material outside of the core into space in a tremendous explosion, named a "\NewTerm{supernova}\index{supernova}":
	\begin{figure}[H]
		\begin{center}
		\includegraphics{img/cosmology/supernova.jpg}
		\caption{Region of the sky before and after the 1987 supernova in visible light}
		\end{center}	
	\end{figure}
	When the helium mass of a star becomes sufficient, the increase in pressure causes an increase of the temperature thereby initiating the fusion of helium ("\NewTerm{helium flash}\index{helium flash}") into carbon, oxygen and neon creating a second combustion front inside the first. For a star of one solar mass, reactions stop at this stage. The star radius increas and its surface temperature decrease until stabilization. It becomes a red giant $10^4$ times more luminous than before. It goes through various phases of instability and eventually gradually expel its outer layers, forming a "\NewTerm{planetary nebula}\index{planetary nebula}" (from a fraction of parsecs like in size like our Solar System to a little bit more of $1$ parsec - approximately $4$ light years - for the biggest known at this date). 
	\begin{figure}[H]
		\begin{center}
		\includegraphics[scale=0.09]{img/cosmology/planetary_nebula.jpg}
		\end{center}	
		\caption{Planetary Nebula gallery (source: Hubble Space Telescope)}
	\end{figure}
	Its core, with a density of several tons per cubic centimeter, cools down slowly: it become a "\NewTerm{white dwarf}\index{white dwarf}" (we will discuss this process mathematically below). The balance in its core is maintained by the pressure of degeneration of electrons.
	
	For a more massive star, the internal temperature becomes quite important so that the carbon and oxygen can fusion into silicon. In turn, if there is enough mass, silicon will fusion into iron. Combustion fronts develop in a pattern said of "onion skins" (see prior previous figure). As iron is the most stable nucleotide,  it is at the bottom of the valley of stability (\SeeChapter{see section Nuclear Physics}). It can not fusion or split. When the density reaches a critical value (this corresponds to a total mass of the star of more than $8$ solar masses!!!), electron degeneracy pressure can no longer maintain the balance against gravity. In a tenth of a second, the iron core collapses. The other layers of the heart of the star rush towards the collapsed core in the for a wave whose maximum speed corresponds to the sonic radius.
	
	The core density then becomes really huge. There occur inverse $\beta^-$ reactions where protons capture electrons forming neutrons (!!!) and releasing a flow of neutrinos. When the core of the star reaches the nuclear density of approximately $10^{18}\;[\text{kg}\cdot\text{m}^{-3}]$, the compaction stops roughly (the remaining radius at this stage is about $10$ [km] only!). The outer layers of the core bounce by a super elastic shock and come into expansion. When this reflected shock wave reached the sonic radius, the temperature rises so high that give him a value is almost meaningless. The material undergoes a complete photodisintegration (all nucleotides are disaggregated into nucleons gas). Finally by an unclear mechanism, all the outer layers of the star are ejected into space: it is a "\NewTerm{type II supernovae}\index{type II supernovae}".
	
	The collapsed core, made almost entirely of neutrons, will be rotating rapidly if the original star had a nonzero angular momentum (conservation of angular momentum oblige!). The magnetic field is also preserved and far exceeds anything that will probably never be feasible a laboratory. This causes a synchrotron beam which gives the illusion that the star flashes. This is why these young "\NewTerm{neutron stars}\index{neutron stars}" are named "\NewTerm{pulsars}\index{pulsars}".
	
	For very massive stars (above $50$ solar masses), the total mass of the core that collapses could exceed $3$ solar masses. In this case, gravity becomes such that its mass collapses beyond the last repulsive forces and compacted into a singularity. The curvature of space becomes such that almost (without going into the details of some theoreties that have until now not been verified) no material information or radiation can escape beyond the horizon or a volume named the "\NewTerm{Schwarzschild sphere}\index{Schwarzschild sphere}". This is a "\NewTerm{black hole}\index{black hole}". Anything that falls inside loses his identity. A black hole has only three properties: its mass, angular momentum and electric charge. We say that a black hole has "no hair". Moreover, such a singularity should always be hidden by a horizon, be: "dressed" (for more details see the section of General Relativity).
	
	To give an idea of the scales you can see the figure below:
	\begin{figure}[H]
		\begin{center}
		\includegraphics[scale=2]{img/cosmology/size_comparison.jpg}
		\end{center}	
		\caption{Comparison of various planets with various Stars}
	\end{figure}
	And here just for the solar system but without respecting distance but only the proportions of the planets and Sun (high definition image so you can zoom on):
	\begin{figure}[H]
		\begin{center}
		\includegraphics[width=\textwidth]{img/cosmology/solar_system.jpg}
		\end{center}	
		\caption{Solar System Proportions}
	\end{figure}
	
	\pagebreak
	\subsubsection{Stellar Physics}
	We will now see how new stars can be born from huge gas clouds that extend between the stars in galaxies. The interstellar medium is a potential source of new stars, which once completed their life (as a red giant or supernova) can inject some of their material in outer space.

	In fact, nobody really knows in this beginning of this 21st centuery the details of how an interstellar cloud leads to a star because it is a very difficult problem, mainly because of the emergence of a hierarchy of structures, sub-structures, etc. in the cloud as it collapses on itself. Turbulent motions appear, which can not be described simply by the hydrodynamic equations (\SeeChapter{see section Continuum Mechanics}). Further complications arise when we consider the magnetic field on the gas contraction, or supernova explosions in the cloud...

	At least, can we give the necessary conditions for a star to form in an interstellar cloud. For this, several barriers must actually be completed. A first thermal barrier. A second rotational barrier is: a protostar that contracts rotates faster and faster and can literally explode if its speed becomes too high (conservation of angular momentum). Let's examine these two effects.
	
	
	\paragraph{Collapse of an Interstellar Cloud}\mbox{}\\\\\
	Two opposing forces are present in a cloud of mass $M$ and radius $R$: an autogravitation a force which tends to contract the cloud, and thermal pressure force, which tends to explode it.
	
	We can quantify these two opposite forces in terms of energy: the cloud has a gravitational potential energy (negative) and a kinetic energy (positive) due to thermal agitation of the molecules.

	We know (\SeeChapter{see section Classical Mechanics}) that the gravitational potential energy of two masses $m$ and $m$ of particles separated from a distance $r$ is written:
	
	So the external potential energy of a spherical cloud (...) of mass $M$ and radius $R$ is of the order of:
	
	\begin{tcolorbox}[title=Remark,colframe=black,arc=10pt]
	Some practitioners (this is our case) prefer for the following developments use the internal potential energy that is given for recall by (see the proof in the section of Classical Mechanics):
	
	\end{tcolorbox}
	In a gas in thermodynamic equilibrium, a particle has a kinetic energy (\SeeChapter{see section Continuum Mechanics}) of $kT/2$ by degree of freedom (translation, rotation, etc.). So if $\mu$ is the average mass of a molecule of the cloud, the total kinetic energy of the latter will be expressed:
	
	The cloud then collapses if its total mechanical energy is negative, or (according to the previous approximation):
	
	The above equation defines the "\NewTerm{Jean's mass}\index{Jean's mass}" (assuming a spherical and homogeneous distribution). This is the minimum mass (limit) at a given temperature $T$ and density $\rho$ for a cloud begins to collapse until another physical process may intervene to stop the contraction of the gas.
	
	By eliminating the radius with:
	
	In the previous relation, we get:
	
	If we would not have make the choice of the external potential, but rather the internal one (more accurate in our point of view) and we did not approximate $2/3\cong 1$ the final result would have been:
	
	That is traditionally written as:
	
	Therefore if $M_\text{cloud}>M_J$ then the cloud collapse!
	
	As they are many approximations, astrophysicists prefer to write this last relation as:
	
	where $C$ is obviously a constant without units.
	
	\pagebreak
	\subparagraph{Limit Mass Cloud for Ionization (rogue planets)}\mbox{}\\\\\
	Now let us come back on the relation:
	
	A famous question is what is the mass required by a hydrogen cloud to start nuclear fusion and become a star. For this, we can say in a first approximation that for the fusion, hydrogen atoms must have a distance equal to their radius such that we have for the density:
	
	where for comparison, density for iron is $7,874\;[\text{kg}\cdot \text{m}^{-3}]$. We will also take $\overline{m}\cong m_p = 1.6726219\cdot 10^{-27}$ [kg] (we neglect the mass of electrons as it is almost $1,800$ smaller) and for temperature fusion of hydrogen\footnote{We have proved that the ionization energy of hydrogen in the section of Corpuscular Quantum Physics was $13.6$ [eV], hence $157,821$ [K] but to avoid recoupling we take a security factor of $10$} $T=T_i=10^6$ [K].
	
	Therefore:
	
	This value perfectly match the value given by then french version of Wikpedia (given without proof...).
	
	With the $10$ security factor for $T_i$ we would have:
	
	In comparison Jupiter is $0.1\%$ of the mass of the Sun...
	
	Therefore for an initial mass is less than $0.066 M_\odot $ the gas sphere liquefies or solidifies and stabilized in the form of a planet. Jupiter as we have just see has a mass that is not for very near this limit value and we observed with telescopes that the planet is still very slowly contracting. 
	
	Such bodies are named "\NewTerm{sub-brown dwarfs}\index{sub-brown dwarfs}", sometimes referred to as "\NewTerm{rogue planets}\index{rogue planets}".
	
	\pagebreak
	\subparagraph{Limit Mass Cloud for Fusion (black dwarf)}\mbox{}\\\\\
	The name "\NewTerm{black dwarf}\index{black dwarf}" has also been applied to substellar objects that do not have sufficient mass, less than approximately $0.08 M_\odot$, to maintain hydrogen-burning nuclear fusion. These objects are now generally called "\NewTerm{brown dwarfs}\index{brown dwarfs}", a term coined in the 1970s. Black dwarfs should not be confused with black holes or neutron stars.
	
	Beyond ionization mass limit, it is an ionized gas ball that will continue gravitational collapse. If during contraction, the temperature of $10^ 7$ [K] is not reached, the nuclear reactions can be triggered and it is the quantum nature of repulsive forces that will oppose gravity. Electrons are fermions (\SeeChapter{see section Statistical Mechanics}), the principle of Pauli exclusion (\SeeChapter{see section Corpuscular Quantum Physics}) prevents the stack of Electron in the same volume of phase space. This is equivalent to a high pressure which is well above the thermal pressure of the atoms.
	
	The mass that can be stabilized in this state is still:
	
	As $T_f\cong 10^7$ [K], the electrons are non-relativist. The linear moment is such that:
	
	From the incertitude principle (\SeeChapter{see section Wave Quantum Physics}):
	
	Roughly:
	
	to compare with the previous $r_0=0.5\cdot 10^{-10}$ [m]...
	
	Therefore:
	
	For the protons, that are $1,800$ times more massive, the minimum volume is much smaller and can be neglected at this level of our discussion.
	
	We use now the same relation as previously but where we take $T_f$ instead of $T_i$ and $\overline{m}=m_e=9.10938356(11)\cdot 10^{-31}$ [kg] as the proton mass as plasma experiences and intuition gives that electrons because of their small mass take the most kinetic energy (in comparison to the proton for the same charge... or as in heated liquids where small molecules are much more agitated than big one):
	
	Therefore theoretically we have so far:
	\begin{itemize}
		\item $M_i<0.066M_\odot$ we have a big gaz planet of the style of Jupiter or Saturn ("\NewTerm{sub-brown dwarfs}\index{sub-brown dwarfs}" or "\NewTerm{rogue planets}\index{rogue planets}" for recall).

		\item $0.066M_\odot<M_f<0.92M_\odot$ we have a ionized hot star at the limit of starting a nuclear fusion it's for recall a "\NewTerm{black dwarf}\index{black dwarf}" or more suited named a "\NewTerm{brown dwarf}\index{brown dwarf}".

		\item For $M_J>0.92M_\odot$ we have then a star able to initiate thermonuclear fusion.
	\end{itemize}
	Brown dwarf seems to be very difficult to observe. It seems as far as we know that the first one was directly observer in 2016 only (HD 4747 B)... because of their low surface energy emitting.
	\begin{figure}[H]
		\begin{center}
		\includegraphics[scale=0.55]{img/cosmology/brown_dwarf.jpg}
		\end{center}	
		\caption{Sun, brown dwarf and rogue planet (source: Wikipedia)}
	\end{figure}
	\begin{tcolorbox}[title=Remark,colframe=black,arc=10pt]
	The devlopements above are obviously approximations and depends on many other factors. For example, Proxima Centauri, located just $4.2$ light-years away has $12\%$ of the mass of the Sun, and it’s estimated to be just $14.5\%$ the size of the Sun with a diameter of Proxima Centauri about $200,000$ [km] (just for comparison, the diameter of Jupiter is $143,000$ [km], so Proxima Centauri is only a little larger than Jupiter).\\

	But that's not the smallest star ever discovered! The smallest known star right now is OGLE-TR-122b that is part of a binary stellar system. This red dwarf has its radius accurately measured!: $0.12$ solar radii. This works out to be $167,000$ [km]. That's only $20\%$ larger than Jupiter. You might be surprised to know that OGLE-TR-122b has $100$ times the mass of Jupiter.
	\end{tcolorbox}
	
	\paragraph{Nuclear Duration Life}\mbox{}\\\\\
	Once again remember that we have proved in the section of Classical Mechanics that insiste a massive body the gravitational potential energy was given by:
	
	What has this have to do with starshine?

	Well, notice that as $R$ gets smaller, $E_p$ gets more negative then energy is being converted to other forms, like heat. If a star can radiate this heat into space, then gravitational contraction might produce the luminosity of the star.
	
	How much of this gravitational energy can be radiated away? 
	
	Remember that we know that radiation is related to heat that is related to velocity (\SeeChapter{see section Statistical Mechanics}) and that in the section of Continuum Mechanics we proved during our study of Virial theorem that:
	 
	But back to the contracting Sun. The Virial theorem says that half the change in gravitational energy stays with the star (it heats the star throught the atomic agitation). The other half is radiated away.
	
	So, let's say that the Sun has been contracting and was originally much, much bigger.  Initially its gravitational potential energy was tiny (why?), so the change in gravitational energy is:
	
	Now, half of this energy could have been radiated as the Sun shrank:
	
	that gives with our Sun values: $\cong 10^{41}\;[J]$. That's a lot of energy! So how long could it sustain the luminosity of the Sun?:
	
	this time is named the "\NewTerm{Kelvin-Helmholtz timescale}\index{Kelvin-Helmholtz timescale}" and with the values of the sun it gives:
	
	and as we see the value is quite problematic... As our Earth would be older than our Sun. In fact this problem comes the fact that we don't take into account the fuel of start is the nuclear fusion. So let us see a little bit more accurate model:
	
	So the age of the stars is as we will see just now mainly a problem of calculation of nuclear fuel. The resolution of this problem was given by relativity, and in particular by the mass-energy equivalence (\SeeChapter{see section Special Relativity}).

	Even if the detailed description of nuclear reactions in the heart of the Sun was made in the mid-1930s by Hans Bethe, astrophysicists have suspected soon after Albert Einstein's work that the mass-energy equivalence could explain the brightness of the Sun on billions of years, for example through the fusion of ionized hydrogen (proton $p$) into ionized helium (two protons, two neutrons) via a series of steps (the specified energy is the kinetic energy of the different elements):
	
	The positron annihilates instantly with one of the electrons of a surrounding hydrogen and theire mass-energy is liberated in the form of two gamma photons:
	
	After this the deuterium produced in the first stage can fuse with another hydrogen nucleus to produce an isotope of helium:
	
	Finally, two isotopes of helium $^3\mathrm{He}$  may fuse and produce the normal isotope of helium $\tensor[^{3}_2]{\mathrm{He}}{}$ and also two hydrogen nuclei that can start again the reaction in three difference ways named PP1, PP2 and PP3:
	
	And these reactions do not occur all with the same probability and at the same temperatures ...
	
	The measurement of the mass of the proton gives $m_p\cong 1.673\cdot 10^{-27}$ [kg], while the helium mass is   $m_{\text{He}}\cong 6.645\cdot 10^{-27}$ [kg], that is to say a loss in atomic mass (we neglect the mass of positrons which is $10,000$ times smaller than that of the neutrino):
	
	So a relative loss of mass by fusion (this is the part of the reactions that escapes from the Sun in the form of kinetic energy):
	
	We will prove further below that the Sun emits a power output of:
	
	Therefore its mass consumption per second is:
	
	i.e. its mass decreases by $4.4$ million tonnes per second...
	
	Now we know that this value corresponds to only $0.72\%$ of the mass put in reaction in a fusion. The total mass reacted  is then (rule of three):
	
	Thus, at every second $627$ million tons of hydrogen 1(ionized) fuse into helium 4 with a weight loss of $4.4$ million tonnes which is converted into energy.
	
	Assuming that only the center of the Sun fills the thermal conditions for the fusion ($\cong 10\%$ of its total mass), this brings us to determine the time of nuclear life of the Sun (or any other Star of the same type whose mass is known):
	
	Transforming this in years we have:
	
	
	\paragraph{Internal Temperature}\mbox{}\\\\\
	The stars are assumed to be spherical clusters of hydrogen gas where the interactions between molecules are governed by the gravitational attraction.

	A star has no wall that delimits it, that is to say that there are no external forces coming from vacuum and therefore:
	
	and also not any bulk modulus.
	\begin{figure}[H]
		\begin{center}
		\includegraphics[scale=0.23]{img/cosmology/sun_corona.jpg}
		\end{center}	
		\caption{Sun Corona Zoom (source: NASA Goddard Space Flight Center)}
	\end{figure}
	\begin{figure}[H]
		\begin{center}
		\includegraphics[scale=0.75]{img/cosmology/global_sun.jpg}
		\end{center}	
		\caption{Global Sun overwivew (source: NASA)}
	\end{figure}
		
	Using the Virial theorem in the section of Continuum Mechanics that gives us:
	
	We have for a homogeneous spherical gas of radius $R$ and mass $N$ composed of $N$ bodies, the relations the following relations proved for the first one in the section of Continuum Mechanics of the second in the section of Classical Mechanics:
	
	Therefore:
	
	where for recall $k$ is the Boltzmann constant.

	Which gives:
	
	in order to not make the confusion between to constant of ideal gaz denoted $R$ and the radius it is more convenient to write the latter relation as (and making the Boltzmann constant more explicit):
	
	With for a given star $N$ being the ratio of the total mass of the star on the average mass of a molecule.
	
	For the Sun it comes that $T\cong 10^7$ [K].
	
	This is the central temperature of the Sun. Optical measurements measured from Earth only give the surface temperature (chromosphere), thus $6,000$ [K]. The calculated internal temperature is about $1,600$ times higher than at the surface. Independent methods based on nuclear reactions in the center of the Sun (measurement of solar neutrino flux) give the same order of magnitude, but the precise values differ by a factor of $2$-$3$.
	
	\paragraph{External temperature}\mbox{}\\\\\
	We have proved in the section of Thermodynamics that the Stefan-Boltzmann law permits to calculate the temperature of a heated body from its emittance or its internal energy in terms of density such as:
	
	with:
	
	being the Stefan-Boltzmann.

	Let us take an interesting example that concerns us directly:
	
	The average emittance also named "\NewTerm{average bolometric emittance}\index{average bolometric emittance}" received by the Earth outside the atmosphere, also named "\NewTerm{solar constant}\index{solar constant}" (which is in fact not constant ... on a scale of several billion years), is directly measurable in orbit and is equal to $\sim 1,373\;[\text{Wm}^2]$.
	
	Knowing the average distance from the Sun to be about $1.496\cdot 10^{11}\;[\text{m}]=1\;[\text{UA}]$  (Astronomical Unit), we can calculate the surface of the sphere $S$ at $R=1$ [UA] and thus the solar power $P$. Thus:
	
	and:
	
	Assuming known the radius of the Sun as being $r_{\odot}\cong 6.9599\cdot 10^8$ [m], we can calculate its surface $S$ and is solar radiative emittance $M_{\odot}(T)$. So:
	
	and:
	
	\begin{tcolorbox}[title=Remark,colframe=black,arc=10pt]
	The radiating surface of a star is named "\NewTerm{photosphere}\index{photosphere}". Indeed, as stars, excepting neutron stars, have no solid surface, the photosphere is typically used to describe the Sun's or another star's visual surface.
	\end{tcolorbox}
	Using the Stefan-Boltzmann law, we can now calculate approximately the thermodynamic temperature of the photosphere:
	
	which is very accurate to direct measurement!!! More generally the previous relation is written:
	
	or respecting the notation of optical geometry:
	
	So since we can estimate the luminosity and the temperature of a star (or something that looks like...) we can also estimate it's radius!
	\begin{figure}[H]
		\begin{center}
		\includegraphics{img/cosmology/photosphere.jpg}
		\end{center}	
		\caption{Layer's view of our Sun (source: NASA)}
	\end{figure}

	Planck's law (\SeeChapter{see section Thermodynamics}) applied at this temperature allow us to calculate the spectral distribution of solar radiation and then we see that the maximum of the intensity is in the visible range (our visible range!!!) spectrum which is from $400$ [nm] to $700$ [nm].
	
	\paragraph{Equation of Hydrostatic Equilibrium}\mbox{}\\\\\
	The absence of changes in most stars over timescales of hours or days indicates that the forces acting on the matter in the stars are essentially perfectly balanced (remember figure showed earlier above). Here we analyze this constraint in more detail.

	In the figure above the we have represented a piece mass shell in a spherically symmetric star where we consider in reality a very small cylindrical element between radius $r$ and radius $r + \mathrm{d}r$ hence the fact that the element surface $\mathrm{dS}$ is considered as being the same (but we can neglect the pressiur variation as it can be very big even in a small height difference):
	\begin{figure}[H]
		\centering
		\includegraphics{img/cosmology/histrostatic_equilibrium.jpg}	
	\end{figure}
	If we denote by $M(r)$ the mass of stat in the smaller radii and $\Delta m$ the mass in the cylinder, we get:
	
	Now for the pressure we have (net force due to difference in pressure between upper and lower faces):
	
	Using the definition of the derivative but applied to the pressure:
	
	Therefore:
	
	Now we have as mass element that is given as we know by:
	
	Applying Newton's second law to the cylinder:
	
	This sum must be equal to $0$ everywhere if the star is indeed static. Therefore:
	
	After simplification we get the "\NewTerm{equation of hydrostatic equilibrium}\index{equation of hydrostatic equilibrium}" or "\NewTerm{stellar structure equation}\index{stellar structure equation}":
	
	So far we already estimated roughly the core temperature of the Sun and of the photosphere. Let us now first estimate roughly its average pressure:
	
	A better estimation is given by using the equation hydrostatic equilibrium:
	
	and integrate (assuming density is constant):
	
	So we have an expression for the central pressure:
	
	That is to $6$ times more than the previous roughly approximation and compared to direct measurement method this seems more accurate!
	
	We also have for mean density of the sun:
	
	to compare with the density of pure water that is $1,000 \; [\text{kg}\cdot \text{m}^{-3}]$ or to that of iron that is $7,874 \; [\text{kg}\cdot m^{-3}]$. So we understand better why space image of the sun looks like a big liquid sphere of gas as because of the gravity conditions the pressure is such that the gas is reduce to a density greater than that of water in average!!!!!
	
	Here is a figure of what we think so far as comparison for Jupiter that is like a non-initiated star:
	\begin{figure}[H]
		\centering
		\includegraphics{img/cosmology/jupiter_layers.jpg}	
	\end{figure}
	
	Using the equation of hydrostatic equation we can estimate roughly the density of the photosphere:
	
	So it is obvious that the density of the sun decreases continuously outward from the center. The visible surface of the Sun (i.e. the photosphere) is a very thin layer, only about $500$ [km] thick as compared to the radius of the Sun. The density of the photosphere is very, very low, about $0.2\cdot 10^{-4} \;[\text{kg}\cdot \text{m}^{-3}]$ as estimated by observations. Therefore the average density of the Sun can only be explained by the density of its core that is very high, about $160,000\;[\text{kg}\cdot\text{m}^{-3}]$ , much higher than any material that we know.

	
	
	\pagebreak
	\paragraph{Brightness}\mbox{}\\\\\
	The "\NewTerm{intrinsic bolometric brightness}\index{intrinsic bolometric brightness}" of a star corresponds to the total power radiated in the entire electromagnetic spectrum in the direction of the observer expressed relatively to the total power radiated by the sun. Assuming all stars spherical and isotropic , we can express it in solar units:
	
	The radiated power $P$ is calculated, as we know, by multiplying the radiative emittance (Stefan-Boltzmann law) by the surface of the star:
	
	The intrinsic bolometric luminosity of a star is therefore proportional to the square of its radius and the fourth power of its surface temperature. Taking the Sun as a reference, the constants are simplified. We can the write:
	
	with $r_{\odot}\cong 6.9559\cdot 10^8$ [m] and $T_{\odot}\cong 5,780$ [K] hence $c^{te}\cong 1.85\cdot 10^{-33}\;[\text{K}^{-4}\text{m}^2]$.
	
	In astrophysics, we also use a logarithmic scale to express the bolometric luminosity of a star: the "absolute magnitude $M$". This unit has an empirical origin that will be explained below.
	
	\paragraph{Shining (apparent brightness)}\mbox{}\\\\\	
	Perhaps the easiest measurement to make of a star is its apparent brightness. I am purposely being careful about my choice of words. When I say apparent brightness, I mean how bright the star appears to a detector here on Earth.
	
	\textbf{Definition (\#\mydef):} The "\NewTerm{brilliance}\index{brilliance}" or "\NewTerm{shining}\index{shining}" or "\NewTerm{apparent brightness}\index{apparent brightness}" $b$ of a star is the density of radiation received by the observer, that is to say equal to the flow of energy divided (power of the star at its surface) divided by the sphere surface with the  radius equal to the distance which separates the observer from the star:
	
	The brilliance decreases therefore with the square of the distance (as in myna other filed of physics):
	\begin{figure}[H]
		\begin{center}
			\includegraphics{img/cosmology/apparent_luminosity_inverse_square.jpg}
		\end{center}
	\end{figure}
	It is important to notice that this quantity has no direct relation with the physical intrinsic  properties of the respective star (unlike the bolometric brightness!).
	
	Thus, two identical stars can have the same apparent brightness if (and only if) they lie at the same distance from Earth. However, as illustrated in Figure below, two different stars can appear equally bright if the more luminous one lies farther away. A bright star (that is, a star with large apparent brightness) is a powerful emitter of radiation, is near Earth, or both. A dim star is a weak emitter, is far from Earth, or both:
	\begin{figure}[H]
		\begin{center}
			\includegraphics{img/cosmology/apparent_luminosity.jpg}
		\end{center}	
		\caption{Apparent luminosity}
	\end{figure}	
	 The luminosity of a star, on the other hand, is the amount of light it emits from its surface. The difference between luminosity and apparent brightness depends on distance as we know now. Another way to look at these quantities is that the luminosity is an intrinsic property of the star, which means that everyone who has some means of measuring the luminosity of a star should find the same value. However, apparent brightness is not an intrinsic property of the star; it depends on your location. So everyone will measure a different apparent brightness for the same star if they are all different distances away from that star.
	
	Apparent brightness is the brightness perceived by an observer on Earth and absolute brightness is the brightness that would be perceived if all stars were magically placed at the same standard distance. There can be a great difference between the total amount of radiation a star emits and the amount of radiation measured at the Earth's surface.
	
	In astrophysics, we also use another scale of measurement where the apparent brightness is given by another magnitude of empirical origin: the apparent magnitude, which will be explained immediately below.
	
	\paragraph{Apparent magnitude}\mbox{}\\\\\
	Ptolemy in 137 AD had defined a scale of six magnitudes to express the brightness (shining) of stars, the first for the brightest and the sixth for the stars just visible to the naked eye ($6$ magntitudes and therefore $5$ gaps).

	During the 19th century, with the arrival of new photometric observations techniques (photographic and photoelectric), the scale of magnitude was replaced by that of "\NewTerm{apparent magnitude}\index{apparent magnitude}" $m$ that has been defined so that this new scale is close to the old one.

	The definition is the following:
	\begin{itemize}
		\item The scale is logarithmic in base $10$ (for convenience of the magnitude of manipulated quantities)

		\item There are $5$ magnitude gaps corresponding to an apparent brightness ratio of $1$ for $100$ ($1: 100$)

		\item The scale is inverse (high magnitude corresponds to a small apparent magnitude/ brightness).
	\end{itemize}
	Using these definitions, we can construct a relative way relating the shining (brilliance) of two stars to their apparent magnitude $m$.

	For a star $1$ two hundred times brighter than a star $2$, the stat $1$ is $5$ magnitude  units above the star $2$ (remember that the scale is reversed). So a ratio of:
	
	corresponds by definition to:
	
	We can then put the relations:
	
	By applying the rule of three, we build:
	
	By simplifying, we find the "\NewTerm{Pogson's Formula}\index{Pogson's Formula}" which expresses the relation between (visual) apparent apparent magnitudes and brilliance (shining) of two stars:
	
	Apart from small corrections, the brightness of Vega\footnote{Brightest star in the constellation Lyra at this day (21st century). It is actually a relatively close star at only $25$ light-years from Earth, and, together with Arcturus and Sirius, one of the most luminous stars in the Sun's neighborhood} ($\alpha$ Lyr) still serves as the definition of zero magnitude for visible and near infrared wavelengths. The brightness of Vega is exceeded by four stars in the night sky at the 21st century at visible wavelengths (and more at infrared wavelengths) as well as bright planets such as Venus, Mars, and Jupiter, and these must be described by negative magnitudes. For example, Sirius, the brightest star of the celestial sphere, has an apparent magnitude of $-1.4$.
	
	To get an idea of the (visual) apparent magnitudes relatively to Vega here are some examples: Sun $m_{\odot}=-26.74$, Full Moon $m=-15$, Venus maximum $m=-4.8$, Sirius $m=-1.4$ (spectral type A1 and distant of $8.6$ light years), limit perceived with the naked eye $6$, perception limit through an amateur telescope of $15$ [cm] at this date (2003) $m=13$ limits of perception Hubble space telescope $m=30$.
	\begin{tcolorbox}[colframe=black,colback=white,sharp corners]
	\textbf{{\Large \ding{45}}Example:}\\\\
	Now as we know that the apparent magnitude of the Sun is $-26.74$ (brighter), and the mean apparent magnitude of the full moon is $-12.74$ (dimmer) the difference in apparent magnitude is obviously that $\delta m=14.00$\\

	With this information reconsider that the Pogson formula also gives by construction the ratio of the luminosity. So relatively to our example we get:
	
	After rearranging we get therefore:
	
	Or better to have a nicer number:
	
	The Sun appears about $400,000$ times brighter than the full moon.
	\end{tcolorbox}
	It should be noticed that the (visual) apparent magnitude does not exactly match the real apparent magnitude, because the eye is not equally sensitive to all wavelengths. The blue or red stars seem less bright to the eye than they actually are because some of the radiation is in the ultraviolet, respectively in the infrared.

	It is therefore necessary to clarify whether it is a  visual or bolometric apparent magnitude. In general, astrophysicists use bolometric magnitudes in their publications.
	
	\paragraph{Absolute magnitude}\mbox{}\\\\\
	The absolute magnitude $M$ (not to be confused with the notation of emittance seen in the section of Geometrical Optics) of a star is also a logarithmic scale , expressing this time the bolometric luminosity $L$!!! It is the quantity presented in ordinate of the Hertzsprung-Russell diagram. The scale of this size is based however on the (visual) apparent magnitude.

	The apparent magnitude and absolute magnitude are bound by the distance from the star. At constant intrinsic apparent brightness, the apparent brightness therefore decreases obviously with the square of the distance as we have already seen. In order to establish a relation, we had to choose a reference distance by a new definition.

	\textbf{Definition (\#\mydef):} The "\NewTerm{absolute magnitude}\index{absolute magnitude}" $M$ of a star is equal to its apparent magnitude $m$ if it is distant of $10$ parsecs ($32.6$ light years).
	
	Therefore taking Pogon's formula that is for recall:
	
	And changing the notations to make it correspond the previous definition:
	
	we get:
	
	And as:
	
	But as it is the same star:
	
	In our case this becomes:
	
	Therefore:
	
	So finally:
	
	\begin{figure}[H]
		\begin{center}
		\includegraphics{img/cosmology/absolute_apparent_magnitudes.jpg}
		\end{center}	
		\caption{Sirius apparent VS absolute magnitudes}
	\end{figure}
	As the Sun-Earth distance in parsec is equal to $4.84814\cdot 10^{-6}$  we get:
	
	Therefore:
	
	\begin{tcolorbox}[title=Remark,colframe=black,arc=10pt]
	For the absolute magnitude $M$ to be accurate, we need stellar models, and know the temperature of the star as we will immediately see it. In practice, the only readily accessible quantity is obviously the observed magnitude, which is actually the combination of the apparent magnitude and the interstellar absorption.
	\end{tcolorbox}
	The absolute magnitude can be obviously rewritten with respect to the absolute bolometric luminosity of the Sun:	
	
	We put for the Sun that $L_{\text{bol},\odot}=1$. Therefore it remains:
	
	
	This latter relation of comparison of the absolute magnitude with the apparent magnitude (which is the actually magnitude observed on Earth) allows estimation $d$ of the distance of the object in astrophysics.
	
	Using the expression of the bolometric luminosity proved earlier above:
	
	the absolute magnitude of star being a direct function of its temperature and radius we can then write:
	
	This is the result we wanted to prove from the beginning: the absolute bolometric magnitude is directly related to the bolometric luminosity of the star, which is why it is one that most interests astrophysicists.
	\begin{tcolorbox}[title=Remark,colframe=black,arc=10pt]
	The distance to nearby stars could be determined by the satellite Hipparcos. By measuring the parallax (measurements of the star position at six-month intervals and applying basic trigonometric rules as seen in the section Astronomy). But beyond a few tens of parsecs, measuring the distance of stars by parallax becomes very imprecise. By studying the spectrum of the star, we can determine its spectral class, its surface temperature and place in the Hertzsprung-Russell diagram. It is therefore possible to estimate its absolute magnitude and roughly calculate its distance.
	\end{tcolorbox}
	This measurement trick is fundamental to cosmology. It is the way we determines the distance to nearby galaxies by measuring the period of some variable stars (we will focus a little bit on that further below).

	The distance of distant galaxies is calculated by measuring the apparent magnitude of supernovae that occur in it. Indeed, the absolute magnitudes of Type Ia supernovae (we recognize them by the lack of hydrogen spectrum lines, and by the decrease in brightness) are well calibrated because the energy released by these stellar explosions is relatively constant.
	
	The stars of the main sequence of the Hertzsprung-Russell diagram are very stable objects. The gravitational force, which tends to contract the star, is exactly compensated by the internal pressure forces, which tend to dilate it. It's when the star becomes a red giant that sometimes the balance is upset. Thus began a phase of instability which results in significant variations in the brightness of the star.

	The breaking of balance is caused by a complex phenomenon that involves variations of transparency of helium layers near the surface of the star. From there, the star begins to experience a series of expansions and contractions controlled by the forces that were formerly balance. When the pressure force prevails, the volume of the star increases. But the gravity slows the movement and eventually cause contraction. The volume of the star will pass below its average value, until the internal pressure opposes the contraction and managed to cause further expansion.

	It is not the size changes that cause the variations in brightness, but those of the temperature. Indeed, as we have prove it earlier above, the brightness of a star varies with the fourth power of the temperature, while it varies with the square of the radius following for recall:
	
	When the volume of the star, however, is lower than average, the temperature is slightly higher and the brightness maximum. At the opposite, the temperature is slightly lower than average and the brightness minimum . The brightness of the star thus changes periodically, hence the name of "variable star" or "pulsative variable star".

	It exists in the Hertzsprung-Russell diagram of a band of instability that crosses this diagram almost vertically just to produce the thermal phenomena in question.

	The two main types of pulsating variables are the Cepheids and RR Lyrae stars. These bodies play a central role in astrophysics. Cepheids are stars of a few solar masses. They are in the helium burning phase after reaching the red giant stage. The stars of solar mass arrived at this point become RR-Lyrae stars. Their brightness varies with a period of between one day and several weeks. The remarkable property of Cepheids is the existence of a relation between the average brightness and the period of their oscillations. For example, the average brightness is $1,000$ times that of the Sun for a period of days and $10,000$ times that amount for a period of several weeks. It is this relation that makes Cepheids one of the basic tools of astrophysics.
	
	\subsubsection{Pulsative Variable Stars}
	The stars of the main sequence of the Hertzsprung-Russell diagram are very stable objects. The gravitational force, which tends to contract the star, is exactly compensated by the internal pressure forces, which tend to dilate it. It's when the star becomes a red giant that sometimes the balance is upset. Thus began a phase of instability which results in significant variations in the brightness of the star.

	The breaking of balance is caused by a complex phenomenon that involves variations of transparency of helium layers near the surface of the star. From there, the star begins to experience a series of expansions and contractions controlled by the forces that were formerly balance. When the pressure force prevails, the volume of the star increases. But the gravity slows the movement and eventually cause contraction. The volume of the star will pass below its average value, until the internal pressure opposes the contraction and managed to cause further expansion.

	It is not the size changes that cause the variations in brightness, but those of the temperature. Indeed, as we have prove it earlier above, the brightness of a star varies with the fourth power of the temperature, while it varies with the square of the radius following for recall:
	
	When the volume of the star, however, is lower than average, the temperature is slightly higher and the brightness maximum. At the opposite, the temperature is slightly lower than average and the brightness minimum . The brightness of the star thus changes periodically, hence the name of "variable star" or "pulsative variable star".

	It exists in the Hertzsprung-Russell diagram of a band of instability that crosses this diagram almost vertically just to produce the thermal phenomena in question.

	The two main types of pulsating variables are the Cepheids and RR Lyrae stars. These bodies play a central role in astrophysics. Cepheids are stars of a few solar masses. They are in the helium burning phase after reaching the red giant stage. The stars of solar mass arrived at this point become RR-Lyrae stars. Their brightness varies with a period of between one day and several weeks. The remarkable property of Cepheids is the existence of a relation between the average brightness and the period of their oscillations. For example, the average brightness is $1,000$ times that of the Sun for a period of days and $10,000$ times that amount for a period of several weeks. It is this relation that makes Cepheids one of the basic tools of astrophysics.
	
	If we know this relationship for a variable star, it is relatively easy, by the determination its period to derive its absolute magnitude $M$. By then measuring its apparent magnitude $m$ we can then calculate the distance in parsec with of the relation proved earlier above:
	
	
	One of the main reasons for constructing the Hubble Space Telescope (HST) was to measure light curves of Cepheid variables in other galaxies. It is especially important to use Cepheids to measure distances to the galaxies in two nearby clusters: the Virgo Cluster (the nearest rich cluster), and the Fornax Cluster (a somewhat sparser collection of galaxies).

	HST can zoom in on a small portion of a galaxy to find and measure Cepheids:
	\begin{figure}[H]
		\begin{center}
		\includegraphics[scale=0.7]{img/cosmology/cepheid_hst_galaxy_ngc1365.jpg}
		\end{center}
	\end{figure}
	and the zoom inside the are of interest:
	\begin{figure}[H]
		\begin{center}
		\includegraphics[scale=0.7]{img/cosmology/cepheid_hst_galaxy_ngc1365_wfpc2_zoom.jpg}
		\caption{Hubble Space Telescope Cepheid Measurement}
		\end{center}
	\end{figure}
	The period-luminosity relation for classical Cepheids was discovered in 1908 by Henrietta Swan Leavitt in an investigation of thousands of variable stars in the Magellanic Clouds. She published it in 1912 with further evidence. Once the period-luminosity relationship is calibrated, the luminosity of a given Cepheid whose period is known can be established. Their distance is then found from their apparent brightness. The period-luminosity relationship has been calibrated by many astronomers throughout the twentieth century, beginning with Hertzsprung. Calibrating the period-luminosity relation has been problematic; however, a firm Galactic calibration was established by Benedict et al. 2007 using precise HST parallaxes for 10 nearby classical Cepheids. Also, in 2008, ESO astronomers estimated with a precision within $1\%$ the distance to the Cepheid RS Puppis, using light echos from a nebula in which it is embedded. However, that latter finding has been actively debated in the literature.

	The following relationship between a Population I Cepheid's period $P$ (in days) and its mean absolute magnitude $\bar{M}$ was established from Hubble Space Telescope trigonometric parallaxes for $10$ nearby Cepheids:
	
	\begin{figure}[H]
		\begin{center}
		\includegraphics[scale=0.7]{img/cosmology/period_cepheid_relation_plot.jpg}
		\caption{Cepheid absolute magnitude - period plot (source: Wikipedia)}
		\end{center}
	\end{figure}
	Cepheids aren't perfect distance indicators. For one thing, their brightness and periods of pulsation can vary with their chemical composition. There's also the problem of crowding and confusion: what if our view of a distant galaxy appears to show a single, varying Cepheid star... but is really a combination of light from the Cepheid and several nearby stars, all mixed together?
	
	\subsubsection{Neutron Stars (magnetars)}
	A neutron star is the collapsed core of a large star ($10$ to $29$ solar masses). Neutron stars are the smallest and densest stars known to exist. With a radius on the order of $10$ [km], they can, however, have a mass of about twice that of the Sun. They result from the supernova explosion of a massive star, combined with gravitational collapse, that compresses the core past the white dwarf star density to that of atomic nuclei. Most of the basic models for these objects imply that neutron stars are composed almost entirely of neutrons, which are subatomic particles with no net electrical charge and with slightly larger mass than protons. They are supported against further collapse by neutron degeneracy pressure, a phenomenon described by the Pauli exclusion principle (see further below). If the remnant has too great a density, something which occurs in excess of an upper limit of the size of neutron stars at $2$-$3$ solar masses, it will continue collapsing to form a black hole (see proof further below).
	
	\paragraph{Chandrasekhar limit}\mbox{}\\\\\
	We have already determined in the section of Classical Mechanics the Schwarzschild radius (in its classical form) that expresses the critical radius of a body for the release speed to it surface to be equal to that of speed of light. We obtained the following relation which typically expressed the radius that have a given celestial boject to have a release speed equal to that of light:
	
	
	In the figure below on the left we have schematic slice through a neutron star. Letters N, n, p, e, $\mu$ refer to the presence of nuclei, fluid neutrons and protons, electrons and muons, respectively. The inner core composition is still uncertain and various exotic possibilities exist, including hyperons and deconfined quark matter. 
	\begin{figure}[H]
		\centering
		\includegraphics[scale=1]{img/cosmology/neutron_star_slice.jpg}	
		\caption{Neutron star slice}
	\end{figure}
	On the figure below we have an overview of what we expect to be the composition of the inner crust:
	\begin{figure}[H]
		\centering
		\includegraphics[scale=0.9]{img/cosmology/neutron_star_inner_crust.jpg}	
		\caption{Neutron star inner crust}
	\end{figure}
	At lower densities, a lattice of superheavy, neutron-rich nuclei is immersed in a fluid of neutrons (which are likely to be superfluid) and a relativistic electron gas. At high enough densities the nuclei might deform and connect along certain directions to form extended tubes, sheets and bubbles of nuclear matter. These nuclear pasta phases might form a layer at the base of the neutron star crust, sometimes referred to as the mantle. Ranges of density and thickness given for each layer represent current uncertainties in the physics of neutron star crusts.
	
	
	\paragraph{Neutron star magnetic field}\mbox{}\\\\\
	As the star's core collapses, its rotation rate increases as a result of conservation of angular momentum, hence newly formed neutron stars rotate at up to several hundred times per second. Some neutron stars emit beams of electromagnetic radiation that make them detectable as pulsars. Indeed, the discovery of pulsars in 1967 was the first observational suggestion that neutron stars exist. The radiation from pulsars is thought to be primarily emitted from regions near their magnetic poles. If the magnetic poles do not coincide with the rotational axis of the neutron star, the emission beam will sweep the sky, and when seen from a distance, if the observer is somewhere in the path of the beam, it will appear as pulses of radiation coming from a fixed point in space (the so-called "lighthouse effect"). The fastest rotation rate for a neutron star was a rate of $716$ times a second or $43,000$ revolutions per minute, giving a linear speed at the surface on the order of $0.165 c$....
	
	So now let us focus on the simplified math approach of the impact of the angular momentum conservation on the magnetic field of the Star. 
	
	From the conservation of angular moment as the core collapses we have (\SeeChapter{see section Classical Mechanics}):
	
	Or,  for a sphere of constant density  (\SeeChapter{see section Geometric Shapes}):
	
	So the final spin frequency is:
	
	or the final spin period is:
	
	The magnetic flux $\Phi$ ($\vec{B}$ times area $S$) through the surface of the core is also conserved in collapse. So roughly   (\SeeChapter{see section Magnetostatics}):
	
	Which means that:
	
	
	Tensile strength against rotation \\
	
	\subsection{Galaxies}
	A galaxy is a gravitationally bound system of stars, stellar remnants, interstellar gas, dust, and (of the supposed...) dark matter. 

	Galaxies range in size from dwarfs with just a few billion ($10^9$) stars to giants with one hundred trillion ($10^{14}$) stars, each orbiting its galaxy's center of mass. Galaxies are categorized according to their visual morphology as elliptical, spiral or irregular:
	\begin{figure}[H]
		\begin{center}
		\includegraphics[scale=0.145]{img/cosmology/classification_galaxies_spitzer.jpg}
		\end{center}	
		\caption{Apparent stars speed anomaly in galaxies rotations}
	\end{figure}
	 Many galaxies are thought to have black holes at their active centers. 

	It seems that there is between $2\cdot 10^{11}$ galaxies in the observable Universe following the actual estimates. Most of the galaxies are $1,000$ to $100,000$ parsecs in diameter and usually separated by distances on the order of millions of parsecs (or megaparsecs). The space between galaxies is filled with a tenuous gas having an average density of less than one atom per cubic meter. The majority of galaxies are gravitationally organized into associations known as galaxy groups, clusters, and superclusters. At the largest scale, these associations are generally arranged into sheets and filaments surrounded by immense voids.
	
	\subsubsection{Radial Speed Anamoly}
	In 1978, Vera Rubin begins to observe that in galaxies, more the stars are distant from the galactic core, the more their angular velocity is high... The initial observation that uniformity of speed was unexpected because the theory of gravity Newton predicted that more distant objects have less speed. For example, the planets of the solar system orbit with a respective speed decreases while growing their respective distance from the sun. We are left with the same problem: how to explain a point measurement is greater than the theoretical value?
	\begin{figure}[H]
		\begin{center}
		\includegraphics{img/cosmology/apparent_anomaly_star_speed_galaxy_rotation.jpg}
		\end{center}	
		\caption{Apparent stars speed anomaly in galaxies rotations}
	\end{figure}
	According to Newton's laws, in a circular path, there is a as we know balance between the centripetal acceleration and gravitational attraction:
	
	The volume of a disk galaxy of radius $R$ and thickness $e$ is (\SeeChapter{see section Geometric Shapes}):
	
	If we consider the mass of the galaxy almost entirely within the radius $R _ {\max}$, corresponding to the maximum speed, of density $\rho$ is given then by:
	
	Making the approximation that the mass is substantially within the range corresponding to the maximum speed, we can write:
	
	Which, introduced into the first equation but rearranged:
	
	 gives:
	
	the law that the maximum speed varies as the $1/4$ power of the mass. After that the speed decrease of the stars should decrease following:
	
	But we must keep in mind that this is a two body relation! In the facts a galaxy should be considered as an isotropic fluid (like the rest of the universe) and therefore it is quite normal that the two-body assumption does not suite the observations. A galaxy can also not be consider as a solid cylinder otherwise by applying $v=\omega r$ the speed of the stars should increase in proportion to the distance to the center of the galaxy.
	
	\begin{flushright}
	\begin{tabular}{l c}
	\circled{90} & \pbox{20cm}{\score{4}{5} \\ {\tiny 28 votes,  80.71\%}} 
	\end{tabular} 
	\end{flushright}

	%to make section start on odd page
	\newpage
	\thispagestyle{empty}
	\mbox{}
	\section{Special Relativity}
	\lettrine[lines=4]{\color{BrickRed}W}e have always considered until now in all our developments that the  interactions (cause and effect) between the body were instantly and the observation of a phenomenon took place instantly after it had taken place. Now, two physicists (Michelson and Morley) during an experiment discovered something that would change radically all of classical physics: the velocity (speed) of light was invariant (constant) regardless of the movement that we had relatively to it!
	
	This observation is even more important that we know that is the light that allows us to perceive and feel things. It should also be taken into consideration that the electrostatic and magnetic fields are, as we have seen in the section of Quantum Field Theory, carried by the vector of interaction that is the photon that moves at the finite speed of light denoted by $c$. This fact also allows us to assume that the gravitational field also has an interaction vector (which would be the "graviton" whose existence seems indirectly proven) that propagates at the speed of light. It is therefore appropriate to take into account this non-immediacy and the consequences that this entails in the observed phenomena to finally be able to decide what is really of what seems to be.
	
	Before we start with the calculations, we need to define a little bit what will be studied in this section (which applies not only to cosmology but... it seemed to us better to put it in this chapter rather than in the chapter of Mechanics or Atomistic).
	
	\textbf{Definition (\#\mydef):} The "\NewTerm{Special Relativity}\index{special relativity}" is a theory confined to isolated inertial frames (Galileans), that is to say, the study of animated frames in a uniform (inertial) rectilinear motion. The reason of this will be given in the statement of Special Relativity principle (see below).
	
	\begin{tcolorbox}[title=Remarks,colframe=black,arc=10pt]
	\textbf{R1.} Restrict the study to inertial frames of course does not does not prohibit that within these, bodies can be animated of a uniform speed or not (an inertial rocket can have bodies inside itself that have a non-uniform movement)!\\
	
	\textbf{R2.} General relativity's purpose (see corresponding section) is to take into account non-inertial frames and in any coordinate system by making use of the power of the tensor calculus to be applicable in any type of space (other than flat one!).
	\end{tcolorbox}	
	Special Relativity is mainly based on three important concepts:
	\begin{enumerate}
		\item The invariance postulate of speed of light
		\item The cosmological principle (see below)
		\item The principle of Special Relativity (see below)
	\end{enumerate}
	It is also important to inform the reader that we will use here many concepts seen in the section of Linear Algebra, Tensor Calculus, Trigonometry, Analytical Mechanics, Classical Mechanics, Electrostatics, Magnetostatics and Electrodynamics. It is therefore strongly advised to have covered these topics before at risk of not understanding what follows.
	
	\subsection{Assumptions and Principles}
	Physics laws express relations between the fundamental physical quantities. If the laws of physics are invariant under Galilean referential change as we have seen in the section of Classical Mechanics, it is not necessarily the same for physical quantities! These can transform from Galilean frame to another according to  simple transformation law as we have seen in the section of Classical Mechanics for velocity for example. It is the same in Special Relativity, but we must now consider what we neglected in our study of Galileo's transformations: the time lag is not the same for two observers if the speed of the light is finite, but the concept of time interval is supposed to be kept invariant!
	
	\subsubsection{Postulate of Invariance}
	Laboratory measurements (Michelson-Morley experiment as we have already mentioned) have, for a long time, shown that the speed $c$ measured in an inertial frame (straight line and at constant speed) is constant regardless of its speed. Then we are taken to state the postulate of invariance of light: the speed of light (vector for the transport of ) can neither be added nor substracted, to the drive speed of the frame in which we measure it (more clearly it means that no matter how fast you move, you will always measure the speed of light as being numerically finite and equal to $c=299,792,458 [\text{km}\cdot  \text{s}^{-1}]$!).
	
	As corollary the principle of Galilean relativity (\SeeChapter{see section of Classical Mechanics}) according to this premise is completely at fault and then we have to develop a new theory that takes into account of this property of light.
	
	\begin{tcolorbox}[title=Remark,colframe=black,arc=10pt]
	It is important to note that we consider that light, within the framework of Special Relativity, the messenger of information from one body to another!!!
	\end{tcolorbox}
	
	\subsubsection{Cosmological Principle}
	We assume that our position in the Universe is typical not only in space as stated in the standard model of the Universe (\SeeChapter{see section Astrophysics}), but also in time. Thus, an astronomer located in a remote galaxy must observe the same general properties of the Universe that we, he lived a billion years ago, or that hed observed it in a billion years.
	
	In fact, it is quite natural to go further and state that: the Universe looks the same in every point, that is to say, it is homogeneous. This homogeneity is therefore sets as the "\NewTerm{Cosmological Principle}\index{Cosmological Principle}".
	
	This principle is not based actual 21th century observations because to much fragmentary compared to the huge size of the cosmos so that they can not establish its validity. It constitutes a presupposition for any physical study of our Universe. Its purpose is relative to its character, essential to any scientific cosmology study, and perhaps to a certain reaction to the geocentric or heliocentric old vision: it is assumed now that no place is special in the cosmos!
	
	\subsubsection{Special Relativity Principle}
	Let us recall (\SeeChapter{see section Classical Mechanics}) that the Galilean transformations tell us that no reference frame can not be considered as an absolute frame because the relations between the physical quantities are identical in all Galileans repositories ("Galilean relativity principle"). The Galilean motion is therefore relative!
	
	In the 20th century physicists noted that an important class of physical phenomena violated the Galilean relativity principle: the electromagnetic phenomena!
	
	By applying the Galilean transformations to Maxwell's equations, we get a different set of equations depending on whether the observer is in a fixed reference or a mobile reference frame.
	
	Indeed, we have proved in the section on Electrodynamics that the electric or magnetic field propagation equation could be written in one-dimensional space as the following d'Alembert equation:
	
	where $\psi$ represents any one of the two fields (electric or magnetic). We name this relation sometimes "\NewTerm{Hertz equation}\index{Hertz equation}".
	
	We also saw in the section of Classical Mechanics that an important factor in the validity of a theory is the invariance of the expression of its laws under a Galilean transformation by putting:
	
	We have also shown in the section of Differential and Integral Calculus that the total differential of a function was written (example with two variables):
	
	Therefore:
	
	Which brings us to simply write (using the physicist method way of life...):
	
	After elimination of $f$ and using the Schwarz theorem (see section Differential and Integral Calculus) and still the physicist way of life:
	
	If we write the same with the time variable:
	
	Ultimately the Galilean transformation of the wave equation supposedly have an invariant form becomes:
	
	To fix the situation, following this example, we can state at least three assumptions:
	\begin{enumerate}
		\item[H1.] Maxwell's equations are false. The correct equations remain to be discovered and must be invariant under a Galilean transformation.
		\item[H2.] Galilean invariance is valid for mechanics but not for electromagnetism (this is the historical solution before Albert Einstein, an "ether" determines the existence of a kind of absolute repository where Maxwell's equations do not change).
		\item[H3.] Galilean invariance is false. There is a more general invariance, it remains to be discovered, which preserves the form of the Maxwell equations. Classical mechanics is to be reformulated so that it is invariant under this new transformation.
	\end{enumerate}
	\begin{tcolorbox}[title=Remark,colframe=black,arc=10pt]
	It turns out that the first two assumptions are excluded by the experimental facts. Moreover, Maxwell's equations integrating the speed of light they are implicitly relativistic.
	\end{tcolorbox}	
	Albert Einstein did not accept the violation of the Galilean relativity principle by electromagnetism. From his perspective, it was necessary to generalize it to all natural laws.
	
	He postulated that the laws of physics should be the same in all repositories Galileans, which means, implicitly, that in the point of view of physical laws, it is not possible to distinguish one from another Galilean frame. This result is most commonly formulated as: no reference is privileged. This principle was named "\NewTerm{principle of relativity}\index{principle of relativity}". Indeed, this relativity is restricted to the case of Galileans frames (also named "inertial frames") exclusively.
	
	In other words, the physic laws should remain unchanged after a change of reference. We must therefore identify new adequate transformations that will substitute to the Galilean transformations.
	
	In the case of non Galileans frames repositories are not indistinguishable anymore. Indeed, imagine a person in a train moving at a constant speed and another person on land. Everyone can then say that it is the other who is in motion (relative) and indiscriminately. By cons, if the train begins to accelerate, although the two individuals can say that this is another speeding, only the one on the train will feel the effect of this acceleration ... and repositories are no more indistinguishable.
	
	Albert Einstein abolished as well as the idea that there is an absolute reference point that does not move and on which we can define an absolute time, an absolute length or absolute mass. However, one can define a privileged reference point for every object in the Universe. It is the frame moving at the same speed and in the same direction as the object. The time measured in this privileged reference frame is minimal and is named the "\NewTerm{proper time}\index{proper time}". Similarly, the size of the object is maximum, it is his "\NewTerm{proper dimension}\index{proper dimension}" or "\NewTerm{proper distance}\index{proper distance}", and its mass is minimal, it is its "\NewTerm{proper mass}\index{proper mass}" (we will do the corresponding detailed mathematical developments further below).
	
		\subsection{Lorentz Transformations/Boost}
		For make possible to $c$ to be invariant (light speed invariance postulate), we must admit that time does not flow the same way for the observer $\text{O}$ that is motionless than for the observer $\text{O}'$ in a reference frame in uniform translation (i.e.: an inertial frame) in the direction of $x$  with relative velocity (the term "relative" is important!) $v$ (caution! the relative speed between the repositories is often denoted in the literature by $u$).
		
	
		\begin{tcolorbox}[title=Remark,colframe=black,arc=10pt]
		A special case of disposal of referential frames in which the space axes are parallel leads to what we name the "\NewTerm{pure Lorentz transformations}\index{pure Lorentz transformations}" or "\NewTerm{special Lorentz transformations}\index{special Lorentz transformations}" and the relative displacement along a particular axis is often named a "\NewTerm{boost}\index{boost}".
		\end{tcolorbox}	
		
		To study the behavior of physic laws, we must bring two clocks that give the times $t$ and $t'$ (the referential frame that contains its clock/measuring instrument is named "\NewTerm{proper referential}\index{proper referential}" or "\NewTerm{proper frame}\index{proper frame}").
		
		Let's set up the following imaginary experiment:
		
		When the observers $\text{O}$ and $\text{O'}$ are superimposed, we set $t = 0$ and $t' = 0$ (clock time sync) and we emit a bright flash\footnote{In fact we should consider the emission of "an element of information". Using light as an example is just convenient for pedagogical purposes. As we will see the results we will get further belov involve a speed limit $c$ that is for sure the speed of light, but in reality we should consider light as a special case of the maximum possible speed of information transfer. The "$c$" can then be seen as the "causality speed".} in the direction of a point $A$ spotted by respectively $\vec{r}$ and $\vec{r}'$:
		\begin{figure}[H]
			\begin{center}
			\includegraphics{img/cosmology/lorentz_pure_transformations_experiment.jpg}
			\end{center}	
			\caption{Configuration for the study of relativistic effects}
		\end{figure}
		It is obvious that when the flash arrive in $A$, the observer $\text{O}$ will measure a time $t$ and $\text{O}'$ a time $t'$.
		
		The observer $\text{O}$ therefore concludes:
		
		The observer $\text{O}'$ therefore concludes:
		
		Since the displacement of $\text{O}'$ is made only along the $\text{O}x$ axis, we have for the two observers:
		
		Moreover, if the path of the light beam coincides within the axe $\text{Ox}$, we have:
		
		This gives us then:
		
		And therefore:
		
		these two relations are equal (zero) at any $x, x', t, t'$ between the two observers. These are the first "relativistic invariant" (equal values regardless of the frame) that we find in a more generalized form when applied to the whole space:
		
		Now iyou should be remembered that in the classical model (Galilean relativity), we would have written that the position of point $A$ for the observer $\text{O}$ from the information given by $\text{O}'$ would be given by $x=x'+vt$ and vice versa (\SeeChapter{see section of Classical Mechanics}) such as:
		
		In the relativistic model, we must admit that time $t$ which is related to $x$ is not the same as $t'$ which is related to $x'$, relativity principle oblige (otherwise it would be difficult to explain the invariance of the speed of light)!
		
		We are then led to try to write the above relation as follows:
		
		where $\lambda$ would be a numerical value to be determined from a given algebraic expression. Because to explain the constancy of the speed of light one possibility is that the space must constantly adjust according to our velocity $v$. What is revolutionary hypothesis as we have already mentioned!
	\begin{tcolorbox}[title=Remark,colframe=black,arc=10pt]
	A reader asked us why we could not write the last relation in the following simplified form (using the relation $x = ct$ obtained above) where point $A$ is on the $X$ axis:
	
	The only reason is that later we will introduce a vector (matrix) notation of this result showing the concept of quadrivector (four-vector) and that it is in the first form of writing (this making explicit reference to time) that we can clearly make the concept of space-time emerge.
	\end{tcolorbox}
	Furthermore, if $t\neq t'$, we must also be able to express $t'$ as a function of $t$ and $x$ in a similar way:
	
	Let us summarize the shape of the problem:
	
	with $\lambda$ to be determined and after:
	
	with $a,b$ to be determined.
	We then seek to determine the relation that give us to know the values of the coefficients $a$,$b$ and $\lambda$ that satisfy simultaneously:
	
	Remembering the previous developments and bearing in mind that in our special case $y '= y $and $z' = z$, the last equation becomes:
	
	Let us distribute:
	
	To satisfy the relation:
	
	We must have:
	
	It is easy to solve (2):
	
	We then introduce this result in (1) and (3) and we arrive at:
	
	If we divide (1') by (2'), we get:
	
	and introducing this latter result into the relation:
	
	we obtain the following remarkable result:
	
	That we frequently denote by:
	
	and which we name "\NewTerm{Michelson-Morley factor}\index{Michelson-Morley factor}" with:
	
	Also introducing:
	
	in:
	
	we get:
	
	Let us write now (to comply with the traditional notations in this field):
	
	with therefore the parameter:
	

	\subsubsection{Displacement four-vector}
	We derive the "\NewTerm{Lorentz transformation}\index{Lorentz transformation}" relations to pass from the values measured by $\text{O}'$ to those measured in $\text{O}$ and vice versa:
	
	who have for property to be covariant (that is to say their relations keep the same structure during a change of a Galilean reference system). We see through these relations that the concept of "time" is something individual relating to the movement we have over other (this is the "\NewTerm{proper time}"). This is why it is not possible to define a "common time" between two people moving relatively and that don't know their respective relative speed (and even here we do not take into account the gravity that distorts space-time ... that we will study in the section of General Relativity).
	\begin{tcolorbox}[title=Remark,colframe=black,arc=10pt]
	If $v$ is much smaller than $c$, we fall back on the Galilean transformation as $\gamma\cong 1$ and $v/c^2\cong 0$
	\end{tcolorbox}
	We can also write the last relations in a more useful way (the reader will notice that this time that for all relations the units of all the terms to the left of equality are the same: it is every time a distance!):
	
	Of course the difference is that the fourth dimension being the time coordinate of "space-time" seems at the contrary of the spatial coordinates to have a privileged direction: the "\NewTerm{arrow of time}\index{arrow of time}" (you can not go back to a given moment time given in the reality - as least as far as we know today - when it is possible when we traverse a purely spatial distance). The direction of time is imposed by the second law of thermodynamics as entropy can only increase (\SeeChapter{see section of Thermodynamics}). If this were not the case then all time could already exist and we could travel in time as we travel on distances and therefore the future should be already written (people that believe in destiny like this...) and we could also go back in time.... However, thermodynamics does not give a particular direction to the time ... so if our time has the direction it has today.... it is because our universe was organized at its creation (so it had a low entropy).
	
	By proceeding in a homogeneisation of units to be able to use more modern and generalized maths than just simple algebra we can see that in fact when we travel in time we travel in physical point of view a distance $ct$. But because $c$ and $t$ are measured in arbitrary human being units physicists prefer to put $c=1$ we mathematical development become more complicate rather than changing the definition of time.
	
	Now can put the Lorentz transformations of coordinate and time in the traditional following matrix form (\SeeChapter{see section Linear Algebra}) which defines the "\NewTerm{Lorentz matrix}\index{Lorentz matrix}" or "\NewTerm{Lorentz-Poincare matrix}\index{Lorentz-Poincare matrix}" or "\NewTerm{Lorentz boost}\index{Lorentz boost}":
	
	and reciprocally:
	
	and the reader can very easily control that with the two previous relations we fall back on:
	
	We have also obviously:
	
	In index form the matrix formulation is written:
	
	which can therefore be written in tensor (SeeChapter{see section Tensor Calculus}) form:
	
	\begin{tcolorbox}[title=Remark,colframe=black,arc=10pt]
	We can see the tensor (the matrix) of Lorentz transformation in some books in the condensed form $L(\beta)$ and sometimes $L_\nu^\mu$ or even $\Lambda_\nu^\mu$.
	\end{tcolorbox}
	The vector:
	
	is named "\NewTerm{space-time four-vector}\index{space-time four-vector}" or "\NewTerm{four-vector displacement}\index{four-vector displacement}".
	
	Notice that since:
	
	the transformation by the matrix $L_\nu^\mu$ conserves the norm (Lorentz invariance). In geometric terms it is thus a "\NewTerm{isometry}\index{isometry}" or an invariance of the dot product by Lorentz transformation as the Lorentz transformations are orthogonal (\SeeChapter{see section Vector Calculus}).
	
	Let us prove this explicitly following the request of a reader! We will take again for the proof only a movement along $x$ and we use:
	
	As $y'=y$ and $z'=z$ to simplify the development we will ignore these both components.
	
	We will also put to simply $c=1$ and therefore $v$ is expressed in percent of $c$ and becomes $v=\beta$:
	
	Now we calculate:
	
	and as $\gamma^2(1-\beta^2)=1$ we get indeed:
	
	
	\paragraph{Wave Equation Invariance}\mbox{}\\\\\
	Now that we have determined the Lorentz transformations, we can check whether the wave equation is invariant with respect to the latter (remember that we have proven earlier that it was not invariant under a Galilean transformation!!!).
	
	Starting from the Lorentz transformation written in explicitly:
	
	we calculate the partial derivatives with respect to $x$ and $t$ (the expression after the second equality has already been proven earlier in this section):
	
	These relation can also be written:
	
	Squared:
	
	In the Maxwell's equations, or rather in the propagation equation of the electric or magnetic field in vacuum, we have proven (\SeeChapter{see section Electrodynamics}) that the following operator appeared:
	
	Substituting in it the previous differential expressions:
	
	We therefore have well:
	
	which shows that a Lorentz transformation leaves invariant this operator (Jackpot!). So we got what we were looking for (the wave equation but in the other reference frame)!

	The reader will also have notice that this only works if and only if the wave propagation speed is the speed of light!

	\paragraph{Hypergeometric interpretation}\mbox{}\\\\\
	Now let us come back to our Lorentz transformations. Let us recall that we have restricted ourselves to the special case where the space axes were parallel (what brought us to define the "pure Lorentz transformations"). This special configuration has an interesting geometric property that sometimes many books use.

	Let us see what this is about:
	
	We have seen in the context of the study of the Lorentz transformations of lengths that we had a special transformation (boost) along one axis, ie the $x$-axis, requiring in this case for the other components:
	
	This allows us to  reduce the transformation matrix $L_\nu^\mu$ ($4\times 4$ matrix that we obtained earlier above) to a $2\times 2$ matrix of components $A$, $B$, $C$ and $D$ such that:
	
	We notice that the components $A$, $B$, $C$, $D$ respect by construction the following expressions:
	
	The first relation can be related with the remarkable identity in hyperbolic trigonometry (\SeeChapter{see section Trigonometry}):
	
	And therefore:
	
	and the second relation that there exists $\alpha_2$ such that:
	
	\begin{tcolorbox}[title=Remark,colframe=black,arc=10pt]
	The choice of the "$-$" sign for $B$ and $C$ is useful because as we always have $\beta \geq 0$ (same for $\gamma$ that is strictly positive) it will impose us at the end of the calculations to have $\alpha\geq 0$. Therefore, as $-\gamma\beta\leq 0$ and $\alpha\geq 0$ the only way for $C$ (and also for $B$) to be negative is to put a "$-$" sign.
	\end{tcolorbox}	
	The third then gives the remarkable addition relation:
	
	and therefore the difference $\alpha_1-\alpha_2$ that we will denoted more simply by $\alpha$ is equal zero. Which validate the relations:
	
	The matrix is therefore presented as follow:
	
	This is (by analogy to the classical one), a "\NewTerm{hyperbolic rotation matrix}\index{hyperbolic rotation matrix}". We will not go further on this analogy as it is not used for practical cases study in this book.
	
	Finally, the special Lorentz transformation of velocity $v$ along the $x$-axis can also be written:
	
	which brings us to write:
	
	The dimensionless quantity $\alpha$ is named "\NewTerm{rapidity}\index{rapidity}" by those who use physics in high energy. The advantage of working with angles is to make the combination of $2$ boosts easier.

	We will stop here regarding the geometric study of Special Relativity finding personally that it has less and less interest to proceed so today (even it is quite funny).
	
	\subsubsection{Velocity four-vector}
	We can also determine the Lorentz transformations for speed. Let us consider again a particle moving in an inertial reference frame O' such that at time $t'$, its coordinates are $(x ', y', z ').$:
	\begin{figure}[H]
		\begin{center}
		\includegraphics{img/cosmology/lorentz_pure_transformations_experiment.jpg}
		\end{center}	
	\end{figure}
	Therefore, the components of the velocity $v'$ are:
	
	So what are the components of the velocity in O (remember that $O'$ go away at speed $v$!)?
	
	Again, we write:
	
	We can differentiate by the time the components of the transformation equations we obtained before and thus we can write:
	
	Therefore we have:
	
	and also same:
	
	and:
	and also same:
	
	And as the constant speed of reference frame $O'$ is given by $\beta=v/c$, we then have:
	
	and vice versa:
	
	Within the limit of classical mechanics, where the speed of light was supposed instantaneous and therefore $c\rightarrow$, we fall back on:
	
	which are the Galilean transformations such as we have seen them in the section of Classical Mechanics.

	As we can see, the speeds transformations do not follow too much the shape of the Lorentz matrix that we determined above for the coordinates. Physicists, not liking what is inhomogeneous, sought to have the same transformations for both.

	Thus, let us take again the speed transformations and let us rewrite rewrite them as below:
	
	These relation can be written differently if we calculate:
	
	Thus simplifying a bit:
	
	Let us put:
	
	and:
	
	and:
	
	where the latter equality means that in order to simplify that the inertial speed and thus the study of only a single component is sufficient and that is the one collinear with the axis of movement.

	With this notation and simplifying it will be easy to determine the temporal component, indeed the relation:
	
	is the written:
	
	The reader will have perhaps notice that we therefore have three $\Gamma$: one related to the inertial speed, the second related the norm of the vector of the particle in the reference frame O and the third related to the norm of the vector in the reference frame O'. But actually following our simplification made earlier above we know that in the repository O' the particle is at the origin in $Y'$ and $Z'$.

	By doing the same for each of the spatial components, we will get in the end:
	
	and here we have reached our goal of homogenization that allows us to write if we put:
	
	the following system:
	
	that is written in tensor form  sometimes as:
	
	The vector:
	
	is itself named the "\NewTerm{four-vector velocity}\index{four-vector velocity}".
	
	\subsubsection{Current four-vector}
	We have defined naturally during our introduction of the electromagnetic tensor field (\SeeChapter{see section Electrodynamics}) the four-vector current:
	
	that we can write:
	
	This means that charge density is related to time, while current density is related to space.
	
	Therefore, considering $\rho_0$ as the charge density in the proper frame moving with velocity $v$ relative to reference frame O' and due to length contraction in the direction of the velocity, the volume occupied by a given load will be multiplied by the factor $\gamma(\vec{v})$ so that:
	
	which is none other than the "\NewTerm{four-current}\index{four-current}" where we see back the four-vector velocity previously determined.
	
	\subsubsection{Acceleration four-vector}
	Having previously obtained a four-vector velocity transformable thanks to the Lorentz matrix let us also look for the equivalent for acceleration.

	The four-vector acceleration is naturally expressed as the derivative with respect to the proper time of the four-velocity $u$ such that:
	
	Let us just recall that the proper time of a particle is the time measured in the coordinate system of the particle, that is to say, in the reference frame where it is motionless. The proper time in the literature is often denoted $\tau$.
	\begin{tcolorbox}[title=Remark,colframe=black,arc=10pt]
	We must be careful and check that the corollary of the assumption of the equivalence principle is true otherwise all General Relativity would collapse (in the early 21st century experiments are still going to try to show a default to this principle)!
	\end{tcolorbox}	
	The reader must first admit that (we will prove this further below) that:
	
	Therefore, we have:
	
	If we introduce the ordinary acceleration $\vec{a}=\mathrm{d}\vec{v}/\mathrm{d}t$ we see that:
	
	then:
	
	Using the vector identity (\SeeChapter{see section Vector Calculus}):
	
	we then find that the four-vector acceleration can be written:
	
	
	The vector:
	
	is named "\NewTerm{four-vector acceleration}\index{four-vector acceleration}" and therefore also transformed using the Lorentz matrix.
	
	We see that if this $v\ll c$ and $\vec{a}=\vec{a}_0$ the last relationship simplifies to:
	
	We thus fall back on the classic acceleration.

	Using the Minkowski metric (see definition further below), denoted $\nu_{uv}$, let us calculate the norm of the four-vector acceleration:
	
	\begin{tcolorbox}[title=Remark,colframe=black,arc=10pt]
	It must well understood that when we write $(\vec{a}+\vec{\beta}\times(\vec{\beta}\times\vec{a}))^2$ it is implicit in this case that we do sum of the squares of the components of the calculations in the brackets.
	\end{tcolorbox}	
	And as:
	
	and:
	
	we put this together to get:
	
	Now we develop the sum $a_ia^i$ of big parenthesis that becomes therefore:
	
	We simplify:
	
	Hence:
	
	But we have the relation:
	
	and the property of the cross product:
	
	Which finally gives us:
	
	Now imagine an object with a uniformly accelerated relative motion $\vec{}_0^2$ (constant acceleration) in our own repository. If we assume our repository as fixed, we have $\vec{v}=\vec{0}\Leftrightarrow \vec{\beta}=\vec{0}$. Therefore:
	
	Verbatim after rearranging the terms and taking the square root if the accelerated motion is made only along a single component:
	
	But, we also have:
	
	So finally, we can write:
	
	Which after integration gives:
	
	We see that the speed $u$ never reaches $c$ while the force (acceleration implicity) is always the same!

	So we have:
	
	which gives us:
	
	After rearranging, we write this:
	
	We are far from the relation of uniformly accelerated motion we have proved in the section of Classical Mechanics and that is for recall:
	
	However, for $t$ close to zero, we fall back on the same Classical Mechanics relation by taking the Taylor expansion to the second order of the square root (\SeeChapter{see section Sequences and Series}):
	
	However, this does not give us the relations of transformation of acceleration components in a simple form. Let's see how to get them.

	First recall that we have obtained for speed:
	
	Then it comes by differentiating:
	
	therefore:
	
	Let us recall now that we have proved that:
	
	differentiating it comes:
	
	We can write:
	
	After simplifying and rearranging we get obviously:
	
	Hence:
	
	hence finally:
	
	and for the components $y$, $z$:
	
	and therefore:
	
	So finally:
	So finally:
	
	Remember that these relations apply when the movements of the reference frames are in uniform translation!
	
	\subsubsection{Relativistic sum of velocities}
	As the speed of light is a speed supposed unsurpassable, we now come to ask ourselves what will be finally the speed of an object launched at a speed close to that of light (for example...) from a reference frame moving also close to that of the speed of light (why not...).

	We must then find a relationship that gives the real speed $V$ from the launch speed $v_2$ and speed of the repository $v_1$.

	We know that for the object launched:
	
	As the one who is concerned does not know the real speed $V$, it should use the Lorentz transformations. Thus, given the expression of $t'$ that we saw earlier it comes:
	
	and given the prior-previous expression of $x'$ we also have:
	
	therefore after rearranging and simplifying a bit:
	
	Hence:
	
	We know that $v=x/t$ so we finally the "\NewTerm{law of compositions relativistic speeds}\index{law of compositions relativistic speeds}" or simply "\NewTerm{velocity-addition formula}\index{velocity-addition formula}" or "\NewTerm{Einstein's velocity addition}\index{Einstein's velocity addition}" relation:
	
	which is then the speed of a moving body in the moving reference frame relatively to that seen as the rest frame (but that in fact should also move at any speed less then $c$).

	And conversely seen from the other moving frame of reference, we have by the same developments (with reverse signs and speed of course):
	
	which is the speed of a moving body in the rest frame relatively to that considered as being in motion (or in other words seen by the moving frame of reference).
	
	\subsubsection{Relativistic lengths variation (length contraction)}
	Let us consider now that the length of an object is given by the distance between its two ends $A$ and $B$. Let us consider this object $\overline{AB}$ motionless in the repository $\text{O}'$ in uniform translation and oriented along the axis $\text{O}'X'$:
	\begin{figure}[H]
		\begin{center}
		\includegraphics{img/cosmology/lorentz_pure_transformations_experiment.jpg}
		\end{center}	
	\end{figure}
	Its length is then the distance between its both ends:
	
	For the observer O, the object is moving. The positions $A$ and $B$ should therefore be measured simultaneously:
	
	So it comes using the relation proved earlier in this section:
	
	the following difference:
	
	hence the remarkable result:
	
	we also find the relation frequently in the literature as follows:
	
	Thus, the length of an observed rule in a moving frame relatively to the proper frame of the rule is less than its own length (which can assimilate in generality to a "\NewTerm{proper length}\index{proper length}"). In other words, the length of a moving object measured by the fixed reference frame will be measured shorter than its real proper size. This phenomenon is named "\NewTerm{length contraction}\index{length contraction}".
	\begin{figure}[H]
		\begin{center}
		\includegraphics[scale=0.95]{img/cosmology/start_trek.jpg}
		\end{center}	
		\caption{Length contraction principle for straight motion (source:?)}
	\end{figure}
	Due to superficial application of the contraction formula some paradoxes can occur. Examples are the ladder paradox and Bell's spaceship paradox. However, those paradoxes can simply be solved by a correct application of relativity of simultaneity. Another famous paradox is the Ehrenfest paradox (high relativistic speed "rigid" rotating disc\footnote{Circumference of a rotating disk should contract but not the radius, as radius is perpendicular to the direction of motion.}), which proves that the concept of rigid bodies is not compatible with relativity, reducing the applicability of Born rigidity, and showing that for a co-rotating observer the geometry is in fact non-euclidean and the we need then to use General Relativity.
	
	\subsubsection{Relativistic time variation (time dilatation)}
	An event is a phenomenon that occurs in a given place at a given time. The origin of time is difficult to determine, we often prefer to define the concept of time interval as the time elapsed between two events as it is often customary (\SeeChapter{see section Principia}).
	
	Let us now consider two consecutive events $A$ and $B$ that occur at the same location $x'$ (!) in the repository in uniform translation:
	\begin{figure}[H]
		\begin{center}
		\includegraphics{img/cosmology/lorentz_pure_transformations_experiment.jpg}
		\end{center}	
	\end{figure}
	For the observer in $\text{O}'$, the time interval is simply:
	
	To measure this time interval, the observer O in the fixed reference repository should also require that $x'$ is common to both events. Then using the relation proved earlier above:
	
	we get:
	
	hence the remarkable result:
	
	what we write under traditional condensed form:
	
	We also deduce taking an infinitesimal time element:
	
	So the observer O (stationary) measures a time interval much larger than that one measured in the moving repository where the phenomenon takes place as it moves quickly. The time in the fixed repository (thus the "\NewTerm{proper time}\index{proper time}" of the fixed reference frame!) seems like dilated compared to that in occurring in the mobile reference frame (that is to say relatively to the "proper time" of mobile reference frame!).
	
	Let us see two application examples that are so famous that they have their even a name so that we will consider them as an table of contents entry of our book:
	
	\paragraph{Hafele–Keating experiment}\mbox{}\\\\\
	In 1971, direct experimental verification of time dilation was performed. Two airplanes in whose had been placed a cesium atomic clock during their regular commercial flights (one flying to the east, the other to west) compared their clocks to a third  atomic clock remained on the ground. This experiment made famous by time is named today "\NewTerm{Hafele-Keating experiment}\index{Hafele-Keating experiment}" ( Joseph C. Hafele, a physicist, and Richard E. Keating, an astronomer).
	\begin{figure}[H]
		\begin{center}
		\includegraphics{img/cosmology/hafele_keating_experiment.jpg}
		\end{center}
	\end{figure}
	Because the Hafele–Keating experiment has been reproduced by increasingly accurate methods, there has been a consensus among physicists since at least the 1970s that the relativistic predictions of gravitational and kinematic effects on time have been conclusively verified.[7] Criticisms of the experiment did not address the subsequent verification of the result by more accurate methods, and have been shown to be in error.

	The idea is obviously that in a frame of reference at rest with respect to the center of the Earth, a clock aboard the plane moving eastward, in the direction of the Earth's rotation, has a greater velocity (resulting in a relative time loss) than one that remained on the ground, while a clock aboard the plane moving westward, against the Earth's rotation, had a lower velocity than one on the ground.
	
	The plane flying eastward lost $59$ [ns] while the plane flying westward gained $273$ [ns] (the Earth rotates on itself in a day, from West to East. It was therefore measured a total difference:
	
	between the two clocks and this difference is even statistically significantly greater than that the one implied by Special Relativity (see detailed calculations just below).

	Let us analyze the experience considering that all repositories are inertial (thus eliminating General Relativity).
	\begin{tcolorbox}[title=Remark,colframe=black,arc=10pt]
	Strictly speaking, the effect of General Relativity (slowing of clocks depending on the altitude in accordance with Einstein's effect proved in the section of General Relativity) is absolutely not negligible since it is equivalent in amplitude that of Special Relativity.
	\end{tcolorbox}
	Let us consider for our study  three inertial reference points, one at the North Pole, one on Earth (elsewhere apart from the North Pole in the idea!) and one in a plane. The time intervals $t_{\text{North}},t_{\text{Earth}}$ and $t_{\text{plane}}$ respectively (which we will abbreviated $t_N,t_E,t_P$ for the following developments), are connected by the previously proven relations (so the North pole is taken as the reference at rest in this experience and therefore the reference proper time!):
	
	where we have:
	
	The repository on Earth and in the plane so have equation relative speeds $v_E$ and $v_P$ relative to the North Pole. The time by plane and on Earth are then linked by:
	
	We gonna rewrite this relation:
	
	We will accept the following approximation:
	
	where we have assumed that at the denominator:
	
	For square roots whose value is anyway close to $1$ (since $c$ is much larger than the considered relative speeds), we can do a Maclaurin expansion to the second order as $x$ approaches zero (\SeeChapter{see section Sequences and Series}):
	
	Then we can write:
	
	Thanks to these tricky successive approximations, we can easily write the difference between the two clocks that is then:
	
	According to the initial assumptions, the cruising speed of the two aircraft from the ground is constant and is denoted $v$. The speed of each plane (!nonrelativistic according to the preceding approximations) is then:
	
	for the plane going eastwards and:
	
	for the aircraft going respectively westward. So:
	
	We will consider that (it's pretty rough ...):
	
	So it remains:
	
	We see well that obviously with the previous approximations we lost the asymmetry of time dilatation between East and West. The reader that this should disturb can then apply directly the numerical values in the prior previous relation.

	The previous result that we get we all successive approximations already lead us to see formally and quickly that the sign of the result will be in agreement with experimental results.

	For a practical numerical application, we will take the constant speed of the commercial planes at that time that was:
	
	and the total time travel of planes was of $41$ hours according to the measurement at the ground, thus:
	
	and a point at equator of the Earth's surface go at the speed:
	
	where the Earth's radius being of $6,371$ [km] (this suppose that the plans are above the equator radius). We then have applying all that numerical values:
	
	which leads to a result very close to the measurement that was performed.

	And using directly the non-approximate version:
	
	where we took this time the speed of the Earth at latitude consistent with the experience in 1971:
	
	So we see that the result is therefore not very consistent with the experience! Indeed, we must now consider in that approximated case the time dilatation due to gravity. We'll have to use the Einstein's effect relation proved in the section of General Relativity for approximated locally flat space:
	
	which expresses for recall the that at the ground time flows slower than the time at altitude $h$.
	
	According to the records of the experiment, the aircraft flew at $10,000$ [m] above sea level. What gives (the acceleration $g$ is not the same on the ground level than in altitude for recall!) and acceleration of time of:
	
	But, we see that the two aircraft were both at the same height, we always have:
	
	So either there are other effects, of the order of General Relativity, which should be taken into account to explain the $67$ [ns] of difference to the experience, or it is a accuracy problem of the time accuracy of the time clock at the time of the experiment.

	In fact, we will see a detailed study of this experience in the section of General Relativity and see that the theoretical values are in very good agreement with experimental results.
	
	\paragraph{Twins paradox}\mbox{}\\\\\
	The twin paradox is a thought experiment in special relativity involving identical twins, one of whom makes a journey into space in a high-speed rocket and returns home to find that the twin who remained on Earth has aged more. This result appears puzzling because each twin sees the other twin as moving, and so, according to an incorrect and naive application of time dilation and the principle of relativity, each should paradoxically find the other to have aged more slowly. However, this scenario can be resolved within the standard framework of special relativity: the traveling twin's trajectory involves two different inertial frames, one for the outbound journey and one for the inbound journey, and so there is no symmetry between the space-time paths of the two twins. Therefore, the twin paradox is not a paradox in the sense of a logical contradiction.
	\begin{figure}[H]
		\begin{center}
		\includegraphics[scale=0.5]{img/cosmology/twin_paradox.jpg}
		\end{center}
	\end{figure}
	We can already consider the famous twin paradox in the framework of Special Relativity to show that the twin paradox does not only apply to non inertial systems. This is a rough approach (knowing that will rigorously discussed the subject in the section of General Relativity).

	Let us consider a rocket taking off at time $t$ zero of the Earth and accelerating to $20$ times the acceleration of Earth's gravity $g$ up to a cruising speed of $90\%$ the speed of light $c$. Let us suppose that the rocket continues at this speed during a terrestrial year and decelerate with the same intensity to resume its journey to Earth and accelerates again for its approach to the Earth and decelerate once again to its final zero velocity.
	
	
	Thus, the total proper time spent for a human remained on Earth is:
	
	For the traveler in the rocket, the proper time during the acceleration phase will be given roughly by:
	
	Thus by integrating (using the usual primitive proved in the section of Differential and Integral Calculus) for one of the phase of acceleration of the rocket it gives:
	
	And the proper time for the part with the constant cruising speed:
	
	And therefore the total proper time in the rocket is:
	
	So compared to the person remained on Earth, the one that was in the rocket has aged about half !!! This is a paradox (rather a "sophism" in reality) because we can not accurately apply Special Relativity to non-inertial frames. Nevertheless, even with General Relativity, there is a time difference!
	
	\subsubsection{Apparent relativistic mass}
	First the reader must take care (!!) the title is misleading by tradition! We will see why a little further below.

	Meanwhile, imagine a frontal collision between two identical objects $(1)$ and $(2)$ having in the repository $R_0$ equal but opposite speeds. We will assume that the collision is elastic, that is to say that the kinetic energy and momentum are conserved.

	Before the shock (collision), the components of objects speeds $(1)$ and $(2)$ are:
	
	as shown below:
	\begin{figure}[H]
		\begin{center}
		\includegraphics[scale=1]{img/cosmology/relativistic_mass_collision_01.jpg}
		\caption{Configuration for the study of the apparent relativistic mass variation seen from $R_0$}
		\end{center}
	\end{figure}
	After the collision, we have:
	
	We will now apply the following Lorentz transformation:
	\begin{itemize}
		\item We give ourselves another repository $R$ and assume that the repositories $R_0$ and $R$ are in uniform translation speed $u_1$ along the $x$-axis in the positive direction (that is to say in the same direction and at the same horizontal speed than the particle $(1)$).
		
		\item For our particle $(1)$ its trajectory became such is present not visible speed anymore along the $x$-axis.
	\end{itemize}
	Let's go! Let us place ourselves in repository $R$ that moves relative to $R_0$ with the speed $u_1$ following the $x$-axis, the components of the speeds in this repository are ten before the collision:
	
	and after the collision:
	
	\begin{figure}[H]
		\begin{center}
		\includegraphics[scale=1]{img/cosmology/relativistic_mass_collision_02.jpg}
		\caption{Configuration for the study of the apparent relativistic mass variation seen from $R$}
		\end{center}
	\end{figure}
	So we have trivially in the reference frame $R$:
	
	but by applying the law of composition of speeds proved earlier above:
	
	for the components of the horizontal axis we always have in the reference frame $R$
	
	and for the vertical movement, we have earlier above that:
	
	Therefore we get:
	
	Passing from $R_0$ to $R$, the component following $y$ of the total momentum must remain zero (as it was the case in $R_0$ initially). But:
	
	To break this deadlock, we must admit that the respective apparent masses $m_1$ and $m_2$ may not be identical in $R$. So that brings us to request that:
	
	which leads us to:
	
	In $R$, the square of the norm of the two objects speeds gives:
	
	The last relation can be written:
	
	so that after rearrangement and factorization we get:
	
	Therefore:
	
	We thus found:
	
	In the case as assumed above where both object are identical we will put $m_1=m_2=m_0$ and therefore:
	
	And we will put them as an apparent relativistic denoted simply by $m$ such that:
	
	And as $V_1^2$ and $U_2^2+V_2^2$ are simply the square norm of the velocity, we can write:
	
	So that finally:
	
	So we see that when $v=0$, we have $m=m_0$ this is why we name $m_0$ the "\NewTerm{rest mass}" or "\NewTerm{invariant mass}\index{invariant mass}".
	
	Since the mass is a function of $v$ (at least in apparence), some physicists note the rest mass as a function, that is to say: $m(0)$. But it is rather more common to use the $m_0$ to not have too much in parentheses in developments...
	
	As the Michelson-Morley factor $\gamma$ tends to infinity when the speed $v$ approaches the speed $c$ of light in a vacuum we have an additional reason to say that $c$ is the upper limit assigned to the speed of any material object otherwise the apparent mass $m$ would be infinite also, which is consistent with both the experience and the consequences already formulated by the Lorentz transformations!!!
	
	It already follows an important conclusion: there are therefore two types of particles, those with a mass and will never go to the speed of light (as it then takes an infinite energy to get them there following our previous result), and those having a zero mass and which will therefore necessarily be at the speed of light.
	
	As we will see it in the section of Quantum Field Theory interaction forces are short-range precisely because of the uncertainty principle and of the above statement. The greater the distance is between large particles that interacts together, the more time will be longer and therefore the small will be the energy involved. But in the case where the particle of interaction have no mass, the "force" is a long range one.
	
	\pagebreak
	\paragraph{Mass–energy equivalence}\mbox{}\\\\\
	Under the action of a force $F$, the speed of a mass $m$ increases or decreases on each portion of the trajectory. The work of the component $F\mathrm{d}x$  can then be interpreted into kinetic energy $\mathrm{d}E_c$ ..

	In the relativistic theory, the mass varies with speed as we have just prove it, therefore:
	
	The integration by parts (\SeeChapter{see section Differential and Integral Calculus}):
	
	give us:
	
	The gain of kinetic energy of a particle can be considered as gain in its apparent mass. Since $m_0$ is the rest mass, the quantity $m_0c^2$ is named "\NewTerm{rest energy}\index{rest energy}" of the particle.

	We then have:
	
	where $E_c$ represents the energy of motion (kinetic energy).

	The sum of:
	
	therefore represents the total energy $E$ of the particle in the absence of the potential field. Which brings us to write:
	
	And therefore:
	
	\begin{figure}[H]
		\begin{center}
		\includegraphics[scale=0.6]{img/cosmology/eintein_trial.jpg}
		\end{center}
	\end{figure}
	Finally we could also have the same result in another way using lagrangien mechanics (\SeeChapter{see section Analytical Mechanics}):
	
	\paragraph{Relativistic Lagrangian}\mbox{}\\\\\
	The following developments will help us in the study of Electrodynamics (if this section has not been read yet), to determine the expression of the tensor of the electromagnetic field and in Relativistic Quantum Physics to determine the Klein-Gordon equation with magnetic field. So be sure to carefully read what follows.

	In special relativity, so we want the equations of motion have the same form in all inertial frames. For this, we need the action $S$ (\SeeChapter{see section Analytical Mechanics}) to be invariant with respect to Lorentz transformations. Guided by this principle, trying to get the action of a free particle. Suppose that the action is in the reference frame O':
	
	\begin{tcolorbox}[title=Remarks,colframe=black,arc=10pt]
	\textbf{R1.} The choice of the minus sign will be evident in our study of electrodynamics.\\
	
	\textbf{R2.} The notation $L_0$ instead of the $L$ of Lagrangian lets just emphasize that this is a case study where the system is free. This distinction of notation will be useful in our study of General Relativity and determination of the Tensor of the Electromagnetic field in the section of Electrodynamics.\\
	
	\textbf{R3.} We are not supposed to know what kind of mass we are dealing (inertial rest mass), hence the fact that in the ignorance, we will work with the inertial mass $m$ to perhaps correct this hypothesis later if necessary.
	\end{tcolorbox}
	And let us recall that:
	
	In the repository O, then we have the "\NewTerm{Lorentz invariant action}\index{Lorentz invariant action}":
	
	So according to our initial hypothesis, we have for the relativistic Lagrangian (in the absence of potential field... since the system is assumed to be "free"):
	
	In the non-relativistic approximation $v\ll c$, we have following the Maclaurin development of the square root (\SeeChapter{see section Sequences and Series}):
	
	We thus fall back on the usual Lagrangian of a free system in movement but more a constant $(-mc^2)$ that does not affect the equations of motion we get in Classical Mechanics but that will be absolutely necessary in to us in Electrodynamics.

	Let us recall now that the generalized momentum (\SeeChapter{see section Analytical Mechanics}) is defined by:
	
	We will now see that this definition is not accidental. Indeed:
	
	The Hamiltonian (\SeeChapter{see section Analytical Mechanics}) is equal to:
	
	Which gives:
	
	The Hamiltonian is in this case equal to the total energy of the particle. Its expression led us to change somewhat our initial hypothesis and finally to write $m_0$ instead of $m$ in the expression of the action $S$.

	So we finally we have:
	
	and:
	
	In the non-relativistic approximation $v\ll c$, $H_0$ becomes with a Maclaurin development (\SeeChapter{see section Sequences And Series}):
	
	We recognize the usual kinetic energy, plus a constant: the energy at rest. Which corresponds to the calculations we had made before where we got:
	
	
	\paragraph{Relativistic (linear) momentum}\mbox{}\\\\\
	The total energy $E$ and the (linear) momentum $p=mv$ of a particle can therefore take any positive value (when the speed approaches the limit value $c$, the apparent mass suits for the product $p=mv$ to not be bounded) .

	In the expression of $E$, we can replace the speed $v^2$ by a function $p^2$:
	
	introduced into:
	
	we have:
	
	Therefore:
	
	hence (we will come back on that relation of the utmost importance during our our proof of Einstein relation):
	
	We have not kept the negative part of the previous relation as it has no meaning in classical physics. However, when we will study Relativistic Quantum Physics, it will be essential to preserve it otherwise we will get absurdities.

	However, we can obviously write this last relation also in the following form named "\NewTerm{relativistic mass momentum relation}\index{relativistic mass momentum relation}":
	
	or also (ugly!):
	
	In other words, the total energy of a moving particle is equal to its mass energy added to its kinetic energy (basically nothing new).
	
	The relation above has two limit cases where we can simplify it:
	\begin{enumerate}
		\item For a particle at rest ($p = 0$), we can reduce the expression to:
		
		by omitting the negative energy ... at least for now.

		\item We can apply the equation to a particle without mass to eliminate the first term, which then gives us:
		
		A photon, for example, has a zero rest mass but it is never at rest ...by definition, it is a quantum of energy, kinetic energy is never zero and so it has a mass corresponding to its kinetic energy. Thus, a massless particle at rest moves at the speed of light, regardless of the chosen repository frame! Conversely, a particle with a non-zero rest mass can never reach the speed of light in any repository.
	\end{enumerate}
	\begin{tcolorbox}[title=Remarks,colframe=black,arc=10pt]
	\textbf{R1.} As we prove it further below (see the "Einstein relation"), from the construction of Planck's law (\SeeChapter{see section Thermodynamics}), we can write $E=pc=hv$.\\
	
	\textbf{R2.} The mass of the photon can hardly be non-zero! Indeed, quantum theory would be false otherwise. But it has never fail until now (\SeeChapter{see section Wave Quantum Physics}). We would also have a small change on th Electrostatic force following that it is given by the Yukawa potential (\SeeChapter{see section Quantum Field Theory}) and this would have been observed in laboratory since...
	\end{tcolorbox}
	Let us now look after the relations between $p$ and $p'$ and between $E$ and $E'$, to make it possible for O' to write:
	
	We then begin to get rid of the square root:
	
	If O write:
	
	O' must be able to write:
	
	Therefore we have:
	
	If we identify:
	
	we obtain similar expressions to those used for the Lorentz transformations of spatial and temporal components. We can then write, by similarity, that the changes to the (linear) momentum and energy are therefore given by:
	
	Again, if we take:
	

	We therefore have by expressing all previous relations of transformation in the same units by remembering that $E\equiv pc$ (for a photon!):
	
	We can the define a matrix such that:
	
	where we fall back on the "Lorentz matrix" or "symmetric Lorentz tensor" ${L'}_\mu^\nu$

	The vector:
	
	is meanwhile, named the "\NewTerm{four-vector energy-momentum}\index{four-vector energy-momentum}" or just "\NewTerm{four-momentu}\index{four-momentum}" Its utility is that its value is also conserved and this is especially useful for the study of nuclear reactions. If we add these vectors on all particles (without forgetting the photons as well!!!) before and after the reaction, we should found the same quantities for the $4$ components!
	\begin{tcolorbox}[title=Remarks,colframe=black,arc=10pt]
	\textbf{R1.} The inverse transformation being done obviously with the inverse matrix that we have already outlined earlier above.\\
	
	\textbf{R2.} We use in Relativistic Optistics the four-vector $(\omega/c,\vec{k})$, where $\omega$ is for recall the pulsation of the wave and $\vec{k}$ the wave vector (\SeeChapter{see sections Wave Mechanics and Wave Optics}). This four-vector is the equivalent for an electromagnetic wave of the four-vector $(E/c,\vec{p})$ for a particle multiplied by the reduced Planck's constant $\hbar=h/2\pi$. Indeed, the wave-particle duality (\SeeChapter{see section Wave Quantum Physics}) attributes to a wave an energy:
	
	and a linear momentum which norm is:
	
	\end{tcolorbox}
	Now let us come back on the following relation is central in some areas of quantum physics:
	
	Therefore:
	
	Which can be written in vector form (very common form):
	
	This latter relation will be very useful in the section of Relativistic Quantum Physics to calculate the energy of virtual photons exchange.

	For photons, since the mass is zero, we have:
	
	Finally let us also notice that the four-momentum is also related to a another quantity name "\NewTerm{four-wave vector}\index{four-wave vector}" as following:
	
	as we know that (for the first component):
	
	and that for all other components that as:
	
	
	\subparagraph{Einstein relation}\mbox{}\\\\\
	Following the principle of relativity, we whish that the relation between the linear momentum and energy of an electromagnetic wave can be written in the same way for two inertial observers in translation relative to the other:

	If O writes:
	
	then O' must be able to write:
	
	Let the take the first relation above and put it to the square without forgetting that the photon has a zero rest mass $m_0$. Therefore:
	
	and as $m_0=0$ for the photon:
	
	Given the known Planck relation (\SeeChapter{see section Thermodynamics}):
	
	we are led to write the famous "\NewTerm{Einstein's relation}\index{Einstein's relation}" that we will find very often in Quantum Physics and in Thermodynamics:
		
	So even if the photo has no mass at rest, it has a "\NewTerm{relativistic mass}\index{relativistic mass}". This relativist mass is in Special Relativity for the photon the equivalent to the electric charge in Electrodynamics.
	
	\subparagraph{Time of flight}\mbox{}\\\\\
	Suppose we get a beam made of massive particles. The rest mass  is $m_0$. The particle travels a distance $L$ in its inertial frame. The particle has an energy $E$ in that frame. Therefore, the so-called "\NewTerm{time of flight}\index{time of flight}” from two points at $x=0$ and $x=L$ will be:
	
 	where we use the relativistic definition of momentum:
	
	Now, knowing that the relativistic energy is:
	
	using the time that a massless light beam uses to travel the proper distance $L$, easily calculated to be:
	
	 we get:
 	
	But we have just proved that:
	
 	Therefore:
 	
	and then:
 	
	Thus, the time of flight is finally written as follows:
 	
	This equation is very important in practical applications. Specially in Astrophysics and baseline beam experiments, like those involving the neutrinos! Indeed, we usually calculate the difference between the photon (or any other massless) time of arrival and that of massive particles, e.g. the neutrinos. Several neutrino experiments can measure this difference using a well designed experimental set-up. The difference between those times of flight (the neutrino time of flight minus the photon time of flight) is:
	
	or equivalently:
 	
	or as well:
 	
	This last expression can also be expressed in terms of the speed of light and neutrinos, since:
	
	so:
	
 	Therefore:
 	
	In the case of know light left-handed neutrinos, the rest mass is tiny (likely sub-[eV] and next to the [meV] scale), and then we can make a Taylor expansion for $Q$ if:
 	
	Then:
	
	Then, we would expect, accordingly to Special Relativity, of course, that:
	
 	We can guess how large it is plugging "typical" values for the neutrino mass and energy. For instance, taking $m_\nu\cong 1$ [meV] and $E\cong 1$ [GeV]  the $Q$ value is about $10^{-24}$.
 
	Nowadays, we have no clock with this precision, so the neutrino mass measurement using this approach is impossible with current technology. However, it is clear that if we could make clocks with that precision, we would measure the neutrino mass with this "time of flight" procedure. It is a challenge. We can not do that in these times (circa 2012), and thus we don't measure any time delay in baseline experiments. Then, neutrinos move with $v_\nu=c$ and since there is no observed delay (beyond the OPERA result, already corrected), neutrinos are, thus, ultra-relativistic particles, and for them $E=pc$ with great accuracy.
	
	\subparagraph{Relativistic force}\mbox{}\\\\\
	Following the principle of relativity, we want that the relation between force and linear momentum to be written in the same way by two inertial observers in translation relative to each other!

	Therefore if O writes:
	
	O' must be able to write:
	
	The relation between $\vec{F}$ and $\vec{F}'$ is quite complicated in the general case. We will limit ourselves here to the particular case where a body is momentarily at rest in O' and therefore where the observer O' will only take into account the force $\vec{F}'$ that he applies. He will name this the "\NewTerm{proper force}\index{proper force}" because it has not to worry about other forces (such as centrifugal force, for example).

	It is necessary to substitute $p'$ and $t'$ by $p$ and $t$ in:
	
	Since:
	
	we will have:
	
	We have seen also previously that:
	
	Therefore it remains:
	
	The component of the force is therefore invariant in the direction of the movement.

	For the directions $y$ and $z$ perpendicular to the movement:
	
	So for summary:
	
	However, to change from one reference frame to another, it is better to use again the "\NewTerm{four-vector force}\index{four-vector force}" defined as the derivative of the four liner moment vector with respect to the proper time:
	
	Indeed, let us recall that:
	
	
	\pagebreak
	\subsubsection{Relativistic electrodynamics}
	With a mass spectrometer we establish that the ratio $m / q$ of the mass $m$ of a particle by its electrical charge $q$ varies in the same way as the mass $m$ when the velocity $v$ of the particle varies:
	
	Thus, it comes that:
	
	The charge of a particle is therefore independent of its velocity, as we have proved in the section of Electromagnetism (\SeeChapter{see section Electrodynamics}) when determining the charge conservation equation.

	Let us consider now two charges $q$ and $Q$ immobile a reference frame O' in translation at speed $v$ with respect to another one centered on O:
	\begin{figure}[H]
		\centering
		\includegraphics[scale=1]{img/cosmology/electric_field_lorenz_transformation.jpg}
		\caption{Configuration for the study of transformations of electric and magnetic fields}
	\end{figure}
	We will restric ourselves to the case where the velocity $\vec{v}$ is paralle to the O$x$-axis:
	
	and we write the vector at the horizontal to spare time and paper...

	The electric charge $Q$ is place on O' and is therefore fixed for O'. The observer O' makes the conclusions that an electrostatic force:
	
	act on the reference particle $q$ placed on $\vec{r}'$:
	
	The observer O also sees an electrostatic field $\vec{E}$ in $\vec{r}$, but he also sees that $Q$ is in movement along the O$x$-axis. It thus deduces the existence of a magnetic field $\vec{B}$ on $\vec{r}$ oriented in the plane $YZ$ plane (\SeeChapter{see section Electrodynamics}):
	
	It therefore measures the supposedly known Lorentz's force (\SeeChapter{see section Magnetostatics}):
	
	But:
	
	Therefore:
	
	We have now seen:
	
	The comparison of the expressions above gives the relativistic transformations of the electric field:
	
	As for the Lorentz transformation of the spatial and temporal components, we have obtained the inverse transformations by exchanging the fields and considering that O' sees O going back away (we therefore replace $v$ by $-v$).

	The above relations, sometimes named "\NewTerm{Joules-Bernoulli equations of the electric field}\index{Joules-Bernoulli equations of the electric field}", make it clear that if, for example, the electric field in one of the reference systems is zero but the magnetic field is not, then an electric field exists from the point of view of the other reference frame !!! It is therefore an absolute victory of relativity in comparison to classical mechanics!

	To obtain the relativistic transformations of the magnetic field, we proceed as follows:
	To get the relativistic transformations of the magnetic field, we proceed as follows:
	
	After some small manipulations of very elementary algebra, we get:
	
	We do the identically:
	
	After a few simple manipulations of very elementary algebra, we get:
	And so on. Finally, we get:
	
	The above relations, sometimes named "\NewTerm{Joules-Bernoulli equations of the magnetic field}\index{Joules-Bernoulli equations of the magnetic field}", make it clear that if, for example, the magnetic field in one of the reference frame is zero but the electric field is not, then a magnetic field exists from the point of view of the other reference fame !!! It is therefore once again an absolute victory of relativity in comparison to classical mechanics!

	Let us now study the behavior of the electromagnetic field of a moving charge:

	Let us consider two parallel referentials O and O', in translation at constant velocity $v$ along the axis $XX'$:
	\begin{figure}[H]
		\centering
		\includegraphics[scale=1]{img/cosmology/lorentz_configuration_study_for_electromagnetic_transformations.jpg}
		\caption{Configuration for the study of electrodynamic transformations}
	\end{figure}
	where a fixed electric charge $Q$ is placed at O '.

	It is clear that the observer O measures $\vec{B}=\vec{0}$ everywhere and that at the point $P$ of the plane $X 'Y'$, on $\vec{r}'=(x',y',0)$ he measures the electrostatic field (\SeeChapter{see section Electrostatic}):
	
	If the observer O is informed of the values of $\vec{E}'$ and of $\vec{B}'=\vec{0}$, he can introduce them into the relativistic transformation giving the electric field $\vec{E}$ that he observes:
	
	To write an expression of the field $\vec{E}$ at the point $P$, the observer O must determine, at a time $t$ of its local time, the components of the vector $\vec{r}=(x,y,0)$ which separates the point $P$ from the electric charge $Q$ (by summing the position vectors of the latter two material points).

	The coordinates of the point $P$ and of the charge $Q$ that he sees in the plane $XYZ$ are given by the usual Lorentz transformations:
	
	He thus easily deduces, by summation, the distances $x$, $y$.

	Another simpler possible method is that since the $x$ component is a length, it therefore undergoes Lorentz transformations and:
	
	Since for recall:
	
	The relativistic transformation of the electric field then gives:
	
	and:
	
	Written in vector form:
	
	We must also determine how to express $r'$ as a function of $r$:
	
	as (Pythagorean theorem):
	
	The writing is simplified if we use the angle formed by the electric field vector and the $x$-axis. We then denote then $\theta'$ in O' and the $\theta$ in O the angles given by:
	
	with $\theta\geq \theta'$ due to the expansion of the lengths along the $x$-axis.

	We eliminate $y$ with:
	
	Thus, the electric field $\vec{E}$ that sees O is given by:
	
	The factor containing $\sin(\theta)$ shows that the electric field $\vec{E}$ of a moving charge no longer has a spherical symmetry!!! It depends on the direction of the vector $\vec{r}$.

	At equal distances, the electric field is more intense in the vertical direction to that of the displacement ($\theta=\pi/2$) than in the direction of the displacement of the electric charge ($\theta=0$).

	If $v = 0$, we fall back on the classic known expression:
	
	\begin{tcolorbox}[title=Remark,colframe=black,arc=10pt]
	Let us recall that we have carried out (and continue in this sense) here a study of an electric charge in uniform rectilinear motion, that is to say at constant speed!
	\end{tcolorbox}
	To find now the expression of the magnetic field $\vec{B}$, we introduce:
	
	and:
	
	in:
	
	We therefore get:
	
	Which are the components of:
	
	To know $\vec{B}$ as a function of $\vec{r}$, we substitute the expression obtained for $\vec{E}$:
	
	In the case where the velocity is small, the relativistic term tends to $1$ and the field $\vec{B}$ of an electric charge $Q$ moving at the velocity $v$ becomes:
	
	because as we have in the section of Electrodynamics: 
	
	\begin{tcolorbox}[title=Remarks,colframe=black,arc=10pt]
	\textbf{R1.} At each location, the lines of the field $\vec{B}$ are contained in a plane perpendicular to the direction of motion of the electric charge $Q$ (vector product oblige...!).\\
	
	\textbf{R2.} If the moving electric charge is seen as a $\mathrm{d}Q$ attached to the point O', we can interpret its displacement at velocity $v$ as a current $I$ at a point of the referential O where O' is located. Therefore:
	
	Therefore:
	
	We then fall back here the "Biot and Savart law" as proved in the section of Electromagnetism. So this a success of the Special Relativity theory again!!!
	\end{tcolorbox}
	It is interesting to remember that an electric charged particle in motion will be seen in the frame of reference of the particle as emitting no electromagnetic field (there will be just an electrostatic field). This is not the case for a repository at rest. There is thus here a sort of flagrant counter-intuitive contradiction.

	But this poses another problem, in a fast-moving frame of reference, a charged particle normally emits an acceleration radiation (\SeeChapter{see section Electrodynamics}), this radiation in quantum mechanics must necessarily be accompanied by the emission of a quanta, which exists either or does not exist (a medium term does not exist). The very existence of photons would therefore be purely relative. And yet it is! Some particles have only a relative existence!!!! The complicated answer is therefore to know what the photons have become.

	But here we reach the limit of what we master perfectly in the physics of the end of the 20th century, because we speak of accelerated references frames (which implies to be in General Gelativity and not the special one) and quantum field theory . The rigorous framework for dealing with this (which would encompass Quantum Gravitation) does not yet exist as far as we know. But a first step has been taken with the development of the Quantum Field Theory in curved space.
	
	\pagebreak
	\paragraph{Tensor field transformation}\mbox{}\\\\\
	We have seen and proved in the section of Electrodynamics that the whole electromagnetic field was summarized by the tensor of the same name. It would then be good to look at how this tensor transforms itself and if it does so correctly in relation to the results obtained above.

	Let us consider the transformation (where the tensor of the electromagnetic field is in natural units !!!):
	
	with the tensor of the electromagnetic field in contravariant components in the Minkowski metric $+---$:
	
	And also by construction:
	
	Let us take, for example, the velocity parallel to the $x$-axis, then we have proved above that:
	
	Therefore:
	
	where as we can see, it is often customary in the field of Special Relativity and Electrodynamics to number the components of matrices / tensors starting from $0$ (instead of $1$ for most of the other chapters of this book) .

	We calculate the transformation (remember that the tensor of the electromagnetic field is antisymmetric!):
	
	We thus deduce, for the electric field (which corresponds perfectly to what we obtained above):
	
	We make a second calculation for the perpendicular component:
	
	hence:
	
	which again corresponds perfectly to what we had obtained earlier above (in natural units, do not forget that we then have $\beta=v$)!

	The same applies to the magnetic field:
	
	and:
	
	Which gives (in natural units, again do not forget that we then have $\beta=v$)! :
	
	etc.
	
	\subsection{Minkowski space-time}
		We have proved earlier above that:
	
	Let us write this in the form:
	
	Let us multiply the two members by $(ct)^2$:
	
	which gives us:
	
	If $v=c$ the equation vanishes:
	
	This result translates the fact that the dimensions of space and time are as stopped in the relativistic referential, because the relative speed of the object is equal to that of the light!

	Let us now imagine that a light beam is emitted at the instant $t=0$ and propagates from the origin of a referential. We know that in space-time (application of the Pythagoras theorem in three-dimensional Euclidean space for recall...) the distance traveled by the photon is:
	
	By changing $t$ of member and bringing the whole to the square to remove the root, we get:
	
	Therefore:
	
	\begin{tcolorbox}[title=Remark,colframe=black,arc=10pt]
	We can assimilate this relation to the representation of a spherical wavefront of a light wave propagating at the speed of light (see the equation of a sphere originally centered in the section of Analytical Geometry) .
	\end{tcolorbox}
	Let us now consider two coordinate events $(x_1,y_1,z_1,t_1)$ and $(x_2,y_2,z_2,t_2)$ and we denote by $\mathrm{d}s$ the "\NewTerm{space-time abscissa}\index{space-time abscissa}" (or "\NewTerm{proper-distance}\index{proper-distance}"). We can then write the spatio-temporal interval as such:
	
	By passing to the limit, we get the quadratic form:
	
	which has the same shape and value regardless of the reference system considered as we have already proved it earlier above. The infinitesimal interval of space-time $\mathrm{d}s^2$ between two infinitely neighboring events is therefore a relativistic invariant that we often name the "\NewTerm{space-time curvilinear abscissa}\index{space-time curvilinear abscissa}". It is the interval of space-time or, as Albert Einstein simply said, the "square of distance" .... The fact that this magnitude may be positive, negative (!) or zero is linked to the absolute character of the speed of light (we will come back to this later).

	We can also now turn our attention to the relativistic character of this metric. If it is invariant, it must also be invariant by Lorentz transformations. We then say that "the metric is invariant by Lorentz transformation". Such a transformation can be found on the basis of that used for the tensor of the electromagnetic field (see above). The reader will readily verify from the detailed example of the electromagnetic field that for the metric tensor we have the relation (as always we can detail on request if necessary):
	
	The curvilinear abscissa can also be expressed by the norm of the quadrivector displacement which we defined above as $(ct,x,y,z)$. Indeed, the norm (\SeeChapter{see section Tensor Calculus}) is written by taking down the indices using the "\NewTerm{Minkowski metric}\index{Minkowski metric}" $\eta_{\mu\nu}$ or "\NewTerm{pseudo-Riemannian metric}\index{Minkowski metric}":
	
	with the definition of the "\NewTerm{Minkowski's matrix}\index{Minkowski's matrix}" (we will return to this in detail at the beginning of our study of General Relativity):
	
	where as usual in this book we make the abuse of notation (already mentioned in the section of Tensorial Calculation) not to put $\eta_{\mu\nu}$ in brackets (since a tensor and its matrix form are normally two distinct things in rigorously speaking).

	If we put the following two relations in correspondence:
	
	we then have $\mathrm{d}s^2=0$ when the two events are connected to the speed of light.

	Moreover, if we put:
	
	we can then write:
	
	This is nothing more than the equation of a cone (\SeeChapter{see section of Analytical Geometry}) of axis of ordinate $c^2\mathrm{d}t^2$... the famous "\NewTerm{light cone of Universe}\index{light cone of Universe}" (to which we devote a study further below). Every event is therefore by extension in this cone and the evolution of any system can thus be described (by its spatial and temporal position), by what we name its "\NewTerm{line of Universe}\index{line of Universe}" or "\NewTerm{World line}\index{World line}". The Universe line of a particle is therefore the sequence of events it unfolds during its lifetime.
	
	\subsubsection{Four-vectors}
	We have just defined what Minkowski's metric was, we can now correctly define the concept of quadrivector that we have already addressed without always knowing what we were doing.
	
	\textbf{Definition (\#\mydef):} In a four-dimensional space of Minkowski type, the four quantities:
	
	(regardless of the order of terms for this definition or whether the indices are numbers or letters corresponding to the four spatio-temporal components) form a covariant "\NewTerm{four-vector}\index{four-vector}\index{quadrivector}" if they transform following the Lorentz transformation:	
	
	The "\NewTerm{pseudo-norm}\index{quadrivector pseudo-norm}\index{four-vector pseudo-norm}" of a quadrivector in a Minkowski space of metric $\eta_{\mu\nu}$ is then:
	
	where we see that the contravariant four-vector multiplied by the metric returns the contravariant four-vector (\SeeChapter{see section Tensor Calculus}).

	The following quantity being invariant by change of Galilean referential as we proved it almost at the beginning of this section:
	
	This property of invariance by change of Galilean referential of the four-vector is their main property. Thus, two observers in relative motion, which are uniform in relation to each other, must compare the results of the same measure using the norm of the four-vectors. Similarly, the laws they seek to determine to be as general as possible must use these invariant quantities! 
	
	We can also write the norm of a four-vector in the form:
	
	and the four-vector themselves:
	
	So for summary let us give the list of the four-vectors we have determined so far in this section but by standardizing the notations (and only those four-vectors that will be useful for other sections of this book!):
	\begin{itemize}
		\item The space-times four-vector ("four-position"):
		

		\item The velocity four-vector ("four-velocity"):
		
		
		\item The current four-vector ("four-current"):
		
		
		\item The acceleration four-vector ("four-acceleration"):
		
		
		\item The energy-momentum four-vector ("four-momentum"):
		

		\item The gradient four-vector ("four-gradient") introduced in the section of Tensor Calculus:
		
	\end{itemize}
	Obviously we have considered here four-vectors in the context of Special Relativity. Although the concept of four-vectors also extends to General Relativity, some of the results stated above require modification in General Relativity.
	
	\subsubsection{Universe light cone}
	The topology of the light cone has its origin in the relations of anteriority and posteriority of relativistic events, which makes it possible to distinguish between an event in the past of another or in the future of it.

	The principal objective of the light cones in the popularization works of theoretical physics is to map out the history of light pulses emitted at a point in the space where certain conditions may prevail. The points are represented in space by a series of snapshots at various times $t_1$, $t_2$, $t_3$, etc. (see figure below), the spherical wave front of the light magnifying in space. In space-time, the same event (at the bottom on the figure) is represented by a "\NewTerm{light cone}\index{light cone}", whose apex is the point of emission.

	On a sheet of paper, we have to remove one of the spatial dimensions. The spatial axes are drawn in the horizontal plane and the time axis directed upwards. The cone sections at the instants $t_1$, $t_2$, $t_3$ correspond to the snapshots of the spatial representation: the two-dimensional wavefronts are circles whose radius is that of the spherical wavefront at the instant considered. The light cone shows in a single diagram the continuous history of the wavefront of a light signal.
	\begin{figure}[H]
		\centering
		\includegraphics{img/cosmology/light_cone.jpg}
		\caption{Idea of light-cone}	
	\end{figure}
	More precisely, the "snapshots" mentioned above are named "\NewTerm{punctual events}" and these appear instantaneous (approximation based on geometric optics) to any observer capable seeing them. A collision between two point particles provides an example of a punctual event. It is quite possible that a non-punctual instantaneous event appears instantaneous to a certain observer but, because of the finite propagation velocity of the light, not instantaneous to another observer.

	\textbf{Definitions (\#\mydef):}
	\begin{itemize}
		\item[D1.] Two punctual events occupy the same "\NewTerm{time-space point}\index{time-space point}" if they appear simultaneously to any observer able to see them.

		\item[D2.] The set $M$ of all points of space-time is named the "\NewTerm{space-time}\index{space-time}".

		\item[D3.] The boundary defined by the Universe cone is named the "\NewTerm{cosmological horizon}\index{cosmological horizon}"
	\end{itemize}
	Let us recall that if no force acts on a point particle, we say from it that is is an "inertial" or "free" particle. We also say that it is in "inertial motion".

	Given the point $p$, $N(p)$ is an absolute geometric structure independent of the observer. Its future component will be denoted $N^{+}(p)$; Its past component $N^{-}(p)$ and it will be represented by the following cone:
	\begin{figure}[H]
		\centering
		\includegraphics{img/cosmology/past_future_light_cone.jpg}
		\caption{Past and future light cones}	
	\end{figure}
	Indeed, let us recall that the Minkowski equation is invariant since:
	
	We have, when we reduced to three parameters (we remove a spatial dimension to simplify the conceptualisation), if the punctual events are related to the speed of light (see earlier above):
	
	What we can also write in the form:
	
	to be compared with the equation of a cone (\SeeChapter{see section Analytical Geometry}):
	
	when we put $c = 1$ (which is frequent in theoretical physics as we have already mentioned many times).

	Therefore the Minkowski equation can be indeed presented by a cone.
	\begin{tcolorbox}[title=Remark,colframe=black,arc=10pt]
	If we would have keep the three spatial parameters and the time interval constant, the reader will then perhaps have notice that we would fall bacl neither on the equation of a cone but on that of a sphere. It is the "\NewTerm{celestial sphere}\index{celestial sphere}" where at a given instant, on its surface, multiple cones of light are created.
	\end{tcolorbox}
	The universe line of any observer which occupies instantly $p$ and whose line of the Universe passes through $p$ itself, is contained within $N(p)$ defined by a single point on its celestial sphere (the one that is described by the information vector - the photon - in all directions of space). This means that there can be, in extenso, as many null rays (foci of cones) passing through $p$ as points on a sphere.

	The following example will (we hope) appear more obvious:
	\begin{figure}[H]
		\centering
		\includegraphics{img/cosmology/light_cone_associated_universe_line.jpg}
		\caption{Universe Line Principle with its associated Cone}	
	\end{figure}
	As illustrated in the figure above, a light event at the point O of the space-time produces a beam of photons, all in the zero cone of the future O, $N^{+}(\text{O})$ (these photons have been emitted by atoms in various states of movements whose universe lines $l$ and $l'$ pass through O, but are entirely contained within the $N^{+}(\text{O})$). The universe line $n$ can only be described by a particle moving at the speed of light because it defines the boundary of the cone (we then say that the line of the Universe is of "light type").
	
	\begin{tcolorbox}[title=Remark,colframe=black,arc=10pt]
	The representation of the Universe lines in the lower part (inverted cone) comes from the fact that an event can also have a past... so the scheme generalizes the particular example.
	\end{tcolorbox}	
	Let $l_p$ be the universe line of a stationary person $P$ (hence the verticality of its Universe line in the figure above) and $n$ that of a light ray having the origin O. Both lie in the four dimension space and they intersect at a single point $P$. The points O and $P$ lie on a zero radius (of a cone of the future), $n$, of $N^{+}(\text{O})$. In $P$, the person $P$ sees a sudden flash in the direction defined by $n$, for him the direction of the luminous event (described only by its velocity therefore, so a universe line of an inertial particle can be described only by time and speed).
	
	An atom whose Universe line cuts $n$ at the point $Q$ absorbs a photon of the luminous event O and re-emits a beam of photons shortly after. These, in turn, form zero rays in $N^{+}(Q)$, but only those of direction $n$ will reach the person $P$ and will be seen by him at the point $P$.

	If $P$ is inside $N(\text{O})$, the zero cone of O, we will say that its universe line is of the "time type". In this case, O and $P$ are located on the universe line of an observer or a massive particle. There are, of course, two types of time displacement:
	\begin{enumerate}
		\item If $P$ is in the future of O (according to an observer whose universe line passes through O and $P$), we will say that $P$ "points to the future".

		\item If not, we will of course say that it "points to the past".
	\end{enumerate}
	If $P$ is on $N(\text{O})$ - that is to say on the surface of the cone - then we say that it is "null" or of "light type" and if $P$ is neither zero nor of time, then $P$ is outside of $N(\text{O})$ and the we say that it is of "space type":
	\begin{figure}[H]
		\centering
		\includegraphics{img/cosmology/type_of_universe_lines.jpg}
		\caption{Types of universe lines}	
	\end{figure}
	This is mathematically translated by remembering (see above) that:
	
	\begin{itemize}
		\item $\mathrm{d}s^2=0$ (then $r^2=c^2\Delta^2$): The universe line is therefore "light-like" and it is that latter which describes the surface of the cone by definition (according to what we have demonstrated previously and whatever the choice of the metric) is such that:
		
		which is the case of a photon (hence the name ...). In other words, the spatial separation is equal to the distance light travels.

		\item $\mathrm{d}s^2<0$ (then $r^2<c^2\Delta^2$: We then say that the Universe line is "space-like", therefore such that:
		
		Two events that take place simultaneously but at different places are therefore space-like. In other words, the spatial separation is less than the distance light travels.

		\item $\mathrm{d}s^2>0$ (then $r^2>c^2\Delta^2$: We then say that the Universe line is "time-like", therefore such that:
		
		In other words the spatial separation is greater than the distance light travels.
		\begin{figure}[H]
			\centering
			\includegraphics{img/cosmology/universe_line_type_with_equation.jpg}
		\end{figure}
		
		\item A "\NewTerm{causal line}" is a time-like or light-like line that is always oriented towards the future.
	\end{itemize}
	Let us return to our equations after this small interlude ... the equations therefore lead us to several observations. Thus, in the four-dimensional Euclidean Universe of Minkowski, the trajectories of objects in space-time are always straight lines. Indeed, the trivial example consists in considering that the object remains at rest, then only the time then continues to flow. We have therefore:
	
	by putting $v=0$, this gives us:
	
	therefore:
	
	hence:
	
	and also:
	
	The primitive being (integration constant taken as zero):
	
	which is indeed a straight line and therefore represents the universe line of the object considered in the universe cone. We can also observe that in this case, the evolution of the phenomenon is purely temporal when the interval is positive (which supports what we said earlier).
	\begin{tcolorbox}[title=Remarks,colframe=black,arc=10pt]
	\textbf{R1.} If the speed of light is infinite, we fall back on the particular case of the Newtonian universe, where a phenomenon can instantly occur. Time is absolute and there is no cosmological horizon because the cone has a maximum aperture (right angle).\\
	
	\textbf{R2.} If we put that the velocity of light as equal to $1$ (natural units), as we have done it already sometimes, the axis of the ordinate of the cone is named a "purely temporal axis".\\
	
	\textbf{R3.} It is necessary to understand that our Universe has its own cone of Universe (cone... if the space is of Minkowsky-like of course...).
	\end{tcolorbox}
	Finally, let us say that the theory of Special Relativity, like that of General Relativity, does not impose a given number of spatial dimensions in order to remain consistent: this is a pity for theoretical physicists who would like a theory which, imposes on itself a finite number of dimensions to remain consistent (that on the other hand the theory of the strings or superstring).
	
	\begin{flushright}
	\begin{tabular}{l c}
	\circled{95} & \pbox{20cm}{\score{4}{5} \\ {\tiny 48 votes,  71.25\%}} 
	\end{tabular} 
	\end{flushright}
		
	%to force start on odd page
	\newpage
	\thispagestyle{empty}
	\mbox{}
	\section{General Relativity}
	\lettrine[lines=4]{\color{BrickRed}A}s we saw it, in the previous section, Special Relativity is a remarkable achievement from a theoretical point of view as well as a practical point of view, forming a continuum of space-time where the space variables and time are given the same physical dimension (that of a distance metric for reminder!). However, this applies only to the Euclidean frames and to inertial/Galileans reference frames(constant speed reminder ...). It is therefore appropriate to first generalize the entire mechanic theory by expressing its principles and fundamental results in a generalized form independent of the type of coordinate system chosen (that is to say: independent of the space properties) using for this purpose tensor calculus and then to take into account the non-inertial systems. The equivalence of inertial systems by Special Relativity and the non-equivalence of inertial systems can then shortly be resume (a little bit basically...) saying that speed is relative but the acceleration is absolute. Thus, we can never rest distinguish a uniform motion, but we can distinguish them from an accelerated motion.
	
	It should also be consider the fact that Special Relativity applies only to Galileans frames is restrictive because any mass creates a gravitational field whose scope is endless. To find a true Galilean frame, it is therefore necessary to lie infinitely far from any mass. Relativistic mechanics built from Special Relativity therefore constitutes an approximation of the laws of nature, where the gravitational fields or accelerations are low enough. This application limitation is not  more suited to relativistic astrophysics whose activity has intensified in the late 20th century.
	
	\subsection{Assumptions and Principles}
	Albert Einstein and some others of his time believed in a physics that not to favor any frame system since that was in their eyes the reality of the Universe (we have already mentioned this point of view). But how can we subtract ourselves to the phenomenon of  acceleration? The brilliant idea was to state the "\NewTerm{equivalence postulate}\index{equivalence postulate}" below (which still in this early 21st century has not show any default by recent known experiences) plus the "invariance postulate" and "cosmological principle" we have already stated in the section of Special Relativity and the assumption that the motion of a particle that does not undergo any other interaction that gravitation follows a geodesic line (see below for the detailed proof).
	
	\subsubsection{Equivalence Postulates}
	At first, Albert Einstein will improve the equivalence postulate (also named "equivalence principle") whose older versions are due to Galileo and Newton:
	
	\textbf{Postulate:} The (uniform!) acceleration of a mass (outside gravitational field) due to application of mechanical force and the acceleration of that mass subjected to a gravitational field are supposed completely equivalent. Thus, the results of mathematical analysis in one case may apply to the other (here this is already smart but consistent ... the idea is very good still had to have it...!). 
	\begin{figure}[H]
		\begin{center}
		\includegraphics{img/cosmology/equivalence_principle.jpg}
		\end{center}	
	\end{figure}
	In other words, the gravity field has a fundamental property which distinguishes it from all other fields known in nature: the free fall movement of bodies is universal, independent of the mass and composition of the bodies.
	
	Corollary: The rest mass of a body must be the same whether it is measured in a frame within a gravitational field or outside a gravitational field (we speak then about "inertial mass" and of "gravitational mass" as we have already study at the beginning of our study in the section of Classical Mechanics).
	
	\begin{tcolorbox}[title=Remark,colframe=black,arc=10pt]
	We must be careful and check that the corollary of the assumption of the equivalence principle is true otherwise all General Relativity would collapse (in the early 21st century experiments are still going to try to show a default to this principle)!
	\end{tcolorbox}	
	
	So all static and uniform gravitational field is equivalent to an accelerated frame in vacuum. We can consider any physical gravitational field as static and uniform in a relatively small region of space, and for a relatively short period of time to avoid the tides effects. We are thus led to state the "\NewTerm{Weak Equivalence Principle WEF}\index{weak equivalence principle}": For any event in space-time in an arbitrary gravitational field, we can choose a frame named "\NewTerm{locally inertial frame}\index{locally inertial frame}" such as in the neighborhood of the event of interest the free movement of all body (which are also in the gravity field!) is straight and uniform as we are able to apply the Lorentz transformations (\SeeChapter{see section Special Relativity}). In other words, it is not possible to distinguish a system in vacuum space far away from any star (gravitational source) from as system falling in a constant homogeneous gravitational field.
	
	If we experimentally show that WEF fails, then we are put into default the equivalence principle itself ... which has never been achieved in the laboratory to this date!
	
	\begin{tcolorbox}[title=Remark,colframe=black,arc=10pt]
	The concept of "locality" is very important because it reality we don't know any natural uniform gravitational field. For example, on Earth, two body distant of a certain length dropped from a certain height will fall to the ground with a shorter distance than the distance between them when they were released. This is what we call in physics the "tide effect": the gravitational field is never uniform (as far as we know...).
	\end{tcolorbox}
	
	So the postulate of equivalence (which includes the principle of weak equivalence) finally asserts that the Newton's force on inertial mass $m_i$:
	
	and that of gravitation in the form of the Newton-Poisson law (\SeeChapter{see section Astronomy}) with gravitational mass $m_g$:
	
	are equivalent such as the inertial mass equals the gravitational mass and acceleration equal to gravity and that it is not possible to distinguish the both such that:
	
	In what this postulate allows to resolve all the problems therefore? It's simple! The idea is the following:
	
	When we will consider a body in acceleration, we first always equate it to the acceleration due to the fall in a gravitational field (by applying the postulate of equivalence). Then, we will assume, and will have to check (see proof further below) by rediscovering Newton's law, that acceleration due to the gravitational field is not due to the field itself but to the geometry of the deformed space by the presence of the mass (i.e. energy) that creates the gravitational field. Thus, the object is no longer in "free fall" but will be seen as sliding on the distorted spatial frame to acquire therefore its acceleration.
	\begin{enumerate}
		\item If the tensor calculus gives the possibility to express the laws of classical and relativistic mechanics in any coordinate system, it is then possible to see how the coordinate system (metric) acts on the expression of the laws of the Universe (Albert Einstein did not know that fact as he had not completed its calculations but had a presentiment about this)!
		\item If the natural tensor expression of the laws of mechanics shows slippage (i.e. acceleration) on the spatial frame following the (local) considered metric, then the bet is won and then the acceleration can be seen as an effect whose cause is purely geometrical.
	\end{enumerate}
	
	\begin{tcolorbox}[colframe=black,colback=white,sharp corners]
\textbf{{\Large \ding{45}}Example:}\\\\
		Suppose that two rockets, which we denoted by $A$ and $B$ are in a region of space away from any body. Their engines are stopped which physically results in a uniform motion. In each rocket, physicists are making mechanics experiments with objects which they know the inert mass. Suddenly the engine of the rocket $A$ starts and communicates to it an acceleration whose effects felt inside the spaceship is an inertia force that constraint objects going to the floor. For physicists $A$ rocket laws of mechanics are then the same as that observed in a gravitational field. They are logically led to interpret the force of inertia as the manifestation of a gravitational field. Using a balance, they can weigh their objects and assign to them a gravitational mass.\\
		
		Suppose now that the physicists in rocket $B$ could observe what happens in the rocket $A$. They know what their colleagues interpret as the weight of objects is in fact a force of inertia. The inertial force is proportional to the acceleration and the inertial mass. If the gravitational mass was different from the inertial mass the physicists of the rocket $A$ could distinguish the effects of inertial forces from those of a gravitational field because the measured masses are distinct. We know that the inertial and gravitational mass are equivalent (Galilean principle of equivalence). It follows that the physicists of the rocket $A$ have no way to differentiate between inertial forces resulting from an accelerated motion of their spaceship and the gravitational attraction forces.\\
		
		However, we must temper the conclusions from this experience: the real gravitational fields differ from an accelerated frame since the gravitational acceleration varies with the distance to the main body while in an accelerated reference frame, the acceleration is the same at any point in space. However, \underline{locally}, a gravitational field and an accelerated frame can not be differentiated!!!
	\end{tcolorbox}
	
	We are led now to state the "\NewTerm{Einstein's equivalence principle}\index{Einstein's equivalence principle}" (EPE) as did Albert Einstein: locally all the laws of physics are the same in a gravitational field and a uniformly accelerated frame.
	
	This has a consequence: If the mass (which is equivalent to the energy as we have proved in the section of Special Relativity) of an object is not differentiable that we are in a gravitational field or a uniformly accelerated frame that means that all types of energy (nuclear cohesion energy, electrostatic energy, proper gravitational energy of the object, etc.) of this object are indistinguishable. So the laws of Special Relativity are also valid whatever the considered frame!
	
	If the laws are not the same, then EPE (Einstein equivalence principle) is faulted, so verbatim WEF (weak equivalence principle) also and more globally the principle of equivalence in general but this has never happened experimentally as far as we know at this day.

	\begin{tcolorbox}[colframe=black,colback=white,sharp corners]
	\textbf{{\Large \ding{45}}Example:}\\\\
	By the WEF, it is interesting to note that the gravitational field also acts on the gravitational potential energy of the other bodies. We say then that the gravitational field is a "coupled field".
	\end{tcolorbox}
	
	Given that in General Relativity, the gravitational field is supposed to be described by the metric $g_{\mu\nu}$ (from which the 4-dimensional differentiable manifold that is space-time is supposed to be made), we can see a locally inertial frame as a coordinate system of spacetime in which the metric becomes flat (pseudo-Riemannian):
	
	using the notation introduce in the section of Tensor Calculus.
	
	Such a coordinate system will by hypothesis always exists, indicating the existence, for any gravitational field, of locally inertial frames!
	
	\subsubsection{Mach Principle}
	
	If the equivalence principle highlights the equality of inert and gravitational mass, it does not enlighten us about the nature of these two masses. Finally, what are the inert and gravitational mass?
	
	The deep nature of the inert mass should inform us about the inertia itself. The inertia is manifested in a passive form - the principle of inertia - and an active form - the second Newton's law. In general, it expresses a universal behavior of bodies to resist to the change of movement. But we know that inertial motion is relative, that is to say that there is no absolute referential frame. Is it the same with the accelerated movement? Consider, to illustrate this question, a rocket in which has taken place a physicist and let us carry two experiments:
	
	
	However when the metric is not flat the coordinates are named "\NewTerm{Riemann normal coordinates}\index{Riemann normal coordinates}" and then describes a Riemann metric space (curved space) and itself depends in a nontrivial way of the coordinates system (see sections Tensor Calculus and Non-euclidean Geometry).
	
	\begin{enumerate}
		\item First experience: The rocket accelerates and the physicist is subjected to inertia force oriented in the direction opposite to that of acceleration.
		
		\item Second experience. Now assume that we gives to the whole Universe - at the exception of the rocket that moves in an inertial motion - an acceleration exactly opposite to the one of the rocket of the preceding experiment.
	\end{enumerate}
	If the accelerated motion is relative then, for an observer, it is not possible to distinguish the two experiments. In particular, the physicist located inside the rocket must observe the emergence of an inertial force absolutely identical to the one he noted in the first experiment. The inert mass could then has its origin in the interactions of the gravitational mass of bodies with all the gravitational mass of the Universe! It is as if by moving all masses of the Universe, they dragged with them the objects in the rocket, the physicist therefore experienced a force that pulls in the same direction as the acceleration applied to stars.
	
	Following Ernst Mach, physicist and philosopher of the 19th century, the movement whatsoever inertial or accelerated, is relative.
	
	This theory was named by Albert Einstein "\NewTerm{Mach principle}\index{Mach principle}". At this date, Mach's principle has not been confirmed, but no more rejected. It is true that its experimental verification far exceeds actual human capacities!
	
	\subsection{Metrics}
	Albert Einstein assumed that gravity was only the manifestation of space-time distortions. To try to illustrate in the more possible simple and illustrated way the idea of Albert Einstein, consider a rolling gear at constant speed (say, one tooth at a second) on a rack. Imagine that we have the power to simultaneously change the pitch of the rack and the wheel when and where we wish. Let us do things such that the pitch of the rack slightly increases from one tooth to another. For fixed observers the gear is then driven with a uniformly accelerated motion as, in effect, at each turn thereof always travels a greater distance. On the other hand, if one chooses the rack as a reference and thereof the pitch as a standard to measure the movement of the wheel is then uniform (one tooth per second). The acceleration of the wheel is the consequence of the increase in the pitch of the rack.
	
	Let us continue the analogy: the pitch of the rack acts as a local measurement standard in our one-dimensional space that represents the rack. In geometry, it is named the "metric". The metric is what determines the distance between two points, it is somehow the standard infinitesimal space unit. In Euclidean geometry, the metric is constant, allowing us to create universal measurement standards. Bernhard Riemann, for example, invented a metric geometry which can vary from one point to another in space, which allowed him to describe curved spaces like the surface of a sphere, for example (\SeeChapter{see section Non-Euclidean Geometries}).
	
	During our study of tensor calculus, non-Euclidean geometries and differential geometry (section that the reading is more than recommended!!!) we have seen that the measurement of the curvilinear distance $ds$ between two points positioned in a two or three dimensions space can be made using a large number of coordinate system by the "\NewTerm{metric equation}\index{metric equation}" (\SeeChapter{see section Tensor Calculus}):
	
	In General Relativity, the idea is to make the theoretical model independent of the background and thus build it in a covariant form (which some physicists liken to assimilate to a postulate named the "\NewTerm{covariance principle}\index{covariance principle}"). An excellent candidate for this type of approach is to use the tensor formalism. This is the reason why the metric equation is therefore one of the pillars.
	
	\begin{tcolorbox}[colframe=black,colback=white,sharp corners]
	\textbf{{\Large \ding{45}}Examples:}\\\\	
	E1. Rectangular coordinates (in $\mathbb{R}^3$):
	
	If the squared distance satisfies this relation then we are in a flat space or at least locally flat (\SeeChapter{see section Non-Euclidean Geometries}).\\
	
	E2. Polar coordinates (in $\mathbb{R}^2$):
	
	Therefore:
	
	Therefore:
	
	If the squared distance satisfies this relation then we are in a flat space or at least locally flat (\SeeChapter{see section Non-Euclidean Geometries}).\\
	
	E3. Cylindrical coordinates (in $\mathbb{R}^3$):
	
	when we put this into $\mathrm{d}s^2=\mathrm{d}x^2+\mathrm{d}y^2+\mathrm{d}z^2$ we get in a similar way as before:
	
	If the squared distance satisfies this relation then we are in a curved space (cylindrical type) but that may may be locally flat (\SeeChapter{see section Non-Euclidean geometries}). In fact, to have the metric of the cylinder surface and not simply of the plane expressed in cylindrical coordinates, we must take the following metric:
	
	whose origin was proved in the section of Differential Geometry and also... just previously...
	\end{tcolorbox}
	
	\pagebreak
	\begin{tcolorbox}[colframe=black,colback=white,sharp corners]	
	E4. Spherical coordinates (in $\mathbb{R}^3$) for which we have:
	
	when we put this into $\mathrm{d}s^2=\mathrm{d}x^2+\mathrm{d}y^2+\mathrm{d}z^2$ we get:
	
	Now remember that (\SeeChapter{see section Algebra Calculus}):
	
	Therefore:
	
	After a first set of factorization and basic simplifications of identical terms, we obtain:
	
	If the squared  distance satisfies this relation then we are in a curved space (spherical type) but that locally may be flat (\SeeChapter{see section Non-Euclidean Geometries}). In fact, for the metric of the surface of the sphere and not simply of the plane expressed in spherical coordinates, we have to take the following metric:
	
	whose origin has been proved in the section of Differential Geometry. We also checked in the section of Tensor Calculus, that the Ricci curvature of the spherical prior-previous metric was zero. By cons, we had right after checked that if we took the previous metric of the surface of the sphere, the Ricci curvature was not zero (and it is still happy!).
	\end{tcolorbox}
	Until then, you may be wondering where we are going? In fact, we try to define from these relations, a mathematical being that consistent with the Einstein's hypothesis, expresses the geometric properties of given space.
	
	How we will do this?: We first change simply change the notations. Instead of using the symbols $(x,y,z,\theta,\phi,r)$ we will write $x^1,x^2,x^3,...$. Caution! The numbers suffixes are not powers!!! These are dummy values that are only there to symbolize the $x$-th coordinate of a given basis.
	
	Now let us write again our metric equations with this new notation by considering it is only specific examples that do not necessarily have relevant physical sense (we also mentioned it earlier!):
	\begin{itemize}
		\item Rectangular coordinates:
		
		\item Polar coordinates:
		
		\item Cylindrical coordinates:
		
		\item Spherical coordinates:
		
	\end{itemize}
	Now let us recall again that the "\NewTerm{metric tensor}\index{metric tensor}" (so named because it calibrates space-time) noted (\SeeChapter{see section Tensor Calculus}):
	
	is involved in the metric equation as follows in the Lorentz invariant:
	
	and notice that the components of the matrix are also dimensionless!
	
	This mathematical entity which is a tensor thus contains the parameters of the curvature (we also sometimes say of the "stress" or "tension") wherein a space is located. But then what contains the metric tensor of space-time for a flat Euclidean space?
	
	According to the summing writing Einstein's convention (\SeeChapter{see section of Tensor Calculus}), for example, for $\mu=\nu=2$ we have:
	
	So if we return to our tensor for the flat Euclidean space, we already know (\SeeChapter{see section Tensor Calculus}) that $m$ and $n$ goes from $1$ to $3$ and we have in our tensor $g_{\mu\nu}=0$ for $\mu\neq \nu$ and $g_{\mu\nu}=1$ for  $\mu= \nu$  (symmetrical tensor ). So:
	
	Therefore:
	
	which as usual in this book we make usage of the abusive notation (already indicated in the section of Tensor Calculus) to not put $g_{\mu\nu}$ (since a tensor and its matrix form are normally two separate things strictly speaking).

	This result is remarkable, because the metric tensor will therefore enable us to define the properties of a space from a simple mathematical being that can easily be handled formally as we already seen in the section of Tensor Calculus, Non-Euclidean Geometry and Differential Geometry.

	In polar coordinates the tensor $g_{\mu\nu}$ is:
	
	Check:
	
	And in cylindrical coordinates the tensor $g_{\mu\nu}$ is written:
	
	We will not do the check as the result is obvious (excepted on reader request).
	
	In spherical coordinates the tensor $g_{\mu\nu}$ is a little more complex and is written:
	
	We will also not do the check as the result is obvious (excepted as always on reader request). 
	\begin{tcolorbox}[title=Remark,colframe=black,arc=10pt]
	As we have mention it, in the section of Tensor Calculus, $g^{ij}=(g_{ij})^{-1}$ and the reader can quickly verify this with Maple 4.00b as inverting a matrix is always a boring work (here the code is given only for the spherical one but the idea is the same for the others):\\
	
	\texttt{>with(linalg):\\
	>A:=array([[1,0,0],[0,r\string^2,0],[0,0,r\string^2*sin(theta)\string^2]]);\\
	>inverse(A);}
	\end{tcolorbox}
	
	In Special Relativity, we have seen that the notions of space and time were implicitly bounded. Thus, to study modern physics (this does not really interest  the pure mathematician), we need to add to our metric tensor a time component  to get what we name the "\NewTerm{space-time metric tensor}\index{space-time metric tensor}".
	
	To determine the writing of this tensor, we will place us at first in a Minkowski space where we have for recall (\SeeChapter{see section Special Relativity}):
	
	which is the infinitesimal interval of space-time between two infinitely close events (or considered as it at a given scale ...).

	Thus by putting:
	
	We have:
	
	with the "\NewTerm{signature}\index{signature}":
	
	\begin{tcolorbox}[title=Remark,colframe=black,arc=10pt]
	For all metric tensor that we have determined before, if we express them in space-time (thus adding time component), the spatial components will all have a negative sign!
	\end{tcolorbox}
	We will see later other metrics that are much less intuitive once we will have proved far later below the Einstein's fields equation.
	
	\subsubsection{Schild Criteria (Einstein red-shift effect Newtonian approach)}
	We will prove later that gravitation as formulated in Newtonian mechanics is completely describable by a curvature formulation of space-time. But first we want to introduce to the reader what in our point of view the easiest development that can be done without General Relativity to compare it also later with the easiest development that can be done with General Relativity: the gravitational red-shift effect!
	
	Imagine first a very height tower of height $h$ built on the surface of the Earth. A man sits at the ground of the tower, and sends a signal of pulsation $\omega_A$ to a colleague $B$ at the top of the tower. There will be, and we will immediately prove it, that the pulsation $\omega_B$ of the wave received by $B$ differs of $\omega_A$ according to the relation:
	
	Hence:
	
	This shift of pulsation (frequencies respectively) in a gravitational field is what we name the "\NewTerm{Einstein's effect}\index{Einstein's effect}", or "\NewTerm{gravitational redshift}\index{gravitational redshift}".
	
	We will first prove this relation using conventional arguments and now well known to us. Later we will prove that in fact this is only an approximation of a result that we will get later using General Relativity curvature properties.
	
	A material body sent from the ground to the sky must fight against the gravitational force that pulls it down. So it will lose a certain amount of energy, equivalent to gravitational potential energy gained during the trip. The total energy $E_A$ of the body at the ground level is therefore its mass energy (\SeeChapter{see section Special Relativity}) to which we add the potential energy at the height of the tower:
	
	The energy of this body when you reach the top of the tower is simply its mass energy:
	
	because he had to spend the energy $mgh$ during the trip to go up. The ratio of energy is then:
	
	This ratio being independent of the mass $m$, we can take the limit $m\rightarrow 0$ in order to have the relation for the photon. We then get:
	
	which implies:
	
	That is to say the clocks run slower in a gravitational field as seen by a distant observer!
	
	We will now study this phenomenon in the context of the Minkowski space-time. We will see then a contradiction, what will motivate the transition to a curved space-time: this is the argument of a curved geometry that was used by Schild.

	Let us consider again the human experience of a human in $A$ which sends a wave to his friend positioned in $B$. Given $\Delta t_A$ the time taken by $A$ to emit exactly $1$ cycle of the wave (\SeeChapter{see section Wave Mechanics}):
	
	and $\Delta t_B$ the time taken for $B$ to receive this cycle:
	
	Because of the Einstein's effect just seen previously, we know that $\omega_A>\omega_B$ and therefore that $\Delta t_A<\Delta t_B$ in proper time! That is to say that time passes more slowly for someone on the ground ($A$) than another person in a mountain top ($B$)!

	But as we are in flat geometry and the gravitational field is assumed static, we deduce that space-time trajectories described by the signals must be parallel! This leads to the conclusion that the proper time interval would be $\Delta t_A=\Delta t_B$ (according to Special Relativity).
	
	If we opt for a curved space, we can preserve the relation $\Delta t_A<\Delta_B$, that is to say that time passes more slowly for $A$ than for $B$. This simply results in the fact that curved geometry, the proper time (!) of an observer depends on the metric.

	Let us now notice that same developments can be made by assimilating the previous experience with a train that moves with constant acceleration $g$ (horizontal situation of the previous one!). The observer $A$ is in the rear compartment (equivalent to the floor of the Earth in the preceding experiment) sends a wave to his colleague $B$ on the front of the train (at a distance $h$).
	
	The observer $B$ receives the wave after a time $\Delta t=h/c$. During this time, the train has accelerated, and its speed has increased of a value $\Delta v=g\Delta t=gh/c$. Therefore, the wave seen by $B$ will be altered by the conventional Doppler effect (\SeeChapter{see section Wave Mechanics}):
	
	We fall back on the initial results of the Einstein's effect by simply writing:
	
	giving gloriously:
	
	We find more often this relation in the form below in the literature using the relation between pulsation and frequency and Newton's gravitational force to explicit $g$:
	
	 and putting $h$ as being equal to $r$:
	 
	and after rearranging we also found sometimes in textbooks:
	
	or even more frequently:
	
	We also find this last relation in the following condensed form:
	
	The same result can be obtained using the Schwarzschild metric (see further below), hence the name of this effect that can also be obtained from the mathematical tools of Einstein's General Relativity. We will prove later, in a simple way, using this metric that time actually flows more slowly in a gravitational field (assumption we made a few paragraphs above).
	
	We see that in all cases:
	
	since the right term is positive and not zero. This simply means that the electromagnetic wave in analogy to the color spectrum shifts toward red. Thus, Einstein's effect is indeed a gravitational redshift!

	The frequency difference is very small and therefore difficult to measure even with the best spectroscopes. The slightest disturbance can completely mask the Einstein's effect. It will be necessary to wait until 1960 that the experience of Pound and Rebka to be capable of measuring a frequency offset with an accuracy of $1\%$ therefore leaving no doubt as to the reality of this phenomenon.
	
	\subsection{Equations of movement}
	We will prove here that the equation of motion of a free particle is constant along its world line by first limiting ourselves to the case of a flat space (Minkowski space type ). Then we will generalize this result in any kind of space using a simple development, to show quite clearly that the equation of motion is independent of the mass and follows the curvature of space !!! Finally, we will present a second proof in any kind of space using the variational principle.

	So let us start by proving the equation of motion of a free particle in a flat space.

	During our study of Special Relativity, we have proved the Relativistic Lagrangian of a free particle given by:
	
	and for this we started from the action (hypothetical):
	
	and we came to write:
	
	Now let us show something interesting! Let us recall that for the Minkowski space-time, we got:
	
	and restricting ourselves to one spatial dimension, we obtain as relation:
	
	Therefore:
	
	and then ... well that's the way, if we put:
	
	we finally have:
	
	so we fall back on the same action from a more general form (pure) action that is:
	
	result that we had also proved in the section of Electrodynamics !! We can even do better in terms of elegance ...! If we observe well the developments of the previous lines, we observe that in facts the relation:
	
	and is the special case to one dimension of the relation:
	
	with as defined earlier above:
	
	and therefore:
	
	Thus we have the "\NewTerm{Fitzgerald-Lorentz factor}\index{Fitzgerald-Lorentz factor}" or simply "\NewTerm{Lorentz factor}\index{Lorentz factor}" that is given in general form by:
	
	as a generalization of Special Relativity!

	This being done, let us come back on our topic... In an space without potential field, we have proved the section of Analytical Mechanics that the Lagrangian is reduced to the simplest expression of the kinetic energy such that:
	
	If we wish to generalize this relation for it to be valid in any type of space (curved or flat), we must introduce the curvilinear coordinates as we have studied them in the section of Tensor Calculus.

	In a first time, this gives:
	
	where for recall $\mathrm{d}s$ is the curvilinear abscissa of the path.
	
	And we have proved in the section of Tensor Calculus that:
	
	The latter relation is written in the context of relativistic mechanics in a most standard way:
	
	where $\tau$ is a parameter that in relativistic mechanics if for recall the proper time of the particle.
	
	Before we focus on curved spaces described by the metric $g_{\alpha\beta}$ (which we will do during our proof of the free generalized Lagrangian ), let us restrict us to Euclidean space with the metric (this will be a good exercise to understand) given by the Minkowski matrix (\SeeChapter{see section Special Relativity}):
	
	which we denote $\eta_{\alpha\beta}$ to differentiate it from others (because most often used). Finally we in Euclidean space:
	
	Now let us apply the variational principle:
	
	The variation $\mathrm{d}s$ can be found simply from the variation of $\mathrm{d}s^2$:
	
	we find:
	
	The factor "$2$" is because by symmetry of the Euclidean space, the variations of $\mathrm{d}x^\alpha$ and $\mathrm{d}x^\beta$ are equal. 
	\begin{tcolorbox}[title=Remark,colframe=black,arc=10pt]
	As we will see later, this relation $\delta(\mathrm{d}s)^2$ will not be the same anymore when dealing with curved spaces.
	\end{tcolorbox}
	Simplifying a bit, we get:
	
	Which is equivalent to write:
	
	We can now go back to the action:
	
	We rewrite the preceding integral as following (it will be easier to treat):
	
	Indeed, let check that this form is similar:
	
	So let us come back to our integral:
	
	We have then two integrals that it will be a bit easier to analyze. The first integral:
	
	simply gives an expression evaluated to the temporal extremities $(\tau_1,\tau_2)$. Therefore, as the values  $x^\alpha$ are perfectly known at the time ends, the variational $\delta x^\alpha$ is zero at the both extremities and this integral is therefore zero.
	
	Then we are left only with this integral:
	
	So for the variational principle (\SeeChapter{see section Analytical Mechanics}):
	 
	is respected, we must have:
	
	Now, we can write this expression explicitly. Indeed, we have:
	 
	Remember also that we have proved earlier above that:
	
	and that we have:
	
	Therefore:
	
	Now, let us recall that during our study of Special Relativity, we have proved the path that led us to define the linear momentum four-vector:
	
	So finally, what cancel the variational of the action integral can be written:
	
	We thus fall back on the conservation equation of linear momentum (momentum conservation) that we name in the framework of General Relativity "\NewTerm{equation of motion}\index{equation of motion}". This form of the equation of motion seems dependent on the mass but by digging a bit, we will see that it is fact not.
	
	Multiplying this relation by $\eta_{\mu\nu}$ we can also write:
	
	and the same for another observer:
	
	In other words, the linear momentum of the particle remains constant along its world line.

	But we can also write:
	
	Therefore:
	
	An even more important form of movement equation can be obtained. Indeed using the relations just proved above we can write:
	
	Therefore:
	
	Hence:
	
	this relation is therefore "massless" equation of motion in Euclidean space or in other words, in a Minkowski space-time type. In other words, there exists a falling coordinate system wherein the motion of the particle is a uniform movement in space-time.
	
	It will be very interesting to compare it later with the equation of motion in a curved space as we will see later (named "geodesic equation").
	\begin{tcolorbox}[title=Remark,colframe=black,arc=10pt]
	It is equivalent to write the relations of equations of motion with respect to the curvilinear abscissa $\mathrm{d}s$ or the proper time $\mathrm{d}t$ (traditionally denoted by $\mathrm{d}\tau$ in the field of General Relativity).
	\end{tcolorbox}
	We can now prove that the previous equation of motion, just like the geodesic equation that we will see afterwards, is invariant under Lorentz transformation. Indeed:
	
	Now let us see a more general form of the equation of motion for any kind of space. The aim is to highlight, and this in a few lines of calculations, that the movement followed by a free particle is independent of its mass (you can already anticipate the interpretation of the path of a photon in a curved space...!).

	Let us first recall that we have proved in the section of Tensor Calculus (and previously) that:
	
	giving us for the generalized Lagrangian of a free particle with $\mathrm{d}\alpha=\mathrm{d}\tau,u^i=x^\alpha,u^j=x^\beta$ (although we fall back on the general expression of the kinetic energy as there is no potential for a free particle):
	
	where for recall $\tau$ is the proper time\footnote{The proper time is for recall a kind of imaginary clock that travels on the particle and whatever observers watch the clock, they will mathematically agree on the value of the time interval between two "Tic" of the clock.} of the particle, it is an invariant!
	\begin{tcolorbox}[title=Remark,colframe=black,arc=10pt]
	This relation is named the "\NewTerm{geodesic lagrangian}\index{geodesic lagrangian}" by some text book authors.
	\end{tcolorbox}
	This allows us to write (caution! the reader must remember the different relations that we had determined during our study of the Lagrangian formalism in the section dealing with Analytical Mechanics):
	
	\begin{tcolorbox}[title=Remark,colframe=black,arc=10pt]
	The elimination of the $1/2$ Lagrangian factor results from the symmetry of the metric tensor. If that latter is not symmetric, we can always characterize it by a tensor that is.\\

	Indeed, for recall (\SeeChapter{see section Tensor Calculus}, given $\vec{x}$ a vector of coordinates $x_1,\ldots,x_n$ and given:
	
	The $T_{ij}$ are not symmetric a priori, but we can write:
	
	We put afterwards:
	
	Therefore:
	
	And the $B_{ij}$ are symmetric.\\
	
	The quadratic form $q$ can thus always be written with a symmetric matrix, there is even a bijection. The conclusion is that a metric tensor must be symmetric if we want to characterize it by the quadratic form it defines.
	\end{tcolorbox}
	The mathematical interlude having ended, let us continue our physical development. As a consequence of the last relation, the expression of the Hamiltonian obviously becomes:
	
	since we consider to be in a space without potential field anymore. Since the square of the velocity is therefore constant over the entire trajectory, we have:
	
	Let us now establish the equations of motion of any body. We have:
	
	and as:
	
	then:
	
	hence:
	
	By putting everything together we get:
	
	that we can write identically for the $\ddot{x}^\alpha$ by proceeding in the same way as above.

	The preceding relation therefore gives the trajectory of a body in motion, in a space without a potential field, as a function of its curvilinear coordinates and of the metric of the space under consideration.

	What is particularly interesting in this result is that mass $m$ (again) is eliminated identically in this equation of motion:
	
	Notice that we could have used another invariant parameter as well as the proper time $\tau$ such as the curvilinear abscissa $\mathrm{d}s$. Hence the preceding equation should be written:
	
	We can still simplify this relation, but we will keep this simplification for the second proof of the equation of motion in any space (by making use of the variational principle this time) just further below.

	It is very (very) interesting to observe that if we restrict the metric to that of a Euclidean space:
	
	with:
	
	We then have the following simplification:
	
	That it remains only:
	
	By lowering the indices with the signature it remains:
	
	We thus fall back on the first equation of the motion obtained for a flat space! The result is remarkable!

	The conclusions is that at the same initial conditions of curvilinear position and velocity in a space (flat or curved) without a potential field (this is what we could think at least according to our initial hypotheses ...), corresponds the same trajectory whatever the mass $m$ of the particle (even for photons - light - whose rest mass is zero!).

	We can now study the principle of least action in order to seek the shortest path (both spatially and temporally!) between two points in a given geometric space before addressing the much more complex case of the Lagrangian which takes In account the tensor field...
	
	\subsubsection{Geodesic equations}
	Let us now turn to the same result, but this time using the variational principle. We will fall on the same equation as before for any kind of space with the difference that this time we will take the time to simplify it to arrive at the "geodesic equation".

	Starting from (see previous developments):
	
	with a parametrization such that $x^i$ and $x^j$ depend of a temporal or spatial parameter.

	For a given surface in parametric form, we therefore seek to minimize the length of an arc $\mathrm{d}s$ by applying the variational principle (not dependent on time) because the photons can not have a faster path in the temporal sense of the term between two points but only a shorter path - in the metric sense of the term!):
	
	in natural units. Or:
	
	By developing, and as the indices have the same range of variation:
	
	hence (we have already multiplied the expression after the second equality by $\mathrm{d}s/\mathrm{d}s$ by anticipating the integral that follows):
	
	Then, we must introduce this development under the integral:
	
	Working on the second integral (after the equality), we put:
	
	So by integration by part (\SeeChapter{see sectoin Differential and Integral Calculus}):
	
	becomes:
	
	Thus finally:
	
	The non-integrated term below:
	
	is negligible because of the presence of the factor $\delta \mathrm{d}x^j$:
	Therefore:
	
	We make a change of index:
	
	Which allows us to factorize $\delta x^k$:
	
	As $\delta \mathrm{d}x^k$ and $\mathrm{d}s$ are different from zero, it is the integrande that must be zero:
	
	By developing the second term:
	
	Which can also be written (in the physicist way of life...)
	
	Which simplifies into:
	
	We fall back (again!) on the system of equations which defines the "\NewTerm{geodesics}\index{geodesic}", that is to say the straight lines of $\mathcal{E}^n$. These latter then constitute the extremities of the integral which measures the length of a curve arc joining two given points in $\mathcal{E}^n$.

	This last equation is the one which interests us in the case of the free Lagrangian. Indeed, if we take the extreme case of light (or photons if you prefer), the latter will not seek the fastest path at the temporal level. This would totally contradict the postulate of invariance to see the light accelerate according to the path !!! In this context, it means that on the spatio-temporal framework, the only thing that has meaning is the shortest spatial path and not the shortest temporal path! This is why the latter equation is named the "\NewTerm{geodesic equation}\index{geodesic equation}" or also "\NewTerm{generalized Euler-Lagrange equation}\index{generalized Euler-Lagrange equation}".

	However, we can write this last equation in a more condensed form by introducing the Christoffel symbols if the metric is a symmetric tensor, that is to say if $g_{\alpha\beta}=-g_{\beta\alpha}$.
	
	Indeed:
	
	And as the Christoffel symbol of the first kind (\SeeChapter{see section Tensor Calculus})
 is defined by:
	
	\begin{tcolorbox}[title=Remark,colframe=black,arc=10pt]
	It is important to remember that this symbol contains almost all information about the space-time metric. We will see an example below as what in a locally inertial frame this Christoffel symbol is equal to zero.
	\end{tcolorbox}
	Then the Euler-Lagrange equation is then written:
	
	The contracted multiplication (\SeeChapter{see section Tensor Calculus}) of the preceding relation in the canonical basis by $g^{kl}$ gives us:
	
	Hence
	
	In the literature a change of index is often carried out in order to at the end (it is still the same expression given that the indices have the same range of variation!):
	
	with $\Gamma_{\alpha\beta}^\mu$ being the Christoffel symbol of the second kind (\SeeChapter{see section Tensor Calculus}) given by:
	
	and is named in the context of General Relativity the "\NewTerm{affine connection}\index{affine connection}" or "\NewTerm{connection coefficients}\index{connection coefficients}" and which makes it possible to find the system of coordinates (through the resolution of a system of differential equations) in free fall in which the particle equation is that of a uniform movement in space-time as a function of a reference system (the two systems are therefore connected by the affine connection).

	This relation, of the highest importance, allows us to determine how a moving body will naturally move in a curved space and this perhaps ... regardless of its mass !!! It therefore gives us the metric in which we must set a frame of reference so that it is inertial with respect to the body in question.

	The previous equation of geodesics is also the differential equation of the second order which must therefore satisfy the parametric representation of a line on a surface where $s$ is the length along the line so that its total length is extremal!!!

	According to the principle of equivalence, we are therefore entitled to interpret this relation as the equation of motion in any gravitational field of and thus to interpret the second additional term of the equation as the opposite of a gravitational term force per unit mass, that is to say as the opposite of a gravitational field!
	\begin{tcolorbox}[title=Remark,colframe=black,arc=10pt]
	We can also write the equation of the geodesics and using the proper time. Indeed:
	
	or by using the four-vector velocity:
	
	\end{tcolorbox}
	Again, if we restrict ourselves to a flat space-time, we see trivially that we fall back on the first equation of motion that we had obtained since for the Minkowski metric $\eta_{\mu\nu}$ we have immediately $\Gamma_{\alpha\beta}^\mu=0$:
	
	because the components of the Minkowski metric being constant the Christoffel coefficients are all zero.
	
	The solutions of the latter equation are ordinary straight lines given by:
		
Obviously, in a general curved space-time, the geodesics can not be globally represented by straight lines. However, with a second-order approximation in Taylor's development (\SeeChapter{see section Sequences and Series}), we fall back on straight lines (which is equivalent to bringing the curved space back to a flat space).

	The important thing in all this is that the equation of geodesics makes it possible to observe that the curvature of space determines the trajectories of the bodies which move there whatever their mass, whether they are in uniform motion or not (observe the second derivative in the geodesic equation!). All that remains is then to complete the work and to relate the curvature of space-time with the energy that is there!
	
	\subsubsection{Newtonian Limit}
	We have shown above (Shild's argument) that to study gravitation (in particular the Einstein's effect), curved geometry is necessary. We promised also to show that it was enough. Now is the time to do it!

	\textbf{Definition (\#\mydef):}  The "\NewTerm{Newtonian limit}" is a physical situation where the three conditions below are satisfied:
	\begin{enumerate}
		\item[C1.] The particles move slowly with respect to the speed of light. This is expressed as the fact that the variations of the spatial components of their quadrivector are much less than those of the temporal component ($t$ being the proper time):
		

		\item[C2.] The gravitational field is static. In other words, any time derivative of the metric is zero!

		\item[C3.] The gravitational field is weak, that is, it can be seen as a weak perturbation of a flat space:
		
		with $|h_{\mu\nu}|\ll 1$ and where $\eta_{\mu\nu}$ is constant (only $h_{\mu\nu}$ depends on the coordinates).
	\end{enumerate}
	Let us consider the geodesic equation obtained previously:
	
	The first condition (C1) leads us to simplify it in the form:
	
	The two other conditions (C2 and C3 whose application has been shown in the development below) offer us several simplifications in the expression of the symbol of Christoffel of the second kind:
	
	The geodesic equation then becomes:
	
	and is then equal for the temporal component to ($\mu=0$):
	
	But (recall of the Minkowski metric):
	
	for $\lambda>0$ and for $\lambda=0$ we have (static metric):
	
	Therefore, we must conclude that $\mathrm{d}x^0/\mathrm{d}t$ is a constant (whatever the choice of the signature of the Minkowski metric).

	And for the spatial components, we know that $\eta^{\mu\nu}$ when reduced to its spatial part is a simple unitary $3\times 3$ matrix, which gives for each spatial component in the case where we choose (by tradition only!) the signature $- + + +$ of the Minkowski metric:
	
	Obviously, the reader can have fun making the development that follows with the inverse signature ($+ - - -$) and he will see that it only changes the sign of potential in the final result of the development):
	Let us now rearrange the above relation:
	
	By dividing by $(\mathrm{d}x^0/\mathrm{d}\tau)^2$ and restoring $x^0=c\tau$, we get by making as sequence of simplifications:
	
	Starting from here we put (because our illustrious predecessors have tried before us):
	
	such as (a relation which will be very useful to us when studying the Schwarzschild metric further below):
	
	where $\varphi$ is the gravitational potential. We fall back here on the expression of the gravitational acceleration (Newton-Poisson equation) of the Newtonian mechanics (\SeeChapter{see section Astronomy}):
	
	with $i=1,2,3$.

	This development, simple but nevertheless remarkable by its interpretation, proves that the curved geometry is sufficient to describe the gravitation (and therefore the theory of Newton)!!!!!!!!!!!! This verification is named by some people the "\NewTerm{principle of correspondence}\index{principle of correspondence}".
	
	\subsection{Stress-Energy Tensor}
	The "\NewTerm{Stress-Energy Tensor SET}\index{Stress-Energy Tensor}" (sometimes named "\NewTerm{stress–energy–momentum tensor}\index{stress–energy–momentum tensor}" or "\NewTerm{energy–momentum tensor}\index{energy–momentum tensor}") is a mathematical tool used (in particular) in General Relativity to represent the density and flux of energy and moment in space-time, generalizing the stress tensor of Newtonian physics of mass and energy. It is therefore an attribute of matter, radiation, and non-gravitational force fields. The stress–energy tensor is the source of the gravitational field in the Einstein's field equations of general relativity, just as mass density is the source of such a field in Newtonian gravity.
	
	Let us take for example the SET which considers matter in General Relativity as being able to be approximated by a perfect fluid. In the section Continuum Mechanics we have proved:
	
	where $N_i$ has for recall the units of a force and $n_j$ those of a surface. Thus with a more conventional writing:
	
	In variational form this gives:
	
	Let us now calculate:
	
	\begin{tcolorbox}[title=Remark,colframe=black,arc=10pt]
	We do not work with differential elements to avoid being trapped later. It is completely a physicist Do It Yourself approach, but it works well (confirmed by experience...).
	\end{tcolorbox}
	Assuming that only the volume and the time makes that the force varies (which assume a constant density and the to be inertial) we then have:
	
	This gives simply the tensor product of the velocities (\SeeChapter{see section of Tensor Calculus}):
	
	If we generalize this relation to the velocity quadrivectors of Special Relativity with the corresponding notations, then we have by definition the "\NewTerm{energy-momentum tensor}\index{energy-momentum tensor}" or "\NewTerm{Stress–energy tensor}\index{Stress–energy tensor}":
	
	or in index form:
	
	Either in contravariant form (most common form in textbooks):
	
	This relation is the justification for which General Relativity is also indicated as a theory of continuous mechanics by some specialists.

	Now let us prove that the derivative:
	
	\begin{tcolorbox}[title=Remark,colframe=black,arc=10pt]
	What we have already pointed out in the section of Tensorial Calculus is written $T^{0j}_{,j}$ in old books or in modern textbooks where the author want to show its technical level...
	\end{tcolorbox}	
	First, let us recall that (\SeeChapter{see section Special Relativity}):
	
	and let us admit that we are in low speeds such as $\gamma=01$. Then, in a Minkowski metric of type $(+, -, -, -)$ we have:
	
	But, we recognize in the parentheses the equation of continuity (conservation of the mass) which we have proved in the section of Thermodynamics and which we know is equal to zero in a system without sources! Therefore:
	
	Let us also look that what contains the componant $T^{00}$ of the stress-energy tensor:
	
	In terms of units, this is an energy density (we see directly that this quantity can only be positive).

	Let us now look at the other components with $i=0$ and $j=1,2,3$:
	
	where $p^i$ has the units of linear momentum density.

	Let us now consider the components of the tensor when $i,j=1\ldots 3$ (we omit then the first row and the first column):
	
	We thus fall back on the components of the stress tensor of a perfect fluid.

	So finally, the stress-energy tensor can be written in the form of a symmetric real $4\times 4$ matrix:
	
	
	This tensor is also sometimes represented as following:
	
	We thus fall back in this tensor on the following interpretations of the physical quantities (although rigorously all the components have units which can be seen as density of energy or as a pressure):
	\begin{figure}[H]
		\centering
		\includegraphics[scale=0.4]{img/cosmology/stress_energy_tensor.jpg}	
	\end{figure}
	We then understand better why the this matrix is named "Energie-Momentum Tensor" or "Stress-Energy-Momentum Tensor" since implicitly it is a question of modeling the space by a perfect fluid under shear stresses (tangential forces) and tensions (normal forces).
	\begin{tcolorbox}[title=Remark,colframe=black,arc=10pt]
	The sub-matrix of spatial components:
	
	is the matrix named the "\NewTerm{matrix of moments flows}\index{matrix of moments flows}" (a name that is quite debatable ...). In Continuum Mechanics (see section of the same name), we have proved that its diagonal corresponds to the pressure, and the other components to the tangential forces due to the dynamic viscosity.
	\end{tcolorbox}
	Let us prove that the covariant derivative (\SeeChapter{see section Tensor Calculatus}) of the stress-energy tensor is zero such that:
	
	Therefore:
	
	Let us begin by developing the first term:
	
	But we have:
	
	hence:
	
	We find in the squared brackets the equation of continuity which is zero in the absence of sources. On the other hand, the first term in parentheses is non-zero as we saw in our study of the four-accelerator acceleration in the section of Special Relativity:
	
	But according to the weak principle of equivalence (WPE), we can always place ourselves in a repository such that locally the acceleration is null, that is to say such that (for recall, we do not put vector arrows for the quadrivectors):
	
	And it comes then:
	
	So we now have:
	
	Let us look at what this last term gives but first recalling that in the section of Special Relativity we had proved that the quadri-acceleration was expressed according to:
	
	Therefore (we take only the first two components as examples):
	
	We will now in fact prove that:
	
	for this we start first to prove that:
	
	For this we calculate first:
	
	But:
	
	Therefore:
	
	Now let us prove that:
	
	the other components $a^2$, $a^3$ are then verified automatically.

	For this we do little bit algebra:
	
	and therefore we have indeed:
	
	but according to the WEP, $a^\nu=0$ therefore:
	
	and finally we have indeed under the assumptions stated above:
	
	Which is the expression of the conservation of energy in general relativity! By lowering the indices it comes:
	
	
	\pagebreak
	\subsection{Einstein's Field Equations}
	It is now time to tackle one of the most beautiful, one of the most famous equations of our time and that shines the eyes of many young students and science passionate: Einstein's field equations. The one that explains why matter (energy) curves space!!! There are several ways to obtain these equations. The two most common ones are either:
	\begin{enumerate}
		\item To have an engineer approach: That is to say we proceed by comparison with a known limiting result which is the law of gravitation of Newton (it is the one that we have chosen)

		\item To have a pure mathematic approach (very elegant but a little fallen from the sky with some circular reasoning): That is to say that we use the Lagrangian formalism and seek by trial and errors a Lagrangian density which allows us to fall back on something known.
	\end{enumerate}
	Well this having been said, let us recall before starting some results that we have obtained so far. First, we have succeeded in proving brilliantly that every particle (assumed to be free but left to interpretation ... in a curved space ...) follows the equation of motion of geodesics:
	
	In the section of Tensor Calculus, we have proved (not without difficulty...) what we name the "\NewTerm{Einstein's tensor}\index{Einstein's tensor}" (which is a constant in a given Riemannian space) is given by:
	
	where $R^{\mu\nu}$ is for recall the Ricci tensor (\SeeChapter{see section Tensor Calculus}).

	Since the covariant derivative of the Einstein's tensor is zero (\SeeChapter{see section Tensor Calculus}) and we have proved that the covariant derivative of stress-energy tensor is also, then it is tempting to put:
	
	where $\kappa$ is a normalization constant and must satisfy the relation so that it is homogeneous at the level of the units. So it comes (we should better say: "we think we can write...") after simplification:
	
	To find the expression of the constant, we will place ourselves in the Newtonian limit and request that the preceding relation reproduce the Poisson's equation for the gravitational potential $\Phi$ (\SeeChapter{see section Astronomy}):
	
	\begin{tcolorbox}[title=Remark,colframe=black,arc=10pt]
	This relation shows that the gravitational potential is connected to the matter density linearly through its second derivatives. Albert Einstein thought, therefore, that the first member of the equations of the field in General Relativity, member supposed to describe the geometry of space-time, must therefore somehow include the second derivatives, not of the gravitational potential, but of the potentials of the metric. In fact, Albert Einstein tried to generalize the right-hand side of the Poisson equation: the desired quantity must include not only the density of matter but also the momentum (as soon as the body is moving, its energy increases and therefore its mass). To evaluate the gravitational effect of a body, it was therefore necessary to combine its mass at rest with its momentum. It was finally the stress-energy tensor of rank $2$ which is the generalization of the quadrivector momentum of Special Relativity.
	\end{tcolorbox}
	We have proved earlier above that in the Newtonian limit (weak field approximation):
	
	and in our definition of stress-energy tensor, for a distribution of matter at rest (or in a coordinate frame according to...) only the following component is non-zero:
	
	It follows that the Poisson equation can be written:
	
	Now let us return to the relation:
	
	By contracting the two members of the preceding relation, it comes:
	
	that is to say more explicitly (\SeeChapter{see section Tensor Calculus}):
	
	But, the Ricci scalar (\SeeChapter{see section Tensorial Calculus}) is given by:
	
	It comes therefore:
	
	Now in the special case of the Minkowski metric (with the signature $(-, +, +, +)$) it is immediate that:
	
	Therefore (implicity we continue the Minkowski metric!):
	
	Using this last relation, the equation:
	
	can finally be written:
	
	Let us focus on the component $\rho=\sigma=0$ (not to be confused with the notation of shear stress and density!!!) such that the preceding relation is written:
	
	Let us write explicitly this last relation by using the definition of the Ricci tensor (\SeeChapter{see section of Tensor Calculus}) that is for recall:
	
	Then it comes:
	
	But, the Riemann-Christoffel tensor developed in this particular case is given for recall by (\SeeChapter{see section of Tensor Calculus}):
	
	\begin{tcolorbox}[title=Remark,colframe=black,arc=10pt]
	In the absence of a gravitational field and in Cartesian coordinates, it is logical that all the Christoffel symbols are null. Indeed, the Christoffel symbols translate nothing more than the forces of inertia. But when we have a field of gravitation, the trajectories followed are no longer straight lines, even in the Newtonian case, then the Christoffels symbols are non-zero.
	\end{tcolorbox}
	In the approximation of the weak field slowly variable over time, the Christoffel symbols are of order $\mathcal{O}^1$ and their products are of order $\mathcal{O}^2$ and the temporal derivatives are negligible in front of the spatial derivatives. It therefore remains only the terms of order $\mathcal{O}^1$ such that:
	
	But, we have proved in the section of Tensor Calculus that:
	
	Since then:
	
	But in the weak field approximation, the variation of the metric with respect to time is negligible compared to the spatial variation (the approximation is somewhat pulled by the hair it must be said ...):
	
	Therefore, the relation:
	
	becomes:
	
	and we immediately notice that we fall back on the Poisson's equation if and only if:
	
	Constant which is sometimes named "\NewTerm{Einstein's constant}\index{Einstein's constant}". It follows immediately that the Ricci scalar is positive and therefore that we are locally in a spherical curvature space.

	The "\NewTerm{Einstein's field equations EFE}\index{Einstein's field equations}" is therefore in definitive form:
	
	or more conventionally:
	
	The left-hand part represents the curvature of space-time as determined by the metric and the right-hand expression represents a modelization of the space-time content of mass / energy. This equation can then be interpreted as a set of equations describing how the curvature of space-time is related to the mass-energy content of the Universe. These equations, as well as the geodesic equation, form the core of the mathematical formulation of General Relativity.
	
	The EFE  is therefore a dynamic equation describing how matter and energy modify the geometry of space-time. This curvature of the geometry around a source of matter is then interpreted as the gravitational field of this source. The movement of objects in this field is described very precisely by the equation of its geodesic.
	
	Similar to the way that electromagnetic fields are determined using charges and currents via Maxwell's equations, the EFE are used to determine the spacetime geometry resulting from the presence of mass–energy and linear momentum, that is, they determine the metric tensor of space-time for a given arrangement of stress–energy in the space-time. 

	On the other hand, we have just seen that Einstein's equation reduces to the laws of Newton's gravity by using the approximation of weak fields and slow movements. 
	
	These differential equations are in general a nightmare to solve, the Ricci scalars and tensors are contractions of the Riemann tensor, which include the derivatives and products of the Christoffel symbols, which are themselves constructed on the inverse metric tensor and on the derivatives of it. To compute the whole, it is possible to construct energy-momentum tensors that can invoke the metric as well. It is therefore very difficult to solve the Albert Einstein equations of fields in the general case. Exact solutions for the EFE can only be found under simplifying assumptions such as symmetry. Special classes of exact solutions are most often studied as they model many gravitational phenomena, such as rotating Black Holes and the expanding universe. Further simplification is achieved in approximating the actual space-time as flat space-time with a small deviation, leading to the linearised EFE. These equations are used to study phenomena such as gravitational waves.

	Since the stress-energy tensor has $16$ components, $10$ of which are actually unique (independent) since the tensor is symmetric, we can see the Einstein equation of the fields as $10$ second-order differential equations coupled on field tensor metric $g_{ij}$.
	
	Some people are confused about how the curvature of space-time and gravity are related. I am going to explain mainly that starting with simpler examples, and moving to more complicated ones.

	Okay, let's say we have a sheet of rubber. This is the classic example of spacetime. Let's say wet ake a bowling ball, and set it on the taut sheet of rubber. It has a large mass (compared to what else we'll be putting on the sheet), therefore the sheet curves a lot for the bowling ball. We now have an image in our head like the one below:
	\begin{figure}[H]
		\centering
		\includegraphics[scale=0.8]{img/cosmology/general_relativity_2d_space_curvature.jpg}	
		\caption{2D naive representation of space curve near Earth}
	\end{figure}
	So mass leads to curvature. Then, let us take a baseball, say, and set it near the bowling ball. It rolls toward the bowling ball, right? This occurs because of the curvature of the sheet. So, then, curvature leads to gravity. So, if an object has large mass, it will curve space-time dramatically, leading to strong gravity.

	This is, of course, an overly simplistic example. It is 2D, and it doesn't take into account other factors. Let us move to 3D (keeping in mind the universe is accepted to be at 4D, ignoring the holographic principle). The mass of a bowling ball now sucks in space around it, sort of like in the picture below:
	\begin{figure}[H]
		\centering
		\includegraphics[scale=0.9]{img/cosmology/general_relativity_3d_space_curvature.jpg}	
		\caption{3D naive representation of space curve near Earth}
	\end{figure}
	And now, in this case, we can see (or understand) that more mass still leads to more curvature. The greater the mass, the more space-time will "contract" around the object. So we still think that mass leads to curvature. Now, if we set an object near this massive object (like the moon next to Earth) it is "sucked in" sort of, by the curvature of space-time, though of course the moon contracts space-time around it as well. At this point, we can reasonably still conclude that in 3D, mass leads to curvature which leads to gravity.
	
	A quick glance at the oconstant of prooertionality in the Einstein field equtions gives one a roguht feeeling of much stres-energy is needed to curve space. In SI units, the gravitational $G$ is about $6.67\cdot 10^{-11}\;[\text{m}^3\cdot\text{kg}^{-1}\cdot\text{s}^{-2}]$ while the spped fo light $c$ is approximately $3.00\cdot 10^8\;[\text{m}\cdot \text{s}^{-1}]$. The field equation reat then in numertical value:
	
	The sun has an average mass-energy density (the dominant component of the stress-energy tensor) of $T^{00}\cong 1.27\cdot 10^{20}\;[\text{kg}\cdot\text{m}^{-1}\cdot\text{s}^{-2}]$. The corresponding component of the Einstein Tensor is therefore $G_{00}\cong 2.64\cdot 10^{-23}\;[\text{m}^{-2}]$. By comparison the Einstein tensor for the flat Minkowski metric is identically zero. So to see a curvature we need to look at hyperenergetic phenomena, like a collapsing star, to fin an Einstein tensor component appreciable greater than this. Even though the space-time metric $g_{\mu\nu}$ is not generally flat, throughout most of the universe it is flat enough to be considered as small perturbation of a flat background metric:
	
	But, as I said earlier, the Universe is generally thought of as 4D. What does our picture look like when we add time? Well, the time dimension is contracted around a massive object. So let us picture our previous example but that the fabric of space-time has a few clocks embedded in it occasionally. As the space stretches and contracts, so will the clocks (the "time") and so the time on those clocks will be "wrong" - it'll differ from the other clocks. And in this case, as the Earth contracts space and time around it, it changes the time and space (it curves space-time) and so when another object enters our region of space-time, it is "sucked in" still, but so is it's time. This is, of course, a very extreme example, but I hope this shows that we can conclude that mass leads to curvature which leads to gravity. 
	
	\pagebreak
	\subsubsection{Cosmological Constant}
	Albert Einstein modified his original field equations to include a cosmological constant term $\Lambda$ proportional to the metric that led afterwards the Universe model to be static (\SeeChapter{see section Cosmology})
	To see how this constant was introduced let us recall that we have proved so far that:
	
	or more explicitly:
	
	That is to say:
	
	But we have proved in the section of Tensor Calculus that the covariant derivative kills the metric, that is to say for recall:
	
	Therefore if we choose a constant $\Lambda$ the latter relation can also we written:
	
	Obviously:
	
	So nothing avoid us to put this covariant derivative in:
	
	as we can write:
	
	and replacing the $0$ by the covariant derivative of the metric:
	
	After factorization we get:
	
	And simplifying we get the "\NewTerm{general Einstein's field equation}\index{general Einstein's field equation}":
	
	where $\Lambda$ is the so named "\NewTerm{cosmological constant}\index{cosmological constant}" (a.k.a. dark energy in conteporary physics).
	
	The latter relation can be found also in many textbooks in natural units (\SeeChapter{see section Principia}) and rearranged a little bit as following:
	
	The effort from Albert Einstein to introduce this constant was unsuccessful because:
	\begin{itemize}
		\item The universe described by this theory was unstable
		\item Observations by Edwin Hubble confirmed that our Universe is expanding
	\end{itemize}
	So, Albert Einstein abandoned $\Lambda$, calling it the "biggest blunder [he] ever made".

		Despite Albert Einstein's motivation for introducing the cosmological constant term, there is nothing inconsistent with the presence of such a term in the equations. For many years the cosmological constant was almost universally considered to be $0$. However, recent improved astronomical techniques have found that a positive value of $\Lambda$  is needed to explain observations that seems to give an accelerating universe.	
	\begin{figure}[H]
		\centering
		\includegraphics[scale=1]{img/cosmology/einstein_efe_leiden.jpg}	
		\caption{Diagram of gravitational lensing  with formula of Albert Einstein on a wall of Museum Boerhaave, Leiden in Netherlands (source: Wikipedia, author: Stichting Tegenbeeld,  photograph: Vysotsky)}
	\end{figure}
	
	\pagebreak
	\subsubsection{Schwarzschild Solution}
	The "\NewTerm{Schwarzschild metric}\index{Schwarzschild metric}" is an approximate solution of the EFE  in the case of an isotropic non-rotating gravitational field, without electric charge, zero universal cosmological constant  and at a great distance from the source. It provides the three main proofs of General Relativity: the shift of clocks, the deviation of light by a dense celestial body and the advance of the perihelion of Mercury. These three proofs are very important because Einstein's equation was not experimentally demonstrated at the time.  The solution is a useful approximation for describing slowly rotating astronomical objects such as many stars and planets, including Earth and the Sun. The solution is named after Karl Schwarzschild, who first published the solution in 1916.

	To introduce this metric, let us imagine a source (for example the Sun) which produces a gravitational field by means of its mass $M$. We seek, in order to compare with the experiment, the solutions of Einstein's equation (in other words: the metric) outside the source (of the Sun therefore ...) of mass $M$.
	\begin{tcolorbox}[title=Remark,colframe=black,arc=10pt]
	There are several mathematical techniques to introduce the Schwarzschild metric. The reader will be able to search, for example, in the literature or on the Internet the one using a gauge transformation ("Einstein gauge" with the "harmonic gauge") for the local perturbation constraint. This method is very elegant but very math oriented and we prefer as the reader alreaydy know it the "engineer" method...
	\end{tcolorbox}
	In other words, this is like assuming to have in the region of space that interests us (considering that there is only the star in question and nothing else around, not even the energy / mass specific to the gravitational field) the following property:
	
	So the EFE proved just above without cosmological constant:
	
	then becomes:
	
	But we proved above that this last relation can also be written using the definition of the Ricci scalar that is given for recall by:
	
	as following:
	
	and since the parenthesis is not null since we have proved above that in the Minkowski metric:
	
	it remains:
	
	and therefore in extenso the scalar of Ricci is also null. This last relation is named the "\NewTerm{vacuum field equations}\index{vacuum field equations}".
	\begin{figure}[H]
		\centering
		\includegraphics[scale=1]{img/cosmology/einstein_coin_vacuum_equation.jpg}	
		\caption{Swiss commemorative coin showing the vacuum field equations with zero cosmological constant (top) and action minimization}
	\end{figure}
	We must therefore find the metric that satisfies this relation (in other words, a metric that far from the source corresponds to a flat space since the Ricci tensor is zero). As there are several possibilities let us focus on a particularly elegant case with as the physicists like it ... full of symmetries.

	The idea is therefore to find a metric, if possible independent of time (therefore the gravitational field as well will be independent of time!) and ... with spherical symmetry (a star or planet being itself of this form), taking into account the mass of the star (this is the major objective!) and such that far enough from the source (...) or when the mass is zero we fall back on the classical metric known and see earlier above:
	
	But this is not totally accurate! Indeed, we work in space-time. But, we have seen that the equation of the curvilinear metric is given in a flat space-time by:
	
	by passing in spherical coordinates we then have:
	
	And it is on this equation of the metric that we must fall back when we are far from the source or that the mass is extremely small ($M=0$). That is to say the Schwarzschild metric must therefore be asymptotically flat, that is to say corresponding to the flat space of Minkowski.

	So let's get to the task. First, we start from what we know (it's better if we can...!). Which means:
	
	And in spherical coordinates including time we have the components $r,\theta,\phi,t$. Rigorously, we denote by:
	
	the "\NewTerm{Schwarzschild coordinates}\index{Schwarzschild coordinates}".
	
	On a total of $16$ terms implied by the prior-previous relation, we finally retain $10$ namely: the $4$ terms of the diagonal and the $6$ other terms of interaction so as to obtain:
	
	where $A$, $B$, $C$, ... are coefficients to be determined.

	Before tackling this work, we know that according to one of our starting constraints, when the mass is weak or we are far from a  non high-speed rotating source, we must therefore fall back on:
	
	therefore intuitively we can already write:
	
	what we must admit it ... is a clear progress ...!

	If as we have imposed it to ourselves at the beginning, the equation of the metric is independent of time, we can by symmetry of time (hypothesis ...) make the following change of variable:
	
	without this changing anything in our $\mathrm{d}s^2$. But, we realize very quick that this will not be the case. Immediately, for this to be satisfied we see that we must have:
	
	Which brings us (it's already better!) to:
	
	Now if the system is indeed spherical, the equation of the metric must be invariant by the transformation $\mathrm{d}\phi=-\mathrm{d}\phi$ (the opposite would be known for a long time if this were not the case experimentally) and/or also for the transformation $\mathrm{d}\theta=-\mathrm{d}\theta$.

	So for this to be correct, we see immediately that in the preceding relation we must impose:
	
	So finally it only remains:
	
	where $A$, $B$, $C$, $D$ will obviously be independent of time (the opposite would contradict our initial constraint) but may by symmetry of the sphere may be dependent of $r$ such that:
	
	Now, let us imagine on the sphere (rigorously it is a hypersphere but it helps anyway...) at a fixed distance $r$ from the center of the source of the field at a given instant $t$ fixed. We then only have:
	
	since $\mathrm{d}t$ is zero (fixed time) and $\mathrm{d}r$ also (fixed distance $r$).

	We have also on the way removed the sign $-$ because we anticipated that it will be eliminated in the third equality that will follow and we will put it then back.

	Now, let imagine we close to the north pole of the sphere ($\theta=0$) we then only have in first approximation:
	
	and at equator ($\theta=\pi/2$):
	
	By symmetry of the field, an infinitesimal angular displacement in each of these two particular zones must, however, be equal. From then on, we can only put:
	
	Hence the equation of the metric is reduced to:
	
	Let us now show that we can choose a system of coordinates for which $C(r)=1$.

	Let us introduce for this a distance defined by:
	
	hence:
	
	Therefore it comes:
	
	hence:
	
	This is further simplified by:
	
	Let's put it all to the square and divide it by left and right by $C(r)r^2=\bar{r}^2$:
	
	Therefore after rearranging a bit:
	
	hence:
	
	hence:
	
	Hence the equation of the metric is written:
	
	It is therefore as if $C(r)=1$:
	
	Therefore:
	
	Therefore:
	
	and the corresponding contravariant metric tensor (that we will further below):
	
	such that for recall (\SeeChapter{see section Tensor Calculus}):
	
	Now, to determine the remaining coefficients (that is, $A$ and $B$) we are going to use the relation that must satisfy metric if it is locally of the Minkowski type:
	
	and therefore the first Bianchi's identity (\SeeChapter{see section Tensor Calculus}) will be automatically satisfied.

	Either in a developed form (\SeeChapter{see section Tensor Calculus}):
	
	with obviously (\SeeChapter{see section Tensor Calculus}):
	
	That is to say that we have quite a lot of work to do... OK! First since the metric is simple the only non-zero derivatives are:	
	
	We then simply deduce the $9$ non-zero elements of the connection (the details are given following the request of a reader):
	
	
	
	
	
	
	
	
	
	
	
	
	
	
	
	
	
	To summarize (we have taken the results with the signature $-, +, +, +$ of the metric instead of $+, -, -, -$ to conform ourselves to the tradition but this does not change the final result):
	
	Now that we have these terms of the connection, we have to calculate their derivative in order to be able to express the first two terms of:
	
	There are then $10$ non-zero terms which are:
	
	We finally have for each component of the Ricci tensor:
	
	The only elements directly non-zero are then:
	
	In a more conventional form (according to the literature) we can simplify a little and moreover keep only the first three equations:
	
	If we add the first two equations, we have:
	
	which equals:
	
	And this also gives us:
	
	We have therefore:
	
	which becomes:
	
	Where we have divide by $2A$ when passing from the second to the third line.

	The reader can verify that a solution of the differential equation is (we can provide the details on request):
	
	Where $S$ is a non-zero real constant. Consequently, the metric for a static solution, symmetrically spherical and in the vacuum (...), is written:
	
	It remains for us to determine a coefficient. But as:
	
	It comes:
	
	Hence:
	
	Finally:
	
	Let us notice that the space-time represented by this metric is asymptotically flat, or, in other words, when $r\rightarrow +\infty$ the metrich approaches that of Minkowski and the space-time variety resembles to that of the Minkowski's space.

	To calculate the constants $K$ and $S$, we use the weak field approximation. In other words, we place ourselves far from the center, where the gravitational field is weak. In this case, the component $g_{tt}$ of the metric can be calculated.

	Indeed, we had studied the Newtonian limit above and obtained the following relation:
	
	with (\SeeChapter{see section Astronomy}) $\varphi=GM/rc^2$. So in extenso we can put without too much fear:
	
	Therefore:
	
	Finally we have for the "\NewTerm{Schwarzschild metric}\index{Schwarzschild metric}":
	
	That is to say in natural units:
	
	What ultimately gives the Schwarzschild metric tensor:
	
	Caution!!! Some reference books have the Schwarzschild metric with different signs because they take the metric $-, +, +, +$ instead of the metric $+, -, -, -$.
	
	Caution!!! The Schwarzschild metric is a solution of Einstein's field equations in \underline{empty space}, meaning that it is valid only outside the gravitating body. That is, for a spherical body of radius $R$ the solution is valid for $r > R$. To describe the gravitational field both inside and outside the gravitating body the Schwarzschild solution must be matched with some suitable interior solution at $r = R$, such as the "\NewTerm{interior Schwarzschild solution}\index{interior Schwarzschild solution}".

	An all (physically) apparent singularity appears when:
	
	Or in other words, when the coordinate of the radius $r$ is equal to:
	
	This radius, which we had already determined during our study of Classical Mechanics, is named the "\NewTerm{Schwarzschild radius}\index{Schwarzschild radius}".

	Therefore the Schwarzschild solution appears to have singularities at $r = 0$ and $r = 2GM/c^2$; some of the metric components "blow up" at these radii. Since the Schwarzschild metric is only expected to be valid for radii larger than the radius $R$ of the gravitating body, there is no problem as long as $R > 2GM/c^2$. For ordinary stars and planets this is always the case. For example, the radius of the Sun is approximately $700'000$ [km], while its Schwarzschild radius is only $3$ [km].
	
	The Schwarzschild radius is defined as the critical radius provided by the Schwarzschild geometry, below which nothing can escape: if a Star or other object reaches a radius equal to or less than its Schwarzschild radius Then it becomes a "\NewTerm{Black Hole}\index{Black Hole}", and any object approaching at a distance from it less than the Schwarzschild's ray will not escape from it. The term is used in physics and astronomy to give an order of magnitude of the characteristic size to which general relativity effects become necessary for the description of objects of a given mass. The only objects that are not Black Holes and whose size is of the same order as their Schwarzschild radius are neutron stars (or pulsars), thus, curiously, also the observable Universe as a whole...
	
	\begin{tcolorbox}[title=Remarks,colframe=black,arc=10pt]
	\textbf{R1.} The singularity in the metric when the Schwarzschild radius is reached is apparent because it is only an effect of the coordinate system used. It is an instance of what is named a "\NewTerm{coordinate singularity}\index{coordinate singularity}". As the name implies, the singularity arises from a bad choice of coordinates or coordinate conditions. When changing to a different coordinate system (for example Lemaitre coordinates, Eddington–Finkelstein coordinates, Kruskal–Szekeres coordinates, Novikov coordinates, or Gullstrand–Painlevé coordinates) the metric becomes regular at $r=2GM/c^2$\\
	
	\textbf{R2.} A remarkable theorem states that the Schwarzschild metric is the only solution to Einstein's equations in vacuum possessing spherical symmetry. As the Schwarzschild metric is also static, this shows that in fact in vacuum any spherical solution is automatically static. One interesting consequence of this theorem is that any pulsating star that remains spherically symmetrical can not generate gravitational waves (since the space-time region outside the star must remain static).\\
	
	\textbf{R3.} As the previous developments are based on the assumption of mathematical tools (Bianchi's identity) that requires a zero torsion tensor (\SeeChapter{see section Tensor Calculus}), there are more complete models that can not by extension use the Einstein equation of fields.
	\end{tcolorbox}	
	Now that we have the Schwarzschild metric we come back to the Schild criterion that we saw in our classical study of the Einstein effect.

	If we rewrite the Schwarzschild metric for an static body, we have the metric which is simplified into:
	
	By using the gravitational potential (\SeeChapter{see section Astronomy}):
	
	The metric is written:
	
	hence by introducing the proper time:
	
	hence:
	
	Therefore:
	
	Maclaurin's second-order expansion in series (\SeeChapter{see section Sequences and Series}) of the negative root gives:
	
	Therefore we have:
	
	Thus, this proof that the curvature (gravitation) generates a larger time dilation (in the sense that it flows faster) that the field of gravity is intense (mass $M$ is large) or that we are close to the body under the influence of the field (small radius $r$).

	For the Earth, the term:
	
	is relatively small. But for a Black Hole or a Neutron star, this is no longer the case and the dilation becomes important and the effects accessible to the measure.
	
	\subsection{Experimental Tests}	
	We will now review the $4$ classical experimental checks of the $20$th century of the General Relativity theory which are:
	\begin{enumerate}
		\item The precession of the perihelion which, in terms of numerical results, posed a problem for us with the tools of Classical Mechanics (\SeeChapter{see section Astronomy}).

		\item The deflection of electromagnetic waves (light) passing close to a massive stellar body which in the numerical results also posed a problem to us with the tools of Classical Mechanics (\SeeChapter{see section Astronomy}).

		\item The proof of the Schild criterion (already made in the preceding paragraphs) as the only way to explain rigorously the gravitational redshift and the hypothesis of slowing down time in a gravitational field.

		\item The delay of electromagnetic signals propagating near dense bodies. Delay referred to as "Shapiro effect" whose numerical applications are used for the operation of the G.P.S and which will be discussed later.
	\end{enumerate}
	\subsubsection{The Precession of Mercury's Perihelion}
	Let us now treat one of the most famous examples of General Relativity: the precession of the Mercury's perihelion. We had already dealt with this case in the Astronomy section, but we had mentioned that the theoretical numerical result did not correspond to the experimental observations. We shall see in the equivalent of almost ten A4 pages of detailed developments how General Relativity makes it possible to reconcile theory and experience.

	To study this case, we will use the Lagrangian formalism seen in the section of Analytical Mechanics.

	First, let us recall that we obtained for the metric of Schwarzschild:
	
	What we will write by dividing by $\mathrm{d}s^2$:
	
	And to abbreviate the notations, we put $l=GM/c^2$ such that:
	
	Now let us recall that (\SeeChapter{see section Analytical Mechanics}) in natural units:
	
	So (it's very rude but it works ... This is physics!...):
	
	Finally it means that the Lagrangian is:
	
	The equations of Lagrange give us for the $\theta$ coordinate \SeeChapter{see section Analytical Mechanics}):
	
	with therefore:
	
	Hence:
	
	and:
	
	From where finally for the coordinate $\theta$:
	
	Let us do the same for $\phi$. First, we have:
	
	and:
	
	And it comes immediately from the application of the Euler-Lagrange equation:
	
	Let us do the same for $t$:
	
	And it comes here also immediately:
	
	Therefore:
	
	Now let us assume that the motion of Mercury is in the equatorial plane such as $\theta=\pi/2$. Hence, the relation obtained above:
	
	simplifies into:
	
	hence:
	
	We have, therefore, the expression of the Universe line, which, for recall, is:
	
	Which since $\theta=\pi/2$ (which is therefore a constant) is simplified into:
	
	Let us now do the following replacement:
	
	Which is therefore a constant as we have proved just above and also the following replacement (which is also a constant as we proved just above):
	
	In the universe line element and we get:
	
	Let us consider also $r$ as a function of $\phi$ then:
	
	hence:
	
	Thus, we can rewrite the universe line in the form:
	
	Let us make a change of variable by putting:
	
	hence:
	
	Which gives for our universe line:
	
	or:
	
	By differentiating:
	
	Or written differently:
	
	Which simplifies and factorize itself into:
	
	The first possible solution is obviously:
	
	Hence as $r=1/u$:
	
	The circular motion is thus also a solution of Kepler's problem in general relativity in a Schwarzschild field (ouf!).

	The other solution will be:
	
	Or written differently:
	
	it corresponds to the orbit of Kepler's problem.

	Let us do the comparison by considering in Newton's mechanics the motion of a particle of mass $m$ in a potential $V$. The Lagrangian (\SeeChapter{see section Analytical Mechanics}) is then:
	
	In polar coordinates we have already seen in different section (Vector Calculus and Astronomy) that the speed is then written:
	
	Using the Euler-Lagrange equation we have the equation of motion:
	
	Which give:
	
	hence:
	
	And as we have seen in the section Astronomy:
	
	Is the constant of areas. Let us introduce:
	
	Hence:
	
	and therefore:
	
	So:
	
	The equation:
	
	therefore becomes:
	
	But:
	
	hence:
	
	therefore:
	
	where:
	
	It is therefore the "\NewTerm{non-relativistic Binet formula}\index{non-relativistic Binet formula}" which gives the relation between $u = 1 / r$ and $\phi$ for a central force (\SeeChapter{see section Astronomy}). In the case of a Newtonian potential:
	
	Hence:
	
	with for recall:
	
	Now let us recall the form of that which we had obtained just before with the General Relativity:
	
	Thus, we see that the analogous term in relativity is:
	
	and that general relativity adds the term $3lu^2$. Now, as in General Relativity:
	
	Then:
	
	However, in the case of the approximation of weak fields:
	
	hence:
	
	So finally:
	
	That said, it is really interesting to note that the equation for General Relativity:
	
	can be interpreted as Binet's equation for Classical Mechanics:
	
	with the potential:
	
	with $\gamma=lK^2$.
	
	Let us now return to our equation:
	
	We would like to know if the second term on the right of the equality is negligible or not with respect to the first term on the right of the equality in order to be able to apply the theory of perturbations.

	We will first put with the help of the weak field approximation given above:
	
	Now let us calculate the ratio:
	
	Recall that in polar coordinates (\SeeChapter{see section Vector Calculus}):
	
	In approximation, we can roughly put that:
	
	Therefore for Mercury ...:
	
	So we see immediately that we can apply the variational theories to the term $3lu^2$. Thus, let us put:
	
	The equation:
	
	takes the shape:
	
	To solve this differential equation, we will use the perturbation theory approach (\SeeChapter{see section Differential and Integral Calculus}). We will therefore focus on a solution of the form of a Taylor expansion in second order only in $\varepsilon$:
	
	where $u_0$, $u_1$ are obviously dependent on $\phi$ and will have to be determined! To do this, we know that we must replace the previous expression in the differential equation such that:
	
	Which simplifies into:
	
	where let us recall that:
	
	is the classical equation obtained earlier above:
	
	Let us consider the solution of the type:
	
	where $D$ is an arbitrary constant. Now, as we have seen in the section of Astronomy in the case of the precession of perihelion:
	
	is actually an ellipse. Which means that any solution of the form:
	
	is also an ellipse!
	
	For the equation in $\varepsilon$:
	
	which is simplifies into:
	
	Since (\SeeChapter{see section of Trigonometry}):
	
	It comes:
	
	To determine $u_1$, let us decompose it into three terms:
	
	This gives us immediately (by injecting the three terms respectively into the second derivative and the term alone):
	
	So finally:
	
	The solution sought is finally:
	
	It is therefore with:
	
	that it is necessary to calculate the displacement of the perihelion (we arrive soon... pfiuuuuu...).
	
	We see relatively quickly by observing the preceding relation that the only term whose amplitude is not constant is $\varepsilon D\phi\sin(\phi)$.

	Let us then recall that (\SeeChapter{see section Trigonometry}):
	
	This can also be roughly written as a first approximation using Maclaurin's first-order expansion (\SeeChapter{see section Sequencers and Series}):
	
	We know that the zero order orbit is:
	
	The effect of the last term:
	
	is therefore to introduce a small periodic variation in the radial distance. This term does not affect the displacement of the perihelion. This is the term $\varepsilon\phi$ in:
	
	which introduces a non-periodicity which can be non-negligible in the case where $\phi$ is large.
	
	The perihelion (the point closest to the Sun for recall) therefore appears when $r$ is the minimum therefore $u=1/r$ maximum. But, $u$ is maximum when the term which interests us is maximum, that is to say:
	
	We have approximately:
	
	For two successive perihelions, we have an interval:
	
	instead of $2\pi$. Thus, the displacement for a revolution is:
	
	where $K$ is therefore the constant of the areas and $M$ the mass of the central star and since:
	
	Finally we have in the end:
	
	Relation to be compared with that obtained in the section of Astronomy with a Classical Newtonian treatment:
	
	We thus fall back at the perfection on the factor $6$ which was lacking in the conventional treatments!

	For Mercury a numerical application gives:
	
	and the experiment gives $\delta\phi\cong 42.5''\pm 1.0''$. By Albert Einstein's own admission, in obtaining this result he had palpitations and the impression of grazing a heart attack and satisfied with his Herculean effort which had exhausted him he took a long period of rest.
	\begin{tcolorbox}[title=Remark,colframe=black,arc=10pt]
	It is perhaps useful for the reader to know that Albert Einstein and Michele Besso took almost $2$ years (!!!) by trials and errors to found the good result above. The first time they had an error of $4000\%$ in comparison to the experimental observed value, the second time an error of $400\%$ and finally the value above (after that Albert Einstein had identified that he choose the wrong Tensor for his theory).
	\end{tcolorbox}
	To conclude on this subject, let us mention a second frequent writing in the literature concerning the result obtained. Indeed, we have proved in the section of Astronomy that the focal parameter was given by:
	
	It therefore remains:
	
	and we have also proved in the section of Analytical Geometry that:
	
	It thus comes in the end the most classic form:
	
	
	\subsubsection{Deflexion of Light}
	We have just proved that:
	
	By replacing the factors by their respective values, we have:
	
	But we have seen above that:
	
	and as $K$ is the areas constant given by the conservation of the momentum itself constant (\SeeChapter{see section Classical Mechanics}):
	
	We then have for a photon $m\rightarrow 0\Rightarrow K\rightarrow +\infty$.
	
	Let us put now to simplify the notations:
	
	Then:
	
	The term to the right of the equality is small (considering the constants that intervene therein) so that an approximate form of the differential equation is:
	
	of which a particular solution, which we know in advance, is interesting:
	
	We carry this approximated solution in the initial differential equation and we get:
	
	Therefore:
	
	Hence:
	
	What follows is going to be very subtle (how to guess something like that ...?). First we will create a new differential equation:
	
	The trick is to multiply this equation by $\mathrm{i}$ and sum it to the original differential equation:
	
	What we will denote by:
	
	Another trick is to look for a particular solution of the previous relation in the form:
	
	Then we have:
	
	This injected into our new differential equation gives:
	
	We deduce immediately:
	
	A particular solution of the original differential equation is thus:
	
	Either by using the remarkable trigonometric relations (\SeeChapter{see section Trigonometry}):
	
	It comes:
	
	The general solution is:
	
	If we admit that the light is very weakly deviated by the Sun, the radius of curvature ($1/r$) of its trajectory will be very small.

	Therefore:
	
	such that:
	
	The first term is predominant relatively to the second because of the factor $r_g$ that is very small on the second. For what will follows, we will proceed as in the in the section Astronomy (only the notations change) for the study of the deflexion angle (if you don't come back to it, it can be difficult to understand the justification of what will follow!). We put without loosing in generality:
	
	Therefore:
	
	and as:
	
	it comes:
	
	Using trigonometric identities again:
	
	It comes:
	
	$\theta$ being supposed as very small we do a Maclaurin development (\SeeChapter{see section Sequences and Series}) to the first order of the trigonometric functions:
	
	Which gives:
	
	Therefore after a series of approximation... and of hypothesis at the limit of what is acceptable..., we get ($R$ is sometimes named the "\NewTerm{impact parameter}, that is to say he distance of nearest approach of the light-beam to the center of mass):
	
	instead of the result that we get following the Newtonian approach in the section Astronomy:
	
	We thus founded the factor $2$ that was missing in the classical treatment, relatively to experimental observations, that we have proved in the section of Astronomy:
	
	What is often pictured in the media by the following drawing:
	\begin{figure}[H]
		\centering
		\includegraphics[scale=1]{img/cosmology/light_deflexion.jpg}	
	\end{figure}
	This deviation have been observed experimentelly by measuring the position of stars in the vicinity of the solar disk during the 1919 eclipse by Arthur Eddington and his team. After the advance of the perihelion of Mercury, this was the second test successfully passed by the General Relativity. It was this event that made Albert Einstein famous among the general public. Today, the deviation of light rays can be measured with much greater precision by considering radio signals emitted by extragalactic sources (quasars, AGN, etc.): the prediction of the General Relativity has been confirmed to the nearest thousandth.

	The deviation of light rays is today very important in observational cosmology,
Since it is at the origin of the phenomenon of gravitational mirage, also named "gravitational lens".

	It is interesting to notice that the whole theory of gravitational mirages is based on the relation:
	
	at least for a point detector. It is the only ingredient of General Relativity used in the calculation of images.
	
	In observational astronomy an "\NewTerm{Einstein ring}\index{Einstein ring}", also known as an "\NewTerm{Einstein-Chwolson}\index{Einstein-Chwolson}" ring or "\NewTerm{Chwolson ring}\index{Chwolson ring}", is the deformation of the light from a source (such as a galaxy or star) into a ring through gravitational lensing of the source's light by an object with an extremely large mass (such as another galaxy or a Black Hole). This occurs when the source, lens, and observer are all aligned.
	\begin{figure}[H]
		\centering
		\includegraphics[scale=0.55]{img/cosmology/einstein_ring_lrg_3_757.jpg}	
		\caption{Einstein Ring LRG 3 757}
	\end{figure}
	
	\subsubsection{Shapiro Effect (delay)}
	In 1964,  Irwin Shapiro demonstrated that a ray of light was not only deflected by passing near a mass, but also that the duration of its path was lengthened in relation to a Euclidean geometry. He calculated that the delay should be about $200$ microseconds, therefore perfectly measurable, for a line of sight shaving the Sun. He then suggested systematically measuring the time taken by a radar signal to make the round trip between the Earth and a planet passing behind the Sun (so that the effect is maximal). This was first accomplished with radar echoes on Mars, Venus or Mercury, with an accuracy of the order of $20\%$. The result was very clear: the time required for a radar signal to make the go and come back between the Earth and the other Planet increases suddenly just before the planet passes behind the Sun and decreases just as suddenly when it reappears.
	\begin{tcolorbox}[title=Remark,colframe=black,arc=10pt]
	We sometimes also talk of "slowing down of the light" near the Sun to describe the Shapiro effect, but it is an awkward and erroneous expression. As we have already been mention it, the speed of light is constant in General Relativity as well as in Relativity for all observers (but for recall this doesn't mean that the speed of light in constant during the life of our Universe!). In the case of the Shapiro effect (and in other similar cases), what changes is the flow of time where the light passes, in relation to what it is where the observer is located.
	\end{tcolorbox}
	Although this is a weak effect, it has been verified precisely since the arrival of the Viking probes on Mars in 1976, using signals sent from Earth to Mars and reflecting on the latter by the probes (see the principle of the experiment in the figure further below). In addition, there is now even an increasingly common object for which the Shapiro effect must be taken into account: the "G.P.S." (Global Positioning System). Indeed, despite the weakness of the field of gravitation, a geographical precision of a few meters requires such details in the calculation! However, a satellite has recently been launched to verify in the Earth's gravitational field an even lower effect predicted by General Relativity and which does not even intervene in GPS: the drag over of space also known as the "\NewTerm{Lense-Thirring effect}" due to the rotation of Earth.
	
	Let us point out for the GPS that two phenomena of error are known within the framework of the Relativity:
	\begin{enumerate}
		\item The satellites rotating around the Earth at a speed of approximately $20,000$ kilometers per hour then delay $7$ millionths of a second per day (Relativity).

		\item At an altitude of $20,200$ kilometers, that of the satellite orbit, the lower gravitational field advances the satellite clocks by $45$ millionths of a second per day.
	\end{enumerate}
	The sum of the two corrections gives a drift of $38$ millionths of a second per day, a staggering figure for a GPS system whose precision must be $50$ billionths of a second per day!!!

	Let us make the calculation for a ray touching the surface of the Sun. For this, we take up our Schwarzschild's metric given for recall by:
	
	with:
	
	For a photon, we know that $\mathrm{d}s=0$ and therefore the equation of the Schwarzschild's metric is then written:
	
	The trajectory of the photon taking place in the equatorial plane of the Sun, we put:
	
	which simplifies even more the equation of the metric by:
	
	To simplify even more, we make the hypothesis that the trajectory (in polar coordinates) of the photon shaving the Sun is rectilinear such that (for one of the polar components of the plane):
	
	where $r_\odot$ is the ray of the Sun. We will use this assumption to simplify the equation of the metric. For this we rearrange:
	
	We derive (\SeeChapter{see section Differential and Integral Calculus}):
	
	If we square everything:
	
	hence:
	
	We can now rewrite the equation of the metric:
	
	Taking the square root:
	
	Since $r>r_\odot$ and $r_g\ll 0$ then:
	
	Therefore, we have using the Maclaurin developments (\SeeChapter{see section Sequences and Series}) to the first order:
	
	We have then:
	
	Finally, we get once condensed:
	
	What it is traditional to write (we take out the $1 / c$ of the different terms):
	
	If there is no mass then space-time is flat and $r_g=0$. Therefore:
	
	We can thus distinguish the classical time from the extra time generated by the curved space. The "delay" will therefore be given by:
	
	Then, to integrate the four functions of $r$, we must place ourselves in a repository placed if possible at the center of the main body (the Sun typically) since the Schwarzschild metric is based on this hypothesis for recall. Thus, to know the delay of a luminous ray starting from the Sun and travelling to the Earth, we logically choose as the radius of departure that of the Sun itself and as the radius of arrival, the distance Sun-Earth (this will correspond once the primitives computed at the integration terminals).
	\begin{figure}[H]
		\centering
		\includegraphics[scale=1]{img/cosmology/shapiro_effect.jpg}	
		\caption{Round-trip time of a signal as a function of the position of Mars}
	\end{figure}
	Well that says it's nice to know the notations of use, but it's even better to do a numerical application! We will therefore first determine the primitive of each of the terms below:
	
	The first two primitives are simple because they are usual primitives proved in detail in the section of Differential and Integral Calculus:
	
	where for the last primitive we have preserved the constant of integration (contrary to what was done in the section of Differential and Integral Calculus because $r_\odot\neq 1$).

	Now it remains to us the last two integrals. Let's start in the order by:
	
	By putting:
	
	and using the results proved in the section of Differential and Integral Calculus, we then have:
	
	Since we have (\SeeChapter{see section Trigonometry}):
	
	Then:
	
	Finally, it remains the last primitive:
	
	We put for what will follow:
	
	Therefore it comes:
	
	In the section of Differential and Integral Calculus we have proved that:
	
	and that:
	
	Therefore:
	
	To return to the integral of the beginning we remember that $r=\dfrac{r_\odot}{x}$. Therefore:
	
	We thus finally have by taking all the primitives calculated above and by choosing a starting and finishing terminal for the calculation:
	
	We see in the Newtonian limit case where $r_g=0$ that this relation is reduced to:
	
	So for a round trip (between planet and satellite for example), then it comes in this simplified case:
	
	In November 1976, when the two Viking spacecraft were operating on the surface of Mars, the planet went
behind the Sun as seen from Earth (see figure below). Scientists had preprogrammed Viking to send a radio wave toward Earth that would go extremely close to the outer regions of the Sun. According to General Relativity there would be a delay because the radio wave would be passing through a region where time ran more slowly. The experiment was able to confirm Einstein’s theory to within $0.1\%$.
	\begin{figure}[H]
		\centering
		\includegraphics[scale=0.7]{img/cosmology/shapiro_effect_viking.jpg}	
		\caption{Radio signals from the Viking lander on Mars were delayed when they passed near the Sun (source: OpenStax}
	\end{figure}
	
	\pagebreak
	\subsubsection{Black Holes}
	Always staying with at our Schwarzschild metric .... a radial trajectory of light-type implies:
	
	therefore:
	
	and in a direct radial trajectory (by definition!) we also have:
	
	therefore:
	
	Therefore:
	
	hence:
	
	so that:
	
	Let us change to natural units $c=1$. It then comes:
	
	When $r\rightarrow 2GM$ the right-hand side of this equality tends to $\pm \infty$, then the evolution of time $t$ (external observer) as a function of $r$ tends to infinity with respect to the proper time of light.

	The sphere given by the radius:
	
	defines the "\NewTerm{horizon}\index{horizon}" of a "\NewTerm{Schwarzschild Black Hole}\index{Black Hole}".
	
	Towards this limit boundary, the light seems to put an infinite time compared to an external observer to move when approaching a Black Hole. It therefore never really reaches it in relation to the observer, hence the fact that the Black Holes can be surrounded according to their environment by a luminous halo near the Schwarzschild radius. Moreover, since time seems to be stopped, the frequency of the light surrounding the Black Hole tends towards zero and therefore towards the infrared.

	A Black Hole is therefore a region of spacetime exhibiting such strong gravitational effects that nothing—not even particles and electromagnetic radiation such as light—can escape from inside it. The theory of general relativity predicts that a sufficiently compact mass can deform spacetime to form a black hole. The boundary of the region from which no escape is possible is named the "event horizon". 

	Objects whose gravitational fields are too strong for light to escape were first considered in the 18th century by John Michell and Pierre-Simon Laplace. The first modern solution of general relativity that would characterize a black hole was found by Karl Schwarzschild in 1916, although its interpretation as a region of space from which nothing can escape was first published by David Finkelstein in 1958. Black Holes were long considered a mathematical curiosity; it was during the 1960s that theoretical work showed they were a generic prediction of general relativity. The discovery of neutron stars sparked interest in gravitationally collapsed compact objects as a possible astrophysical reality.

	Black Holes of stellar mass are expected to form when very massive stars collapse at the end of their life cycle. After a Black Hole has formed, it can continue to grow by absorbing mass from its surroundings. By absorbing other stars and merging with other Black Holes, supermassive Black Holes of millions of solar masses may form. There is general consensus that supermassive Black Holes exist in the centers of most galaxies.

	Despite its invisible interior, the presence of a Black Hole can be inferred through its interaction with other matter and with electromagnetic radiation such as visible light. Matter that falls onto a black hole can form an external accretion disk heated by friction, forming some of the brightest objects in the universe. If there are other stars orbiting a black hole, their orbits can be used to determine the black hole's mass and location. Such observations can be used to exclude possible alternatives such as neutron stars. In this way, astronomers have identified numerous stellar black hole candidates in binary systems, and established that the radio source known as Sagittarius A*, at the core of our own Milky Way galaxy, contains a supermassive black hole of about $4.3$ million solar masses. 
	\begin{tcolorbox}[title=Remark,colframe=black,arc=10pt]
	If the Sun or any other celestial object collapse into a Black Hole, this doesn't affect its orbiting element as the Black Holes still have the same mass (only the density change) and Newton's law still remains valid at a quite significant distant of it. So it is wrong to imagine that if the Sun collapse into a Black Hole of $6$ [km] diameter it would exert a more big gravitational force and that all planets turning around it will be sucked in.
	\end{tcolorbox}
	
	On 11 February 2016, the LIGO collaboration announced the first observation of gravitational waves; because these waves were generated from a Black Hole merger it was the first ever direct detection of a binary Black Hole merger. On 15 June 2016, a second detection of a gravitational wave event from colliding black holes was announced.
	\begin{figure}[H]
		\centering
		\includegraphics[scale=0.6]{img/cosmology/black_hole.jpg}	
		\caption{2D naive representation of space curve near some celestial objects (source: OpenStax)}
	\end{figure}
	The Galactic Center Group members have been measuring the positions of thousands of stars in the vicinity of the Galactic Center for more than $20$ years. This unique data set allowed us to measure directly short-period orbits of stars. In particular, a full phase coverage has been measured for two stars: S0-2 with an orbital period of $15.56$ years, and S0-102 with $11.5$ years. At the closest approach, S0-2 is only $17$ light hours away from the center of the Galaxy, about four times the distance of Neptune from the Sun. From these orbital data, we can determine the mass of the central black hole in our own Galaxy.

	The Milky Way Galaxy center is the best candidate to what seems to a black hole, and especially and example of the closest supermassive black holes (SMBH), located at $\sim 25,000$ light years away from us and corresponds with the location of Sagittarius A* a bright and very compact astronomical radio source at the center of the Milky Way. Its mass is estimated to be $4$ million times the mass of the Sun, which implies that the Schwarzschild radius is about $17$ times that of Sun's radius. As a comparison, Mercury's orbit is located at a distance of $\sim 83$ solar radii. Because the Galactic Center is the site of the closest supermassive black hole by a factor of $100$, it is a unique laboratory for solving some of the greatest mysteries associated with the fundamental physics of supermassive black holes and the role that they play in the formation and evolution of galaxies. Furthermore, it is the only galactic nucleus in which direct measurements of stellar orbits is possible, with either the current or the next-generation instruments.
	
	Observations of stellar orbits around the Galactic black hole also yields precision measurement of the distance to the Galactic Center, which is important as it affects almost all questions not only of Galactic structure, dynamics and mass, but those of extragalactic distance scales and the value of Hubble's constant as well.
	\begin{figure}[H]
		\centering
		\includegraphics[scale=1]{img/cosmology/galactic_black_hole_center.jpg}	
		\caption{The orbits of stars within the central 1.0 X 1.0 arcseconds of our Galaxy (source: Galactic Group Center)}
	\end{figure}
	The origin of supermassive black holes  and also the assumption that each galaxy center is a black hole remains an open field of research. Astrophysicists agree that once a black hole is in place in the center of a galaxy, it can grow by accretion of matter and by merging with other black holes this is why some observations gives for example for the hyperluminous quasar S5 0014+81 a weight of approximately $40\cdot 10^10$ times the mass of the Sun.
	
	Are there any other magnitudes we should note or calculate in Black Hole physics and thermodynamics! Yes, there are. Assuming spherical symmetry, we can calculate the Schwarzschild area or event horizon/surface area of the Schwarzschild's Black Hole simple by:
   
	We can also calculate the surface gravity $\kappa$, if the gravitational field of the Black Hole reads
	
	then, at the Schwarzschild radius it becomes the mentioned surface gravity $\kappa=g(R=R_S)$:
	
	Interestingly, this surface gravity is $1/M$ the maximal force $c^4/4G$ allowed by natural units... What else? Surface tides, or more precisely, the tidal acceleration at the Black Hole surface. The tidal acceleration is calculated with (the reader notice that we take the version of the tidal force with the factor $2$ instead of factor $4$ as discusses during the proof of this relation in the section of Astronomy):
	
	If it is evaluated at $R_S$ we get:
	
	
	\pagebreak
	\subsubsection{Gravitational waves}
	Before we focus on the maths, let us do a simple introduction.
	
	"\NewTerm{Gravitational waves}\index{gravitational waves}" are ripples in the curvature of space-time that propagate as waves at the speed of light, generated in certain gravitational interactions that propagate outward from their source. The possibility of gravitational waves was discussed in 1893 by Oliver Heaviside using the analogy between the inverse-square law in gravitation and electricity. In 1905 Henri Poincaré first proposed gravitational waves emanating from a body and propagating at the speed of light as being required by the Lorentz transformations. Predicted in 1916 by Albert Einstein on the basis of his theory of General Relativity, gravitational waves transport energy as gravitational radiation, a form of radiant energy similar to electromagnetic radiation. Gravitational waves cannot exist in the Newton's law of universal gravitation, since that law is predicated on the assumption that physical interactions propagate at infinite speed.
	
	Gravitational-wave astronomy is an emerging branch of observational astronomy which aims to use gravitational waves to collect observational data about sources of detectable gravitational waves such as binary star systems composed of white dwarfs, neutron stars, and Black Holes; and events such as supernovae, and the formation of the early universe shortly after the Big Bang.
	
	On February 11, 2016, the LIGO Scientific Collaboration and Virgo Collaboration teams announced that they had made the first observation of gravitational waves, originating from a pair of merging black holes using the Advanced LIGO detectors. On June 15, 2016, a second detection of gravitational waves from coalescing Black Holes was announced. Besides LIGO, many other gravitational-wave observatories (detectors) are under construction.
	\begin{figure}[H]
		\centering
		\includegraphics[scale=0.75]{img/cosmology/ligo_measurements.jpg}	
		\caption{LIGO measurement of the gravitational waves at the Hanford (left) and Livingston (right) detectors}
	\end{figure}
	\begin{figure}[H]
		\centering
		\includegraphics[scale=0.75]{img/cosmology/ligo.jpg}
	\end{figure}
	In real life it looks like this:
	\begin{figure}[H]
		\centering
		\includegraphics[scale=0.95]{img/cosmology/ligo_washington.jpg}
		\caption[Hanford LIGO detector Washington state (source: LIGO)]{Hanford LIGO detector - October 30, 2000 - Washington state (source: LIGO)}
	\end{figure}
	The most typical illustration that we can see of gravitational wave of newspapers are that generated by the special case of a high speed dynamic gravitation field due to the rotation of two massive object around each other. Otherwise, and it is obvious, a collapsing Star at the end of its life also generated a variational gravitational field but that is quite difficult to detect with actual existing instruments as a star alone is not massive enough to generate detectable gravitation waves:
	\begin{figure}[H]
		\centering
		\includegraphics[scale=0.8]{img/cosmology/gravitational_wave.jpg}	
		\caption{Pseudo-3D (wrong) common visualization of gravitational wave of a binary system}
	\end{figure}
	But obviously gravitational waves don't make things go up and down like ocean waves as illustrated above, and they're definitely not like that planet on a trampoline — after all, there's nothing "below" to pull things downward so there can't be a dent.  And gravitational waves don't do spirals, much....
	
	In a gravitational wave, space itself is compressed and stretched.  A particle caught in a gravitational wave doesn't get pushed back and forth.  Instead, it shrinks and expands in place.  If you encounter a gravitational wave, you and all your calibrated measurement gear (yardsticks, digital rangers, that slide rule you’re so proud of) shrink and expand together.  You would only notice the experience if you happened to be comparing two extremely precise laser rangers set perpendicular to each other (LIGO!).  One would briefly register a slight change compared to the other one.
	
	
	Now let us deal with the maths. As almost always in science there are multiple ways to introduce a new tool. In this book, as most of times, we will use what we consider the most easy one and that is "physicist" or "engineer" oriented...
	
	So first, remember that earlier we have proved under some assumptions that:
	
	That what we have seen afterwards can be written more generally as:
	
	If we explicit it by keeping in mind Classical Mechanics it would be written in 3D:
	
	or a bit better:
	
	But... but...! We are in General Relativity and we have to introduce the 4-th dimensions:
	
	This is better but now let u explicit this using in cartesian coordinates using the metric $(- + + +)$. We then get:
	
	Assuming that we a observing a piece (volume) of space where there is no matter and no radiation in any form, then $T_{\mu\nu}=0$ and we get:
	
	Using the d'Alembertian already introduced in the section of Electrodynamics but now in its General Relativity form named "\NewTerm{flat-space d'Alembertian}\index{flat-space d'Alembertian}":
	
	that latter relation is commonly written in textbooks:
	
	Taking back and rearranging the explicit relation we get (we see that this equation also give us that space transmits gravitational waves at the speed of light!):
	
	and named the "\NewTerm{gravitational wave equation}\index{gravitational wave equation}", "\NewTerm{gravitational propagation equation}\index{gravitational propagation equation}" or "\NewTerm{gravitational d'Alembert's equation}\index{gravitatgional d'Alembert's equation}".
	
	As gravitational waves pass through boundaries that light cannot. They can transport information about what happens inside the event horizons of Black Holes, and they can pass through the cosmic background (CMB) radiation (\SeeChapter{see section Cosmology}), the barrier of light that prevent us from seeing our Universe before it turned $380,000$ years old!
	\begin{figure}[H]
		\centering
		\includegraphics[scale=0.8]{img/cosmology/gravitational_wave_cmb.jpg}	
		\caption{Gravitational cosmic background "radiation"}
	\end{figure}
	
	So for summary General Relativity has successfully with high accuracy pass the following experimental tests in order of verification (top oldest):
	\begin{enumerate}
		\item Mercury perhelie precession
		\item Light deviation
		\item Black Holes
		\item Universe expansion
		\item Time Dilatation
		\item Gravitational waves
	\end{enumerate}	

	Let us point out once again a very important point. Before Albert Einstein, geometry was considered an integral part of the laws. Albert Einstein has shown that the geometry of space evolves in time according to other, even deeper, laws. It is important to understand this point. The geometry of space is not part of the laws of nature (this was criticized at his time by many French physicists). Therefore, nothing that we can find in these laws tells us what geometry of space we are working in. Thus, before we begin to solve the equations of Einstein's General Theory of relativity, we have absolutely no idea what geometry we ear dealing with. We only discover it once the equations are solved!
	
	In extenso, the choice of $4$ dimensions is part of the background. Could it be possible that another deeper theory does not require presupposing the number of dimensions? 

	To sum up, the idea of independence relatively to the background, in its most general formulation, is a wise way of making physics: made up of better theories, in which the things which before were postulated, will be explained in allowing such things to evolve over time according to new laws.

	This is also a difficulty of Quantum Theory, that latter is essentially background dependent at the opposite of General Relativity that is "\NewTerm{background independent}\index{background independent}".
	
	\begin{flushright}
	\begin{tabular}{l c}
	\circled{90} & \pbox{20cm}{\score{3}{5} \\ {\tiny 33 votes,  61.82\%}} 
	\end{tabular} 
	\end{flushright}
	
	%to force start on odd page
	\newpage
	\thispagestyle{empty}
	\mbox{}
	\section{Cosmology}
	\lettrine[lines=4]{\color{BrickRed}C}osmology is concerned with understanding the birth and evolution of the Universe by the scientific method. It is only through this game between physical theories, models and observations that we will discuss this issue here. We will try to avoid carefully any metaphysical digression. The specific problems of cosmology fit in its definition: Statistics that is one of the great scientific methods is apparently poor: we only have one Universe visible to us at this day. Furthermore, we can only observe the past of our Universe. Can we speak about "predictions" in these conditions? The theories, however, are reliable since they predict behaviors that can be tested by observations.
	
	Cosmology mainly uses the arsenal of mathematics, theoretical physics, particle physics, nuclear physics, physics of detectors and astrophysics. It is interdisciplinary! Cosmology deals with scales larger than the size of a galaxy to the scales defined as itself as "Horizon". Even if the limit is deliberately vague, cosmology does not address the internal details of the birth and evolution of astrophysical objects (such as galaxies, globular clusters, and clusters of galaxies) that fall more into the study field of "\NewTerm{cosmogony}\index{cosmogony}".
	
	\begin{fquote}[]If you can explain a God without a creator, you can explain a Universe without a creator...
 	\end{fquote}
	
	\subsection{Newtonian Cosmological Model}
	A cosmological model is a mathematical representation of the Universe that seeks to explain the reasons for its present appearance, and describe its evolution over time (named "\NewTerm{cosmological time}\index{cosmological time}") but not its creation!
	
	The Newtonian model applies under the assumptions of Newtonian mechanics (instantaneous action for example). The results we are going to study here were discovered before the development of General Relativity but published after! But this model has the advantage of simplicity while being able to identify and discuss the dynamics of the Universe and to prepare for the study of the Universe models making use of the results of General Relativity afterwards. Its disadvantage, besides the fact that it is not quite fit the experimental results, it is to beno longer valid under extreme conditions and therefore cannot be extrapolated to the instant of the Big Bang.
	
	Before we begin, we must define the "\NewTerm{cosmological principle}\index{cosmological principle}" consisting of the following two assumptions (basically, it ensures that we are not privileged observers, and that what we are seeing is a representative of the whole of the Universe) :
	
	\pagebreak
	\begin{itemize}
		\item[H1.] The space (the Universe) is homogeneous, that is to say, it has the same properties in all its regions. This must be at a very large scale, beyond the thousand Mpc (megaparsecs). It is clear that small-scale inhomogeneities exist... we for example... \Winkey
		
		\item[H2.] The space (Universe) is isotropic, ie there is no specific direction in space, such as flattening a direction or an overall movement on a Universal scale for example and all its physical properties are almost identical in any point.
	\end{itemize}
	
	\begin{tcolorbox}[title=Remark,colframe=black,arc=10pt]
	This hypothesis of the isotropy of the Universe and that works relatively well in theoretical models (see below) requires an interesting fact if we admit a beginning to the Universe. This fact implies that the Universe had a phase in its history where it did not leave the time to the matter to clump together to form from its beginnings inhomogeneous and anisotropic material groups that are visible today in our telescopes. From this it follows that at a moment of its history, the Universe had an non-quasistatique expansion rate that we could make match with the speed of light (this is badly formulated but I hope it is still acceptable in the idea).
	\end{tcolorbox}
	The figure below shows the Automated Plate Measurement (APM) Galaxy Survey. Over $2$ million galaxies are depicted in a region of $100$
degrees across centered toward the Milky Way's south pole:
	\begin{figure}[H]
		\begin{center}
		\includegraphics[scale=0.55]{img/cosmology/apm_galaxy_survey.jpg}
		\end{center}	
		\caption{Automated Plate Measurement (APM) Galaxy Survey}
	\end{figure}
	\pagebreak
	We will also request other working assumptions/hypothesis:
	\begin{itemize}
		\item[H1.] The Universe is a non viscous gaseous fluid whose particles are galaxies. Assuming the cosmological principle, the movement of galaxies, constituents in-fine of this "fluid" by construction, are statistically at rest.
		
		\item[H2.] The Universe is a thermodynamically closed system, without work and adiabatic (no heat exchange with the outside).
		
		\item[H3.] The Universe in a homothetic expansion (in proportional expansion in all its dimensions) is taken as having a spherical shape with a center (yes this is the Newtonian model...).
		
		\item[H4.] Its density is only a function of time and there is mass conservation (and therefore energy). Therefore the amount of material is constant (yes this is always the Newtonian model...)!
		
		\item[H5.] We accept the Newtonanian dynamics (approximation of General Relativity) to build the Newtonian models that will follow.
		
		\item[H6.] The origin of time is treated as the origin of creation (horizon) of the universe and repository of study is comoving with the particles (and therefore moves with the galaxies placed on the space-time pattern) and named "\NewTerm{reference material}\index{reference material}" (galaxies are stationary in this repository!).
	\end{itemize}

	\subsubsection{Hubble's Law}
	Assuming the cosmological principle and the above assumptions, the distance from an origin point O to any point $M$ of the universe can vary in function of time (in a undetectable way to the human scale) in the form:
	
	where $F(t)$ is the "\NewTerm{scaling factor}\index{scaling factor}" (denoted by $R(t)$ depending on the context...).
	
	In writing this relation, we consider that the points O and $M$ are on a plane of zero curvature. Indeed, if we imagine two points on a circular curved surface (e.g. the surface of a sphere) let us see what happens:
	\begin{figure}[H]
		\begin{center}
		\includegraphics{img/cosmology/newtonian_universe_model.jpg}
		\end{center}	
		\caption{Illustration of the validity limit of the model}
	\end{figure}
	The distance between two points on the circle (i.e. spherical space) is given by (\SeeChapter{see section Trigonometry}):
	
	We see very well in this relation that if the radius (of the spherical Universe) changes by a factor $F$, then the change in the distance between the two points is not linearly proportional to this factor!! Which is not the case in a zero curvature plane!
	
	Consequence: Our Newtonian model is valid only in a flat universe where General Relativity or purely classical energy approach (see further below) can take into account different types of curvature!
	
	We see immediately that the relation:
	
	is independent of the chosen origin, in fact, if we apply it on any two points $A, B$ we have:
	
	Then by difference:
	
	\begin{tcolorbox}[title=Remarks,colframe=black,arc=10pt]
	At the time $t_0$ it is obvious that the above equation is:
	
	and imposes $F(t_0)=F(0)=1$. This is important and we will come back on it many times during the developments that will follow.
	\end{tcolorbox}
	The law above therefore applies to any segment $\overline{AB}$ in the Universe. This is why the Universe has no geometrical center (at least as far as we know) and that we can represent ourselves the expansion of the frame of the Universe: consider a half-inflated balloon on whose surface we draw two marks (eg: two crosses drawn with ink). Inflating the balloon more we find that these two cross diverge from each other and therefore the distance between them will increase. This is what we are seeing with the galaxies:
	\begin{figure}[H]
		\begin{center}
		\includegraphics{img/cosmology/inflated_universe.jpg}
		\end{center}	
		\caption{Illustration of the inflated balloon Universe model}
	\end{figure}
	
	Deriving with respect to time the relation:
	
	The first member then gives the particle velocity (or other any object) to the point $\vec{r}(t)=\overrightarrow{\text{O}M}(t)$:
	
	Therefore eliminating $\overrightarrow{\text{O}M}(t_0)$:
	
	We put to simplify the notations:
	
	Therefore we have:
	
	This relation is known as the "\NewTerm{Hubble law}\index{Hubble law}" (and which, according to historical research should have for paternity rather Georges Lemaître...).
	
	Before going further, it is necessary to pause on this equation for the present moment:
	
	This equation says that the object of the Universe recede with a speed proportional to their distance in all points of the Universe without special repository (no galaxy seems to be fixed while they are in the material repository!).
	\begin{tcolorbox}[title=Remark,colframe=black,arc=10pt]
	This relation allows for speeds higher than those of light... But this is not a violation of Special Relativity regarding the constancy of the speed of light! Indeed, we must not forget that the Hubble law takes into account the expansion of the "frame" of space-time on which light moves. Also Special Relativity deal with speed of light and not with speed of space itself. Therefore if the frame extends along an expansion factor $F$ greater than one, it gives the impression that light travels faster than $c$, and this is what gives the sometimes a redshift of $4$ or $5$!
	\end{tcolorbox}
	The constant $H(t_0)=H_0$ being of course being identifiable to the "\NewTerm{Hubble constant}\index{Hubble constant}" as currently measured in the early 2000s as being about: $\sim 75\; [\text{km s}^{-1} \text{Mpc}^{-1}]$.
	
	In IS units, since one megaparsec is almost equal to $\sim 3.085\cdot 10^{22}\; [\text{m}]$ then we have:
	
	Thus, a current estimate of the age (horizon) of the Universe could be interpreted as the inverse of the Hubble constant that gives the "\NewTerm{Hubble time}\index{Hubble time}":
	
	that is to say about 13 billion years ago (we will further below a better approach).
	
	Conversely, we can have fun to calculate the distance from which we can reach the speed of light $299,792/75\cong 3,997\;[\text{Mpc}]$ thanks to the equation:
	
	and a numerical application gives roughly $13$ billion light years. This is the distance of the "\NewTerm{cosmological horizon}\index{cosmological horizon}".
	
	In cosmology, a "\NewTerm{Hubble volume}\index{Hubble volume}, or "\NewTerm{Hubble sphere}\index{Hubble sphere}", is a spherical region of the Universe surrounding an observer beyond which objects recede from that observer at a rate greater than the speed of light due to the expansion of the Universe. Regarding the relation we get above, the corresponding radius today is given by $r_{\text{HS}}(t_0)=c/H_0$ and almost equal to $13$ billion light years as mentioned just earlier above. As observations seems to indicate that our Universe is accelerating, so that some objects that we can currently exchange signals with will one day cross our Hubble limit.
	
	\subsection{Friedmann Equations}
	Consider now a spherical ring of material of radius $r$ and of constant mass $m$ expanding at velocity $v$, and containing a ball of material of mass $M$ (also in expansion at speed $v$).
	
	where $k_1$ is a constant. By dividing by $m$ and replacing each member $M$ by its expression as a function of the density, we obtain:
	
	\begin{tcolorbox}[title=Remark,colframe=black,arc=10pt]
	If it can help the reader to understand what we did with the term of the potential energy, it can refer to section Classical Mechanics when we developed the calculations of the potential energy of a material sphere.
	\end{tcolorbox}
	
	and:
	
	We get:
	
	That we simplify in:
	
	However, $k_1,m,r_0$ are constants. We introduce a new constant $k$ defined by (to simplify the notations):
	
	So we get the equation:
	
	which is none other than the "\NewTerm{Friedmann's first equation}\index{Friedmann's first equation}" that we frequently find in the literature as follows (among others notations...):
	
	It is possible to obtain the same equation from Einstein's Field equations (\SeeChapter{see section General Relativity}) and the metric of Friedmann-Robertson-Walker (see further below).
	
	Let us still notice a very common form of that relation. When using Einstein's mass and energy equivalence ($E=mc^2$) the above density $\rho$ is no longer a mass density but an energy density, we will have to divid it again by the squared speed of light to have a mass density again. It is the same if at the denominator of the constant $k$, the mass $m$ is replaced by energy, then we will have to multiply $k$ by the squared speed of light to fall back on a masse We then noting the radius $a$ (as is often customary in the literature) and redistributing the terms, the following form of the first Friedmann equation:
	
	where $a$ is also known as the "\NewTerm{cosmic scale factor}\index{cosmic scale factor}" or sometimes the "\NewTerm{Robertson-Walker scale factor}\index{Robertson-Walker scale factor}".
	
	\pagebreak
	\begin{tcolorbox}[title=Remark,colframe=black,arc=10pt]
	Albert Einstein added to this equation for personal and quasi-religious beliefs a cosmological constant equation that allowed him to make static the scale factor of the Universe. I (as the main redactor of the book) reject this arbitrary constant (at least until this date), even if in the contemporary physics, it has become a trendy constant (its value was defined more mathematically rather than religiously) because it would explain the origin of a supposed "dark matter" that seem to accelerate the expansion of our Universe (this acceleration could also be a "local" departure of the Hubble constant from its globally averaged value perhaps caused by a local void in the mass density of our space neighborhood named a "\NewTerm{Hubble bubble}\index{Hubble bubble}"), the current laws of our Universe, the inflationary period of our universe and its geometry. Thus, the first Friedmann equation with the cosmological constant, which is a total artifice of work, is then: 
	
	with:
	
	This is Andrei Sakharov who defined the value of the above cosmological constant, which supposedly corresponds to the quantum energy of vacuum (depending on the Higgs fields).\\
	
	In quantum physics the equations of the field associated with elementary particles that are used to define the Big Bang theory are one of the main flow of the beginning of this 21st century. The famous Einstein's field equation tells us that energy creates a gravitational field like the electron in motion causes an electromagnetic field. It follows from these two observations that by measuring the gravitational field we have a way to determine the energy of vacuum. The gravitational field is no longer about the matter but about the energy density of vacuum. But the cosmological constant is directly proportional to the constant of gravity $G$. Its measurement is a quite very dangerous game because its value depends on several fundamental laws of physics and of significant properties on the dynamics of our Universe. The debate remains completely open and if I (the main redactor of this book) find a valid and rigorous proof of this constant, we will provide to the reader the  consequences of this constant on the models that we will see below.
	\end{tcolorbox}
	Let us now use the first law of thermodynamics (\SeeChapter{see section of Thermodynamics}) for a system by definition that will be closed and isolated, which the sum of kinetic and potential energy is constant (and therefore the amount of total energy variation is zero for these two energies). We then have that the change in total energy is only given by the variation of internal energy (the most common case in thermodynamics for macroscopic objects):
	
	and we have also proven in the section of Thermodynamics the characteristic equation of a fluid in equilibrium:
	
	If the system is adiabatic (no heat transfer between the system and outside), then we have also proven in the section of Thermodynamics that the variation of entropy was:
	
	Then:
	
	Since the universe is spherical assumed in our model, we have:
	
	and in the material repository where galaxies (cosmic fluid particles) are immobile:
	
	Therefore:
	
	which simplifies in:
	
	taking the derivative with respect to the cosmic time $t$:
	
	therefore:
	
	Now let us take again the first Friedmann equation, obtained above, in the form:
	
	and let us write it as follows:
	
	If we differentiate:
	
	Then we get:
	
	Let us inject:
	
	in the relation:
	
	Then we get:
	
	Thus:
	
	The following relation:
	
	is the "\NewTerm{Friedmann's second equation}\index{Friedmann's second equation}" that is also sometimes named "\NewTerm{Raychaudhuri equation}\index{Raychaudhuri equation}".
		
	\paragraph{Critical Density}\mbox{}\\\\\
	Let us come back to our first Friedmann equation without cosmological constant. So we have shown above that:
	
	We obtain then by injecting the latter relation in the first Friedmann equation the following relation:
	
	which rearranges with:
	
	into:
	
	The exponent of the left term requires that the right term is positive or zero as:
	
	Recall that the initial conditions impose us that at time $t_0=0$ we have:
	
	Indeed:
	
	Then it comes:
	
	This term should be accessible to observation, sadly $H_0^2$ is very poorly known and $\rho_0$ even more. In other words, given the "$-$" sign in the expression of $k$, we do not even know today the sign of this constant.
	
	However, it may be important to notice that there is a value $\rho_0$ named "\NewTerm{critical density}\index{critical densit}" that cancels the $k$ above and therefore also (see above):
	
	as it is also equal to $k$. This imply that the thotal Energy of the Universe would be zero (following considerations of quantum cosmology).

	This value $\rho_0$ is given immediately by:
	
	For $H_0=67.80\pm 0.7\;[\text{km} \cdot\text{s}^{-1}\cdot\text{Mpc}^{-1}]$ (actual value) we get (for the value in SI units see our tables of constants in the section Principia):
	
	In comparison, a hydrogen atom weighs $1.7\cdot 10^{-27}\;[\text{kg}]$, the critical density would therefore correspond to six hydrogen atoms per cubic meter.

	Physicists have defined a constant (time-varying... so therefore not so constant...) denoted by the Greek letter $\Omega$ and named "\NewTerm{cosmological density parameter}\index{cosmological density parameter}" and given by the ratio of the mass density (or energy densities since the ratio will be the same!):
	
	astrophysicists often break the cosmological density parameter in three terms:
	
	estimated  experimentally in this early 21st century at $\Omega=1.00\pm0.02$.
	
	It is interesting to work with this constant because in the case:
	\begin{itemize}
		\item $\Omega=1$:
		We have:
		
		which gives by replacing in the Friedmann $k=0$ (flat Universe as we shall see in our study of the relativistic model).

		\item $\Omega>1$:
		By performing the same reasoning, and still using inequality, we have then: $k>0$ (a Universe with positive curvature (closed) as we shall see in our study of the relativistic model).

		\item $\Omega<1$:
		By performing the same reasoning, and still using inequality, we have then: $k<0$ (a Universe with negative curvature (open) as we shall see in our study of the relativistic model).
	\end{itemize}
	These three situations can be summarized geometrically by the following well known figure:
	\begin{figure}[H]
		\begin{center}
		\includegraphics[scale=0.5]{img/cosmology/type_universe.jpg}
		\end{center}	
		\caption{Illustration of the different types of curvature (source: Wikipedia)}
	\end{figure}
	All measurement which have been made so far have failed to show a curvature of the Universe (anisotropy). The measurements of the microwave background (see mathematical details further below) by the BOOMERANG balloon and COBE satellite however, tend to support the hypothesis of a relatively flat Universe validating therefore numerical simulations:
	\begin{figure}[H]
		\begin{center}
		\includegraphics{img/cosmology/anisotropy_boomerang.jpg}
		\end{center}	
		\caption{Illustration of what would give observations depending on the curvature type}
	\end{figure}
	Also satellites sensibility don't stop to increase as how the figure below but it's still hard to conclude anything about the temperature background anisotropy:
	\begin{figure}[H]
		\begin{center}
		\includegraphics[scale=0.12]{img/cosmology/anisotropy_performance.jpg}
		\end{center}	
		\caption{Comparison of CMB results from COBE, WMAP and Planck – 2013-03-21 (source: Wikipedia)}
	\end{figure}
	\begin{tcolorbox}[title=Remark,colframe=black,arc=10pt]
	The concept of Universe topology and openness are actually normally two separate concepts. When we speak of "open" or "closed" Universe we do not normally talk about its topology but his destiny! Thus, an "open" Universe is expanding indefinitely and a "closed" Universe recontracte on itself after a given time. That said, in the models that we study in this section (cosmological constant equal to zero), the curvature is directly related to the density, and thus to its openness.
	\end{tcolorbox}
	Let us come back to the relation:
	
	We can write:
	
	By adopting the notation:
	
	\begin{tcolorbox}[title=Remark,colframe=black,arc=10pt]
	The actual measurement gives (year 2007): $A\cong 0.2793923067\cdot 10^{-35}$.
	\end{tcolorbox}
	Therefore:
	
	It is now appropriate for us to consider three situations:
	
	thus correspond respectively to the cosmological density parameters:
	
	\begin{tcolorbox}[title=Remark,colframe=black,arc=10pt]
	We can't put $\rho_0=0$ because our initial assumptions was the principle of conservation of energy.
	\end{tcolorbox}
	
	\pagebreak	
	\subsection{Cosmological models of Friedmann-Lemaitre}
	The Euclidean cosmological models of Friedmann-Lemaitre consist in the Newtonian  limit to study the  "\NewTerm{fundamental equation of Friedmann models}\index{fundamental equation of Friedmann models}":
	
	considering the three situations:
	
	\begin{tcolorbox}[title=Remark,colframe=black,arc=10pt]
	It is possible within the framework of General Relativity to rigorously find a solution to Einstein's field equations named the "\NewTerm{Robertson-Walker metric}\index{Robertson-Walker metric}" which in the case of a Newtonian approximation gives us the Friedmann equations obtained in the present text (often these are the approximations that are used in the literature for because the exact solution is out of the  framework of the traditional universities courses of the 21st century).
	\end{tcolorbox}
	\subsubsection{Flat spaces ($k=0$)}
	The flat (Euclidean) space model consist to assume that $k=0$. In other words, we are in a Universe whose density is named "\NewTerm{critical density}\index{critical density}" or also simply "\NewTerm{flat}\index{flat}" (as we will see with the relativistic model).

	The we have the following equation:
	
	By arranging appropriately the terms:
	
	and by integrating, it comes:
	
	Which simplifies in (we raise to the square, hence the removal of the double sign $\pm$):
	
	So we have in this model the relation:
	
	to which we must add a constant for the condition corresponding to today:
	
	that remains satisfied. Therefore:
	
	This gives us a plot function that looks like this (do not trust the values shown on the horizontal axis as they are arbitrary):
	\begin{figure}[H]
		\begin{center}
		\includegraphics{img/cosmology/friedmann_flat_space.jpg}
		\end{center}	
		\caption{Evolution of the scale factor for a zero curvature space}
	\end{figure}
	We put the area where $F(t)<1$ in evidence to remember that this part of the solution is to reject.

	So we have a model of Universe in which the scale factor is growing exponentially and this and indefinitely.
	\begin{tcolorbox}[title=Remark,colframe=black,arc=10pt]
	More $\rho_0$ is big, more the scale factor incrase fast (meaning that the slope is obviously larger).
	\end{tcolorbox}
	
	\paragraph{Flat space dominated by matter}\mbox{}\\\\\
	There is also another approach much more elegant and subtle from my point of view than the previous proof (I have discovered that many years after writing the previous version). It has the benefit of highlighting a hypothesis that does not appear with previous developments.

	We start from the first Friedmann equation:
	
	put always putting $k$ as being equal to zero and then we use the trick that consist to start from the relation prove also earlier above (used to proved the second Friedmann equation):
	
	to require that the fluid pressure $P$ (whatever it is: gas or radiation) is zero. We then say that the Universe is a universe dominated by matter and we deduce:
	
	which gives us:
	
	Under these conditions, the first Friedmann equation becomes:
	
	rearranging and simplifying, we then have:
	
	which gives:
	
	At the time $t=0$, we have the scale factor that is equal to $R=0$ and thus the constant is zero. Then we have:
	
	By putting that at time $t=0$, the scaling factor was unitary, the latter relationis simplified:
	
	Using the common notation in theoretical cosmology we get the famous relation often given without proof:
	
	If we assume that the scale factor is today taken as unitary, then it comes:
	
	and by replacing in it the numerical values currently known of the Hubble constant, it follows that the universe is now aged about $8.6$ billion years (compared to $13$ billion of the Hubble time obtained earlier above!).
	
	\paragraph{Flat space dominated by radiation}\mbox{}\\\\\
	We have proved in the section of Thermodynamics dur our study of the Stefan-Boltzmann law that the pressure of radiation was related to the energy density by the following equation:
	
	In a universe dominated by radiation, the relation:
	
	 proved earlier, expressed with a density of energy and not a density of mass becomes:
	
	Therefore:
	
	and using the relationship between radiation pressure and energy density, we get:
	
	After a little rearrangement, we get:
	
	from which we get that:
	
	Under these conditions, the first Friedmann equation:
	
	becomes first by changing into energy density and with $k$ being put as equal to zero:
	
	and so we can replace the energy density by the result we get just before:
	
	Then we have:
	
	hence:
	
	The primitive is obvious and is immediate:
	
	At the time of the Big Bang in $t=0$, we have the scale factor that is supposely $R=0$ and therefore the constant is zero. Then we have:
	
	But assuming that at $t=0$ we had $R_0=1$ the we have:
	
	Therefore:
	
	Thus after simplification it remains only:
	
	Using the common notation in theoretical cosmology we get the famous relation often given without proof:
	
	Then a flat universe dominated by radiation has a scaling factor that is growing slightly more slowly than a flat universe dominated by matter.

	In comparison with Maple 4.00b (blue: flat universe dominated by matter, in red: flat universe dominated by radiation):

	\texttt{>plot([t\string^(2/3),t\string^(1/2)],t=0..2*Pi,0..3,color=[blue,red]);}\\
	\begin{figure}[H]
		\begin{center}
		\includegraphics{img/cosmology/universe_scale_factor_evolution_flat_matter_radiation_maple.jpg}
		\end{center}	
		\caption[]{Evolution of $R$ for a zero curvature space dominated by matter or radiation.}
	\end{figure}
	
	\subsubsection{Spherical spaces ($k>0$)}
	In this model (also sometimes named "elliptical model"), we consider $k>0$. So the equation to deal with remains:
	
	Which can also be written:
	
	Let us recall that we have assumed that for $t=t_0$ we had $F(t_0)=1$, if we make the change of variable $U=1/F(t)$, we get the following integral:
	
	So we are looking of a primitive of:
	
	and we will discuss the sign $\pm$ after having found the primitive.

	We still carry a change of variable:
	
	thus:
	
 	which gives us the following primitive to calculate:
	
	by doing again a change of variable:
	
	Therefore to a given multiplicative constant $2Ak^{-3/2}$ we have finally the following integral:
	
	In the section of Differential andIntegral Calculus we proved that this form of primitive is resovled by (we add the constant of integration at the end because we do physics and we must satisfy to some initial conditions at which are not necessarily interested to in mathematics):
	
	with:
	
	hence:
	
	We still have to calculate $I_1$ (\SeeChapter{see section of Differential and Integral Calculus}):
	
	Finally:
	
	by inverting all the changes of variables and introducing the multiplicative constant again, we have finally in the case where $k>0$:
	
	Between the two terminals of integration $(1/F,1)$ we therefore have (the integration constant cancels and we take back the $\pm$ which was originally in the primitive):
	
	where for recall the theory request that $\Delta t>0$ (otherwise it's pure speculation and philosophy...).
	
	We see that as expected, if we $F=1$ in the above relation we have indeed $\Delta t=0$ (that means that at the moment in time we take the Universe actual size as reference, the time variation is indeed equal to $0$ as expected).
	
	If we plot this function for a fixed value $k>0$. We have the following  animated plot Maple in 17.00 for the negative sign:
	
	\texttt{>restart:\\
	>f:=(A,k,F)->-[(F*sqrt(A/F-k)/k+A/k\string^(3/2)*arctan(sqrt(A/f-k)/sqrt(k)))\\
	-(sqrt(A-k)/k+A/k\string^(3/2)*arctan(sqrt(A-k)/sqrt(k)))]:\\
	>plots:-animate(plot3d,[f(A,k,F),k=0.1..1,F=0..10,],A=0..1)
	}
	\begin{figure}[H]
		\begin{center}
		\includegraphics{img/cosmology/spherical_universe_maple_animation_minus_sign_solution.jpg}
		\end{center}
	\end{figure}
	We can see with this solution that as time increase the Universe reach an asymptote size whatever the value of $k$, but the bigger is $k$ the faster the asymptote is reached. In other words... we must be lucky not to be in a Universe with a to big positive curvature...
	
	And for the positive sign:\\
	
	\texttt{>restart:\\
	>f:=(A,k,F)->+[(F*sqrt(A/F-k)/k+A/k\string^(3/2)*arctan(sqrt(A/f-k)/sqrt(k)))\\
	-(sqrt(A-k)/k+A/k\string^(3/2)*arctan(sqrt(A-k)/sqrt(k)))]:\\
	>plots:-animate(plot3d,[f(A,k,F),k=0.1..1,F=0..10,],A=0..1)
	}
	\begin{figure}[H]
		\begin{center}
		\includegraphics{img/cosmology/spherical_universe_maple_animation_plus_sign_solution.jpg}
		\end{center}
	\end{figure}
	We can see with that this solution see to be the symmetric one of the above plot. That means as time reached back to the zero value, the Universe contracts on single point back and once again this effect is slower as $k$ is big! 
	
	\begin{tcolorbox}[title=Remark,colframe=black,arc=10pt]
	The time $\Delta t$ in the above plots is always represented on the vertical axis and also for all the following charts further below (you have to turn your head a little if as usually you want to put the time on the horizontal axis...).
	\end{tcolorbox}
	By fixing a small value of $A$ and for $k$, we get the following two-dimensional plot first for the negative sign:
	
	\texttt{>k:=0.0001;A:=1;\\
	 >plot([-(F*sqrt(A/F-k)/k+A/k\string^(3/2)*arctan(sqrt(A/F-k)/sqrt(k)))-(sqrt(A-k)/k\\
	+A/k\string^(3/2)*arctan(sqrt(A-k)/sqrt(k)))],F=1..10000,labels=[F,t])}
	\begin{figure}[H]
		\begin{center}
		\includegraphics[scale=0.6]{img/cosmology/universe_factor_evolution_for_constant_A_negative_sign.jpg}
		\end{center}
	\end{figure}
	where we as have already mention, the read must keep in mind that the values below $1$ must be rejected!
	
	And for the positive sign:
	\begin{figure}[H]
		\begin{center}
		\includegraphics[scale=0.6]{img/cosmology/universe_factor_evolution_for_constant_A_positive_sign.jpg}
		\end{center}
	\end{figure}
	By looking at the both plots above it is obvious to see that we have after a rotation and putting each one next to the other (we also could change the time reference to have a logical time axis):
	\begin{figure}[H]
		\centering
		\includegraphics[scale=0.5]{img/cosmology/universe_big_bang_big_crunch.jpg}
		\caption{Big Bang and Big Crunch plot side by side}
	\end{figure}
	and and what we see here is the Big Ban and the Big Crunch!
	
	Now let us recall that to build the previous model we started from:
	
	A limit condition (condition of integration) is that the right term to be positive. That is:
	
	or:
	
	So for our previous $2$D plots this limit is locate at $F=k/A=10,000$ and this is according to the maximum value before the Universe turn into a Big Crunch.
	
	So if $F^{-1}$ is smaller than $F_{\lim}^{-1}$, we are not in a valid (real) domain model anymore.
	
	In fact, beyond the time limit $t_{\lim}$ corresponding to this $F_{\lim}$, what does not know the computer that has drawn our function plot is that it should switch to the scaling function with the "$+$". So when we execute the plot of both functions we should get the previous figure.
	
	We see then that for $t<t_{\lim}$ the Universe is entering a phase of contraction that we commonly name the "\NewTerm{Big Crunch}\index{Big Crunch}". After this phase of contraction, it is possible that either the Universe disappears completely in a singularity or that it re-enters a cyclical dynamic phase (mathematically the two outcomes seems to be possible).
	
	\paragraph{Spherical space dominated by matter}\mbox{}\\\\\
	Just as the model for flat space, there is also another approach much more elegant and subtle for my taste than the previous proof (I have also discovered that many years after writing the previous text). It also has the advantage of highlighting a hypothesis that has not appear with previous developments and allows to plot more simply in Maple 4.00b the behavior of the scale factor of the Universe. We thus find exactly the famous plot representing the evolution of the scale factor of the Universe available in almost all popular books on the subject (without proofs obviously...)

	We start again from the first Friedmann equation:
	
	It is customary for this model equation to put $k=1$ (even take any positive number at least pick one that is friendly...) and we have shown that when matter dominates, we had:
	
	Since then:
	
	and it comes immediately:
	
	Therefore:
	
	If we move now to the comoving time also named "\NewTerm{conformal time}\index{conformal time}" (already introduced at the beginning of this section) defined mathematically by (don't forget that $R$ is note a radius but a ratio of two radius!):
	
	Then we have:
	
	Let us write that in the form:
	
	where $A$ is strictly positive. Let us do the substitution:
	
	Then we have:
	
	and we have proved in the section of Differential and Integral Calculus that the primitive is:
	
	Therefore:
	
	As at time $\eta=0$ we have $R=0$, it is necessary that the constant is such that:
	
	Therefore:
	
	Hence:
	
	Now, let us recall that:
	
	Therefore:
	
	Hence:
	
	and as we must have at time $t=0$ the comoving time that is also zero, the constant is then zero. Therefore we have the following parametric system in the end (something very strange with this result... it is the parametric function of the brachistochrone curve!!!!):
	
	With Maple 4.00b we then have by comparing the flat Universe dominated by matter (blue), the flat Universe dominated by radiation (red) and finally the positive curvature Universe dominated by matter (green) and putting artificial coefficients to better distinguish the plots:
	
	\texttt{>plot([t\string^(2/3),t\string^(1/2),[0.5*(t-sin(t)),0.5*(1-cos(t)),t=0..2*Pi]],\\
	t=0...Pi,0..2.5,color=[blue,red,green]);}
	\begin{figure}[H]
		\centering
		\includegraphics[scale=0.8]{img/cosmology/universe_models_01_maple_plot.jpg}
		\caption[]{Evolution of the $R$ factor for the resulting space configurations studied so far with Maple 4.00b}
	\end{figure}
	
	\paragraph{Spherical space dominated by radiation}\mbox{}\\\\\
	Let us now consider a universe dominated by radiation. We have proved earlier above that in this situation we had:
	
	and:
	
	In this case the Friedmann equation in terms of energy density can be written by putting $k=1$:
	
	What becomes:
	
	By injecting $R^4\rho_E=R_0^4\rho_{E,0}$, it comes:
	
	Therefore:
	Therefore:
	
	Let us write this in the form:
		
	If we change to the comoving time again:
	
	Then we have:
		
	In the section of Differential and Integral Calculus we have proved how to determine exactly the same primitive (because it is a usual primitive). We have:
	
	For at time $\eta=0$ we have $R=0$, it is necessary that the constant is such that:
	
	Hence:
	
	Now, remember that:
	
	Therefore:
	
	Hence:
	
	and as at time $t=0$ time comoving time is also zero, the constant is therefore equal to $\sqrt{A}$. Therefore we have finally the following parametric system:
	
	With Maple 4.00b we then comparing the flat Universe dominated by matter (blue), the flat Universe dominated by radiation (red), the Universe with positive curvature dominated by matter (green), the Universe with positive curvature dominated by radiation (black):\\
	
	\texttt{>plot([t\string^(2/3),t\string^(1/2),[0.5*(t-sin(t)),0.5*(1-cos(t)),t=0..2*Pi],[0.5*(1-cos(t)),\\0.5*sin(t),t=0..2*Pi]],t=0...Pi,0..3,color=[blue,red,green,black]);}
	\begin{figure}[H]
		\centering
		\includegraphics[scale=0.8]{img/cosmology/universe_models_01_maple_plot.jpg}
		\caption[]{Evolution of the $R$ factor for the resulting space configurations studied so far with Maple 4.00b}
	\end{figure}
	
	\subsubsection{Hyperbolic spaces ($ky0$)}
	In this model, we consider $k<0$. So the equation to be treated can be written:
	
	Which is also written:
	
	Let us recall that we assumed that $t=t_0$ that $F(t_0)=1$. If we make the change of variable $U=1/F(t)$, we get the following integral:
	
	So we are looking for a primitive of:
	
	and we will discuss the sign $\pm$ after finding the primitive.

	We still carry a change of variable by putting:
	
	 therefore:
	
	which gives us the following primitive to calculate:
	
	Doing again a change of variable:
	
	hence to a given multiplicative constant:
	
	We have:
	
	In the section of Differential and Integral Calculus we saw that this form of primitive is resolved by the relation (we added the constant of integration in the end because we do physics and must satisfy the initial conditions to which we were not interested to in pure mathematics):
		
	with:
	
	hence:
	
	hence:
	We still need to calculate $I_1$:
	
	Finally:
	
	by putting back all the changes of variables and introducing the multiplicative constant again, we have therefore in the case $k>0$:
	
	Between the two terminals of integration $(1/F,1)$ so we have (the integration constant is zero):
	
	We must obviously have (we take back the $\pm$ which was originally in the integral):
	
	If we plot this function for a fixed value $k>0$. We have the following  animated plot Maple in 17.00 for the negative sign (as the positive one has no physical meaning):
	
	\texttt{>restart:\\
	>f:=(A,k,F)->-(((-F(sqrt(A/F+abs(k)))/k+A/2(2*abs(k)\string^(3/2))*ln(abs(((sqrt(\\
	abs(k))+sqrt(A/F+abs(k)))/(sqrt(abs(k))-sqrt(A/F+abs(k))))/((sqrt(abs(k))+sqrt(A+abs(k)))/(sqrt(abs(k))-sqrt(A+abs(k))))))))):\\
	>plots:-animate(plot3d,[f(A,k,F),k=0.1..1,F=0..10,],A=0.1..1)
	}
	\begin{figure}[H]
		\begin{center}
		\includegraphics{img/cosmology/hyperbolic_universe_maple_animation_minus_sign_solution.jpg}
		\end{center}
	\end{figure}
	We see that small the constant $A$ is, the fastest the Universe increases indefinitely quickly. Furthermore for a fixed value of $k$, some values of $A$ are prohibited (it is in fact still the integration condition).

	Again we see that the in the equation above the criterion $F(t_0)=F(0)=1$ is naturally fully respected. All values $F (t) $ below $1$ are to be rejected!

	So we have in this hyperbolic model a Universe that grows indefinitely in an exponential away (as the flat Friedmann-Lemaitre mdoel) because since $k<0$, there is no integration condition limit anymore (unlike the previous spherical model).
	
	\subsubsection{Matter dominated hyperbolic space}
	Just as for the models for flat and spherical space, there is also another approach much more elegant and subtle for my taste than the previous proof (I have also discovered that one many years after writing the previous version) . It also has the advantage of highlighting a hypothesis that has not occurred with previous developments and allows to draw more simply in Maple 4.00b the behavior of the scale factor of the Universe. We thus find exactly the famous plot representing the evolution of the scale factor of the Universe available in almost all popular books on the subject but without proof.

	We always start from the first Friedmann equation:
	
	It is customary for this model to put $k=-1$ (as we have to choose any negative number at least we pick one that is friendly ...) and we have proved that when the radiation dominates, we had:
	
	In this case the Friedmann equation in terms of energy density can be written by putting $k=-1$:
	
	The first Friedmann equation then becomes:
	
	Then we have:
	
	Hence:
	
	This is exactly the same integral than that of the spherical Universe dominated by matter at the difference that in the root, we $+1$ instead of $-1$. We will proceed in the same manner using the time comoving time:
	
	It comes then:
	
	Let us write that in the form:
	
	where $A$ is strictly positive. Let us make the substitution:
	
	Then we have:
	
	Using the usual primitive proved in the section of Differential and Integral Calculus it comes:
	then we have:
	
	Using the usual primitive proved in the section of Differential and Integral Calculus it comes:
	
	Or redoing the change of variables:
	
	Therefore:
	
	So that at time $\eta=0$ we have $R=0$ , it is necessary that the constant is such that:
	
	Which brings us to that the constant is zero and thus:
	
	Therefore:
	
	Hence:
	
	and as:
	
	We have:
	
	Which gives:
	
	As at time $t=0$, we must have $\eta=0$, it follows that the constant must be zero. So finally, we have:
	
	with Maple 4.00b we then have by comparing the flat Universe dominated by matter (blue), the flat Universe dominated by radiation (red), the Universe with positive curvature dominated by matter (green), the Universe with positive curvature dominated by radiation (black), the negatively curved Universe dominated by matter (gray):
	
	\texttt{>plot([t\string^(2/3),t\string^(1/2),[0.5*(t-sin(t)),0.5*(1-cos(t)),t=0..2*Pi],}\\
	\texttt{[0.5*(1-cos(t)),0.5*sin(t),t=0..2*Pi],[0.5*(sinh(t)-t),0.5*(cosh(t)-1)}
	\texttt{,t=0..2*Pi]],t=0...Pi,0..3,color=[blue,red,green,black,gray]);}
	\begin{figure}[H]
		\centering
		\includegraphics[scale=0.8]{img/cosmology/universe_models_03_maple_plot.jpg}
		\caption[]{Evolution of the $R$ factor for the resulting space configurations studied so far with Maple 4.00b}
	\end{figure}
	We can therefore observe that for a negative curvature (hyperbolic type), expansion is growing significantly faster than for a flat Universe and this without end.
	
	\subsubsection{Hyperbolic space dominated by radiation}
	Let us now consider a Universe dominated by radiation. We have proved that in this situation we had:
	
	and:
	
	The first Friedmann equation the becomes:
	
	Then we have:
	
	hence:
	
	This is exactly the same integral than that of the spherical universe dominated by matter at the difference that in the root, we have $+1$ instead of $-1$. We will proceed in the same manner using the comoving time:
	
	It comes then:
	
	Let us write this in the form:
	
	where $A$ is strictly positive.In the section of Differential and Integral Calculus we have proved how to determine exactly the same primitive (because it is a usual primitive). We have:
	
	So that at time $\eta=0$ we have $R=0$ , it is necessary that the constant is zero. Therefore:
	
	Hence:
	
	and as:
	
	We have:
	
	Which gives:
	
	As at the time $t=0$, we must have $\eta=0$, it follows that the constant must be equal to $-\sqrt{A}$. So finally, we have:
	
	with Maple 4.00b we then have by comparing the flat Universe dominated by matter (blue), the flat Universe dominated by radiation (red), the Universe with positive curvature dominated by matter (green), the Universe with positive curvature dominated by radiation (black), the Universe with negative curvature dominated by matter (gray), the negative curvature Universe dominated by radiation (brown):
	
	\texttt{>plot([t\string^(2/3),t\string^(1/2),[0.5*(t-sin(t)),0.5*(1-cos(t)),t=0..2*Pi],}\\
	\texttt{[0.5*(1-cos(t)),0.5*sin(t),t=0..2*Pi],[0.5*(sinh(t)-t),0.5*(cosh(t)-1)}\\
	\texttt{t=0..2*Pi],[0.5*(cosh(t)-1),0.5*(sinh(t)),t=0..2*Pi]],t=0...Pi,0..3
,color=[blue,red,green,black,grey,brown]);}
	\begin{figure}[H]
		\centering
		\includegraphics[scale=0.8]{img/cosmology/universe_models_04_maple_plot.jpg}
		\caption[]{Evolution of the $R$ factor for the resulting space configurations studied so far with Maple 4.00b}
	\end{figure}
	We can therefore observe that for a negative curvature (hyperbolic type), the expansion of a Universe dominated by radiation grows slower than a universe dominated by matter (it's a bit intuitive...).

	Finally to summarize a little better all this with captions it we get the following important plot (its important to be implicated by this plot as the Universe affects us all ...):
	\begin{figure}[H]
		\centering
		\includegraphics[scale=1]{img/cosmology/summary_newtonian_universe.jpg}
		\caption{Summary of Newtonian Universe models}
	\end{figure}
	
	\subsection{Observable Universe}
	We have determined previously a possible interpretations of the current estimate of the age (horizon) of our Universe as being as the inverse of the Hubble constant that has given us for recall:
	
	that is to say about $13$ billion years.
	\begin{tcolorbox}[title=Remarks,colframe=black,arc=10pt]
	\textbf{R1.} It is important to know that popular research articles in cosmology  often use the term "Universe" in the sense of "observable Universe".\\
	
	\textbf{R2.} There should be more rigorous in fact when we speak of the Universe "age". In fact, we should rather say that the horizon of the Universe, for a comoving observer since the earliest times, is $13$ billion years. In other words, it's time that someone would have measure if he has remained an inertial observer (in free fall: not subjected to any force other than gravity) throughout the evolution of the Universe and in a repository such that he would always perceived this Universe as homogeneous and isotropic.
	\end{tcolorbox}
	The word "observable" used in this sense does not depend on whether modern technology actually permits detection of radiation from an object in this region (or indeed on whether there is any radiation to detect). It simply indicates that it is possible in principle for light or other signals from the object to reach an observer on Earth. In practice, we can see light only from as far back as the time of photon decoupling in the recombination epoch. That is when particles were first able to emit photons that were not quickly re-absorbed by other particles. Before then, the Universe was filled with a plasma that was opaque to photons. The detection of gravitational waves indicates there is now a possibility of detecting non-light signals from before the recombination epoch.
	
	At the beginning of the early 21st century, we do still do not know if the Universe is finite or infinite, although the majority of theorists currently favor an infinite universe.

	The observable Universe is thus composed of all locations that could have affected us since the Big Bang (beware!... despite its name, the Big Bang theory has nothing to say on its start! It merely describes the evolution and the expansion of the Universe).

	The current size (the "\NewTerm{comoving distance}\index{comoving distance}") of the observable Universe is larger, since the Universe continued to expand during the time that light takes to reach us, we believe it is about $40$ billion light years.

	This value can be obtained by taking the actual most distant visible object which is $13.39$ billion years of Earth. This will therefore have needed $13.39$ billion years to get away from us, its light will have needed $13.39$ billion years to reach us and during the time of light travel, it will have move away of $13.39$ billion years (since objects at cosmological horizon are going at the speed of light). Thus a total of about $40$ billion years.
	
	This observable Universe contains according to today's heuristics  estimates (year 2011) about $7\cdot 10^{22}$ stars, distributed in approximately $10^{10}$ galaxies, themselves organized into clusters and superclusters of galaxies. The number of galaxies may be even larger, as the "Hubble Ultra-Deep Field" observed with the Hubble space telescope seems to indicate us. 
	
	The Hubble Ultra-Deep Field is an image of a region of the observable universe (equivalent sky area size shown in bottom left corner), near the constellation Fornax. Each spot is a galaxy, consisting of billions of stars. The light from the smallest, most red-shifted galaxies originated nearly $14$ billion years ago.
	\begin{figure}[H]
		\centering
		\includegraphics[scale=0.19]{img/cosmology/hubble_deep_space.jpg}
		\caption{Hubble Ultra-Deep (source: Wikipedia, author: NASA and ESA)}
	\end{figure}
	However it is difficult to imagine what that represents. As we found on the Internet some wonderful series of illustrations we would like to share them with the reader before the disappear from the Internet.

	First here is a high-resolution summary of various structures you can found at different scales of our Universe (you can considerably zoom in!):
	\begin{figure}[H]
		\centering
		\includegraphics[scale=0.09]{img/cosmology/universe_scales.jpg}
		\caption{Universe scales (source: Wikipedia, author: Andrew Z. Colvin)}
	\end{figure}
	And a more detailed way to discover its structure:
	\begin{enumerate}
		\item The universe to $14$ billion light years (the visible Universe as we approximately know it today):
		\begin{figure}[H]
			\centering
			\includegraphics[scale=0.75]{img/cosmology/universe_zoom_0.jpg}
			\caption{Simplified illustration of the observable Universe (source: http://atunivers.free.fr, author: Richard Powell)}
		\end{figure}
		This illustration attempts to show the entire visible Universe. The galaxies in the universe tend to collect into vast sheets and "supercluster" of galaxies surrounding large vacuums zones, giving the universe a cellular appearance. Because light in the Universe only travel a finite speed, we see objects at the edge of the Universe as when it was very young, there is $14$ billion years ago

		Some numbers (estimates):
		\begin{itemize}
			\item Number of superclusters in the visible universe: $10$ million
			\item Number of galaxy groups in the visible universe: $25$ billion
			\item Number of large galaxies in the visible universe: $350$ billion
			\item Number of dwarf galaxies in the visible universe: $7$ trillion
			\item Number of stars in the visible universe: $30$ billion trillion  ($3\cdot 10^{22}$)
		\end{itemize}
		
		\item After a $\times 14$ zoom we get the Universe withing a $1$ billion light years, that is the neighboring superclusters:
		\begin{figure}[H]
			\centering
			\includegraphics[scale=0.75]{img/cosmology/universe_zoom_1.jpg}
			\caption{Simplified illustration of the neighboring superclusters (source: http://atunivers.free.fr, author: Richard Powell)}
		\end{figure}
		Galaxies and clusters of galaxies are not distributed uniformly in the Universe. Instead, they gathered in large clusters, sheets and walls of galaxies separated by large gaps in which few galaxies appear to be. The illustration above shows a number of these superclusters including the Virgo - a rather small supercluster of which our galaxy is part of. The entire map is approximately $7\%$ of the diameter of the visible universe. The galaxies are too small to appear individually on this map, each point there is a group of galaxies.
		
		Some numbers (estimates):
		\begin{itemize}
			\item Number of superclusters within $1$ billion light years: $100$
			\item Number of galaxy groups within $1$ billion light years: $240,000$
			\item Number of large galaxies within $1$ billion light years: $3$ million
			\item Number of dwarf galaxies within $1$ billion light years: $60$ million
			\item Number of stars within $1$ billion light years: $250,000$ trillion
		\end{itemize}
		
		\item After a $\times 10$ zoom we get the Universe within $100$ million light years or the Virgo Supercluster:
		\begin{figure}[H]
			\centering
			\includegraphics[scale=0.75]{img/cosmology/universe_zoom_2.jpg}
			\caption{Simplified illustration of the Virgo Supercluster (source: http://atunivers.free.fr, author: Richard Powell)}
		\end{figure}
		Our galaxy is just one of thousands that lie within $100$ million light years. The above illustration shows how galaxies tend to gather in groups, the largest nearby cluster is the Virgo cluster (Virgo), a concentration of several hundred galaxies which dominates the surrounding groups of galaxies. Collectively, all of these groups is known to Supercluster Virgo. The second richest cluster in this volume is the Fornax cluster (Fornax), but it is as rich as that of the Virgin. Only bright galaxies are drawn here, our galaxy is the point at center.	
		
		Some numbers (estimates):
		\begin{itemize}
			\item Number of galaxy groups within $100$ million light years: $200$
			\item Number of large galaxies within $100$ million light years: $2,500$
			\item Number of dwarf galaxies within $100$ million light years: $50,000$
			\item Number of stars within $100$ million light years: $200$ trillion
		\end{itemize}
		
		\item After a $\times 20$ zoom we get the Universe within $5$ million Light Years, that is the Local Group of Galaxies:
		\begin{figure}[H]
			\centering
			\includegraphics[scale=0.75]{img/cosmology/universe_zoom_3.jpg}
			\caption{Simplified illustration of the Local Group (source: http://atunivers.free.fr, author: Richard Powell)}
		\end{figure}
		The Milky Way is one of three large galaxies in the group named "\NewTerm{Local Group}\index{Local Group}" which also contains several dozen of dwarf galaxies. Most of these galaxies are plotted on the illustration above, but note that many of these dwarf galaxies a very small magnitude, so that there are certainly more to discover.	
		
		Some numbers (estimates):
		\begin{itemize}
			\item Number of large galaxies within $5$ million light years: $3$
			\item Number of dwarf galaxies within $5$ million light years: $46$
			\item Number of stars within $5$ million light years: $700$ billion
		\end{itemize}
		
		\item After a $\times 10$ zoom we get the Universe within $500,000$ light years, that is the Satellite Galaxies:
		\begin{figure}[H]
			\centering
			\includegraphics[scale=0.75]{img/cosmology/universe_zoom_4.jpg}
			\caption{Simplified illustration of Satellite Galaxies (source: http://atunivers.free.fr, author: Richard Powell)}
		\end{figure}
		The Milky Way is surrounded by several dwarf galaxies, each containing tens of millions of stars, which is insignificant compared to the population of the Milky Way itself. The map above shows all of the nearest dwarf galaxies that are gravitationally bound to the Milky Way, and revolve around it in a few billion years.
		
		Some numbers (estimates):
		\begin{itemize}
			\item Number of large galaxies within $500,000$ light years: $1$
			\item Number of dwarf galaxies within $500,000$ light years: $12$
			\item Number of stars within $500,000$ light years: $225$ billion billion
		\end{itemize}
		
		\item After a $\times 10$ zoom we get the Universe within $50,000$ light years, that is the Milky Way Galaxy:
		\begin{figure}[H]
			\centering
			\includegraphics[scale=0.75]{img/cosmology/universe_zoom_5.jpg}
			\caption{Simplified illustration of the Milky Way Galaxy (source: http://atunivers.free.fr, author: Richard Powell)}
		\end{figure}
		This map shows the Milky Way as a whole - a spiral galaxy of at least two hundred billion stars. Our Sun is buried deep within the Orion Arm about $26,000$ light years from the center. Toward the center of the galaxy, stars are much closer to each other than at the periphery where we live. Also notice the presence of small globular clusters far outside the galactic plane, and the presence of a neighboring dwarf galaxy - named "Sagittarius" - which is slowly being swallowed by our own Galaxy.
		
		Some numbers (estimates):
		\begin{itemize}
			\item Number of stars within $50,000$ light years: $200$ billion billion
		\end{itemize}
		
		\item After a $\times 10$ zoom we get the Universe within $5,000$ light years, that is to say the Orion Arm:
		\begin{figure}[H]
			\centering
			\includegraphics[scale=0.75]{img/cosmology/universe_zoom_6.jpg}
			\caption{Simplified illustration of the Orion Arm (source: http://atunivers.free.fr, author: Richard Powell)}
		\end{figure}
		This is a map of our corner of the Milky Way. The Sun is located in the Orion Arm - a fairly small arms compared to the Sagittarius Arm, which is closer to the galactic center. The map shows several stars visible to the naked eye, located far away in the Orion arm. The most notable group of stars is composed of the main stars of the constellation of Orion - from which the spiral arm gets its name. All these stars are bright giant and supergiant stars, thousands of times more luminous than the sun. The brightest star of the map is Rho Cassiopeia - to $4,000$ light-years from us is just barely visible to the naked eye star, but in reality it is a supergiant $100,000$ times brighter than our Sun.
		
		Some numbers (estimates):
		\begin{itemize}
			\item Number of stars within $5,000$ light years: $600$ million
		\end{itemize}
		
		\item After a $\times 20$ zoom we get the Universe within $250$ light years, that is to say the solar neighborhood:
		\begin{figure}[H]
			\centering
			\includegraphics[scale=0.75]{img/cosmology/universe_zoom_7.jpg}
			\caption{Simplified illustration of the solar neighborhood (source: http://atunivers.free.fr, author: Richard Powell)}
		\end{figure}
		This map shows the $1,500$ most luminous stars within $250$ light years. All these stars are much more luminous than the Sun, and most are visible to the naked eye. About a third of the stars visible to the naked eye are within $250$ light years, even though that area represents only a small part of our galaxy.
		
		Some numbers (estimates):
		\begin{itemize}
			\item Number of stars within $250$ light years: $260,000$
		\end{itemize}
		
		\item After a $\times 20$ zoom we get the Universe within $12.5$ light years (the nearest stars):
		\begin{figure}[H]
			\centering
			\includegraphics[scale=0.75]{img/cosmology/universe_zoom_8.jpg}
			\caption{Simplified illustration of the nearest stars (source: http://atunivers.free.fr, author: Richard Powell)}
		\end{figure}
		This map shows some stars up to a distance of $12.5$ light years from our Sun (there would be $33$ identified to this date). Most of these stars are red dwarfs - stars with a tenth of the mass of the Sun and a hundred times less bright. About $80\%$ of stars in the Universe are red dwarfs, and the nearest star - Proxima Centaure - is a typical example.
		
		The map below show all known stars within $20$ light years. There are a total of $77$ systems containing $110$ stars:
		\begin{figure}[H]
			\centering
			\includegraphics[scale=0.75]{img/cosmology/universe_zoom_9.jpg}
		\end{figure}
		The distances between stars are huge. The distance from the Sun to Proxima Centauri is $4.22$ light years, or $40$ trillion kilometers. Walk this distance would take a billion years. Even the fastest space probes in this early 21st would need $6,000$ years to make the trip. There are currently four probes leaving the solar system - Pioneer 10 and 11 and Voyager 1 and 2 but we will likely lose contact with them within the next two years (if it's not already done when the reader see these lines). The following diagram attempts to show these distances by broadening the scope from the inner solar system to Alpha Centauri:
	\begin{figure}[H]
		\centering
		\includegraphics[scale=0.75]{img/cosmology/universe_zoom_10.jpg}
		\end{figure}
	\end{enumerate}
	And finally an artist's logarithmic scale conception of the observable universe with the Solar System at the center, inner and outer planets, Kuiper belt, Oort cloud, Alpha Centauri, Perseus Arm, Milky Way galaxy, Andromeda galaxy, nearby galaxies, Cosmic Web, Cosmic microwave radiation and the Big Bang's invisible plasma on the edge.
	\begin{figure}[H]
		\centering
		\includegraphics[scale=0.4]{img/cosmology/observable_universe_logarithmic_illustration.jpg}
		\caption{Artist's logarithmic scale conception of the observable universe (source: Wikipedia, author: ?)}
	\end{figure}
	
	\pagebreak
	\subsection{Cosmic Microwave Background (CMB)}
	We have already mention the cosmic microwave background earlier above and we have given numerous illustrations of it and the corresponding satellites names and observations related to its experimental study. The purpose here is to have a mathematical approach of that latter.
	
	The existence and properties of the cosmic radiation discovered by Arno Penzias and Robert Woodrow Wilson were mainly due to the two physical phenomena that we will now describe in broad outline.

	The expansion of the Universe has for consequence in its gradual cooling. From the fantastically high values that have reigned immediately after the Big Bang that created the Universe, the temperature gradually decreased. When it reaches about $3,000$ [K] occurs the first of two crucial phenomena that interest us here: the radiation, which until then was in thermal equilibrium with the material particles practically ceases to interact with them and became independent. In the "standard model" of evolution of the Universe, we calculate that this crucial moment is situated $0.68\cdot 10^6$ years after the Big Bang (see the proofs further below).
	\begin{figure}[H]
		\centering
		\includegraphics[scale=0.35]{img/cosmology/planck_history_of_universe.jpg}
		\caption{2D illustration of Universe past history following 21st century hypothesis (source: ESA)}
	\end{figure}
	We can first qualitatively understand the physical reasons for this (further below we will approach this with maths stuff!). Shortly before, when for example the temperature was $100,000$ [K], the Universe contained mostly photons, electrons and bare atomic nuclei (mostly protons, and, to a lesser extent, $\alpha$ particles, helium $4$ nuclei). The temperature was too high so that the electrons and nuclei may form stable atoms. The interaction between the photons and charged particles (mainly electrons, the lighter of them) is sufficiently intense, and the density of the latter was then sufficiently strong, so that the photons were continuously diffused, transmitted and absorbed . Despite its expansion, the Universe was at every moment at equilibrium; its temperature $T$ was consistently well defined, although decreasing over time, the photon energy, that is to say the pulse of radiation, was therefore distributed according to Planck's law for this temperature $T$!
	
	The temperature decrease then permit the formation of atoms from the electrons and nuclei. This process led to a rapid drop in average cross section of interaction between photons and material particles (mainly due to the disappearance of free electrons), so that the Universe became transparent to photons. A quantitative assessment of the characteristics of the phenomenon is this decoupling occurred when the temperature dropped to $3,000$ [K] (see the simple mathematical approach further below).

	At the moment of decoupling, the volume density of the radiation energy is distributed in the pulsations spectrum according to Planck's law (\SeeChapter{see section Thermodynamics}):
	
	where we assume that $T$ is the temperature ($3,000$ [K] approximately - ionization temperature of the simplest atoms) at the moment of decoupling. This distribution will then evolve under the influence of the expansion of the Universe.

	Let us consider the photons located at time $t$ in the volume $r^3$, and whose pulsation $\omega$ with a variation $\mathrm{d}\omega$. Their number is then using previous relation equal to:
	where we assume that $T$ is the temperature ($3,000$ [K] approximately - ionization temperature of the simplest atoms) at the moment of decoupling. This distribution will then evolve under the influence of the expansion of the Universe.

	Let us consider the photons located at time $t$ in the volume $4/3r^3\cong r^3$, and whose pulsation $\omega$ with a variation $\mathrm{d}\omega$. Their number is then using previous relation equal to:
	
	As there is no absorption or emission of photons at this temperature (it is a hypothesis but as experimental measurements seem to confirm this model...), this number will remain constant. But because of the expansion of the Universe, these photons constant number will occupy a larger volume, and gain greater wavelength $\lambda$ (following the expansion of the structure of space due to the positive value of the Hubble constant) that is to say, a rather smaller pulsation $\omega$ (the equivalent of the Doppler effect). To clarify, let consider the situation at time a further time $t'$. All lengths of the Universe have increased between, between $t$ and $t'$, by the same scaling factor $F$ following the Hubble's law: the chosen radius $r$ of our previous selected sphere volume has become obviously:
	
	and the wavelength of the photons considered:
	
	so that their pulsation is equal at the instant $t'$ to:
	
	So the energy contained at this time in the volume $V\cong {r'}^3$ and in the pulsation range $(\omega',\omega'+\mathrm{d}\omega')$ that is given obviously by:
	
	is given by:
	
	The volumetric energy density $R'(\omega',T')\mathrm{d}\omega')$ at time $t'$, for the pulsation range $(\omega',\omega'+\mathrm{d}\omega')$, is then written:
	
	It follows that the spectral energy distribution is still at the instant $t'$ that of the black body:
	
	where the corresponding temperature $T'$ is immediately such that:
	
	that is more often written:
	
	Thus, after decoupling with matter, the cosmic radiation evolves maintaining the distribution of a black body whose temperature decreases regularly in the same proportion as the distances are increased during the expansion of the Universe.

	From the moment of decoupling, the $F$ factor of scale is very close to $1,000$ since from to the estimated $3,000$ [K] to go to the $2.7$ [K] measured today there is a factor of:
	
	also named "\NewTerm{recombination redshift}\index{recombination redshift}".
	
	 This value of approximately $1,000$ allows us from the Friedmann-Lemaitre model we have introduced earlier above to easily calculate at what time (horizon) of the Universe this decoupling occurred.
	 
	 If we consider a radiation dominate flat Universs we have proved roughly that the scale factor $R(t)$ fwas proportion to the $t^{2/3}$:
	
	So clearly we have the age age ratio on:
	
	That is:
	
	So to get the age of the universe at $z=1100$ we would therefore have to divide the age now, by the factor $36482$. This gives:
	
	Or more generally:
	
	 Thus we find a value of approximately $380,000$ years. This is according to what most textbooks gives without proof.
	 
	 Now let us see how we can get an an approximation of the $3,000$ [K]. Actually observations give that the density of radiations seems to be:
	
	and for that of matter:
	
	Then we have the photon to baryon ratio::
	
	\begin{tcolorbox}[title=Remark,colframe=black,arc=10pt]
	It is more common in textbooks to define the baryon to photo ratio as:
	
	\end{tcolorbox}
	Now let us consider that the first atoms absorbing photons and avoiding the transparency of the Universe were hydrogen atoms. As electrons "orbiting" the hydrogen proton never had by assumption the time to "relax" (reached their fundamental state) because of the density of the early Univers we can assume that all absorbed photons were for electrons between the first and second main quantum number. But as we have proved in the section of Corpuscular Quantum Physics:
	
	and for Hydrogen:
	
	So keep the value $13.605\;[\text{eV}]$ in mind.
	
	\subsection{Einstein Cosmological Model}
	Ok... We have seen a lot of stuff and all this without involving General Relativity! Let us see now a more robust way to derivate a more complete version of Friedman equations using a special metric AND General Theory of relativity. The result will be what we name the "\NewTerm{standard cosmological model}\index{standard cosmological model}".
	
	\subsubsection{Robertson-Walker metric}
	We will now introduce an important metric for General Relativity and Comsology that is the "\NewTerm{Friedmann–Lemaître–Robertson–Walker (FLRW) metric}\index{Friedmann–Lemaître–Robertson–Walker metric}\index{Friedmann–Lemaître–Robertson metric}\index{Robertson-Walker metric}".
	
	To see how we derive this metric with first consider the elemetary distance on a $2$D hypersurface:
	
	on if the $2$D hypersurface is a sphere, we have:
	
	For that latter relation if $R=c^{te}$ we have:
	
	After simplification:
	
	Therefore:
	
	We see that if $R\rightarrow +\infty$, then:
	
	Now let us consider a $3D$ hypersurface in a $4D$ space such that:
	
	The trick that explain with we put ourselves on the $\mathcal{S}^3$ sphere is that we anticipate in advance that by doing a change of variable we will reduce this metric to three coordinates only!
	
	We have also obviously:
	
	If we assume $R=c^{te}$ we get by differentiation:
	
	After simplification:
	
	That is:
	
	Now we use spherical coordinates (\SeeChapter{see section Vector Calculus}):
	
	But as we know that in the general:
	
	Therefore having all this, we can rewrite:
	
	as following:
	
	We will denote obviously $||\vec{r}||$ as $r$, and $||\mathrm{d}\vec{r}||$ as $\mathrm{d}r$, and as they are colinear, we have $\cos(\alpha)=1$, therefore:
	
	It is traditional to put:
	
	Therefore:
	
	As for the comoving time introduce earlier before, let us introduce the "\NewTerm{comoving coordinates}\index{comoving coordinates}" as being:
	
	Therefore:
	
	If we had made the same development but with the hyperbolic metric (\SeeChapter{see section Differential Geometry}), we would have obtain:
	
	This is why we write more generally:
	
	with:
	\begin{itemize}
		\item If $k=0$ we fall back on the flat Euclidean space (but don't forget we are on comoving coordinates!)
		
		\item If $k=+1$ we are on a spherical space
		
		\item If $k=-1$ we are in a hyperbolic space
	\end{itemize}
	Then the metric tensor including time coordinates with $-,+,+,+$ signature is given by:
	
	So that we can write obviously if $x^\mu=(ct,x^i)$:
	
	But $R$ can be an arbitrary function of time so we write traditionally:
	
	and as we have seen earlier it usage to denote the radius $a$ so therefore:
	
	and this is the final form of the Friedmann–Lemaître–Robertson–Walker (FLRW) metric.
	
	It is important to remember that $x$ has no units and is often denoted in texbooks with the following letter $\chi$ and that rigorously the scale factor $a(t)$ is defined by $R(t)/R_0$.

	The main results of the FLRW model were first derived by the Soviet mathematician Alexander Friedmann in 1922 and 1924. Although his work was published in the prestigious physics journal Zeitschrift für Physik, it remained relatively unnoticed by his contemporaries. Friedmann was in direct communication with Albert Einstein, who, on behalf of Zeitschrift für Physik, acted as the scientific referee of Friedmann's work. Eventually Einstein acknowledged the correctness of Friedmann's calculations, but failed to appreciate the physical significance of Friedmann's predictions.
	
	Friedmann died in 1925. In 1927, Georges Lemaître, a Belgian priest,	astronomer and periodic professor of physics at the Catholic University of Leuven, arrived independently at similar results as Friedmann had and published them in Annals of the Scientific Society of Brussels. In the face of the observational evidence for the expansion of the universe obtained by Edwin Hubble in the late 1920s, Lemaître's results were noticed in particular by Arthur Eddington, and in 1930–31 his paper was translated into English and published in the Monthly Notices of the Royal Astronomical Society.

	Howard P. Robertson from the US and Arthur Geoffrey Walker from the UK explored the problem further during the 1930s. In 1935 Robertson and Walker rigorously proved that the FLRW metric is the only one on a spacetime that is spatially homogeneous and isotropic (as noted above, this is a geometric result and is not tied specifically to the equations of general relativity, which were always assumed by Friedmann and Lemaître).

	Because the dynamics of the FLRW model were derived by Friedmann and Lemaître, the latter two names are often omitted by scientists outside the US. Conversely, US physicists often refer to it as simply "Robertson–Walker". The full four-name title is the most democratic and it is frequently used.[citation needed] Often the "Robertson–Walker" metric, so-called since they proved its generic properties, is distinguished from the dynamical "Friedmann-Lemaître" models, specific solutions for a(t) which assume that the only contributions to stress-energy are cold matter ("dust"), radiation, and a cosmological constant.
	
	
	
	
	\pagebreak
	\subsection{The Black Hole Universe}
	A recent hypothesis in the history of cosmology (since the 1970 as far as we know...) which is at the heart of many theoretical research (Stephen Hawking, Roger Penrose and others) is the possibility of assimilating our Universe to a Black Hole (\SeeChapter{see section of General Relativity}).

	The origin of the idea can be made from a very simple calculation:

	We know that approximately the radius of the (current) Universe is given according to our previous calculations by:
	
	But we have proved in the section of General Relativity (and Classical Mechanics) that the Schwarzschild radius is given by:
	
	What we can write for the Universe in the following form (under many assumptions: isotropy, homogeneity, spherical, etc.):
	
	which with the values of the critical density and the radius of the cosmological horizon calculated earlier above gives:
	
	So, roughly speaking, knowing all the uncertainties that we have accumulated in particular that on the Hubble constant we see that the Schwarzschild radius is not very far from the radius of the present Universe.

	As curious as it may seem, this question is not so far-fetched and is very seriously studied. It is therefore theoretically possible that our whole universe is encapsulated in a gigantic Black Hole (therefore of very large mass and very low density as we see it with our numerical values) of another inaccessible Universe ...

	What is certain is that if this were the case, the expansion of the Universe (now observed) could not continue beyond the horizon of this super Black Hole because nothing coming from within can cross this horizon. However, recent observations seem to show that the expansion of the Universe is far from slowing and tends to accelerate with time, which is in contradiction with such a Black Hole Universe...

	\begin{flushright}
	\begin{tabular}{l c}
	\circled{90} & \pbox{20cm}{\score{3}{5} \\ {\tiny 17 votes,  70.59\%}} 
	\end{tabular} 
	\end{flushright}

	%to make section start on odd page
	\newpage
	\thispagestyle{empty}
	\mbox{}
	\section{String Theory}
	\lettrine[lines=4]{\color{BrickRed}I}t must be considered in this section that string theory (and verbatim superstring theory) is currently speculative and could not be verified (confirmed) or falsified by experience following the scientific method. We should therefore take with caution the developments that follow and to be the most critical possible!
	
	It is also a theory (we can not talk currently about "model") of unification of the forces that is not new since it soon have over thirty years and is trying to bridge the issues of the standard model of particles and also to unite General Relativity and quantum physics (which is not without difficulty since the latter is dependent on the background unlike General Relativity). She is one of the many theories that exist in modern physics and is trying in the early 21st century this unification (there are dozens of other more or less known).
	
	\begin{tcolorbox}[title=Remark,colframe=black,arc=10pt]
	If this subject is covered in the Cosmology chapter and not Atomistic one it is only for an pedagogical reason. Indeed, the basic formalism of string theory is much closer to relativistic mechanics (special and General Relativity) than that of the wave quantum physics or quantum physics fields. It seemed therefore more suited to this day (!), to provide continuity in the mathematical formalism and its interpretation rather than a thematic continuity with a relatively different approach than the usual formalism of quantum physics.
	\end{tcolorbox}
	Well, string theory is a theoretical framework in which the point-like particles of particle physics are replaced by one-dimensional objects named "\NewTerm{strings}\index{strings}" as its name suggest it... It describes how these strings propagate through space and interact with each other. On distance scales larger than the string scale, a string looks just like an ordinary particle, with its mass, charge, and other properties determined by the vibrational state of the string. In string theory, one of the many vibrational states of the string corresponds to the graviton, a quantum mechanical particle that carries gravitational force. Thus string theory is a theory of quantum gravity.

	String theory is a broad and varied subject that attempts to address a number of deep questions of fundamental physics. String theory has been applied to a variety of problems in black hole physics, early universe cosmology, nuclear physics, and condensed matter physics, and it has stimulated a number of major developments in pure mathematics. Because string theory potentially provides a unified description of gravity and particle physics, it is a candidate for a theory of everything, a self-contained mathematical model that describes all fundamental forces and forms of matter. Despite much work on these problems, it is not known to what extent string theory describes the real world or how much freedom the theory allows to choose the details.
	
	The undeniable advantage of string theory, besides the fact that mathematically it is quite indigestible but not really worse than General Relativity, is that it avoids in a certain order... many singularities in the calculations unlike other contemporary theories that consider the objects as points (so zero volume and length...).
	
	The undeniable advantage of string theory, besides the fact that mathematically it is quite indigestible but not really worse than General Relativity, is that it avoids in a certain order ... many singularities in the calculations unlike other contemporary theories that consider the objects as points (so zero volume and length...).
	
	This theory, although aesthetic and remarkable in that it uses for its calculations bases foundations that have over 200 years, is lacking in our opinion to work with successive analogies, as we shall see, with current relativistic and quantum theories. While this is not dramatic in itself, the theory may seem to lose a little of its own autonomy even though the fact it is not. The reader should therefore not be badly surprised in the development that will follow...
	
	The main particularity of string theory is that his ambition does not stop to this reconciliation, but it claims to successfully unify the four known elementary interactions, we speak therefore about "theory of everything", while staying on two hypothesis/assumptions:
	
	\begin{enumerate}
		\item[H1.] The fundamental building blocks of the universe would not be point particles, but a variety of vibrating string having a given stress in the manner of an elastic. What we perceive as particles with special characteristics (mass, charge, etc.) are merely distinct strings vibrating differently. With this assumption, string theories admit a minimum scale and make it easy to avoid the emergence of some infinite amounts that are inevitable in usual quantum field theories.
		\item[H2.] The Universe could contain more than three spatial dimensions. Some of them, folded on themselves, being invisible to our scales (by a procedure named "\NewTerm{dimensional reduction}\index{dimensional reduction}").
	\end{enumerate}
	
	Despite promising first partial promising results and also remarkable rich mathematical background the string theory remains however incomplete. On the one hand, a multitude of solutions to string theory equations exist, which poses a selection problem for our Universe and, secondly, even if many similar models were obtained, none of them allows to reproduce accurately the standard model of particle physics...
	
	That said ... let us begin our initiation:
	
	\subsection{Wave equation of a transervsal string}
	The aim here will be in a first step to determine the non-relativistic wave equation of a string excited transversely using calculations that we made in the section of Wave Mechanics. Once this work done, we will pass to the study of relativistic strings and see their wave equation, as well as the non-relativistic version, can be assimilated to the current conservation equation we had proved in the section of Electrodynamics.
	
	We begin by recalling the form of the action that we proved in the section of Wave Mechanics for a non-relativistic string:
	
	with:
	
	Now, in the same way as we did in section of Analytic Mechanics (and in that of Quantum Field Theory), we will define a notation by an analogy to the canonical moments of the string:
	
	with $y'=\partial y/\partial x$. It is simply the derivatives from the Lagrangian density with respectively the first and second argument. More explicitly, we get then directly by doing the calculation (\SeeChapter{see section Wave Mechanics}):
	
	So if we rewrite the variational of the action we obtained in the section of Wave Mechanics with this canonical notation, we get:
	
	Making use of the same methods as in the section of Wave Mechanics, our variational can be express after simplification again in the form of three terms:
	
	The conditions to find the extreme value (according to the principle of least action) are the same as those seen in the section of Wave Mechanics. Thus, for the third term, we therefore have the wave equation of a transversely excited string with the canonical form given by:
	
	\begin{tcolorbox}[title=Remark,colframe=black,arc=10pt]
	It obviously should be noted that this form of writing will greatly facilitate our work!
	\end{tcolorbox}
	It must be observed (as it is remarkable!) also that as in the sectionof Analytical Mechanics, the canonical moment $\mathcal{P}^t$ as defined above, coincides perfectly (the hazard makes things well) with the density momentum that we obtained in the section of Wave Mechanics. Effectively:
	
	Thus, by analogy with the Analytical Mechanics (where we recall, the derivative of the Lagrangian with respect to the speed gives the momentum), $\dot{y}$ plays well the role of speed and thus the derivative of the Lagrangian density by this latter gives the momentum density $\mathcal{P}^t$!!!
	
	Remember also another point that has been seen in the section of Wave Mechanics, the extremum of the action ($\delta S=0$) imposes use the Neumann boundary conditions, which leads us to write $\mathcal{P}^x=0$.
	\begin{tcolorbox}[title=Remark,colframe=black,arc=10pt]
	In the context of string theory to relativistic with more than 3 dimensions, it is possible to generalize the concept of boundary conditions considering the constraints in space like hypersurfaces named Dp-branes with $p$ dimensions. The usual Dirichlet boundary conditions then correspond to the situation where the ends of a strings are constraint by a 0-brane. The Neumann condition for a free string in $p$ dimensions corresponds to a constraint on a Dp-brane.
	\begin{figure}[H]
		\begin{center}
		\includegraphics{img/cosmology/dp_brane.jpg}
		\end{center}	
		\caption[]{Illustration of Dp-branes}
	\end{figure}
	\end{tcolorbox}
	
	\subsection{Non-relativistic Wave equation of a transversal string}
	We will now determine the action of a relativistic string. We can, to lay the foundations of our study, remember that a point particle draws a line in space-time (each point on the line is marked by a time coordinate and three space coordinates). Therefore, by extension, a string that is a two-dimensional element (if we consider it with no thickness) plots a surface in the spacetime.
	
	Thus, just like the line that that draw a point particle in space-time is named a "world line" (\SeeChapter{see section Special Relativity}), the surface traced by a string will by analogy be named a "\NewTerm{surface Universe}\index{surface Universe}".
	
	A string in a closed space-time Minkowski trace, for example, a tube, while an open string trace a band:
	\begin{figure}[H]
		\begin{center}
		\includegraphics{img/cosmology/open_closed_string.jpg}
		\end{center}	
		\caption{Universe surface generated by respectively an open/closed string}
	\end{figure}
	in the figure above, with two spatial dimensions and one implicit temporal dimension, the string is motionless in our current space. It moves in space-time (as time goes on the vertical axis) but not in space in the example above (it would take an additional spatial component to see such a movement).
	\begin{tcolorbox}[title=Remark,colframe=black,arc=10pt]
	\textbf{R1.} Caution! Remember that the diagram above is in three dimensions while the space-time has four dimensions.\\
	
	\textbf{R2.} Remember also that the time vector of the orthogonal basis is always perpendicular to all other spatial components (this remark will be useful during our proof of the Nambu-Goto action).
	\end{tcolorbox}
	
	During our proof of the equation of motion in the section of General Relativity, we reparameterized the particle's Universe line with a parameter that was the proper time of the particle $t$. Indeed, you only have to remember parametric equations that represent curves. For example with Maple 4.00b:
	
	\texttt{>with(plots):}\\
	\texttt{>spacecurve([cos(t),sin(t),t],t=0..4*Pi,axes=boxed);}
	
	\begin{figure}[H]
		\begin{center}
		\includegraphics{img/cosmology/curve_parametrization.jpg}
		\end{center}	
		\caption{Elementary illustrated refresh of a parametric curve}
	\end{figure}
	and the same procedure is valid for a line in four dimensions (time + space).
	
	We were thus arrived to construct the expression of the action $S$ of it before applying the variational principle.
	
	We will do the same for a relativistic string with the difference that we will reparametrized the surfaces generated by the strings this time. The constraints we impose are that the chosen parameters will also have to be (in reference to the case of the particle) relativistic invariants.
	
	As we have therefore seen in the section of General Relativity, a world line can be reparametrized naturally using only one parameter (curvilinear abscissa). A surface in space, however, is a two-dimensional object, we assume by extension that it requires two parameters $\zeta^1,\zeta^2$ (one more) to be described completely.
	
	Indeed, we guess, that one of the two parameters will be the proper time (to make the surface evolve in the time), the second parameter will give a "thickness" of what would be only a Universe line if it did not exist. It would be sufficient in a three-dimensional space that the second parameter had, to generate a surface, the dimensions of length but in the four dimension space-time the second parameter must have the units of a surface.
	
	Given a parameterized surface, we can draw on them isolines of the parameters (lines where the two parameters $\zeta_,\zeta_2$ are constant over the entire surface). These contours cover the surface as a grid (see figure a little bit further below).
	
	The parametric equation of a volume in space requires three parameters as we saw in the section of Analytic Geometry. Thus, if a parametrized area in Euclidean space can be represented by a vector of the type:
	
	during a reparameterization and making use of the tensor notation of Minkowski space-time as seen in the section of General Relativity, we have (by restricting for the moment to the particular case of two spatial dimensions and one of time):
	
	Thus, the surface is the image of the parameters $(\xi^1,\xi^2)$. Alternatively, we can see the components $(\xi^1,\xi^2)$ as the time and space coordinates of the surface, at least locally!
	
	We now want to calculate the area of an element of any type of space in the same way as we did for the curvilinear abscissa of any world line in the section of General Relativity. This raises the question of the form of the differential surface element??? Should we take the multiplication of the differential of the two previously selected parameters as being a square, rectangle, circle, or other?
	
	In fact, we will put our choice on a parallelogram! This choice may seem completely arbitrary for now but as we'll see a few lines later, this choice coincides for mathematical reasons with that we name the "\NewTerm{induced metric}\index{induced metric}" of the surface itself (rather remarkable result!).
	
	Thus, let us denote by $\mathrm{d}\vec{v}_1$ and $\mathrm{d}\vec{v}_2$ the sides of the parallelogram. They are the image by $\vec{x}$ of the couples $(\xi^1,0)$ and $(0,\xi^2)$ respectively:
	\begin{figure}[H]
		\begin{center}
		\includegraphics{img/cosmology/elementary_surface_study_configuration.jpg}
		\end{center}	
		\caption{Configuration for the study of an elementary surface element}
	\end{figure}
	Therefore we can write:
	
	and then:
	
	Now let us calculate the surface $\mathrm{d}A$ (we will not take the letter $S$ to avoid confusion with the variable representing the action in this section) of the parallelogram (\SeeChapter{see section Vector Calculus}):
	
	using the dot product, this can be rewritten as:
	
	using the previously established relations this can be written:
	
	the latter relation is the general shape of a surface element of a parametrized pattern. The total surface is obviously given by:
	
	Just as in the framework of the study of the principle of least action (\SeeChapter{see section Analytical Mechanics}) we searched the optimum path for a particle browsing a universe line, for a string, we have to search for the optimum of the surface $A$ by minimizing the function $\vec{x}=\left(\xi^1,\xi^2\right)$.
	
	This latter form is, however, a little heavy and does not show anything known particularly or is not similar to any form already known in another field of physics. We will see, however that by digging a little bit however it is possible to get something pretty interesting.
	
	Consider now a vector $\mathrm{d}\vec{x}$ and its squared length (norm) given by the scalar product:
	
	\begin{tcolorbox}[title=Remark,colframe=black,arc=10pt]
	This approach of separating the wave function into the composition of a wave function of the center of mass and the relative movement is also used in the context of the study of poly-electronic atoms, but with one difference: as the nucleus is much more massive than the processing electrons (in approximation ...), the center of mass is assimilated to the nucleus of the atom and the relative motion to the entire electron cloud. This approximate approach is well known under the designation "\NewTerm{Born-Oppenheimer approximation}\index{Born-Oppenheimer approximation}".
	\end{tcolorbox}
	Careful in the future not to "see" the $s$ as squared in the $\mathrm{d}s$ (as it is the case in Special and General Relativity) but remember well that it is the $\mathrm{s}$ that is squared (the notation may lead to confusion...).
	
	Thus, the squared length of $\mathrm{d}\vec{x}$ can be expressed in tensor form:
	
	what we will note by convention in the future:
	
	The quantity $g_{ij}(\xi)$  is named the "\NewTerm{induced metric of the parametrized area}\index{induced metric of the parametrized area}" (because contains a scalar product which quite generally uses a metric... hence the term "induced") and is therefore a matrix of dimension $2 \times 2$. It is obvious that the choice of this name comes from the resemblance with the usual metric as we have defined in our study of tensor calculus and from its use in special and General Relativity.
	
	The matrix $g_{ij}(\xi)$ has therefore by design and definition the form:
	
	Now let us come back to our expression of the surface generated by the string:
	
	and let us quickly calculate the determinant (\SeeChapter{see section Linear Algebra}") of the matrix $g_{ij}(\xi)$:
	
	and so what? Well here it is! $A$ can now been expressed as:
	
	Thus, the choice of the parallelogram as elementary surface is best explained here!
	
	Now we will adopt the traditional notations of string theory in relation to the expression of the surface. Thus, just as the time-space coordinates are described in Special Relativity the space-time four-vector:
	
	we will describe the surfaces Universe by (we now turn to the notation making use of the four dimensions of space-time):
	
	This notation will save us in the future to have to confuse, if the theory leads us there, the traditional space-time coordinates $x^\mu$ with the image function of the surface Universe $x^\mu(\tau,\sigma)$  and this especial because physicists being sometimes a little lazy shorten this latter $x^\mu$... hence the choice of the capital letter.
	
	It is then much more appropriate and wise to change the notation.
	
	From now on we will name "\NewTerm{string coordinates}\index{string coordinates}" the surface Universe described by $X^\mu$.
	
	This small change in notation obviously not change the interpretation of the image of the function. Given a couple $(\tau,\sigma)$ involving proper-time element and surface element of the pre-image, this point is projected onto a surface element of the space-time coordinates of the string:
	
	
	\subsubsection{Nambu-Goto Action}
	The Nambu–Goto action is the simplest invariant action in bosonic string theory, and is also used in other theories that investigate string-like objects (for example, cosmic strings). It is the starting point of the analysis of zero-thickness (infinitely thin) string behavior, using the principles of Lagrangian mechanics. Just as the action for a free point particle is proportional to its proper time—i.e., the "length" of its world-line—a relativistic string's action is proportional to the area of the sheet which the string traces as it travels through spacetime.
	
	In the case of a Universe surface the parameters are then by convention $\tau$ and $\sigma$, where as in special and General Relativity, the proper time may be in the range:
	
	the second can only be positive since this is a surface:
	
	and the coordinates of this surface corresponding to the parameter space is therefore:
	
	where once again, for recall, the parameter $\tau$ is considered as the variable describing the time (there must be one!), and $\sigma$ the variable describing the spatial extension of a string (that is to say that the condition $\sigma\in]0,\sigma_1[$ involving the finite length of the string).

	The parameters $(\tau,\sigma)$ describe therefore a surface in the space of pre-images:
	\begin{figure}[H]
		\begin{center}
		\includegraphics{img/cosmology/space_time_surface_parametrization.jpg}
		\end{center}	
		\caption{Parameterization of a space-time surface}
	\end{figure}
	The ends of the string have a constant value $\sigma$. However, as time passes and that the ends of the string on the Universe surface move we must notice an essential condition of the Universe surface concerning the both ends of an open string:
	
	\begin{tcolorbox}[title=Remark,colframe=black,arc=10pt]
	This condition is made on the component $X^0$ because it corresponds to the component $x^0$ of the space-time four-vector which is nothing else, in natural units, $t$ (the proper time). Therefore, time passes and is never constant, this is why we  impose this derivative as being nonzero.
	\end{tcolorbox}
	And using the standard conventions in physics for writing the derivatives with respect to time or space components, we agree to adopt also now the following notations:
	
	since as:
	
	then:
	
	The surface is therefore written:
	
	However, there is a problem here! Indeed, let us look if the radicand (term under the root) has a tangible physical reality ...

	For this, we must first consider the left side of the figure below representing the surface patch described by an open string:
	\begin{figure}[H]
		\begin{center}
		\includegraphics{img/cosmology/surface_patch_for_study_radicand_string_action.jpg}
		\end{center}	
	\end{figure}
	At each point $P$ of this surface patch (assumed differentiable at every point) there are endless tangents, all in the same plane, which we will denote for example by $\vec{v}$ and thus form a surface tangent at $P$.

	Now, as the space in which the surface patch of the string held in a spatial and temporal orthonormal basis, the tangent vectors $\vec{v}$ can then in turn be decomposed into a two-dimensional spatial and temporal local orthogonal base at the point $P$ such that the vectors of the base are two vectors (see our study of surface patches in the section of Differential Geometry):
	
	all other tangent vectors expressing as a linear combination thereof.
	
	However, a problem remains in our decomposition (...): the unit vectors of the local orthogonal basis at $P$ have units that differ... For this let us add a dimensional factor  to the spatial component (this is arbitrary because the conclusion will be the same regardless of the component on which you put the sizing factor) as we did in Special Relativity with the time axis:
	
	This dimensional factor can also be used to get all the tangent vectors such as:
	
	Indeed, if $\lambda\in [-\infty,+\infty]$, then for $\lambda=0$ we get the vector $\partial X^\mu/\partial \tau$ and for $\lambda=+\infty$ the vector $\partial X^\mu/\partial \sigma$ . And all intermediate values, we get all the tangent vectors as shown on the left side of the previous figure.

	Now, let us recall that we saw in the section of Special Relativity that there exist according to the curvilinear abscissa:
	
	of the Universe line of the light type ($\mathrm{d}s^2$), space ($\mathrm{d}s^2<0$) or time ($\mathrm{d}s^2>0$) if we consider the four-vectors $x^\mu$.

	It must be true by analogy the same for the tangent vectors to the surface and given by:
	
	Therefore:
	
	that corresponds to an equation of the second degree on $\lambda$, and that must, to have negative values (Universe surface patch of the type of space) or positive (Universe surface type of the type time) have at least two roots (see the right part of the previous figure ). This brings us back to the condition that the discriminant is strictly positive (\SeeChapter{see section Calculus}):
	
	Therefore:
	
	Into condensed form this is equivalent as writing:
	
	The surface must therefore be written as:
	
	if we want the radicand has a physical sense.
	
	By analogy with the Lagrangien of General Relativity we write the latter:
	
	that we will justify a little more robustly further below.
	
	Recall now that the action $S$ of a point particle is proportional to its world line (proper time). Thus by analogy, the action $S$ of string will be proportional to the Universe area:
	
	which gives:
	
	Which brings us very frequently in the literature to find the action of a string in the following form:
	
	or more stylized:
	
	Relation to compare with the Lagrangian of a free particle (\SeeChapter{see section Analytical Mechanics}) and with the Lagrangian density of a field (\SeeChapter{see section of Quantum Field Theory}):
	
	The functional $S$ has for units the one of a surface. This because the $X^\mu$ have a unit of length and inside the root each is at the fourth power and the units of $\tau,s\sigma$ cancel between the inner root and the differential that are outside it.
	
	Now, by definition of the action, the units that we have to get should match that of energy multiplied by time. That is to say joules [J] or using the international system of units $[\text{kg}\cdot\text{m}^2\cdot s^{-1}]$. For now, we have:
	
		To get to the action units we want, then we have to multiply the expression of the surface $A$ by a quantity whose units are $[\text{kg}\cdot\text{s}^{-1}]$. To chose these quantities, we will inspire us from our study of Wave Mechanics. When we worked with (non-relativistic) strings we saw that the properties to be considered were the stress and the velocity of string wave propagation. We'll try to take the following stress/speed ratio:
	
	where appears therefore that the stress of the string at rest $T_0$ and the speed of light $c$. 
	
	\begin{tcolorbox}[title=Remark,colframe=black,arc=10pt]
	This is similar to the point material physics where in the action we find the rest mass (equivalent to the tension at rest of the string) and the speed of light (\SeeChapter{see section Special Relativity}).
	\end{tcolorbox}
	Thus, the "\NewTerm{Nambu-Goto action}\index{Nambu-Goto action}" can now be written:
	
	\begin{tcolorbox}[title=Remark,colframe=black,arc=10pt]
	We will prove later why we put a factor "$-$". However, a small analogy with the action of a point particle, for which we also have a "$-$" sign (\SeeChapter{see section of Special Relativity}), can easily be done...
	\end{tcolorbox}
	Let us define for what will follow:
	
	What we can also write in matrix form:
	
	using the determinant of the matrix, it comes:
	
	So we can then write the action of a relativistic string in the final following condensed form:
	
	which is nothing else than the "\NewTerm{Nambu-Goto condensed form}\index{Nambu-Goto condensed form}" form of a relativistic string.
	
	We will now obtain the equation of motion by varying the action. For this, we will exactly inspire us of the methods seen when determining the non-relativistic wave equation of string in the section of Wave Mechanics.

	Thus, we rewrite the Nambu-Goto action by defining a Lagrangian density $\mathcal{L}$ such that:
	
	where $\mathcal{L}$ is therefore defined by:
	
	We will now apply the variational principle on the action in order to get the equation of movement of a string. The development and approximation are perfectly similar to those seen at the beginning of this section for the non-relativistic string. Let us recall that we obtained as Lagrangian density and as an expression of the action:
	
	and that the application of variational gave us:
	
	But what we did not see in the section Wave Mechanics is that the latter relation could easily be written also from the Lagrangian density:
	
	Therefore, for the relativistic string, we have an identical form by applying developments in all points similar (even if the Lagrangian density has a different form):
	
	and as we did at the beginning of this section for non-relativistic strings, we will introduce the canonical momentum (pulse density/momentum if you prefer) of the string by choosing for the notation:
	
	where in the details, we get very easily (it's a simple derivative but if you wish by contacting us, we can detail the developments as always in this book) the longitudinal and transverse momentum:
	
	\begin{figure}[H]
		\begin{center}
		\includegraphics{img/atomistic/string_momentum_density.jpg}
		\end{center}	
	\end{figure}
	by making use of this notation, we can then write:
	
	Making use once again of exactly the same methods as those seen in the section Wave Mechanics, our variational can be written, after simplification, again in the form of three terms:
	
	The conditions to find the extrem8k (according to the principle of least action) remain the same as in Wave Mechanics. Thus, for the third term, we have well the wave equation of a transversely excited stringwith the following canonical form:
	
	As far as we know this equation is horribly difficult to solve but choosing an appropriate parameterization can nevertheless simplify the task.
	
	\subsection{Lagrangian of a String}
	Let us recall that we have:
	
	and that with this choice, we have:
	
	\begin{gather*}
		\begin{aligned}
		&=-\dfrac{T_0}{c}\int \mathrm{d}t\mathrm{d}\sigma\sqrt{\left(\dfrac{\partial \vec{X}}{\partial t}\circ\dfrac{\partial \vec{X}}{\partial \sigma}\right)^2+c^2\left(\dfrac{\partial \vec{X}}{\partial \sigma}\right)^2-\left(\dfrac{\partial \vec{X}}{\partial t}\right)^2\left(\dfrac{\partial \vec{X}}{\partial \sigma}\right)^2}\\
		\end{aligned}
	\end{gather*}
	Now let use what we have seen in section of Differential Geometry with the Frenet's triad:
	
	where $\vec{T}$ is the tangent to the Universe surface at a time $t$ at the neighborhood of a given point. We had also notice in this same section that by definition:
	where $\vec{T}$ is the tangent to the Universe surface at a time $t$ at the neighborhood of a given point. We had also notice in this same section that by definition:
	
	But we can write:
	
	where it should be remembered that $\partial \vec{X}/\partial \sigma$ is taken at a fixed time $t$. As the lines of the surface Universe of constant $ $t describe the string, then $\partial \vec{X}/\partial \sigma$ is tangent to the string.

	And as:
	
	Then $\mathrm{d}\vec{X}/\mathrm{d}s$ is collinear to $\partial \vec{X}/\partial \sigma$ and thus also tangential to the string (information that we did not have a few lines before!). These small discovers being made, let us go back to:
	
	it already gets a little more interesting!

	Now let consider the following figure:
	\begin{figure}[H]
		\begin{center}
		\includegraphics{img/cosmology/recall_dot_product.jpg}
		\end{center}	
		\caption[]{Illustrated recall of the dot product}
	\end{figure}
	where $\vec{u}$ is any vector and $\vec{n}$ a unit (dimensionless) vector and $\vec{v}$, the orthogonal projection of $\vec{u}$ on $\vec{n}$. We then have (\SeeChapter{see section Vector Calculus}):
	
	Now if we seek for the vector $\vec{v}$ we will have to multiply  the whole $\vec{n}$:
	
	Finally, if we seek the expression of the vector $\vec{}$ it comes immediately:
	
	Then by similarity, we can write:
	
	where $\vec{w}$ is perpendicular to $\partial\vec{X}/\partial s$ and as the unit of speed. By construction, $\vec{w}$ is therefore the transverse velocity of the speed at a time $t$ t since $\partial\vec{X}/\partial s$ is tangent thereto. We will denote then:
	
	Let us now take, for future needs, the square norm of this last relation (be careful we process the components of the vectors directly by generalizing the vector notation!):
	
	and if we now go back to:
	
	The associated Lagrangian is then directly (not to be confused with the Lagrangian density!):
	
	as:
	
	The Lagrangian of the prior-previous relation is considered by the specialists in string theory as a natural generalization of the Lagrangian of the free particle as we get it in the section of Special Relativity and that was for recall given by:
		
	
	\begin{flushright}
	\begin{tabular}{l c}
	\circled{20} & \pbox{20cm}{\score{3}{5} \\ {\tiny 37 votes,  80\%}} 
	\end{tabular} 
	\end{flushright}